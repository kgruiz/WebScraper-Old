\title{typst.app/universe/package/letter-pro}

\phantomsection\label{banner}
\phantomsection\label{template-thumbnail}
\pandocbounded{\includegraphics[keepaspectratio]{https://packages.typst.org/preview/thumbnails/letter-pro-3.0.0-small.webp}}

\section{letter-pro}\label{letter-pro}

{ 3.0.0 }

DIN 5008 letter template for Typst.

\href{/app?template=letter-pro&version=3.0.0}{Create project in app}

\phantomsection\label{readme}
A template for creating business letters following the DIN 5008
standard.

\subsection{Overview}\label{overview}

typst-letter-pro provides a convenient and professional way to generate
business letters with a standardized layout. The template follows the
guidelines specified in the DIN 5008 standard, ensuring that your
letters adhere to the commonly accepted business communication
practices.

The goal of typst-letter-pro is to simplify the process of creating
business letters while maintaining a clean and professional appearance.
It offers predefined sections for the sender and recipient information,
subject, date, header, footer and more.

\subsection{\texorpdfstring{\href{https://raw.githubusercontent.com/wiki/Sematre/typst-letter-pro/documentation-v3.0.0.pdf}{Documentation}}{Documentation}}\label{documentation}

\subsection{Example}\label{example}

Text source:
\href{https://web.archive.org/web/20230927152049/https://www.deutschepost.de/de/b/briefvorlagen/beschwerden.html\#Einspruch}{Musterbrief
Widerspruch gegen Einkommensteuerbescheid}

\subsubsection{\texorpdfstring{Preview (
\href{https://raw.githubusercontent.com/wiki/Sematre/typst-letter-pro/simple_letter.pdf}{PDF
version} )}{Preview ( PDF version )}}\label{preview-pdf-version}

\pandocbounded{\includegraphics[keepaspectratio]{https://github.com/typst/packages/raw/main/packages/preview/letter-pro/3.0.0/template/thumbnail.png}}

\subsubsection{Code}\label{code}

\begin{Shaded}
\begin{Highlighting}[]
\NormalTok{\#import "@preview/letter{-}pro:3.0.0": letter{-}simple}

\NormalTok{\#set text(lang: "de")}

\NormalTok{\#show: letter{-}simple.with(}
\NormalTok{  sender: (}
\NormalTok{    name: "Anja Ahlsen",}
\NormalTok{    address: "Deutschherrenufer 28, 60528 Frankfurt",}
\NormalTok{    extra: [}
\NormalTok{      Telefon: \#link("tel:+4915228817386")[+49 152 28817386]\textbackslash{}}
\NormalTok{      E{-}Mail: \#link("mailto:aahlsen@example.com")[aahlsen\textbackslash{}@example.com]\textbackslash{}}
\NormalTok{    ],}
\NormalTok{  ),}
  
\NormalTok{  annotations: [Einschreiben {-} Rückschein],}
\NormalTok{  recipient: [}
\NormalTok{    Finanzamt Frankfurt\textbackslash{}}
\NormalTok{    Einkommenssteuerstelle\textbackslash{}}
\NormalTok{    Gutleutstraße 5\textbackslash{}}
\NormalTok{    60329 Frankfurt}
\NormalTok{  ],}
  
\NormalTok{  reference{-}signs: (}
\NormalTok{    ([Steuernummer], [333/24692/5775]),}
\NormalTok{  ),}
  
\NormalTok{  date: "12. November 2014",}
\NormalTok{  subject: "Einspruch gegen den ESt{-}Bescheid",}
\NormalTok{)}

\NormalTok{Sehr geehrte Damen und Herren,}

\NormalTok{die von mir bei den Werbekosten geltend gemachte Abschreibung für den im}
\NormalTok{vergangenen Jahr angeschafften Fotokopierer wurde von Ihnen nicht berücksichtigt.}
\NormalTok{Der Fotokopierer steht in meinem Büro und wird von mir ausschließlich zu beruflichen}
\NormalTok{Zwecken verwendet.}

\NormalTok{Ich lege deshalb Einspruch gegen den oben genannten Einkommensteuerbescheid ein}
\NormalTok{und bitte Sie, die Abschreibung anzuerkennen.}

\NormalTok{Anbei erhalten Sie eine Kopie der Rechnung des Gerätes.}

\NormalTok{Mit freundlichen Grüßen}
\NormalTok{\#v(1cm)}
\NormalTok{Anja Ahlsen}

\NormalTok{\#v(1fr)}
\NormalTok{*Anlagen:*}
\NormalTok{{-} Rechnung}
\end{Highlighting}
\end{Shaded}

\subsection{Usage}\label{usage}

\subsubsection{Preview repository}\label{preview-repository}

Import the package in your document:

\begin{Shaded}
\begin{Highlighting}[]
\NormalTok{\#import "@preview/letter{-}pro:3.0.0": letter{-}simple}
\end{Highlighting}
\end{Shaded}

\subsubsection{Local namespace}\label{local-namespace}

Download the repository to the local package namespace using Git:

\begin{Shaded}
\begin{Highlighting}[]
\ExtensionTok{$}\NormalTok{ git clone }\AttributeTok{{-}c}\NormalTok{ advice.detachedHead=false https://github.com/Sematre/typst{-}letter{-}pro.git }\AttributeTok{{-}{-}depth}\NormalTok{ 1 }\AttributeTok{{-}{-}branch}\NormalTok{ v3.0.0 \textasciitilde{}/.local/share/typst/packages/local/letter{-}pro/3.0.0}
\end{Highlighting}
\end{Shaded}

Then import the package in your document:

\begin{Shaded}
\begin{Highlighting}[]
\NormalTok{\#import "@local/letter{-}pro:3.0.0": letter{-}simple}
\end{Highlighting}
\end{Shaded}

\subsubsection{Manual}\label{manual}

Download the \texttt{\ letter-pro-v3.0.0.typ\ } file from the
\href{https://github.com/Sematre/typst-letter-pro/releases}{releases
page} and place it next to your document file, e.g., using \emph{wget} :

\begin{Shaded}
\begin{Highlighting}[]
\ExtensionTok{$}\NormalTok{ wget https://github.com/Sematre/typst{-}letter{-}pro/releases/download/v3.0.0/letter{-}pro{-}v3.0.0.typ}
\end{Highlighting}
\end{Shaded}

Then import the package in your document:

\begin{Shaded}
\begin{Highlighting}[]
\NormalTok{\#import "letter{-}pro{-}v3.0.0.typ": letter{-}simple}
\end{Highlighting}
\end{Shaded}

\subsection{Contributing}\label{contributing}

Contributions to typst-letter-pro are welcome! If you encounter any
issues or have suggestions for improvements, please open an issue on
GitHub or submit a pull request.

Before making any significant changes, please discuss your ideas with
the project maintainers to ensure they align with the project’s goals
and direction.

\subsection{Acknowledgments}\label{acknowledgments}

This project is inspired by the following projects and resources:

\begin{itemize}
\tightlist
\item
  \href{https://de.wikipedia.org/wiki/DIN_5008}{Wikipedia / DIN 5008}
\item
  \href{https://web.archive.org/web/20240223035339/https://www.deutschepost.de/de/b/briefvorlagen/normbrief-din-5008-vorlage.html}{Deutsche
  Post / DIN 5008 Vorlage}
\item
  \href{https://www.deutschepost.de/dam/dpag/images/P_p/printmailing/downloads/dp-automationsfaehige-briefsendungen-2024.pdf}{Deutsche
  Post / Automationsfähige Briefsendungen}
\item
  \href{https://www.edv-lehrgang.de/din-5008/}{EDV Lehrgang / DIN-5008}
\item
  \href{https://github.com/ludwig-austermann/typst-din-5008-letter}{Ludwig
  Austermann / typst-din-5008-letter}
\item
  \href{https://github.com/pascal-huber/typst-letter-template}{Pascal
  Huber / typst-letter-template}
\end{itemize}

\subsection{License}\label{license}

Distributed under the \textbf{MIT License} . See \texttt{\ LICENSE\ }
for more information.

\href{/app?template=letter-pro&version=3.0.0}{Create project in app}

\subsubsection{How to use}\label{how-to-use}

Click the button above to create a new project using this template in
the Typst app.

You can also use the Typst CLI to start a new project on your computer
using this command:

\begin{verbatim}
typst init @preview/letter-pro:3.0.0
\end{verbatim}

\includesvg[width=0.16667in,height=0.16667in]{/assets/icons/16-copy.svg}

\subsubsection{About}\label{about}

\begin{description}
\tightlist
\item[Author :]
Sematre
\item[License:]
MIT
\item[Current version:]
3.0.0
\item[Last updated:]
October 28, 2024
\item[First released:]
April 2, 2024
\item[Archive size:]
7.08 kB
\href{https://packages.typst.org/preview/letter-pro-3.0.0.tar.gz}{\pandocbounded{\includesvg[keepaspectratio]{/assets/icons/16-download.svg}}}
\item[Repository:]
\href{https://github.com/Sematre/typst-letter-pro}{GitHub}
\item[Categor y :]
\begin{itemize}
\tightlist
\item[]
\item
  \pandocbounded{\includesvg[keepaspectratio]{/assets/icons/16-envelope.svg}}
  \href{https://typst.app/universe/search/?category=office}{Office}
\end{itemize}
\end{description}

\subsubsection{Where to report issues?}\label{where-to-report-issues}

This template is a project of Sematre . Report issues on
\href{https://github.com/Sematre/typst-letter-pro}{their repository} .
You can also try to ask for help with this template on the
\href{https://forum.typst.app}{Forum} .

Please report this template to the Typst team using the
\href{https://typst.app/contact}{contact form} if you believe it is a
safety hazard or infringes upon your rights.

\phantomsection\label{versions}
\subsubsection{Version history}\label{version-history}

\begin{longtable}[]{@{}ll@{}}
\toprule\noalign{}
Version & Release Date \\
\midrule\noalign{}
\endhead
\bottomrule\noalign{}
\endlastfoot
3.0.0 & October 28, 2024 \\
\href{https://typst.app/universe/package/letter-pro/2.1.0/}{2.1.0} &
April 2, 2024 \\
\end{longtable}

Typst GmbH did not create this template and cannot guarantee correct
functionality of this template or compatibility with any version of the
Typst compiler or app.
