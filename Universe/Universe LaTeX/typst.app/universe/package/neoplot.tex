\title{typst.app/universe/package/neoplot}

\phantomsection\label{banner}
\section{neoplot}\label{neoplot}

{ 0.0.2 }

Gnuplot in Typst

\phantomsection\label{readme}
A Typst package to use \href{http://www.gnuplot.info/}{gnuplot} in
Typst.

\begin{Shaded}
\begin{Highlighting}[]
\NormalTok{\#import "@preview/neoplot:0.0.2" as gp}
\end{Highlighting}
\end{Shaded}

Execute gnuplot commands:

\begin{Shaded}
\begin{Highlighting}[]
\NormalTok{\#gp.exec(}
\NormalTok{    kind: "command",}
\NormalTok{    \textasciigrave{}\textasciigrave{}\textasciigrave{}gnuplot}
\NormalTok{    reset;}
\NormalTok{    set samples 1000;}
\NormalTok{    plot sin(x),}
\NormalTok{         cos(x)}
\NormalTok{    \textasciigrave{}\textasciigrave{}\textasciigrave{}}
\NormalTok{)}
\end{Highlighting}
\end{Shaded}

Execute a gnuplot script:

\begin{Shaded}
\begin{Highlighting}[]
\NormalTok{\#gp.exec(}
\NormalTok{    \textasciigrave{}\textasciigrave{}\textasciigrave{}gnuplot}
\NormalTok{    reset}
\NormalTok{    \# Can add comments since it is a script}
\NormalTok{    set samples 1000}
\NormalTok{    \# Use a backslash to extend commands}
\NormalTok{    plot sin(x), \textbackslash{}}
\NormalTok{         cos(x)}
\NormalTok{    \textasciigrave{}\textasciigrave{}\textasciigrave{}}
\NormalTok{)}
\end{Highlighting}
\end{Shaded}

To read a data file:

\begin{verbatim}
# datafile.dat
# x  y
  0  0
  2  4
  4  0
\end{verbatim}

\begin{Shaded}
\begin{Highlighting}[]
\NormalTok{\#gp.exec(}
\NormalTok{    \textasciigrave{}\textasciigrave{}\textasciigrave{}gnuplot}
\NormalTok{    $data \textless{}\textless{}EOD}
\NormalTok{    0  0}
\NormalTok{    2  4}
\NormalTok{    4  0}
\NormalTok{    EOD}
\NormalTok{    plot $data with linespoints}
\NormalTok{    \textasciigrave{}\textasciigrave{}\textasciigrave{}}
\NormalTok{)}
\end{Highlighting}
\end{Shaded}

or

\begin{Shaded}
\begin{Highlighting}[]
\NormalTok{\#gp.exec(}
\NormalTok{    // Use a datablock since Typst doesn\textquotesingle{}t support WASI}
\NormalTok{    "$data \textless{}\textless{}EOD\textbackslash{}n" +}
\NormalTok{    // Load "datafile.dat" using Typst}
\NormalTok{    read("datafile.dat") +}
\NormalTok{    "EOD\textbackslash{}n" +}
\NormalTok{    "plot $data with linespoints"}
\NormalTok{)}
\end{Highlighting}
\end{Shaded}

To print \texttt{\ \$data\ } :

\begin{Shaded}
\begin{Highlighting}[]
\NormalTok{\#gp.exec("print $data")}
\end{Highlighting}
\end{Shaded}

\subsubsection{How to add}\label{how-to-add}

Copy this into your project and use the import as \texttt{\ neoplot\ }

\begin{verbatim}
#import "@preview/neoplot:0.0.2"
\end{verbatim}

\includesvg[width=0.16667in,height=0.16667in]{/assets/icons/16-copy.svg}

Check the docs for
\href{https://typst.app/docs/reference/scripting/\#packages}{more
information on how to import packages} .

\subsubsection{About}\label{about}

\begin{description}
\tightlist
\item[Author :]
\href{https://github.com/KNnut}{KNnut}
\item[License:]
BSD-3-Clause
\item[Current version:]
0.0.2
\item[Last updated:]
July 16, 2024
\item[First released:]
June 17, 2024
\item[Minimum Typst version:]
0.11.1
\item[Archive size:]
512 kB
\href{https://packages.typst.org/preview/neoplot-0.0.2.tar.gz}{\pandocbounded{\includesvg[keepaspectratio]{/assets/icons/16-download.svg}}}
\item[Repository:]
\href{https://github.com/KNnut/neoplot}{GitHub}
\item[Discipline :]
\begin{itemize}
\tightlist
\item[]
\item
  \href{https://typst.app/universe/search/?discipline=mathematics}{Mathematics}
\end{itemize}
\item[Categor ies :]
\begin{itemize}
\tightlist
\item[]
\item
  \pandocbounded{\includesvg[keepaspectratio]{/assets/icons/16-chart.svg}}
  \href{https://typst.app/universe/search/?category=visualization}{Visualization}
\item
  \pandocbounded{\includesvg[keepaspectratio]{/assets/icons/16-integration.svg}}
  \href{https://typst.app/universe/search/?category=integration}{Integration}
\end{itemize}
\end{description}

\subsubsection{Where to report issues?}\label{where-to-report-issues}

This package is a project of KNnut . Report issues on
\href{https://github.com/KNnut/neoplot}{their repository} . You can also
try to ask for help with this package on the
\href{https://forum.typst.app}{Forum} .

Please report this package to the Typst team using the
\href{https://typst.app/contact}{contact form} if you believe it is a
safety hazard or infringes upon your rights.

\phantomsection\label{versions}
\subsubsection{Version history}\label{version-history}

\begin{longtable}[]{@{}ll@{}}
\toprule\noalign{}
Version & Release Date \\
\midrule\noalign{}
\endhead
\bottomrule\noalign{}
\endlastfoot
0.0.2 & July 16, 2024 \\
\href{https://typst.app/universe/package/neoplot/0.0.1/}{0.0.1} & June
17, 2024 \\
\end{longtable}

Typst GmbH did not create this package and cannot guarantee correct
functionality of this package or compatibility with any version of the
Typst compiler or app.
