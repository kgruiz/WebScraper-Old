\title{typst.app/universe/package/clear-iclr}

\phantomsection\label{banner}
\phantomsection\label{template-thumbnail}
\pandocbounded{\includegraphics[keepaspectratio]{https://packages.typst.org/preview/thumbnails/clear-iclr-0.4.0-small.webp}}

\section{clear-iclr}\label{clear-iclr}

{ 0.4.0 }

Paper template for submission to International Conference on Learning
Representations (ICLR)

{ } Featured Template

\href{/app?template=clear-iclr&version=0.4.0}{Create project in app}

\phantomsection\label{readme}
\subsection{Usage}\label{usage}

You can use this template in the Typst web app by clicking \emph{Start
from template} on the dashboard and searching for
\texttt{\ clear-iclr\ } .

Alternatively, you can use the CLI to kick this project off using the
command

\begin{Shaded}
\begin{Highlighting}[]
\NormalTok{typst init @preview/clear{-}iclr}
\end{Highlighting}
\end{Shaded}

Typst will create a new directory with all the files needed to get you
started.

\subsection{Configuration}\label{configuration}

This template exports the \texttt{\ iclr\ } function with the following
named arguments.

\begin{itemize}
\item
  \texttt{\ title\ } : The paper’s title as content.
\item
  \texttt{\ authors\ } : An array of author dictionaries. Each of the
  author dictionaries must have a name key and can have the keys
  department, organization, location, and email.

\begin{Shaded}
\begin{Highlighting}[]
\NormalTok{\#let authors = (}
\NormalTok{  ...,}
\NormalTok{  (}
\NormalTok{    names: ([Coauthor1], [Coauthor2]),}
\NormalTok{    affilation: [Affiliation],}
\NormalTok{    address: [Address],}
\NormalTok{    email: "correspondent@example.org",}
\NormalTok{  ),}
\NormalTok{  ...}
\NormalTok{)}
\end{Highlighting}
\end{Shaded}
\item
  \texttt{\ keywords\ } : Publication keywords (used in PDF metadata).
\item
  \texttt{\ date\ } : Creation date (used in PDF metadata).
\item
  \texttt{\ abstract\ } : The content of a brief summary of the paper or
  none. Appears at the top under the title.
\item
  \texttt{\ bibliography\ } : The result of a call to the bibliography
  function or none. The function also accepts a single, positional
  argument for the body of the paper.
\item
  \texttt{\ appendix\ } : Content to append after bibliography section
  (can be included).
\item
  \texttt{\ accepted\ } : If this is set to \texttt{\ false\ } then
  anonymized ready for submission document is produced;
  \texttt{\ accepted:\ true\ } produces camera-redy version. If the
  argument is set to \texttt{\ none\ } then preprint version is produced
  (can be uploaded to arXiv).
\end{itemize}

The template will initialize your package with a sample call to the
\texttt{\ iclr\ } function in a show rule. If you want to change an
existing project to use this template, you can add a show rule at the
top of your file.

\subsection{Issues}\label{issues}

This template is developed at
\href{https://github.com/daskol/typst-templates}{daskol/typst-templates}
repo. Please report all issues there.

\begin{itemize}
\item
  Common issue is related to Typst’s inablity to produce colored
  annotation. In order to mitigte the issue, we add a script which
  modifies annotations and make them colored.

\begin{Shaded}
\begin{Highlighting}[]
\NormalTok{../colorize{-}annotations.py \textbackslash{}}
\NormalTok{    example{-}paper.typst.pdf example{-}paper{-}colored.typst.pdf}
\end{Highlighting}
\end{Shaded}

  See {[} \href{http://readme.md/}{README.md} {]}{[}3{]} for details.
\item
  The author instructions says that preferable font is MS Times New
  Roman but the official example paper uses serifs like Computer Modern
  and Nimbus font families. Monospace fonts are not specified.
\item
  ICML-like bibliography style. The bibliography slightly differs from
  the one in the original example paper. The main difference is that we
  prefer to use author’s lastname at first place to search an entry
  faster.
\end{itemize}

\href{/app?template=clear-iclr&version=0.4.0}{Create project in app}

\subsubsection{How to use}\label{how-to-use}

Click the button above to create a new project using this template in
the Typst app.

You can also use the Typst CLI to start a new project on your computer
using this command:

\begin{verbatim}
typst init @preview/clear-iclr:0.4.0
\end{verbatim}

\includesvg[width=0.16667in,height=0.16667in]{/assets/icons/16-copy.svg}

\subsubsection{About}\label{about}

\begin{description}
\tightlist
\item[Author :]
\href{mailto:d.bershatsky2@skoltech.ru}{Daniel Bershatsky}
\item[License:]
MIT
\item[Current version:]
0.4.0
\item[Last updated:]
September 11, 2024
\item[First released:]
September 11, 2024
\item[Minimum Typst version:]
0.11.1
\item[Archive size:]
21.4 kB
\href{https://packages.typst.org/preview/clear-iclr-0.4.0.tar.gz}{\pandocbounded{\includesvg[keepaspectratio]{/assets/icons/16-download.svg}}}
\item[Repository:]
\href{https://github.com/daskol/typst-templates}{GitHub}
\item[Discipline s :]
\begin{itemize}
\tightlist
\item[]
\item
  \href{https://typst.app/universe/search/?discipline=computer-science}{Computer
  Science}
\item
  \href{https://typst.app/universe/search/?discipline=mathematics}{Mathematics}
\end{itemize}
\item[Categor y :]
\begin{itemize}
\tightlist
\item[]
\item
  \pandocbounded{\includesvg[keepaspectratio]{/assets/icons/16-atom.svg}}
  \href{https://typst.app/universe/search/?category=paper}{Paper}
\end{itemize}
\end{description}

\subsubsection{Where to report issues?}\label{where-to-report-issues}

This template is a project of Daniel Bershatsky . Report issues on
\href{https://github.com/daskol/typst-templates}{their repository} . You
can also try to ask for help with this template on the
\href{https://forum.typst.app}{Forum} .

Please report this template to the Typst team using the
\href{https://typst.app/contact}{contact form} if you believe it is a
safety hazard or infringes upon your rights.

\phantomsection\label{versions}
\subsubsection{Version history}\label{version-history}

\begin{longtable}[]{@{}ll@{}}
\toprule\noalign{}
Version & Release Date \\
\midrule\noalign{}
\endhead
\bottomrule\noalign{}
\endlastfoot
0.4.0 & September 11, 2024 \\
\end{longtable}

Typst GmbH did not create this template and cannot guarantee correct
functionality of this template or compatibility with any version of the
Typst compiler or app.
