\title{typst.app/universe/package/touying}

\phantomsection\label{banner}
\section{touying}\label{touying}

{ 0.5.3 }

A powerful package for creating presentation slides in Typst.

{ } Featured Package

\phantomsection\label{readme}
\href{https://github.com/touying-typ/touying}{Touying} (投影 in
chinese, /tóuyÇ?ng/, meaning projection) is a user-friendly, powerful
and efficient package for creating presentation slides in Typst. Partial
code is inherited from
\href{https://github.com/andreasKroepelin/polylux}{Polylux} . Therefore,
some concepts and APIs remain consistent with Polylux.

Touying provides automatically injected global configurations, which is
convenient for configuring themes. Besides, Touying does not rely on
\texttt{\ counter\ } and \texttt{\ context\ } to implement
\texttt{\ \#pause\ } , resulting in better performance.

If you like it, consider
\href{https://github.com/touying-typ/touying}{giving a star on GitHub} .
Touying is a community-driven project, feel free to suggest any ideas
and contribute.

\href{https://touying-typ.github.io/}{\pandocbounded{\includegraphics[keepaspectratio]{https://img.shields.io/badge/docs-book-green}}}
\href{https://github.com/touying-typ/touying/wiki}{\pandocbounded{\includegraphics[keepaspectratio]{https://img.shields.io/badge/docs-gallery-orange}}}
\pandocbounded{\includegraphics[keepaspectratio]{https://img.shields.io/github/license/touying-typ/touying}}
\pandocbounded{\includegraphics[keepaspectratio]{https://img.shields.io/github/v/release/touying-typ/touying}}
\pandocbounded{\includegraphics[keepaspectratio]{https://img.shields.io/github/stars/touying-typ/touying}}
\pandocbounded{\includegraphics[keepaspectratio]{https://img.shields.io/badge/themes-6-aqua}}

\subsection{Document}\label{document}

Read \href{https://touying-typ.github.io/}{the document} to learn all
about Touying.

We will maintain \textbf{English} and \textbf{Chinese} versions of the
documentation for Touying, and for each major version, we will maintain
a documentation copy. This allows you to easily refer to old versions of
the Touying documentation and migrate to new versions.

\textbf{Note that the documentation may be outdated, and you can also
use Tinymist to view Touying’s annotated documentation by hovering
over the code.}

\subsection{Gallery}\label{gallery}

Touying offers \href{https://github.com/touying-typ/touying/wiki}{a
gallery page} via wiki, where you can browse elegant slides created by
Touying users. You’re also encouraged to contribute your own beautiful
slides here!

\subsection{Special Features}\label{special-features}

\begin{enumerate}
\tightlist
\item
  Split slides by headings
  \href{https://touying-typ.github.io/docs/sections}{document}
\end{enumerate}

\begin{Shaded}
\begin{Highlighting}[]
\NormalTok{= Section}

\NormalTok{== Subsection}

\NormalTok{=== First Slide}

\NormalTok{Hello, Touying!}

\NormalTok{=== Second Slide}

\NormalTok{Hello, Typst!}
\end{Highlighting}
\end{Shaded}

\begin{enumerate}
\setcounter{enumi}{1}
\tightlist
\item
  \texttt{\ \#pause\ } and \texttt{\ \#meanwhile\ } animations
  \href{https://touying-typ.github.io/docs/dynamic/simple}{document}
\end{enumerate}

\begin{Shaded}
\begin{Highlighting}[]
\NormalTok{\#slide[}
\NormalTok{  First}

\NormalTok{  \#pause}

\NormalTok{  Second}

\NormalTok{  \#meanwhile}

\NormalTok{  Third}

\NormalTok{  \#pause}

\NormalTok{  Fourth}
\NormalTok{]}
\end{Highlighting}
\end{Shaded}

\pandocbounded{\includegraphics[keepaspectratio]{https://github.com/touying-typ/touying/assets/34951714/24ca19a3-b27c-4d31-ab75-09c37911e6ac}}

\begin{enumerate}
\setcounter{enumi}{2}
\tightlist
\item
  Math Equation Animation
  \href{https://touying-typ.github.io/docs/dynamic/equation}{document}
\end{enumerate}

\pandocbounded{\includegraphics[keepaspectratio]{https://github.com/touying-typ/touying/assets/34951714/8640fe0a-95e4-46ac-b570-c8c79f993de4}}

\begin{enumerate}
\setcounter{enumi}{3}
\tightlist
\item
  \texttt{\ touying-reducer\ } Cetz and Fletcher Animations
  \href{https://touying-typ.github.io/docs/dynamic/other}{document}
\end{enumerate}

\pandocbounded{\includegraphics[keepaspectratio]{https://github.com/touying-typ/touying/assets/34951714/9ba71f54-2a5d-4144-996c-4a42833cc5cc}}

\begin{enumerate}
\setcounter{enumi}{4}
\tightlist
\item
  Correct outline and bookmark (no duplicate and correct page number)
\end{enumerate}

\pandocbounded{\includegraphics[keepaspectratio]{https://github.com/touying-typ/touying/assets/34951714/7b62fcaf-6342-4dba-901b-818c16682529}}

\begin{enumerate}
\setcounter{enumi}{5}
\tightlist
\item
  Dewdrop Theme Navigation Bar
  \href{https://touying-typ.github.io/docs/themes/dewdrop}{document}
\end{enumerate}

\pandocbounded{\includegraphics[keepaspectratio]{https://github.com/touying-typ/touying/assets/34951714/0426516d-aa3c-4b7a-b7b6-2d5d276fb971}}

\begin{enumerate}
\setcounter{enumi}{6}
\tightlist
\item
  Semi-transparent cover mode
  \href{https://touying-typ.github.io/docs/dynamic/cover}{document}
\end{enumerate}

\pandocbounded{\includegraphics[keepaspectratio]{https://github.com/touying-typ/touying/assets/34951714/22a9ea66-c8b5-431e-a52c-2c8ca3f18e49}}

\begin{enumerate}
\setcounter{enumi}{7}
\tightlist
\item
  Speaker notes for dual-screen
  \href{https://touying-typ.github.io/docs/external/pympress}{document}
\end{enumerate}

\pandocbounded{\includegraphics[keepaspectratio]{https://github.com/touying-typ/touying/assets/34951714/afbe17cb-46d4-4507-90e8-959c53de95d5}}

\begin{enumerate}
\setcounter{enumi}{8}
\tightlist
\item
  Export slides to PPTX and HTML formats and show presentation online.
  \href{https://github.com/touying-typ/touying-exporter}{touying-exporter}
  \href{https://github.com/touying-typ/touying-template}{touying-template}
  \href{https://touying-typ.github.io/touying-template/}{online}
\end{enumerate}

\pandocbounded{\includegraphics[keepaspectratio]{https://github.com/touying-typ/touying-exporter/assets/34951714/207ddffc-87c8-4976-9bf4-4c6c5e2573ea}}

\subsection{Quick start}\label{quick-start}

Before you begin, make sure you have installed the Typst environment. If
not, you can use the \href{https://typst.app/}{Web App} or the
\href{https://marketplace.visualstudio.com/items?itemName=myriad-dreamin.tinymist}{Tinymist
LSP} extensions for VS Code.

To use Touying, you only need to include the following code in your
document:

\begin{Shaded}
\begin{Highlighting}[]
\NormalTok{\#import "@preview/touying:0.5.3": *}
\NormalTok{\#import themes.simple: *}

\NormalTok{\#show: simple{-}theme.with(aspect{-}ratio: "16{-}9")}

\NormalTok{= Title}

\NormalTok{== First Slide}

\NormalTok{Hello, Touying!}

\NormalTok{\#pause}

\NormalTok{Hello, Typst!}
\end{Highlighting}
\end{Shaded}

\pandocbounded{\includegraphics[keepaspectratio]{https://github.com/touying-typ/touying/assets/34951714/f5bdbf8f-7bf9-45fd-9923-0fa5d66450b2}}

It’s simple. Congratulations on creating your first Touying slide!
🎉

\textbf{Tip:} You can use Typst syntax like
\texttt{\ \#import\ "config.typ":\ *\ } or
\texttt{\ \#include\ "content.typ"\ } to implement Touying’s
multi-file architecture.

\subsection{More Complex Examples}\label{more-complex-examples}

In fact, Touying provides various styles for writing slides. For
example, the above example uses first-level and second-level titles to
create new slides. However, you can also use the
\texttt{\ \#slide{[}..{]}\ } format to access more powerful features
provided by Touying.

\begin{Shaded}
\begin{Highlighting}[]
\NormalTok{\#import "@preview/touying:0.5.3": *}
\NormalTok{\#import themes.university: *}
\NormalTok{\#import "@preview/cetz:0.2.2"}
\NormalTok{\#import "@preview/fletcher:0.5.1" as fletcher: node, edge}
\NormalTok{\#import "@preview/ctheorems:1.1.2": *}
\NormalTok{\#import "@preview/numbly:0.1.0": numbly}

\NormalTok{// cetz and fletcher bindings for touying}
\NormalTok{\#let cetz{-}canvas = touying{-}reducer.with(reduce: cetz.canvas, cover: cetz.draw.hide.with(bounds: true))}
\NormalTok{\#let fletcher{-}diagram = touying{-}reducer.with(reduce: fletcher.diagram, cover: fletcher.hide)}

\NormalTok{// Theorems configuration by ctheorems}
\NormalTok{\#show: thmrules.with(qed{-}symbol: $square$)}
\NormalTok{\#let theorem = thmbox("theorem", "Theorem", fill: rgb("\#eeffee"))}
\NormalTok{\#let corollary = thmplain(}
\NormalTok{  "corollary",}
\NormalTok{  "Corollary",}
\NormalTok{  base: "theorem",}
\NormalTok{  titlefmt: strong}
\NormalTok{)}
\NormalTok{\#let definition = thmbox("definition", "Definition", inset: (x: 1.2em, top: 1em))}
\NormalTok{\#let example = thmplain("example", "Example").with(numbering: none)}
\NormalTok{\#let proof = thmproof("proof", "Proof")}

\NormalTok{\#show: university{-}theme.with(}
\NormalTok{  aspect{-}ratio: "16{-}9",}
\NormalTok{  // config{-}common(handout: true),}
\NormalTok{  config{-}info(}
\NormalTok{    title: [Title],}
\NormalTok{    subtitle: [Subtitle],}
\NormalTok{    author: [Authors],}
\NormalTok{    date: datetime.today(),}
\NormalTok{    institution: [Institution],}
\NormalTok{    logo: emoji.school,}
\NormalTok{  ),}
\NormalTok{)}

\NormalTok{\#set heading(numbering: numbly("\{1\}.", default: "1.1"))}

\NormalTok{\#title{-}slide()}

\NormalTok{== Outline \textless{}touying:hidden\textgreater{}}

\NormalTok{\#components.adaptive{-}columns(outline(title: none, indent: 1em))}

\NormalTok{= Animation}

\NormalTok{== Simple Animation}

\NormalTok{We can use \textasciigrave{}\#pause\textasciigrave{} to \#pause display something later.}

\NormalTok{\#pause}

\NormalTok{Just like this.}

\NormalTok{\#meanwhile}

\NormalTok{Meanwhile, \#pause we can also use \textasciigrave{}\#meanwhile\textasciigrave{} to \#pause display other content synchronously.}

\NormalTok{\#speaker{-}note[}
\NormalTok{  + This is a speaker note.}
\NormalTok{  + You won\textquotesingle{}t see it unless you use \textasciigrave{}config{-}common(show{-}notes{-}on{-}second{-}screen: right)\textasciigrave{}}
\NormalTok{]}


\NormalTok{== Complex Animation}

\NormalTok{At subslide \#touying{-}fn{-}wrapper((self: none) =\textgreater{} str(self.subslide)), we can}

\NormalTok{use \#uncover("2{-}")[\textasciigrave{}\#uncover\textasciigrave{} function] for reserving space,}

\NormalTok{use \#only("2{-}")[\textasciigrave{}\#only\textasciigrave{} function] for not reserving space,}

\NormalTok{\#alternatives[call \textasciigrave{}\#only\textasciigrave{} multiple times \textbackslash{}u\{2717\}][use \textasciigrave{}\#alternatives\textasciigrave{} function \#sym.checkmark] for choosing one of the alternatives.}


\NormalTok{== Callback Style Animation}

\NormalTok{\#slide(repeat: 3, self =\textgreater{} [}
\NormalTok{  \#let (uncover, only, alternatives) = utils.methods(self)}

\NormalTok{  At subslide \#self.subslide, we can}

\NormalTok{  use \#uncover("2{-}")[\textasciigrave{}\#uncover\textasciigrave{} function] for reserving space,}

\NormalTok{  use \#only("2{-}")[\textasciigrave{}\#only\textasciigrave{} function] for not reserving space,}

\NormalTok{  \#alternatives[call \textasciigrave{}\#only\textasciigrave{} multiple times \textbackslash{}u\{2717\}][use \textasciigrave{}\#alternatives\textasciigrave{} function \#sym.checkmark] for choosing one of the alternatives.}
\NormalTok{])}


\NormalTok{== Math Equation Animation}

\NormalTok{Equation with \textasciigrave{}pause\textasciigrave{}:}

\NormalTok{$}
\NormalTok{  f(x) \&= pause x\^{}2 + 2x + 1  \textbackslash{}}
\NormalTok{       \&= pause (x + 1)\^{}2  \textbackslash{}}
\NormalTok{$}

\NormalTok{\#meanwhile}

\NormalTok{Here, \#pause we have the expression of $f(x)$.}

\NormalTok{\#pause}

\NormalTok{By factorizing, we can obtain this result.}


\NormalTok{== CeTZ Animation}

\NormalTok{CeTZ Animation in Touying:}

\NormalTok{\#cetz{-}canvas(\{}
\NormalTok{  import cetz.draw: *}
  
\NormalTok{  rect((0,0), (5,5))}

\NormalTok{  (pause,)}

\NormalTok{  rect((0,0), (1,1))}
\NormalTok{  rect((1,1), (2,2))}
\NormalTok{  rect((2,2), (3,3))}

\NormalTok{  (pause,)}

\NormalTok{  line((0,0), (2.5, 2.5), name: "line")}
\NormalTok{\})}


\NormalTok{== Fletcher Animation}

\NormalTok{Fletcher Animation in Touying:}

\NormalTok{\#fletcher{-}diagram(}
\NormalTok{  node{-}stroke: .1em,}
\NormalTok{  node{-}fill: gradient.radial(blue.lighten(80\%), blue, center: (30\%, 20\%), radius: 80\%),}
\NormalTok{  spacing: 4em,}
\NormalTok{  edge(({-}1,0), "r", "{-}|\textgreater{}", \textasciigrave{}open(path)\textasciigrave{}, label{-}pos: 0, label{-}side: center),}
\NormalTok{  node((0,0), \textasciigrave{}reading\textasciigrave{}, radius: 2em),}
\NormalTok{  edge((0,0), (0,0), \textasciigrave{}read()\textasciigrave{}, "{-}{-}|\textgreater{}", bend: 130deg),}
\NormalTok{  pause,}
\NormalTok{  edge(\textasciigrave{}read()\textasciigrave{}, "{-}|\textgreater{}"),}
\NormalTok{  node((1,0), \textasciigrave{}eof\textasciigrave{}, radius: 2em),}
\NormalTok{  pause,}
\NormalTok{  edge(\textasciigrave{}close()\textasciigrave{}, "{-}|\textgreater{}"),}
\NormalTok{  node((2,0), \textasciigrave{}closed\textasciigrave{}, radius: 2em, extrude: ({-}2.5, 0)),}
\NormalTok{  edge((0,0), (2,0), \textasciigrave{}close()\textasciigrave{}, "{-}|\textgreater{}", bend: {-}40deg),}
\NormalTok{)}


\NormalTok{= Theorems}

\NormalTok{== Prime numbers}

\NormalTok{\#definition[}
\NormalTok{  A natural number is called a \#highlight[\_prime number\_] if it is greater}
\NormalTok{  than 1 and cannot be written as the product of two smaller natural numbers.}
\NormalTok{]}
\NormalTok{\#example[}
\NormalTok{  The numbers $2$, $3$, and $17$ are prime.}
\NormalTok{  @cor\_largest\_prime shows that this list is not exhaustive!}
\NormalTok{]}

\NormalTok{\#theorem("Euclid")[}
\NormalTok{  There are infinitely many primes.}
\NormalTok{]}
\NormalTok{\#proof[}
\NormalTok{  Suppose to the contrary that $p\_1, p\_2, dots, p\_n$ is a finite enumeration}
\NormalTok{  of all primes. Set $P = p\_1 p\_2 dots p\_n$. Since $P + 1$ is not in our list,}
\NormalTok{  it cannot be prime. Thus, some prime factor $p\_j$ divides $P + 1$.  Since}
\NormalTok{  $p\_j$ also divides $P$, it must divide the difference $(P + 1) {-} P = 1$, a}
\NormalTok{  contradiction.}
\NormalTok{]}

\NormalTok{\#corollary[}
\NormalTok{  There is no largest prime number.}
\NormalTok{] \textless{}cor\_largest\_prime\textgreater{}}
\NormalTok{\#corollary[}
\NormalTok{  There are infinitely many composite numbers.}
\NormalTok{]}

\NormalTok{\#theorem[}
\NormalTok{  There are arbitrarily long stretches of composite numbers.}
\NormalTok{]}

\NormalTok{\#proof[}
\NormalTok{  For any $n \textgreater{} 2$, consider $}
\NormalTok{    n! + 2, quad n! + 3, quad ..., quad n! + n \#qedhere}
\NormalTok{  $}
\NormalTok{]}


\NormalTok{= Others}

\NormalTok{== Side{-}by{-}side}

\NormalTok{\#slide(composer: (1fr, 1fr))[}
\NormalTok{  First column.}
\NormalTok{][}
\NormalTok{  Second column.}
\NormalTok{]}


\NormalTok{== Multiple Pages}

\NormalTok{\#lorem(200)}


\NormalTok{\#show: appendix}

\NormalTok{= Appendix}

\NormalTok{== Appendix}

\NormalTok{Please pay attention to the current slide number.}
\end{Highlighting}
\end{Shaded}

\pandocbounded{\includegraphics[keepaspectratio]{https://github.com/user-attachments/assets/3488f256-a0b3-43d0-a266-009d9d0a7bd3}}

\subsection{Acknowledgements}\label{acknowledgements}

Thanks to…

\begin{itemize}
\tightlist
\item
  \href{https://github.com/andreasKroepelin}{@andreasKroepelin} for the
  \texttt{\ polylux\ } package
\item
  \href{https://github.com/Enivex}{@Enivex} for the
  \texttt{\ metropolis\ } theme
\item
  \href{https://github.com/drupol}{@drupol} for the
  \texttt{\ university\ } theme
\item
  \href{https://github.com/pride7}{@pride7} for the \texttt{\ aqua\ }
  theme
\item
  \href{https://github.com/Coekjan}{@Coekjan} and
  \href{https://github.com/QuadnucYard}{@QuadnucYard} for the
  \texttt{\ stargazer\ } theme
\item
  \href{https://github.com/ntjess}{@ntjess} for contributing to
  \texttt{\ fit-to-height\ } , \texttt{\ fit-to-width\ } and
  \texttt{\ cover-with-rect\ }
\end{itemize}

\subsection{Poster}\label{poster}

\pandocbounded{\includegraphics[keepaspectratio]{https://github.com/user-attachments/assets/e1ddb672-8e8f-472d-b364-b8caed1da16b}}

\href{https://github.com/touying-typ/touying-poster}{View Code}

\subsection{Star History}\label{star-history}

\href{https://star-history.com/\#touying-typ/touying&Date}{\pandocbounded{\includegraphics[keepaspectratio]{https://api.star-history.com/svg?repos=touying-typ/touying&type=Date}}}

\subsubsection{How to add}\label{how-to-add}

Copy this into your project and use the import as \texttt{\ touying\ }

\begin{verbatim}
#import "@preview/touying:0.5.3"
\end{verbatim}

\includesvg[width=0.16667in,height=0.16667in]{/assets/icons/16-copy.svg}

Check the docs for
\href{https://typst.app/docs/reference/scripting/\#packages}{more
information on how to import packages} .

\subsubsection{About}\label{about}

\begin{description}
\tightlist
\item[Author s :]
OrangeX4 , Andreas Kröpelin , ntjess , Enivex , Pol Dellaiera , pride7
, \& Coekjan
\item[License:]
MIT
\item[Current version:]
0.5.3
\item[Last updated:]
October 15, 2024
\item[First released:]
January 11, 2024
\item[Minimum Typst version:]
0.11.0
\item[Archive size:]
302 kB
\href{https://packages.typst.org/preview/touying-0.5.3.tar.gz}{\pandocbounded{\includesvg[keepaspectratio]{/assets/icons/16-download.svg}}}
\item[Repository:]
\href{https://github.com/touying-typ/touying}{GitHub}
\item[Categor y :]
\begin{itemize}
\tightlist
\item[]
\item
  \pandocbounded{\includesvg[keepaspectratio]{/assets/icons/16-presentation.svg}}
  \href{https://typst.app/universe/search/?category=presentation}{Presentation}
\end{itemize}
\end{description}

\subsubsection{Where to report issues?}\label{where-to-report-issues}

This package is a project of OrangeX4, Andreas Kröpelin, ntjess,
Enivex, Pol Dellaiera, pride7, and Coekjan . Report issues on
\href{https://github.com/touying-typ/touying}{their repository} . You
can also try to ask for help with this package on the
\href{https://forum.typst.app}{Forum} .

Please report this package to the Typst team using the
\href{https://typst.app/contact}{contact form} if you believe it is a
safety hazard or infringes upon your rights.

\phantomsection\label{versions}
\subsubsection{Version history}\label{version-history}

\begin{longtable}[]{@{}ll@{}}
\toprule\noalign{}
Version & Release Date \\
\midrule\noalign{}
\endhead
\bottomrule\noalign{}
\endlastfoot
0.5.3 & October 15, 2024 \\
\href{https://typst.app/universe/package/touying/0.5.2/}{0.5.2} &
September 3, 2024 \\
\href{https://typst.app/universe/package/touying/0.5.1/}{0.5.1} &
September 3, 2024 \\
\href{https://typst.app/universe/package/touying/0.5.0/}{0.5.0} &
September 2, 2024 \\
\href{https://typst.app/universe/package/touying/0.4.2/}{0.4.2} & May
27, 2024 \\
\href{https://typst.app/universe/package/touying/0.4.1/}{0.4.1} & May
13, 2024 \\
\href{https://typst.app/universe/package/touying/0.4.0/}{0.4.0} & April
6, 2024 \\
\href{https://typst.app/universe/package/touying/0.3.3/}{0.3.3} & March
26, 2024 \\
\href{https://typst.app/universe/package/touying/0.3.2/}{0.3.2} & March
15, 2024 \\
\href{https://typst.app/universe/package/touying/0.3.1/}{0.3.1} & March
7, 2024 \\
\href{https://typst.app/universe/package/touying/0.3.0/}{0.3.0} & March
6, 2024 \\
\href{https://typst.app/universe/package/touying/0.2.1/}{0.2.1} &
February 17, 2024 \\
\href{https://typst.app/universe/package/touying/0.2.0/}{0.2.0} &
January 20, 2024 \\
\href{https://typst.app/universe/package/touying/0.1.0/}{0.1.0} &
January 11, 2024 \\
\end{longtable}

Typst GmbH did not create this package and cannot guarantee correct
functionality of this package or compatibility with any version of the
Typst compiler or app.
