\title{typst.app/docs/reference/introspection/state}

\begin{itemize}
\tightlist
\item
  \href{/docs}{\includesvg[width=0.16667in,height=0.16667in]{/assets/icons/16-docs-dark.svg}}
\item
  \includesvg[width=0.16667in,height=0.16667in]{/assets/icons/16-arrow-right.svg}
\item
  \href{/docs/reference/}{Reference}
\item
  \includesvg[width=0.16667in,height=0.16667in]{/assets/icons/16-arrow-right.svg}
\item
  \href{/docs/reference/introspection/}{Introspection}
\item
  \includesvg[width=0.16667in,height=0.16667in]{/assets/icons/16-arrow-right.svg}
\item
  \href{/docs/reference/introspection/state/}{State}
\end{itemize}

\section{\texorpdfstring{{ state }}{ state }}\label{summary}

Manages stateful parts of your document.

Let\textquotesingle s say you have some computations in your document
and want to remember the result of your last computation to use it in
the next one. You might try something similar to the code below and
expect it to output 10, 13, 26, and 21. However this \textbf{does not
work} in Typst. If you test this code, you will see that Typst complains
with the following error message: \emph{Variables from outside the
function are read-only and cannot be modified.}

\begin{verbatim}
// This doesn't work!
#let x = 0
#let compute(expr) = {
  x = eval(
    expr.replace("x", str(x))
  )
  [New value is #x. ]
}

#compute("10") \
#compute("x + 3") \
#compute("x * 2") \
#compute("x - 5")
\end{verbatim}

\subsection{State and document markup}\label{state-and-markup}

Why does it do that? Because, in general, this kind of computation with
side effects is problematic in document markup and Typst is upfront
about that. For the results to make sense, the computation must proceed
in the same order in which the results will be laid out in the document.
In our simple example, that\textquotesingle s the case, but in general
it might not be.

Let\textquotesingle s look at a slightly different, but similar kind of
state: The heading numbering. We want to increase the heading counter at
each heading. Easy enough, right? Just add one. Well,
it\textquotesingle s not that simple. Consider the following example:

\begin{verbatim}
#set heading(numbering: "1.")
#let template(body) = [
  = Outline
  ...
  #body
]

#show: template

= Introduction
...
\end{verbatim}

\includegraphics[width=5in,height=\textheight,keepaspectratio]{/assets/docs/OC8Yphz4-mFQhH6Mm9lwwAAAAAAAAAAA.png}

Here, Typst first processes the body of the document after the show
rule, sees the \texttt{\ Introduction\ } heading, then passes the
resulting content to the \texttt{\ template\ } function and only then
sees the \texttt{\ Outline\ } . Just counting up would number the
\texttt{\ Introduction\ } with \texttt{\ 1\ } and the
\texttt{\ Outline\ } with \texttt{\ 2\ } .

\subsection{Managing state in Typst}\label{state-in-typst}

So what do we do instead? We use Typst\textquotesingle s state
management system. Calling the \texttt{\ state\ } function with an
identifying string key and an optional initial value gives you a state
value which exposes a few functions. The two most important ones are
\texttt{\ get\ } and \texttt{\ update\ } :

\begin{itemize}
\item
  The
  \href{/docs/reference/introspection/state/\#definitions-get}{\texttt{\ get\ }}
  function retrieves the current value of the state. Because the value
  can vary over the course of the document, it is a \emph{contextual}
  function that can only be used when
  \href{/docs/reference/context/}{context} is available.
\item
  The
  \href{/docs/reference/introspection/state/\#definitions-update}{\texttt{\ update\ }}
  function modifies the state. You can give it any value. If given a
  non-function value, it sets the state to that value. If given a
  function, that function receives the previous state and has to return
  the new state.
\end{itemize}

Our initial example would now look like this:

\begin{verbatim}
#let s = state("x", 0)
#let compute(expr) = [
  #s.update(x =>
    eval(expr.replace("x", str(x)))
  )
  New value is #context s.get().
]

#compute("10") \
#compute("x + 3") \
#compute("x * 2") \
#compute("x - 5")
\end{verbatim}

\includegraphics[width=5in,height=\textheight,keepaspectratio]{/assets/docs/TvB3cSxy6XwQVp0EXZ9-ewAAAAAAAAAA.png}

State managed by Typst is always updated in layout order, not in
evaluation order. The \texttt{\ update\ } method returns content and its
effect occurs at the position where the returned content is inserted
into the document.

As a result, we can now also store some of the computations in
variables, but they still show the correct results:

\begin{verbatim}
...

#let more = [
  #compute("x * 2") \
  #compute("x - 5")
]

#compute("10") \
#compute("x + 3") \
#more
\end{verbatim}

\includegraphics[width=5in,height=\textheight,keepaspectratio]{/assets/docs/leSHwxlkl8fBohZKt4lM4AAAAAAAAAAA.png}

This example is of course a bit silly, but in practice this is often
exactly what you want! A good example are heading counters, which is why
Typst\textquotesingle s
\href{/docs/reference/introspection/counter/}{counting system} is very
similar to its state system.

\subsection{Time Travel}\label{time-travel}

By using Typst\textquotesingle s state management system you also get
time travel capabilities! We can find out what the value of the state
will be at any position in the document from anywhere else. In
particular, the \texttt{\ at\ } method gives us the value of the state
at any particular location and the \texttt{\ final\ } methods gives us
the value of the state at the end of the document.

\begin{verbatim}
...

Value at `<here>` is
#context s.at(<here>)

#compute("10") \
#compute("x + 3") \
*Here.* <here> \
#compute("x * 2") \
#compute("x - 5")
\end{verbatim}

\includegraphics[width=5in,height=\textheight,keepaspectratio]{/assets/docs/FSbY2IZskPNKeQtPqbjroAAAAAAAAAAA.png}

\subsection{A word of caution}\label{caution}

To resolve the values of all states, Typst evaluates parts of your code
multiple times. However, there is no guarantee that your state
manipulation can actually be completely resolved.

For instance, if you generate state updates depending on the final value
of a state, the results might never converge. The example below
illustrates this. We initialize our state with \texttt{\ 1\ } and then
update it to its own final value plus 1. So it should be \texttt{\ 2\ }
, but then its final value is \texttt{\ 2\ } , so it should be
\texttt{\ 3\ } , and so on. This example displays a finite value because
Typst simply gives up after a few attempts.

\begin{verbatim}
// This is bad!
#let s = state("x", 1)
#context s.update(s.final() + 1)
#context s.get()
\end{verbatim}

\includegraphics[width=5in,height=\textheight,keepaspectratio]{/assets/docs/4ABrNAaHVbvzCF9JEmUebAAAAAAAAAAA.png}

In general, you should try not to generate state updates from within
context expressions. If possible, try to express your updates as
non-contextual values or functions that compute the new value from the
previous value. Sometimes, it cannot be helped, but in those cases it is
up to you to ensure that the result converges.

\subsection{\texorpdfstring{Constructor
{}}{Constructor }}\label{constructor}

\phantomsection\label{constructor-constructor-tooltip}
If a type has a constructor, you can call it like a function to create a
new value of the type.

Create a new state identified by a key.

{ state } (

{ \href{/docs/reference/foundations/str/}{str} , } { { any } , }

) -\textgreater{} \href{/docs/reference/introspection/state/}{state}

\paragraph{\texorpdfstring{\texttt{\ key\ }}{ key }}\label{constructor-key}

\href{/docs/reference/foundations/str/}{str}

{Required} {{ Positional }}

\phantomsection\label{constructor-key-positional-tooltip}
Positional parameters are specified in order, without names.

The key that identifies this state.

\paragraph{\texorpdfstring{\texttt{\ init\ }}{ init }}\label{constructor-init}

{ any }

{{ Positional }}

\phantomsection\label{constructor-init-positional-tooltip}
Positional parameters are specified in order, without names.

The initial value of the state.

Default: \texttt{\ }{\texttt{\ none\ }}\texttt{\ }

\subsection{\texorpdfstring{{ Definitions
}}{ Definitions }}\label{definitions}

\phantomsection\label{definitions-tooltip}
Functions and types and can have associated definitions. These are
accessed by specifying the function or type, followed by a period, and
then the definition\textquotesingle s name.

\subsubsection{\texorpdfstring{\texttt{\ get\ } {{ Contextual
}}}{ get   Contextual }}\label{definitions-get}

\phantomsection\label{definitions-get-contextual-tooltip}
Contextual functions can only be used when the context is known

Retrieves the value of the state at the current location.

This is equivalent to
\texttt{\ state\ }{\texttt{\ .\ }}\texttt{\ }{\texttt{\ at\ }}\texttt{\ }{\texttt{\ (\ }}\texttt{\ }{\texttt{\ here\ }}\texttt{\ }{\texttt{\ (\ }}\texttt{\ }{\texttt{\ )\ }}\texttt{\ }{\texttt{\ )\ }}\texttt{\ }
.

self { . } { get } (

) -\textgreater{} { any }

\subsubsection{\texorpdfstring{\texttt{\ at\ } {{ Contextual
}}}{ at   Contextual }}\label{definitions-at}

\phantomsection\label{definitions-at-contextual-tooltip}
Contextual functions can only be used when the context is known

Retrieves the value of the state at the given selector\textquotesingle s
unique match.

The \texttt{\ selector\ } must match exactly one element in the
document. The most useful kinds of selectors for this are
\href{/docs/reference/foundations/label/}{labels} and
\href{/docs/reference/introspection/location/}{locations} .

\emph{Compatibility:} For compatibility with Typst 0.10 and lower, this
function also works without a known context if the \texttt{\ selector\ }
is a location. This behaviour will be removed in a future release.

self { . } { at } (

{ \href{/docs/reference/foundations/label/}{label}
\href{/docs/reference/foundations/selector/}{selector}
\href{/docs/reference/introspection/location/}{location}
\href{/docs/reference/foundations/function/}{function} }

) -\textgreater{} { any }

\paragraph{\texorpdfstring{\texttt{\ selector\ }}{ selector }}\label{definitions-at-selector}

\href{/docs/reference/foundations/label/}{label} {or}
\href{/docs/reference/foundations/selector/}{selector} {or}
\href{/docs/reference/introspection/location/}{location} {or}
\href{/docs/reference/foundations/function/}{function}

{Required} {{ Positional }}

\phantomsection\label{definitions-at-selector-positional-tooltip}
Positional parameters are specified in order, without names.

The place at which the state\textquotesingle s value should be
retrieved.

\subsubsection{\texorpdfstring{\texttt{\ final\ } {{ Contextual
}}}{ final   Contextual }}\label{definitions-final}

\phantomsection\label{definitions-final-contextual-tooltip}
Contextual functions can only be used when the context is known

Retrieves the value of the state at the end of the document.

self { . } { final } (

{ \href{/docs/reference/foundations/none/}{none}
\href{/docs/reference/introspection/location/}{location} }

) -\textgreater{} { any }

\paragraph{\texorpdfstring{\texttt{\ location\ }}{ location }}\label{definitions-final-location}

\href{/docs/reference/foundations/none/}{none} {or}
\href{/docs/reference/introspection/location/}{location}

{{ Positional }}

\phantomsection\label{definitions-final-location-positional-tooltip}
Positional parameters are specified in order, without names.

\emph{Compatibility:} This argument is deprecated. It only exists for
compatibility with Typst 0.10 and lower and shouldn\textquotesingle t be
used anymore.

Default: \texttt{\ }{\texttt{\ none\ }}\texttt{\ }

\subsubsection{\texorpdfstring{\texttt{\ update\ }}{ update }}\label{definitions-update}

Update the value of the state.

The update will be in effect at the position where the returned content
is inserted into the document. If you don\textquotesingle t put the
output into the document, nothing happens! This would be the case, for
example, if you write
\texttt{\ }{\texttt{\ let\ }}\texttt{\ \_\ }{\texttt{\ =\ }}\texttt{\ }{\texttt{\ state\ }}\texttt{\ }{\texttt{\ (\ }}\texttt{\ }{\texttt{\ "key"\ }}\texttt{\ }{\texttt{\ )\ }}\texttt{\ }{\texttt{\ .\ }}\texttt{\ }{\texttt{\ update\ }}\texttt{\ }{\texttt{\ (\ }}\texttt{\ }{\texttt{\ 7\ }}\texttt{\ }{\texttt{\ )\ }}\texttt{\ }
. State updates are always applied in layout order and in that case,
Typst wouldn\textquotesingle t know when to update the state.

self { . } { update } (

{ { any } \href{/docs/reference/foundations/function/}{function} }

) -\textgreater{} \href{/docs/reference/foundations/content/}{content}

\paragraph{\texorpdfstring{\texttt{\ update\ }}{ update }}\label{definitions-update-update}

{ any } {or} \href{/docs/reference/foundations/function/}{function}

{Required} {{ Positional }}

\phantomsection\label{definitions-update-update-positional-tooltip}
Positional parameters are specified in order, without names.

If given a non function-value, sets the state to that value. If given a
function, that function receives the previous state and has to return
the new state.

\subsubsection{\texorpdfstring{\texttt{\ display\ }}{ display }}\label{definitions-display}

Displays the current value of the state.

\textbf{Deprecation planned:} Use
\href{/docs/reference/introspection/state/\#definitions-get}{\texttt{\ get\ }}
instead.

self { . } { display } (

{ \href{/docs/reference/foundations/none/}{none}
\href{/docs/reference/foundations/function/}{function} }

) -\textgreater{} \href{/docs/reference/foundations/content/}{content}

\paragraph{\texorpdfstring{\texttt{\ func\ }}{ func }}\label{definitions-display-func}

\href{/docs/reference/foundations/none/}{none} {or}
\href{/docs/reference/foundations/function/}{function}

{{ Positional }}

\phantomsection\label{definitions-display-func-positional-tooltip}
Positional parameters are specified in order, without names.

A function which receives the value of the state and can return
arbitrary content which is then displayed. If this is omitted, the value
is directly displayed.

Default: \texttt{\ }{\texttt{\ none\ }}\texttt{\ }

\href{/docs/reference/introspection/query/}{\pandocbounded{\includesvg[keepaspectratio]{/assets/icons/16-arrow-right.svg}}}

{ Query } { Previous page }

\href{/docs/reference/data-loading/}{\pandocbounded{\includesvg[keepaspectratio]{/assets/icons/16-arrow-right.svg}}}

{ Data Loading } { Next page }
