\title{typst.app/docs/reference/text/underline}

\begin{itemize}
\tightlist
\item
  \href{/docs}{\includesvg[width=0.16667in,height=0.16667in]{/assets/icons/16-docs-dark.svg}}
\item
  \includesvg[width=0.16667in,height=0.16667in]{/assets/icons/16-arrow-right.svg}
\item
  \href{/docs/reference/}{Reference}
\item
  \includesvg[width=0.16667in,height=0.16667in]{/assets/icons/16-arrow-right.svg}
\item
  \href{/docs/reference/text/}{Text}
\item
  \includesvg[width=0.16667in,height=0.16667in]{/assets/icons/16-arrow-right.svg}
\item
  \href{/docs/reference/text/underline/}{Underline}
\end{itemize}

\section{\texorpdfstring{\texttt{\ underline\ } {{ Element
}}}{ underline   Element }}\label{summary}

\phantomsection\label{element-tooltip}
Element functions can be customized with \texttt{\ set\ } and
\texttt{\ show\ } rules.

Underlines text.

\subsection{Example}\label{example}

\begin{verbatim}
This is #underline[important].
\end{verbatim}

\includegraphics[width=5in,height=\textheight,keepaspectratio]{/assets/docs/xV-Fy8zwdVIfyHyOpdk_9AAAAAAAAAAA.png}

\subsection{\texorpdfstring{{ Parameters
}}{ Parameters }}\label{parameters}

\phantomsection\label{parameters-tooltip}
Parameters are the inputs to a function. They are specified in
parentheses after the function name.

{ underline } (

{ \hyperref[parameters-stroke]{stroke :}
\href{/docs/reference/foundations/auto/}{auto}
\href{/docs/reference/layout/length/}{length}
\href{/docs/reference/visualize/color/}{color}
\href{/docs/reference/visualize/gradient/}{gradient}
\href{/docs/reference/visualize/stroke/}{stroke}
\href{/docs/reference/visualize/pattern/}{pattern}
\href{/docs/reference/foundations/dictionary/}{dictionary} , } {
\hyperref[parameters-offset]{offset :}
\href{/docs/reference/foundations/auto/}{auto}
\href{/docs/reference/layout/length/}{length} , } {
\hyperref[parameters-extent]{extent :}
\href{/docs/reference/layout/length/}{length} , } {
\hyperref[parameters-evade]{evade :}
\href{/docs/reference/foundations/bool/}{bool} , } {
\hyperref[parameters-background]{background :}
\href{/docs/reference/foundations/bool/}{bool} , } {
\href{/docs/reference/foundations/content/}{content} , }

) -\textgreater{} \href{/docs/reference/foundations/content/}{content}

\subsubsection{\texorpdfstring{\texttt{\ stroke\ }}{ stroke }}\label{parameters-stroke}

\href{/docs/reference/foundations/auto/}{auto} {or}
\href{/docs/reference/layout/length/}{length} {or}
\href{/docs/reference/visualize/color/}{color} {or}
\href{/docs/reference/visualize/gradient/}{gradient} {or}
\href{/docs/reference/visualize/stroke/}{stroke} {or}
\href{/docs/reference/visualize/pattern/}{pattern} {or}
\href{/docs/reference/foundations/dictionary/}{dictionary}

{{ Settable }}

\phantomsection\label{parameters-stroke-settable-tooltip}
Settable parameters can be customized for all following uses of the
function with a \texttt{\ set\ } rule.

How to \href{/docs/reference/visualize/stroke/}{stroke} the line.

If set to \texttt{\ }{\texttt{\ auto\ }}\texttt{\ } , takes on the
text\textquotesingle s color and a thickness defined in the current
font.

Default: \texttt{\ }{\texttt{\ auto\ }}\texttt{\ }

\includesvg[width=0.16667in,height=0.16667in]{/assets/icons/16-arrow-right.svg}
View example

\begin{verbatim}
Take #underline(
  stroke: 1.5pt + red,
  offset: 2pt,
  [care],
)
\end{verbatim}

\includegraphics[width=5in,height=\textheight,keepaspectratio]{/assets/docs/tbLKc9iYaghdhC9NcJaJOQAAAAAAAAAA.png}

\subsubsection{\texorpdfstring{\texttt{\ offset\ }}{ offset }}\label{parameters-offset}

\href{/docs/reference/foundations/auto/}{auto} {or}
\href{/docs/reference/layout/length/}{length}

{{ Settable }}

\phantomsection\label{parameters-offset-settable-tooltip}
Settable parameters can be customized for all following uses of the
function with a \texttt{\ set\ } rule.

The position of the line relative to the baseline, read from the font
tables if \texttt{\ }{\texttt{\ auto\ }}\texttt{\ } .

Default: \texttt{\ }{\texttt{\ auto\ }}\texttt{\ }

\includesvg[width=0.16667in,height=0.16667in]{/assets/icons/16-arrow-right.svg}
View example

\begin{verbatim}
#underline(offset: 5pt)[
  The Tale Of A Faraway Line I
]
\end{verbatim}

\includegraphics[width=5in,height=\textheight,keepaspectratio]{/assets/docs/p2tUWXcYq-E_ZbDtwzCDrAAAAAAAAAAA.png}

\subsubsection{\texorpdfstring{\texttt{\ extent\ }}{ extent }}\label{parameters-extent}

\href{/docs/reference/layout/length/}{length}

{{ Settable }}

\phantomsection\label{parameters-extent-settable-tooltip}
Settable parameters can be customized for all following uses of the
function with a \texttt{\ set\ } rule.

The amount by which to extend the line beyond (or within if negative)
the content.

Default: \texttt{\ }{\texttt{\ 0pt\ }}\texttt{\ }

\includesvg[width=0.16667in,height=0.16667in]{/assets/icons/16-arrow-right.svg}
View example

\begin{verbatim}
#align(center,
  underline(extent: 2pt)[Chapter 1]
)
\end{verbatim}

\includegraphics[width=5in,height=\textheight,keepaspectratio]{/assets/docs/tbT2BOLPtcXW-alQPb8q6wAAAAAAAAAA.png}

\subsubsection{\texorpdfstring{\texttt{\ evade\ }}{ evade }}\label{parameters-evade}

\href{/docs/reference/foundations/bool/}{bool}

{{ Settable }}

\phantomsection\label{parameters-evade-settable-tooltip}
Settable parameters can be customized for all following uses of the
function with a \texttt{\ set\ } rule.

Whether the line skips sections in which it would collide with the
glyphs.

Default: \texttt{\ }{\texttt{\ true\ }}\texttt{\ }

\includesvg[width=0.16667in,height=0.16667in]{/assets/icons/16-arrow-right.svg}
View example

\begin{verbatim}
This #underline(evade: true)[is great].
This #underline(evade: false)[is less great].
\end{verbatim}

\includegraphics[width=5in,height=\textheight,keepaspectratio]{/assets/docs/PaJc2qUpoh1s97E6NZYz0QAAAAAAAAAA.png}

\subsubsection{\texorpdfstring{\texttt{\ background\ }}{ background }}\label{parameters-background}

\href{/docs/reference/foundations/bool/}{bool}

{{ Settable }}

\phantomsection\label{parameters-background-settable-tooltip}
Settable parameters can be customized for all following uses of the
function with a \texttt{\ set\ } rule.

Whether the line is placed behind the content it underlines.

Default: \texttt{\ }{\texttt{\ false\ }}\texttt{\ }

\includesvg[width=0.16667in,height=0.16667in]{/assets/icons/16-arrow-right.svg}
View example

\begin{verbatim}
#set underline(stroke: (thickness: 1em, paint: maroon, cap: "round"))
#underline(background: true)[This is stylized.] \
#underline(background: false)[This is partially hidden.]
\end{verbatim}

\includegraphics[width=5in,height=\textheight,keepaspectratio]{/assets/docs/W98M7AlnFoSVnlt9g5bIsAAAAAAAAAAA.png}

\subsubsection{\texorpdfstring{\texttt{\ body\ }}{ body }}\label{parameters-body}

\href{/docs/reference/foundations/content/}{content}

{Required} {{ Positional }}

\phantomsection\label{parameters-body-positional-tooltip}
Positional parameters are specified in order, without names.

The content to underline.

\href{/docs/reference/text/text/}{\pandocbounded{\includesvg[keepaspectratio]{/assets/icons/16-arrow-right.svg}}}

{ Text } { Previous page }

\href{/docs/reference/text/upper/}{\pandocbounded{\includesvg[keepaspectratio]{/assets/icons/16-arrow-right.svg}}}

{ Uppercase } { Next page }
