\title{typst.app/docs/reference/math/equation}

\begin{itemize}
\tightlist
\item
  \href{/docs}{\includesvg[width=0.16667in,height=0.16667in]{/assets/icons/16-docs-dark.svg}}
\item
  \includesvg[width=0.16667in,height=0.16667in]{/assets/icons/16-arrow-right.svg}
\item
  \href{/docs/reference/}{Reference}
\item
  \includesvg[width=0.16667in,height=0.16667in]{/assets/icons/16-arrow-right.svg}
\item
  \href{/docs/reference/math/}{Math}
\item
  \includesvg[width=0.16667in,height=0.16667in]{/assets/icons/16-arrow-right.svg}
\item
  \href{/docs/reference/math/equation/}{Equation}
\end{itemize}

\section{\texorpdfstring{\texttt{\ equation\ } {{ Element
}}}{ equation   Element }}\label{summary}

\phantomsection\label{element-tooltip}
Element functions can be customized with \texttt{\ set\ } and
\texttt{\ show\ } rules.

A mathematical equation.

Can be displayed inline with text or as a separate block.

\subsection{Example}\label{example}

\begin{verbatim}
#set text(font: "New Computer Modern")

Let $a$, $b$, and $c$ be the side
lengths of right-angled triangle.
Then, we know that:
$ a^2 + b^2 = c^2 $

Prove by induction:
$ sum_(k=1)^n k = (n(n+1)) / 2 $
\end{verbatim}

\includegraphics[width=5in,height=\textheight,keepaspectratio]{/assets/docs/JtxOgQArvspfmmStl8-3_gAAAAAAAAAA.png}

By default, block-level equations will not break across pages. This can
be changed through
\texttt{\ }{\texttt{\ show\ }}\texttt{\ math\ }{\texttt{\ .\ }}\texttt{\ }{\texttt{\ equation\ }}\texttt{\ }{\texttt{\ :\ }}\texttt{\ }{\texttt{\ set\ }}\texttt{\ }{\texttt{\ block\ }}\texttt{\ }{\texttt{\ (\ }}\texttt{\ breakable\ }{\texttt{\ :\ }}\texttt{\ }{\texttt{\ true\ }}\texttt{\ }{\texttt{\ )\ }}\texttt{\ }
.

\subsection{Syntax}\label{syntax}

This function also has dedicated syntax: Write mathematical markup
within dollar signs to create an equation. Starting and ending the
equation with at least one space lifts it into a separate block that is
centered horizontally. For more details about math syntax, see the
\href{/docs/reference/math/}{main math page} .

\subsection{\texorpdfstring{{ Parameters
}}{ Parameters }}\label{parameters}

\phantomsection\label{parameters-tooltip}
Parameters are the inputs to a function. They are specified in
parentheses after the function name.

math { . } { equation } (

{ \hyperref[parameters-block]{block :}
\href{/docs/reference/foundations/bool/}{bool} , } {
\hyperref[parameters-numbering]{numbering :}
\href{/docs/reference/foundations/none/}{none}
\href{/docs/reference/foundations/str/}{str}
\href{/docs/reference/foundations/function/}{function} , } {
\hyperref[parameters-number-align]{number-align :}
\href{/docs/reference/layout/alignment/}{alignment} , } {
\hyperref[parameters-supplement]{supplement :}
\href{/docs/reference/foundations/none/}{none}
\href{/docs/reference/foundations/auto/}{auto}
\href{/docs/reference/foundations/content/}{content}
\href{/docs/reference/foundations/function/}{function} , } {
\href{/docs/reference/foundations/content/}{content} , }

) -\textgreater{} \href{/docs/reference/foundations/content/}{content}

\subsubsection{\texorpdfstring{\texttt{\ block\ }}{ block }}\label{parameters-block}

\href{/docs/reference/foundations/bool/}{bool}

{{ Settable }}

\phantomsection\label{parameters-block-settable-tooltip}
Settable parameters can be customized for all following uses of the
function with a \texttt{\ set\ } rule.

Whether the equation is displayed as a separate block.

Default: \texttt{\ }{\texttt{\ false\ }}\texttt{\ }

\subsubsection{\texorpdfstring{\texttt{\ numbering\ }}{ numbering }}\label{parameters-numbering}

\href{/docs/reference/foundations/none/}{none} {or}
\href{/docs/reference/foundations/str/}{str} {or}
\href{/docs/reference/foundations/function/}{function}

{{ Settable }}

\phantomsection\label{parameters-numbering-settable-tooltip}
Settable parameters can be customized for all following uses of the
function with a \texttt{\ set\ } rule.

How to \href{/docs/reference/model/numbering/}{number} block-level
equations.

Default: \texttt{\ }{\texttt{\ none\ }}\texttt{\ }

\includesvg[width=0.16667in,height=0.16667in]{/assets/icons/16-arrow-right.svg}
View example

\begin{verbatim}
#set math.equation(numbering: "(1)")

We define:
$ phi.alt := (1 + sqrt(5)) / 2 $ <ratio>

With @ratio, we get:
$ F_n = floor(1 / sqrt(5) phi.alt^n) $
\end{verbatim}

\includegraphics[width=5in,height=\textheight,keepaspectratio]{/assets/docs/ICkRN4qFA2wn3VV_dGJcKAAAAAAAAAAA.png}

\subsubsection{\texorpdfstring{\texttt{\ number-align\ }}{ number-align }}\label{parameters-number-align}

\href{/docs/reference/layout/alignment/}{alignment}

{{ Settable }}

\phantomsection\label{parameters-number-align-settable-tooltip}
Settable parameters can be customized for all following uses of the
function with a \texttt{\ set\ } rule.

The alignment of the equation numbering.

By default, the alignment is
\texttt{\ end\ }{\texttt{\ +\ }}\texttt{\ horizon\ } . For the
horizontal component, you can use \texttt{\ right\ } , \texttt{\ left\ }
, or \texttt{\ start\ } and \texttt{\ end\ } of the text direction; for
the vertical component, you can use \texttt{\ top\ } ,
\texttt{\ horizon\ } , or \texttt{\ bottom\ } .

Default: \texttt{\ end\ }{\texttt{\ +\ }}\texttt{\ horizon\ }

\includesvg[width=0.16667in,height=0.16667in]{/assets/icons/16-arrow-right.svg}
View example

\begin{verbatim}
#set math.equation(numbering: "(1)", number-align: bottom)

We can calculate:
$ E &= sqrt(m_0^2 + p^2) \
    &approx 125 "GeV" $
\end{verbatim}

\includegraphics[width=5in,height=\textheight,keepaspectratio]{/assets/docs/EjQKswH-OBAc5Rwhl-7WNQAAAAAAAAAA.png}

\subsubsection{\texorpdfstring{\texttt{\ supplement\ }}{ supplement }}\label{parameters-supplement}

\href{/docs/reference/foundations/none/}{none} {or}
\href{/docs/reference/foundations/auto/}{auto} {or}
\href{/docs/reference/foundations/content/}{content} {or}
\href{/docs/reference/foundations/function/}{function}

{{ Settable }}

\phantomsection\label{parameters-supplement-settable-tooltip}
Settable parameters can be customized for all following uses of the
function with a \texttt{\ set\ } rule.

A supplement for the equation.

For references to equations, this is added before the referenced number.

If a function is specified, it is passed the referenced equation and
should return content.

Default: \texttt{\ }{\texttt{\ auto\ }}\texttt{\ }

\includesvg[width=0.16667in,height=0.16667in]{/assets/icons/16-arrow-right.svg}
View example

\begin{verbatim}
#set math.equation(numbering: "(1)", supplement: [Eq.])

We define:
$ phi.alt := (1 + sqrt(5)) / 2 $ <ratio>

With @ratio, we get:
$ F_n = floor(1 / sqrt(5) phi.alt^n) $
\end{verbatim}

\includegraphics[width=5in,height=\textheight,keepaspectratio]{/assets/docs/LsvSGn7Nchg2dddv3zDBtAAAAAAAAAAA.png}

\subsubsection{\texorpdfstring{\texttt{\ body\ }}{ body }}\label{parameters-body}

\href{/docs/reference/foundations/content/}{content}

{Required} {{ Positional }}

\phantomsection\label{parameters-body-positional-tooltip}
Positional parameters are specified in order, without names.

The contents of the equation.

\href{/docs/reference/math/class/}{\pandocbounded{\includesvg[keepaspectratio]{/assets/icons/16-arrow-right.svg}}}

{ Class } { Previous page }

\href{/docs/reference/math/frac/}{\pandocbounded{\includesvg[keepaspectratio]{/assets/icons/16-arrow-right.svg}}}

{ Fraction } { Next page }
