\title{typst.app/universe/package/mannot}

\phantomsection\label{banner}
\section{mannot}\label{mannot}

{ 0.1.0 }

A package for highlighting and annotating in math blocks.

\phantomsection\label{readme}
A package for highlighting and annotating in math blocks in
\href{https://typst.app/}{Typst} .

A full documentation is
\href{https://github.com/typst/packages/raw/main/packages/preview/mannot/0.1.0/docs/doc.pdf}{here}
.

\subsection{Example}\label{example}

\begin{Shaded}
\begin{Highlighting}[]
\NormalTok{$}
\NormalTok{mark(1, tag: \#\textless{}num\textgreater{}) / mark(x + 1, tag: \#\textless{}den\textgreater{}, color: \#blue)}
\NormalTok{+ mark(2, tag: \#\textless{}quo\textgreater{}, color: \#red)}

\NormalTok{\#annot(\textless{}num\textgreater{}, pos: top)[Numerator]}
\NormalTok{\#annot(\textless{}den\textgreater{})[Denominator]}
\NormalTok{\#annot(\textless{}quo\textgreater{}, pos: right, yshift: 1em)[Quotient]}
\NormalTok{$}
\end{Highlighting}
\end{Shaded}

\pandocbounded{\includesvg[keepaspectratio]{https://github.com/typst/packages/raw/main/packages/preview/mannot/0.1.0/examples/showcase.svg}}

\subsection{Usage}\label{usage}

Import and initialize the package \texttt{\ mannot\ } on the top of your
document.

\begin{Shaded}
\begin{Highlighting}[]
\NormalTok{\#import "@preview/mannot:0.1.0": *}
\NormalTok{\#show: mannot{-}init}
\end{Highlighting}
\end{Shaded}

To highlight a part of a math block, use the \texttt{\ mark\ } function:

\begin{Shaded}
\begin{Highlighting}[]
\NormalTok{$}
\NormalTok{mark(x)}
\NormalTok{$}
\end{Highlighting}
\end{Shaded}

\pandocbounded{\includesvg[keepaspectratio]{https://github.com/typst/packages/raw/main/packages/preview/mannot/0.1.0/examples/usage1.svg}}

You can also specify a color for the highlighted part:

\begin{Shaded}
\begin{Highlighting}[]
\NormalTok{$ // Need \# before color names.}
\NormalTok{mark(3, color: \#red) mark(x, color: \#blue)}
\NormalTok{+ mark(integral x dif x, color: \#green)}
\NormalTok{$}
\end{Highlighting}
\end{Shaded}

\pandocbounded{\includesvg[keepaspectratio]{https://github.com/typst/packages/raw/main/packages/preview/mannot/0.1.0/examples/usage2.svg}}

To add an annotation to a highlighted part, use the \texttt{\ annot\ }
function. You need to specify the tag of the marked content:

\begin{Shaded}
\begin{Highlighting}[]
\NormalTok{$}
\NormalTok{mark(x, tag: \#\textless{}x\textgreater{})  // Need \# before tags.}
\NormalTok{\#annot(\textless{}x\textgreater{})[Annotation]}
\NormalTok{$}
\end{Highlighting}
\end{Shaded}

\pandocbounded{\includesvg[keepaspectratio]{https://github.com/typst/packages/raw/main/packages/preview/mannot/0.1.0/examples/usage3.svg}}

You can customize the position of the annotation and the vertical
distance from the marked content:

\begin{Shaded}
\begin{Highlighting}[]
\NormalTok{$}
\NormalTok{mark(integral x dif x, tag: \#\textless{}i\textgreater{}, color: \#green)}
\NormalTok{+ mark(3, tag: \#\textless{}3\textgreater{}, color: \#red) mark(x, tag: \#\textless{}x\textgreater{}, color: \#blue)}

\NormalTok{\#annot(\textless{}i\textgreater{}, pos: left)[Set pos to left.]}
\NormalTok{\#annot(\textless{}i\textgreater{}, pos: top + left)[Top left.]}
\NormalTok{\#annot(\textless{}3\textgreater{}, pos: top, yshift: 1.2em)[Use yshift.]}
\NormalTok{\#annot(\textless{}x\textgreater{}, pos: right, yshift: 1.2em)[Auto arrow.]}
\NormalTok{$}
\end{Highlighting}
\end{Shaded}

\pandocbounded{\includesvg[keepaspectratio]{https://github.com/typst/packages/raw/main/packages/preview/mannot/0.1.0/examples/usage4.svg}}

For convenience, you can define custom mark functions:

\begin{Shaded}
\begin{Highlighting}[]
\NormalTok{\#let rmark = mark.with(color: red)}
\NormalTok{\#let gmark = mark.with(color: green)}
\NormalTok{\#let bmark = mark.with(color: blue)}

\NormalTok{$}
\NormalTok{integral\_rmark(0, tag: \#\textless{}i0\textgreater{})\^{}bmark(1, tag: \#\textless{}i1\textgreater{})}
\NormalTok{mark(x\^{}2 + 1, tag: \#\textless{}i2\textgreater{}) dif gmark(x, tag: \#\textless{}i3\textgreater{})}

\NormalTok{\#annot(\textless{}i0\textgreater{})[Begin]}
\NormalTok{\#annot(\textless{}i1\textgreater{}, pos: top)[End]}
\NormalTok{\#annot(\textless{}i2\textgreater{}, pos: top + right)[Integrand]}
\NormalTok{\#annot(\textless{}i3\textgreater{}, pos: right, yshift: .6em)[Variable]}
\NormalTok{$}
\end{Highlighting}
\end{Shaded}

\pandocbounded{\includesvg[keepaspectratio]{https://github.com/typst/packages/raw/main/packages/preview/mannot/0.1.0/examples/usage5.svg}}

\subsubsection{How to add}\label{how-to-add}

Copy this into your project and use the import as \texttt{\ mannot\ }

\begin{verbatim}
#import "@preview/mannot:0.1.0"
\end{verbatim}

\includesvg[width=0.16667in,height=0.16667in]{/assets/icons/16-copy.svg}

Check the docs for
\href{https://typst.app/docs/reference/scripting/\#packages}{more
information on how to import packages} .

\subsubsection{About}\label{about}

\begin{description}
\tightlist
\item[Author :]
ryuryu-ymj
\item[License:]
MIT
\item[Current version:]
0.1.0
\item[Last updated:]
October 21, 2024
\item[First released:]
October 21, 2024
\item[Minimum Typst version:]
0.12.0
\item[Archive size:]
6.84 kB
\href{https://packages.typst.org/preview/mannot-0.1.0.tar.gz}{\pandocbounded{\includesvg[keepaspectratio]{/assets/icons/16-download.svg}}}
\item[Repository:]
\href{https://github.com/ryuryu-ymj/mannot}{GitHub}
\item[Categor ies :]
\begin{itemize}
\tightlist
\item[]
\item
  \pandocbounded{\includesvg[keepaspectratio]{/assets/icons/16-chart.svg}}
  \href{https://typst.app/universe/search/?category=visualization}{Visualization}
\item
  \pandocbounded{\includesvg[keepaspectratio]{/assets/icons/16-layout.svg}}
  \href{https://typst.app/universe/search/?category=layout}{Layout}
\end{itemize}
\end{description}

\subsubsection{Where to report issues?}\label{where-to-report-issues}

This package is a project of ryuryu-ymj . Report issues on
\href{https://github.com/ryuryu-ymj/mannot}{their repository} . You can
also try to ask for help with this package on the
\href{https://forum.typst.app}{Forum} .

Please report this package to the Typst team using the
\href{https://typst.app/contact}{contact form} if you believe it is a
safety hazard or infringes upon your rights.

\phantomsection\label{versions}
\subsubsection{Version history}\label{version-history}

\begin{longtable}[]{@{}ll@{}}
\toprule\noalign{}
Version & Release Date \\
\midrule\noalign{}
\endhead
\bottomrule\noalign{}
\endlastfoot
0.1.0 & October 21, 2024 \\
\end{longtable}

Typst GmbH did not create this package and cannot guarantee correct
functionality of this package or compatibility with any version of the
Typst compiler or app.
