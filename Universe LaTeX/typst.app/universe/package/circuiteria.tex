\title{typst.app/universe/package/circuiteria}

\phantomsection\label{banner}
\section{circuiteria}\label{circuiteria}

{ 0.1.0 }

Drawing block circuits with Typst made easy, using CeTZ

\phantomsection\label{readme}
Circuiteria is a \href{https://typst.app/}{Typst} package for drawing
block circuit diagrams using the
\href{https://typst.app/universe/package/cetz}{CeTZ} package.

\pandocbounded{\includegraphics[keepaspectratio]{https://github.com/typst/packages/raw/main/packages/preview/circuiteria/0.1.0/gallery/platypus.png}}

\subsection{Examples}\label{examples}

\begin{longtable}[]{@{}ll@{}}
\toprule\noalign{}
\endhead
\bottomrule\noalign{}
\endlastfoot
\multicolumn{2}{@{}l@{}}{%
\href{https://github.com/typst/packages/raw/main/packages/preview/circuiteria/0.1.0/gallery/test.typ}{\includegraphics[width=5.20833in,height=\textheight,keepaspectratio]{https://github.com/typst/packages/raw/main/packages/preview/circuiteria/0.1.0/gallery/test.png}}} \\
\multicolumn{2}{@{}l@{}}{%
A bit of eveything} \\
\multicolumn{2}{@{}l@{}}{%
\href{https://github.com/typst/packages/raw/main/packages/preview/circuiteria/0.1.0/gallery/test5.typ}{\includegraphics[width=5.20833in,height=\textheight,keepaspectratio]{https://github.com/typst/packages/raw/main/packages/preview/circuiteria/0.1.0/gallery/test5.png}}} \\
\multicolumn{2}{@{}l@{}}{%
Wires everywhere} \\
\href{https://github.com/typst/packages/raw/main/packages/preview/circuiteria/0.1.0/gallery/test4.typ}{\includegraphics[width=2.60417in,height=\textheight,keepaspectratio]{https://github.com/typst/packages/raw/main/packages/preview/circuiteria/0.1.0/gallery/test4.png}}
&
\href{https://github.com/typst/packages/raw/main/packages/preview/circuiteria/0.1.0/gallery/test6.typ}{\includegraphics[width=2.60417in,height=\textheight,keepaspectratio]{https://github.com/typst/packages/raw/main/packages/preview/circuiteria/0.1.0/gallery/test6.png}} \\
Groups & Rotated \\
\end{longtable}

\begin{quote}
\textbf{Note}\\
These circuit layouts were copied from a digital design course given by
prof. S. Zahno and recreated using this package
\end{quote}

\emph{Click on the example image to jump to the code.}

\subsection{Usage}\label{usage}

For more information, see the
\href{https://github.com/typst/packages/raw/main/packages/preview/circuiteria/0.1.0/manual.pdf}{manual}

To use this package, simply import
\href{https://typst.app/universe/package/circuiteria}{circuiteria} and
call the \texttt{\ circuit\ } function:

\begin{Shaded}
\begin{Highlighting}[]
\NormalTok{\#import "@preview/circuiteria:0.1.0"}
\NormalTok{\#circuiteria.circuit(\{}
\NormalTok{  import circuiteria: *}
\NormalTok{  ...}
\NormalTok{\})}
\end{Highlighting}
\end{Shaded}

\subsubsection{How to add}\label{how-to-add}

Copy this into your project and use the import as
\texttt{\ circuiteria\ }

\begin{verbatim}
#import "@preview/circuiteria:0.1.0"
\end{verbatim}

\includesvg[width=0.16667in,height=0.16667in]{/assets/icons/16-copy.svg}

Check the docs for
\href{https://typst.app/docs/reference/scripting/\#packages}{more
information on how to import packages} .

\subsubsection{About}\label{about}

\begin{description}
\tightlist
\item[Author :]
\href{https://git.kb28.ch/HEL}{Louis Heredero}
\item[License:]
Apache-2.0
\item[Current version:]
0.1.0
\item[Last updated:]
October 3, 2024
\item[First released:]
October 3, 2024
\item[Minimum Typst version:]
0.11.0
\item[Archive size:]
193 kB
\href{https://packages.typst.org/preview/circuiteria-0.1.0.tar.gz}{\pandocbounded{\includesvg[keepaspectratio]{/assets/icons/16-download.svg}}}
\item[Repository:]
\href{https://git.kb28.ch/HEL/circuiteria}{git.kb28.ch}
\item[Categor y :]
\begin{itemize}
\tightlist
\item[]
\item
  \pandocbounded{\includesvg[keepaspectratio]{/assets/icons/16-chart.svg}}
  \href{https://typst.app/universe/search/?category=visualization}{Visualization}
\end{itemize}
\end{description}

\subsubsection{Where to report issues?}\label{where-to-report-issues}

This package is a project of Louis Heredero . Report issues on
\href{https://git.kb28.ch/HEL/circuiteria}{their repository} . You can
also try to ask for help with this package on the
\href{https://forum.typst.app}{Forum} .

Please report this package to the Typst team using the
\href{https://typst.app/contact}{contact form} if you believe it is a
safety hazard or infringes upon your rights.

\phantomsection\label{versions}
\subsubsection{Version history}\label{version-history}

\begin{longtable}[]{@{}ll@{}}
\toprule\noalign{}
Version & Release Date \\
\midrule\noalign{}
\endhead
\bottomrule\noalign{}
\endlastfoot
0.1.0 & October 3, 2024 \\
\end{longtable}

Typst GmbH did not create this package and cannot guarantee correct
functionality of this package or compatibility with any version of the
Typst compiler or app.
