\title{typst.app/universe/package/iridis}

\phantomsection\label{banner}
\section{iridis}\label{iridis}

{ 0.1.0 }

A package to colors matching parenthesis

\phantomsection\label{readme}
Iridis is a package to colorize parenthesis in your typst document.
Iridis is a latin word that means “rainbow�. This package is
inspired by the many rainbow parenthesis plugins available for code
editors.

\subsection{Usage}\label{usage}

The package provides a single show-rule \texttt{\ iridis-show\ } that
can be used to colorize parenthesis in your document and a palette
\texttt{\ iridis-palette\ } that can be used to define the colors used.

The rule takes 3 arguments:

\begin{itemize}
\tightlist
\item
  \texttt{\ opening-parenthesis\ } : The opening parenthesis character.
  Default is \texttt{\ ("(",\ "{[}",\ "\{")\ } .
\item
  \texttt{\ closing-parenthesis\ } : The closing parenthesis character.
  Default is \texttt{\ (")",\ "{]}",\ "\}")\ } .
\item
  \texttt{\ palette\ } : The color palette to use. Default is
  \texttt{\ iridis-palette\ } .
\end{itemize}

If the symbols are single characters, they are interpreted as normal
strings but if you use multi-character strings, then they are
interpreted as regular expressions.

\subsection{Exemples}\label{exemples}

\begin{Shaded}
\begin{Highlighting}[]
\NormalTok{\#show: iridis.iridis{-}show}

\NormalTok{\textasciigrave{}\textasciigrave{}\textasciigrave{}rs}
\NormalTok{fn main() \{}
\NormalTok{    let n = false;}
\NormalTok{    if n \{}
\NormalTok{        println!("Hello, world!");}
\NormalTok{    \} else \{}
\NormalTok{        println!("Goodbye, world!");}
\NormalTok{    \}}
\NormalTok{\}}
\NormalTok{\textasciigrave{}\textasciigrave{}\textasciigrave{}}

\NormalTok{\textasciigrave{}\textasciigrave{}\textasciigrave{}cpp}
\NormalTok{\#include \textless{}iostream\textgreater{}}

\NormalTok{int main() \{}
\NormalTok{    bool n = false;}
\NormalTok{    if (n) \{}
\NormalTok{        std::cout \textless{}\textless{} "Hello, world!" \textless{}\textless{} std::endl;}
\NormalTok{    \} else \{}
\NormalTok{        std::cout \textless{}\textless{} "Goodbye, world!" \textless{}\textless{} std::endl;}
\NormalTok{    \}}
\NormalTok{\}}
\NormalTok{\textasciigrave{}\textasciigrave{}\textasciigrave{}}

\NormalTok{\textasciigrave{}\textasciigrave{}\textasciigrave{}py}
\NormalTok{if \_\_name\_\_ == "\_\_main\_\_":}
\NormalTok{    n = False}
\NormalTok{    if n:}
\NormalTok{        print("Hello, world!")}
\NormalTok{    else:}
\NormalTok{        print("Goodbye, world!")}
\NormalTok{\textasciigrave{}\textasciigrave{}\textasciigrave{}}
\end{Highlighting}
\end{Shaded}

\pandocbounded{\includegraphics[keepaspectratio]{https://raw.githubusercontent.com/Robotechnic/iridis/master/images/code1.png}}

\begin{Shaded}
\begin{Highlighting}[]
\NormalTok{\#show: iridis.iridis{-}show}

\NormalTok{$}
\NormalTok{    "plus" equiv lambda m. f lambda n. lambda f. lambda x. m f (n f x) \textbackslash{}}
\NormalTok{    "succ" equiv lambda n. lambda f. lambda x. f (n f x) \textbackslash{}}
\NormalTok{    "mult" equiv lambda m. lambda n. lambda f. lambda x. m (n f) x \textbackslash{}}
\NormalTok{    "pred" equiv lambda n. lambda f. lambda x. n (lambda g. lambda h. h (g f)) (lambda u. x) (lambda u. u) \textbackslash{}}
\NormalTok{    "zero" equiv lambda f. lambda x. x \textbackslash{}}
\NormalTok{    "one" equiv lambda f. lambda x. f x \textbackslash{}}
\NormalTok{    "two" equiv lambda f. lambda x. f (f x) \textbackslash{}}
\NormalTok{    "three" equiv lambda f. lambda x. f (f (f x)) \textbackslash{}}
\NormalTok{    "four" equiv lambda f. lambda x. f (f (f (f x))) \textbackslash{}}
\NormalTok{$}

\NormalTok{$}
\NormalTok{    (((1 + 5) * 7) / (5 {-} 8 * 7) + 3) * 2 approx 4.352941176}
\NormalTok{$}

\NormalTok{$ mat(}
\NormalTok{  1, 2, ..., (10 / 2);}
\NormalTok{  2, 2, ..., 10;}
\NormalTok{  dots.v, dots.v, dots.down, dots.v;}
\NormalTok{  10, (10 {-} (5 * 8)) / 2, ..., 10;}
\NormalTok{) $}
\end{Highlighting}
\end{Shaded}

\pandocbounded{\includegraphics[keepaspectratio]{https://raw.githubusercontent.com/Robotechnic/iridis/master/images/math1.png}}

\subsection{Changelog}\label{changelog}

\subsubsection{0.1.0}\label{section}

\begin{itemize}
\tightlist
\item
  Initial release
\end{itemize}

\subsubsection{How to add}\label{how-to-add}

Copy this into your project and use the import as \texttt{\ iridis\ }

\begin{verbatim}
#import "@preview/iridis:0.1.0"
\end{verbatim}

\includesvg[width=0.16667in,height=0.16667in]{/assets/icons/16-copy.svg}

Check the docs for
\href{https://typst.app/docs/reference/scripting/\#packages}{more
information on how to import packages} .

\subsubsection{About}\label{about}

\begin{description}
\tightlist
\item[Author :]
\href{https://github.com/Robotechnic}{Robotechnic}
\item[License:]
MIT
\item[Current version:]
0.1.0
\item[Last updated:]
June 24, 2024
\item[First released:]
June 24, 2024
\item[Minimum Typst version:]
0.11.0
\item[Archive size:]
3.17 kB
\href{https://packages.typst.org/preview/iridis-0.1.0.tar.gz}{\pandocbounded{\includesvg[keepaspectratio]{/assets/icons/16-download.svg}}}
\end{description}

\subsubsection{Where to report issues?}\label{where-to-report-issues}

This package is a project of Robotechnic . You can also try to ask for
help with this package on the \href{https://forum.typst.app}{Forum} .

Please report this package to the Typst team using the
\href{https://typst.app/contact}{contact form} if you believe it is a
safety hazard or infringes upon your rights.

\phantomsection\label{versions}
\subsubsection{Version history}\label{version-history}

\begin{longtable}[]{@{}ll@{}}
\toprule\noalign{}
Version & Release Date \\
\midrule\noalign{}
\endhead
\bottomrule\noalign{}
\endlastfoot
0.1.0 & June 24, 2024 \\
\end{longtable}

Typst GmbH did not create this package and cannot guarantee correct
functionality of this package or compatibility with any version of the
Typst compiler or app.
