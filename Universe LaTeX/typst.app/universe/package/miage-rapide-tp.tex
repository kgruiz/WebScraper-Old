\title{typst.app/universe/package/miage-rapide-tp}

\phantomsection\label{banner}
\phantomsection\label{template-thumbnail}
\pandocbounded{\includegraphics[keepaspectratio]{https://packages.typst.org/preview/thumbnails/miage-rapide-tp-0.1.2-small.webp}}

\section{miage-rapide-tp}\label{miage-rapide-tp}

{ 0.1.2 }

Quickly generate a report for MIAGE practical work.

\href{/app?template=miage-rapide-tp&version=0.1.2}{Create project in
app}

\phantomsection\label{readme}
Typst template to generate a practical work report for students of the
MIAGE (Méthodes Informatiques Appliquées Ã~ la Gestion des
Entreprises).

\subsection{ðŸ§`â€?ðŸ'» Usage}\label{uxf0uxffuxe2uxf0uxff-usage}

\begin{itemize}
\item
  Directly from \href{https://typst.app/}{Typst web app} by clicking
  “Start from template� on the dashboard and searching for
  \texttt{\ miage-rapide-tp\ } .
\item
  With CLI:
\end{itemize}

\begin{verbatim}
typst init @preview/miage-rapide-tp:{version}
\end{verbatim}

\subsection{🚀 Features}\label{uxf0uxffux161-features}

\begin{itemize}
\tightlist
\item
  Cover page
\item
  Table of contents (optionnal)
\item
  \texttt{\ question\ } = automatically generates a question number
  (optionnal) with the content of the question
\item
  \texttt{\ code-block\ } = code block with syntax highlighting. You can
  pass a filepath or code directly to display in the block
\item
  \texttt{\ remarque\ } = a remark block with content and color
\end{itemize}

\subsubsection{Cover page}\label{cover-page}

The conf looks like this:

\begin{Shaded}
\begin{Highlighting}[]
\NormalTok{\#let conf(}
\NormalTok{  subtitle: none,}
\NormalTok{  authors: (),}
\NormalTok{  toc: true,}
\NormalTok{  lang: "fr",}
\NormalTok{  font: "Satoshi",}
\NormalTok{  date: none,}
\NormalTok{  years: (2024, 2025),}
\NormalTok{  years{-}label: "Année universitaire",}
\NormalTok{  title,}
\NormalTok{  doc,}
\NormalTok{)}
\end{Highlighting}
\end{Shaded}

\subsubsection{Question}\label{question}

A question can be added like this:

\begin{Shaded}
\begin{Highlighting}[]
\NormalTok{\#question("Une question avec numéro ?")}
\NormalTok{\#question("Une question sans numéro ?", counter: false)}
\end{Highlighting}
\end{Shaded}

The first argument is the question content, and the second (OPTIONAL) is
the counter. If \texttt{\ counter\ } is set to \texttt{\ false\ } , the
question will not be numbered.

\subsubsection{Code-block}\label{code-block}

To use a \texttt{\ code-block\ } , you can do as follows :

\begin{Shaded}
\begin{Highlighting}[]
\NormalTok{\#code{-}block(read("code/main.py"), "py")}
\NormalTok{\#code{-}block(read("code/example.sql"), "sql", title: "Classic SQL")}
\end{Highlighting}
\end{Shaded}

The first argument is the code to display, the second is the language of
the code, and the third is the title of the code block.

\subsubsection{Remarque}\label{remarque}

To use a \texttt{\ remarque\ } , you can do as follows :

\begin{Shaded}
\begin{Highlighting}[]
\NormalTok{\#remarque("Ceci est une remarque")}
\NormalTok{\#remarque("Remarque personnalisée", bg{-}color: olive, text{-}color: white)}
\end{Highlighting}
\end{Shaded}

You can change the bg-color and text-color of the remark block to match
your needs.

\subsection{ðŸ``? License}\label{uxf0uxff-license}

This is MIT licensed.

\begin{quote}
Rapide means fast in French. tp is the abbreviation of “travaux
pratiques� which means practical work. MIAGE is a French degree in
computer science applied to management.
\end{quote}

\href{/app?template=miage-rapide-tp&version=0.1.2}{Create project in
app}

\subsubsection{How to use}\label{how-to-use}

Click the button above to create a new project using this template in
the Typst app.

You can also use the Typst CLI to start a new project on your computer
using this command:

\begin{verbatim}
typst init @preview/miage-rapide-tp:0.1.2
\end{verbatim}

\includesvg[width=0.16667in,height=0.16667in]{/assets/icons/16-copy.svg}

\subsubsection{About}\label{about}

\begin{description}
\tightlist
\item[Author :]
Rémi Saurel
\item[License:]
MIT-0
\item[Current version:]
0.1.2
\item[Last updated:]
September 25, 2024
\item[First released:]
September 11, 2024
\item[Archive size:]
299 kB
\href{https://packages.typst.org/preview/miage-rapide-tp-0.1.2.tar.gz}{\pandocbounded{\includesvg[keepaspectratio]{/assets/icons/16-download.svg}}}
\item[Discipline s :]
\begin{itemize}
\tightlist
\item[]
\item
  \href{https://typst.app/universe/search/?discipline=education}{Education}
\item
  \href{https://typst.app/universe/search/?discipline=engineering}{Engineering}
\item
  \href{https://typst.app/universe/search/?discipline=computer-science}{Computer
  Science}
\end{itemize}
\item[Categor y :]
\begin{itemize}
\tightlist
\item[]
\item
  \pandocbounded{\includesvg[keepaspectratio]{/assets/icons/16-speak.svg}}
  \href{https://typst.app/universe/search/?category=report}{Report}
\end{itemize}
\end{description}

\subsubsection{Where to report issues?}\label{where-to-report-issues}

This template is a project of Rémi Saurel . You can also try to ask for
help with this template on the \href{https://forum.typst.app}{Forum} .

Please report this template to the Typst team using the
\href{https://typst.app/contact}{contact form} if you believe it is a
safety hazard or infringes upon your rights.

\phantomsection\label{versions}
\subsubsection{Version history}\label{version-history}

\begin{longtable}[]{@{}ll@{}}
\toprule\noalign{}
Version & Release Date \\
\midrule\noalign{}
\endhead
\bottomrule\noalign{}
\endlastfoot
0.1.2 & September 25, 2024 \\
\href{https://typst.app/universe/package/miage-rapide-tp/0.1.1/}{0.1.1}
& September 14, 2024 \\
\href{https://typst.app/universe/package/miage-rapide-tp/0.1.0/}{0.1.0}
& September 11, 2024 \\
\end{longtable}

Typst GmbH did not create this template and cannot guarantee correct
functionality of this template or compatibility with any version of the
Typst compiler or app.
