\title{typst.app/universe/package/in-dexter}

\phantomsection\label{banner}
\section{in-dexter}\label{in-dexter}

{ 0.5.3 }

Hand Picked Index for Typst.

\phantomsection\label{readme}
Automatically create a handcrafted index in
\href{https://typst.app/}{typst} . This typst component allows the
automatic creation of an Index page with entries that have been manually
marked in the document by its authors. This, in times of advanced search
functionality, seems somewhat outdated, but a handcrafted index like
this allows the authors to point the reader to just the right location
in the document.

âš~ï¸? Typst is in beta and evolving, and this package evolves with it.
As a result, no backward compatibility is guaranteed yet. Also, the
package itself is under development and fine-tuning.

\subsection{Table of Contents}\label{table-of-contents}

\begin{itemize}
\tightlist
\item
  \href{https://github.com/typst/packages/raw/main/packages/preview/in-dexter/0.5.3/\#usage}{Usage}

  \begin{itemize}
  \tightlist
  \item
    \href{https://github.com/typst/packages/raw/main/packages/preview/in-dexter/0.5.3/\#importing-the-component}{Importing
    the Component}
  \item
    \href{https://github.com/typst/packages/raw/main/packages/preview/in-dexter/0.5.3/\#remarks-for-new-version}{Remarks
    for new version}
  \item
    \href{https://github.com/typst/packages/raw/main/packages/preview/in-dexter/0.5.3/\#marking-entries}{Marking
    Entries}

    \begin{itemize}
    \tightlist
    \item
      \href{https://github.com/typst/packages/raw/main/packages/preview/in-dexter/0.5.3/\#generating-the-index-page}{Generating
      the index page}
    \item
      \href{https://github.com/typst/packages/raw/main/packages/preview/in-dexter/0.5.3/\#brief-sample-document}{Brief
      Sample Document}
    \item
      \href{https://github.com/typst/packages/raw/main/packages/preview/in-dexter/0.5.3/\#full-sample-document}{Full
      Sample Document}
    \end{itemize}
  \end{itemize}
\item
  \href{https://github.com/typst/packages/raw/main/packages/preview/in-dexter/0.5.3/\#changelog}{Changelog}

  \begin{itemize}
  \tightlist
  \item
    \href{https://github.com/typst/packages/raw/main/packages/preview/in-dexter/0.5.3/\#v053}{v0.5.3}
  \item
    \href{https://github.com/typst/packages/raw/main/packages/preview/in-dexter/0.5.3/\#v052}{v0.5.2}
  \item
    \href{https://github.com/typst/packages/raw/main/packages/preview/in-dexter/0.5.3/\#v051}{v0.5.1}
  \item
    \href{https://github.com/typst/packages/raw/main/packages/preview/in-dexter/0.5.3/\#v050}{v0.5.0}
  \item
    \href{https://github.com/typst/packages/raw/main/packages/preview/in-dexter/0.5.3/\#v043}{v0.4.3}
  \item
    \href{https://github.com/typst/packages/raw/main/packages/preview/in-dexter/0.5.3/\#v042}{v0.4.2}
  \item
    \href{https://github.com/typst/packages/raw/main/packages/preview/in-dexter/0.5.3/\#v041}{v0.4.1}
  \item
    \href{https://github.com/typst/packages/raw/main/packages/preview/in-dexter/0.5.3/\#v040}{v0.4.0}
  \item
    \href{https://github.com/typst/packages/raw/main/packages/preview/in-dexter/0.5.3/\#v032}{v0.3.2}
  \item
    \href{https://github.com/typst/packages/raw/main/packages/preview/in-dexter/0.5.3/\#v031}{v0.3.1}
  \item
    \href{https://github.com/typst/packages/raw/main/packages/preview/in-dexter/0.5.3/\#v030}{v0.3.0}
  \item
    \href{https://github.com/typst/packages/raw/main/packages/preview/in-dexter/0.5.3/\#v020}{v0.2.0}
  \item
    \href{https://github.com/typst/packages/raw/main/packages/preview/in-dexter/0.5.3/\#v010}{v0.1.0}
  \item
    \href{https://github.com/typst/packages/raw/main/packages/preview/in-dexter/0.5.3/\#v006}{v0.0.6}
  \item
    \href{https://github.com/typst/packages/raw/main/packages/preview/in-dexter/0.5.3/\#v005}{v0.0.5}
  \item
    \href{https://github.com/typst/packages/raw/main/packages/preview/in-dexter/0.5.3/\#v004}{v0.0.4}
  \item
    \href{https://github.com/typst/packages/raw/main/packages/preview/in-dexter/0.5.3/\#v003}{v0.0.3}
  \item
    \href{https://github.com/typst/packages/raw/main/packages/preview/in-dexter/0.5.3/\#v002}{v0.0.2}
  \end{itemize}
\end{itemize}

\subsection{Usage}\label{usage}

\subsection{Importing the Component}\label{importing-the-component}

To use the index functionality, the component must be available. This
can be achieved by importing the package \texttt{\ in-dexter\ } into the
project:

Add the following code to the head of the document file(s) that want to
use the index:

\begin{Shaded}
\begin{Highlighting}[]
\NormalTok{  \#import "@preview/in{-}dexter:0.5.3": *}
\end{Highlighting}
\end{Shaded}

Alternatively it can be loaded from the file, if you have it copied into
your project.

\begin{Shaded}
\begin{Highlighting}[]
\NormalTok{  \#import "in{-}dexter.typ": *}
\end{Highlighting}
\end{Shaded}

\subsection{Remarks for new version}\label{remarks-for-new-version}

In previous versions (before 0.0.6) of in-dexter, it was required to
hide the index entries with a show rule. This is not required anymore.

\subsection{Marking Entries}\label{marking-entries}

To mark a word to be included in the index, a simple function can be
used. In the following sample code, the word “elit� is marked to be
included into the index.

\begin{Shaded}
\begin{Highlighting}[]
\NormalTok{= Sample Text}
\NormalTok{Lorem ipsum dolor sit amet, consectetur adipiscing \#index[elit], sed do eiusmod tempor}
\NormalTok{incididunt ut labore et dolore.}
\end{Highlighting}
\end{Shaded}

Nested entries can be created - the following would create an entry
\texttt{\ adipiscing\ } with sub entry \texttt{\ elit\ } .

\begin{Shaded}
\begin{Highlighting}[]
\NormalTok{= Sample Text}
\NormalTok{Lorem ipsum dolor sit amet, consectetur adipiscing elit\#index("adipiscing", "elit"), sed do eiusmod}
\NormalTok{tempor incididunt ut labore et dolore.}
\end{Highlighting}
\end{Shaded}

The marking, by default, is invisible in the resulting text, while the
marked word will still be visible. With the marking in place, the index
component knows about the word, as well as its location in the document.

\subsection{Generating the Index Page}\label{generating-the-index-page}

The index page can be generated by the following function:

\begin{Shaded}
\begin{Highlighting}[]
\NormalTok{= Index}
\NormalTok{\#columns(3)[}
\NormalTok{  \#make{-}index(title: none)}
\NormalTok{]}
\end{Highlighting}
\end{Shaded}

This sample uses the optional title, outline, and use-page-counter
parameters:

\begin{Shaded}
\begin{Highlighting}[]
\NormalTok{\#make{-}index(title: [Index], outlined: true, use{-}page{-}counter: true)}
\end{Highlighting}
\end{Shaded}

The \texttt{\ make-index()\ } function takes three optional arguments:
\texttt{\ title\ } , \texttt{\ outlined\ } , and
\texttt{\ use-page-counter\ } .

\begin{itemize}
\tightlist
\item
  \texttt{\ title\ } adds a title (with \texttt{\ heading\ } ) and
\item
  \texttt{\ outlined\ } is \texttt{\ false\ } by default and is passed
  to the heading function
\item
  \texttt{\ use-page-counter\ } is \texttt{\ false\ } by default. If set
  to \texttt{\ true\ } it will use \texttt{\ counter(page).display()\ }
  for the page number text in the index instead of the absolute page
  position (the absolute position is still used for the actual link
  target)
\end{itemize}

If no title is given the heading should never appear in the layout.
Note: The heading is (currently) not numbered.

The first sample emits the index in three columns. Note: The actual
appearance depends on your template or other settings of your document.

You can find a preview image of the resulting page on
\href{https://github.com/RolfBremer/in-dexter}{in-dexter´s GitHub
repository} .

You may have noticed that some page numbers are displayed as bold. These
are index entries which are marked as “main� entries. Such entries
are meant to be the most important for the given entry. They can be
marked as follows:

\begin{Shaded}
\begin{Highlighting}[]
\NormalTok{\#index(fmt: strong, [Willkommen])}
\end{Highlighting}
\end{Shaded}

or you can use the predefined semantically helper function

\begin{Shaded}
\begin{Highlighting}[]
\NormalTok{\#index{-}main[Willkommen]}
\end{Highlighting}
\end{Shaded}

\subsubsection{Brief Sample Document}\label{brief-sample-document}

This is a very brief sample to demonstrate how in-dexter can be used.
The next chapter contains a more fleshed out sample.

\begin{Shaded}
\begin{Highlighting}[]
\NormalTok{\#import "@preview/in{-}dexter:0.5.3": *}


\NormalTok{= My Sample Document with \textasciigrave{}in{-}dexter\textasciigrave{}}

\NormalTok{In this document the usage of the \textasciigrave{}in{-}dexter\textasciigrave{} package is demonstrated to create}
\NormalTok{a hand picked \#index[Hand Picked] index. This sample \#index{-}main[Sample]}
\NormalTok{document \#index[Document] is quite short, and so is its index.}


\NormalTok{= Index}

\NormalTok{This section contains the generated Index.}

\NormalTok{\#make{-}index()}
\end{Highlighting}
\end{Shaded}

\subsubsection{Full Sample Document}\label{full-sample-document}

\begin{Shaded}
\begin{Highlighting}[]
\NormalTok{\#import "@preview/in{-}dexter:0.5.3": *}

\NormalTok{\#let index{-}main(..args) = index(fmt: strong, ..args)}

\NormalTok{// Document settings}
\NormalTok{\#set page("a5")}
\NormalTok{\#set text(font: ("Arial", "Trebuchet MS"), size: 12pt)}


\NormalTok{= My Sample Document with \textasciigrave{}in{-}dexter\textasciigrave{}}

\NormalTok{In this document \#index[Document] the usage of the \textasciigrave{}in{-}dexter\textasciigrave{} package \#index[Package]}
\NormalTok{is demonstrated to create a hand picked index \#index{-}main[Index]. This sample document}
\NormalTok{is quite short, and so is its index. So to fill this sample with some real text,}
\NormalTok{let´s elaborate on some aspects of a hand picked \#index[Hand Picked] index. So, "hand}
\NormalTok{picked" means, the entries \#index[Entries] in the index are carefully chosen by the}
\NormalTok{author(s) of the document to point the reader, who is interested in a specific topic}
\NormalTok{within the documents domain \#index[Domain], to the right spot \#index[Spot]. Thats, how}
\NormalTok{it should be; and it is quite different to what is done in this sample text, where the}
\NormalTok{objective \#index{-}main[Objective] was to put many different index markers}
\NormalTok{\#index[Markers] into a small text, because a sample should be as brief as possible,}
\NormalTok{while providing enough substance \#index[Substance] to demo the subject}
\NormalTok{\#index[Subject]. The resulting index in this demo is somewhat pointless}
\NormalTok{\#index[Pointless], because all entries are pointing to few different pages}
\NormalTok{\#index[Pages], due to the fact that the demo text only has few pages \#index[Page].}
\NormalTok{That is also the reason for what we chose the DIN A5 \#index[DIN A5] format, and we}
\NormalTok{also continue with some remarks \#index[Remarks] on the next page.}


\NormalTok{== Some more demo content without deeper meaning}

\NormalTok{\#lorem(50) \#index[Lorem]}

\NormalTok{\#pagebreak()}

\NormalTok{== Remarks}

\NormalTok{Here are some more remarks \#index{-}main[Remarks] to have some content on a second page, what}
\NormalTok{is a precondition \#index[Precondition] to demo that Index \#index[Index] entries}
\NormalTok{\#index[Entries] may point to multiple pages.}


\NormalTok{= Index}

\NormalTok{This section \#index[Section] contains the generated Index \#index[Index], in a nice}
\NormalTok{two{-}column{-}layout.}

\NormalTok{\#set text(size: 10pt)}
\NormalTok{\#columns(2)[}
\NormalTok{    \#make{-}index()}
\NormalTok{]}
\end{Highlighting}
\end{Shaded}

The following image shows a generated index page of another document,
with additional formatting for headers applied.

\pandocbounded{\includegraphics[keepaspectratio]{https://github.com/typst/packages/raw/main/packages/preview/in-dexter/0.5.3/gallery/SampleIndex.png}}

More usage samples are shown in the document
\texttt{\ sample-usage.typ\ } on
\href{https://github.com/RolfBremer/in-dexter}{in-dexter´s GitHub} .

A more complex sample PDF is available there as well.

\subsection{Changelog}\label{changelog}

\subsubsection{v0.5.3}\label{v0.5.3}

\begin{itemize}
\tightlist
\item
  fix error in typst.toml file.
\item
  Add a sample for raw display.
\end{itemize}

\subsubsection{v0.5.2}\label{v0.5.2}

\begin{itemize}
\tightlist
\item
  Fix a bug with bang notation.
\item
  Add compiler to toml file.
\end{itemize}

\subsubsection{v0.5.1}\label{v0.5.1}

\begin{itemize}
\tightlist
\item
  Migrate deprecated locate to context.
\end{itemize}

\subsubsection{v0.5.0}\label{v0.5.0}

\begin{itemize}
\tightlist
\item
  Support page numbering formats (i.e. roman), when
  \texttt{\ use-page-counter\ } is set to true. Thanks to
  @ThePuzzlemaker!
\end{itemize}

\subsubsection{v0.4.3}\label{v0.4.3}

\begin{itemize}
\tightlist
\item
  Suppress extra space character emitted by the \texttt{\ index()\ }
  function.
\item
  Fix a bug where math formulas are not displayed.
\item
  Introduce \texttt{\ apply-casing\ } parameter to \texttt{\ index()\ }
  to suppress entry-casing for individual entries.
\end{itemize}

\subsubsection{v0.4.2}\label{v0.4.2}

\begin{itemize}
\tightlist
\item
  Improve internal method \texttt{\ as-text\ } to be more robust.
\item
  tidy up sample-usage.typ.
\end{itemize}

\subsubsection{v0.4.1}\label{v0.4.1}

\begin{itemize}
\tightlist
\item
  Bug fixed: Fix a bug where an index entry with same name as a group
  hides the group.
\item
  Fixed typos in the sample-usage document.
\end{itemize}

\subsubsection{v0.4.0}\label{v0.4.0}

\begin{itemize}
\tightlist
\item
  Support for a \texttt{\ display\ } parameter for entries. This allows
  the usage of complex content, like math expressions in the index.
  (based on suggestions by @lukasjuhrich)
\item
  Also support a tuple value for display and key parameters of the
  entry.
\item
  Improve internal robustness and fix some errors in the sample
  document.
\end{itemize}

\subsubsection{v0.3.2}\label{v0.3.2}

\begin{itemize}
\tightlist
\item
  Fix initial parsing and returning fist letter (thanks to
  @lukasjuhrich, \#14)
\end{itemize}

\subsubsection{v0.3.1}\label{v0.3.1}

\begin{itemize}
\tightlist
\item
  Fix handling of trailing or multiple spaces and crlf in index entries.
\end{itemize}

\subsubsection{v0.3.0}\label{v0.3.0}

\begin{itemize}
\tightlist
\item
  Support multiple named indexes. Also allow the generation of combined
  index pages.
\item
  Support for LaTeX index group syntax (
  \texttt{\ \#index("Group1!Group2@Entry"\ } ).
\item
  Support for advanced case handling for the entries in the index. Note:
  The new default ist to ignore the casing for the sorting of the
  entries. The behavior can be changed by providing a
  \texttt{\ sort-order()\ } function to the \texttt{\ make-index\ }
  function.
\item
  The casing for the index entry can also be altered by providing a
  \texttt{\ entry-casing()\ } function to the \texttt{\ make-index\ }
  function. So it is possible that all entries have an uppercase first
  letter (which is also the new default!).
\end{itemize}

\subsubsection{v0.2.0}\label{v0.2.0}

\begin{itemize}
\tightlist
\item
  Allow index to respect unnumbered physical pages at the start of the
  document (Thanks to @jewelpit). See “Skipping physical pages� in
  the sample-usage document.
\end{itemize}

\subsubsection{v0.1.0}\label{v0.1.0}

\begin{itemize}
\tightlist
\item
  big refactor (by @epsilonhalbe).
\item
  changing “marker classes� to support direct format function
  \texttt{\ fmt:\ content\ -\textgreater{}\ content\ } e.g.
  \texttt{\ index(fmt:\ strong,\ {[}entry{]})\ } .
\item
  Implemented:

  \begin{itemize}
  \tightlist
  \item
    nested entries.
  \item
    custom initials + custom sorting.
  \end{itemize}
\end{itemize}

\subsubsection{v0.0.6}\label{v0.0.6}

\begin{itemize}
\tightlist
\item
  Change internal index marker to use metadata instead of figures. This
  allows a cleaner implementation and does not require a show rule to
  hide the marker-figure anymore.
\item
  This version requires Typst 0.8.0 due to the use of metadata().
\item
  Consolidated the \texttt{\ PackageReadme.md\ } into a single
  \texttt{\ README.md\ } .
\end{itemize}

\subsubsection{v0.0.5}\label{v0.0.5}

\begin{itemize}
\tightlist
\item
  Address change in \texttt{\ figure.caption\ } in typst (commit:
  976abdf ).
\end{itemize}

\subsubsection{v0.0.4}\label{v0.0.4}

\begin{itemize}
\tightlist
\item
  Add title and outline arguments to \#make-index() by @sbatial in \#4
\end{itemize}

\subsubsection{v0.0.3}\label{v0.0.3}

\begin{itemize}
\tightlist
\item
  Breaking: Renamed the main file from \texttt{\ index.typ\ } to
  \texttt{\ in-dexter.typ\ } to match package.
\item
  Added a Changelog to this README.
\item
  Introduced a brief and a full sample code to this README.
\item
  Added support for package manager in Typst.
\end{itemize}

\subsubsection{v.0.0.2}\label{v.0.0.2}

\begin{itemize}
\tightlist
\item
  Moved version to GitHub.
\end{itemize}

\subsubsection{How to add}\label{how-to-add}

Copy this into your project and use the import as \texttt{\ in-dexter\ }

\begin{verbatim}
#import "@preview/in-dexter:0.5.3"
\end{verbatim}

\includesvg[width=0.16667in,height=0.16667in]{/assets/icons/16-copy.svg}

Check the docs for
\href{https://typst.app/docs/reference/scripting/\#packages}{more
information on how to import packages} .

\subsubsection{About}\label{about}

\begin{description}
\tightlist
\item[Author s :]
\href{https://github.com/RolfBremer}{JKRB} \& in-dexter Contributors
\item[License:]
Apache-2.0
\item[Current version:]
0.5.3
\item[Last updated:]
August 14, 2024
\item[First released:]
July 10, 2023
\item[Minimum Typst version:]
0.11.0
\item[Archive size:]
11.3 kB
\href{https://packages.typst.org/preview/in-dexter-0.5.3.tar.gz}{\pandocbounded{\includesvg[keepaspectratio]{/assets/icons/16-download.svg}}}
\item[Repository:]
\href{https://github.com/RolfBremer/in-dexter}{GitHub}
\item[Categor y :]
\begin{itemize}
\tightlist
\item[]
\item
  \pandocbounded{\includesvg[keepaspectratio]{/assets/icons/16-package.svg}}
  \href{https://typst.app/universe/search/?category=components}{Components}
\end{itemize}
\end{description}

\subsubsection{Where to report issues?}\label{where-to-report-issues}

This package is a project of JKRB and in-dexter Contributors . Report
issues on \href{https://github.com/RolfBremer/in-dexter}{their
repository} . You can also try to ask for help with this package on the
\href{https://forum.typst.app}{Forum} .

Please report this package to the Typst team using the
\href{https://typst.app/contact}{contact form} if you believe it is a
safety hazard or infringes upon your rights.

\phantomsection\label{versions}
\subsubsection{Version history}\label{version-history}

\begin{longtable}[]{@{}ll@{}}
\toprule\noalign{}
Version & Release Date \\
\midrule\noalign{}
\endhead
\bottomrule\noalign{}
\endlastfoot
0.5.3 & August 14, 2024 \\
\href{https://typst.app/universe/package/in-dexter/0.5.2/}{0.5.2} &
August 23, 2024 \\
\href{https://typst.app/universe/package/in-dexter/0.4.2/}{0.4.2} & June
14, 2024 \\
\href{https://typst.app/universe/package/in-dexter/0.3.0/}{0.3.0} & May
13, 2024 \\
\href{https://typst.app/universe/package/in-dexter/0.2.0/}{0.2.0} &
April 30, 2024 \\
\href{https://typst.app/universe/package/in-dexter/0.1.0/}{0.1.0} &
January 8, 2024 \\
\href{https://typst.app/universe/package/in-dexter/0.0.6/}{0.0.6} &
October 1, 2023 \\
\href{https://typst.app/universe/package/in-dexter/0.0.5/}{0.0.5} &
September 13, 2023 \\
\href{https://typst.app/universe/package/in-dexter/0.0.4/}{0.0.4} &
August 6, 2023 \\
\href{https://typst.app/universe/package/in-dexter/0.0.3/}{0.0.3} & July
10, 2023 \\
\end{longtable}

Typst GmbH did not create this package and cannot guarantee correct
functionality of this package or compatibility with any version of the
Typst compiler or app.
