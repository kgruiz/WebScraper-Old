\title{typst.app/universe/package/problemst}

\phantomsection\label{banner}
\phantomsection\label{template-thumbnail}
\pandocbounded{\includegraphics[keepaspectratio]{https://packages.typst.org/preview/thumbnails/problemst-0.1.0-small.webp}}

\section{problemst}\label{problemst}

{ 0.1.0 }

Simple and easy-to-use template for problem sets/homeworks/assignments.

\href{/app?template=problemst&version=0.1.0}{Create project in app}

\phantomsection\label{readme}
Simple and easy-to-use template for problem sets/homeworks/assignments.

\pandocbounded{\includegraphics[keepaspectratio]{https://github.com/typst/packages/raw/main/packages/preview/problemst/0.1.0/template/thumbnail.png}}

\subsection{Usage}\label{usage}

Click “Start from template� in the Typst web app and search for
\texttt{\ problemst\ } .

Alternatively, run the following command to create a directory
initialized with all necessary files:

\begin{verbatim}
typst init @preview/problemst:0.1.0
\end{verbatim}

\subsection{Configuration}\label{configuration}

The \texttt{\ pset\ } function takes the following named arguments:

\begin{itemize}
\tightlist
\item
  \texttt{\ class\ } (string): Class the assignment is for.
\item
  \texttt{\ student\ } (string): Student completing the assignment.
\item
  \texttt{\ title\ } (string): Title of the assignment.
\item
  \texttt{\ date\ } (datetime): Date to be displayed on the assignment.
\item
  \texttt{\ collaborators\ } (array of strings): Collaborators that
  worked on the assignment with the student. Can be \texttt{\ ()\ } .
\item
  \texttt{\ subproblems\ } (string): Numbering scheme for the
  subproblems.
\end{itemize}

\href{/app?template=problemst&version=0.1.0}{Create project in app}

\subsubsection{How to use}\label{how-to-use}

Click the button above to create a new project using this template in
the Typst app.

You can also use the Typst CLI to start a new project on your computer
using this command:

\begin{verbatim}
typst init @preview/problemst:0.1.0
\end{verbatim}

\includesvg[width=0.16667in,height=0.16667in]{/assets/icons/16-copy.svg}

\subsubsection{About}\label{about}

\begin{description}
\tightlist
\item[Author :]
\href{https://github.com/carreter}{Willow Carretero Chavez}
\item[License:]
MIT
\item[Current version:]
0.1.0
\item[Last updated:]
April 17, 2024
\item[First released:]
April 17, 2024
\item[Minimum Typst version:]
0.11.0
\item[Archive size:]
2.63 kB
\href{https://packages.typst.org/preview/problemst-0.1.0.tar.gz}{\pandocbounded{\includesvg[keepaspectratio]{/assets/icons/16-download.svg}}}
\item[Categor y :]
\begin{itemize}
\tightlist
\item[]
\item
  \pandocbounded{\includesvg[keepaspectratio]{/assets/icons/16-speak.svg}}
  \href{https://typst.app/universe/search/?category=report}{Report}
\end{itemize}
\end{description}

\subsubsection{Where to report issues?}\label{where-to-report-issues}

This template is a project of Willow Carretero Chavez . You can also try
to ask for help with this template on the
\href{https://forum.typst.app}{Forum} .

Please report this template to the Typst team using the
\href{https://typst.app/contact}{contact form} if you believe it is a
safety hazard or infringes upon your rights.

\phantomsection\label{versions}
\subsubsection{Version history}\label{version-history}

\begin{longtable}[]{@{}ll@{}}
\toprule\noalign{}
Version & Release Date \\
\midrule\noalign{}
\endhead
\bottomrule\noalign{}
\endlastfoot
0.1.0 & April 17, 2024 \\
\end{longtable}

Typst GmbH did not create this template and cannot guarantee correct
functionality of this template or compatibility with any version of the
Typst compiler or app.
