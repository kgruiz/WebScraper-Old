\title{typst.app/universe/package/algorithmic}

\phantomsection\label{banner}
\section{algorithmic}\label{algorithmic}

{ 0.1.0 }

Algorithm pseudocode typesetting for Typst, inspired by algorithmicx in
LaTeX

\phantomsection\label{readme}
This is a package inspired by the LaTeX
\href{https://ctan.org/pkg/algorithmicx}{\texttt{\ algorithmicx\ }}
package for Typst. It’s useful for writing pseudocode and typesetting
it all nicely.

\pandocbounded{\includegraphics[keepaspectratio]{https://github.com/typst/packages/raw/main/packages/preview/algorithmic/0.1.0/docs/assets/demo-rendered.png}}

Example:

\begin{Shaded}
\begin{Highlighting}[]
\NormalTok{\#import "@preview/algorithmic:0.1.0"}
\NormalTok{\#import algorithmic: algorithm}

\NormalTok{\#algorithm(\{}
\NormalTok{  import algorithmic: *}
\NormalTok{  Function("Binary{-}Search", args: ("A", "n", "v"), \{}
\NormalTok{    Cmt[Initialize the search range]}
\NormalTok{    Assign[$l$][$1$]}
\NormalTok{    Assign[$r$][$n$]}
\NormalTok{    State[]}
\NormalTok{    While(cond: $l \textless{}= r$, \{}
\NormalTok{      Assign([mid], FnI[floor][$(l + r)/2$])}
\NormalTok{      If(cond: $A ["mid"] \textless{} v$, \{}
\NormalTok{        Assign[$l$][$m + 1$]}
\NormalTok{      \})}
\NormalTok{      ElsIf(cond: [$A ["mid"] \textgreater{} v$], \{}
\NormalTok{        Assign[$r$][$m {-} 1$]}
\NormalTok{      \})}
\NormalTok{      Else(\{}
\NormalTok{        Return[$m$]}
\NormalTok{      \})}
\NormalTok{    \})}
\NormalTok{    Return[*null*]}
\NormalTok{  \})}
\NormalTok{\})}
\end{Highlighting}
\end{Shaded}

This DSL is implemented using the same trick as
\href{https://github.com/johannes-wolf/typst-canvas}{CeTZ} uses: a code
block of arrays gets those arrays joined together.

Currently this library is not really customizable. Please vendor it and
hack it up as needed then file an issue for the customization option
you’re missing.

\subsection{Reference}\label{reference}

\paragraph{stmt}\label{stmt}

Statement-level contexts in \texttt{\ algorithmic\ } generally accept
the type \texttt{\ body\ } in the following:

\begin{verbatim}
body = (ast|content)[] | ast | content
ast = (change_indent: int, body: body)
\end{verbatim}

\paragraph{inline}\label{inline}

Inline functions will generate plain content.

\paragraph{\texorpdfstring{\texttt{\ algorithmic(..bits)\ }}{ algorithmic(..bits) }}\label{algorithmic..bits}

Takes one or more lists of \texttt{\ ast\ } and creates an algorithmic
block with line numbers.

\subsubsection{Control flow}\label{control-flow}

\paragraph{\texorpdfstring{\texttt{\ Function\ } /
\texttt{\ Procedure\ }
(stmt)}{ Function  /  Procedure  (stmt)}}\label{function-procedure-stmt}

Defined as
\texttt{\ f(name:\ string\textbar{}content,\ args:\ content{[}{]}?,\ ..body)\ }
. Body can be one or more \texttt{\ body\ } values.

\paragraph{\texorpdfstring{\texttt{\ If\ } / \texttt{\ ElseIf\ } /
\texttt{\ Else\ } / \texttt{\ For\ } / \texttt{\ While\ }
(stmt)}{ If  /  ElseIf  /  Else  /  For  /  While  (stmt)}}\label{if-elseif-else-for-while-stmt}

Defined as \texttt{\ f(cond:\ string\textbar{}content,\ ..body)\ } .
Body can be one or more \texttt{\ body\ } values.

Generates an indented block with the body, and the specified
\texttt{\ cond\ } between the two keywords as condition.

\subsubsection{Statements}\label{statements}

\paragraph{\texorpdfstring{\texttt{\ Assign\ }
(stmt)}{ Assign  (stmt)}}\label{assign-stmt}

Defined as \texttt{\ Assign(var:\ content,\ val:\ content)\ } .

Generates \texttt{\ \#var\ \textless{}-\ \#val\ } .

\paragraph{\texorpdfstring{\texttt{\ CallI\ } (inline),
\texttt{\ Call\ }
(stmt)}{ CallI  (inline),  Call  (stmt)}}\label{calli-inline-call-stmt}

Defined as \texttt{\ f(name,\ args:\ content\textbar{}content{[}{]})\ }
.

Calls a function with the function name styled in smallcaps and the args
joined by commas.

\paragraph{\texorpdfstring{\texttt{\ Cmt\ }
(stmt)}{ Cmt  (stmt)}}\label{cmt-stmt}

Defined as \texttt{\ Cmt(body:\ content)\ } .

Makes a line comment.

\paragraph{\texorpdfstring{\texttt{\ FnI\ } (inline), \texttt{\ Fn\ }
(stmt)}{ FnI  (inline),  Fn  (stmt)}}\label{fni-inline-fn-stmt}

Defined as \texttt{\ f(name,\ args:\ content\textbar{}content{[}{]})\ }
.

Calls a function with the function name styled in bold and the args
joined by commas.

\paragraph{\texorpdfstring{\texttt{\ Ic\ }
(inline)}{ Ic  (inline)}}\label{ic-inline}

Defined as \texttt{\ Ic(body:\ content)\ -\textgreater{}\ content\ } .

Makes an inline comment.

\paragraph{\texorpdfstring{\texttt{\ Return\ }
(stmt)}{ Return  (stmt)}}\label{return-stmt}

Defined as \texttt{\ Return(arg:\ content)\ } .

Generates \texttt{\ return\ \#arg\ } .

\paragraph{\texorpdfstring{\texttt{\ State\ }
(stmt)}{ State  (stmt)}}\label{state-stmt}

Defined as \texttt{\ State(body:\ content)\ } .

Turns any content into a line.

\subsubsection{How to add}\label{how-to-add}

Copy this into your project and use the import as
\texttt{\ algorithmic\ }

\begin{verbatim}
#import "@preview/algorithmic:0.1.0"
\end{verbatim}

\includesvg[width=0.16667in,height=0.16667in]{/assets/icons/16-copy.svg}

Check the docs for
\href{https://typst.app/docs/reference/scripting/\#packages}{more
information on how to import packages} .

\subsubsection{About}\label{about}

\begin{description}
\tightlist
\item[Author :]
Jade Lovelace
\item[License:]
MIT
\item[Current version:]
0.1.0
\item[Last updated:]
August 19, 2023
\item[First released:]
August 19, 2023
\item[Archive size:]
3.29 kB
\href{https://packages.typst.org/preview/algorithmic-0.1.0.tar.gz}{\pandocbounded{\includesvg[keepaspectratio]{/assets/icons/16-download.svg}}}
\item[Repository:]
\href{https://github.com/lf-/typst-algorithmic}{GitHub}
\end{description}

\subsubsection{Where to report issues?}\label{where-to-report-issues}

This package is a project of Jade Lovelace . Report issues on
\href{https://github.com/lf-/typst-algorithmic}{their repository} . You
can also try to ask for help with this package on the
\href{https://forum.typst.app}{Forum} .

Please report this package to the Typst team using the
\href{https://typst.app/contact}{contact form} if you believe it is a
safety hazard or infringes upon your rights.

\phantomsection\label{versions}
\subsubsection{Version history}\label{version-history}

\begin{longtable}[]{@{}ll@{}}
\toprule\noalign{}
Version & Release Date \\
\midrule\noalign{}
\endhead
\bottomrule\noalign{}
\endlastfoot
0.1.0 & August 19, 2023 \\
\end{longtable}

Typst GmbH did not create this package and cannot guarantee correct
functionality of this package or compatibility with any version of the
Typst compiler or app.
