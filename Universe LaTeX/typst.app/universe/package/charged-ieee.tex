\title{typst.app/universe/package/charged-ieee}

\phantomsection\label{banner}
\phantomsection\label{template-thumbnail}
\pandocbounded{\includegraphics[keepaspectratio]{https://packages.typst.org/preview/thumbnails/charged-ieee-0.1.3-small.webp}}

\section{charged-ieee}\label{charged-ieee}

{ 0.1.3 }

An IEEE-style paper template to publish at conferences and journals for
Electrical Engineering, Computer Science, and Computer Engineering

{ } Featured Template

\href{/app?template=charged-ieee&version=0.1.3}{Create project in app}

\phantomsection\label{readme}
This is a Typst template for a two-column paper from the proceedings of
the Institute of Electrical and Electronics Engineers. The paper is
tightly spaced, fits a lot of content and comes preconfigured for
numeric citations from BibLaTeX or Hayagriva files.

\subsection{Usage}\label{usage}

You can use this template in the Typst web app by clicking “Start from
template� on the dashboard and searching for \texttt{\ charged-ieee\ }
.

Alternatively, you can use the CLI to kick this project off using the
command

\begin{verbatim}
typst init @preview/charged-ieee
\end{verbatim}

Typst will create a new directory with all the files needed to get you
started.

\subsection{Configuration}\label{configuration}

This template exports the \texttt{\ ieee\ } function with the following
named arguments:

\begin{itemize}
\tightlist
\item
  \texttt{\ title\ } : The paper’s title as content.
\item
  \texttt{\ authors\ } : An array of author dictionaries. Each of the
  author dictionaries must have a \texttt{\ name\ } key and can have the
  keys \texttt{\ department\ } , \texttt{\ organization\ } ,
  \texttt{\ location\ } , and \texttt{\ email\ } . All keys accept
  content.
\item
  \texttt{\ abstract\ } : The content of a brief summary of the paper or
  \texttt{\ none\ } . Appears at the top of the first column in
  boldface.
\item
  \texttt{\ index-terms\ } : Array of index terms to display after the
  abstract. Shall be \texttt{\ content\ } .
\item
  \texttt{\ paper-size\ } : Defaults to \texttt{\ us-letter\ } . Specify
  a
  \href{https://typst.app/docs/reference/layout/page/\#parameters-paper}{paper
  size string} to change the page format.
\item
  \texttt{\ bibliography\ } : The result of a call to the
  \texttt{\ bibliography\ } function or \texttt{\ none\ } . Specifying
  this will configure numeric, IEEE-style citations.
\end{itemize}

The function also accepts a single, positional argument for the body of
the paper.

The template will initialize your package with a sample call to the
\texttt{\ ieee\ } function in a show rule. If you want to change an
existing project to use this template, you can add a show rule like this
at the top of your file:

\begin{Shaded}
\begin{Highlighting}[]
\NormalTok{\#import "@preview/charged{-}ieee:0.1.3": ieee}

\NormalTok{\#show: ieee.with(}
\NormalTok{  title: [A typesetting system to untangle the scientific writing process],}
\NormalTok{  abstract: [}
\NormalTok{    The process of scientific writing is often tangled up with the intricacies of typesetting, leading to frustration and wasted time for researchers. In this paper, we introduce Typst, a new typesetting system designed specifically for scientific writing. Typst untangles the typesetting process, allowing researchers to compose papers faster. In a series of experiments we demonstrate that Typst offers several advantages, including faster document creation, simplified syntax, and increased ease{-}of{-}use.}
\NormalTok{  ],}
\NormalTok{  authors: (}
\NormalTok{    (}
\NormalTok{      name: "Martin Haug",}
\NormalTok{      department: [Co{-}Founder],}
\NormalTok{      organization: [Typst GmbH],}
\NormalTok{      location: [Berlin, Germany],}
\NormalTok{      email: "haug@typst.app"}
\NormalTok{    ),}
\NormalTok{    (}
\NormalTok{      name: "Laurenz Mädje",}
\NormalTok{      department: [Co{-}Founder],}
\NormalTok{      organization: [Typst GmbH],}
\NormalTok{      location: [Berlin, Germany],}
\NormalTok{      email: "maedje@typst.app"}
\NormalTok{    ),}
\NormalTok{  ),}
\NormalTok{  index{-}terms: ("Scientific writing", "Typesetting", "Document creation", "Syntax"),}
\NormalTok{  bibliography: bibliography("refs.bib"),}
\NormalTok{)}

\NormalTok{// Your content goes below.}
\end{Highlighting}
\end{Shaded}

\href{/app?template=charged-ieee&version=0.1.3}{Create project in app}

\subsubsection{How to use}\label{how-to-use}

Click the button above to create a new project using this template in
the Typst app.

You can also use the Typst CLI to start a new project on your computer
using this command:

\begin{verbatim}
typst init @preview/charged-ieee:0.1.3
\end{verbatim}

\includesvg[width=0.16667in,height=0.16667in]{/assets/icons/16-copy.svg}

\subsubsection{About}\label{about}

\begin{description}
\tightlist
\item[Author :]
\href{https://typst.app}{Typst GmbH}
\item[License:]
MIT-0
\item[Current version:]
0.1.3
\item[Last updated:]
October 29, 2024
\item[First released:]
March 6, 2024
\item[Minimum Typst version:]
0.12.0
\item[Archive size:]
6.39 kB
\href{https://packages.typst.org/preview/charged-ieee-0.1.3.tar.gz}{\pandocbounded{\includesvg[keepaspectratio]{/assets/icons/16-download.svg}}}
\item[Repository:]
\href{https://github.com/typst/templates}{GitHub}
\item[Discipline s :]
\begin{itemize}
\tightlist
\item[]
\item
  \href{https://typst.app/universe/search/?discipline=computer-science}{Computer
  Science}
\item
  \href{https://typst.app/universe/search/?discipline=engineering}{Engineering}
\end{itemize}
\item[Categor y :]
\begin{itemize}
\tightlist
\item[]
\item
  \pandocbounded{\includesvg[keepaspectratio]{/assets/icons/16-atom.svg}}
  \href{https://typst.app/universe/search/?category=paper}{Paper}
\end{itemize}
\end{description}

\subsubsection{Where to report issues?}\label{where-to-report-issues}

This template is a project of Typst GmbH . Report issues on
\href{https://github.com/typst/templates}{their repository} . You can
also try to ask for help with this template on the
\href{https://forum.typst.app}{Forum} .

\phantomsection\label{versions}
\subsubsection{Version history}\label{version-history}

\begin{longtable}[]{@{}ll@{}}
\toprule\noalign{}
Version & Release Date \\
\midrule\noalign{}
\endhead
\bottomrule\noalign{}
\endlastfoot
0.1.3 & October 29, 2024 \\
\href{https://typst.app/universe/package/charged-ieee/0.1.2/}{0.1.2} &
August 15, 2024 \\
\href{https://typst.app/universe/package/charged-ieee/0.1.1/}{0.1.1} &
August 8, 2024 \\
\href{https://typst.app/universe/package/charged-ieee/0.1.0/}{0.1.0} &
March 6, 2024 \\
\end{longtable}
