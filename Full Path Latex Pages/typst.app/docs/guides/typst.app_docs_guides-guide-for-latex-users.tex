\title{typst.app/docs/guides/guide-for-latex-users}

\begin{itemize}
\tightlist
\item
  \href{/docs}{\includesvg[width=0.16667in,height=0.16667in]{/assets/icons/16-docs-dark.svg}}
\item
  \includesvg[width=0.16667in,height=0.16667in]{/assets/icons/16-arrow-right.svg}
\item
  \href{/docs/guides/}{Guides}
\item
  \includesvg[width=0.16667in,height=0.16667in]{/assets/icons/16-arrow-right.svg}
\item
  \href{/docs/guides/guide-for-latex-users/}{Guide for LaTeX users}
\end{itemize}

\section{Guide for LaTeX users}\label{guide-for-latex-users}

This page is a good starting point if you have used LaTeX before and
want to try out Typst. We will explore the main differences between
these two systems from a user perspective. Although Typst is not built
upon LaTeX and has a different syntax, you will learn how to use your
LaTeX skills to get a head start.

Just like LaTeX, Typst is a markup-based typesetting system: You compose
your document in a text file and mark it up with commands and other
syntax. Then, you use a compiler to typeset the source file into a PDF.
However, Typst also differs from LaTeX in several aspects: For one,
Typst uses more dedicated syntax (like you may know from Markdown) for
common tasks. Typst\textquotesingle s commands are also more principled:
They all work the same, so unlike in LaTeX, you just need to understand
a few general concepts instead of learning different conventions for
each package. Moreover Typst compiles faster than LaTeX: Compilation
usually takes milliseconds, not seconds, so the web app and the compiler
can both provide instant previews.

In the following, we will cover some of the most common questions a user
switching from LaTeX will have when composing a document in Typst. If
you prefer a step-by-step introduction to Typst, check out our
\href{/docs/tutorial/}{tutorial} .

\subsection{Installation}\label{installation}

You have two ways to use Typst: In \href{https://typst.app/signup/}{our
web app} or by \href{https://github.com/typst/typst/releases}{installing
the compiler} on your computer. When you use the web app, we provide a
batteries-included collaborative editor and run Typst in your browser,
no installation required.

If you choose to use Typst on your computer instead, you can download
the compiler as a single, small binary which any user can run, no root
privileges required. Unlike LaTeX, packages are downloaded when you
first use them and then cached locally, keeping your Typst installation
lean. You can use your own editor and decide where to store your files
with the local compiler.

\subsection{How do I create a new, empty
document?}\label{getting-started}

That\textquotesingle s easy. You just create a new, empty text file (the
file extension is \texttt{\ .typ\ } ). No boilerplate is needed to get
started. Simply start by writing your text. It will be set on an empty
A4-sized page. If you are using the web app, click "+ Empty document" to
create a new project with a file and enter the editor.
\href{/docs/reference/model/parbreak/}{Paragraph breaks} work just as
they do in LaTeX, just use a blank line.

\begin{verbatim}
Hey there!

Here are two paragraphs. The
output is shown to the right.
\end{verbatim}

\includegraphics[width=5in,height=\textheight,keepaspectratio]{/assets/docs/1Xaxf-Fz-W7agcXqKmLyjQAAAAAAAAAA.png}

If you want to start from an preexisting LaTeX document instead, you can
use \href{https://pandoc.org}{Pandoc} to convert your source code to
Typst markup. This conversion is also built into our web app, so you can
upload your \texttt{\ .tex\ } file to start your project in Typst.

\subsection{How do I create section headings, emphasis,
...?}\label{elements}

LaTeX uses the command \texttt{\ \textbackslash{}section\ } to create a
section heading. Nested headings are indicated with
\texttt{\ \textbackslash{}subsection\ } ,
\texttt{\ \textbackslash{}subsubsection\ } , etc. Depending on your
document class, there is also \texttt{\ \textbackslash{}part\ } or
\texttt{\ \textbackslash{}chapter\ } .

In Typst, \href{/docs/reference/model/heading/}{headings} are less
verbose: You prefix the line with the heading on it with an equals sign
and a space to get a first-order heading:
\texttt{\ }{\texttt{\ =\ Introduction\ }}\texttt{\ } . If you need a
second-order heading, you use two equals signs:
\texttt{\ }{\texttt{\ ==\ In\ this\ paper\ }}\texttt{\ } . You can nest
headings as deeply as you\textquotesingle d like by adding more equals
signs.

Emphasis (usually rendered as italic text) is expressed by enclosing
text in \texttt{\ }{\texttt{\ \_underscores\_\ }}\texttt{\ } and strong
emphasis (usually rendered in boldface) by using
\texttt{\ }{\texttt{\ *stars*\ }}\texttt{\ } instead.

Here is a list of common markup commands used in LaTeX and their Typst
equivalents. You can also check out the
\href{/docs/reference/syntax/}{full syntax cheat sheet} .

\begin{longtable}[]{@{}llll@{}}
\toprule\noalign{}
Element & LaTeX & Typst & See \\
\midrule\noalign{}
\endhead
\bottomrule\noalign{}
\endlastfoot
Strong emphasis & \texttt{\ \textbackslash{}textbf\{strong\}\ } &
\texttt{\ }{\texttt{\ *strong*\ }}\texttt{\ } &
\href{/docs/reference/model/strong/}{\texttt{\ strong\ }} \\
Emphasis & \texttt{\ \textbackslash{}emph\{emphasis\}\ } &
\texttt{\ }{\texttt{\ \_emphasis\_\ }}\texttt{\ } &
\href{/docs/reference/model/emph/}{\texttt{\ emph\ }} \\
Monospace / code & \texttt{\ \textbackslash{}texttt\{print(1)\}\ } &
\texttt{\ }{\texttt{\ \textasciigrave{}print(1)\textasciigrave{}\ }}\texttt{\ }
& \href{/docs/reference/text/raw/}{\texttt{\ raw\ }} \\
Link & \texttt{\ \textbackslash{}url\{https://typst.app\}\ } &
\texttt{\ }{\texttt{\ https://typst.app/\ }}\texttt{\ } &
\href{/docs/reference/model/link/}{\texttt{\ link\ }} \\
Label & \texttt{\ \textbackslash{}label\{intro\}\ } &
\texttt{\ }{\texttt{\ \textless{}intro\textgreater{}\ }}\texttt{\ } &
\href{/docs/reference/foundations/label/}{\texttt{\ label\ }} \\
Reference & \texttt{\ \textbackslash{}ref\{intro\}\ } &
\texttt{\ }{\texttt{\ @intro\ }}\texttt{\ } &
\href{/docs/reference/model/ref/}{\texttt{\ ref\ }} \\
Citation & \texttt{\ \textbackslash{}cite\{humphrey97\}\ } &
\texttt{\ }{\texttt{\ @humphrey97\ }}\texttt{\ } &
\href{/docs/reference/model/cite/}{\texttt{\ cite\ }} \\
Bullet list & \texttt{\ itemize\ } environment &
\texttt{\ }{\texttt{\ -\ }}\texttt{\ List\ } &
\href{/docs/reference/model/list/}{\texttt{\ list\ }} \\
Numbered list & \texttt{\ enumerate\ } environment &
\texttt{\ }{\texttt{\ +\ }}\texttt{\ List\ } &
\href{/docs/reference/model/enum/}{\texttt{\ enum\ }} \\
Term list & \texttt{\ description\ } environment &
\texttt{\ }{\texttt{\ /\ }}\texttt{\ }{\texttt{\ Term\ }}\texttt{\ }{\texttt{\ :\ }}\texttt{\ List\ }
& \href{/docs/reference/model/terms/}{\texttt{\ terms\ }} \\
Figure & \texttt{\ figure\ } environment & \texttt{\ figure\ } function
& \href{/docs/reference/model/figure/}{\texttt{\ figure\ }} \\
Table & \texttt{\ table\ } environment & \texttt{\ table\ } function &
\href{/docs/reference/model/table/}{\texttt{\ table\ }} \\
Equation & \texttt{\ \$x\$\ } , \texttt{\ align\ } /
\texttt{\ equation\ } environments &
\texttt{\ }{\texttt{\ \$\ }}\texttt{\ x\ }{\texttt{\ \$\ }}\texttt{\ } ,
\texttt{\ }{\texttt{\ \$\ }}\texttt{\ x\ =\ y\ }{\texttt{\ \$\ }}\texttt{\ }
& \href{/docs/reference/math/equation/}{\texttt{\ equation\ }} \\
\end{longtable}

\href{/docs/reference/model/list/}{Lists} do not rely on environments in
Typst. Instead, they have lightweight syntax like headings. To create an
unordered list ( \texttt{\ itemize\ } ), prefix each line of an item
with a hyphen:

\begin{verbatim}
To write this list in Typst...

```latex
\begin{itemize}
  \item Fast
  \item Flexible
  \item Intuitive
\end{itemize}
```

...just type this:

- Fast
- Flexible
- Intuitive
\end{verbatim}

\includegraphics[width=5in,height=\textheight,keepaspectratio]{/assets/docs/EtxJBpmzn-Q3caUem89mOQAAAAAAAAAA.png}

Nesting lists works just by using proper indentation. Adding a blank
line in between items results in a more
\href{/docs/reference/model/list/\#parameters-tight}{widely} spaced
list.

To get a \href{/docs/reference/model/enum/}{numbered list} (
\texttt{\ enumerate\ } ) instead, use a \texttt{\ +\ } instead of the
hyphen. For a \href{/docs/reference/model/terms/}{term list} (
\texttt{\ description\ } ), write
\texttt{\ }{\texttt{\ /\ }}\texttt{\ }{\texttt{\ Term\ }}\texttt{\ }{\texttt{\ :\ }}\texttt{\ Description\ }
instead.

\subsection{How do I use a command?}\label{commands}

LaTeX heavily relies on commands (prefixed by backslashes). It uses
these \emph{macros} to affect the typesetting process and to insert and
manipulate content. Some commands accept arguments, which are most
frequently enclosed in curly braces:
\texttt{\ \textbackslash{}cite\{rasmus\}\ } .

Typst differentiates between
\href{/docs/reference/scripting/\#blocks}{markup mode and code mode} .
The default is markup mode, where you compose text and apply syntactic
constructs such as
\texttt{\ }{\texttt{\ *stars\ for\ bold\ text*\ }}\texttt{\ } . Code
mode, on the other hand, parallels programming languages like Python,
providing the option to input and execute segments of code.

Within Typst\textquotesingle s markup, you can switch to code mode for a
single command (or rather, \emph{expression} ) using a hash (
\texttt{\ \#\ } ). This is how you call functions to, for example, split
your project into different
\href{/docs/reference/scripting/\#modules}{files} or render text based
on some \href{/docs/reference/scripting/\#conditionals}{condition} .
Within code mode, it is possible to include normal markup
\href{/docs/reference/foundations/content/}{\emph{content}} by using
square brackets. Within code mode, this content is treated just as any
other normal value for a variable.

\begin{verbatim}
First, a rectangle:
#rect()

Let me show how to do
#underline([_underlined_ text])

We can also do some maths:
#calc.max(3, 2 * 4)

And finally a little loop:
#for x in range(3) [
  Hi #x.
]
\end{verbatim}

\includegraphics[width=5in,height=\textheight,keepaspectratio]{/assets/docs/YRh-AUkIq4C1mI8DVTggswAAAAAAAAAA.png}

A function call always involves the name of the function (
\href{/docs/reference/visualize/rect/}{\texttt{\ rect\ }} ,
\href{/docs/reference/text/underline/}{\texttt{\ underline\ }} ,
\href{/docs/reference/foundations/calc/\#functions-max}{\texttt{\ calc.max\ }}
,
\href{/docs/reference/foundations/array/\#definitions-range}{\texttt{\ range\ }}
) followed by parentheses (as opposed to LaTeX where the square brackets
and curly braces are optional if the macro requires no arguments). The
expected list of arguments passed within those parentheses depends on
the concrete function and is specified in the
\href{/docs/reference/}{reference} .

\subsubsection{Arguments}\label{arguments}

A function can have multiple arguments. Some arguments are positional,
i.e., you just provide the value: The function
\texttt{\ }{\texttt{\ \#\ }}\texttt{\ }{\texttt{\ lower\ }}\texttt{\ }{\texttt{\ (\ }}\texttt{\ }{\texttt{\ "SCREAM"\ }}\texttt{\ }{\texttt{\ )\ }}\texttt{\ }
returns its argument in all-lowercase. Many functions use named
arguments instead of positional arguments to increase legibility. For
example, the dimensions and stroke of a rectangle are defined with named
arguments:

\begin{verbatim}
#rect(
  width: 2cm,
  height: 1cm,
  stroke: red,
)
\end{verbatim}

\includegraphics[width=5in,height=\textheight,keepaspectratio]{/assets/docs/qhelTU2eEzhkL0zp__ciIAAAAAAAAAAA.png}

You specify a named argument by first entering its name (above,
it\textquotesingle s \texttt{\ width\ } , \texttt{\ height\ } , and
\texttt{\ stroke\ } ), then a colon, followed by the value (
\texttt{\ 2cm\ } , \texttt{\ 1cm\ } , \texttt{\ red\ } ). You can find
the available named arguments in the \href{/docs/reference/}{reference
page} for each function or in the autocomplete panel when typing. Named
arguments are similar to how some LaTeX environments are configured, for
example, you would type
\texttt{\ \textbackslash{}begin\{enumerate\}{[}label=\{\textbackslash{}alph*)\}{]}\ }
to start a list with the labels \texttt{\ a)\ } , \texttt{\ b)\ } , and
so on.

Often, you want to provide some
\href{/docs/reference/foundations/content/}{content} to a function. For
example, the LaTeX command
\texttt{\ \textbackslash{}underline\{Alternative\ A\}\ } would translate
to
\texttt{\ }{\texttt{\ \#\ }}\texttt{\ }{\texttt{\ underline\ }}\texttt{\ }{\texttt{\ (\ }}\texttt{\ }{\texttt{\ {[}\ }}\texttt{\ Alternative\ A\ }{\texttt{\ {]}\ }}\texttt{\ }{\texttt{\ )\ }}\texttt{\ }
in Typst. The square brackets indicate that a value is
\href{/docs/reference/foundations/content/}{content} . Within these
brackets, you can use normal markup. However, that\textquotesingle s a
lot of parentheses for a pretty simple construct. This is why you can
also move trailing content arguments after the parentheses (and omit the
parentheses if they would end up empty).

\begin{verbatim}
Typst is an #underline[alternative]
to LaTeX.

#rect(fill: aqua)[Get started here!]
\end{verbatim}

\includegraphics[width=5in,height=\textheight,keepaspectratio]{/assets/docs/mXv-36m3C2iLosBLIxMjHwAAAAAAAAAA.png}

\subsubsection{Data types}\label{data-types}

You likely already noticed that the arguments have distinctive data
types. Typst supports many \href{/docs/reference/foundations/type/}{data
types} . Below, there is a table with some of the most important ones
and how to write them. In order to specify values of any of these types,
you have to be in code mode!

\begin{longtable}[]{@{}ll@{}}
\toprule\noalign{}
Data type & Example \\
\midrule\noalign{}
\endhead
\bottomrule\noalign{}
\endlastfoot
\href{/docs/reference/foundations/content/}{Content} &
\texttt{\ }{\texttt{\ {[}\ }}\texttt{\ }{\texttt{\ *fast*\ }}\texttt{\ typesetting\ }{\texttt{\ {]}\ }}\texttt{\ } \\
\href{/docs/reference/foundations/str/}{String} &
\texttt{\ }{\texttt{\ "Pietro\ S.\ Author"\ }}\texttt{\ } \\
\href{/docs/reference/foundations/int/}{Integer} &
\texttt{\ }{\texttt{\ 23\ }}\texttt{\ } \\
\href{/docs/reference/foundations/float/}{Floating point number} &
\texttt{\ }{\texttt{\ 1.459\ }}\texttt{\ } \\
\href{/docs/reference/layout/length/}{Absolute length} &
\texttt{\ }{\texttt{\ 12pt\ }}\texttt{\ } ,
\texttt{\ }{\texttt{\ 5in\ }}\texttt{\ } ,
\texttt{\ }{\texttt{\ 0.3cm\ }}\texttt{\ } , ... \\
\href{/docs/reference/layout/ratio/}{Relative length} &
\texttt{\ }{\texttt{\ 65\%\ }}\texttt{\ } \\
\end{longtable}

The difference between content and string is that content can contain
markup, including function calls, while a string really is just a plain
sequence of characters.

Typst provides \href{/docs/reference/scripting/\#conditionals}{control
flow constructs} and
\href{/docs/reference/scripting/\#operators}{operators} such as
\texttt{\ +\ } for adding things or \texttt{\ ==\ } for checking
equality between two variables.

You can also store values, including functions, in your own
\href{/docs/reference/scripting/\#bindings}{variables} . This can be
useful to perform computations on them, create reusable automations, or
reference a value multiple times. The variable binding is accomplished
with the let keyword, which works similar to
\texttt{\ \textbackslash{}newcommand\ } :

\begin{verbatim}
// Store the integer `5`.
#let five = 5

// Define a function that
// increments a value.
#let inc(i) = i + 1

// Reference the variables.
I have #five fingers.

If I had one more, I'd have
#inc(five) fingers. Whoa!
\end{verbatim}

\includegraphics[width=5in,height=\textheight,keepaspectratio]{/assets/docs/B033fxNtvCwYya9MX4R6xwAAAAAAAAAA.png}

\subsubsection{Commands to affect the remaining document}\label{rules}

In LaTeX, some commands like
\texttt{\ \textbackslash{}textbf\{bold\ text\}\ } receive an argument in
curly braces and only affect that argument. Other commands such as
\texttt{\ \textbackslash{}bfseries\ bold\ text\ } act as switches (LaTeX
calls this a declaration), altering the appearance of all subsequent
content within the document or current scope.

In Typst, the same function can be used both to affect the appearance
for the remainder of the document, a block (or scope), or just its
arguments. For example,
\texttt{\ }{\texttt{\ \#\ }}\texttt{\ }{\texttt{\ text\ }}\texttt{\ }{\texttt{\ (\ }}\texttt{\ weight\ }{\texttt{\ :\ }}\texttt{\ }{\texttt{\ "bold"\ }}\texttt{\ }{\texttt{\ )\ }}\texttt{\ }{\texttt{\ {[}\ }}\texttt{\ bold\ text\ }{\texttt{\ {]}\ }}\texttt{\ }
will only embolden its argument, while
\texttt{\ }{\texttt{\ \#\ }}\texttt{\ }{\texttt{\ set\ }}\texttt{\ }{\texttt{\ text\ }}\texttt{\ }{\texttt{\ (\ }}\texttt{\ weight\ }{\texttt{\ :\ }}\texttt{\ }{\texttt{\ "bold"\ }}\texttt{\ }{\texttt{\ )\ }}\texttt{\ }
will embolden any text until the end of the current block, or, if there
is none, document. The effects of a function are immediately obvious
based on whether it is used in a call or a
\href{/docs/reference/styling/\#set-rules}{set rule.}

\begin{verbatim}
I am starting out with small text.

#set text(14pt)

This is a bit #text(18pt)[larger,]
don't you think?
\end{verbatim}

\includegraphics[width=5in,height=\textheight,keepaspectratio]{/assets/docs/aX-wYquEk7ekHdAWcGZ3-wAAAAAAAAAA.png}

Set rules may appear anywhere in the document. They can be thought of as
default argument values of their respective function:

\begin{verbatim}
#set enum(numbering: "I.")

Good results can only be obtained by
+ following best practices
+ being aware of current results
  of other researchers
+ checking the data for biases
\end{verbatim}

\includegraphics[width=5in,height=\textheight,keepaspectratio]{/assets/docs/FTzkApPthDlHdofpWCjnfwAAAAAAAAAA.png}

The \texttt{\ +\ } is syntactic sugar (think of it as an abbreviation)
for a call to the \href{/docs/reference/model/enum/}{\texttt{\ enum\ }}
function, to which we apply a set rule above.
\href{/docs/reference/syntax/}{Most syntax is linked to a function in
this way.} If you need to style an element beyond what its arguments
enable, you can completely redefine its appearance with a
\href{/docs/reference/styling/\#show-rules}{show rule} (somewhat
comparable to \texttt{\ \textbackslash{}renewcommand\ } ).

You can achieve the effects of LaTeX commands like
\texttt{\ \textbackslash{}textbf\ } ,
\texttt{\ \textbackslash{}textsf\ } ,
\texttt{\ \textbackslash{}rmfamily\ } ,
\texttt{\ \textbackslash{}mdseries\ } , and
\texttt{\ \textbackslash{}itshape\ } with the
\href{/docs/reference/text/text/\#parameters-font}{\texttt{\ font\ }} ,
\href{/docs/reference/text/text/\#parameters-style}{\texttt{\ style\ }}
, and
\href{/docs/reference/text/text/\#parameters-weight}{\texttt{\ weight\ }}
arguments of the \texttt{\ text\ } function. The text function can be
used in a set rule (declaration style) or with a content argument. To
replace \texttt{\ \textbackslash{}textsc\ } , you can use the
\href{/docs/reference/text/smallcaps/}{\texttt{\ smallcaps\ }} function,
which renders its content argument as smallcaps. Should you want to use
it declaration style (like \texttt{\ \textbackslash{}scshape\ } ), you
can use an \href{/docs/reference/styling/\#show-rules}{\emph{everything}
show rule} that applies the function to the rest of the scope:

\begin{verbatim}
#show: smallcaps

Boisterous Accusations
\end{verbatim}

\includegraphics[width=5in,height=\textheight,keepaspectratio]{/assets/docs/uKURZLdKdC2JssMYxVk2sQAAAAAAAAAA.png}

\subsection{How do I load a document class?}\label{templates}

In LaTeX, you start your main \texttt{\ .tex\ } file with the
\texttt{\ \textbackslash{}documentclass\{article\}\ } command to define
how your document is supposed to look. In that command, you may have
replaced \texttt{\ article\ } with another value such as
\texttt{\ report\ } and \texttt{\ amsart\ } to select a different look.

When using Typst, you style your documents with
\href{/docs/reference/foundations/function/}{functions} . Typically, you
use a template that provides a function that styles your whole document.
First, you import the function from a template file. Then, you apply it
to your whole document. This is accomplished with a
\href{/docs/reference/styling/\#show-rules}{show rule} that wraps the
following document in a given function. The following example
illustrates how it works:

\begin{verbatim}
#import "conf.typ": conf
#show: conf.with(
  title: [
    Towards Improved Modelling
  ],
  authors: (
    (
      name: "Theresa Tungsten",
      affiliation: "Artos Institute",
      email: "tung@artos.edu",
    ),
    (
      name: "Eugene Deklan",
      affiliation: "Honduras State",
      email: "e.deklan@hstate.hn",
    ),
  ),
  abstract: lorem(80),
)

Let's get started writing this
article by putting insightful
paragraphs right here!
\end{verbatim}

\includegraphics[width=12.75in,height=\textheight,keepaspectratio]{/assets/docs/k47Rrbrcbzmkzl-uxGd1cAAAAAAAAAAA.png}

The
\href{/docs/reference/scripting/\#modules}{\texttt{\ }{\texttt{\ import\ }}\texttt{\ }}
statement makes \href{/docs/reference/foundations/function/}{functions}
(and other definitions) from another file available. In this example, it
imports the \texttt{\ conf\ } function from the \texttt{\ conf.typ\ }
file. This function formats a document as a conference article. We use a
show rule to apply it to the document and also configure some metadata
of the article. After applying the show rule, we can start writing our
article right away!

You can also use templates from Typst Universe (which is
Typst\textquotesingle s equivalent of CTAN) using an import statement
like this:
\texttt{\ }{\texttt{\ \#\ }}\texttt{\ }{\texttt{\ import\ }}\texttt{\ }{\texttt{\ "@preview/elsearticle:0.2.1"\ }}\texttt{\ }{\texttt{\ :\ }}\texttt{\ elsearticle\ }
. Check the documentation of an individual template to learn the name of
its template function. Templates and packages from Typst Universe are
automatically downloaded when you first use them.

In the web app, you can choose to create a project from a template on
Typst Universe or even create your own using the template wizard.
Locally, you can use the \texttt{\ typst\ init\ } CLI to create a new
project from a template. Check out
\href{https://typst.app/universe/search/?kind=templates}{the list of
templates} published on Typst Universe. You can also take a look at the
\href{https://github.com/qjcg/awesome-typst}{\texttt{\ awesome-typst\ }
repository} to find community templates that aren\textquotesingle t
available through Universe.

You can also \href{/docs/tutorial/making-a-template/}{create your own,
custom templates.} They are shorter and more readable than the
corresponding LaTeX \texttt{\ .sty\ } files by orders of magnitude, so
give it a try!

Functions are Typst\textquotesingle s "commands" and can transform their
arguments to an output value, including document \emph{content.}
Functions are "pure", which means that they cannot have any effects
beyond creating an output value / output content. This is in stark
contrast to LaTeX macros that can have arbitrary effects on your
document.

To let a function style your whole document, the show rule processes
everything that comes after it and calls the function specified after
the colon with the result as an argument. The \texttt{\ .with\ } part is
a \emph{method} that takes the \texttt{\ conf\ } function and
pre-configures some if its arguments before passing it on to the show
rule.

\subsection{How do I load packages?}\label{packages}

Typst is "batteries included," so the equivalent of many popular LaTeX
packages is built right-in. Below, we compiled a table with frequently
loaded packages and their corresponding Typst functions.

\begin{longtable}[]{@{}ll@{}}
\toprule\noalign{}
LaTeX Package & Typst Alternative \\
\midrule\noalign{}
\endhead
\bottomrule\noalign{}
\endlastfoot
graphicx, svg &
\href{/docs/reference/visualize/image/}{\texttt{\ image\ }} function \\
tabularx & \href{/docs/reference/model/table/}{\texttt{\ table\ }} ,
\href{/docs/reference/layout/grid/}{\texttt{\ grid\ }} functions \\
fontenc, inputenc, unicode-math & Just start writing! \\
babel, polyglossia &
\href{/docs/reference/text/text/\#parameters-lang}{\texttt{\ text\ }}
function:
\texttt{\ }{\texttt{\ \#\ }}\texttt{\ }{\texttt{\ set\ }}\texttt{\ }{\texttt{\ text\ }}\texttt{\ }{\texttt{\ (\ }}\texttt{\ lang\ }{\texttt{\ :\ }}\texttt{\ }{\texttt{\ "zh"\ }}\texttt{\ }{\texttt{\ )\ }}\texttt{\ } \\
amsmath & \href{/docs/reference/math/}{Math mode} \\
amsfonts, amssymb & \href{/docs/reference/symbols/}{\texttt{\ sym\ }}
module and \href{/docs/reference/syntax/\#math}{syntax} \\
geometry, fancyhdr &
\href{/docs/reference/layout/page/}{\texttt{\ page\ }} function \\
xcolor &
\href{/docs/reference/text/text/\#parameters-fill}{\texttt{\ text\ }}
function:
\texttt{\ }{\texttt{\ \#\ }}\texttt{\ }{\texttt{\ set\ }}\texttt{\ }{\texttt{\ text\ }}\texttt{\ }{\texttt{\ (\ }}\texttt{\ fill\ }{\texttt{\ :\ }}\texttt{\ }{\texttt{\ rgb\ }}\texttt{\ }{\texttt{\ (\ }}\texttt{\ }{\texttt{\ "\#0178A4"\ }}\texttt{\ }{\texttt{\ )\ }}\texttt{\ }{\texttt{\ )\ }}\texttt{\ } \\
hyperref & \href{/docs/reference/model/link/}{\texttt{\ link\ }}
function \\
bibtex, biblatex, natbib &
\href{/docs/reference/model/cite/}{\texttt{\ cite\ }} ,
\href{/docs/reference/model/bibliography/}{\texttt{\ bibliography\ }}
functions \\
lstlisting, minted & \href{/docs/reference/text/raw/}{\texttt{\ raw\ }}
function and syntax \\
parskip &
\href{/docs/reference/layout/block/\#parameters-spacing}{\texttt{\ block\ }}
and
\href{/docs/reference/model/par/\#parameters-first-line-indent}{\texttt{\ par\ }}
functions \\
csquotes & Set the
\href{/docs/reference/text/text/\#parameters-lang}{\texttt{\ text\ }}
language and type \texttt{\ "\ } or \texttt{\ \textquotesingle{}\ } \\
caption & \href{/docs/reference/model/figure/}{\texttt{\ figure\ }}
function \\
enumitem & \href{/docs/reference/model/list/}{\texttt{\ list\ }} ,
\href{/docs/reference/model/enum/}{\texttt{\ enum\ }} ,
\href{/docs/reference/model/terms/}{\texttt{\ terms\ }} functions \\
\end{longtable}

Although \emph{many} things are built-in, not everything can be.
That\textquotesingle s why Typst has its own
\href{https://typst.app/universe/}{package ecosystem} where the
community share its creations and automations. Let\textquotesingle s
take, for instance, the \emph{cetz} package: This package allows you to
create complex drawings and plots. To use cetz in your document, you can
just write:

\begin{verbatim}
#import "@preview/cetz:0.2.1"
\end{verbatim}

(The \texttt{\ @preview\ } is a \emph{namespace} that is used while the
package manager is still in its early and experimental state. It will be
replaced in the future.)

Aside from the official package hub, you might also want to check out
the \href{https://github.com/qjcg/awesome-typst}{awesome-typst
repository} , which compiles a curated list of resources created for
Typst.

If you need to load functions and variables from another file within
your project, for example to use a template, you can use the same
\href{/docs/reference/scripting/\#modules}{\texttt{\ import\ }}
statement with a file name rather than a package specification. To
instead include the textual content of another file, you can use an
\href{/docs/reference/scripting/\#modules}{\texttt{\ include\ }}
statement. It will retrieve the content of the specified file and put it
in your document.

\subsection{How do I input maths?}\label{maths}

To enter math mode in Typst, just enclose your equation in dollar signs.
You can enter display mode by adding spaces or newlines between the
equation\textquotesingle s contents and its enclosing dollar signs.

\begin{verbatim}
The sum of the numbers from
$1$ to $n$ is:

$ sum_(k=1)^n k = (n(n+1))/2 $
\end{verbatim}

\includegraphics[width=5in,height=\textheight,keepaspectratio]{/assets/docs/_cQQzHlyveS6_BoZXnzzQQAAAAAAAAAA.png}

\href{/docs/reference/math/}{Math mode} works differently than regular
markup or code mode. Numbers and single characters are displayed
verbatim, while multiple consecutive (non-number) characters will be
interpreted as Typst variables.

Typst pre-defines a lot of useful variables in math mode. All Greek (
\texttt{\ alpha\ } , \texttt{\ beta\ } , ...) and some Hebrew letters (
\texttt{\ alef\ } , \texttt{\ bet\ } , ...) are available through their
name. Some symbols are additionally available through shorthands, such
as \texttt{\ \textless{}=\ } , \texttt{\ \textgreater{}=\ } , and
\texttt{\ -\textgreater{}\ } .

Refer to the \href{/docs/reference/symbols/}{symbol pages} for a full
list of the symbols. If a symbol is missing, you can also access it
through a \href{/docs/reference/syntax/\#escapes}{Unicode escape
sequence} .

Alternate and related forms of symbols can often be selected by
\href{/docs/reference/symbols/symbol/}{appending a modifier} after a
period. For example, \texttt{\ arrow.l.squiggly\ } inserts a squiggly
left-pointing arrow. If you want to insert multiletter text in your
expression instead, enclose it in double quotes:

\begin{verbatim}
$ delta "if" x <= 5 $
\end{verbatim}

\includegraphics[width=5in,height=\textheight,keepaspectratio]{/assets/docs/jk1sG5lCs-ZQ-s2nXNTbTwAAAAAAAAAA.png}

In Typst, delimiters will scale automatically for their expressions,
just as if \texttt{\ \textbackslash{}left\ } and
\texttt{\ \textbackslash{}right\ } commands were implicitly inserted in
LaTeX. You can customize delimiter behaviour using the
\href{/docs/reference/math/lr/\#functions-lr}{\texttt{\ lr\ } function}
. To prevent a pair of delimiters from scaling, you can escape them with
backslashes.

Typst will automatically set terms around a slash \texttt{\ /\ } as a
fraction while honoring operator precedence. All round parentheses not
made redundant by the fraction will appear in the output.

\begin{verbatim}
$ f(x) = (x + 1) / x $
\end{verbatim}

\includegraphics[width=5in,height=\textheight,keepaspectratio]{/assets/docs/hWpFm1Wb3In32sS_SEIHvgAAAAAAAAAA.png}

\href{/docs/reference/math/attach/\#functions-attach}{Sub- and
superscripts} work similarly in Typst and LaTeX.
\texttt{\ }{\texttt{\ \$\ }}\texttt{\ x\ }{\texttt{\ \^{}\ }}\texttt{\ 2\ }{\texttt{\ \$\ }}\texttt{\ }
will produce a superscript,
\texttt{\ }{\texttt{\ \$\ }}\texttt{\ x\ }{\texttt{\ \_\ }}\texttt{\ 2\ }{\texttt{\ \$\ }}\texttt{\ }
yields a subscript. If you want to include more than one value in a sub-
or superscript, enclose their contents in parentheses:
\texttt{\ }{\texttt{\ \$\ }}\texttt{\ x\ }{\texttt{\ \_\ }}\texttt{\ }{\texttt{\ (\ }}\texttt{\ a\ }{\texttt{\ -\textgreater{}\ }}\texttt{\ }{\texttt{\ epsilon\ }}\texttt{\ }{\texttt{\ )\ }}\texttt{\ }{\texttt{\ \$\ }}\texttt{\ }
.

Since variables in math mode do not need to be prepended with a
\texttt{\ \#\ } or a \texttt{\ /\ } , you can also call functions
without these special characters:

\begin{verbatim}
$ f(x, y) := cases(
  1 "if" (x dot y)/2 <= 0,
  2 "if" x "is even",
  3 "if" x in NN,
  4 "else",
) $
\end{verbatim}

\includegraphics[width=5in,height=\textheight,keepaspectratio]{/assets/docs/0X1AFPDieBd9jiawKpc0-AAAAAAAAAAA.png}

The above example uses the
\href{/docs/reference/math/cases/}{\texttt{\ cases\ } function} to
describe f. Within the cases function, arguments are delimited using
commas and the arguments are also interpreted as math. If you need to
interpret arguments as Typst values instead, prefix them with a
\texttt{\ \#\ } :

\begin{verbatim}
$ (a + b)^2
  = a^2
  + text(fill: #maroon, 2 a b)
  + b^2 $
\end{verbatim}

\includegraphics[width=5in,height=\textheight,keepaspectratio]{/assets/docs/Wmx0wcFeGyknnvRFv8bI5QAAAAAAAAAA.png}

You can use all Typst functions within math mode and insert any content.
If you want them to work normally, with code mode in the argument list,
you can prefix their call with a \texttt{\ \#\ } . Nobody can stop you
from using rectangles or emoji as your variables anymore:

\begin{verbatim}
$ sum^10_(🥸=1)
  #rect(width: 4mm, height: 2mm)/🥸
  = 🧠 maltese $
\end{verbatim}

\includegraphics[width=5in,height=\textheight,keepaspectratio]{/assets/docs/aSVPcEv5ICbnBtcXfhpmfgAAAAAAAAAA.png}

If you\textquotesingle d like to enter your mathematical symbols
directly as Unicode, that is possible, too!

Math calls can have two-dimensional argument lists using \texttt{\ ;\ }
as a delimiter. The most common use for this is the
\href{/docs/reference/math/mat/}{\texttt{\ mat\ } function} that creates
matrices:

\begin{verbatim}
$ mat(
  1, 2, ..., 10;
  2, 2, ..., 10;
  dots.v, dots.v, dots.down, dots.v;
  10, 10, ..., 10;
) $
\end{verbatim}

\includegraphics[width=5in,height=\textheight,keepaspectratio]{/assets/docs/yiSilYGQ1wRBpIK3ON349AAAAAAAAAAA.png}

\subsection{How do I get the "LaTeX look?"}\label{latex-look}

Papers set in LaTeX have an unmistakeable look. This is mostly due to
their font, Computer Modern, justification, narrow line spacing, and
wide margins.

The example below

\begin{itemize}
\tightlist
\item
  sets wide
  \href{/docs/reference/layout/page/\#parameters-margin}{margins}
\item
  enables
  \href{/docs/reference/model/par/\#parameters-justify}{justification} ,
  \href{/docs/reference/model/par/\#parameters-leading}{tighter lines}
  and
  \href{/docs/reference/model/par/\#parameters-first-line-indent}{first-line-indent}
\item
  \href{/docs/reference/text/text/\#parameters-font}{sets the font} to
  "New Computer Modern", an OpenType derivative of Computer Modern for
  both text and \href{/docs/reference/text/raw/}{code blocks}
\item
  disables paragraph
  \href{/docs/reference/layout/block/\#parameters-spacing}{spacing}
\item
  increases
  \href{/docs/reference/layout/block/\#parameters-spacing}{spacing}
  around \href{/docs/reference/model/heading/}{headings}
\end{itemize}

\begin{verbatim}
#set page(margin: 1.75in)
#set par(leading: 0.55em, spacing: 0.55em, first-line-indent: 1.8em, justify: true)
#set text(font: "New Computer Modern")
#show raw: set text(font: "New Computer Modern Mono")
#show heading: set block(above: 1.4em, below: 1em)
\end{verbatim}

This should be a good starting point! If you want to go further, why not
create a reusable template?

\subsection{Bibliographies}\label{bibliographies}

Typst includes a fully-featured bibliography system that is compatible
with BibTeX files. You can continue to use your \texttt{\ .bib\ }
literature libraries by loading them with the
\href{/docs/reference/model/bibliography/}{\texttt{\ bibliography\ }}
function. Another possibility is to use
\href{https://github.com/typst/hayagriva/blob/main/docs/file-format.md}{Typst\textquotesingle s
YAML-based native format} .

Typst uses the Citation Style Language to define and process citation
and bibliography styles. You can compare CSL files to
BibLaTeX\textquotesingle s \texttt{\ .bbx\ } files. The compiler already
includes
\href{/docs/reference/model/bibliography/\#parameters-style}{over 80
citation styles} , but you can use any CSL-compliant style from the
\href{https://github.com/citation-style-language/styles}{CSL repository}
or write your own.

You can cite an entry in your bibliography or reference a label in your
document with the same syntax: \texttt{\ }{\texttt{\ @key\ }}\texttt{\ }
(this would reference an entry called \texttt{\ key\ } ). Alternatively,
you can use the \href{/docs/reference/model/cite/}{\texttt{\ cite\ }}
function.

Alternative forms for your citation, such as year only and citations for
natural use in prose (cf. \texttt{\ \textbackslash{}citet\ } and
\texttt{\ \textbackslash{}textcite\ } ) are available with
\href{/docs/reference/model/cite/\#parameters-form}{\texttt{\ }{\texttt{\ \#\ }}\texttt{\ }{\texttt{\ cite\ }}\texttt{\ }{\texttt{\ (\ }}\texttt{\ }{\texttt{\ \textless{}key\textgreater{}\ }}\texttt{\ }{\texttt{\ ,\ }}\texttt{\ form\ }{\texttt{\ :\ }}\texttt{\ }{\texttt{\ "prose"\ }}\texttt{\ }{\texttt{\ )\ }}\texttt{\ }}
.

You can find more information on the documentation page of the
\href{/docs/reference/model/bibliography/}{\texttt{\ bibliography\ }}
function.

\subsection{What limitations does Typst currently have compared to
LaTeX?}\label{limitations}

Although Typst can be a LaTeX replacement for many today, there are
still features that Typst does not (yet) support. Here is a list of them
which, where applicable, contains possible workarounds.

\begin{itemize}
\item
  \textbf{Well-established plotting ecosystem.} LaTeX users often create
  elaborate charts along with their documents in PGF/TikZ. The Typst
  ecosystem does not yet offer the same breadth of available options,
  but the ecosystem around the
  \href{https://github.com/cetz-package/cetz}{\texttt{\ cetz\ }} package
  is catching up quickly.
\item
  \textbf{Change page margins without a pagebreak.} In LaTeX, margins
  can always be adjusted, even without a pagebreak. To change margins in
  Typst, you use the
  \href{/docs/reference/layout/page/}{\texttt{\ page\ } function} which
  will force a page break. If you just want a few paragraphs to stretch
  into the margins, then reverting to the old margins, you can use the
  \href{/docs/reference/layout/pad/}{\texttt{\ pad\ } function} with
  negative padding.
\item
  \textbf{Include PDFs as images.} In LaTeX, it has become customary to
  insert vector graphics as PDF or EPS files. Typst supports neither
  format as an image format, but you can easily convert both into SVG
  files with \href{https://cloudconvert.com/pdf-to-svg}{online tools} or
  \href{https://inkscape.org/}{Inkscape} . The web app will
  automatically convert PDF files to SVG files upon uploading them.
\end{itemize}

\href{/docs/guides/}{\pandocbounded{\includesvg[keepaspectratio]{/assets/icons/16-arrow-right.svg}}}

{ Guides } { Previous page }

\href{/docs/guides/page-setup-guide/}{\pandocbounded{\includesvg[keepaspectratio]{/assets/icons/16-arrow-right.svg}}}

{ Page setup guide } { Next page }
