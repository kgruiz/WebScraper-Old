\title{typst.app/docs/reference/math/class}

\begin{itemize}
\tightlist
\item
  \href{/docs}{\includesvg[width=0.16667in,height=0.16667in]{/assets/icons/16-docs-dark.svg}}
\item
  \includesvg[width=0.16667in,height=0.16667in]{/assets/icons/16-arrow-right.svg}
\item
  \href{/docs/reference/}{Reference}
\item
  \includesvg[width=0.16667in,height=0.16667in]{/assets/icons/16-arrow-right.svg}
\item
  \href{/docs/reference/math/}{Math}
\item
  \includesvg[width=0.16667in,height=0.16667in]{/assets/icons/16-arrow-right.svg}
\item
  \href{/docs/reference/math/class/}{Class}
\end{itemize}

\section{\texorpdfstring{\texttt{\ class\ } {{ Element
}}}{ class   Element }}\label{summary}

\phantomsection\label{element-tooltip}
Element functions can be customized with \texttt{\ set\ } and
\texttt{\ show\ } rules.

Forced use of a certain math class.

This is useful to treat certain symbols as if they were of a different
class, e.g. to make a symbol behave like a relation. The class of a
symbol defines the way it is laid out, including spacing around it, and
how its scripts are attached by default. Note that the latter can always
be overridden using
\href{/docs/reference/math/attach/\#functions-limits}{\texttt{\ limits\ }}
and
\href{/docs/reference/math/attach/\#functions-scripts}{\texttt{\ scripts\ }}
.

\subsection{Example}\label{example}

\begin{verbatim}
#let loves = math.class(
  "relation",
  sym.suit.heart,
)

$x loves y and y loves 5$
\end{verbatim}

\includegraphics[width=5in,height=\textheight,keepaspectratio]{/assets/docs/4-1urHqzMZfIf7fLTw_1MAAAAAAAAAAA.png}

\subsection{\texorpdfstring{{ Parameters
}}{ Parameters }}\label{parameters}

\phantomsection\label{parameters-tooltip}
Parameters are the inputs to a function. They are specified in
parentheses after the function name.

math { . } { class } (

{ \href{/docs/reference/foundations/str/}{str} , } {
\href{/docs/reference/foundations/content/}{content} , }

) -\textgreater{} \href{/docs/reference/foundations/content/}{content}

\subsubsection{\texorpdfstring{\texttt{\ class\ }}{ class }}\label{parameters-class}

\href{/docs/reference/foundations/str/}{str}

{Required} {{ Positional }}

\phantomsection\label{parameters-class-positional-tooltip}
Positional parameters are specified in order, without names.

The class to apply to the content.

\includesvg[width=0.16667in,height=0.16667in]{/assets/icons/16-arrow-right.svg}
View options

\begin{longtable}[]{@{}ll@{}}
\toprule\noalign{}
Variant & Details \\
\midrule\noalign{}
\endhead
\bottomrule\noalign{}
\endlastfoot
\texttt{\ "\ normal\ "\ } & The default class for non-special things. \\
\texttt{\ "\ punctuation\ "\ } & Punctuation, e.g. a comma. \\
\texttt{\ "\ opening\ "\ } & An opening delimiter, e.g. \texttt{\ (\ }
. \\
\texttt{\ "\ closing\ "\ } & A closing delimiter, e.g. \texttt{\ )\ }
. \\
\texttt{\ "\ fence\ "\ } & A delimiter that is the same on both sides,
e.g. \texttt{\ \textbar{}\ } . \\
\texttt{\ "\ large\ "\ } & A large operator like \texttt{\ sum\ } . \\
\texttt{\ "\ relation\ "\ } & A relation like \texttt{\ =\ } or
\texttt{\ prec\ } . \\
\texttt{\ "\ unary\ "\ } & A unary operator like \texttt{\ not\ } . \\
\texttt{\ "\ binary\ "\ } & A binary operator like \texttt{\ times\ }
. \\
\texttt{\ "\ vary\ "\ } & An operator that can be both unary or binary
like \texttt{\ +\ } . \\
\end{longtable}

\subsubsection{\texorpdfstring{\texttt{\ body\ }}{ body }}\label{parameters-body}

\href{/docs/reference/foundations/content/}{content}

{Required} {{ Positional }}

\phantomsection\label{parameters-body-positional-tooltip}
Positional parameters are specified in order, without names.

The content to which the class is applied.

\href{/docs/reference/math/cases/}{\pandocbounded{\includesvg[keepaspectratio]{/assets/icons/16-arrow-right.svg}}}

{ Cases } { Previous page }

\href{/docs/reference/math/equation/}{\pandocbounded{\includesvg[keepaspectratio]{/assets/icons/16-arrow-right.svg}}}

{ Equation } { Next page }

\textbf{On this page}

\begin{itemize}
\tightlist
\item
  \hyperref[summary]{Summary}
\item
  \hyperref[example]{Example}
\item
  \hyperref[parameters]{Parameters}

  \begin{itemize}
  \tightlist
  \item
    \hyperref[parameters-class]{class}
  \item
    \hyperref[parameters-body]{body}
  \end{itemize}
\end{itemize}

\begin{itemize}
\tightlist
\item
  \href{/}{Home}
\item
  \href{/pricing/}{Pricing}
\item
  \href{/docs/}{Documentation}
\item
  \href{/universe/}{Universe}
\item
  \href{/about/}{About Us}
\item
  \href{/contact/}{Contact Us}
\item
  \href{/privacy/}{Privacy}
\item
  \href{https://typst.app/terms}{Terms and Conditions}
\item
  \href{/legal/}{Legal (Impressum)}
\end{itemize}

\begin{itemize}
\tightlist
\item
  \href{https://forum.typst.app}{Forum}
\item
  \href{/tools/}{Tools}
\item
  \href{/blog/}{Blog}
\item
  \href{https://github.com/typst/}{GitHub}
\item
  \href{https://discord.gg/2uDybryKPe}{Discord}
\item
  \href{https://mastodon.social/@typst}{Mastodon}
\item
  \href{https://bsky.app/profile/typst.app}{Bluesky}
\item
  \href{https://www.linkedin.com/company/typst/}{LinkedIn}
\item
  \href{https://instagram.com/typstapp/}{Instagram}
\end{itemize}

Made in Berlin
