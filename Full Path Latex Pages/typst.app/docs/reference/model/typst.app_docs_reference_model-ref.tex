\title{typst.app/docs/reference/model/ref}

\begin{itemize}
\tightlist
\item
  \href{/docs}{\includesvg[width=0.16667in,height=0.16667in]{/assets/icons/16-docs-dark.svg}}
\item
  \includesvg[width=0.16667in,height=0.16667in]{/assets/icons/16-arrow-right.svg}
\item
  \href{/docs/reference/}{Reference}
\item
  \includesvg[width=0.16667in,height=0.16667in]{/assets/icons/16-arrow-right.svg}
\item
  \href{/docs/reference/model/}{Model}
\item
  \includesvg[width=0.16667in,height=0.16667in]{/assets/icons/16-arrow-right.svg}
\item
  \href{/docs/reference/model/ref/}{Reference}
\end{itemize}

\section{\texorpdfstring{\texttt{\ ref\ } {{ Element
}}}{ ref   Element }}\label{summary}

\phantomsection\label{element-tooltip}
Element functions can be customized with \texttt{\ set\ } and
\texttt{\ show\ } rules.

A reference to a label or bibliography.

Produces a textual reference to a label. For example, a reference to a
heading will yield an appropriate string such as "Section 1" for a
reference to the first heading. The references are also links to the
respective element. Reference syntax can also be used to
\href{/docs/reference/model/cite/}{cite} from a bibliography.

Referenceable elements include
\href{/docs/reference/model/heading/}{headings} ,
\href{/docs/reference/model/figure/}{figures} ,
\href{/docs/reference/math/equation/}{equations} , and
\href{/docs/reference/model/footnote/}{footnotes} . To create a custom
referenceable element like a theorem, you can create a figure of a
custom
\href{/docs/reference/model/figure/\#parameters-kind}{\texttt{\ kind\ }}
and write a show rule for it. In the future, there might be a more
direct way to define a custom referenceable element.

If you just want to link to a labelled element and not get an automatic
textual reference, consider using the
\href{/docs/reference/model/link/}{\texttt{\ link\ }} function instead.

\subsection{Example}\label{example}

\begin{verbatim}
#set heading(numbering: "1.")
#set math.equation(numbering: "(1)")

= Introduction <intro>
Recent developments in
typesetting software have
rekindled hope in previously
frustrated researchers. @distress
As shown in @results, we ...

= Results <results>
We discuss our approach in
comparison with others.

== Performance <perf>
@slow demonstrates what slow
software looks like.
$ T(n) = O(2^n) $ <slow>

#bibliography("works.bib")
\end{verbatim}

\includegraphics[width=5in,height=\textheight,keepaspectratio]{/assets/docs/bzf3klNJ674BqVarCEGU8wAAAAAAAAAA.png}

\subsection{Syntax}\label{syntax}

This function also has dedicated syntax: A reference to a label can be
created by typing an \texttt{\ @\ } followed by the name of the label
(e.g.
\texttt{\ }{\texttt{\ =\ Introduction\ }}\texttt{\ }{\texttt{\ \textless{}intro\textgreater{}\ }}\texttt{\ }
can be referenced by typing \texttt{\ }{\texttt{\ @intro\ }}\texttt{\ }
).

To customize the supplement, add content in square brackets after the
reference:
\texttt{\ }{\texttt{\ @intro\ }{\texttt{\ {[}\ }}\texttt{\ Chapter\ }{\texttt{\ {]}\ }}\texttt{\ }}\texttt{\ }
.

\subsection{Customization}\label{customization}

If you write a show rule for references, you can access the referenced
element through the \texttt{\ element\ } field of the reference. The
\texttt{\ element\ } may be \texttt{\ }{\texttt{\ none\ }}\texttt{\ }
even if it exists if Typst hasn\textquotesingle t discovered it yet, so
you always need to handle that case in your code.

\begin{verbatim}
#set heading(numbering: "1.")
#set math.equation(numbering: "(1)")

#show ref: it => {
  let eq = math.equation
  let el = it.element
  if el != none and el.func() == eq {
    // Override equation references.
    link(el.location(),numbering(
      el.numbering,
      ..counter(eq).at(el.location())
    ))
  } else {
    // Other references as usual.
    it
  }
}

= Beginnings <beginning>
In @beginning we prove @pythagoras.
$ a^2 + b^2 = c^2 $ <pythagoras>
\end{verbatim}

\includegraphics[width=5in,height=\textheight,keepaspectratio]{/assets/docs/_2kRnAjhpZZ-kvJsytflygAAAAAAAAAA.png}

\subsection{\texorpdfstring{{ Parameters
}}{ Parameters }}\label{parameters}

\phantomsection\label{parameters-tooltip}
Parameters are the inputs to a function. They are specified in
parentheses after the function name.

{ ref } (

{ \href{/docs/reference/foundations/label/}{label} , } {
\hyperref[parameters-supplement]{supplement :}
\href{/docs/reference/foundations/none/}{none}
\href{/docs/reference/foundations/auto/}{auto}
\href{/docs/reference/foundations/content/}{content}
\href{/docs/reference/foundations/function/}{function} , }

) -\textgreater{} \href{/docs/reference/foundations/content/}{content}

\subsubsection{\texorpdfstring{\texttt{\ target\ }}{ target }}\label{parameters-target}

\href{/docs/reference/foundations/label/}{label}

{Required} {{ Positional }}

\phantomsection\label{parameters-target-positional-tooltip}
Positional parameters are specified in order, without names.

The target label that should be referenced.

Can be a label that is defined in the document or an entry from the
\href{/docs/reference/model/bibliography/}{\texttt{\ bibliography\ }} .

\subsubsection{\texorpdfstring{\texttt{\ supplement\ }}{ supplement }}\label{parameters-supplement}

\href{/docs/reference/foundations/none/}{none} {or}
\href{/docs/reference/foundations/auto/}{auto} {or}
\href{/docs/reference/foundations/content/}{content} {or}
\href{/docs/reference/foundations/function/}{function}

{{ Settable }}

\phantomsection\label{parameters-supplement-settable-tooltip}
Settable parameters can be customized for all following uses of the
function with a \texttt{\ set\ } rule.

A supplement for the reference.

For references to headings or figures, this is added before the
referenced number. For citations, this can be used to add a page number.

If a function is specified, it is passed the referenced element and
should return content.

Default: \texttt{\ }{\texttt{\ auto\ }}\texttt{\ }

\includesvg[width=0.16667in,height=0.16667in]{/assets/icons/16-arrow-right.svg}
View example

\begin{verbatim}
#set heading(numbering: "1.")
#set ref(supplement: it => {
  if it.func() == heading {
    "Chapter"
  } else {
    "Thing"
  }
})

= Introduction <intro>
In @intro, we see how to turn
Sections into Chapters. And
in @intro[Part], it is done
manually.
\end{verbatim}

\includegraphics[width=5in,height=\textheight,keepaspectratio]{/assets/docs/fh477CUxS1KmPvq1dqsQ5QAAAAAAAAAA.png}

\href{/docs/reference/model/quote/}{\pandocbounded{\includesvg[keepaspectratio]{/assets/icons/16-arrow-right.svg}}}

{ Quote } { Previous page }

\href{/docs/reference/model/strong/}{\pandocbounded{\includesvg[keepaspectratio]{/assets/icons/16-arrow-right.svg}}}

{ Strong Emphasis } { Next page }
