\title{typst.app/universe/package/prequery}

\phantomsection\label{banner}
\section{prequery}\label{prequery}

{ 0.1.0 }

library for extracting metadata for preprocessing from a typst document

\phantomsection\label{readme}
This package helps extracting metadata for preprocessing from a typst
document, for example image URLs for download from the web. Typst
compilations are sandboxed: it is not possible for Typst packages, or
even just a Typst document itself, to access the “ouside world�.
This sandboxing of Typst has good reasons. Yet, it is often convenient
to trade a bit of security for convenience by weakening it. Prequery
helps with that by providing some simple scaffolding for supporting
preprocessing of documents.

Here’s an example for referencing images from the internet:

\begin{Shaded}
\begin{Highlighting}[]
\NormalTok{\#import "@preview/prequery:0.1.0"}

\NormalTok{// toggle this comment or pass \textasciigrave{}{-}{-}input prequery{-}fallback=true\textasciigrave{} to enable fallback}
\NormalTok{// \#prequery.fallback.update(true)}

\NormalTok{\#prequery.image(}
\NormalTok{  "https://en.wikipedia.org/static/images/icons/wikipedia.png",}
\NormalTok{  "assets/wikipedia.png")}
\end{Highlighting}
\end{Shaded}

Using \texttt{\ typst\ query\ } , the image URL(s) are extracted from
the document:

\begin{Shaded}
\begin{Highlighting}[]
\ExtensionTok{typst}\NormalTok{ query }\AttributeTok{{-}{-}input}\NormalTok{ prequery{-}fallback=true }\AttributeTok{{-}{-}field}\NormalTok{ value }\DataTypeTok{\textbackslash{}}
\NormalTok{    main.typ }\StringTok{\textquotesingle{}\textquotesingle{}}
\end{Highlighting}
\end{Shaded}

This will output the following piece of JSON:

\begin{Shaded}
\begin{Highlighting}[]
\OtherTok{[}\FunctionTok{\{}\DataTypeTok{"url"}\FunctionTok{:} \StringTok{"https://en.wikipedia.org/static/images/icons/wikipedia.png"}\FunctionTok{,} \DataTypeTok{"path"}\FunctionTok{:} \StringTok{"assets/wikipedia.png"}\FunctionTok{\}}\OtherTok{]}
\end{Highlighting}
\end{Shaded}

Which can then be used to download all images to the expected locations.

See the
\href{https://github.com/typst/packages/raw/main/packages/preview/prequery/0.1.0/docs/manual.pdf}{manual}
for details.

\subsubsection{How to add}\label{how-to-add}

Copy this into your project and use the import as \texttt{\ prequery\ }

\begin{verbatim}
#import "@preview/prequery:0.1.0"
\end{verbatim}

\includesvg[width=0.16667in,height=0.16667in]{/assets/icons/16-copy.svg}

Check the docs for
\href{https://typst.app/docs/reference/scripting/\#packages}{more
information on how to import packages} .

\subsubsection{About}\label{about}

\begin{description}
\tightlist
\item[Author :]
\href{https://github.com/SillyFreak/}{Clemens Koza}
\item[License:]
MIT
\item[Current version:]
0.1.0
\item[Last updated:]
July 15, 2024
\item[First released:]
July 15, 2024
\item[Archive size:]
3.29 kB
\href{https://packages.typst.org/preview/prequery-0.1.0.tar.gz}{\pandocbounded{\includesvg[keepaspectratio]{/assets/icons/16-download.svg}}}
\item[Repository:]
\href{https://github.com/SillyFreak/typst-prequery}{GitHub}
\item[Categor ies :]
\begin{itemize}
\tightlist
\item[]
\item
  \pandocbounded{\includesvg[keepaspectratio]{/assets/icons/16-code.svg}}
  \href{https://typst.app/universe/search/?category=scripting}{Scripting}
\item
  \pandocbounded{\includesvg[keepaspectratio]{/assets/icons/16-hammer.svg}}
  \href{https://typst.app/universe/search/?category=utility}{Utility}
\end{itemize}
\end{description}

\subsubsection{Where to report issues?}\label{where-to-report-issues}

This package is a project of Clemens Koza . Report issues on
\href{https://github.com/SillyFreak/typst-prequery}{their repository} .
You can also try to ask for help with this package on the
\href{https://forum.typst.app}{Forum} .

Please report this package to the Typst team using the
\href{https://typst.app/contact}{contact form} if you believe it is a
safety hazard or infringes upon your rights.

\phantomsection\label{versions}
\subsubsection{Version history}\label{version-history}

\begin{longtable}[]{@{}ll@{}}
\toprule\noalign{}
Version & Release Date \\
\midrule\noalign{}
\endhead
\bottomrule\noalign{}
\endlastfoot
0.1.0 & July 15, 2024 \\
\end{longtable}

Typst GmbH did not create this package and cannot guarantee correct
functionality of this package or compatibility with any version of the
Typst compiler or app.
