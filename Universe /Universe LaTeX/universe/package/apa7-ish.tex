\title{typst.app/universe/package/apa7-ish}

\phantomsection\label{banner}
\phantomsection\label{template-thumbnail}
\pandocbounded{\includegraphics[keepaspectratio]{https://packages.typst.org/preview/thumbnails/apa7-ish-0.2.0-small.webp}}

\section{apa7-ish}\label{apa7-ish}

{ 0.2.0 }

Typst Template that (mostly) complies with APA7 Style (Work in
Progress).

\href{/app?template=apa7-ish&version=0.2.0}{Create project in app}

\phantomsection\label{readme}
\href{https://typst.app/}{Typst} Template that (mostly) complies with
APA7 Style (Work in Progress).

The template does not follow all recommendations by the APA Manual,
especially when the suggestions break with typographic conventions (such
as double line spacing :vomiting\_face:). Instead, the goal of this
template is that it generates you a high-quality manuscript that has all
the important components of the APA7 format, but is aesthetically
pleasing.

The following works already quite well:

\begin{itemize}
\tightlist
\item
  consistent and simple typesetting
\item
  correct display of author information / author note
\item
  citations anfalsed references
\item
  Page headers and footers (show short title in header)
\item
  Option to anonymize the paper
\item
  Tables: consisting of 3 parts (caption, content, and table notes)
\end{itemize}

This is still not finished:

\begin{itemize}
\tightlist
\item
  figures
\item
  complete pandoc integration (template for pandoc to replace
  Latex-based workflows)
\item
  automatic calculation of page margins (like memoir-class for Latex)
\end{itemize}

The easiest way to get started is to edit the example file, which has
sensible default values. Most fields in the configuration are optional
and will safely be ignored (not rendered) when you set them to
\texttt{\ none\ } .

\subsection{Authors}\label{authors}

The \texttt{\ authors\ } setting expects an array of dictionaries with
the following fields:

\begin{Shaded}
\begin{Highlighting}[]
\NormalTok{(}
\NormalTok{  name: "First Name Last Name", // Name of author as it should appear on the paper title page}
\NormalTok{  affiliation: "University, Department", // affiliation(s) of author as it should appear on the title page}
\NormalTok{  orcid: "0000{-}0000{-}0000{-}0000", // optional for author note}
\NormalTok{  corresponding: true, // optional to mark an author as corresponding author}
\NormalTok{  email: "email@upenn.edu", // optional email address, required if author is corresponding}
\NormalTok{  postal: "Longer string", // optional postal address for corresponding author}
\NormalTok{)}
\end{Highlighting}
\end{Shaded}

Note that the \texttt{\ affiliation\ } field also accepts an array, in
case an author has several affiliations:

\begin{Shaded}
\begin{Highlighting}[]
\NormalTok{(}
\NormalTok{  name: "First Name Last Name",}
\NormalTok{  affiliation: ("University A", "University B")}
\NormalTok{  ...}
\NormalTok{)}
\end{Highlighting}
\end{Shaded}

\subsection{Anonymization}\label{anonymization}

Sometimes you need to submit a paper without any author information. In
such cases you can set \texttt{\ anonymous\ } to \texttt{\ true\ } .

\subsection{Tables}\label{tables}

The template has basic support for tables with a handful of utilities.
Analogous to the \href{https://ctan.org/pkg/booktabs}{Latex booktabs
package} , there are pre-defined horizontal lines (“rules�):

\begin{itemize}
\tightlist
\item
  \texttt{\ \#toprule\ } : used at the top of the table, before the
  first row
\item
  \texttt{\ \#midrule\ } : used to separate the header row, or to
  separate a totals row at the bottom
\item
  \texttt{\ \#bottomrule\ } : used after the last row (technically the
  same as toprule, but may be useful later to define custom behaviour)
\end{itemize}

Addtionally, there is a \texttt{\ \#tablenote\ } function to be used to
place a table note below the table.

A minimal usage example (taken from the typst documentation):

\begin{Shaded}
\begin{Highlighting}[]
\NormalTok{// wrap everything in a \#figure}
\NormalTok{\#figure(}
\NormalTok{  [}
\NormalTok{    \#table(}
\NormalTok{      columns: 2,}
\NormalTok{      align: left,}
\NormalTok{      toprule, // separate table from other content}
\NormalTok{      table.header([Amount], [Ingredient]),}
\NormalTok{      midrule, // separation between table header and body}
\NormalTok{      [360g], [Baking flour],}
\NormalTok{      [250g], [Butter (room temp.)],}
\NormalTok{      [150g], [Brown sugar],}
\NormalTok{      [100g], [Cane sugar],}
\NormalTok{      [100g], [70\% cocoa chocolate],}
\NormalTok{      [100g], [35{-}40\% cocoa chocolate],}
\NormalTok{      [2], [Eggs],}
\NormalTok{      [Pinch], [Salt],}
\NormalTok{      [Drizzle], [Vanilla extract],}
\NormalTok{      bottomrule // separation after the last row}
\NormalTok{    )}
\NormalTok{    // tablenote goes after the \#table}
\NormalTok{    \#tablenote([Here are some additional notes.])}
\NormalTok{  ],}
\NormalTok{  // caption is part of the \#figure}
\NormalTok{  caption: [Here is the table caption]}
\NormalTok{)}
\end{Highlighting}
\end{Shaded}

\href{/app?template=apa7-ish&version=0.2.0}{Create project in app}

\subsubsection{How to use}\label{how-to-use}

Click the button above to create a new project using this template in
the Typst app.

You can also use the Typst CLI to start a new project on your computer
using this command:

\begin{verbatim}
typst init @preview/apa7-ish:0.2.0
\end{verbatim}

\includesvg[width=0.16667in,height=0.16667in]{/assets/icons/16-copy.svg}

\subsubsection{About}\label{about}

\begin{description}
\tightlist
\item[Author :]
MrWunderbar
\item[License:]
MIT
\item[Current version:]
0.2.0
\item[Last updated:]
October 30, 2024
\item[First released:]
October 21, 2024
\item[Minimum Typst version:]
0.12.0
\item[Archive size:]
8.21 kB
\href{https://packages.typst.org/preview/apa7-ish-0.2.0.tar.gz}{\pandocbounded{\includesvg[keepaspectratio]{/assets/icons/16-download.svg}}}
\item[Repository:]
\href{https://github.com/mrwunderbar666/typst-apa7ish}{GitHub}
\item[Categor y :]
\begin{itemize}
\tightlist
\item[]
\item
  \pandocbounded{\includesvg[keepaspectratio]{/assets/icons/16-atom.svg}}
  \href{https://typst.app/universe/search/?category=paper}{Paper}
\end{itemize}
\end{description}

\subsubsection{Where to report issues?}\label{where-to-report-issues}

This template is a project of MrWunderbar . Report issues on
\href{https://github.com/mrwunderbar666/typst-apa7ish}{their repository}
. You can also try to ask for help with this template on the
\href{https://forum.typst.app}{Forum} .

Please report this template to the Typst team using the
\href{https://typst.app/contact}{contact form} if you believe it is a
safety hazard or infringes upon your rights.

\phantomsection\label{versions}
\subsubsection{Version history}\label{version-history}

\begin{longtable}[]{@{}ll@{}}
\toprule\noalign{}
Version & Release Date \\
\midrule\noalign{}
\endhead
\bottomrule\noalign{}
\endlastfoot
0.2.0 & October 30, 2024 \\
\href{https://typst.app/universe/package/apa7-ish/0.1.0/}{0.1.0} &
October 21, 2024 \\
\end{longtable}

Typst GmbH did not create this template and cannot guarantee correct
functionality of this template or compatibility with any version of the
Typst compiler or app.
