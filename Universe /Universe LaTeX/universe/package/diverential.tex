\title{typst.app/universe/package/diverential}

\phantomsection\label{banner}
\section{diverential}\label{diverential}

{ 0.2.0 }

Format differentials conveniently.

\phantomsection\label{readme}
\texttt{\ diverential\ } is a
\href{https://github.com/typst/typst}{Typst} package simplifying the
typesetting of differentials. It is the equivalent to LaTeX’s
\texttt{\ diffcoeff\ } , though not as mature.

\subsection{Overview}\label{overview}

\texttt{\ diverential\ } allows normal, partial, compact, and separated
derivatives with smart degree calculations.

\begin{Shaded}
\begin{Highlighting}[]
\NormalTok{\#}\ImportTok{import} \StringTok{"@preview/diverential:0.2.0"}\OperatorTok{:} \OperatorTok{*}

\NormalTok{$ }\FunctionTok{dv}\NormalTok{(f}\OperatorTok{,}\NormalTok{ x}\OperatorTok{,}\NormalTok{ deg}\OperatorTok{:} \DecValTok{2}\OperatorTok{,}\NormalTok{ eval}\OperatorTok{:} \DecValTok{0}\NormalTok{) $}
\NormalTok{$ }\FunctionTok{dvp}\NormalTok{(f}\OperatorTok{,}\NormalTok{ x}\OperatorTok{,}\NormalTok{ y}\OperatorTok{,}\NormalTok{ eval}\OperatorTok{:} \DecValTok{0}\OperatorTok{,}\NormalTok{ evalsym}\OperatorTok{:} \StringTok{"["}\NormalTok{) $}
\NormalTok{$ }\FunctionTok{dvpc}\NormalTok{(f}\OperatorTok{,}\NormalTok{ x) $}
\NormalTok{$ }\FunctionTok{dvps}\NormalTok{(f}\OperatorTok{,}\NormalTok{ \#([x]}\OperatorTok{,} \DecValTok{2}\NormalTok{)}\OperatorTok{,}\NormalTok{ \#([y]}\OperatorTok{,}\NormalTok{ [n])}\OperatorTok{,}\NormalTok{ \#([z]}\OperatorTok{,}\NormalTok{ [m])}\OperatorTok{,}\NormalTok{ eval}\OperatorTok{:} \DecValTok{0}\NormalTok{) $}
\end{Highlighting}
\end{Shaded}

\includegraphics[width=1.5625in,height=\textheight,keepaspectratio]{https://github.com/typst/packages/raw/main/packages/preview/diverential/0.2.0/examples/overview.jpg}

\subsection{\texorpdfstring{\texttt{\ dv\ }}{ dv }}\label{dv}

\texttt{\ dv\ } is an ordinary derivative. It takes the function as its
first argument and the variable as its second one. A degree can be
specified with \texttt{\ deg\ } . The derivate can be specified to be
evaluated at a point with \texttt{\ eval\ } , the brackets of which can
be changed with \texttt{\ evalsym\ } . \texttt{\ space\ } influences the
space between derivative and evaluation bracket. Unless defined
otherwise, no space is set by default, except for
\texttt{\ \textbar{}\ } , where a small gap is introduced.

\subsubsection{\texorpdfstring{\texttt{\ dvs\ }}{ dvs }}\label{dvs}

Same as \texttt{\ dv\ } , but separates the function from the
derivative.\\
Example: \$\$
\textbackslash frac\{\textbackslash mathrm\{d\}\}\{\textbackslash mathrm\{d\}x\}
f \$\$

\subsubsection{\texorpdfstring{\texttt{\ dvc\ }}{ dvc }}\label{dvc}

Same as \texttt{\ dv\ } , but uses a compact derivative.\\
Example: \$\$ \textbackslash mathrm\{d\}\_x f \$\$

\subsection{\texorpdfstring{\texttt{\ dvp\ }}{ dvp }}\label{dvp}

\texttt{\ dv\ } is a partial derivative. It takes the function as its
first argument and the variable as the rest. For information on
\texttt{\ eval\ } , \texttt{\ evalsym\ } , and \texttt{\ space\ } , read
the description of \texttt{\ dv\ } .\\
The variable can be one of these options:

\begin{itemize}
\tightlist
\item
  plain variable, e.g. \texttt{\ x\ }
\item
  list of variables, e.g. \texttt{\ x,\ y\ }
\item
  list of variables with higher degrees (type
  \texttt{\ (content,\ integer)\ } ), e.g.
  \texttt{\ x,\ \#({[}y{]},\ 2)\ } The degree is smartly calculated: If
  all degrees of the variables are integers, the total degree is their
  sum. If some are content, the integer ones are summed (arithmetically)
  up and added to the visual sum of the content degrees. Example:
  \texttt{\ \#({[}x{]},\ n),\ \#({[}y{]},\ 2),\ z\ } â†'
  \$\textbackslash frac\{\textbackslash partial\^{}\{n+3\}\}\{\textbackslash partial
  x\^{}n,\textbackslash partial y\^{}2,\textbackslash partial z\}\$\\
  Specifying \texttt{\ deg\ } manually is always possible and might be
  required in more complicated cases.
\end{itemize}

\subsubsection{\texorpdfstring{\texttt{\ dvps\ }}{ dvps }}\label{dvps}

Same as \texttt{\ dvp\ } , but separates the function from the
derivative.\\
Example: \$\$
\textbackslash frac\{\textbackslash partial\}\{\textbackslash partial
x\} f \$\$

\subsubsection{\texorpdfstring{\texttt{\ dvpc\ }}{ dvpc }}\label{dvpc}

Same as \texttt{\ dvp\ } , but uses a compact derivative.\\
Note: If supplying multiple variables, \texttt{\ deg\ } is ignored.\\
Example: \$\$ \textbackslash partial\_x f \$\$

\subsubsection{How to add}\label{how-to-add}

Copy this into your project and use the import as
\texttt{\ diverential\ }

\begin{verbatim}
#import "@preview/diverential:0.2.0"
\end{verbatim}

\includesvg[width=0.16667in,height=0.16667in]{/assets/icons/16-copy.svg}

Check the docs for
\href{https://typst.app/docs/reference/scripting/\#packages}{more
information on how to import packages} .

\subsubsection{About}\label{about}

\begin{description}
\tightlist
\item[Author :]
Christopher Hecker
\item[License:]
MIT
\item[Current version:]
0.2.0
\item[Last updated:]
July 29, 2023
\item[First released:]
July 29, 2023
\item[Archive size:]
3.38 kB
\href{https://packages.typst.org/preview/diverential-0.2.0.tar.gz}{\pandocbounded{\includesvg[keepaspectratio]{/assets/icons/16-download.svg}}}
\end{description}

\subsubsection{Where to report issues?}\label{where-to-report-issues}

This package is a project of Christopher Hecker . You can also try to
ask for help with this package on the
\href{https://forum.typst.app}{Forum} .

Please report this package to the Typst team using the
\href{https://typst.app/contact}{contact form} if you believe it is a
safety hazard or infringes upon your rights.

\phantomsection\label{versions}
\subsubsection{Version history}\label{version-history}

\begin{longtable}[]{@{}ll@{}}
\toprule\noalign{}
Version & Release Date \\
\midrule\noalign{}
\endhead
\bottomrule\noalign{}
\endlastfoot
0.2.0 & July 29, 2023 \\
\end{longtable}

Typst GmbH did not create this package and cannot guarantee correct
functionality of this package or compatibility with any version of the
Typst compiler or app.
