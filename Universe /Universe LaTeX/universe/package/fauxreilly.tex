\title{typst.app/universe/package/fauxreilly}

\phantomsection\label{banner}
\section{fauxreilly}\label{fauxreilly}

{ 0.1.0 }

A package for creating O\textquotesingle Rly- /
O\textquotesingle Reilly-type cover pages

\phantomsection\label{readme}
\href{https://forthebadge.com/}{\pandocbounded{\includesvg[keepaspectratio]{https://raw.githubusercontent.com/dei-layborer/o-rly-typst/refs/heads/main/made-with-(2s)-2\%2C6-diamino-n-\%5B(2s)-1-phenylpropan-2-yl\%5Dhexanamide-n-\%5B(2s)-1-phenyl-2-propanyl\%5D-l-lysinamide.svg}}}

\href{https://deilayborer.neocities.org/funding}{\includesvg[width=\linewidth,height=0.3125in,keepaspectratio]{https://raw.githubusercontent.com/dei-layborer/o-rly-typst/refs/heads/main/\%24\%24\%24-gimmie.svg}}

A \texttt{\ typst\ } package for creating \textbf{O’RLY?} -style cover
pages.

\subsection{Example}\label{example}

\begin{Shaded}
\begin{Highlighting}[]
\NormalTok{\#import }\StringTok{"@preview/o{-}rly{-}cover:0.1.0"}\OperatorTok{:} \OperatorTok{*}

\NormalTok{\#orly(}
\NormalTok{    color}\OperatorTok{:}\NormalTok{ rgb(}\StringTok{"\#85144b"}\NormalTok{)}\OperatorTok{,}
\NormalTok{    title}\OperatorTok{:} \StringTok{"Learn to Stop Worrying and Love Metathesis"}\OperatorTok{,}
\NormalTok{    top}\OperatorTok{{-}}\NormalTok{text}\OperatorTok{:} \StringTok{"Axe nat why (or do)"}\OperatorTok{,}
\NormalTok{    subtitle}\OperatorTok{:} \StringTok{"Free yourself from prescriptivism"}\OperatorTok{,}
\NormalTok{    pic}\OperatorTok{:} \StringTok{"chomskydoz.png"}\OperatorTok{,}
\NormalTok{    signature}\OperatorTok{:} \StringTok{"Dr. N. Supponent"}
\NormalTok{)}
\end{Highlighting}
\end{Shaded}

\pandocbounded{\includegraphics[keepaspectratio]{https://github.com/typst/packages/raw/main/packages/preview/fauxreilly/0.1.0/example.png}}

\subsection{Usage}\label{usage}

First, import the package at the top of your \texttt{\ typst\ } file:
\texttt{\ \#import\ "@preview/o-rly-cover:0.1.0":\ *\ }

Only one function is exposed, \texttt{\ \#orly()\ } . This will create
its own page in the document at whatever location you call the function.
In other words, any content in the \texttt{\ typst\ } document that
appears before \texttt{\ \#orly()\ } is called will be before the
O’Rly? page in the PDF that \texttt{\ typst\ } renders. Anything after
the function call will be on subsequent page(s).

All content for the title page is passed as options to
\texttt{\ \#orly()\ } . I included what I figured were the most likely
things you’d want to customize without having a million options.
Meanwhile, most of the layout parameters (font sizes, the heights of
individual pieces, etc.) are variables within the code, so hopefully
aren’t too hard to alter if need-be. None of the options are strictly
required, although the text fields are the only ones that can be left
empty without potentially breaking the layout. A few have defaults
instead, and those are listed below where applicable.

\subsubsection{Options}\label{options}

The order that the options appear in the table is the order they must be
sent to the function, unless you specify the option’s key along with
its value.

Data types listed are based on \texttt{\ typst\ } ’s internal types,
so are entered the same way as they would be in any other function that
takes that data type. For example, the data type needed for the
\texttt{\ font\ } option is the same as what is used for
\texttt{\ typst\ } ’s built-in \texttt{\ \#text()\ } function, which
is linked in the table below. (All links go to their specific usage in
the \texttt{\ typst\ } documentation.)

\begin{longtable}[]{@{}llcc@{}}
\toprule\noalign{}
Option & Description & Type & Default \\
\midrule\noalign{}
\endhead
\bottomrule\noalign{}
\endlastfoot
\texttt{\ font\ } & The font for all text except the “publisher� in
the bottom-left corner &
\href{https://typst.app/docs/reference/text/text/\#parameters-font}{\texttt{\ string(s)\ }}
& Whatever is set in the document context \\
\texttt{\ color\ } & Accent color. Used for the background of the title
block and of the colored bar at the very top. &
\href{https://typst.app/docs/reference/visualize/color/}{\texttt{\ color\ }}
& \texttt{\ blue\ } (typst built-in) \\
\texttt{\ top-text\ } & The text at the top, just under the color bar &
\href{https://typst.app/docs/reference/foundations/str/}{\texttt{\ string\ }}
& Empty \\
\texttt{\ pic\ } & Image to be used above the title block &
\href{https://typst.app/docs/reference/visualize/image/\#parameters-path}{\texttt{\ string\ }}
with path to the image & Empty \\
\texttt{\ title\ } & The title of the book &
\href{https://typst.app/docs/reference/foundations/str/}{\texttt{\ string\ }}
& Empty \\
\texttt{\ title-align\ } & How the text is aligned (horizontally) in the
title block &
\href{https://typst.app/docs/reference/layout/alignment/}{\texttt{\ alignment\ }}
& \texttt{\ left\ } \\
\texttt{\ subtitle\ } & Text that appears just below the title block &
\href{https://typst.app/docs/reference/foundations/str/}{\texttt{\ string\ }}
& Empty \\
\texttt{\ publisher\ } & The name of the “publisher� in the
bottom-left &
\href{https://typst.app/docs/reference/foundations/str/}{\texttt{\ string\ }}
& O RLY \textsuperscript{?} (see example above) \\
\texttt{\ publisher-font\ } & Font to be used for “publisher� name &
\href{https://typst.app/docs/reference/text/text/\#parameters-font}{\texttt{\ string(s)\ }}
& Noto Sans, Arial Rounded MT, document context (in that order) \\
\texttt{\ signature\ } & Text in the bottom-right corner &
\href{https://typst.app/docs/reference/foundations/str/}{\texttt{\ string\ }}
& Empty \\
\texttt{\ margin\ } & Page margins &
\href{https://typst.app/docs/reference/layout/page/\#parameters-margin}{\texttt{\ length\ }
or \texttt{\ dictionary\ }} & \texttt{\ top:\ 0\ } , all others will use
the document context \\
\end{longtable}

\subsubsection{Usage Notes}\label{usage-notes}

There are a couple quirks to data types and the like that may not be
obvious.

\begin{enumerate}
\tightlist
\item
  \texttt{\ string\ } s typically must be contained in quotation marks.
  But note that this will render quotation marks \emph{within} those
  strings without using
  \href{https://typst.app/docs/reference/text/smartquote/}{smartquotes}
  . To avoid this, you may use content mode instead (by enclosing the
  text in square brackets \texttt{\ {[}{]}\ } ). For example,
  \texttt{\ "Some\ title"\ } â†' \texttt{\ {[}Some\ title{]}\ }

  \begin{itemize}
  \tightlist
  \item
    Similarly, you can use this to toggle italics (e.g.
    \texttt{\ {[}Italic\ text,\ \_except\_\ this\ one{]}\ } ) or apply
    other formatting
  \end{itemize}
\item
  Other types may look like strings but do \textbf{not} take quotes,
  specifically \texttt{\ color\ } (including when using the built-in
  color names) and \texttt{\ alignment\ }
\item
  With the \texttt{\ margin\ } type, if a single value is entered
  (without quotes), that value is applied to all four sides. All other
  usage must be done as a dictionary (meaning enclosed in parentheses),
  even if you’re only specifying one side. For example, the default is
  written \texttt{\ (top:\ 0in)\ } .

  \begin{itemize}
  \tightlist
  \item
    If you’re going to pass any value other than the top as an option,
    you’ll likely want to add \texttt{\ top:\ 0in\ } back in to avoid
    a gap between the top of the page and the color bar
  \item
    Any values passed to the function (or the default value if none are)
    will override any margin(s) set earlier in the \texttt{\ typst\ }
    file. So you can use a \texttt{\ set\ } rule at the beginning of the
    document without affecting the cover page
  \end{itemize}
\end{enumerate}

\subsubsection{Images}\label{images}

The package uses \texttt{\ typst\ } ’s built-in image handling, which
means it only supports PNG, JPG, and SVG.

The image will be resized to as close to 100\% page width (inside the
margins) as possible while both maintaining proportions and avoiding any
cropping. The rest of the layout \emph{should} flow reasonably well
around any image height, but outliers may exist.

O’Reilly-style animals can be found in the
\href{https://etc.usf.edu/clipart/galleries/730-animals}{relevant
section} of the Florida Center for Instructional Technology’s
\href{https://etc.usf.edu/clipart/}{ClipArt ETC} project. Just be aware
that these are provided as GIFs(!), so conversion to one of
\texttt{\ typst\ } ’s supported formats will be required.

\subsection{Bugs \& Feature Requests}\label{bugs-feature-requests}

I put this whole thing together in an afternoon when I should’ve been
doing work for my day job. Granted I’d already done a basic version
for a seminary writing assignment (I love to spoof academic writing),
but either way, I’ve gotten this project to a basic level of
functionality and no further. I’m entirely open to suggestions for
additional functionality, however, so feel free to
\href{https://github.com/dei-layborer/o-rly-typst/issues}{create an
issue} if there’s something you’d like to see added.

It hopefully goes without saying that the same is true if something
breaks!

Tested on \texttt{\ typst\ } versions \texttt{\ 0.11.1\ } and
\texttt{\ 0.12.0-rc1\ } .

\subsection{Thanks \& Shout-Outs}\label{thanks-shout-outs}

Shout out to Arthur Beaulieu (
\href{https://github.com/ArthurBeaulieu}{@arthurbeaulieu} ), whose
\href{https://arthurbeaulieu.github.io/ORlyGenerator/}{web-based
generator} served as both inspiration and reference (I pretty much stole
his layout settings).

Significant thanks to the folks in the
\href{https://discord.gg/2uDybryKPe}{typst discord} who helped me sort
out some layout woes.

Extra double appreciation to Enivex on the discord for the name.

\subsubsection{How to add}\label{how-to-add}

Copy this into your project and use the import as
\texttt{\ fauxreilly\ }

\begin{verbatim}
#import "@preview/fauxreilly:0.1.0"
\end{verbatim}

\includesvg[width=0.16667in,height=0.16667in]{/assets/icons/16-copy.svg}

Check the docs for
\href{https://typst.app/docs/reference/scripting/\#packages}{more
information on how to import packages} .

\subsubsection{About}\label{about}

\begin{description}
\tightlist
\item[Author :]
Dei Layborer
\item[License:]
GPL-3.0
\item[Current version:]
0.1.0
\item[Last updated:]
October 16, 2024
\item[First released:]
October 16, 2024
\item[Minimum Typst version:]
0.11.1
\item[Archive size:]
4.48 kB
\href{https://packages.typst.org/preview/fauxreilly-0.1.0.tar.gz}{\pandocbounded{\includesvg[keepaspectratio]{/assets/icons/16-download.svg}}}
\item[Repository:]
\href{https://github.com/dei-layborer/o-rly-typst}{GitHub}
\item[Categor ies :]
\begin{itemize}
\tightlist
\item[]
\item
  \pandocbounded{\includesvg[keepaspectratio]{/assets/icons/16-package.svg}}
  \href{https://typst.app/universe/search/?category=components}{Components}
\item
  \pandocbounded{\includesvg[keepaspectratio]{/assets/icons/16-layout.svg}}
  \href{https://typst.app/universe/search/?category=layout}{Layout}
\item
  \pandocbounded{\includesvg[keepaspectratio]{/assets/icons/16-smile.svg}}
  \href{https://typst.app/universe/search/?category=fun}{Fun}
\end{itemize}
\end{description}

\subsubsection{Where to report issues?}\label{where-to-report-issues}

This package is a project of Dei Layborer . Report issues on
\href{https://github.com/dei-layborer/o-rly-typst}{their repository} .
You can also try to ask for help with this package on the
\href{https://forum.typst.app}{Forum} .

Please report this package to the Typst team using the
\href{https://typst.app/contact}{contact form} if you believe it is a
safety hazard or infringes upon your rights.

\phantomsection\label{versions}
\subsubsection{Version history}\label{version-history}

\begin{longtable}[]{@{}ll@{}}
\toprule\noalign{}
Version & Release Date \\
\midrule\noalign{}
\endhead
\bottomrule\noalign{}
\endlastfoot
0.1.0 & October 16, 2024 \\
\end{longtable}

Typst GmbH did not create this package and cannot guarantee correct
functionality of this package or compatibility with any version of the
Typst compiler or app.
