\title{typst.app/universe/package/pyrunner}

\phantomsection\label{banner}
\section{pyrunner}\label{pyrunner}

{ 0.2.0 }

Run python code in typst.

\phantomsection\label{readme}
Run python code in \href{https://typst.app/}{typst} using
\href{https://github.com/RustPython/RustPython}{RustPython} .

\begin{Shaded}
\begin{Highlighting}[]
\NormalTok{\#import "@preview/pyrunner:0.1.0" as py}

\NormalTok{\#let compiled = py.compile(}
\NormalTok{\textasciigrave{}\textasciigrave{}\textasciigrave{}python}
\NormalTok{def find\_emails(string):}
\NormalTok{    import re}
\NormalTok{    return re.findall(r"\textbackslash{}b[a{-}zA{-}Z0{-}9.\_\%+{-}]+@[a{-}zA{-}Z0{-}9.{-}]+\textbackslash{}.[a{-}zA{-}Z]\{2,\}\textbackslash{}b", string)}

\NormalTok{def sum\_all(*array):}
\NormalTok{    return sum(array)}
\NormalTok{\textasciigrave{}\textasciigrave{}\textasciigrave{})}

\NormalTok{\#let txt = "My email address is john.doe@example.com and my friend\textquotesingle{}s email address is jane.doe@example.net."}

\NormalTok{\#py.call(compiled, "find\_emails", txt)}
\NormalTok{\#py.call(compiled, "sum\_all", 1, 2, 3)}
\end{Highlighting}
\end{Shaded}

Block mode is also available.

\begin{Shaded}
\begin{Highlighting}[]
\NormalTok{\#let code = \textasciigrave{}\textasciigrave{}\textasciigrave{}}
\NormalTok{f\textquotesingle{}\{a+b=\}\textquotesingle{}}
\NormalTok{\textasciigrave{}\textasciigrave{}\textasciigrave{}}

\NormalTok{\#py.block(code, globals: (a: 1, b: 2))}

\NormalTok{\#py.block(code, globals: (a: "1", b: "2"))}
\end{Highlighting}
\end{Shaded}

The result will be \texttt{\ a+b=3\ } and
\texttt{\ a+b=\textquotesingle{}12\textquotesingle{}\ } .

\subsection{Current limitations}\label{current-limitations}

Due to restrictions of typst and its plugin system, some Python function
will not work as expected:

\begin{itemize}
\tightlist
\item
  File and network IO will always raise an exception.
\item
  \texttt{\ datatime.now\ } will always return 1970-01-01.
\end{itemize}

Also, there is no way to import third-party modules. Only bundled stdlib
modules are available. We might find a way to lift this restriction, so
feel free to submit an issue if you want this functionality.

\subsection{API}\label{api}

\subsubsection{\texorpdfstring{\texttt{\ block\ }}{ block }}\label{block}

Run Python code block and get its result.

\paragraph{Arguments}\label{arguments}

\begin{itemize}
\tightlist
\item
  \texttt{\ code\ } : string \textbar{} raw content - The Python code to
  run.
\item
  \texttt{\ globals\ } : dict (named optional) - The global variables to
  bring into scope.
\end{itemize}

\paragraph{Returns}\label{returns}

The last expression of the code block.

\subsubsection{\texorpdfstring{\texttt{\ compile\ }}{ compile }}\label{compile}

Compile Python code to bytecode.

\paragraph{Arguments}\label{arguments-1}

\begin{itemize}
\tightlist
\item
  \texttt{\ code\ } : string \textbar{} raw content - The Python code to
  compile.
\end{itemize}

\paragraph{Returns}\label{returns-1}

The bytecode representation in \texttt{\ bytes\ } .

\subsubsection{\texorpdfstring{\texttt{\ call\ }}{ call }}\label{call}

Call a python function with arguments.

\paragraph{Arguments}\label{arguments-2}

\begin{itemize}
\tightlist
\item
  \texttt{\ compiled\ } : bytes - The bytecode representation of Python
  code.
\item
  \texttt{\ fn\_name\ } : string - The name of the function to be
  called.
\item
  \texttt{\ ..args\ } : any - The arguments to pass to the function.
\end{itemize}

\paragraph{Returns}\label{returns-2}

The result of the function call.

\subsection{Build from source}\label{build-from-source}

Install
\href{https://github.com/astrale-sharp/wasm-minimal-protocol}{\texttt{\ wasi-stub\ }}
. You should use a slightly modified one. See
\href{https://github.com/astrale-sharp/wasm-minimal-protocol/issues/22\#issuecomment-1827379467}{the
related issue} .

Build pyrunner.

\begin{verbatim}
rustup target add wasm32-wasi
cargo build --target wasm32-wasi
make pkg/typst-pyrunner.wasm
\end{verbatim}

Add to local package.

\begin{verbatim}
mkdir -p ~/.local/share/typst/packages/local/pyrunner/0.0.1
cp pkg/* ~/.local/share/typst/packages/local/pyrunner/0.0.1
\end{verbatim}

\subsubsection{How to add}\label{how-to-add}

Copy this into your project and use the import as \texttt{\ pyrunner\ }

\begin{verbatim}
#import "@preview/pyrunner:0.2.0"
\end{verbatim}

\includesvg[width=0.16667in,height=0.16667in]{/assets/icons/16-copy.svg}

Check the docs for
\href{https://typst.app/docs/reference/scripting/\#packages}{more
information on how to import packages} .

\subsubsection{About}\label{about}

\begin{description}
\tightlist
\item[Author :]
Peng Guanwen
\item[License:]
MIT
\item[Current version:]
0.2.0
\item[Last updated:]
March 18, 2024
\item[First released:]
December 4, 2023
\item[Minimum Typst version:]
0.10.0
\item[Archive size:]
5.89 MB
\href{https://packages.typst.org/preview/pyrunner-0.2.0.tar.gz}{\pandocbounded{\includesvg[keepaspectratio]{/assets/icons/16-download.svg}}}
\item[Repository:]
\href{https://github.com/peng1999/typst-pyrunner}{GitHub}
\item[Categor ies :]
\begin{itemize}
\tightlist
\item[]
\item
  \pandocbounded{\includesvg[keepaspectratio]{/assets/icons/16-code.svg}}
  \href{https://typst.app/universe/search/?category=scripting}{Scripting}
\item
  \pandocbounded{\includesvg[keepaspectratio]{/assets/icons/16-integration.svg}}
  \href{https://typst.app/universe/search/?category=integration}{Integration}
\end{itemize}
\end{description}

\subsubsection{Where to report issues?}\label{where-to-report-issues}

This package is a project of Peng Guanwen . Report issues on
\href{https://github.com/peng1999/typst-pyrunner}{their repository} .
You can also try to ask for help with this package on the
\href{https://forum.typst.app}{Forum} .

Please report this package to the Typst team using the
\href{https://typst.app/contact}{contact form} if you believe it is a
safety hazard or infringes upon your rights.

\phantomsection\label{versions}
\subsubsection{Version history}\label{version-history}

\begin{longtable}[]{@{}ll@{}}
\toprule\noalign{}
Version & Release Date \\
\midrule\noalign{}
\endhead
\bottomrule\noalign{}
\endlastfoot
0.2.0 & March 18, 2024 \\
\href{https://typst.app/universe/package/pyrunner/0.1.0/}{0.1.0} &
December 4, 2023 \\
\end{longtable}

Typst GmbH did not create this package and cannot guarantee correct
functionality of this package or compatibility with any version of the
Typst compiler or app.
