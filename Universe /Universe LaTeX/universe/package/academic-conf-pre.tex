\title{typst.app/universe/package/academic-conf-pre}

\phantomsection\label{banner}
\phantomsection\label{template-thumbnail}
\pandocbounded{\includegraphics[keepaspectratio]{https://packages.typst.org/preview/thumbnails/academic-conf-pre-0.1.0-small.webp}}

\section{academic-conf-pre}\label{academic-conf-pre}

{ 0.1.0 }

Slide Theme for Acadmic Presentations in Australia

\href{/app?template=academic-conf-pre&version=0.1.0}{Create project in
app}

\phantomsection\label{readme}
\subsection{1. Introduction}\label{introduction}

This is a template for \textbf{academic conference presentations} . It
is designed for the use in Typst which is simplier and more
user-friendly than LaTeX.

\subsection{2. How to use this template}\label{how-to-use-this-template}

themes documents are under \textbf{/themes/}

examples documents are under \textbf{/examples/}

if you don’t want to redesign the template, just follow the typ files
under the examples.

\subsection{3. How it looks}\label{how-it-looks}

The colors are entirely controllable, and I have provided three
relatively comfortable color schemes with \textbf{green} , \textbf{blue}
, and \textbf{red} as the base tones, which can be easily adjusted.

It is notable, there is a logo displayed in the center of the
presentation, and its \textbf{appearance} , \textbf{transparency} , and
\textbf{position} can be fully adjusted. Here, I use the University of
Sydney’s logo as an example.

You could see the PDFs under \textbf{examples/xxx.pdf}

\href{/app?template=academic-conf-pre&version=0.1.0}{Create project in
app}

\subsubsection{How to use}\label{how-to-use}

Click the button above to create a new project using this template in
the Typst app.

You can also use the Typst CLI to start a new project on your computer
using this command:

\begin{verbatim}
typst init @preview/academic-conf-pre:0.1.0
\end{verbatim}

\includesvg[width=0.16667in,height=0.16667in]{/assets/icons/16-copy.svg}

\subsubsection{About}\label{about}

\begin{description}
\tightlist
\item[Author :]
\href{mailto:isjun.liu@gmail.com}{JL-ghcoder}
\item[License:]
MIT
\item[Current version:]
0.1.0
\item[Last updated:]
November 5, 2024
\item[First released:]
November 5, 2024
\item[Archive size:]
1.29 MB
\href{https://packages.typst.org/preview/academic-conf-pre-0.1.0.tar.gz}{\pandocbounded{\includesvg[keepaspectratio]{/assets/icons/16-download.svg}}}
\item[Repository:]
\href{https://github.com/JL-ghcoder/Typst-Pre-Template}{GitHub}
\item[Categor y :]
\begin{itemize}
\tightlist
\item[]
\item
  \pandocbounded{\includesvg[keepaspectratio]{/assets/icons/16-presentation.svg}}
  \href{https://typst.app/universe/search/?category=presentation}{Presentation}
\end{itemize}
\end{description}

\subsubsection{Where to report issues?}\label{where-to-report-issues}

This template is a project of JL-ghcoder . Report issues on
\href{https://github.com/JL-ghcoder/Typst-Pre-Template}{their
repository} . You can also try to ask for help with this template on the
\href{https://forum.typst.app}{Forum} .

Please report this template to the Typst team using the
\href{https://typst.app/contact}{contact form} if you believe it is a
safety hazard or infringes upon your rights.

\phantomsection\label{versions}
\subsubsection{Version history}\label{version-history}

\begin{longtable}[]{@{}ll@{}}
\toprule\noalign{}
Version & Release Date \\
\midrule\noalign{}
\endhead
\bottomrule\noalign{}
\endlastfoot
0.1.0 & November 5, 2024 \\
\end{longtable}

Typst GmbH did not create this template and cannot guarantee correct
functionality of this template or compatibility with any version of the
Typst compiler or app.
