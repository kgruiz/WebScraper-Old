\title{typst.app/docs/reference/foundations/content}

\begin{itemize}
\tightlist
\item
  \href{/docs}{\includesvg[width=0.16667in,height=0.16667in]{/assets/icons/16-docs-dark.svg}}
\item
  \includesvg[width=0.16667in,height=0.16667in]{/assets/icons/16-arrow-right.svg}
\item
  \href{/docs/reference/}{Reference}
\item
  \includesvg[width=0.16667in,height=0.16667in]{/assets/icons/16-arrow-right.svg}
\item
  \href{/docs/reference/foundations/}{Foundations}
\item
  \includesvg[width=0.16667in,height=0.16667in]{/assets/icons/16-arrow-right.svg}
\item
  \href{/docs/reference/foundations/content/}{Content}
\end{itemize}

\section{\texorpdfstring{{ content }}{ content }}\label{summary}

A piece of document content.

This type is at the heart of Typst. All markup you write and most
\href{/docs/reference/foundations/function/}{functions} you call produce
content values. You can create a content value by enclosing markup in
square brackets. This is also how you pass content to functions.

\subsection{Example}\label{example}

\begin{verbatim}
Type of *Hello!* is
#type([*Hello!*])
\end{verbatim}

\includegraphics[width=5in,height=\textheight,keepaspectratio]{/assets/docs/X4qekl24YgH3SaXf1J0tagAAAAAAAAAA.png}

Content can be added with the \texttt{\ +\ } operator,
\href{/docs/reference/scripting/\#blocks}{joined together} and
multiplied with integers. Wherever content is expected, you can also
pass a \href{/docs/reference/foundations/str/}{string} or
\texttt{\ }{\texttt{\ none\ }}\texttt{\ } .

\subsection{Representation}\label{representation}

Content consists of elements with fields. When constructing an element
with its \emph{element function,} you provide these fields as arguments
and when you have a content value, you can access its fields with
\href{/docs/reference/scripting/\#field-access}{field access syntax} .

Some fields are required: These must be provided when constructing an
element and as a consequence, they are always available through field
access on content of that type. Required fields are marked as such in
the documentation.

Most fields are optional: Like required fields, they can be passed to
the element function to configure them for a single element. However,
these can also be configured with
\href{/docs/reference/styling/\#set-rules}{set rules} to apply them to
all elements within a scope. Optional fields are only available with
field access syntax when they were explicitly passed to the element
function, not when they result from a set rule.

Each element has a default appearance. However, you can also completely
customize its appearance with a
\href{/docs/reference/styling/\#show-rules}{show rule} . The show rule
is passed the element. It can access the element\textquotesingle s field
and produce arbitrary content from it.

In the web app, you can hover over a content variable to see exactly
which elements the content is composed of and what fields they have.
Alternatively, you can inspect the output of the
\href{/docs/reference/foundations/repr/}{\texttt{\ repr\ }} function.

\subsection{\texorpdfstring{{ Definitions
}}{ Definitions }}\label{definitions}

\phantomsection\label{definitions-tooltip}
Functions and types and can have associated definitions. These are
accessed by specifying the function or type, followed by a period, and
then the definition\textquotesingle s name.

\subsubsection{\texorpdfstring{\texttt{\ func\ }}{ func }}\label{definitions-func}

The content\textquotesingle s element function. This function can be
used to create the element contained in this content. It can be used in
set and show rules for the element. Can be compared with global
functions to check whether you have a specific kind of element.

self { . } { func } (

) -\textgreater{} \href{/docs/reference/foundations/function/}{function}

\subsubsection{\texorpdfstring{\texttt{\ has\ }}{ has }}\label{definitions-has}

Whether the content has the specified field.

self { . } { has } (

{ \href{/docs/reference/foundations/str/}{str} }

) -\textgreater{} \href{/docs/reference/foundations/bool/}{bool}

\paragraph{\texorpdfstring{\texttt{\ field\ }}{ field }}\label{definitions-has-field}

\href{/docs/reference/foundations/str/}{str}

{Required} {{ Positional }}

\phantomsection\label{definitions-has-field-positional-tooltip}
Positional parameters are specified in order, without names.

The field to look for.

\subsubsection{\texorpdfstring{\texttt{\ at\ }}{ at }}\label{definitions-at}

Access the specified field on the content. Returns the default value if
the field does not exist or fails with an error if no default value was
specified.

self { . } { at } (

{ \href{/docs/reference/foundations/str/}{str} , } {
\hyperref[definitions-at-parameters-default]{default :} { any } , }

) -\textgreater{} { any }

\paragraph{\texorpdfstring{\texttt{\ field\ }}{ field }}\label{definitions-at-field}

\href{/docs/reference/foundations/str/}{str}

{Required} {{ Positional }}

\phantomsection\label{definitions-at-field-positional-tooltip}
Positional parameters are specified in order, without names.

The field to access.

\paragraph{\texorpdfstring{\texttt{\ default\ }}{ default }}\label{definitions-at-default}

{ any }

A default value to return if the field does not exist.

\subsubsection{\texorpdfstring{\texttt{\ fields\ }}{ fields }}\label{definitions-fields}

Returns the fields of this content.

self { . } { fields } (

) -\textgreater{}
\href{/docs/reference/foundations/dictionary/}{dictionary}

\begin{verbatim}
#rect(
  width: 10cm,
  height: 10cm,
).fields()
\end{verbatim}

\includegraphics[width=5in,height=\textheight,keepaspectratio]{/assets/docs/zNlYUwJ_V8GS40gGav-GlwAAAAAAAAAA.png}

\subsubsection{\texorpdfstring{\texttt{\ location\ }}{ location }}\label{definitions-location}

The location of the content. This is only available on content returned
by \href{/docs/reference/introspection/query/}{query} or provided by a
\href{/docs/reference/styling/\#show-rules}{show rule} , for other
content it will be \texttt{\ }{\texttt{\ none\ }}\texttt{\ } . The
resulting location can be used with
\href{/docs/reference/introspection/counter/}{counters} ,
\href{/docs/reference/introspection/state/}{state} and
\href{/docs/reference/introspection/query/}{queries} .

self { . } { location } (

) -\textgreater{} \href{/docs/reference/foundations/none/}{none}
\href{/docs/reference/introspection/location/}{location}

\href{/docs/reference/foundations/calc/}{\pandocbounded{\includesvg[keepaspectratio]{/assets/icons/16-arrow-right.svg}}}

{ Calculation } { Previous page }

\href{/docs/reference/foundations/datetime/}{\pandocbounded{\includesvg[keepaspectratio]{/assets/icons/16-arrow-right.svg}}}

{ Datetime } { Next page }


\title{typst.app/docs/reference/foundations/float}

\begin{itemize}
\tightlist
\item
  \href{/docs}{\includesvg[width=0.16667in,height=0.16667in]{/assets/icons/16-docs-dark.svg}}
\item
  \includesvg[width=0.16667in,height=0.16667in]{/assets/icons/16-arrow-right.svg}
\item
  \href{/docs/reference/}{Reference}
\item
  \includesvg[width=0.16667in,height=0.16667in]{/assets/icons/16-arrow-right.svg}
\item
  \href{/docs/reference/foundations/}{Foundations}
\item
  \includesvg[width=0.16667in,height=0.16667in]{/assets/icons/16-arrow-right.svg}
\item
  \href{/docs/reference/foundations/float/}{Float}
\end{itemize}

\section{\texorpdfstring{{ float }}{ float }}\label{summary}

A floating-point number.

A limited-precision representation of a real number. Typst uses 64 bits
to store floats. Wherever a float is expected, you can also pass an
\href{/docs/reference/foundations/int/}{integer} .

You can convert a value to a float with this type\textquotesingle s
constructor.

NaN and positive infinity are available as
\texttt{\ float\ }{\texttt{\ .\ }}\texttt{\ nan\ } and
\texttt{\ float\ }{\texttt{\ .\ }}\texttt{\ inf\ } respectively.

\subsection{Example}\label{example}

\begin{verbatim}
#3.14 \
#1e4 \
#(10 / 4)
\end{verbatim}

\includegraphics[width=5in,height=\textheight,keepaspectratio]{/assets/docs/Oh7PyPKhSHHcwVH4CSb0KwAAAAAAAAAA.png}

\subsection{\texorpdfstring{Constructor
{}}{Constructor }}\label{constructor}

\phantomsection\label{constructor-constructor-tooltip}
If a type has a constructor, you can call it like a function to create a
new value of the type.

Converts a value to a float.

\begin{itemize}
\tightlist
\item
  Booleans are converted to \texttt{\ 0.0\ } or \texttt{\ 1.0\ } .
\item
  Integers are converted to the closest 64-bit float. For integers with
  absolute value less than
  \texttt{\ calc\ }{\texttt{\ .\ }}\texttt{\ }{\texttt{\ pow\ }}\texttt{\ }{\texttt{\ (\ }}\texttt{\ }{\texttt{\ 2\ }}\texttt{\ }{\texttt{\ ,\ }}\texttt{\ }{\texttt{\ 53\ }}\texttt{\ }{\texttt{\ )\ }}\texttt{\ }
  , this conversion is exact.
\item
  Ratios are divided by 100\%.
\item
  Strings are parsed in base 10 to the closest 64-bit float. Exponential
  notation is supported.
\end{itemize}

{ float } (

{ \href{/docs/reference/foundations/bool/}{bool}
\href{/docs/reference/foundations/int/}{int}
\href{/docs/reference/foundations/float/}{float}
\href{/docs/reference/layout/ratio/}{ratio}
\href{/docs/reference/foundations/str/}{str}
\href{/docs/reference/foundations/decimal/}{decimal} }

) -\textgreater{} \href{/docs/reference/foundations/float/}{float}

\begin{verbatim}
#float(false) \
#float(true) \
#float(4) \
#float(40%) \
#float("2.7") \
#float("1e5")
\end{verbatim}

\includegraphics[width=5in,height=\textheight,keepaspectratio]{/assets/docs/PMa-HqZaL4--FN_1I0OHagAAAAAAAAAA.png}

\paragraph{\texorpdfstring{\texttt{\ value\ }}{ value }}\label{constructor-value}

\href{/docs/reference/foundations/bool/}{bool} {or}
\href{/docs/reference/foundations/int/}{int} {or}
\href{/docs/reference/foundations/float/}{float} {or}
\href{/docs/reference/layout/ratio/}{ratio} {or}
\href{/docs/reference/foundations/str/}{str} {or}
\href{/docs/reference/foundations/decimal/}{decimal}

{Required} {{ Positional }}

\phantomsection\label{constructor-value-positional-tooltip}
Positional parameters are specified in order, without names.

The value that should be converted to a float.

\subsection{\texorpdfstring{{ Definitions
}}{ Definitions }}\label{definitions}

\phantomsection\label{definitions-tooltip}
Functions and types and can have associated definitions. These are
accessed by specifying the function or type, followed by a period, and
then the definition\textquotesingle s name.

\subsubsection{\texorpdfstring{\texttt{\ is-nan\ }}{ is-nan }}\label{definitions-is-nan}

Checks if a float is not a number.

In IEEE 754, more than one bit pattern represents a NaN. This function
returns \texttt{\ true\ } if the float is any of those bit patterns.

self { . } { is-nan } (

) -\textgreater{} \href{/docs/reference/foundations/bool/}{bool}

\begin{verbatim}
#float.is-nan(0) \
#float.is-nan(1) \
#float.is-nan(float.nan)
\end{verbatim}

\includegraphics[width=5in,height=\textheight,keepaspectratio]{/assets/docs/9jd8hxPcunH7CdCSXWd1dwAAAAAAAAAA.png}

\subsubsection{\texorpdfstring{\texttt{\ is-infinite\ }}{ is-infinite }}\label{definitions-is-infinite}

Checks if a float is infinite.

Floats can represent positive infinity and negative infinity. This
function returns \texttt{\ }{\texttt{\ true\ }}\texttt{\ } if the float
is an infinity.

self { . } { is-infinite } (

) -\textgreater{} \href{/docs/reference/foundations/bool/}{bool}

\begin{verbatim}
#float.is-infinite(0) \
#float.is-infinite(1) \
#float.is-infinite(float.inf)
\end{verbatim}

\includegraphics[width=5in,height=\textheight,keepaspectratio]{/assets/docs/AIoKhvpoq-xeueiSPD9O7gAAAAAAAAAA.png}

\subsubsection{\texorpdfstring{\texttt{\ signum\ }}{ signum }}\label{definitions-signum}

Calculates the sign of a floating point number.

\begin{itemize}
\tightlist
\item
  If the number is positive (including
  \texttt{\ }{\texttt{\ +\ }}\texttt{\ }{\texttt{\ 0.0\ }}\texttt{\ } ),
  returns \texttt{\ }{\texttt{\ 1.0\ }}\texttt{\ } .
\item
  If the number is negative (including
  \texttt{\ }{\texttt{\ -\ }}\texttt{\ }{\texttt{\ 0.0\ }}\texttt{\ } ),
  returns
  \texttt{\ }{\texttt{\ -\ }}\texttt{\ }{\texttt{\ 1.0\ }}\texttt{\ } .
\item
  If the number is NaN, returns
  \texttt{\ float\ }{\texttt{\ .\ }}\texttt{\ nan\ } .
\end{itemize}

self { . } { signum } (

) -\textgreater{} \href{/docs/reference/foundations/float/}{float}

\begin{verbatim}
#(5.0).signum() \
#(-5.0).signum() \
#(0.0).signum() \
#float.nan.signum()
\end{verbatim}

\includegraphics[width=5in,height=\textheight,keepaspectratio]{/assets/docs/HHp-pldXJoLbqAEsg_2mmQAAAAAAAAAA.png}

\subsubsection{\texorpdfstring{\texttt{\ from-bytes\ }}{ from-bytes }}\label{definitions-from-bytes}

Converts bytes to a float.

float { . } { from-bytes } (

{ \href{/docs/reference/foundations/bytes/}{bytes} , } {
\hyperref[definitions-from-bytes-parameters-endian]{endian :}
\href{/docs/reference/foundations/str/}{str} , }

) -\textgreater{} \href{/docs/reference/foundations/float/}{float}

\begin{verbatim}
#float.from-bytes(bytes((0, 0, 0, 0, 0, 0, 240, 63))) \
#float.from-bytes(bytes((63, 240, 0, 0, 0, 0, 0, 0)), endian: "big")
\end{verbatim}

\includegraphics[width=5in,height=\textheight,keepaspectratio]{/assets/docs/TbCinqru71JKOm73kOJYdwAAAAAAAAAA.png}

\paragraph{\texorpdfstring{\texttt{\ bytes\ }}{ bytes }}\label{definitions-from-bytes-bytes}

\href{/docs/reference/foundations/bytes/}{bytes}

{Required} {{ Positional }}

\phantomsection\label{definitions-from-bytes-bytes-positional-tooltip}
Positional parameters are specified in order, without names.

The bytes that should be converted to a float.

Must be of length exactly 8 so that the result fits into a 64-bit float.

\paragraph{\texorpdfstring{\texttt{\ endian\ }}{ endian }}\label{definitions-from-bytes-endian}

\href{/docs/reference/foundations/str/}{str}

The endianness of the conversion.

\begin{longtable}[]{@{}ll@{}}
\toprule\noalign{}
Variant & Details \\
\midrule\noalign{}
\endhead
\bottomrule\noalign{}
\endlastfoot
\texttt{\ "\ big\ "\ } & Big-endian byte order: The highest-value byte
is at the beginning of the bytes. \\
\texttt{\ "\ little\ "\ } & Little-endian byte order: The lowest-value
byte is at the beginning of the bytes. \\
\end{longtable}

Default: \texttt{\ }{\texttt{\ "little"\ }}\texttt{\ }

\subsubsection{\texorpdfstring{\texttt{\ to-bytes\ }}{ to-bytes }}\label{definitions-to-bytes}

Converts a float to bytes.

self { . } { to-bytes } (

{ \hyperref[definitions-to-bytes-parameters-endian]{endian :}
\href{/docs/reference/foundations/str/}{str} }

) -\textgreater{} \href{/docs/reference/foundations/bytes/}{bytes}

\begin{verbatim}
#array(1.0.to-bytes(endian: "big")) \
#array(1.0.to-bytes())
\end{verbatim}

\includegraphics[width=5in,height=\textheight,keepaspectratio]{/assets/docs/oyz50tHIOoQRj_5WM6JIbAAAAAAAAAAA.png}

\paragraph{\texorpdfstring{\texttt{\ endian\ }}{ endian }}\label{definitions-to-bytes-endian}

\href{/docs/reference/foundations/str/}{str}

The endianness of the conversion.

\begin{longtable}[]{@{}ll@{}}
\toprule\noalign{}
Variant & Details \\
\midrule\noalign{}
\endhead
\bottomrule\noalign{}
\endlastfoot
\texttt{\ "\ big\ "\ } & Big-endian byte order: The highest-value byte
is at the beginning of the bytes. \\
\texttt{\ "\ little\ "\ } & Little-endian byte order: The lowest-value
byte is at the beginning of the bytes. \\
\end{longtable}

Default: \texttt{\ }{\texttt{\ "little"\ }}\texttt{\ }

\href{/docs/reference/foundations/eval/}{\pandocbounded{\includesvg[keepaspectratio]{/assets/icons/16-arrow-right.svg}}}

{ Evaluate } { Previous page }

\href{/docs/reference/foundations/function/}{\pandocbounded{\includesvg[keepaspectratio]{/assets/icons/16-arrow-right.svg}}}

{ Function } { Next page }


\title{typst.app/docs/reference/foundations/label}

\begin{itemize}
\tightlist
\item
  \href{/docs}{\includesvg[width=0.16667in,height=0.16667in]{/assets/icons/16-docs-dark.svg}}
\item
  \includesvg[width=0.16667in,height=0.16667in]{/assets/icons/16-arrow-right.svg}
\item
  \href{/docs/reference/}{Reference}
\item
  \includesvg[width=0.16667in,height=0.16667in]{/assets/icons/16-arrow-right.svg}
\item
  \href{/docs/reference/foundations/}{Foundations}
\item
  \includesvg[width=0.16667in,height=0.16667in]{/assets/icons/16-arrow-right.svg}
\item
  \href{/docs/reference/foundations/label/}{Label}
\end{itemize}

\section{\texorpdfstring{{ label }}{ label }}\label{summary}

A label for an element.

Inserting a label into content attaches it to the closest preceding
element that is not a space. The preceding element must be in the same
scope as the label, which means that
\texttt{\ Hello\ }{\texttt{\ \#\ }}\texttt{\ }{\texttt{\ {[}\ }}\texttt{\ }{\texttt{\ \textless{}label\textgreater{}\ }}\texttt{\ }{\texttt{\ {]}\ }}\texttt{\ }
, for instance, wouldn\textquotesingle t work.

A labelled element can be \href{/docs/reference/model/ref/}{referenced}
, \href{/docs/reference/introspection/query/}{queried} for, and
\href{/docs/reference/styling/}{styled} through its label.

Once constructed, you can get the name of a label using
\href{/docs/reference/foundations/str/\#constructor}{\texttt{\ str\ }} .

\subsection{Example}\label{example}

\begin{verbatim}
#show <a>: set text(blue)
#show label("b"): set text(red)

= Heading <a>
*Strong* #label("b")
\end{verbatim}

\includegraphics[width=5in,height=\textheight,keepaspectratio]{/assets/docs/l3ZXI9iv-ZpcNuL82oagnwAAAAAAAAAA.png}

\subsection{Syntax}\label{syntax}

This function also has dedicated syntax: You can create a label by
enclosing its name in angle brackets. This works both in markup and
code. A label\textquotesingle s name can contain letters, numbers,
\texttt{\ \_\ } , \texttt{\ -\ } , \texttt{\ :\ } , and \texttt{\ .\ } .

Note that there is a syntactical difference when using the dedicated
syntax for this function. In the code below, the
\texttt{\ }{\texttt{\ \textless{}a\textgreater{}\ }}\texttt{\ }
terminates the heading and thus attaches to the heading itself, whereas
the
\texttt{\ }{\texttt{\ \#\ }}\texttt{\ }{\texttt{\ label\ }}\texttt{\ }{\texttt{\ (\ }}\texttt{\ }{\texttt{\ "b"\ }}\texttt{\ }{\texttt{\ )\ }}\texttt{\ }
is part of the heading and thus attaches to the
heading\textquotesingle s text.

\begin{verbatim}
// Equivalent to `#heading[Introduction] <a>`.
= Introduction <a>

// Equivalent to `#heading[Conclusion #label("b")]`.
= Conclusion #label("b")
\end{verbatim}

Currently, labels can only be attached to elements in markup mode, not
in code mode. This might change in the future.

\subsection{\texorpdfstring{Constructor
{}}{Constructor }}\label{constructor}

\phantomsection\label{constructor-constructor-tooltip}
If a type has a constructor, you can call it like a function to create a
new value of the type.

Creates a label from a string.

{ label } (

{ \href{/docs/reference/foundations/str/}{str} }

) -\textgreater{} \href{/docs/reference/foundations/label/}{label}

\paragraph{\texorpdfstring{\texttt{\ name\ }}{ name }}\label{constructor-name}

\href{/docs/reference/foundations/str/}{str}

{Required} {{ Positional }}

\phantomsection\label{constructor-name-positional-tooltip}
Positional parameters are specified in order, without names.

The name of the label.

\href{/docs/reference/foundations/int/}{\pandocbounded{\includesvg[keepaspectratio]{/assets/icons/16-arrow-right.svg}}}

{ Integer } { Previous page }

\href{/docs/reference/foundations/module/}{\pandocbounded{\includesvg[keepaspectratio]{/assets/icons/16-arrow-right.svg}}}

{ Module } { Next page }


\title{typst.app/docs/reference/foundations/plugin}

\begin{itemize}
\tightlist
\item
  \href{/docs}{\includesvg[width=0.16667in,height=0.16667in]{/assets/icons/16-docs-dark.svg}}
\item
  \includesvg[width=0.16667in,height=0.16667in]{/assets/icons/16-arrow-right.svg}
\item
  \href{/docs/reference/}{Reference}
\item
  \includesvg[width=0.16667in,height=0.16667in]{/assets/icons/16-arrow-right.svg}
\item
  \href{/docs/reference/foundations/}{Foundations}
\item
  \includesvg[width=0.16667in,height=0.16667in]{/assets/icons/16-arrow-right.svg}
\item
  \href{/docs/reference/foundations/plugin/}{Plugin}
\end{itemize}

\section{\texorpdfstring{{ plugin }}{ plugin }}\label{summary}

A WebAssembly plugin.

Typst is capable of interfacing with plugins compiled to WebAssembly.
Plugin functions may accept multiple
\href{/docs/reference/foundations/bytes/}{byte buffers} as arguments and
return a single byte buffer. They should typically be wrapped in
idiomatic Typst functions that perform the necessary conversions between
native Typst types and bytes.

Plugins run in isolation from your system, which means that printing,
reading files, or anything like that will not be supported for security
reasons. To run as a plugin, a program needs to be compiled to a 32-bit
shared WebAssembly library. Many compilers will use the
\href{https://wasi.dev/}{WASI ABI} by default or as their only option
(e.g. emscripten), which allows printing, reading files, etc. This ABI
will not directly work with Typst. You will either need to compile to a
different target or
\href{https://github.com/astrale-sharp/wasm-minimal-protocol/blob/master/wasi-stub}{stub
all functions} .

\subsection{Plugins and Packages}\label{plugins-and-packages}

Plugins are distributed as packages. A package can make use of a plugin
simply by including a WebAssembly file and loading it. Because the
byte-based plugin interface is quite low-level, plugins are typically
exposed through wrapper functions, that also live in the same package.

\subsection{Purity}\label{purity}

Plugin functions must be pure: Given the same arguments, they must
always return the same value. The reason for this is that Typst
functions must be pure (which is quite fundamental to the language
design) and, since Typst function can call plugin functions, this
requirement is inherited. In particular, if a plugin function is called
twice with the same arguments, Typst might cache the results and call
your function only once.

\subsection{Example}\label{example}

\begin{verbatim}
#let myplugin = plugin("hello.wasm")
#let concat(a, b) = str(
  myplugin.concatenate(
    bytes(a),
    bytes(b),
  )
)

#concat("hello", "world")
\end{verbatim}

\includegraphics[width=5in,height=\textheight,keepaspectratio]{/assets/docs/Vj65tyYDxxD3OHZUaiQ94QAAAAAAAAAA.png}

\subsection{Protocol}\label{protocol}

To be used as a plugin, a WebAssembly module must conform to the
following protocol:

\subsubsection{Exports}\label{exports}

A plugin module can export functions to make them callable from Typst.
To conform to the protocol, an exported function should:

\begin{itemize}
\item
  Take \texttt{\ n\ } 32-bit integer arguments \texttt{\ a\_1\ } ,
  \texttt{\ a\_2\ } , ..., \texttt{\ a\_n\ } (interpreted as lengths, so
  \texttt{\ usize/size\_t\ } may be preferable), and return one 32-bit
  integer.
\item
  The function should first allocate a buffer \texttt{\ buf\ } of length
  \texttt{\ a\_1\ +\ a\_2\ +\ ...\ +\ a\_n\ } , and then call
  \texttt{\ wasm\_minimal\_protocol\_write\_args\_to\_buffer(buf.ptr)\ }
  .
\item
  The \texttt{\ a\_1\ } first bytes of the buffer now constitute the
  first argument, the \texttt{\ a\_2\ } next bytes the second argument,
  and so on.
\item
  The function can now do its job with the arguments and produce an
  output buffer. Before returning, it should call
  \texttt{\ wasm\_minimal\_protocol\_send\_result\_to\_host\ } to send
  its result back to the host.
\item
  To signal success, the function should return \texttt{\ 0\ } .
\item
  To signal an error, the function should return \texttt{\ 1\ } . The
  written buffer is then interpreted as an UTF-8 encoded error message.
\end{itemize}

\subsubsection{Imports}\label{imports}

Plugin modules need to import two functions that are provided by the
runtime. (Types and functions are described using WAT syntax.)

\begin{itemize}
\item
  \texttt{\ (import\ "typst\_env"\ "wasm\_minimal\_protocol\_write\_args\_to\_buffer"\ (func\ (param\ i32)))\ }

  Writes the arguments for the current function into a plugin-allocated
  buffer. When a plugin function is called, it
  \hyperref[exports]{receives the lengths} of its input buffers as
  arguments. It should then allocate a buffer whose capacity is at least
  the sum of these lengths. It should then call this function with a
  \texttt{\ ptr\ } to the buffer to fill it with the arguments, one
  after another.
\item
  \texttt{\ (import\ "typst\_env"\ "wasm\_minimal\_protocol\_send\_result\_to\_host"\ (func\ (param\ i32\ i32)))\ }

  Sends the output of the current function to the host (Typst). The
  first parameter shall be a pointer to a buffer ( \texttt{\ ptr\ } ),
  while the second is the length of that buffer ( \texttt{\ len\ } ).
  The memory pointed at by \texttt{\ ptr\ } can be freed immediately
  after this function returns. If the message should be interpreted as
  an error message, it should be encoded as UTF-8.
\end{itemize}

\subsection{Resources}\label{resources}

For more resources, check out the
\href{https://github.com/astrale-sharp/wasm-minimal-protocol}{wasm-minimal-protocol
repository} . It contains:

\begin{itemize}
\tightlist
\item
  A list of example plugin implementations and a test runner for these
  examples
\item
  Wrappers to help you write your plugin in Rust (Zig wrapper in
  development)
\item
  A stubber for WASI
\end{itemize}

\subsection{\texorpdfstring{Constructor
{}}{Constructor }}\label{constructor}

\phantomsection\label{constructor-constructor-tooltip}
If a type has a constructor, you can call it like a function to create a
new value of the type.

Creates a new plugin from a WebAssembly file.

{ plugin } (

{ \href{/docs/reference/foundations/str/}{str} }

) -\textgreater{} \href{/docs/reference/foundations/plugin/}{plugin}

\paragraph{\texorpdfstring{\texttt{\ path\ }}{ path }}\label{constructor-path}

\href{/docs/reference/foundations/str/}{str}

{Required} {{ Positional }}

\phantomsection\label{constructor-path-positional-tooltip}
Positional parameters are specified in order, without names.

Path to a WebAssembly file.

For more details, see the \href{/docs/reference/syntax/\#paths}{Paths
section} .

\href{/docs/reference/foundations/panic/}{\pandocbounded{\includesvg[keepaspectratio]{/assets/icons/16-arrow-right.svg}}}

{ Panic } { Previous page }

\href{/docs/reference/foundations/regex/}{\pandocbounded{\includesvg[keepaspectratio]{/assets/icons/16-arrow-right.svg}}}

{ Regex } { Next page }


\title{typst.app/docs/reference/foundations/panic}

\begin{itemize}
\tightlist
\item
  \href{/docs}{\includesvg[width=0.16667in,height=0.16667in]{/assets/icons/16-docs-dark.svg}}
\item
  \includesvg[width=0.16667in,height=0.16667in]{/assets/icons/16-arrow-right.svg}
\item
  \href{/docs/reference/}{Reference}
\item
  \includesvg[width=0.16667in,height=0.16667in]{/assets/icons/16-arrow-right.svg}
\item
  \href{/docs/reference/foundations/}{Foundations}
\item
  \includesvg[width=0.16667in,height=0.16667in]{/assets/icons/16-arrow-right.svg}
\item
  \href{/docs/reference/foundations/panic/}{Panic}
\end{itemize}

\section{\texorpdfstring{\texttt{\ panic\ }}{ panic }}\label{summary}

Fails with an error.

Arguments are displayed to the user (not rendered in the document) as
strings, converting with \texttt{\ repr\ } if necessary.

\subsection{Example}\label{example}

The code below produces the error
\texttt{\ panicked\ with:\ "this\ is\ wrong"\ } .

\begin{verbatim}
#panic("this is wrong")
\end{verbatim}

\subsection{\texorpdfstring{{ Parameters
}}{ Parameters }}\label{parameters}

\phantomsection\label{parameters-tooltip}
Parameters are the inputs to a function. They are specified in
parentheses after the function name.

{ panic } (

{ \hyperref[parameters-values]{..} { any } }

)

\subsubsection{\texorpdfstring{\texttt{\ values\ }}{ values }}\label{parameters-values}

{ any }

{Required} {{ Positional }}

\phantomsection\label{parameters-values-positional-tooltip}
Positional parameters are specified in order, without names.

{{ Variadic }}

\phantomsection\label{parameters-values-variadic-tooltip}
Variadic parameters can be specified multiple times.

The values to panic with and display to the user.

\href{/docs/reference/foundations/none/}{\pandocbounded{\includesvg[keepaspectratio]{/assets/icons/16-arrow-right.svg}}}

{ None } { Previous page }

\href{/docs/reference/foundations/plugin/}{\pandocbounded{\includesvg[keepaspectratio]{/assets/icons/16-arrow-right.svg}}}

{ Plugin } { Next page }


\title{typst.app/docs/reference/foundations/array}

\begin{itemize}
\tightlist
\item
  \href{/docs}{\includesvg[width=0.16667in,height=0.16667in]{/assets/icons/16-docs-dark.svg}}
\item
  \includesvg[width=0.16667in,height=0.16667in]{/assets/icons/16-arrow-right.svg}
\item
  \href{/docs/reference/}{Reference}
\item
  \includesvg[width=0.16667in,height=0.16667in]{/assets/icons/16-arrow-right.svg}
\item
  \href{/docs/reference/foundations/}{Foundations}
\item
  \includesvg[width=0.16667in,height=0.16667in]{/assets/icons/16-arrow-right.svg}
\item
  \href{/docs/reference/foundations/array/}{Array}
\end{itemize}

\section{\texorpdfstring{{ array }}{ array }}\label{summary}

A sequence of values.

You can construct an array by enclosing a comma-separated sequence of
values in parentheses. The values do not have to be of the same type.

You can access and update array items with the \texttt{\ .at()\ }
method. Indices are zero-based and negative indices wrap around to the
end of the array. You can iterate over an array using a
\href{/docs/reference/scripting/\#loops}{for loop} . Arrays can be added
together with the \texttt{\ +\ } operator,
\href{/docs/reference/scripting/\#blocks}{joined together} and
multiplied with integers.

\textbf{Note:} An array of length one needs a trailing comma, as in
\texttt{\ }{\texttt{\ (\ }}\texttt{\ }{\texttt{\ 1\ }}\texttt{\ }{\texttt{\ ,\ }}\texttt{\ }{\texttt{\ )\ }}\texttt{\ }
. This is to disambiguate from a simple parenthesized expressions like
\texttt{\ }{\texttt{\ (\ }}\texttt{\ }{\texttt{\ 1\ }}\texttt{\ }{\texttt{\ +\ }}\texttt{\ }{\texttt{\ 2\ }}\texttt{\ }{\texttt{\ )\ }}\texttt{\ }{\texttt{\ *\ }}\texttt{\ }{\texttt{\ 3\ }}\texttt{\ }
. An empty array is written as
\texttt{\ }{\texttt{\ (\ }}\texttt{\ }{\texttt{\ )\ }}\texttt{\ } .

\subsection{Example}\label{example}

\begin{verbatim}
#let values = (1, 7, 4, -3, 2)

#values.at(0) \
#(values.at(0) = 3)
#values.at(-1) \
#values.find(calc.even) \
#values.filter(calc.odd) \
#values.map(calc.abs) \
#values.rev() \
#(1, (2, 3)).flatten() \
#(("A", "B", "C")
    .join(", ", last: " and "))
\end{verbatim}

\includegraphics[width=5in,height=\textheight,keepaspectratio]{/assets/docs/uC3P-2nGePaWZlTLapiUowAAAAAAAAAA.png}

\subsection{\texorpdfstring{Constructor
{}}{Constructor }}\label{constructor}

\phantomsection\label{constructor-constructor-tooltip}
If a type has a constructor, you can call it like a function to create a
new value of the type.

Converts a value to an array.

Note that this function is only intended for conversion of a
collection-like value to an array, not for creation of an array from
individual items. Use the array syntax \texttt{\ (1,\ 2,\ 3)\ } (or
\texttt{\ (1,)\ } for a single-element array) instead.

{ array } (

{ \href{/docs/reference/foundations/bytes/}{bytes}
\href{/docs/reference/foundations/array/}{array}
\href{/docs/reference/foundations/version/}{version} }

) -\textgreater{} \href{/docs/reference/foundations/array/}{array}

\begin{verbatim}
#let hi = "Hello 😃"
#array(bytes(hi))
\end{verbatim}

\includegraphics[width=5in,height=\textheight,keepaspectratio]{/assets/docs/X4h0etegVnRbtNlLnkRA5AAAAAAAAAAA.png}

\paragraph{\texorpdfstring{\texttt{\ value\ }}{ value }}\label{constructor-value}

\href{/docs/reference/foundations/bytes/}{bytes} {or}
\href{/docs/reference/foundations/array/}{array} {or}
\href{/docs/reference/foundations/version/}{version}

{Required} {{ Positional }}

\phantomsection\label{constructor-value-positional-tooltip}
Positional parameters are specified in order, without names.

The value that should be converted to an array.

\subsection{\texorpdfstring{{ Definitions
}}{ Definitions }}\label{definitions}

\phantomsection\label{definitions-tooltip}
Functions and types and can have associated definitions. These are
accessed by specifying the function or type, followed by a period, and
then the definition\textquotesingle s name.

\subsubsection{\texorpdfstring{\texttt{\ len\ }}{ len }}\label{definitions-len}

The number of values in the array.

self { . } { len } (

) -\textgreater{} \href{/docs/reference/foundations/int/}{int}

\subsubsection{\texorpdfstring{\texttt{\ first\ }}{ first }}\label{definitions-first}

Returns the first item in the array. May be used on the left-hand side
of an assignment. Fails with an error if the array is empty.

self { . } { first } (

) -\textgreater{} { any }

\subsubsection{\texorpdfstring{\texttt{\ last\ }}{ last }}\label{definitions-last}

Returns the last item in the array. May be used on the left-hand side of
an assignment. Fails with an error if the array is empty.

self { . } { last } (

) -\textgreater{} { any }

\subsubsection{\texorpdfstring{\texttt{\ at\ }}{ at }}\label{definitions-at}

Returns the item at the specified index in the array. May be used on the
left-hand side of an assignment. Returns the default value if the index
is out of bounds or fails with an error if no default value was
specified.

self { . } { at } (

{ \href{/docs/reference/foundations/int/}{int} , } {
\hyperref[definitions-at-parameters-default]{default :} { any } , }

) -\textgreater{} { any }

\paragraph{\texorpdfstring{\texttt{\ index\ }}{ index }}\label{definitions-at-index}

\href{/docs/reference/foundations/int/}{int}

{Required} {{ Positional }}

\phantomsection\label{definitions-at-index-positional-tooltip}
Positional parameters are specified in order, without names.

The index at which to retrieve the item. If negative, indexes from the
back.

\paragraph{\texorpdfstring{\texttt{\ default\ }}{ default }}\label{definitions-at-default}

{ any }

A default value to return if the index is out of bounds.

\subsubsection{\texorpdfstring{\texttt{\ push\ }}{ push }}\label{definitions-push}

Adds a value to the end of the array.

self { . } { push } (

{ { any } }

)

\paragraph{\texorpdfstring{\texttt{\ value\ }}{ value }}\label{definitions-push-value}

{ any }

{Required} {{ Positional }}

\phantomsection\label{definitions-push-value-positional-tooltip}
Positional parameters are specified in order, without names.

The value to insert at the end of the array.

\subsubsection{\texorpdfstring{\texttt{\ pop\ }}{ pop }}\label{definitions-pop}

Removes the last item from the array and returns it. Fails with an error
if the array is empty.

self { . } { pop } (

) -\textgreater{} { any }

\subsubsection{\texorpdfstring{\texttt{\ insert\ }}{ insert }}\label{definitions-insert}

Inserts a value into the array at the specified index, shifting all
subsequent elements to the right. Fails with an error if the index is
out of bounds.

To replace an element of an array, use
\href{/docs/reference/foundations/array/\#definitions-at}{\texttt{\ at\ }}
.

self { . } { insert } (

{ \href{/docs/reference/foundations/int/}{int} , } { { any } , }

)

\paragraph{\texorpdfstring{\texttt{\ index\ }}{ index }}\label{definitions-insert-index}

\href{/docs/reference/foundations/int/}{int}

{Required} {{ Positional }}

\phantomsection\label{definitions-insert-index-positional-tooltip}
Positional parameters are specified in order, without names.

The index at which to insert the item. If negative, indexes from the
back.

\paragraph{\texorpdfstring{\texttt{\ value\ }}{ value }}\label{definitions-insert-value}

{ any }

{Required} {{ Positional }}

\phantomsection\label{definitions-insert-value-positional-tooltip}
Positional parameters are specified in order, without names.

The value to insert into the array.

\subsubsection{\texorpdfstring{\texttt{\ remove\ }}{ remove }}\label{definitions-remove}

Removes the value at the specified index from the array and return it.

self { . } { remove } (

{ \href{/docs/reference/foundations/int/}{int} , } {
\hyperref[definitions-remove-parameters-default]{default :} { any } , }

) -\textgreater{} { any }

\paragraph{\texorpdfstring{\texttt{\ index\ }}{ index }}\label{definitions-remove-index}

\href{/docs/reference/foundations/int/}{int}

{Required} {{ Positional }}

\phantomsection\label{definitions-remove-index-positional-tooltip}
Positional parameters are specified in order, without names.

The index at which to remove the item. If negative, indexes from the
back.

\paragraph{\texorpdfstring{\texttt{\ default\ }}{ default }}\label{definitions-remove-default}

{ any }

A default value to return if the index is out of bounds.

\subsubsection{\texorpdfstring{\texttt{\ slice\ }}{ slice }}\label{definitions-slice}

Extracts a subslice of the array. Fails with an error if the start or
end index is out of bounds.

self { . } { slice } (

{ \href{/docs/reference/foundations/int/}{int} , } {
\href{/docs/reference/foundations/none/}{none}
\href{/docs/reference/foundations/int/}{int} , } {
\hyperref[definitions-slice-parameters-count]{count :}
\href{/docs/reference/foundations/int/}{int} , }

) -\textgreater{} \href{/docs/reference/foundations/array/}{array}

\paragraph{\texorpdfstring{\texttt{\ start\ }}{ start }}\label{definitions-slice-start}

\href{/docs/reference/foundations/int/}{int}

{Required} {{ Positional }}

\phantomsection\label{definitions-slice-start-positional-tooltip}
Positional parameters are specified in order, without names.

The start index (inclusive). If negative, indexes from the back.

\paragraph{\texorpdfstring{\texttt{\ end\ }}{ end }}\label{definitions-slice-end}

\href{/docs/reference/foundations/none/}{none} {or}
\href{/docs/reference/foundations/int/}{int}

{{ Positional }}

\phantomsection\label{definitions-slice-end-positional-tooltip}
Positional parameters are specified in order, without names.

The end index (exclusive). If omitted, the whole slice until the end of
the array is extracted. If negative, indexes from the back.

Default: \texttt{\ }{\texttt{\ none\ }}\texttt{\ }

\paragraph{\texorpdfstring{\texttt{\ count\ }}{ count }}\label{definitions-slice-count}

\href{/docs/reference/foundations/int/}{int}

The number of items to extract. This is equivalent to passing
\texttt{\ start\ +\ count\ } as the \texttt{\ end\ } position. Mutually
exclusive with \texttt{\ end\ } .

\subsubsection{\texorpdfstring{\texttt{\ contains\ }}{ contains }}\label{definitions-contains}

Whether the array contains the specified value.

This method also has dedicated syntax: You can write
\texttt{\ }{\texttt{\ 2\ }}\texttt{\ }{\texttt{\ in\ }}\texttt{\ }{\texttt{\ (\ }}\texttt{\ }{\texttt{\ 1\ }}\texttt{\ }{\texttt{\ ,\ }}\texttt{\ }{\texttt{\ 2\ }}\texttt{\ }{\texttt{\ ,\ }}\texttt{\ }{\texttt{\ 3\ }}\texttt{\ }{\texttt{\ )\ }}\texttt{\ }
instead of
\texttt{\ }{\texttt{\ (\ }}\texttt{\ }{\texttt{\ 1\ }}\texttt{\ }{\texttt{\ ,\ }}\texttt{\ }{\texttt{\ 2\ }}\texttt{\ }{\texttt{\ ,\ }}\texttt{\ }{\texttt{\ 3\ }}\texttt{\ }{\texttt{\ )\ }}\texttt{\ }{\texttt{\ .\ }}\texttt{\ }{\texttt{\ contains\ }}\texttt{\ }{\texttt{\ (\ }}\texttt{\ }{\texttt{\ 2\ }}\texttt{\ }{\texttt{\ )\ }}\texttt{\ }
.

self { . } { contains } (

{ { any } }

) -\textgreater{} \href{/docs/reference/foundations/bool/}{bool}

\paragraph{\texorpdfstring{\texttt{\ value\ }}{ value }}\label{definitions-contains-value}

{ any }

{Required} {{ Positional }}

\phantomsection\label{definitions-contains-value-positional-tooltip}
Positional parameters are specified in order, without names.

The value to search for.

\subsubsection{\texorpdfstring{\texttt{\ find\ }}{ find }}\label{definitions-find}

Searches for an item for which the given function returns
\texttt{\ }{\texttt{\ true\ }}\texttt{\ } and returns the first match or
\texttt{\ }{\texttt{\ none\ }}\texttt{\ } if there is no match.

self { . } { find } (

{ \href{/docs/reference/foundations/function/}{function} }

) -\textgreater{} { any } \href{/docs/reference/foundations/none/}{none}

\paragraph{\texorpdfstring{\texttt{\ searcher\ }}{ searcher }}\label{definitions-find-searcher}

\href{/docs/reference/foundations/function/}{function}

{Required} {{ Positional }}

\phantomsection\label{definitions-find-searcher-positional-tooltip}
Positional parameters are specified in order, without names.

The function to apply to each item. Must return a boolean.

\subsubsection{\texorpdfstring{\texttt{\ position\ }}{ position }}\label{definitions-position}

Searches for an item for which the given function returns
\texttt{\ }{\texttt{\ true\ }}\texttt{\ } and returns the index of the
first match or \texttt{\ }{\texttt{\ none\ }}\texttt{\ } if there is no
match.

self { . } { position } (

{ \href{/docs/reference/foundations/function/}{function} }

) -\textgreater{} \href{/docs/reference/foundations/none/}{none}
\href{/docs/reference/foundations/int/}{int}

\paragraph{\texorpdfstring{\texttt{\ searcher\ }}{ searcher }}\label{definitions-position-searcher}

\href{/docs/reference/foundations/function/}{function}

{Required} {{ Positional }}

\phantomsection\label{definitions-position-searcher-positional-tooltip}
Positional parameters are specified in order, without names.

The function to apply to each item. Must return a boolean.

\subsubsection{\texorpdfstring{\texttt{\ range\ }}{ range }}\label{definitions-range}

Create an array consisting of a sequence of numbers.

If you pass just one positional parameter, it is interpreted as the
\texttt{\ end\ } of the range. If you pass two, they describe the
\texttt{\ start\ } and \texttt{\ end\ } of the range.

This function is available both in the array function\textquotesingle s
scope and globally.

array { . } { range } (

{ \href{/docs/reference/foundations/int/}{int} , } {
\href{/docs/reference/foundations/int/}{int} , } {
\hyperref[definitions-range-parameters-step]{step :}
\href{/docs/reference/foundations/int/}{int} , }

) -\textgreater{} \href{/docs/reference/foundations/array/}{array}

\begin{verbatim}
#range(5) \
#range(2, 5) \
#range(20, step: 4) \
#range(21, step: 4) \
#range(5, 2, step: -1)
\end{verbatim}

\includegraphics[width=5in,height=\textheight,keepaspectratio]{/assets/docs/zrh5Y9Alyv5p1PUCuyz0bAAAAAAAAAAA.png}

\paragraph{\texorpdfstring{\texttt{\ start\ }}{ start }}\label{definitions-range-start}

\href{/docs/reference/foundations/int/}{int}

{{ Positional }}

\phantomsection\label{definitions-range-start-positional-tooltip}
Positional parameters are specified in order, without names.

The start of the range (inclusive).

Default: \texttt{\ }{\texttt{\ 0\ }}\texttt{\ }

\paragraph{\texorpdfstring{\texttt{\ end\ }}{ end }}\label{definitions-range-end}

\href{/docs/reference/foundations/int/}{int}

{Required} {{ Positional }}

\phantomsection\label{definitions-range-end-positional-tooltip}
Positional parameters are specified in order, without names.

The end of the range (exclusive).

\paragraph{\texorpdfstring{\texttt{\ step\ }}{ step }}\label{definitions-range-step}

\href{/docs/reference/foundations/int/}{int}

The distance between the generated numbers.

Default: \texttt{\ }{\texttt{\ 1\ }}\texttt{\ }

\subsubsection{\texorpdfstring{\texttt{\ filter\ }}{ filter }}\label{definitions-filter}

Produces a new array with only the items from the original one for which
the given function returns true.

self { . } { filter } (

{ \href{/docs/reference/foundations/function/}{function} }

) -\textgreater{} \href{/docs/reference/foundations/array/}{array}

\paragraph{\texorpdfstring{\texttt{\ test\ }}{ test }}\label{definitions-filter-test}

\href{/docs/reference/foundations/function/}{function}

{Required} {{ Positional }}

\phantomsection\label{definitions-filter-test-positional-tooltip}
Positional parameters are specified in order, without names.

The function to apply to each item. Must return a boolean.

\subsubsection{\texorpdfstring{\texttt{\ map\ }}{ map }}\label{definitions-map}

Produces a new array in which all items from the original one were
transformed with the given function.

self { . } { map } (

{ \href{/docs/reference/foundations/function/}{function} }

) -\textgreater{} \href{/docs/reference/foundations/array/}{array}

\paragraph{\texorpdfstring{\texttt{\ mapper\ }}{ mapper }}\label{definitions-map-mapper}

\href{/docs/reference/foundations/function/}{function}

{Required} {{ Positional }}

\phantomsection\label{definitions-map-mapper-positional-tooltip}
Positional parameters are specified in order, without names.

The function to apply to each item.

\subsubsection{\texorpdfstring{\texttt{\ enumerate\ }}{ enumerate }}\label{definitions-enumerate}

Returns a new array with the values alongside their indices.

The returned array consists of \texttt{\ (index,\ value)\ } pairs in the
form of length-2 arrays. These can be
\href{/docs/reference/scripting/\#bindings}{destructured} with a let
binding or for loop.

self { . } { enumerate } (

{ \hyperref[definitions-enumerate-parameters-start]{start :}
\href{/docs/reference/foundations/int/}{int} }

) -\textgreater{} \href{/docs/reference/foundations/array/}{array}

\paragraph{\texorpdfstring{\texttt{\ start\ }}{ start }}\label{definitions-enumerate-start}

\href{/docs/reference/foundations/int/}{int}

The index returned for the first pair of the returned list.

Default: \texttt{\ }{\texttt{\ 0\ }}\texttt{\ }

\subsubsection{\texorpdfstring{\texttt{\ zip\ }}{ zip }}\label{definitions-zip}

Zips the array with other arrays.

Returns an array of arrays, where the \texttt{\ i\ } th inner array
contains all the \texttt{\ i\ } th elements from each original array.

If the arrays to be zipped have different lengths, they are zipped up to
the last element of the shortest array and all remaining elements are
ignored.

This function is variadic, meaning that you can zip multiple arrays
together at once:
\texttt{\ }{\texttt{\ (\ }}\texttt{\ }{\texttt{\ 1\ }}\texttt{\ }{\texttt{\ ,\ }}\texttt{\ }{\texttt{\ 2\ }}\texttt{\ }{\texttt{\ )\ }}\texttt{\ }{\texttt{\ .\ }}\texttt{\ }{\texttt{\ zip\ }}\texttt{\ }{\texttt{\ (\ }}\texttt{\ }{\texttt{\ (\ }}\texttt{\ }{\texttt{\ "A"\ }}\texttt{\ }{\texttt{\ ,\ }}\texttt{\ }{\texttt{\ "B"\ }}\texttt{\ }{\texttt{\ )\ }}\texttt{\ }{\texttt{\ ,\ }}\texttt{\ }{\texttt{\ (\ }}\texttt{\ }{\texttt{\ 10\ }}\texttt{\ }{\texttt{\ ,\ }}\texttt{\ }{\texttt{\ 20\ }}\texttt{\ }{\texttt{\ )\ }}\texttt{\ }{\texttt{\ )\ }}\texttt{\ }
yields
\texttt{\ }{\texttt{\ (\ }}\texttt{\ }{\texttt{\ (\ }}\texttt{\ }{\texttt{\ 1\ }}\texttt{\ }{\texttt{\ ,\ }}\texttt{\ }{\texttt{\ "A"\ }}\texttt{\ }{\texttt{\ ,\ }}\texttt{\ }{\texttt{\ 10\ }}\texttt{\ }{\texttt{\ )\ }}\texttt{\ }{\texttt{\ ,\ }}\texttt{\ }{\texttt{\ (\ }}\texttt{\ }{\texttt{\ 2\ }}\texttt{\ }{\texttt{\ ,\ }}\texttt{\ }{\texttt{\ "B"\ }}\texttt{\ }{\texttt{\ ,\ }}\texttt{\ }{\texttt{\ 20\ }}\texttt{\ }{\texttt{\ )\ }}\texttt{\ }{\texttt{\ )\ }}\texttt{\ }
.

self { . } { zip } (

{ \hyperref[definitions-zip-parameters-exact]{exact :}
\href{/docs/reference/foundations/bool/}{bool} , } {
\hyperref[definitions-zip-parameters-others]{..}
\href{/docs/reference/foundations/array/}{array} , }

) -\textgreater{} \href{/docs/reference/foundations/array/}{array}

\paragraph{\texorpdfstring{\texttt{\ exact\ }}{ exact }}\label{definitions-zip-exact}

\href{/docs/reference/foundations/bool/}{bool}

Whether all arrays have to have the same length. For example,
\texttt{\ }{\texttt{\ (\ }}\texttt{\ }{\texttt{\ 1\ }}\texttt{\ }{\texttt{\ ,\ }}\texttt{\ }{\texttt{\ 2\ }}\texttt{\ }{\texttt{\ )\ }}\texttt{\ }{\texttt{\ .\ }}\texttt{\ }{\texttt{\ zip\ }}\texttt{\ }{\texttt{\ (\ }}\texttt{\ }{\texttt{\ (\ }}\texttt{\ }{\texttt{\ 1\ }}\texttt{\ }{\texttt{\ ,\ }}\texttt{\ }{\texttt{\ 2\ }}\texttt{\ }{\texttt{\ ,\ }}\texttt{\ }{\texttt{\ 3\ }}\texttt{\ }{\texttt{\ )\ }}\texttt{\ }{\texttt{\ ,\ }}\texttt{\ exact\ }{\texttt{\ :\ }}\texttt{\ }{\texttt{\ true\ }}\texttt{\ }{\texttt{\ )\ }}\texttt{\ }
produces an error.

Default: \texttt{\ }{\texttt{\ false\ }}\texttt{\ }

\paragraph{\texorpdfstring{\texttt{\ others\ }}{ others }}\label{definitions-zip-others}

\href{/docs/reference/foundations/array/}{array}

{Required} {{ Positional }}

\phantomsection\label{definitions-zip-others-positional-tooltip}
Positional parameters are specified in order, without names.

{{ Variadic }}

\phantomsection\label{definitions-zip-others-variadic-tooltip}
Variadic parameters can be specified multiple times.

The arrays to zip with.

\subsubsection{\texorpdfstring{\texttt{\ fold\ }}{ fold }}\label{definitions-fold}

Folds all items into a single value using an accumulator function.

self { . } { fold } (

{ { any } , } { \href{/docs/reference/foundations/function/}{function} ,
}

) -\textgreater{} { any }

\paragraph{\texorpdfstring{\texttt{\ init\ }}{ init }}\label{definitions-fold-init}

{ any }

{Required} {{ Positional }}

\phantomsection\label{definitions-fold-init-positional-tooltip}
Positional parameters are specified in order, without names.

The initial value to start with.

\paragraph{\texorpdfstring{\texttt{\ folder\ }}{ folder }}\label{definitions-fold-folder}

\href{/docs/reference/foundations/function/}{function}

{Required} {{ Positional }}

\phantomsection\label{definitions-fold-folder-positional-tooltip}
Positional parameters are specified in order, without names.

The folding function. Must have two parameters: One for the accumulated
value and one for an item.

\subsubsection{\texorpdfstring{\texttt{\ sum\ }}{ sum }}\label{definitions-sum}

Sums all items (works for all types that can be added).

self { . } { sum } (

{ \hyperref[definitions-sum-parameters-default]{default :} { any } }

) -\textgreater{} { any }

\paragraph{\texorpdfstring{\texttt{\ default\ }}{ default }}\label{definitions-sum-default}

{ any }

What to return if the array is empty. Must be set if the array can be
empty.

\subsubsection{\texorpdfstring{\texttt{\ product\ }}{ product }}\label{definitions-product}

Calculates the product all items (works for all types that can be
multiplied).

self { . } { product } (

{ \hyperref[definitions-product-parameters-default]{default :} { any } }

) -\textgreater{} { any }

\paragraph{\texorpdfstring{\texttt{\ default\ }}{ default }}\label{definitions-product-default}

{ any }

What to return if the array is empty. Must be set if the array can be
empty.

\subsubsection{\texorpdfstring{\texttt{\ any\ }}{ any }}\label{definitions-any}

Whether the given function returns
\texttt{\ }{\texttt{\ true\ }}\texttt{\ } for any item in the array.

self { . } { any } (

{ \href{/docs/reference/foundations/function/}{function} }

) -\textgreater{} \href{/docs/reference/foundations/bool/}{bool}

\paragraph{\texorpdfstring{\texttt{\ test\ }}{ test }}\label{definitions-any-test}

\href{/docs/reference/foundations/function/}{function}

{Required} {{ Positional }}

\phantomsection\label{definitions-any-test-positional-tooltip}
Positional parameters are specified in order, without names.

The function to apply to each item. Must return a boolean.

\subsubsection{\texorpdfstring{\texttt{\ all\ }}{ all }}\label{definitions-all}

Whether the given function returns
\texttt{\ }{\texttt{\ true\ }}\texttt{\ } for all items in the array.

self { . } { all } (

{ \href{/docs/reference/foundations/function/}{function} }

) -\textgreater{} \href{/docs/reference/foundations/bool/}{bool}

\paragraph{\texorpdfstring{\texttt{\ test\ }}{ test }}\label{definitions-all-test}

\href{/docs/reference/foundations/function/}{function}

{Required} {{ Positional }}

\phantomsection\label{definitions-all-test-positional-tooltip}
Positional parameters are specified in order, without names.

The function to apply to each item. Must return a boolean.

\subsubsection{\texorpdfstring{\texttt{\ flatten\ }}{ flatten }}\label{definitions-flatten}

Combine all nested arrays into a single flat one.

self { . } { flatten } (

) -\textgreater{} \href{/docs/reference/foundations/array/}{array}

\subsubsection{\texorpdfstring{\texttt{\ rev\ }}{ rev }}\label{definitions-rev}

Return a new array with the same items, but in reverse order.

self { . } { rev } (

) -\textgreater{} \href{/docs/reference/foundations/array/}{array}

\subsubsection{\texorpdfstring{\texttt{\ split\ }}{ split }}\label{definitions-split}

Split the array at occurrences of the specified value.

self { . } { split } (

{ { any } }

) -\textgreater{} \href{/docs/reference/foundations/array/}{array}

\paragraph{\texorpdfstring{\texttt{\ at\ }}{ at }}\label{definitions-split-at}

{ any }

{Required} {{ Positional }}

\phantomsection\label{definitions-split-at-positional-tooltip}
Positional parameters are specified in order, without names.

The value to split at.

\subsubsection{\texorpdfstring{\texttt{\ join\ }}{ join }}\label{definitions-join}

Combine all items in the array into one.

self { . } { join } (

{ { any } \href{/docs/reference/foundations/none/}{none} , } {
\hyperref[definitions-join-parameters-last]{last :} { any } , }

) -\textgreater{} { any }

\paragraph{\texorpdfstring{\texttt{\ separator\ }}{ separator }}\label{definitions-join-separator}

{ any } {or} \href{/docs/reference/foundations/none/}{none}

{{ Positional }}

\phantomsection\label{definitions-join-separator-positional-tooltip}
Positional parameters are specified in order, without names.

A value to insert between each item of the array.

Default: \texttt{\ }{\texttt{\ none\ }}\texttt{\ }

\paragraph{\texorpdfstring{\texttt{\ last\ }}{ last }}\label{definitions-join-last}

{ any }

An alternative separator between the last two items.

\subsubsection{\texorpdfstring{\texttt{\ intersperse\ }}{ intersperse }}\label{definitions-intersperse}

Returns an array with a copy of the separator value placed between
adjacent elements.

self { . } { intersperse } (

{ { any } }

) -\textgreater{} \href{/docs/reference/foundations/array/}{array}

\paragraph{\texorpdfstring{\texttt{\ separator\ }}{ separator }}\label{definitions-intersperse-separator}

{ any }

{Required} {{ Positional }}

\phantomsection\label{definitions-intersperse-separator-positional-tooltip}
Positional parameters are specified in order, without names.

The value that will be placed between each adjacent element.

\subsubsection{\texorpdfstring{\texttt{\ chunks\ }}{ chunks }}\label{definitions-chunks}

Splits an array into non-overlapping chunks, starting at the beginning,
ending with a single remainder chunk.

All chunks but the last have \texttt{\ chunk-size\ } elements. If
\texttt{\ exact\ } is set to \texttt{\ }{\texttt{\ true\ }}\texttt{\ } ,
the remainder is dropped if it contains less than
\texttt{\ chunk-size\ } elements.

self { . } { chunks } (

{ \href{/docs/reference/foundations/int/}{int} , } {
\hyperref[definitions-chunks-parameters-exact]{exact :}
\href{/docs/reference/foundations/bool/}{bool} , }

) -\textgreater{} \href{/docs/reference/foundations/array/}{array}

\begin{verbatim}
#let array = (1, 2, 3, 4, 5, 6, 7, 8)
#array.chunks(3)
#array.chunks(3, exact: true)
\end{verbatim}

\includegraphics[width=5in,height=\textheight,keepaspectratio]{/assets/docs/Nt1-jyrzTUv2d90xr98pvAAAAAAAAAAA.png}

\paragraph{\texorpdfstring{\texttt{\ chunk-size\ }}{ chunk-size }}\label{definitions-chunks-chunk-size}

\href{/docs/reference/foundations/int/}{int}

{Required} {{ Positional }}

\phantomsection\label{definitions-chunks-chunk-size-positional-tooltip}
Positional parameters are specified in order, without names.

How many elements each chunk may at most contain.

\paragraph{\texorpdfstring{\texttt{\ exact\ }}{ exact }}\label{definitions-chunks-exact}

\href{/docs/reference/foundations/bool/}{bool}

Whether to keep the remainder if its size is less than
\texttt{\ chunk-size\ } .

Default: \texttt{\ }{\texttt{\ false\ }}\texttt{\ }

\subsubsection{\texorpdfstring{\texttt{\ windows\ }}{ windows }}\label{definitions-windows}

Returns sliding windows of \texttt{\ window-size\ } elements over an
array.

If the array length is less than \texttt{\ window-size\ } , this will
return an empty array.

self { . } { windows } (

{ \href{/docs/reference/foundations/int/}{int} }

) -\textgreater{} \href{/docs/reference/foundations/array/}{array}

\begin{verbatim}
#let array = (1, 2, 3, 4, 5, 6, 7, 8)
#array.windows(5)
\end{verbatim}

\includegraphics[width=5in,height=\textheight,keepaspectratio]{/assets/docs/Gacy-jUdBfccX43fxzwqPgAAAAAAAAAA.png}

\paragraph{\texorpdfstring{\texttt{\ window-size\ }}{ window-size }}\label{definitions-windows-window-size}

\href{/docs/reference/foundations/int/}{int}

{Required} {{ Positional }}

\phantomsection\label{definitions-windows-window-size-positional-tooltip}
Positional parameters are specified in order, without names.

How many elements each window will contain.

\subsubsection{\texorpdfstring{\texttt{\ sorted\ }}{ sorted }}\label{definitions-sorted}

Return a sorted version of this array, optionally by a given key
function. The sorting algorithm used is stable.

Returns an error if two values could not be compared or if the key
function (if given) yields an error.

self { . } { sorted } (

{ \hyperref[definitions-sorted-parameters-key]{key :}
\href{/docs/reference/foundations/function/}{function} }

) -\textgreater{} \href{/docs/reference/foundations/array/}{array}

\paragraph{\texorpdfstring{\texttt{\ key\ }}{ key }}\label{definitions-sorted-key}

\href{/docs/reference/foundations/function/}{function}

If given, applies this function to the elements in the array to
determine the keys to sort by.

\subsubsection{\texorpdfstring{\texttt{\ dedup\ }}{ dedup }}\label{definitions-dedup}

Deduplicates all items in the array.

Returns a new array with all duplicate items removed. Only the first
element of each duplicate is kept.

self { . } { dedup } (

{ \hyperref[definitions-dedup-parameters-key]{key :}
\href{/docs/reference/foundations/function/}{function} }

) -\textgreater{} \href{/docs/reference/foundations/array/}{array}

\begin{verbatim}
#(1, 1, 2, 3, 1).dedup()
\end{verbatim}

\includegraphics[width=5in,height=\textheight,keepaspectratio]{/assets/docs/N8Cp27Nhseeu9VhP3b-g0gAAAAAAAAAA.png}

\paragraph{\texorpdfstring{\texttt{\ key\ }}{ key }}\label{definitions-dedup-key}

\href{/docs/reference/foundations/function/}{function}

If given, applies this function to the elements in the array to
determine the keys to deduplicate by.

\subsubsection{\texorpdfstring{\texttt{\ to-dict\ }}{ to-dict }}\label{definitions-to-dict}

Converts an array of pairs into a dictionary. The first value of each
pair is the key, the second the value.

If the same key occurs multiple times, the last value is selected.

self { . } { to-dict } (

) -\textgreater{}
\href{/docs/reference/foundations/dictionary/}{dictionary}

\begin{verbatim}
#(
  ("apples", 2),
  ("peaches", 3),
  ("apples", 5),
).to-dict()
\end{verbatim}

\includegraphics[width=5in,height=\textheight,keepaspectratio]{/assets/docs/LmuORFz3ft0CLd-WiUZHngAAAAAAAAAA.png}

\subsubsection{\texorpdfstring{\texttt{\ reduce\ }}{ reduce }}\label{definitions-reduce}

Reduces the elements to a single one, by repeatedly applying a reducing
operation.

If the array is empty, returns \texttt{\ }{\texttt{\ none\ }}\texttt{\ }
, otherwise, returns the result of the reduction.

The reducing function is a closure with two arguments: an "accumulator",
and an element.

For arrays with at least one element, this is the same as
\href{/docs/reference/foundations/array/\#definitions-fold}{\texttt{\ array.fold\ }}
with the first element of the array as the initial accumulator value,
folding every subsequent element into it.

self { . } { reduce } (

{ \href{/docs/reference/foundations/function/}{function} }

) -\textgreater{} { any }

\paragraph{\texorpdfstring{\texttt{\ reducer\ }}{ reducer }}\label{definitions-reduce-reducer}

\href{/docs/reference/foundations/function/}{function}

{Required} {{ Positional }}

\phantomsection\label{definitions-reduce-reducer-positional-tooltip}
Positional parameters are specified in order, without names.

The reducing function. Must have two parameters: One for the accumulated
value and one for an item.

\href{/docs/reference/foundations/arguments/}{\pandocbounded{\includesvg[keepaspectratio]{/assets/icons/16-arrow-right.svg}}}

{ Arguments } { Previous page }

\href{/docs/reference/foundations/assert/}{\pandocbounded{\includesvg[keepaspectratio]{/assets/icons/16-arrow-right.svg}}}

{ Assert } { Next page }


\title{typst.app/docs/reference/foundations/calc}

\begin{itemize}
\tightlist
\item
  \href{/docs}{\includesvg[width=0.16667in,height=0.16667in]{/assets/icons/16-docs-dark.svg}}
\item
  \includesvg[width=0.16667in,height=0.16667in]{/assets/icons/16-arrow-right.svg}
\item
  \href{/docs/reference/}{Reference}
\item
  \includesvg[width=0.16667in,height=0.16667in]{/assets/icons/16-arrow-right.svg}
\item
  \href{/docs/reference/foundations/}{Foundations}
\item
  \includesvg[width=0.16667in,height=0.16667in]{/assets/icons/16-arrow-right.svg}
\item
  \href{/docs/reference/foundations/calc}{Calculation}
\end{itemize}

\section{Calculation}\label{summary}

Module for calculations and processing of numeric values.

These definitions are part of the \texttt{\ calc\ } module and not
imported by default. In addition to the functions listed below, the
\texttt{\ calc\ } module also defines the constants \texttt{\ pi\ } ,
\texttt{\ tau\ } , \texttt{\ e\ } , and \texttt{\ inf\ } .

\subsection{Functions}\label{functions}

\subsubsection{\texorpdfstring{\texttt{\ abs\ }}{ abs }}\label{functions-abs}

Calculates the absolute value of a numeric value.

calc { . } { abs } (

{ \href{/docs/reference/foundations/int/}{int}
\href{/docs/reference/foundations/float/}{float}
\href{/docs/reference/layout/length/}{length}
\href{/docs/reference/layout/angle/}{angle}
\href{/docs/reference/layout/ratio/}{ratio}
\href{/docs/reference/layout/fraction/}{fraction}
\href{/docs/reference/foundations/decimal/}{decimal} }

) -\textgreater{} { any }

\begin{verbatim}
#calc.abs(-5) \
#calc.abs(5pt - 2cm) \
#calc.abs(2fr) \
#calc.abs(decimal("-342.440"))
\end{verbatim}

\includegraphics[width=5in,height=\textheight,keepaspectratio]{/assets/docs/1nPNk-RAyXUEHrAszyCnUgAAAAAAAAAA.png}

\paragraph{\texorpdfstring{\texttt{\ value\ }}{ value }}\label{functions-abs-value}

\href{/docs/reference/foundations/int/}{int} {or}
\href{/docs/reference/foundations/float/}{float} {or}
\href{/docs/reference/layout/length/}{length} {or}
\href{/docs/reference/layout/angle/}{angle} {or}
\href{/docs/reference/layout/ratio/}{ratio} {or}
\href{/docs/reference/layout/fraction/}{fraction} {or}
\href{/docs/reference/foundations/decimal/}{decimal}

{Required} {{ Positional }}

\phantomsection\label{functions-abs-value-positional-tooltip}
Positional parameters are specified in order, without names.

The value whose absolute value to calculate.

\subsubsection{\texorpdfstring{\texttt{\ pow\ }}{ pow }}\label{functions-pow}

Raises a value to some exponent.

calc { . } { pow } (

{ \href{/docs/reference/foundations/int/}{int}
\href{/docs/reference/foundations/float/}{float}
\href{/docs/reference/foundations/decimal/}{decimal} , } {
\href{/docs/reference/foundations/int/}{int}
\href{/docs/reference/foundations/float/}{float} , }

) -\textgreater{} \href{/docs/reference/foundations/int/}{int}
\href{/docs/reference/foundations/float/}{float}
\href{/docs/reference/foundations/decimal/}{decimal}

\begin{verbatim}
#calc.pow(2, 3) \
#calc.pow(decimal("2.5"), 2)
\end{verbatim}

\includegraphics[width=5in,height=\textheight,keepaspectratio]{/assets/docs/YQoOsFNxPEgW0b-n9B_VrAAAAAAAAAAA.png}

\paragraph{\texorpdfstring{\texttt{\ base\ }}{ base }}\label{functions-pow-base}

\href{/docs/reference/foundations/int/}{int} {or}
\href{/docs/reference/foundations/float/}{float} {or}
\href{/docs/reference/foundations/decimal/}{decimal}

{Required} {{ Positional }}

\phantomsection\label{functions-pow-base-positional-tooltip}
Positional parameters are specified in order, without names.

The base of the power.

If this is a
\href{/docs/reference/foundations/decimal/}{\texttt{\ decimal\ }} , the
exponent can only be an \href{/docs/reference/foundations/int/}{integer}
.

\paragraph{\texorpdfstring{\texttt{\ exponent\ }}{ exponent }}\label{functions-pow-exponent}

\href{/docs/reference/foundations/int/}{int} {or}
\href{/docs/reference/foundations/float/}{float}

{Required} {{ Positional }}

\phantomsection\label{functions-pow-exponent-positional-tooltip}
Positional parameters are specified in order, without names.

The exponent of the power.

\subsubsection{\texorpdfstring{\texttt{\ exp\ }}{ exp }}\label{functions-exp}

Raises a value to some exponent of e.

calc { . } { exp } (

{ \href{/docs/reference/foundations/int/}{int}
\href{/docs/reference/foundations/float/}{float} }

) -\textgreater{} \href{/docs/reference/foundations/float/}{float}

\begin{verbatim}
#calc.exp(1)
\end{verbatim}

\includegraphics[width=5in,height=\textheight,keepaspectratio]{/assets/docs/D3jiA5mgoQIx6MVn4Oy4zwAAAAAAAAAA.png}

\paragraph{\texorpdfstring{\texttt{\ exponent\ }}{ exponent }}\label{functions-exp-exponent}

\href{/docs/reference/foundations/int/}{int} {or}
\href{/docs/reference/foundations/float/}{float}

{Required} {{ Positional }}

\phantomsection\label{functions-exp-exponent-positional-tooltip}
Positional parameters are specified in order, without names.

The exponent of the power.

\subsubsection{\texorpdfstring{\texttt{\ sqrt\ }}{ sqrt }}\label{functions-sqrt}

Calculates the square root of a number.

calc { . } { sqrt } (

{ \href{/docs/reference/foundations/int/}{int}
\href{/docs/reference/foundations/float/}{float} }

) -\textgreater{} \href{/docs/reference/foundations/float/}{float}

\begin{verbatim}
#calc.sqrt(16) \
#calc.sqrt(2.5)
\end{verbatim}

\includegraphics[width=5in,height=\textheight,keepaspectratio]{/assets/docs/rSjz1bWkkKYqxmWjxezFTwAAAAAAAAAA.png}

\paragraph{\texorpdfstring{\texttt{\ value\ }}{ value }}\label{functions-sqrt-value}

\href{/docs/reference/foundations/int/}{int} {or}
\href{/docs/reference/foundations/float/}{float}

{Required} {{ Positional }}

\phantomsection\label{functions-sqrt-value-positional-tooltip}
Positional parameters are specified in order, without names.

The number whose square root to calculate. Must be non-negative.

\subsubsection{\texorpdfstring{\texttt{\ root\ }}{ root }}\label{functions-root}

Calculates the real nth root of a number.

If the number is negative, then n must be odd.

calc { . } { root } (

{ \href{/docs/reference/foundations/float/}{float} , } {
\href{/docs/reference/foundations/int/}{int} , }

) -\textgreater{} \href{/docs/reference/foundations/float/}{float}

\begin{verbatim}
#calc.root(16.0, 4) \
#calc.root(27.0, 3)
\end{verbatim}

\includegraphics[width=5in,height=\textheight,keepaspectratio]{/assets/docs/g3rxlqoTGgoCjLtiE7bcKAAAAAAAAAAA.png}

\paragraph{\texorpdfstring{\texttt{\ radicand\ }}{ radicand }}\label{functions-root-radicand}

\href{/docs/reference/foundations/float/}{float}

{Required} {{ Positional }}

\phantomsection\label{functions-root-radicand-positional-tooltip}
Positional parameters are specified in order, without names.

The expression to take the root of

\paragraph{\texorpdfstring{\texttt{\ index\ }}{ index }}\label{functions-root-index}

\href{/docs/reference/foundations/int/}{int}

{Required} {{ Positional }}

\phantomsection\label{functions-root-index-positional-tooltip}
Positional parameters are specified in order, without names.

Which root of the radicand to take

\subsubsection{\texorpdfstring{\texttt{\ sin\ }}{ sin }}\label{functions-sin}

Calculates the sine of an angle.

When called with an integer or a float, they will be interpreted as
radians.

calc { . } { sin } (

{ \href{/docs/reference/foundations/int/}{int}
\href{/docs/reference/foundations/float/}{float}
\href{/docs/reference/layout/angle/}{angle} }

) -\textgreater{} \href{/docs/reference/foundations/float/}{float}

\begin{verbatim}
#calc.sin(1.5) \
#calc.sin(90deg)
\end{verbatim}

\includegraphics[width=5in,height=\textheight,keepaspectratio]{/assets/docs/DRz-f64JvhHssm4WgSEL2QAAAAAAAAAA.png}

\paragraph{\texorpdfstring{\texttt{\ angle\ }}{ angle }}\label{functions-sin-angle}

\href{/docs/reference/foundations/int/}{int} {or}
\href{/docs/reference/foundations/float/}{float} {or}
\href{/docs/reference/layout/angle/}{angle}

{Required} {{ Positional }}

\phantomsection\label{functions-sin-angle-positional-tooltip}
Positional parameters are specified in order, without names.

The angle whose sine to calculate.

\subsubsection{\texorpdfstring{\texttt{\ cos\ }}{ cos }}\label{functions-cos}

Calculates the cosine of an angle.

When called with an integer or a float, they will be interpreted as
radians.

calc { . } { cos } (

{ \href{/docs/reference/foundations/int/}{int}
\href{/docs/reference/foundations/float/}{float}
\href{/docs/reference/layout/angle/}{angle} }

) -\textgreater{} \href{/docs/reference/foundations/float/}{float}

\begin{verbatim}
#calc.cos(1.5) \
#calc.cos(90deg)
\end{verbatim}

\includegraphics[width=5in,height=\textheight,keepaspectratio]{/assets/docs/RQec6gdJF5QvRwZkvI-50gAAAAAAAAAA.png}

\paragraph{\texorpdfstring{\texttt{\ angle\ }}{ angle }}\label{functions-cos-angle}

\href{/docs/reference/foundations/int/}{int} {or}
\href{/docs/reference/foundations/float/}{float} {or}
\href{/docs/reference/layout/angle/}{angle}

{Required} {{ Positional }}

\phantomsection\label{functions-cos-angle-positional-tooltip}
Positional parameters are specified in order, without names.

The angle whose cosine to calculate.

\subsubsection{\texorpdfstring{\texttt{\ tan\ }}{ tan }}\label{functions-tan}

Calculates the tangent of an angle.

When called with an integer or a float, they will be interpreted as
radians.

calc { . } { tan } (

{ \href{/docs/reference/foundations/int/}{int}
\href{/docs/reference/foundations/float/}{float}
\href{/docs/reference/layout/angle/}{angle} }

) -\textgreater{} \href{/docs/reference/foundations/float/}{float}

\begin{verbatim}
#calc.tan(1.5) \
#calc.tan(90deg)
\end{verbatim}

\includegraphics[width=5in,height=\textheight,keepaspectratio]{/assets/docs/Mu6UfN_4464KJhy78wvp_wAAAAAAAAAA.png}

\paragraph{\texorpdfstring{\texttt{\ angle\ }}{ angle }}\label{functions-tan-angle}

\href{/docs/reference/foundations/int/}{int} {or}
\href{/docs/reference/foundations/float/}{float} {or}
\href{/docs/reference/layout/angle/}{angle}

{Required} {{ Positional }}

\phantomsection\label{functions-tan-angle-positional-tooltip}
Positional parameters are specified in order, without names.

The angle whose tangent to calculate.

\subsubsection{\texorpdfstring{\texttt{\ asin\ }}{ asin }}\label{functions-asin}

Calculates the arcsine of a number.

calc { . } { asin } (

{ \href{/docs/reference/foundations/int/}{int}
\href{/docs/reference/foundations/float/}{float} }

) -\textgreater{} \href{/docs/reference/layout/angle/}{angle}

\begin{verbatim}
#calc.asin(0) \
#calc.asin(1)
\end{verbatim}

\includegraphics[width=5in,height=\textheight,keepaspectratio]{/assets/docs/R-fP2bsKqek6CrHRxsmlvQAAAAAAAAAA.png}

\paragraph{\texorpdfstring{\texttt{\ value\ }}{ value }}\label{functions-asin-value}

\href{/docs/reference/foundations/int/}{int} {or}
\href{/docs/reference/foundations/float/}{float}

{Required} {{ Positional }}

\phantomsection\label{functions-asin-value-positional-tooltip}
Positional parameters are specified in order, without names.

The number whose arcsine to calculate. Must be between -1 and 1.

\subsubsection{\texorpdfstring{\texttt{\ acos\ }}{ acos }}\label{functions-acos}

Calculates the arccosine of a number.

calc { . } { acos } (

{ \href{/docs/reference/foundations/int/}{int}
\href{/docs/reference/foundations/float/}{float} }

) -\textgreater{} \href{/docs/reference/layout/angle/}{angle}

\begin{verbatim}
#calc.acos(0) \
#calc.acos(1)
\end{verbatim}

\includegraphics[width=5in,height=\textheight,keepaspectratio]{/assets/docs/34tvtgPRx9Zb0oFQtdkEngAAAAAAAAAA.png}

\paragraph{\texorpdfstring{\texttt{\ value\ }}{ value }}\label{functions-acos-value}

\href{/docs/reference/foundations/int/}{int} {or}
\href{/docs/reference/foundations/float/}{float}

{Required} {{ Positional }}

\phantomsection\label{functions-acos-value-positional-tooltip}
Positional parameters are specified in order, without names.

The number whose arcsine to calculate. Must be between -1 and 1.

\subsubsection{\texorpdfstring{\texttt{\ atan\ }}{ atan }}\label{functions-atan}

Calculates the arctangent of a number.

calc { . } { atan } (

{ \href{/docs/reference/foundations/int/}{int}
\href{/docs/reference/foundations/float/}{float} }

) -\textgreater{} \href{/docs/reference/layout/angle/}{angle}

\begin{verbatim}
#calc.atan(0) \
#calc.atan(1)
\end{verbatim}

\includegraphics[width=5in,height=\textheight,keepaspectratio]{/assets/docs/Ks5iB4MwWXNAVXeSMihDJAAAAAAAAAAA.png}

\paragraph{\texorpdfstring{\texttt{\ value\ }}{ value }}\label{functions-atan-value}

\href{/docs/reference/foundations/int/}{int} {or}
\href{/docs/reference/foundations/float/}{float}

{Required} {{ Positional }}

\phantomsection\label{functions-atan-value-positional-tooltip}
Positional parameters are specified in order, without names.

The number whose arctangent to calculate.

\subsubsection{\texorpdfstring{\texttt{\ atan2\ }}{ atan2 }}\label{functions-atan2}

Calculates the four-quadrant arctangent of a coordinate.

The arguments are \texttt{\ (x,\ y)\ } , not \texttt{\ (y,\ x)\ } .

calc { . } { atan2 } (

{ \href{/docs/reference/foundations/int/}{int}
\href{/docs/reference/foundations/float/}{float} , } {
\href{/docs/reference/foundations/int/}{int}
\href{/docs/reference/foundations/float/}{float} , }

) -\textgreater{} \href{/docs/reference/layout/angle/}{angle}

\begin{verbatim}
#calc.atan2(1, 1) \
#calc.atan2(-2, -3)
\end{verbatim}

\includegraphics[width=5in,height=\textheight,keepaspectratio]{/assets/docs/R3PgftYITRsSLYBeQqKe3wAAAAAAAAAA.png}

\paragraph{\texorpdfstring{\texttt{\ x\ }}{ x }}\label{functions-atan2-x}

\href{/docs/reference/foundations/int/}{int} {or}
\href{/docs/reference/foundations/float/}{float}

{Required} {{ Positional }}

\phantomsection\label{functions-atan2-x-positional-tooltip}
Positional parameters are specified in order, without names.

The X coordinate.

\paragraph{\texorpdfstring{\texttt{\ y\ }}{ y }}\label{functions-atan2-y}

\href{/docs/reference/foundations/int/}{int} {or}
\href{/docs/reference/foundations/float/}{float}

{Required} {{ Positional }}

\phantomsection\label{functions-atan2-y-positional-tooltip}
Positional parameters are specified in order, without names.

The Y coordinate.

\subsubsection{\texorpdfstring{\texttt{\ sinh\ }}{ sinh }}\label{functions-sinh}

Calculates the hyperbolic sine of a hyperbolic angle.

calc { . } { sinh } (

{ \href{/docs/reference/foundations/float/}{float} }

) -\textgreater{} \href{/docs/reference/foundations/float/}{float}

\begin{verbatim}
#calc.sinh(0) \
#calc.sinh(1.5)
\end{verbatim}

\includegraphics[width=5in,height=\textheight,keepaspectratio]{/assets/docs/Si7LVr220y-yjr6frD5mYQAAAAAAAAAA.png}

\paragraph{\texorpdfstring{\texttt{\ value\ }}{ value }}\label{functions-sinh-value}

\href{/docs/reference/foundations/float/}{float}

{Required} {{ Positional }}

\phantomsection\label{functions-sinh-value-positional-tooltip}
Positional parameters are specified in order, without names.

The hyperbolic angle whose hyperbolic sine to calculate.

\subsubsection{\texorpdfstring{\texttt{\ cosh\ }}{ cosh }}\label{functions-cosh}

Calculates the hyperbolic cosine of a hyperbolic angle.

calc { . } { cosh } (

{ \href{/docs/reference/foundations/float/}{float} }

) -\textgreater{} \href{/docs/reference/foundations/float/}{float}

\begin{verbatim}
#calc.cosh(0) \
#calc.cosh(1.5)
\end{verbatim}

\includegraphics[width=5in,height=\textheight,keepaspectratio]{/assets/docs/Vut_ujHW8enJAdOI95v6bgAAAAAAAAAA.png}

\paragraph{\texorpdfstring{\texttt{\ value\ }}{ value }}\label{functions-cosh-value}

\href{/docs/reference/foundations/float/}{float}

{Required} {{ Positional }}

\phantomsection\label{functions-cosh-value-positional-tooltip}
Positional parameters are specified in order, without names.

The hyperbolic angle whose hyperbolic cosine to calculate.

\subsubsection{\texorpdfstring{\texttt{\ tanh\ }}{ tanh }}\label{functions-tanh}

Calculates the hyperbolic tangent of an hyperbolic angle.

calc { . } { tanh } (

{ \href{/docs/reference/foundations/float/}{float} }

) -\textgreater{} \href{/docs/reference/foundations/float/}{float}

\begin{verbatim}
#calc.tanh(0) \
#calc.tanh(1.5)
\end{verbatim}

\includegraphics[width=5in,height=\textheight,keepaspectratio]{/assets/docs/8omHKWMEXh9ltcsWpm4RDQAAAAAAAAAA.png}

\paragraph{\texorpdfstring{\texttt{\ value\ }}{ value }}\label{functions-tanh-value}

\href{/docs/reference/foundations/float/}{float}

{Required} {{ Positional }}

\phantomsection\label{functions-tanh-value-positional-tooltip}
Positional parameters are specified in order, without names.

The hyperbolic angle whose hyperbolic tangent to calculate.

\subsubsection{\texorpdfstring{\texttt{\ log\ }}{ log }}\label{functions-log}

Calculates the logarithm of a number.

If the base is not specified, the logarithm is calculated in base 10.

calc { . } { log } (

{ \href{/docs/reference/foundations/int/}{int}
\href{/docs/reference/foundations/float/}{float} , } {
\hyperref[functions-log-parameters-base]{base :}
\href{/docs/reference/foundations/float/}{float} , }

) -\textgreater{} \href{/docs/reference/foundations/float/}{float}

\begin{verbatim}
#calc.log(100)
\end{verbatim}

\includegraphics[width=5in,height=\textheight,keepaspectratio]{/assets/docs/4te-fP3EFYf9CFfXTNeLbgAAAAAAAAAA.png}

\paragraph{\texorpdfstring{\texttt{\ value\ }}{ value }}\label{functions-log-value}

\href{/docs/reference/foundations/int/}{int} {or}
\href{/docs/reference/foundations/float/}{float}

{Required} {{ Positional }}

\phantomsection\label{functions-log-value-positional-tooltip}
Positional parameters are specified in order, without names.

The number whose logarithm to calculate. Must be strictly positive.

\paragraph{\texorpdfstring{\texttt{\ base\ }}{ base }}\label{functions-log-base}

\href{/docs/reference/foundations/float/}{float}

The base of the logarithm. May not be zero.

Default: \texttt{\ }{\texttt{\ 10.0\ }}\texttt{\ }

\subsubsection{\texorpdfstring{\texttt{\ ln\ }}{ ln }}\label{functions-ln}

Calculates the natural logarithm of a number.

calc { . } { ln } (

{ \href{/docs/reference/foundations/int/}{int}
\href{/docs/reference/foundations/float/}{float} }

) -\textgreater{} \href{/docs/reference/foundations/float/}{float}

\begin{verbatim}
#calc.ln(calc.e)
\end{verbatim}

\includegraphics[width=5in,height=\textheight,keepaspectratio]{/assets/docs/ahMgc30uVaXMdJx4f9b76gAAAAAAAAAA.png}

\paragraph{\texorpdfstring{\texttt{\ value\ }}{ value }}\label{functions-ln-value}

\href{/docs/reference/foundations/int/}{int} {or}
\href{/docs/reference/foundations/float/}{float}

{Required} {{ Positional }}

\phantomsection\label{functions-ln-value-positional-tooltip}
Positional parameters are specified in order, without names.

The number whose logarithm to calculate. Must be strictly positive.

\subsubsection{\texorpdfstring{\texttt{\ fact\ }}{ fact }}\label{functions-fact}

Calculates the factorial of a number.

calc { . } { fact } (

{ \href{/docs/reference/foundations/int/}{int} }

) -\textgreater{} \href{/docs/reference/foundations/int/}{int}

\begin{verbatim}
#calc.fact(5)
\end{verbatim}

\includegraphics[width=5in,height=\textheight,keepaspectratio]{/assets/docs/Hx0vydXttNRUJbbdDSvGlwAAAAAAAAAA.png}

\paragraph{\texorpdfstring{\texttt{\ number\ }}{ number }}\label{functions-fact-number}

\href{/docs/reference/foundations/int/}{int}

{Required} {{ Positional }}

\phantomsection\label{functions-fact-number-positional-tooltip}
Positional parameters are specified in order, without names.

The number whose factorial to calculate. Must be non-negative.

\subsubsection{\texorpdfstring{\texttt{\ perm\ }}{ perm }}\label{functions-perm}

Calculates a permutation.

Returns the \texttt{\ k\ } -permutation of \texttt{\ n\ } , or the
number of ways to choose \texttt{\ k\ } items from a set of
\texttt{\ n\ } with regard to order.

calc { . } { perm } (

{ \href{/docs/reference/foundations/int/}{int} , } {
\href{/docs/reference/foundations/int/}{int} , }

) -\textgreater{} \href{/docs/reference/foundations/int/}{int}

\begin{verbatim}
$ "perm"(n, k) &= n!/((n - k)!) \
  "perm"(5, 3) &= #calc.perm(5, 3) $
\end{verbatim}

\includegraphics[width=5in,height=\textheight,keepaspectratio]{/assets/docs/7mAf4sPmhe6rKKzamBE-iAAAAAAAAAAA.png}

\paragraph{\texorpdfstring{\texttt{\ base\ }}{ base }}\label{functions-perm-base}

\href{/docs/reference/foundations/int/}{int}

{Required} {{ Positional }}

\phantomsection\label{functions-perm-base-positional-tooltip}
Positional parameters are specified in order, without names.

The base number. Must be non-negative.

\paragraph{\texorpdfstring{\texttt{\ numbers\ }}{ numbers }}\label{functions-perm-numbers}

\href{/docs/reference/foundations/int/}{int}

{Required} {{ Positional }}

\phantomsection\label{functions-perm-numbers-positional-tooltip}
Positional parameters are specified in order, without names.

The number of permutations. Must be non-negative.

\subsubsection{\texorpdfstring{\texttt{\ binom\ }}{ binom }}\label{functions-binom}

Calculates a binomial coefficient.

Returns the \texttt{\ k\ } -combination of \texttt{\ n\ } , or the
number of ways to choose \texttt{\ k\ } items from a set of
\texttt{\ n\ } without regard to order.

calc { . } { binom } (

{ \href{/docs/reference/foundations/int/}{int} , } {
\href{/docs/reference/foundations/int/}{int} , }

) -\textgreater{} \href{/docs/reference/foundations/int/}{int}

\begin{verbatim}
#calc.binom(10, 5)
\end{verbatim}

\includegraphics[width=5in,height=\textheight,keepaspectratio]{/assets/docs/3evQc1ME4eQqbzXhrmJ5lAAAAAAAAAAA.png}

\paragraph{\texorpdfstring{\texttt{\ n\ }}{ n }}\label{functions-binom-n}

\href{/docs/reference/foundations/int/}{int}

{Required} {{ Positional }}

\phantomsection\label{functions-binom-n-positional-tooltip}
Positional parameters are specified in order, without names.

The upper coefficient. Must be non-negative.

\paragraph{\texorpdfstring{\texttt{\ k\ }}{ k }}\label{functions-binom-k}

\href{/docs/reference/foundations/int/}{int}

{Required} {{ Positional }}

\phantomsection\label{functions-binom-k-positional-tooltip}
Positional parameters are specified in order, without names.

The lower coefficient. Must be non-negative.

\subsubsection{\texorpdfstring{\texttt{\ gcd\ }}{ gcd }}\label{functions-gcd}

Calculates the greatest common divisor of two integers.

calc { . } { gcd } (

{ \href{/docs/reference/foundations/int/}{int} , } {
\href{/docs/reference/foundations/int/}{int} , }

) -\textgreater{} \href{/docs/reference/foundations/int/}{int}

\begin{verbatim}
#calc.gcd(7, 42)
\end{verbatim}

\includegraphics[width=5in,height=\textheight,keepaspectratio]{/assets/docs/qOIwQyXpCnrSkONAOnIDgAAAAAAAAAAA.png}

\paragraph{\texorpdfstring{\texttt{\ a\ }}{ a }}\label{functions-gcd-a}

\href{/docs/reference/foundations/int/}{int}

{Required} {{ Positional }}

\phantomsection\label{functions-gcd-a-positional-tooltip}
Positional parameters are specified in order, without names.

The first integer.

\paragraph{\texorpdfstring{\texttt{\ b\ }}{ b }}\label{functions-gcd-b}

\href{/docs/reference/foundations/int/}{int}

{Required} {{ Positional }}

\phantomsection\label{functions-gcd-b-positional-tooltip}
Positional parameters are specified in order, without names.

The second integer.

\subsubsection{\texorpdfstring{\texttt{\ lcm\ }}{ lcm }}\label{functions-lcm}

Calculates the least common multiple of two integers.

calc { . } { lcm } (

{ \href{/docs/reference/foundations/int/}{int} , } {
\href{/docs/reference/foundations/int/}{int} , }

) -\textgreater{} \href{/docs/reference/foundations/int/}{int}

\begin{verbatim}
#calc.lcm(96, 13)
\end{verbatim}

\includegraphics[width=5in,height=\textheight,keepaspectratio]{/assets/docs/BsSZQG52_995RG9zgRumuAAAAAAAAAAA.png}

\paragraph{\texorpdfstring{\texttt{\ a\ }}{ a }}\label{functions-lcm-a}

\href{/docs/reference/foundations/int/}{int}

{Required} {{ Positional }}

\phantomsection\label{functions-lcm-a-positional-tooltip}
Positional parameters are specified in order, without names.

The first integer.

\paragraph{\texorpdfstring{\texttt{\ b\ }}{ b }}\label{functions-lcm-b}

\href{/docs/reference/foundations/int/}{int}

{Required} {{ Positional }}

\phantomsection\label{functions-lcm-b-positional-tooltip}
Positional parameters are specified in order, without names.

The second integer.

\subsubsection{\texorpdfstring{\texttt{\ floor\ }}{ floor }}\label{functions-floor}

Rounds a number down to the nearest integer.

If the number is already an integer, it is returned unchanged.

Note that this function will always return an
\href{/docs/reference/foundations/int/}{integer} , and will error if the
resulting \href{/docs/reference/foundations/float/}{\texttt{\ float\ }}
or \href{/docs/reference/foundations/decimal/}{\texttt{\ decimal\ }} is
larger than the maximum 64-bit signed integer or smaller than the
minimum for that type.

calc { . } { floor } (

{ \href{/docs/reference/foundations/int/}{int}
\href{/docs/reference/foundations/float/}{float}
\href{/docs/reference/foundations/decimal/}{decimal} }

) -\textgreater{} \href{/docs/reference/foundations/int/}{int}

\begin{verbatim}
#calc.floor(500.1)
#assert(calc.floor(3) == 3)
#assert(calc.floor(3.14) == 3)
#assert(calc.floor(decimal("-3.14")) == -4)
\end{verbatim}

\includegraphics[width=5in,height=\textheight,keepaspectratio]{/assets/docs/3pMWbIkij09wRgebD43VQgAAAAAAAAAA.png}

\paragraph{\texorpdfstring{\texttt{\ value\ }}{ value }}\label{functions-floor-value}

\href{/docs/reference/foundations/int/}{int} {or}
\href{/docs/reference/foundations/float/}{float} {or}
\href{/docs/reference/foundations/decimal/}{decimal}

{Required} {{ Positional }}

\phantomsection\label{functions-floor-value-positional-tooltip}
Positional parameters are specified in order, without names.

The number to round down.

\subsubsection{\texorpdfstring{\texttt{\ ceil\ }}{ ceil }}\label{functions-ceil}

Rounds a number up to the nearest integer.

If the number is already an integer, it is returned unchanged.

Note that this function will always return an
\href{/docs/reference/foundations/int/}{integer} , and will error if the
resulting \href{/docs/reference/foundations/float/}{\texttt{\ float\ }}
or \href{/docs/reference/foundations/decimal/}{\texttt{\ decimal\ }} is
larger than the maximum 64-bit signed integer or smaller than the
minimum for that type.

calc { . } { ceil } (

{ \href{/docs/reference/foundations/int/}{int}
\href{/docs/reference/foundations/float/}{float}
\href{/docs/reference/foundations/decimal/}{decimal} }

) -\textgreater{} \href{/docs/reference/foundations/int/}{int}

\begin{verbatim}
#calc.ceil(500.1)
#assert(calc.ceil(3) == 3)
#assert(calc.ceil(3.14) == 4)
#assert(calc.ceil(decimal("-3.14")) == -3)
\end{verbatim}

\includegraphics[width=5in,height=\textheight,keepaspectratio]{/assets/docs/XVF6AbxDnXwmraGN-Eh1MgAAAAAAAAAA.png}

\paragraph{\texorpdfstring{\texttt{\ value\ }}{ value }}\label{functions-ceil-value}

\href{/docs/reference/foundations/int/}{int} {or}
\href{/docs/reference/foundations/float/}{float} {or}
\href{/docs/reference/foundations/decimal/}{decimal}

{Required} {{ Positional }}

\phantomsection\label{functions-ceil-value-positional-tooltip}
Positional parameters are specified in order, without names.

The number to round up.

\subsubsection{\texorpdfstring{\texttt{\ trunc\ }}{ trunc }}\label{functions-trunc}

Returns the integer part of a number.

If the number is already an integer, it is returned unchanged.

Note that this function will always return an
\href{/docs/reference/foundations/int/}{integer} , and will error if the
resulting \href{/docs/reference/foundations/float/}{\texttt{\ float\ }}
or \href{/docs/reference/foundations/decimal/}{\texttt{\ decimal\ }} is
larger than the maximum 64-bit signed integer or smaller than the
minimum for that type.

calc { . } { trunc } (

{ \href{/docs/reference/foundations/int/}{int}
\href{/docs/reference/foundations/float/}{float}
\href{/docs/reference/foundations/decimal/}{decimal} }

) -\textgreater{} \href{/docs/reference/foundations/int/}{int}

\begin{verbatim}
#calc.trunc(15.9)
#assert(calc.trunc(3) == 3)
#assert(calc.trunc(-3.7) == -3)
#assert(calc.trunc(decimal("8493.12949582390")) == 8493)
\end{verbatim}

\includegraphics[width=5in,height=\textheight,keepaspectratio]{/assets/docs/0ASdokWmhACxp3cbdBzSiwAAAAAAAAAA.png}

\paragraph{\texorpdfstring{\texttt{\ value\ }}{ value }}\label{functions-trunc-value}

\href{/docs/reference/foundations/int/}{int} {or}
\href{/docs/reference/foundations/float/}{float} {or}
\href{/docs/reference/foundations/decimal/}{decimal}

{Required} {{ Positional }}

\phantomsection\label{functions-trunc-value-positional-tooltip}
Positional parameters are specified in order, without names.

The number to truncate.

\subsubsection{\texorpdfstring{\texttt{\ fract\ }}{ fract }}\label{functions-fract}

Returns the fractional part of a number.

If the number is an integer, returns \texttt{\ 0\ } .

calc { . } { fract } (

{ \href{/docs/reference/foundations/int/}{int}
\href{/docs/reference/foundations/float/}{float}
\href{/docs/reference/foundations/decimal/}{decimal} }

) -\textgreater{} \href{/docs/reference/foundations/int/}{int}
\href{/docs/reference/foundations/float/}{float}
\href{/docs/reference/foundations/decimal/}{decimal}

\begin{verbatim}
#calc.fract(-3.1)
#assert(calc.fract(3) == 0)
#assert(calc.fract(decimal("234.23949211")) == decimal("0.23949211"))
\end{verbatim}

\includegraphics[width=5in,height=\textheight,keepaspectratio]{/assets/docs/3TGIWh2MEGFIDAB8C1nEQAAAAAAAAAAA.png}

\paragraph{\texorpdfstring{\texttt{\ value\ }}{ value }}\label{functions-fract-value}

\href{/docs/reference/foundations/int/}{int} {or}
\href{/docs/reference/foundations/float/}{float} {or}
\href{/docs/reference/foundations/decimal/}{decimal}

{Required} {{ Positional }}

\phantomsection\label{functions-fract-value-positional-tooltip}
Positional parameters are specified in order, without names.

The number to truncate.

\subsubsection{\texorpdfstring{\texttt{\ round\ }}{ round }}\label{functions-round}

Rounds a number to the nearest integer away from zero.

Optionally, a number of decimal places can be specified.

If the number of digits is negative, its absolute value will indicate
the amount of significant integer digits to remove before the decimal
point.

Note that this function will return the same type as the operand. That
is, applying \texttt{\ round\ } to a
\href{/docs/reference/foundations/float/}{\texttt{\ float\ }} will
return a \texttt{\ float\ } , and to a
\href{/docs/reference/foundations/decimal/}{\texttt{\ decimal\ }} ,
another \texttt{\ decimal\ } . You may explicitly convert the output of
this function to an integer with
\href{/docs/reference/foundations/int/}{\texttt{\ int\ }} , but note
that such a conversion will error if the \texttt{\ float\ } or
\texttt{\ decimal\ } is larger than the maximum 64-bit signed integer or
smaller than the minimum integer.

In addition, this function can error if there is an attempt to round
beyond the maximum or minimum integer or \texttt{\ decimal\ } . If the
number is a \texttt{\ float\ } , such an attempt will cause
\texttt{\ float\ }{\texttt{\ .\ }}\texttt{\ inf\ } or
\texttt{\ }{\texttt{\ -\ }}\texttt{\ float\ }{\texttt{\ .\ }}\texttt{\ inf\ }
to be returned for maximum and minimum respectively.

calc { . } { round } (

{ \href{/docs/reference/foundations/int/}{int}
\href{/docs/reference/foundations/float/}{float}
\href{/docs/reference/foundations/decimal/}{decimal} , } {
\hyperref[functions-round-parameters-digits]{digits :}
\href{/docs/reference/foundations/int/}{int} , }

) -\textgreater{} \href{/docs/reference/foundations/int/}{int}
\href{/docs/reference/foundations/float/}{float}
\href{/docs/reference/foundations/decimal/}{decimal}

\begin{verbatim}
#calc.round(3.1415, digits: 2)
#assert(calc.round(3) == 3)
#assert(calc.round(3.14) == 3)
#assert(calc.round(3.5) == 4.0)
#assert(calc.round(3333.45, digits: -2) == 3300.0)
#assert(calc.round(-48953.45, digits: -3) == -49000.0)
#assert(calc.round(3333, digits: -2) == 3300)
#assert(calc.round(-48953, digits: -3) == -49000)
#assert(calc.round(decimal("-6.5")) == decimal("-7"))
#assert(calc.round(decimal("7.123456789"), digits: 6) == decimal("7.123457"))
#assert(calc.round(decimal("3333.45"), digits: -2) == decimal("3300"))
#assert(calc.round(decimal("-48953.45"), digits: -3) == decimal("-49000"))
\end{verbatim}

\includegraphics[width=5in,height=\textheight,keepaspectratio]{/assets/docs/S2fXMNcPylTq6uwl7ZRpoAAAAAAAAAAA.png}

\paragraph{\texorpdfstring{\texttt{\ value\ }}{ value }}\label{functions-round-value}

\href{/docs/reference/foundations/int/}{int} {or}
\href{/docs/reference/foundations/float/}{float} {or}
\href{/docs/reference/foundations/decimal/}{decimal}

{Required} {{ Positional }}

\phantomsection\label{functions-round-value-positional-tooltip}
Positional parameters are specified in order, without names.

The number to round.

\paragraph{\texorpdfstring{\texttt{\ digits\ }}{ digits }}\label{functions-round-digits}

\href{/docs/reference/foundations/int/}{int}

If positive, the number of decimal places.

If negative, the number of significant integer digits that should be
removed before the decimal point.

Default: \texttt{\ }{\texttt{\ 0\ }}\texttt{\ }

\subsubsection{\texorpdfstring{\texttt{\ clamp\ }}{ clamp }}\label{functions-clamp}

Clamps a number between a minimum and maximum value.

calc { . } { clamp } (

{ \href{/docs/reference/foundations/int/}{int}
\href{/docs/reference/foundations/float/}{float}
\href{/docs/reference/foundations/decimal/}{decimal} , } {
\href{/docs/reference/foundations/int/}{int}
\href{/docs/reference/foundations/float/}{float}
\href{/docs/reference/foundations/decimal/}{decimal} , } {
\href{/docs/reference/foundations/int/}{int}
\href{/docs/reference/foundations/float/}{float}
\href{/docs/reference/foundations/decimal/}{decimal} , }

) -\textgreater{} \href{/docs/reference/foundations/int/}{int}
\href{/docs/reference/foundations/float/}{float}
\href{/docs/reference/foundations/decimal/}{decimal}

\begin{verbatim}
#calc.clamp(5, 0, 4)
#assert(calc.clamp(5, 0, 10) == 5)
#assert(calc.clamp(5, 6, 10) == 6)
#assert(calc.clamp(decimal("5.45"), 2, decimal("45.9")) == decimal("5.45"))
#assert(calc.clamp(decimal("5.45"), decimal("6.75"), 12) == decimal("6.75"))
\end{verbatim}

\includegraphics[width=5in,height=\textheight,keepaspectratio]{/assets/docs/IT7doIU2fH1UJf0E_SPc6QAAAAAAAAAA.png}

\paragraph{\texorpdfstring{\texttt{\ value\ }}{ value }}\label{functions-clamp-value}

\href{/docs/reference/foundations/int/}{int} {or}
\href{/docs/reference/foundations/float/}{float} {or}
\href{/docs/reference/foundations/decimal/}{decimal}

{Required} {{ Positional }}

\phantomsection\label{functions-clamp-value-positional-tooltip}
Positional parameters are specified in order, without names.

The number to clamp.

\paragraph{\texorpdfstring{\texttt{\ min\ }}{ min }}\label{functions-clamp-min}

\href{/docs/reference/foundations/int/}{int} {or}
\href{/docs/reference/foundations/float/}{float} {or}
\href{/docs/reference/foundations/decimal/}{decimal}

{Required} {{ Positional }}

\phantomsection\label{functions-clamp-min-positional-tooltip}
Positional parameters are specified in order, without names.

The inclusive minimum value.

\paragraph{\texorpdfstring{\texttt{\ max\ }}{ max }}\label{functions-clamp-max}

\href{/docs/reference/foundations/int/}{int} {or}
\href{/docs/reference/foundations/float/}{float} {or}
\href{/docs/reference/foundations/decimal/}{decimal}

{Required} {{ Positional }}

\phantomsection\label{functions-clamp-max-positional-tooltip}
Positional parameters are specified in order, without names.

The inclusive maximum value.

\subsubsection{\texorpdfstring{\texttt{\ min\ }}{ min }}\label{functions-min}

Determines the minimum of a sequence of values.

calc { . } { min } (

{ \hyperref[functions-min-parameters-values]{..} { any } }

) -\textgreater{} { any }

\begin{verbatim}
#calc.min(1, -3, -5, 20, 3, 6) \
#calc.min("typst", "is", "cool")
\end{verbatim}

\includegraphics[width=5in,height=\textheight,keepaspectratio]{/assets/docs/afOSrjdOAc_1RzzU2hxUIgAAAAAAAAAA.png}

\paragraph{\texorpdfstring{\texttt{\ values\ }}{ values }}\label{functions-min-values}

{ any }

{Required} {{ Positional }}

\phantomsection\label{functions-min-values-positional-tooltip}
Positional parameters are specified in order, without names.

{{ Variadic }}

\phantomsection\label{functions-min-values-variadic-tooltip}
Variadic parameters can be specified multiple times.

The sequence of values from which to extract the minimum. Must not be
empty.

\subsubsection{\texorpdfstring{\texttt{\ max\ }}{ max }}\label{functions-max}

Determines the maximum of a sequence of values.

calc { . } { max } (

{ \hyperref[functions-max-parameters-values]{..} { any } }

) -\textgreater{} { any }

\begin{verbatim}
#calc.max(1, -3, -5, 20, 3, 6) \
#calc.max("typst", "is", "cool")
\end{verbatim}

\includegraphics[width=5in,height=\textheight,keepaspectratio]{/assets/docs/B8vbsVaOK7Ilt-aRhfDiFwAAAAAAAAAA.png}

\paragraph{\texorpdfstring{\texttt{\ values\ }}{ values }}\label{functions-max-values}

{ any }

{Required} {{ Positional }}

\phantomsection\label{functions-max-values-positional-tooltip}
Positional parameters are specified in order, without names.

{{ Variadic }}

\phantomsection\label{functions-max-values-variadic-tooltip}
Variadic parameters can be specified multiple times.

The sequence of values from which to extract the maximum. Must not be
empty.

\subsubsection{\texorpdfstring{\texttt{\ even\ }}{ even }}\label{functions-even}

Determines whether an integer is even.

calc { . } { even } (

{ \href{/docs/reference/foundations/int/}{int} }

) -\textgreater{} \href{/docs/reference/foundations/bool/}{bool}

\begin{verbatim}
#calc.even(4) \
#calc.even(5) \
#range(10).filter(calc.even)
\end{verbatim}

\includegraphics[width=5in,height=\textheight,keepaspectratio]{/assets/docs/YVF-q96_WeIoAbwweursAAAAAAAAAAAA.png}

\paragraph{\texorpdfstring{\texttt{\ value\ }}{ value }}\label{functions-even-value}

\href{/docs/reference/foundations/int/}{int}

{Required} {{ Positional }}

\phantomsection\label{functions-even-value-positional-tooltip}
Positional parameters are specified in order, without names.

The number to check for evenness.

\subsubsection{\texorpdfstring{\texttt{\ odd\ }}{ odd }}\label{functions-odd}

Determines whether an integer is odd.

calc { . } { odd } (

{ \href{/docs/reference/foundations/int/}{int} }

) -\textgreater{} \href{/docs/reference/foundations/bool/}{bool}

\begin{verbatim}
#calc.odd(4) \
#calc.odd(5) \
#range(10).filter(calc.odd)
\end{verbatim}

\includegraphics[width=5in,height=\textheight,keepaspectratio]{/assets/docs/54xiVFQnQ9FIdgInF0A_jAAAAAAAAAAA.png}

\paragraph{\texorpdfstring{\texttt{\ value\ }}{ value }}\label{functions-odd-value}

\href{/docs/reference/foundations/int/}{int}

{Required} {{ Positional }}

\phantomsection\label{functions-odd-value-positional-tooltip}
Positional parameters are specified in order, without names.

The number to check for oddness.

\subsubsection{\texorpdfstring{\texttt{\ rem\ }}{ rem }}\label{functions-rem}

Calculates the remainder of two numbers.

The value \texttt{\ calc.rem(x,\ y)\ } always has the same sign as
\texttt{\ x\ } , and is smaller in magnitude than \texttt{\ y\ } .

This can error if given a
\href{/docs/reference/foundations/decimal/}{\texttt{\ decimal\ }} input
and the dividend is too small in magnitude compared to the divisor.

calc { . } { rem } (

{ \href{/docs/reference/foundations/int/}{int}
\href{/docs/reference/foundations/float/}{float}
\href{/docs/reference/foundations/decimal/}{decimal} , } {
\href{/docs/reference/foundations/int/}{int}
\href{/docs/reference/foundations/float/}{float}
\href{/docs/reference/foundations/decimal/}{decimal} , }

) -\textgreater{} \href{/docs/reference/foundations/int/}{int}
\href{/docs/reference/foundations/float/}{float}
\href{/docs/reference/foundations/decimal/}{decimal}

\begin{verbatim}
#calc.rem(7, 3) \
#calc.rem(7, -3) \
#calc.rem(-7, 3) \
#calc.rem(-7, -3) \
#calc.rem(1.75, 0.5)
\end{verbatim}

\includegraphics[width=5in,height=\textheight,keepaspectratio]{/assets/docs/h9kAd8BZ_4qaZUm7WWIpgQAAAAAAAAAA.png}

\paragraph{\texorpdfstring{\texttt{\ dividend\ }}{ dividend }}\label{functions-rem-dividend}

\href{/docs/reference/foundations/int/}{int} {or}
\href{/docs/reference/foundations/float/}{float} {or}
\href{/docs/reference/foundations/decimal/}{decimal}

{Required} {{ Positional }}

\phantomsection\label{functions-rem-dividend-positional-tooltip}
Positional parameters are specified in order, without names.

The dividend of the remainder.

\paragraph{\texorpdfstring{\texttt{\ divisor\ }}{ divisor }}\label{functions-rem-divisor}

\href{/docs/reference/foundations/int/}{int} {or}
\href{/docs/reference/foundations/float/}{float} {or}
\href{/docs/reference/foundations/decimal/}{decimal}

{Required} {{ Positional }}

\phantomsection\label{functions-rem-divisor-positional-tooltip}
Positional parameters are specified in order, without names.

The divisor of the remainder.

\subsubsection{\texorpdfstring{\texttt{\ div-euclid\ }}{ div-euclid }}\label{functions-div-euclid}

Performs euclidean division of two numbers.

The result of this computation is that of a division rounded to the
integer \texttt{\ n\ } such that the dividend is greater than or equal
to \texttt{\ n\ } times the divisor.

calc { . } { div-euclid } (

{ \href{/docs/reference/foundations/int/}{int}
\href{/docs/reference/foundations/float/}{float}
\href{/docs/reference/foundations/decimal/}{decimal} , } {
\href{/docs/reference/foundations/int/}{int}
\href{/docs/reference/foundations/float/}{float}
\href{/docs/reference/foundations/decimal/}{decimal} , }

) -\textgreater{} \href{/docs/reference/foundations/int/}{int}
\href{/docs/reference/foundations/float/}{float}
\href{/docs/reference/foundations/decimal/}{decimal}

\begin{verbatim}
#calc.div-euclid(7, 3) \
#calc.div-euclid(7, -3) \
#calc.div-euclid(-7, 3) \
#calc.div-euclid(-7, -3) \
#calc.div-euclid(1.75, 0.5) \
#calc.div-euclid(decimal("1.75"), decimal("0.5"))
\end{verbatim}

\includegraphics[width=5in,height=\textheight,keepaspectratio]{/assets/docs/496IGVvoarlmERajiTFs_gAAAAAAAAAA.png}

\paragraph{\texorpdfstring{\texttt{\ dividend\ }}{ dividend }}\label{functions-div-euclid-dividend}

\href{/docs/reference/foundations/int/}{int} {or}
\href{/docs/reference/foundations/float/}{float} {or}
\href{/docs/reference/foundations/decimal/}{decimal}

{Required} {{ Positional }}

\phantomsection\label{functions-div-euclid-dividend-positional-tooltip}
Positional parameters are specified in order, without names.

The dividend of the division.

\paragraph{\texorpdfstring{\texttt{\ divisor\ }}{ divisor }}\label{functions-div-euclid-divisor}

\href{/docs/reference/foundations/int/}{int} {or}
\href{/docs/reference/foundations/float/}{float} {or}
\href{/docs/reference/foundations/decimal/}{decimal}

{Required} {{ Positional }}

\phantomsection\label{functions-div-euclid-divisor-positional-tooltip}
Positional parameters are specified in order, without names.

The divisor of the division.

\subsubsection{\texorpdfstring{\texttt{\ rem-euclid\ }}{ rem-euclid }}\label{functions-rem-euclid}

This calculates the least nonnegative remainder of a division.

Warning: Due to a floating point round-off error, the remainder may
equal the absolute value of the divisor if the dividend is much smaller
in magnitude than the divisor and the dividend is negative. This only
applies for floating point inputs.

In addition, this can error if given a
\href{/docs/reference/foundations/decimal/}{\texttt{\ decimal\ }} input
and the dividend is too small in magnitude compared to the divisor.

calc { . } { rem-euclid } (

{ \href{/docs/reference/foundations/int/}{int}
\href{/docs/reference/foundations/float/}{float}
\href{/docs/reference/foundations/decimal/}{decimal} , } {
\href{/docs/reference/foundations/int/}{int}
\href{/docs/reference/foundations/float/}{float}
\href{/docs/reference/foundations/decimal/}{decimal} , }

) -\textgreater{} \href{/docs/reference/foundations/int/}{int}
\href{/docs/reference/foundations/float/}{float}
\href{/docs/reference/foundations/decimal/}{decimal}

\begin{verbatim}
#calc.rem-euclid(7, 3) \
#calc.rem-euclid(7, -3) \
#calc.rem-euclid(-7, 3) \
#calc.rem-euclid(-7, -3) \
#calc.rem-euclid(1.75, 0.5) \
#calc.rem-euclid(decimal("1.75"), decimal("0.5"))
\end{verbatim}

\includegraphics[width=5in,height=\textheight,keepaspectratio]{/assets/docs/ysX2HLC-rfWACinwigYcWgAAAAAAAAAA.png}

\paragraph{\texorpdfstring{\texttt{\ dividend\ }}{ dividend }}\label{functions-rem-euclid-dividend}

\href{/docs/reference/foundations/int/}{int} {or}
\href{/docs/reference/foundations/float/}{float} {or}
\href{/docs/reference/foundations/decimal/}{decimal}

{Required} {{ Positional }}

\phantomsection\label{functions-rem-euclid-dividend-positional-tooltip}
Positional parameters are specified in order, without names.

The dividend of the remainder.

\paragraph{\texorpdfstring{\texttt{\ divisor\ }}{ divisor }}\label{functions-rem-euclid-divisor}

\href{/docs/reference/foundations/int/}{int} {or}
\href{/docs/reference/foundations/float/}{float} {or}
\href{/docs/reference/foundations/decimal/}{decimal}

{Required} {{ Positional }}

\phantomsection\label{functions-rem-euclid-divisor-positional-tooltip}
Positional parameters are specified in order, without names.

The divisor of the remainder.

\subsubsection{\texorpdfstring{\texttt{\ quo\ }}{ quo }}\label{functions-quo}

Calculates the quotient (floored division) of two numbers.

Note that this function will always return an
\href{/docs/reference/foundations/int/}{integer} , and will error if the
resulting \href{/docs/reference/foundations/float/}{\texttt{\ float\ }}
or \href{/docs/reference/foundations/decimal/}{\texttt{\ decimal\ }} is
larger than the maximum 64-bit signed integer or smaller than the
minimum for that type.

calc { . } { quo } (

{ \href{/docs/reference/foundations/int/}{int}
\href{/docs/reference/foundations/float/}{float}
\href{/docs/reference/foundations/decimal/}{decimal} , } {
\href{/docs/reference/foundations/int/}{int}
\href{/docs/reference/foundations/float/}{float}
\href{/docs/reference/foundations/decimal/}{decimal} , }

) -\textgreater{} \href{/docs/reference/foundations/int/}{int}

\begin{verbatim}
$ "quo"(a, b) &= floor(a/b) \
  "quo"(14, 5) &= #calc.quo(14, 5) \
  "quo"(3.46, 0.5) &= #calc.quo(3.46, 0.5) $
\end{verbatim}

\includegraphics[width=5in,height=\textheight,keepaspectratio]{/assets/docs/AEhIvOjgCcBZo0GMCLQ9tQAAAAAAAAAA.png}

\paragraph{\texorpdfstring{\texttt{\ dividend\ }}{ dividend }}\label{functions-quo-dividend}

\href{/docs/reference/foundations/int/}{int} {or}
\href{/docs/reference/foundations/float/}{float} {or}
\href{/docs/reference/foundations/decimal/}{decimal}

{Required} {{ Positional }}

\phantomsection\label{functions-quo-dividend-positional-tooltip}
Positional parameters are specified in order, without names.

The dividend of the quotient.

\paragraph{\texorpdfstring{\texttt{\ divisor\ }}{ divisor }}\label{functions-quo-divisor}

\href{/docs/reference/foundations/int/}{int} {or}
\href{/docs/reference/foundations/float/}{float} {or}
\href{/docs/reference/foundations/decimal/}{decimal}

{Required} {{ Positional }}

\phantomsection\label{functions-quo-divisor-positional-tooltip}
Positional parameters are specified in order, without names.

The divisor of the quotient.

\href{/docs/reference/foundations/bytes/}{\pandocbounded{\includesvg[keepaspectratio]{/assets/icons/16-arrow-right.svg}}}

{ Bytes } { Previous page }

\href{/docs/reference/foundations/content/}{\pandocbounded{\includesvg[keepaspectratio]{/assets/icons/16-arrow-right.svg}}}

{ Content } { Next page }


\title{typst.app/docs/reference/foundations/function}

\begin{itemize}
\tightlist
\item
  \href{/docs}{\includesvg[width=0.16667in,height=0.16667in]{/assets/icons/16-docs-dark.svg}}
\item
  \includesvg[width=0.16667in,height=0.16667in]{/assets/icons/16-arrow-right.svg}
\item
  \href{/docs/reference/}{Reference}
\item
  \includesvg[width=0.16667in,height=0.16667in]{/assets/icons/16-arrow-right.svg}
\item
  \href{/docs/reference/foundations/}{Foundations}
\item
  \includesvg[width=0.16667in,height=0.16667in]{/assets/icons/16-arrow-right.svg}
\item
  \href{/docs/reference/foundations/function/}{Function}
\end{itemize}

\section{\texorpdfstring{{ function }}{ function }}\label{summary}

A mapping from argument values to a return value.

You can call a function by writing a comma-separated list of function
\emph{arguments} enclosed in parentheses directly after the function
name. Additionally, you can pass any number of trailing content blocks
arguments to a function \emph{after} the normal argument list. If the
normal argument list would become empty, it can be omitted. Typst
supports positional and named arguments. The former are identified by
position and type, while the latter are written as
\texttt{\ name:\ value\ } .

Within math mode, function calls have special behaviour. See the
\href{/docs/reference/math/}{math documentation} for more details.

\subsection{Example}\label{example}

\begin{verbatim}
// Call a function.
#list([A], [B])

// Named arguments and trailing
// content blocks.
#enum(start: 2)[A][B]

// Version without parentheses.
#list[A][B]
\end{verbatim}

\includegraphics[width=5in,height=\textheight,keepaspectratio]{/assets/docs/h8ulslRsTYE05Pu4qy5C6AAAAAAAAAAA.png}

Functions are a fundamental building block of Typst. Typst provides
functions for a variety of typesetting tasks. Moreover, the markup you
write is backed by functions and all styling happens through functions.
This reference lists all available functions and how you can use them.
Please also refer to the documentation about
\href{/docs/reference/styling/\#set-rules}{set} and
\href{/docs/reference/styling/\#show-rules}{show} rules to learn about
additional ways you can work with functions in Typst.

\subsection{Element functions}\label{element-functions}

Some functions are associated with \emph{elements} like
\href{/docs/reference/model/heading/}{headings} or
\href{/docs/reference/model/table/}{tables} . When called, these create
an element of their respective kind. In contrast to normal functions,
they can further be used in
\href{/docs/reference/styling/\#set-rules}{set rules} ,
\href{/docs/reference/styling/\#show-rules}{show rules} , and
\href{/docs/reference/foundations/selector/}{selectors} .

\subsection{Function scopes}\label{function-scopes}

Functions can hold related definitions in their own scope, similar to a
\href{/docs/reference/scripting/\#modules}{module} . Examples of this
are
\href{/docs/reference/foundations/assert/\#definitions-eq}{\texttt{\ assert.eq\ }}
or
\href{/docs/reference/model/list/\#definitions-item}{\texttt{\ list.item\ }}
. However, this feature is currently only available for built-in
functions.

\subsection{Defining functions}\label{defining-functions}

You can define your own function with a
\href{/docs/reference/scripting/\#bindings}{let binding} that has a
parameter list after the binding\textquotesingle s name. The parameter
list can contain mandatory positional parameters, named parameters with
default values and
\href{/docs/reference/foundations/arguments/}{argument sinks} .

The right-hand side of a function binding is the function body, which
can be a block or any other expression. It defines the
function\textquotesingle s return value and can depend on the
parameters. If the function body is a
\href{/docs/reference/scripting/\#blocks}{code block} , the return value
is the result of joining the values of each expression in the block.

Within a function body, the \texttt{\ return\ } keyword can be used to
exit early and optionally specify a return value. If no explicit return
value is given, the body evaluates to the result of joining all
expressions preceding the \texttt{\ return\ } .

Functions that don\textquotesingle t return any meaningful value return
\href{/docs/reference/foundations/none/}{\texttt{\ none\ }} instead. The
return type of such functions is not explicitly specified in the
documentation. (An example of this is
\href{/docs/reference/foundations/array/\#definitions-push}{\texttt{\ array.push\ }}
).

\begin{verbatim}
#let alert(body, fill: red) = {
  set text(white)
  set align(center)
  rect(
    fill: fill,
    inset: 8pt,
    radius: 4pt,
    [*Warning:\ #body*],
  )
}

#alert[
  Danger is imminent!
]

#alert(fill: blue)[
  KEEP OFF TRACKS
]
\end{verbatim}

\includegraphics[width=5in,height=\textheight,keepaspectratio]{/assets/docs/56wK4TQtzRt7_B3OQOxb7QAAAAAAAAAA.png}

\subsection{Importing functions}\label{importing-functions}

Functions can be imported from one file (
\href{/docs/reference/scripting/\#modules}{\texttt{\ module\ }} ) into
another using \texttt{\ }{\texttt{\ import\ }}\texttt{\ } . For example,
assume that we have defined the \texttt{\ alert\ } function from the
previous example in a file called \texttt{\ foo.typ\ } . We can import
it into another file by writing
\texttt{\ }{\texttt{\ import\ }}\texttt{\ }{\texttt{\ "foo.typ"\ }}\texttt{\ }{\texttt{\ :\ }}\texttt{\ alert\ }
.

\subsection{Unnamed functions}\label{unnamed}

You can also created an unnamed function without creating a binding by
specifying a parameter list followed by \texttt{\ =\textgreater{}\ } and
the function body. If your function has just one parameter, the
parentheses around the parameter list are optional. Unnamed functions
are mainly useful for show rules, but also for settable properties that
take functions like the page function\textquotesingle s
\href{/docs/reference/layout/page/\#parameters-footer}{\texttt{\ footer\ }}
property.

\begin{verbatim}
#show "once?": it => [#it #it]
once?
\end{verbatim}

\includegraphics[width=5in,height=\textheight,keepaspectratio]{/assets/docs/yXee-w_McX_Uho7Ghovc-QAAAAAAAAAA.png}

\subsection{Note on function purity}\label{note-on-function-purity}

In Typst, all functions are \emph{pure.} This means that for the same
arguments, they always return the same result. They cannot "remember"
things to produce another value when they are called a second time.

The only exception are built-in methods like
\href{/docs/reference/foundations/array/\#definitions-push}{\texttt{\ array.push(value)\ }}
. These can modify the values they are called on.

\subsection{\texorpdfstring{{ Definitions
}}{ Definitions }}\label{definitions}

\phantomsection\label{definitions-tooltip}
Functions and types and can have associated definitions. These are
accessed by specifying the function or type, followed by a period, and
then the definition\textquotesingle s name.

\subsubsection{\texorpdfstring{\texttt{\ with\ }}{ with }}\label{definitions-with}

Returns a new function that has the given arguments pre-applied.

self { . } { with } (

{ \hyperref[definitions-with-parameters-arguments]{..} { any } }

) -\textgreater{} \href{/docs/reference/foundations/function/}{function}

\paragraph{\texorpdfstring{\texttt{\ arguments\ }}{ arguments }}\label{definitions-with-arguments}

{ any }

{Required} {{ Positional }}

\phantomsection\label{definitions-with-arguments-positional-tooltip}
Positional parameters are specified in order, without names.

{{ Variadic }}

\phantomsection\label{definitions-with-arguments-variadic-tooltip}
Variadic parameters can be specified multiple times.

The arguments to apply to the function.

\subsubsection{\texorpdfstring{\texttt{\ where\ }}{ where }}\label{definitions-where}

Returns a selector that filters for elements belonging to this function
whose fields have the values of the given arguments.

self { . } { where } (

{ \hyperref[definitions-where-parameters-fields]{..} { any } }

) -\textgreater{} \href{/docs/reference/foundations/selector/}{selector}

\begin{verbatim}
#show heading.where(level: 2): set text(blue)
= Section
== Subsection
=== Sub-subsection
\end{verbatim}

\includegraphics[width=5in,height=\textheight,keepaspectratio]{/assets/docs/VOR4DpWIitR8ukDkDB2RigAAAAAAAAAA.png}

\paragraph{\texorpdfstring{\texttt{\ fields\ }}{ fields }}\label{definitions-where-fields}

{ any }

{Required} {{ Positional }}

\phantomsection\label{definitions-where-fields-positional-tooltip}
Positional parameters are specified in order, without names.

{{ Variadic }}

\phantomsection\label{definitions-where-fields-variadic-tooltip}
Variadic parameters can be specified multiple times.

The fields to filter for.

\href{/docs/reference/foundations/float/}{\pandocbounded{\includesvg[keepaspectratio]{/assets/icons/16-arrow-right.svg}}}

{ Float } { Previous page }

\href{/docs/reference/foundations/int/}{\pandocbounded{\includesvg[keepaspectratio]{/assets/icons/16-arrow-right.svg}}}

{ Integer } { Next page }


\title{typst.app/docs/reference/foundations/regex}

\begin{itemize}
\tightlist
\item
  \href{/docs}{\includesvg[width=0.16667in,height=0.16667in]{/assets/icons/16-docs-dark.svg}}
\item
  \includesvg[width=0.16667in,height=0.16667in]{/assets/icons/16-arrow-right.svg}
\item
  \href{/docs/reference/}{Reference}
\item
  \includesvg[width=0.16667in,height=0.16667in]{/assets/icons/16-arrow-right.svg}
\item
  \href{/docs/reference/foundations/}{Foundations}
\item
  \includesvg[width=0.16667in,height=0.16667in]{/assets/icons/16-arrow-right.svg}
\item
  \href{/docs/reference/foundations/regex/}{Regex}
\end{itemize}

\section{\texorpdfstring{{ regex }}{ regex }}\label{summary}

A regular expression.

Can be used as a \href{/docs/reference/styling/\#show-rules}{show rule
selector} and with \href{/docs/reference/foundations/str/}{string
methods} like \texttt{\ find\ } , \texttt{\ split\ } , and
\texttt{\ replace\ } .

\href{https://docs.rs/regex/latest/regex/\#syntax}{See here} for a
specification of the supported syntax.

\subsection{Example}\label{example}

\begin{verbatim}
// Works with string methods.
#"a,b;c".split(regex("[,;]"))

// Works with show rules.
#show regex("\d+"): set text(red)

The numbers 1 to 10.
\end{verbatim}

\includegraphics[width=5in,height=\textheight,keepaspectratio]{/assets/docs/UtfXJAklKdjyBZ3HmRwY-AAAAAAAAAAA.png}

\subsection{\texorpdfstring{Constructor
{}}{Constructor }}\label{constructor}

\phantomsection\label{constructor-constructor-tooltip}
If a type has a constructor, you can call it like a function to create a
new value of the type.

Create a regular expression from a string.

{ regex } (

{ \href{/docs/reference/foundations/str/}{str} }

) -\textgreater{} \href{/docs/reference/foundations/regex/}{regex}

\paragraph{\texorpdfstring{\texttt{\ regex\ }}{ regex }}\label{constructor-regex}

\href{/docs/reference/foundations/str/}{str}

{Required} {{ Positional }}

\phantomsection\label{constructor-regex-positional-tooltip}
Positional parameters are specified in order, without names.

The regular expression as a string.

Most regex escape sequences just work because they are not valid Typst
escape sequences. To produce regex escape sequences that are also valid
in Typst (e.g.
\texttt{\ }{\texttt{\ \textbackslash{}\textbackslash{}\ }}\texttt{\ } ),
you need to escape twice. Thus, to match a verbatim backslash, you would
need to write
\texttt{\ }{\texttt{\ regex\ }}\texttt{\ }{\texttt{\ (\ }}\texttt{\ }{\texttt{\ "\textbackslash{}\textbackslash{}\textbackslash{}\textbackslash{}"\ }}\texttt{\ }{\texttt{\ )\ }}\texttt{\ }
.

If you need many escape sequences, you can also create a raw element and
extract its text to use it for your regular expressions:

\includesvg[width=0.16667in,height=0.16667in]{/assets/icons/16-arrow-right.svg}
View example

\texttt{\ }{\texttt{\ regex\ }}\texttt{\ }{\texttt{\ (\ }}\texttt{\ }{\texttt{\ \textasciigrave{}\textbackslash{}d+\textbackslash{}.\textbackslash{}d+\textbackslash{}.\textbackslash{}d+\textasciigrave{}\ }}\texttt{\ }{\texttt{\ .\ }}\texttt{\ text\ }{\texttt{\ )\ }}\texttt{\ }
.

\href{/docs/reference/foundations/plugin/}{\pandocbounded{\includesvg[keepaspectratio]{/assets/icons/16-arrow-right.svg}}}

{ Plugin } { Previous page }

\href{/docs/reference/foundations/repr/}{\pandocbounded{\includesvg[keepaspectratio]{/assets/icons/16-arrow-right.svg}}}

{ Representation } { Next page }


\title{typst.app/docs/reference/foundations/type}

\begin{itemize}
\tightlist
\item
  \href{/docs}{\includesvg[width=0.16667in,height=0.16667in]{/assets/icons/16-docs-dark.svg}}
\item
  \includesvg[width=0.16667in,height=0.16667in]{/assets/icons/16-arrow-right.svg}
\item
  \href{/docs/reference/}{Reference}
\item
  \includesvg[width=0.16667in,height=0.16667in]{/assets/icons/16-arrow-right.svg}
\item
  \href{/docs/reference/foundations/}{Foundations}
\item
  \includesvg[width=0.16667in,height=0.16667in]{/assets/icons/16-arrow-right.svg}
\item
  \href{/docs/reference/foundations/type/}{Type}
\end{itemize}

\section{\texorpdfstring{{ type }}{ type }}\label{summary}

Describes a kind of value.

To style your document, you need to work with values of different kinds:
Lengths specifying the size of your elements, colors for your text and
shapes, and more. Typst categorizes these into clearly defined
\emph{types} and tells you where it expects which type of value.

Apart from basic types for numeric values and
\href{/docs/reference/foundations/int/}{typical}
\href{/docs/reference/foundations/float/}{types}
\href{/docs/reference/foundations/str/}{known}
\href{/docs/reference/foundations/array/}{from}
\href{/docs/reference/foundations/dictionary/}{programming} languages,
Typst provides a special type for
\href{/docs/reference/foundations/content/}{\emph{content.}} A value of
this type can hold anything that you can enter into your document: Text,
elements like headings and shapes, and style information.

\subsection{Example}\label{example}

\begin{verbatim}
#let x = 10
#if type(x) == int [
  #x is an integer!
] else [
  #x is another value...
]

An image is of type
#type(image("glacier.jpg")).
\end{verbatim}

\includegraphics[width=5in,height=\textheight,keepaspectratio]{/assets/docs/dTjHaEMO5150e0-XVg1OzwAAAAAAAAAA.png}

The type of \texttt{\ 10\ } is \texttt{\ int\ } . Now, what is the type
of \texttt{\ int\ } or even \texttt{\ type\ } ?

\begin{verbatim}
#type(int) \
#type(type)
\end{verbatim}

\includegraphics[width=5in,height=\textheight,keepaspectratio]{/assets/docs/HqIgZy_wqBbnboRlZ-Iv4AAAAAAAAAAA.png}

\subsection{Compatibility}\label{compatibility}

In Typst 0.7 and lower, the \texttt{\ type\ } function returned a string
instead of a type. Compatibility with the old way will remain for a
while to give package authors time to upgrade, but it will be removed at
some point.

\begin{itemize}
\tightlist
\item
  Checks like
  \texttt{\ int\ }{\texttt{\ ==\ }}\texttt{\ }{\texttt{\ "integer"\ }}\texttt{\ }
  evaluate to \texttt{\ }{\texttt{\ true\ }}\texttt{\ }
\item
  Adding/joining a type and string will yield a string
\item
  The \texttt{\ in\ } operator on a type and a dictionary will evaluate
  to \texttt{\ }{\texttt{\ true\ }}\texttt{\ } if the dictionary has a
  string key matching the type\textquotesingle s name
\end{itemize}

\subsection{\texorpdfstring{Constructor
{}}{Constructor }}\label{constructor}

\phantomsection\label{constructor-constructor-tooltip}
If a type has a constructor, you can call it like a function to create a
new value of the type.

Determines a value\textquotesingle s type.

{ type } (

{ { any } }

) -\textgreater{} \href{/docs/reference/foundations/type/}{type}

\begin{verbatim}
#type(12) \
#type(14.7) \
#type("hello") \
#type(<glacier>) \
#type([Hi]) \
#type(x => x + 1) \
#type(type)
\end{verbatim}

\includegraphics[width=5in,height=\textheight,keepaspectratio]{/assets/docs/A7_wGHgPK0Jhrp3CDC6IegAAAAAAAAAA.png}

\paragraph{\texorpdfstring{\texttt{\ value\ }}{ value }}\label{constructor-value}

{ any }

{Required} {{ Positional }}

\phantomsection\label{constructor-value-positional-tooltip}
Positional parameters are specified in order, without names.

The value whose type\textquotesingle s to determine.

\href{/docs/reference/foundations/sys/}{\pandocbounded{\includesvg[keepaspectratio]{/assets/icons/16-arrow-right.svg}}}

{ System } { Previous page }

\href{/docs/reference/foundations/version/}{\pandocbounded{\includesvg[keepaspectratio]{/assets/icons/16-arrow-right.svg}}}

{ Version } { Next page }


\title{typst.app/docs/reference/foundations/dictionary}

\begin{itemize}
\tightlist
\item
  \href{/docs}{\includesvg[width=0.16667in,height=0.16667in]{/assets/icons/16-docs-dark.svg}}
\item
  \includesvg[width=0.16667in,height=0.16667in]{/assets/icons/16-arrow-right.svg}
\item
  \href{/docs/reference/}{Reference}
\item
  \includesvg[width=0.16667in,height=0.16667in]{/assets/icons/16-arrow-right.svg}
\item
  \href{/docs/reference/foundations/}{Foundations}
\item
  \includesvg[width=0.16667in,height=0.16667in]{/assets/icons/16-arrow-right.svg}
\item
  \href{/docs/reference/foundations/dictionary/}{Dictionary}
\end{itemize}

\section{\texorpdfstring{{ dictionary }}{ dictionary }}\label{summary}

A map from string keys to values.

You can construct a dictionary by enclosing comma-separated
\texttt{\ key:\ value\ } pairs in parentheses. The values do not have to
be of the same type. Since empty parentheses already yield an empty
array, you have to use the special \texttt{\ (:)\ } syntax to create an
empty dictionary.

A dictionary is conceptually similar to an array, but it is indexed by
strings instead of integers. You can access and create dictionary
entries with the \texttt{\ .at()\ } method. If you know the key
statically, you can alternatively use
\href{/docs/reference/scripting/\#fields}{field access notation} (
\texttt{\ .key\ } ) to access the value. Dictionaries can be added with
the \texttt{\ +\ } operator and
\href{/docs/reference/scripting/\#blocks}{joined together} . To check
whether a key is present in the dictionary, use the \texttt{\ in\ }
keyword.

You can iterate over the pairs in a dictionary using a
\href{/docs/reference/scripting/\#loops}{for loop} . This will iterate
in the order the pairs were inserted / declared.

\subsection{Example}\label{example}

\begin{verbatim}
#let dict = (
  name: "Typst",
  born: 2019,
)

#dict.name \
#(dict.launch = 20)
#dict.len() \
#dict.keys() \
#dict.values() \
#dict.at("born") \
#dict.insert("city", "Berlin ")
#("name" in dict)
\end{verbatim}

\includegraphics[width=5in,height=\textheight,keepaspectratio]{/assets/docs/1ByIQqDPZ4VVxPmFNoQXgwAAAAAAAAAA.png}

\subsection{\texorpdfstring{Constructor
{}}{Constructor }}\label{constructor}

\phantomsection\label{constructor-constructor-tooltip}
If a type has a constructor, you can call it like a function to create a
new value of the type.

Converts a value into a dictionary.

Note that this function is only intended for conversion of a
dictionary-like value to a dictionary, not for creation of a dictionary
from individual pairs. Use the dictionary syntax
\texttt{\ (key:\ value)\ } instead.

{ dictionary } (

{ \href{/docs/reference/foundations/module/}{module} }

) -\textgreater{}
\href{/docs/reference/foundations/dictionary/}{dictionary}

\begin{verbatim}
#dictionary(sys).at("version")
\end{verbatim}

\includegraphics[width=5in,height=\textheight,keepaspectratio]{/assets/docs/vrwNZ5Jfl6kz7gYnEOsM0AAAAAAAAAAA.png}

\paragraph{\texorpdfstring{\texttt{\ value\ }}{ value }}\label{constructor-value}

\href{/docs/reference/foundations/module/}{module}

{Required} {{ Positional }}

\phantomsection\label{constructor-value-positional-tooltip}
Positional parameters are specified in order, without names.

The value that should be converted to a dictionary.

\subsection{\texorpdfstring{{ Definitions
}}{ Definitions }}\label{definitions}

\phantomsection\label{definitions-tooltip}
Functions and types and can have associated definitions. These are
accessed by specifying the function or type, followed by a period, and
then the definition\textquotesingle s name.

\subsubsection{\texorpdfstring{\texttt{\ len\ }}{ len }}\label{definitions-len}

The number of pairs in the dictionary.

self { . } { len } (

) -\textgreater{} \href{/docs/reference/foundations/int/}{int}

\subsubsection{\texorpdfstring{\texttt{\ at\ }}{ at }}\label{definitions-at}

Returns the value associated with the specified key in the dictionary.
May be used on the left-hand side of an assignment if the key is already
present in the dictionary. Returns the default value if the key is not
part of the dictionary or fails with an error if no default value was
specified.

self { . } { at } (

{ \href{/docs/reference/foundations/str/}{str} , } {
\hyperref[definitions-at-parameters-default]{default :} { any } , }

) -\textgreater{} { any }

\paragraph{\texorpdfstring{\texttt{\ key\ }}{ key }}\label{definitions-at-key}

\href{/docs/reference/foundations/str/}{str}

{Required} {{ Positional }}

\phantomsection\label{definitions-at-key-positional-tooltip}
Positional parameters are specified in order, without names.

The key at which to retrieve the item.

\paragraph{\texorpdfstring{\texttt{\ default\ }}{ default }}\label{definitions-at-default}

{ any }

A default value to return if the key is not part of the dictionary.

\subsubsection{\texorpdfstring{\texttt{\ insert\ }}{ insert }}\label{definitions-insert}

Inserts a new pair into the dictionary. If the dictionary already
contains this key, the value is updated.

self { . } { insert } (

{ \href{/docs/reference/foundations/str/}{str} , } { { any } , }

)

\paragraph{\texorpdfstring{\texttt{\ key\ }}{ key }}\label{definitions-insert-key}

\href{/docs/reference/foundations/str/}{str}

{Required} {{ Positional }}

\phantomsection\label{definitions-insert-key-positional-tooltip}
Positional parameters are specified in order, without names.

The key of the pair that should be inserted.

\paragraph{\texorpdfstring{\texttt{\ value\ }}{ value }}\label{definitions-insert-value}

{ any }

{Required} {{ Positional }}

\phantomsection\label{definitions-insert-value-positional-tooltip}
Positional parameters are specified in order, without names.

The value of the pair that should be inserted.

\subsubsection{\texorpdfstring{\texttt{\ remove\ }}{ remove }}\label{definitions-remove}

Removes a pair from the dictionary by key and return the value.

self { . } { remove } (

{ \href{/docs/reference/foundations/str/}{str} , } {
\hyperref[definitions-remove-parameters-default]{default :} { any } , }

) -\textgreater{} { any }

\paragraph{\texorpdfstring{\texttt{\ key\ }}{ key }}\label{definitions-remove-key}

\href{/docs/reference/foundations/str/}{str}

{Required} {{ Positional }}

\phantomsection\label{definitions-remove-key-positional-tooltip}
Positional parameters are specified in order, without names.

The key of the pair to remove.

\paragraph{\texorpdfstring{\texttt{\ default\ }}{ default }}\label{definitions-remove-default}

{ any }

A default value to return if the key does not exist.

\subsubsection{\texorpdfstring{\texttt{\ keys\ }}{ keys }}\label{definitions-keys}

Returns the keys of the dictionary as an array in insertion order.

self { . } { keys } (

) -\textgreater{} \href{/docs/reference/foundations/array/}{array}

\subsubsection{\texorpdfstring{\texttt{\ values\ }}{ values }}\label{definitions-values}

Returns the values of the dictionary as an array in insertion order.

self { . } { values } (

) -\textgreater{} \href{/docs/reference/foundations/array/}{array}

\subsubsection{\texorpdfstring{\texttt{\ pairs\ }}{ pairs }}\label{definitions-pairs}

Returns the keys and values of the dictionary as an array of pairs. Each
pair is represented as an array of length two.

self { . } { pairs } (

) -\textgreater{} \href{/docs/reference/foundations/array/}{array}

\href{/docs/reference/foundations/decimal/}{\pandocbounded{\includesvg[keepaspectratio]{/assets/icons/16-arrow-right.svg}}}

{ Decimal } { Previous page }

\href{/docs/reference/foundations/duration/}{\pandocbounded{\includesvg[keepaspectratio]{/assets/icons/16-arrow-right.svg}}}

{ Duration } { Next page }


\title{typst.app/docs/reference/foundations/str}

\begin{itemize}
\tightlist
\item
  \href{/docs}{\includesvg[width=0.16667in,height=0.16667in]{/assets/icons/16-docs-dark.svg}}
\item
  \includesvg[width=0.16667in,height=0.16667in]{/assets/icons/16-arrow-right.svg}
\item
  \href{/docs/reference/}{Reference}
\item
  \includesvg[width=0.16667in,height=0.16667in]{/assets/icons/16-arrow-right.svg}
\item
  \href{/docs/reference/foundations/}{Foundations}
\item
  \includesvg[width=0.16667in,height=0.16667in]{/assets/icons/16-arrow-right.svg}
\item
  \href{/docs/reference/foundations/str/}{String}
\end{itemize}

\section{\texorpdfstring{{ str }}{ str }}\label{summary}

A sequence of Unicode codepoints.

You can iterate over the grapheme clusters of the string using a
\href{/docs/reference/scripting/\#loops}{for loop} . Grapheme clusters
are basically characters but keep together things that belong together,
e.g. multiple codepoints that together form a flag emoji. Strings can be
added with the \texttt{\ +\ } operator,
\href{/docs/reference/scripting/\#blocks}{joined together} and
multiplied with integers.

Typst provides utility methods for string manipulation. Many of these
methods (e.g., \texttt{\ split\ } , \texttt{\ trim\ } and
\texttt{\ replace\ } ) operate on \emph{patterns:} A pattern can be
either a string or a \href{/docs/reference/foundations/regex/}{regular
expression} . This makes the methods quite versatile.

All lengths and indices are expressed in terms of UTF-8 bytes. Indices
are zero-based and negative indices wrap around to the end of the
string.

You can convert a value to a string with this type\textquotesingle s
constructor.

\subsection{Example}\label{example}

\begin{verbatim}
#"hello world!" \
#"\"hello\n  world\"!" \
#"1 2 3".split() \
#"1,2;3".split(regex("[,;]")) \
#(regex("\d+") in "ten euros") \
#(regex("\d+") in "10 euros")
\end{verbatim}

\includegraphics[width=5in,height=\textheight,keepaspectratio]{/assets/docs/gK89AnI9k7dy82m9R3F1jgAAAAAAAAAA.png}

\subsection{Escape sequences}\label{escapes}

Just like in markup, you can escape a few symbols in strings:

\begin{itemize}
\tightlist
\item
  \texttt{\ }{\texttt{\ \textbackslash{}\textbackslash{}\ }}\texttt{\ }
  for a backslash
\item
  \texttt{\ }{\texttt{\ \textbackslash{}"\ }}\texttt{\ } for a quote
\item
  \texttt{\ }{\texttt{\ \textbackslash{}n\ }}\texttt{\ } for a newline
\item
  \texttt{\ }{\texttt{\ \textbackslash{}r\ }}\texttt{\ } for a carriage
  return
\item
  \texttt{\ }{\texttt{\ \textbackslash{}t\ }}\texttt{\ } for a tab
\item
  \texttt{\ }{\texttt{\ \textbackslash{}u\{1f600\}\ }}\texttt{\ } for a
  hexadecimal Unicode escape sequence
\end{itemize}

\subsection{\texorpdfstring{Constructor
{}}{Constructor }}\label{constructor}

\phantomsection\label{constructor-constructor-tooltip}
If a type has a constructor, you can call it like a function to create a
new value of the type.

Converts a value to a string.

\begin{itemize}
\tightlist
\item
  Integers are formatted in base 10. This can be overridden with the
  optional \texttt{\ base\ } parameter.
\item
  Floats are formatted in base 10 and never in exponential notation.
\item
  From labels the name is extracted.
\item
  Bytes are decoded as UTF-8.
\end{itemize}

If you wish to convert from and to Unicode code points, see the
\href{/docs/reference/foundations/str/\#definitions-to-unicode}{\texttt{\ to-unicode\ }}
and
\href{/docs/reference/foundations/str/\#definitions-from-unicode}{\texttt{\ from-unicode\ }}
functions.

{ str } (

{ \href{/docs/reference/foundations/int/}{int}
\href{/docs/reference/foundations/float/}{float}
\href{/docs/reference/foundations/str/}{str}
\href{/docs/reference/foundations/bytes/}{bytes}
\href{/docs/reference/foundations/label/}{label}
\href{/docs/reference/foundations/decimal/}{decimal}
\href{/docs/reference/foundations/version/}{version}
\href{/docs/reference/foundations/type/}{type} , } {
\hyperref[constructor-parameters-base]{base :}
\href{/docs/reference/foundations/int/}{int} , }

) -\textgreater{} \href{/docs/reference/foundations/str/}{str}

\begin{verbatim}
#str(10) \
#str(4000, base: 16) \
#str(2.7) \
#str(1e8) \
#str(<intro>)
\end{verbatim}

\includegraphics[width=5in,height=\textheight,keepaspectratio]{/assets/docs/06jR9z-fP-M4eu8XB2MFnAAAAAAAAAAA.png}

\paragraph{\texorpdfstring{\texttt{\ value\ }}{ value }}\label{constructor-value}

\href{/docs/reference/foundations/int/}{int} {or}
\href{/docs/reference/foundations/float/}{float} {or}
\href{/docs/reference/foundations/str/}{str} {or}
\href{/docs/reference/foundations/bytes/}{bytes} {or}
\href{/docs/reference/foundations/label/}{label} {or}
\href{/docs/reference/foundations/decimal/}{decimal} {or}
\href{/docs/reference/foundations/version/}{version} {or}
\href{/docs/reference/foundations/type/}{type}

{Required} {{ Positional }}

\phantomsection\label{constructor-value-positional-tooltip}
Positional parameters are specified in order, without names.

The value that should be converted to a string.

\paragraph{\texorpdfstring{\texttt{\ base\ }}{ base }}\label{constructor-base}

\href{/docs/reference/foundations/int/}{int}

The base (radix) to display integers in, between 2 and 36.

Default: \texttt{\ }{\texttt{\ 10\ }}\texttt{\ }

\subsection{\texorpdfstring{{ Definitions
}}{ Definitions }}\label{definitions}

\phantomsection\label{definitions-tooltip}
Functions and types and can have associated definitions. These are
accessed by specifying the function or type, followed by a period, and
then the definition\textquotesingle s name.

\subsubsection{\texorpdfstring{\texttt{\ len\ }}{ len }}\label{definitions-len}

The length of the string in UTF-8 encoded bytes.

self { . } { len } (

) -\textgreater{} \href{/docs/reference/foundations/int/}{int}

\subsubsection{\texorpdfstring{\texttt{\ first\ }}{ first }}\label{definitions-first}

Extracts the first grapheme cluster of the string. Fails with an error
if the string is empty.

self { . } { first } (

) -\textgreater{} \href{/docs/reference/foundations/str/}{str}

\subsubsection{\texorpdfstring{\texttt{\ last\ }}{ last }}\label{definitions-last}

Extracts the last grapheme cluster of the string. Fails with an error if
the string is empty.

self { . } { last } (

) -\textgreater{} \href{/docs/reference/foundations/str/}{str}

\subsubsection{\texorpdfstring{\texttt{\ at\ }}{ at }}\label{definitions-at}

Extracts the first grapheme cluster after the specified index. Returns
the default value if the index is out of bounds or fails with an error
if no default value was specified.

self { . } { at } (

{ \href{/docs/reference/foundations/int/}{int} , } {
\hyperref[definitions-at-parameters-default]{default :} { any } , }

) -\textgreater{} { any }

\paragraph{\texorpdfstring{\texttt{\ index\ }}{ index }}\label{definitions-at-index}

\href{/docs/reference/foundations/int/}{int}

{Required} {{ Positional }}

\phantomsection\label{definitions-at-index-positional-tooltip}
Positional parameters are specified in order, without names.

The byte index. If negative, indexes from the back.

\paragraph{\texorpdfstring{\texttt{\ default\ }}{ default }}\label{definitions-at-default}

{ any }

A default value to return if the index is out of bounds.

\subsubsection{\texorpdfstring{\texttt{\ slice\ }}{ slice }}\label{definitions-slice}

Extracts a substring of the string. Fails with an error if the start or
end index is out of bounds.

self { . } { slice } (

{ \href{/docs/reference/foundations/int/}{int} , } {
\href{/docs/reference/foundations/none/}{none}
\href{/docs/reference/foundations/int/}{int} , } {
\hyperref[definitions-slice-parameters-count]{count :}
\href{/docs/reference/foundations/int/}{int} , }

) -\textgreater{} \href{/docs/reference/foundations/str/}{str}

\paragraph{\texorpdfstring{\texttt{\ start\ }}{ start }}\label{definitions-slice-start}

\href{/docs/reference/foundations/int/}{int}

{Required} {{ Positional }}

\phantomsection\label{definitions-slice-start-positional-tooltip}
Positional parameters are specified in order, without names.

The start byte index (inclusive). If negative, indexes from the back.

\paragraph{\texorpdfstring{\texttt{\ end\ }}{ end }}\label{definitions-slice-end}

\href{/docs/reference/foundations/none/}{none} {or}
\href{/docs/reference/foundations/int/}{int}

{{ Positional }}

\phantomsection\label{definitions-slice-end-positional-tooltip}
Positional parameters are specified in order, without names.

The end byte index (exclusive). If omitted, the whole slice until the
end of the string is extracted. If negative, indexes from the back.

Default: \texttt{\ }{\texttt{\ none\ }}\texttt{\ }

\paragraph{\texorpdfstring{\texttt{\ count\ }}{ count }}\label{definitions-slice-count}

\href{/docs/reference/foundations/int/}{int}

The number of bytes to extract. This is equivalent to passing
\texttt{\ start\ +\ count\ } as the \texttt{\ end\ } position. Mutually
exclusive with \texttt{\ end\ } .

\subsubsection{\texorpdfstring{\texttt{\ clusters\ }}{ clusters }}\label{definitions-clusters}

Returns the grapheme clusters of the string as an array of substrings.

self { . } { clusters } (

) -\textgreater{} \href{/docs/reference/foundations/array/}{array}

\subsubsection{\texorpdfstring{\texttt{\ codepoints\ }}{ codepoints }}\label{definitions-codepoints}

Returns the Unicode codepoints of the string as an array of substrings.

self { . } { codepoints } (

) -\textgreater{} \href{/docs/reference/foundations/array/}{array}

\subsubsection{\texorpdfstring{\texttt{\ to-unicode\ }}{ to-unicode }}\label{definitions-to-unicode}

Converts a character into its corresponding code point.

str { . } { to-unicode } (

{ \href{/docs/reference/foundations/str/}{str} }

) -\textgreater{} \href{/docs/reference/foundations/int/}{int}

\begin{verbatim}
#"a".to-unicode() \
#("a\u{0300}"
   .codepoints()
   .map(str.to-unicode))
\end{verbatim}

\includegraphics[width=5in,height=\textheight,keepaspectratio]{/assets/docs/q50tz6WAJPnwtBCYWbHrIwAAAAAAAAAA.png}

\paragraph{\texorpdfstring{\texttt{\ character\ }}{ character }}\label{definitions-to-unicode-character}

\href{/docs/reference/foundations/str/}{str}

{Required} {{ Positional }}

\phantomsection\label{definitions-to-unicode-character-positional-tooltip}
Positional parameters are specified in order, without names.

The character that should be converted.

\subsubsection{\texorpdfstring{\texttt{\ from-unicode\ }}{ from-unicode }}\label{definitions-from-unicode}

Converts a unicode code point into its corresponding string.

str { . } { from-unicode } (

{ \href{/docs/reference/foundations/int/}{int} }

) -\textgreater{} \href{/docs/reference/foundations/str/}{str}

\begin{verbatim}
#str.from-unicode(97)
\end{verbatim}

\includegraphics[width=5in,height=\textheight,keepaspectratio]{/assets/docs/vNzcsGO4Zd_u-P4qNnxrDQAAAAAAAAAA.png}

\paragraph{\texorpdfstring{\texttt{\ value\ }}{ value }}\label{definitions-from-unicode-value}

\href{/docs/reference/foundations/int/}{int}

{Required} {{ Positional }}

\phantomsection\label{definitions-from-unicode-value-positional-tooltip}
Positional parameters are specified in order, without names.

The code point that should be converted.

\subsubsection{\texorpdfstring{\texttt{\ contains\ }}{ contains }}\label{definitions-contains}

Whether the string contains the specified pattern.

This method also has dedicated syntax: You can write
\texttt{\ }{\texttt{\ "bc"\ }}\texttt{\ }{\texttt{\ in\ }}\texttt{\ }{\texttt{\ "abcd"\ }}\texttt{\ }
instead of
\texttt{\ }{\texttt{\ "abcd"\ }}\texttt{\ }{\texttt{\ .\ }}\texttt{\ }{\texttt{\ contains\ }}\texttt{\ }{\texttt{\ (\ }}\texttt{\ }{\texttt{\ "bc"\ }}\texttt{\ }{\texttt{\ )\ }}\texttt{\ }
.

self { . } { contains } (

{ \href{/docs/reference/foundations/str/}{str}
\href{/docs/reference/foundations/regex/}{regex} }

) -\textgreater{} \href{/docs/reference/foundations/bool/}{bool}

\paragraph{\texorpdfstring{\texttt{\ pattern\ }}{ pattern }}\label{definitions-contains-pattern}

\href{/docs/reference/foundations/str/}{str} {or}
\href{/docs/reference/foundations/regex/}{regex}

{Required} {{ Positional }}

\phantomsection\label{definitions-contains-pattern-positional-tooltip}
Positional parameters are specified in order, without names.

The pattern to search for.

\subsubsection{\texorpdfstring{\texttt{\ starts-with\ }}{ starts-with }}\label{definitions-starts-with}

Whether the string starts with the specified pattern.

self { . } { starts-with } (

{ \href{/docs/reference/foundations/str/}{str}
\href{/docs/reference/foundations/regex/}{regex} }

) -\textgreater{} \href{/docs/reference/foundations/bool/}{bool}

\paragraph{\texorpdfstring{\texttt{\ pattern\ }}{ pattern }}\label{definitions-starts-with-pattern}

\href{/docs/reference/foundations/str/}{str} {or}
\href{/docs/reference/foundations/regex/}{regex}

{Required} {{ Positional }}

\phantomsection\label{definitions-starts-with-pattern-positional-tooltip}
Positional parameters are specified in order, without names.

The pattern the string might start with.

\subsubsection{\texorpdfstring{\texttt{\ ends-with\ }}{ ends-with }}\label{definitions-ends-with}

Whether the string ends with the specified pattern.

self { . } { ends-with } (

{ \href{/docs/reference/foundations/str/}{str}
\href{/docs/reference/foundations/regex/}{regex} }

) -\textgreater{} \href{/docs/reference/foundations/bool/}{bool}

\paragraph{\texorpdfstring{\texttt{\ pattern\ }}{ pattern }}\label{definitions-ends-with-pattern}

\href{/docs/reference/foundations/str/}{str} {or}
\href{/docs/reference/foundations/regex/}{regex}

{Required} {{ Positional }}

\phantomsection\label{definitions-ends-with-pattern-positional-tooltip}
Positional parameters are specified in order, without names.

The pattern the string might end with.

\subsubsection{\texorpdfstring{\texttt{\ find\ }}{ find }}\label{definitions-find}

Searches for the specified pattern in the string and returns the first
match as a string or \texttt{\ }{\texttt{\ none\ }}\texttt{\ } if there
is no match.

self { . } { find } (

{ \href{/docs/reference/foundations/str/}{str}
\href{/docs/reference/foundations/regex/}{regex} }

) -\textgreater{} \href{/docs/reference/foundations/none/}{none}
\href{/docs/reference/foundations/str/}{str}

\paragraph{\texorpdfstring{\texttt{\ pattern\ }}{ pattern }}\label{definitions-find-pattern}

\href{/docs/reference/foundations/str/}{str} {or}
\href{/docs/reference/foundations/regex/}{regex}

{Required} {{ Positional }}

\phantomsection\label{definitions-find-pattern-positional-tooltip}
Positional parameters are specified in order, without names.

The pattern to search for.

\subsubsection{\texorpdfstring{\texttt{\ position\ }}{ position }}\label{definitions-position}

Searches for the specified pattern in the string and returns the index
of the first match as an integer or
\texttt{\ }{\texttt{\ none\ }}\texttt{\ } if there is no match.

self { . } { position } (

{ \href{/docs/reference/foundations/str/}{str}
\href{/docs/reference/foundations/regex/}{regex} }

) -\textgreater{} \href{/docs/reference/foundations/none/}{none}
\href{/docs/reference/foundations/int/}{int}

\paragraph{\texorpdfstring{\texttt{\ pattern\ }}{ pattern }}\label{definitions-position-pattern}

\href{/docs/reference/foundations/str/}{str} {or}
\href{/docs/reference/foundations/regex/}{regex}

{Required} {{ Positional }}

\phantomsection\label{definitions-position-pattern-positional-tooltip}
Positional parameters are specified in order, without names.

The pattern to search for.

\subsubsection{\texorpdfstring{\texttt{\ match\ }}{ match }}\label{definitions-match}

Searches for the specified pattern in the string and returns a
dictionary with details about the first match or
\texttt{\ }{\texttt{\ none\ }}\texttt{\ } if there is no match.

The returned dictionary has the following keys:

\begin{itemize}
\tightlist
\item
  \texttt{\ start\ } : The start offset of the match
\item
  \texttt{\ end\ } : The end offset of the match
\item
  \texttt{\ text\ } : The text that matched.
\item
  \texttt{\ captures\ } : An array containing a string for each matched
  capturing group. The first item of the array contains the first
  matched capturing, not the whole match! This is empty unless the
  \texttt{\ pattern\ } was a regex with capturing groups.
\end{itemize}

self { . } { match } (

{ \href{/docs/reference/foundations/str/}{str}
\href{/docs/reference/foundations/regex/}{regex} }

) -\textgreater{} \href{/docs/reference/foundations/none/}{none}
\href{/docs/reference/foundations/dictionary/}{dictionary}

\paragraph{\texorpdfstring{\texttt{\ pattern\ }}{ pattern }}\label{definitions-match-pattern}

\href{/docs/reference/foundations/str/}{str} {or}
\href{/docs/reference/foundations/regex/}{regex}

{Required} {{ Positional }}

\phantomsection\label{definitions-match-pattern-positional-tooltip}
Positional parameters are specified in order, without names.

The pattern to search for.

\subsubsection{\texorpdfstring{\texttt{\ matches\ }}{ matches }}\label{definitions-matches}

Searches for the specified pattern in the string and returns an array of
dictionaries with details about all matches. For details about the
returned dictionaries, see above.

self { . } { matches } (

{ \href{/docs/reference/foundations/str/}{str}
\href{/docs/reference/foundations/regex/}{regex} }

) -\textgreater{} \href{/docs/reference/foundations/array/}{array}

\paragraph{\texorpdfstring{\texttt{\ pattern\ }}{ pattern }}\label{definitions-matches-pattern}

\href{/docs/reference/foundations/str/}{str} {or}
\href{/docs/reference/foundations/regex/}{regex}

{Required} {{ Positional }}

\phantomsection\label{definitions-matches-pattern-positional-tooltip}
Positional parameters are specified in order, without names.

The pattern to search for.

\subsubsection{\texorpdfstring{\texttt{\ replace\ }}{ replace }}\label{definitions-replace}

Replace at most \texttt{\ count\ } occurrences of the given pattern with
a replacement string or function (beginning from the start). If no count
is given, all occurrences are replaced.

self { . } { replace } (

{ \href{/docs/reference/foundations/str/}{str}
\href{/docs/reference/foundations/regex/}{regex} , } {
\href{/docs/reference/foundations/str/}{str}
\href{/docs/reference/foundations/function/}{function} , } {
\hyperref[definitions-replace-parameters-count]{count :}
\href{/docs/reference/foundations/int/}{int} , }

) -\textgreater{} \href{/docs/reference/foundations/str/}{str}

\paragraph{\texorpdfstring{\texttt{\ pattern\ }}{ pattern }}\label{definitions-replace-pattern}

\href{/docs/reference/foundations/str/}{str} {or}
\href{/docs/reference/foundations/regex/}{regex}

{Required} {{ Positional }}

\phantomsection\label{definitions-replace-pattern-positional-tooltip}
Positional parameters are specified in order, without names.

The pattern to search for.

\paragraph{\texorpdfstring{\texttt{\ replacement\ }}{ replacement }}\label{definitions-replace-replacement}

\href{/docs/reference/foundations/str/}{str} {or}
\href{/docs/reference/foundations/function/}{function}

{Required} {{ Positional }}

\phantomsection\label{definitions-replace-replacement-positional-tooltip}
Positional parameters are specified in order, without names.

The string to replace the matches with or a function that gets a
dictionary for each match and can return individual replacement strings.

\paragraph{\texorpdfstring{\texttt{\ count\ }}{ count }}\label{definitions-replace-count}

\href{/docs/reference/foundations/int/}{int}

If given, only the first \texttt{\ count\ } matches of the pattern are
placed.

\subsubsection{\texorpdfstring{\texttt{\ trim\ }}{ trim }}\label{definitions-trim}

Removes matches of a pattern from one or both sides of the string, once
or repeatedly and returns the resulting string.

self { . } { trim } (

{ \href{/docs/reference/foundations/none/}{none}
\href{/docs/reference/foundations/str/}{str}
\href{/docs/reference/foundations/regex/}{regex} , } {
\hyperref[definitions-trim-parameters-at]{at :}
\href{/docs/reference/layout/alignment/}{alignment} , } {
\hyperref[definitions-trim-parameters-repeat]{repeat :}
\href{/docs/reference/foundations/bool/}{bool} , }

) -\textgreater{} \href{/docs/reference/foundations/str/}{str}

\paragraph{\texorpdfstring{\texttt{\ pattern\ }}{ pattern }}\label{definitions-trim-pattern}

\href{/docs/reference/foundations/none/}{none} {or}
\href{/docs/reference/foundations/str/}{str} {or}
\href{/docs/reference/foundations/regex/}{regex}

{{ Positional }}

\phantomsection\label{definitions-trim-pattern-positional-tooltip}
Positional parameters are specified in order, without names.

The pattern to search for. If \texttt{\ }{\texttt{\ none\ }}\texttt{\ }
, trims white spaces.

Default: \texttt{\ }{\texttt{\ none\ }}\texttt{\ }

\paragraph{\texorpdfstring{\texttt{\ at\ }}{ at }}\label{definitions-trim-at}

\href{/docs/reference/layout/alignment/}{alignment}

Can be \texttt{\ start\ } or \texttt{\ end\ } to only trim the start or
end of the string. If omitted, both sides are trimmed.

\paragraph{\texorpdfstring{\texttt{\ repeat\ }}{ repeat }}\label{definitions-trim-repeat}

\href{/docs/reference/foundations/bool/}{bool}

Whether to repeatedly removes matches of the pattern or just once.
Defaults to \texttt{\ }{\texttt{\ true\ }}\texttt{\ } .

Default: \texttt{\ }{\texttt{\ true\ }}\texttt{\ }

\subsubsection{\texorpdfstring{\texttt{\ split\ }}{ split }}\label{definitions-split}

Splits a string at matches of a specified pattern and returns an array
of the resulting parts.

self { . } { split } (

{ \href{/docs/reference/foundations/none/}{none}
\href{/docs/reference/foundations/str/}{str}
\href{/docs/reference/foundations/regex/}{regex} }

) -\textgreater{} \href{/docs/reference/foundations/array/}{array}

\paragraph{\texorpdfstring{\texttt{\ pattern\ }}{ pattern }}\label{definitions-split-pattern}

\href{/docs/reference/foundations/none/}{none} {or}
\href{/docs/reference/foundations/str/}{str} {or}
\href{/docs/reference/foundations/regex/}{regex}

{{ Positional }}

\phantomsection\label{definitions-split-pattern-positional-tooltip}
Positional parameters are specified in order, without names.

The pattern to split at. Defaults to whitespace.

Default: \texttt{\ }{\texttt{\ none\ }}\texttt{\ }

\subsubsection{\texorpdfstring{\texttt{\ rev\ }}{ rev }}\label{definitions-rev}

Reverse the string.

self { . } { rev } (

) -\textgreater{} \href{/docs/reference/foundations/str/}{str}

\href{/docs/reference/foundations/selector/}{\pandocbounded{\includesvg[keepaspectratio]{/assets/icons/16-arrow-right.svg}}}

{ Selector } { Previous page }

\href{/docs/reference/foundations/style/}{\pandocbounded{\includesvg[keepaspectratio]{/assets/icons/16-arrow-right.svg}}}

{ Style } { Next page }


\title{typst.app/docs/reference/foundations/selector}

\begin{itemize}
\tightlist
\item
  \href{/docs}{\includesvg[width=0.16667in,height=0.16667in]{/assets/icons/16-docs-dark.svg}}
\item
  \includesvg[width=0.16667in,height=0.16667in]{/assets/icons/16-arrow-right.svg}
\item
  \href{/docs/reference/}{Reference}
\item
  \includesvg[width=0.16667in,height=0.16667in]{/assets/icons/16-arrow-right.svg}
\item
  \href{/docs/reference/foundations/}{Foundations}
\item
  \includesvg[width=0.16667in,height=0.16667in]{/assets/icons/16-arrow-right.svg}
\item
  \href{/docs/reference/foundations/selector/}{Selector}
\end{itemize}

\section{\texorpdfstring{{ selector }}{ selector }}\label{summary}

A filter for selecting elements within the document.

You can construct a selector in the following ways:

\begin{itemize}
\tightlist
\item
  you can use an element
  \href{/docs/reference/foundations/function/}{function}
\item
  you can filter for an element function with
  \href{/docs/reference/foundations/function/\#definitions-where}{specific
  fields}
\item
  you can use a \href{/docs/reference/foundations/str/}{string} or
  \href{/docs/reference/foundations/regex/}{regular expression}
\item
  you can use a
  \href{/docs/reference/foundations/label/}{\texttt{\ }{\texttt{\ \textless{}label\textgreater{}\ }}\texttt{\ }}
\item
  you can use a
  \href{/docs/reference/introspection/location/}{\texttt{\ location\ }}
\item
  call the
  \href{/docs/reference/foundations/selector/}{\texttt{\ selector\ }}
  constructor to convert any of the above types into a selector value
  and use the methods below to refine it
\end{itemize}

Selectors are used to \href{/docs/reference/styling/\#show-rules}{apply
styling rules} to elements. You can also use selectors to
\href{/docs/reference/introspection/query/}{query} the document for
certain types of elements.

Furthermore, you can pass a selector to several of
Typst\textquotesingle s built-in functions to configure their behaviour.
One such example is the \href{/docs/reference/model/outline/}{outline}
where it can be used to change which elements are listed within the
outline.

Multiple selectors can be combined using the methods shown below.
However, not all kinds of selectors are supported in all places, at the
moment.

\subsection{Example}\label{example}

\begin{verbatim}
#context query(
  heading.where(level: 1)
    .or(heading.where(level: 2))
)

= This will be found
== So will this
=== But this will not.
\end{verbatim}

\includegraphics[width=5in,height=\textheight,keepaspectratio]{/assets/docs/SW-2iLP1LIGQ0ITsB7LGEQAAAAAAAAAA.png}

\subsection{\texorpdfstring{Constructor
{}}{Constructor }}\label{constructor}

\phantomsection\label{constructor-constructor-tooltip}
If a type has a constructor, you can call it like a function to create a
new value of the type.

Turns a value into a selector. The following values are accepted:

\begin{itemize}
\tightlist
\item
  An element function like a \texttt{\ heading\ } or \texttt{\ figure\ }
  .
\item
  A \texttt{\ }{\texttt{\ \textless{}label\textgreater{}\ }}\texttt{\ }
  .
\item
  A more complex selector like
  \texttt{\ heading\ }{\texttt{\ .\ }}\texttt{\ }{\texttt{\ where\ }}\texttt{\ }{\texttt{\ (\ }}\texttt{\ level\ }{\texttt{\ :\ }}\texttt{\ }{\texttt{\ 1\ }}\texttt{\ }{\texttt{\ )\ }}\texttt{\ }
  .
\end{itemize}

{ selector } (

{ \href{/docs/reference/foundations/str/}{str}
\href{/docs/reference/foundations/regex/}{regex}
\href{/docs/reference/foundations/label/}{label}
\href{/docs/reference/foundations/selector/}{selector}
\href{/docs/reference/introspection/location/}{location}
\href{/docs/reference/foundations/function/}{function} }

) -\textgreater{} \href{/docs/reference/foundations/selector/}{selector}

\paragraph{\texorpdfstring{\texttt{\ target\ }}{ target }}\label{constructor-target}

\href{/docs/reference/foundations/str/}{str} {or}
\href{/docs/reference/foundations/regex/}{regex} {or}
\href{/docs/reference/foundations/label/}{label} {or}
\href{/docs/reference/foundations/selector/}{selector} {or}
\href{/docs/reference/introspection/location/}{location} {or}
\href{/docs/reference/foundations/function/}{function}

{Required} {{ Positional }}

\phantomsection\label{constructor-target-positional-tooltip}
Positional parameters are specified in order, without names.

Can be an element function like a \texttt{\ heading\ } or
\texttt{\ figure\ } , a
\texttt{\ }{\texttt{\ \textless{}label\textgreater{}\ }}\texttt{\ } or a
more complex selector like
\texttt{\ heading\ }{\texttt{\ .\ }}\texttt{\ }{\texttt{\ where\ }}\texttt{\ }{\texttt{\ (\ }}\texttt{\ level\ }{\texttt{\ :\ }}\texttt{\ }{\texttt{\ 1\ }}\texttt{\ }{\texttt{\ )\ }}\texttt{\ }
.

\subsection{\texorpdfstring{{ Definitions
}}{ Definitions }}\label{definitions}

\phantomsection\label{definitions-tooltip}
Functions and types and can have associated definitions. These are
accessed by specifying the function or type, followed by a period, and
then the definition\textquotesingle s name.

\subsubsection{\texorpdfstring{\texttt{\ or\ }}{ or }}\label{definitions-or}

Selects all elements that match this or any of the other selectors.

self { . } { or } (

{ \hyperref[definitions-or-parameters-others]{..}
\href{/docs/reference/foundations/str/}{str}
\href{/docs/reference/foundations/regex/}{regex}
\href{/docs/reference/foundations/label/}{label}
\href{/docs/reference/foundations/selector/}{selector}
\href{/docs/reference/introspection/location/}{location}
\href{/docs/reference/foundations/function/}{function} }

) -\textgreater{} \href{/docs/reference/foundations/selector/}{selector}

\paragraph{\texorpdfstring{\texttt{\ others\ }}{ others }}\label{definitions-or-others}

\href{/docs/reference/foundations/str/}{str} {or}
\href{/docs/reference/foundations/regex/}{regex} {or}
\href{/docs/reference/foundations/label/}{label} {or}
\href{/docs/reference/foundations/selector/}{selector} {or}
\href{/docs/reference/introspection/location/}{location} {or}
\href{/docs/reference/foundations/function/}{function}

{Required} {{ Positional }}

\phantomsection\label{definitions-or-others-positional-tooltip}
Positional parameters are specified in order, without names.

{{ Variadic }}

\phantomsection\label{definitions-or-others-variadic-tooltip}
Variadic parameters can be specified multiple times.

The other selectors to match on.

\subsubsection{\texorpdfstring{\texttt{\ and\ }}{ and }}\label{definitions-and}

Selects all elements that match this and all of the other selectors.

self { . } { and } (

{ \hyperref[definitions-and-parameters-others]{..}
\href{/docs/reference/foundations/str/}{str}
\href{/docs/reference/foundations/regex/}{regex}
\href{/docs/reference/foundations/label/}{label}
\href{/docs/reference/foundations/selector/}{selector}
\href{/docs/reference/introspection/location/}{location}
\href{/docs/reference/foundations/function/}{function} }

) -\textgreater{} \href{/docs/reference/foundations/selector/}{selector}

\paragraph{\texorpdfstring{\texttt{\ others\ }}{ others }}\label{definitions-and-others}

\href{/docs/reference/foundations/str/}{str} {or}
\href{/docs/reference/foundations/regex/}{regex} {or}
\href{/docs/reference/foundations/label/}{label} {or}
\href{/docs/reference/foundations/selector/}{selector} {or}
\href{/docs/reference/introspection/location/}{location} {or}
\href{/docs/reference/foundations/function/}{function}

{Required} {{ Positional }}

\phantomsection\label{definitions-and-others-positional-tooltip}
Positional parameters are specified in order, without names.

{{ Variadic }}

\phantomsection\label{definitions-and-others-variadic-tooltip}
Variadic parameters can be specified multiple times.

The other selectors to match on.

\subsubsection{\texorpdfstring{\texttt{\ before\ }}{ before }}\label{definitions-before}

Returns a modified selector that will only match elements that occur
before the first match of \texttt{\ end\ } .

self { . } { before } (

{ \href{/docs/reference/foundations/label/}{label}
\href{/docs/reference/foundations/selector/}{selector}
\href{/docs/reference/introspection/location/}{location}
\href{/docs/reference/foundations/function/}{function} , } {
\hyperref[definitions-before-parameters-inclusive]{inclusive :}
\href{/docs/reference/foundations/bool/}{bool} , }

) -\textgreater{} \href{/docs/reference/foundations/selector/}{selector}

\paragraph{\texorpdfstring{\texttt{\ end\ }}{ end }}\label{definitions-before-end}

\href{/docs/reference/foundations/label/}{label} {or}
\href{/docs/reference/foundations/selector/}{selector} {or}
\href{/docs/reference/introspection/location/}{location} {or}
\href{/docs/reference/foundations/function/}{function}

{Required} {{ Positional }}

\phantomsection\label{definitions-before-end-positional-tooltip}
Positional parameters are specified in order, without names.

The original selection will end at the first match of \texttt{\ end\ } .

\paragraph{\texorpdfstring{\texttt{\ inclusive\ }}{ inclusive }}\label{definitions-before-inclusive}

\href{/docs/reference/foundations/bool/}{bool}

Whether \texttt{\ end\ } itself should match or not. This is only
relevant if both selectors match the same type of element. Defaults to
\texttt{\ }{\texttt{\ true\ }}\texttt{\ } .

Default: \texttt{\ }{\texttt{\ true\ }}\texttt{\ }

\subsubsection{\texorpdfstring{\texttt{\ after\ }}{ after }}\label{definitions-after}

Returns a modified selector that will only match elements that occur
after the first match of \texttt{\ start\ } .

self { . } { after } (

{ \href{/docs/reference/foundations/label/}{label}
\href{/docs/reference/foundations/selector/}{selector}
\href{/docs/reference/introspection/location/}{location}
\href{/docs/reference/foundations/function/}{function} , } {
\hyperref[definitions-after-parameters-inclusive]{inclusive :}
\href{/docs/reference/foundations/bool/}{bool} , }

) -\textgreater{} \href{/docs/reference/foundations/selector/}{selector}

\paragraph{\texorpdfstring{\texttt{\ start\ }}{ start }}\label{definitions-after-start}

\href{/docs/reference/foundations/label/}{label} {or}
\href{/docs/reference/foundations/selector/}{selector} {or}
\href{/docs/reference/introspection/location/}{location} {or}
\href{/docs/reference/foundations/function/}{function}

{Required} {{ Positional }}

\phantomsection\label{definitions-after-start-positional-tooltip}
Positional parameters are specified in order, without names.

The original selection will start at the first match of
\texttt{\ start\ } .

\paragraph{\texorpdfstring{\texttt{\ inclusive\ }}{ inclusive }}\label{definitions-after-inclusive}

\href{/docs/reference/foundations/bool/}{bool}

Whether \texttt{\ start\ } itself should match or not. This is only
relevant if both selectors match the same type of element. Defaults to
\texttt{\ }{\texttt{\ true\ }}\texttt{\ } .

Default: \texttt{\ }{\texttt{\ true\ }}\texttt{\ }

\href{/docs/reference/foundations/repr/}{\pandocbounded{\includesvg[keepaspectratio]{/assets/icons/16-arrow-right.svg}}}

{ Representation } { Previous page }

\href{/docs/reference/foundations/str/}{\pandocbounded{\includesvg[keepaspectratio]{/assets/icons/16-arrow-right.svg}}}

{ String } { Next page }


\title{typst.app/docs/reference/foundations/eval}

\begin{itemize}
\tightlist
\item
  \href{/docs}{\includesvg[width=0.16667in,height=0.16667in]{/assets/icons/16-docs-dark.svg}}
\item
  \includesvg[width=0.16667in,height=0.16667in]{/assets/icons/16-arrow-right.svg}
\item
  \href{/docs/reference/}{Reference}
\item
  \includesvg[width=0.16667in,height=0.16667in]{/assets/icons/16-arrow-right.svg}
\item
  \href{/docs/reference/foundations/}{Foundations}
\item
  \includesvg[width=0.16667in,height=0.16667in]{/assets/icons/16-arrow-right.svg}
\item
  \href{/docs/reference/foundations/eval/}{Evaluate}
\end{itemize}

\section{\texorpdfstring{\texttt{\ eval\ }}{ eval }}\label{summary}

Evaluates a string as Typst code.

This function should only be used as a last resort.

\subsection{Example}\label{example}

\begin{verbatim}
#eval("1 + 1") \
#eval("(1, 2, 3, 4)").len() \
#eval("*Markup!*", mode: "markup") \
\end{verbatim}

\includegraphics[width=5in,height=\textheight,keepaspectratio]{/assets/docs/KZfqDZ_7V1ElK4um94vvjwAAAAAAAAAA.png}

\subsection{\texorpdfstring{{ Parameters
}}{ Parameters }}\label{parameters}

\phantomsection\label{parameters-tooltip}
Parameters are the inputs to a function. They are specified in
parentheses after the function name.

{ eval } (

{ \href{/docs/reference/foundations/str/}{str} , } {
\hyperref[parameters-mode]{mode :}
\href{/docs/reference/foundations/str/}{str} , } {
\hyperref[parameters-scope]{scope :}
\href{/docs/reference/foundations/dictionary/}{dictionary} , }

) -\textgreater{} { any }

\subsubsection{\texorpdfstring{\texttt{\ source\ }}{ source }}\label{parameters-source}

\href{/docs/reference/foundations/str/}{str}

{Required} {{ Positional }}

\phantomsection\label{parameters-source-positional-tooltip}
Positional parameters are specified in order, without names.

A string of Typst code to evaluate.

\subsubsection{\texorpdfstring{\texttt{\ mode\ }}{ mode }}\label{parameters-mode}

\href{/docs/reference/foundations/str/}{str}

The \href{/docs/reference/syntax/\#modes}{syntactical mode} in which the
string is parsed.

\begin{longtable}[]{@{}ll@{}}
\toprule\noalign{}
Variant & Details \\
\midrule\noalign{}
\endhead
\bottomrule\noalign{}
\endlastfoot
\texttt{\ "\ code\ "\ } & Evaluate as code, as after a hash. \\
\texttt{\ "\ markup\ "\ } & Evaluate as markup, like in a Typst file. \\
\texttt{\ "\ math\ "\ } & Evaluate as math, as in an equation. \\
\end{longtable}

Default: \texttt{\ }{\texttt{\ "code"\ }}\texttt{\ }

\includesvg[width=0.16667in,height=0.16667in]{/assets/icons/16-arrow-right.svg}
View example

\begin{verbatim}
#eval("= Heading", mode: "markup")
#eval("1_2^3", mode: "math")
\end{verbatim}

\includegraphics[width=5in,height=\textheight,keepaspectratio]{/assets/docs/4OYmfbro6ZT1td5j4R5wyAAAAAAAAAAA.png}

\subsubsection{\texorpdfstring{\texttt{\ scope\ }}{ scope }}\label{parameters-scope}

\href{/docs/reference/foundations/dictionary/}{dictionary}

A scope of definitions that are made available.

Default:
\texttt{\ }{\texttt{\ (\ }}\texttt{\ }{\texttt{\ :\ }}\texttt{\ }{\texttt{\ )\ }}\texttt{\ }

\includesvg[width=0.16667in,height=0.16667in]{/assets/icons/16-arrow-right.svg}
View example

\begin{verbatim}
#eval("x + 1", scope: (x: 2)) \
#eval(
  "abc/xyz",
  mode: "math",
  scope: (
    abc: $a + b + c$,
    xyz: $x + y + z$,
  ),
)
\end{verbatim}

\includegraphics[width=5in,height=\textheight,keepaspectratio]{/assets/docs/0vD-OzSZwxX0Gqmm8_Sk9AAAAAAAAAAA.png}

\href{/docs/reference/foundations/duration/}{\pandocbounded{\includesvg[keepaspectratio]{/assets/icons/16-arrow-right.svg}}}

{ Duration } { Previous page }

\href{/docs/reference/foundations/float/}{\pandocbounded{\includesvg[keepaspectratio]{/assets/icons/16-arrow-right.svg}}}

{ Float } { Next page }


\title{typst.app/docs/reference/foundations/bool}

\begin{itemize}
\tightlist
\item
  \href{/docs}{\includesvg[width=0.16667in,height=0.16667in]{/assets/icons/16-docs-dark.svg}}
\item
  \includesvg[width=0.16667in,height=0.16667in]{/assets/icons/16-arrow-right.svg}
\item
  \href{/docs/reference/}{Reference}
\item
  \includesvg[width=0.16667in,height=0.16667in]{/assets/icons/16-arrow-right.svg}
\item
  \href{/docs/reference/foundations/}{Foundations}
\item
  \includesvg[width=0.16667in,height=0.16667in]{/assets/icons/16-arrow-right.svg}
\item
  \href{/docs/reference/foundations/bool/}{Boolean}
\end{itemize}

\section{\texorpdfstring{{ bool }}{ bool }}\label{summary}

A type with two states.

The boolean type has two values:
\texttt{\ }{\texttt{\ true\ }}\texttt{\ } and
\texttt{\ }{\texttt{\ false\ }}\texttt{\ } . It denotes whether
something is active or enabled.

\subsection{Example}\label{example}

\begin{verbatim}
#false \
#true \
#(1 < 2)
\end{verbatim}

\includegraphics[width=5in,height=\textheight,keepaspectratio]{/assets/docs/kY06WRyR--IwV2unWZl-NwAAAAAAAAAA.png}

\href{/docs/reference/foundations/auto/}{\pandocbounded{\includesvg[keepaspectratio]{/assets/icons/16-arrow-right.svg}}}

{ Auto } { Previous page }

\href{/docs/reference/foundations/bytes/}{\pandocbounded{\includesvg[keepaspectratio]{/assets/icons/16-arrow-right.svg}}}

{ Bytes } { Next page }


\title{typst.app/docs/reference/foundations/repr}

\begin{itemize}
\tightlist
\item
  \href{/docs}{\includesvg[width=0.16667in,height=0.16667in]{/assets/icons/16-docs-dark.svg}}
\item
  \includesvg[width=0.16667in,height=0.16667in]{/assets/icons/16-arrow-right.svg}
\item
  \href{/docs/reference/}{Reference}
\item
  \includesvg[width=0.16667in,height=0.16667in]{/assets/icons/16-arrow-right.svg}
\item
  \href{/docs/reference/foundations/}{Foundations}
\item
  \includesvg[width=0.16667in,height=0.16667in]{/assets/icons/16-arrow-right.svg}
\item
  \href{/docs/reference/foundations/repr/}{Representation}
\end{itemize}

\section{\texorpdfstring{\texttt{\ repr\ }}{ repr }}\label{summary}

Returns the string representation of a value.

When inserted into content, most values are displayed as this
representation in monospace with syntax-highlighting. The exceptions are
\texttt{\ }{\texttt{\ none\ }}\texttt{\ } , integers, floats, strings,
content, and functions.

\textbf{Note:} This function is for debugging purposes. Its output
should not be considered stable and may change at any time!

\subsection{Example}\label{example}

\begin{verbatim}
#none vs #repr(none) \
#"hello" vs #repr("hello") \
#(1, 2) vs #repr((1, 2)) \
#[*Hi*] vs #repr([*Hi*])
\end{verbatim}

\includegraphics[width=5in,height=\textheight,keepaspectratio]{/assets/docs/hOvQAQDTPr3WAVu4x8HkgwAAAAAAAAAA.png}

\subsection{\texorpdfstring{{ Parameters
}}{ Parameters }}\label{parameters}

\phantomsection\label{parameters-tooltip}
Parameters are the inputs to a function. They are specified in
parentheses after the function name.

{ repr } (

{ { any } }

) -\textgreater{} \href{/docs/reference/foundations/str/}{str}

\subsubsection{\texorpdfstring{\texttt{\ value\ }}{ value }}\label{parameters-value}

{ any }

{Required} {{ Positional }}

\phantomsection\label{parameters-value-positional-tooltip}
Positional parameters are specified in order, without names.

The value whose string representation to produce.

\href{/docs/reference/foundations/regex/}{\pandocbounded{\includesvg[keepaspectratio]{/assets/icons/16-arrow-right.svg}}}

{ Regex } { Previous page }

\href{/docs/reference/foundations/selector/}{\pandocbounded{\includesvg[keepaspectratio]{/assets/icons/16-arrow-right.svg}}}

{ Selector } { Next page }


\title{typst.app/docs/reference/foundations/auto}

\begin{itemize}
\tightlist
\item
  \href{/docs}{\includesvg[width=0.16667in,height=0.16667in]{/assets/icons/16-docs-dark.svg}}
\item
  \includesvg[width=0.16667in,height=0.16667in]{/assets/icons/16-arrow-right.svg}
\item
  \href{/docs/reference/}{Reference}
\item
  \includesvg[width=0.16667in,height=0.16667in]{/assets/icons/16-arrow-right.svg}
\item
  \href{/docs/reference/foundations/}{Foundations}
\item
  \includesvg[width=0.16667in,height=0.16667in]{/assets/icons/16-arrow-right.svg}
\item
  \href{/docs/reference/foundations/auto/}{Auto}
\end{itemize}

\section{\texorpdfstring{{ auto }}{ auto }}\label{summary}

A value that indicates a smart default.

The auto type has exactly one value:
\texttt{\ }{\texttt{\ auto\ }}\texttt{\ } .

Parameters that support the \texttt{\ }{\texttt{\ auto\ }}\texttt{\ }
value have some smart default or contextual behaviour. A good example is
the \href{/docs/reference/text/text/\#parameters-dir}{text direction}
parameter. Setting it to \texttt{\ }{\texttt{\ auto\ }}\texttt{\ } lets
Typst automatically determine the direction from the
\href{/docs/reference/text/text/\#parameters-lang}{text language} .

\href{/docs/reference/foundations/assert/}{\pandocbounded{\includesvg[keepaspectratio]{/assets/icons/16-arrow-right.svg}}}

{ Assert } { Previous page }

\href{/docs/reference/foundations/bool/}{\pandocbounded{\includesvg[keepaspectratio]{/assets/icons/16-arrow-right.svg}}}

{ Boolean } { Next page }


\title{typst.app/docs/reference/foundations/datetime}

\begin{itemize}
\tightlist
\item
  \href{/docs}{\includesvg[width=0.16667in,height=0.16667in]{/assets/icons/16-docs-dark.svg}}
\item
  \includesvg[width=0.16667in,height=0.16667in]{/assets/icons/16-arrow-right.svg}
\item
  \href{/docs/reference/}{Reference}
\item
  \includesvg[width=0.16667in,height=0.16667in]{/assets/icons/16-arrow-right.svg}
\item
  \href{/docs/reference/foundations/}{Foundations}
\item
  \includesvg[width=0.16667in,height=0.16667in]{/assets/icons/16-arrow-right.svg}
\item
  \href{/docs/reference/foundations/datetime/}{Datetime}
\end{itemize}

\section{\texorpdfstring{{ datetime }}{ datetime }}\label{summary}

Represents a date, a time, or a combination of both.

Can be created by either specifying a custom datetime using this
type\textquotesingle s constructor function or getting the current date
with
\href{/docs/reference/foundations/datetime/\#definitions-today}{\texttt{\ datetime.today\ }}
.

\subsection{Example}\label{example}

\begin{verbatim}
#let date = datetime(
  year: 2020,
  month: 10,
  day: 4,
)

#date.display() \
#date.display(
  "y:[year repr:last_two]"
)

#let time = datetime(
  hour: 18,
  minute: 2,
  second: 23,
)

#time.display() \
#time.display(
  "h:[hour repr:12][period]"
)
\end{verbatim}

\includegraphics[width=5in,height=\textheight,keepaspectratio]{/assets/docs/aJRkqg11vpsxBq0NzqAo0gAAAAAAAAAA.png}

\subsection{Datetime and Duration}\label{datetime-and-duration}

You can get a \href{/docs/reference/foundations/duration/}{duration} by
subtracting two datetime:

\begin{verbatim}
#let first-of-march = datetime(day: 1, month: 3, year: 2024)
#let first-of-jan = datetime(day: 1, month: 1, year: 2024)
#let distance = first-of-march - first-of-jan
#distance.hours()
\end{verbatim}

\includegraphics[width=5in,height=\textheight,keepaspectratio]{/assets/docs/xJIPnvV5Iiw8osdkiAUb_AAAAAAAAAAA.png}

You can also add/subtract a datetime and a duration to retrieve a new,
offset datetime:

\begin{verbatim}
#let date = datetime(day: 1, month: 3, year: 2024)
#let two-days = duration(days: 2)
#let two-days-earlier = date - two-days
#let two-days-later = date + two-days

#date.display() \
#two-days-earlier.display() \
#two-days-later.display()
\end{verbatim}

\includegraphics[width=5in,height=\textheight,keepaspectratio]{/assets/docs/R-BPj6xQMFasAxM1n3h_iwAAAAAAAAAA.png}

\subsection{Format}\label{format}

You can specify a customized formatting using the
\href{/docs/reference/foundations/datetime/\#definitions-display}{\texttt{\ display\ }}
method. The format of a datetime is specified by providing
\emph{components} with a specified number of \emph{modifiers} . A
component represents a certain part of the datetime that you want to
display, and with the help of modifiers you can define how you want to
display that component. In order to display a component, you wrap the
name of the component in square brackets (e.g. \texttt{\ {[}year{]}\ }
will display the year). In order to add modifiers, you add a space after
the component name followed by the name of the modifier, a colon and the
value of the modifier (e.g. \texttt{\ {[}month\ repr:short{]}\ } will
display the short representation of the month).

The possible combination of components and their respective modifiers is
as follows:

\begin{itemize}
\tightlist
\item
  \texttt{\ year\ } : Displays the year of the datetime.

  \begin{itemize}
  \tightlist
  \item
    \texttt{\ padding\ } : Can be either \texttt{\ zero\ } ,
    \texttt{\ space\ } or \texttt{\ none\ } . Specifies how the year is
    padded.
  \item
    \texttt{\ repr\ } Can be either \texttt{\ full\ } in which case the
    full year is displayed or \texttt{\ last\_two\ } in which case only
    the last two digits are displayed.
  \item
    \texttt{\ sign\ } : Can be either \texttt{\ automatic\ } or
    \texttt{\ mandatory\ } . Specifies when the sign should be
    displayed.
  \end{itemize}
\item
  \texttt{\ month\ } : Displays the month of the datetime.

  \begin{itemize}
  \tightlist
  \item
    \texttt{\ padding\ } : Can be either \texttt{\ zero\ } ,
    \texttt{\ space\ } or \texttt{\ none\ } . Specifies how the month is
    padded.
  \item
    \texttt{\ repr\ } : Can be either \texttt{\ numerical\ } ,
    \texttt{\ long\ } or \texttt{\ short\ } . Specifies if the month
    should be displayed as a number or a word. Unfortunately, when
    choosing the word representation, it can currently only display the
    English version. In the future, it is planned to support
    localization.
  \end{itemize}
\item
  \texttt{\ day\ } : Displays the day of the datetime.

  \begin{itemize}
  \tightlist
  \item
    \texttt{\ padding\ } : Can be either \texttt{\ zero\ } ,
    \texttt{\ space\ } or \texttt{\ none\ } . Specifies how the day is
    padded.
  \end{itemize}
\item
  \texttt{\ week\_number\ } : Displays the week number of the datetime.

  \begin{itemize}
  \tightlist
  \item
    \texttt{\ padding\ } : Can be either \texttt{\ zero\ } ,
    \texttt{\ space\ } or \texttt{\ none\ } . Specifies how the week
    number is padded.
  \item
    \texttt{\ repr\ } : Can be either \texttt{\ ISO\ } ,
    \texttt{\ sunday\ } or \texttt{\ monday\ } . In the case of
    \texttt{\ ISO\ } , week numbers are between 1 and 53, while the
    other ones are between 0 and 53.
  \end{itemize}
\item
  \texttt{\ weekday\ } : Displays the weekday of the date.

  \begin{itemize}
  \tightlist
  \item
    \texttt{\ repr\ } Can be either \texttt{\ long\ } ,
    \texttt{\ short\ } , \texttt{\ sunday\ } or \texttt{\ monday\ } . In
    the case of \texttt{\ long\ } and \texttt{\ short\ } , the
    corresponding English name will be displayed (same as for the month,
    other languages are currently not supported). In the case of
    \texttt{\ sunday\ } and \texttt{\ monday\ } , the numerical value
    will be displayed (assuming Sunday and Monday as the first day of
    the week, respectively).
  \item
    \texttt{\ one\_indexed\ } : Can be either \texttt{\ true\ } or
    \texttt{\ false\ } . Defines whether the numerical representation of
    the week starts with 0 or 1.
  \end{itemize}
\item
  \texttt{\ hour\ } : Displays the hour of the date.

  \begin{itemize}
  \tightlist
  \item
    \texttt{\ padding\ } : Can be either \texttt{\ zero\ } ,
    \texttt{\ space\ } or \texttt{\ none\ } . Specifies how the hour is
    padded.
  \item
    \texttt{\ repr\ } : Can be either \texttt{\ 24\ } or \texttt{\ 12\ }
    . Changes whether the hour is displayed in the 24-hour or 12-hour
    format.
  \end{itemize}
\item
  \texttt{\ period\ } : The AM/PM part of the hour

  \begin{itemize}
  \tightlist
  \item
    \texttt{\ case\ } : Can be \texttt{\ lower\ } to display it in lower
    case and \texttt{\ upper\ } to display it in upper case.
  \end{itemize}
\item
  \texttt{\ minute\ } : Displays the minute of the date.

  \begin{itemize}
  \tightlist
  \item
    \texttt{\ padding\ } : Can be either \texttt{\ zero\ } ,
    \texttt{\ space\ } or \texttt{\ none\ } . Specifies how the minute
    is padded.
  \end{itemize}
\item
  \texttt{\ second\ } : Displays the second of the date.

  \begin{itemize}
  \tightlist
  \item
    \texttt{\ padding\ } : Can be either \texttt{\ zero\ } ,
    \texttt{\ space\ } or \texttt{\ none\ } . Specifies how the second
    is padded.
  \end{itemize}
\end{itemize}

Keep in mind that not always all components can be used. For example, if
you create a new datetime with
\texttt{\ }{\texttt{\ datetime\ }}\texttt{\ }{\texttt{\ (\ }}\texttt{\ year\ }{\texttt{\ :\ }}\texttt{\ }{\texttt{\ 2023\ }}\texttt{\ }{\texttt{\ ,\ }}\texttt{\ month\ }{\texttt{\ :\ }}\texttt{\ }{\texttt{\ 10\ }}\texttt{\ }{\texttt{\ ,\ }}\texttt{\ day\ }{\texttt{\ :\ }}\texttt{\ }{\texttt{\ 13\ }}\texttt{\ }{\texttt{\ )\ }}\texttt{\ }
, it will be stored as a plain date internally, meaning that you cannot
use components such as \texttt{\ hour\ } or \texttt{\ minute\ } , which
would only work on datetimes that have a specified time.

\subsection{\texorpdfstring{Constructor
{}}{Constructor }}\label{constructor}

\phantomsection\label{constructor-constructor-tooltip}
If a type has a constructor, you can call it like a function to create a
new value of the type.

Creates a new datetime.

You can specify the
\href{/docs/reference/foundations/datetime/}{datetime} using a year,
month, day, hour, minute, and second.

\emph{Note} : Depending on which components of the datetime you specify,
Typst will store it in one of the following three ways:

\begin{itemize}
\tightlist
\item
  If you specify year, month and day, Typst will store just a date.
\item
  If you specify hour, minute and second, Typst will store just a time.
\item
  If you specify all of year, month, day, hour, minute and second, Typst
  will store a full datetime.
\end{itemize}

Depending on how it is stored, the
\href{/docs/reference/foundations/datetime/\#definitions-display}{\texttt{\ display\ }}
method will choose a different formatting by default.

{ datetime } (

{ \hyperref[constructor-parameters-year]{year :}
\href{/docs/reference/foundations/int/}{int} , } {
\hyperref[constructor-parameters-month]{month :}
\href{/docs/reference/foundations/int/}{int} , } {
\hyperref[constructor-parameters-day]{day :}
\href{/docs/reference/foundations/int/}{int} , } {
\hyperref[constructor-parameters-hour]{hour :}
\href{/docs/reference/foundations/int/}{int} , } {
\hyperref[constructor-parameters-minute]{minute :}
\href{/docs/reference/foundations/int/}{int} , } {
\hyperref[constructor-parameters-second]{second :}
\href{/docs/reference/foundations/int/}{int} , }

) -\textgreater{} \href{/docs/reference/foundations/datetime/}{datetime}

\begin{verbatim}
#datetime(
  year: 2012,
  month: 8,
  day: 3,
).display()
\end{verbatim}

\includegraphics[width=5in,height=\textheight,keepaspectratio]{/assets/docs/6mpnNRypNysjXvXstSouiwAAAAAAAAAA.png}

\paragraph{\texorpdfstring{\texttt{\ year\ }}{ year }}\label{constructor-year}

\href{/docs/reference/foundations/int/}{int}

The year of the datetime.

\paragraph{\texorpdfstring{\texttt{\ month\ }}{ month }}\label{constructor-month}

\href{/docs/reference/foundations/int/}{int}

The month of the datetime.

\paragraph{\texorpdfstring{\texttt{\ day\ }}{ day }}\label{constructor-day}

\href{/docs/reference/foundations/int/}{int}

The day of the datetime.

\paragraph{\texorpdfstring{\texttt{\ hour\ }}{ hour }}\label{constructor-hour}

\href{/docs/reference/foundations/int/}{int}

The hour of the datetime.

\paragraph{\texorpdfstring{\texttt{\ minute\ }}{ minute }}\label{constructor-minute}

\href{/docs/reference/foundations/int/}{int}

The minute of the datetime.

\paragraph{\texorpdfstring{\texttt{\ second\ }}{ second }}\label{constructor-second}

\href{/docs/reference/foundations/int/}{int}

The second of the datetime.

\subsection{\texorpdfstring{{ Definitions
}}{ Definitions }}\label{definitions}

\phantomsection\label{definitions-tooltip}
Functions and types and can have associated definitions. These are
accessed by specifying the function or type, followed by a period, and
then the definition\textquotesingle s name.

\subsubsection{\texorpdfstring{\texttt{\ today\ }}{ today }}\label{definitions-today}

Returns the current date.

datetime { . } { today } (

{ \hyperref[definitions-today-parameters-offset]{offset :}
\href{/docs/reference/foundations/auto/}{auto}
\href{/docs/reference/foundations/int/}{int} }

) -\textgreater{} \href{/docs/reference/foundations/datetime/}{datetime}

\begin{verbatim}
Today's date is
#datetime.today().display().
\end{verbatim}

\includegraphics[width=5in,height=\textheight,keepaspectratio]{/assets/docs/SOSDKByfy_YbHbk7NejgOQAAAAAAAAAA.png}

\paragraph{\texorpdfstring{\texttt{\ offset\ }}{ offset }}\label{definitions-today-offset}

\href{/docs/reference/foundations/auto/}{auto} {or}
\href{/docs/reference/foundations/int/}{int}

An offset to apply to the current UTC date. If set to
\texttt{\ }{\texttt{\ auto\ }}\texttt{\ } , the offset will be the local
offset.

Default: \texttt{\ }{\texttt{\ auto\ }}\texttt{\ }

\subsubsection{\texorpdfstring{\texttt{\ display\ }}{ display }}\label{definitions-display}

Displays the datetime in a specified format.

Depending on whether you have defined just a date, a time or both, the
default format will be different. If you specified a date, it will be
\texttt{\ {[}year{]}-{[}month{]}-{[}day{]}\ } . If you specified a time,
it will be \texttt{\ {[}hour{]}:{[}minute{]}:{[}second{]}\ } . In the
case of a datetime, it will be
\texttt{\ {[}year{]}-{[}month{]}-{[}day{]}\ {[}hour{]}:{[}minute{]}:{[}second{]}\ }
.

See the \href{/docs/reference/foundations/datetime/\#format}{format
syntax} for more information.

self { . } { display } (

{ \href{/docs/reference/foundations/auto/}{auto}
\href{/docs/reference/foundations/str/}{str} }

) -\textgreater{} \href{/docs/reference/foundations/str/}{str}

\paragraph{\texorpdfstring{\texttt{\ pattern\ }}{ pattern }}\label{definitions-display-pattern}

\href{/docs/reference/foundations/auto/}{auto} {or}
\href{/docs/reference/foundations/str/}{str}

{{ Positional }}

\phantomsection\label{definitions-display-pattern-positional-tooltip}
Positional parameters are specified in order, without names.

The format used to display the datetime.

Default: \texttt{\ }{\texttt{\ auto\ }}\texttt{\ }

\subsubsection{\texorpdfstring{\texttt{\ year\ }}{ year }}\label{definitions-year}

The year if it was specified, or
\texttt{\ }{\texttt{\ none\ }}\texttt{\ } for times without a date.

self { . } { year } (

) -\textgreater{} \href{/docs/reference/foundations/none/}{none}
\href{/docs/reference/foundations/int/}{int}

\subsubsection{\texorpdfstring{\texttt{\ month\ }}{ month }}\label{definitions-month}

The month if it was specified, or
\texttt{\ }{\texttt{\ none\ }}\texttt{\ } for times without a date.

self { . } { month } (

) -\textgreater{} \href{/docs/reference/foundations/none/}{none}
\href{/docs/reference/foundations/int/}{int}

\subsubsection{\texorpdfstring{\texttt{\ weekday\ }}{ weekday }}\label{definitions-weekday}

The weekday (counting Monday as 1) or
\texttt{\ }{\texttt{\ none\ }}\texttt{\ } for times without a date.

self { . } { weekday } (

) -\textgreater{} \href{/docs/reference/foundations/none/}{none}
\href{/docs/reference/foundations/int/}{int}

\subsubsection{\texorpdfstring{\texttt{\ day\ }}{ day }}\label{definitions-day}

The day if it was specified, or
\texttt{\ }{\texttt{\ none\ }}\texttt{\ } for times without a date.

self { . } { day } (

) -\textgreater{} \href{/docs/reference/foundations/none/}{none}
\href{/docs/reference/foundations/int/}{int}

\subsubsection{\texorpdfstring{\texttt{\ hour\ }}{ hour }}\label{definitions-hour}

The hour if it was specified, or
\texttt{\ }{\texttt{\ none\ }}\texttt{\ } for dates without a time.

self { . } { hour } (

) -\textgreater{} \href{/docs/reference/foundations/none/}{none}
\href{/docs/reference/foundations/int/}{int}

\subsubsection{\texorpdfstring{\texttt{\ minute\ }}{ minute }}\label{definitions-minute}

The minute if it was specified, or
\texttt{\ }{\texttt{\ none\ }}\texttt{\ } for dates without a time.

self { . } { minute } (

) -\textgreater{} \href{/docs/reference/foundations/none/}{none}
\href{/docs/reference/foundations/int/}{int}

\subsubsection{\texorpdfstring{\texttt{\ second\ }}{ second }}\label{definitions-second}

The second if it was specified, or
\texttt{\ }{\texttt{\ none\ }}\texttt{\ } for dates without a time.

self { . } { second } (

) -\textgreater{} \href{/docs/reference/foundations/none/}{none}
\href{/docs/reference/foundations/int/}{int}

\subsubsection{\texorpdfstring{\texttt{\ ordinal\ }}{ ordinal }}\label{definitions-ordinal}

The ordinal (day of the year), or
\texttt{\ }{\texttt{\ none\ }}\texttt{\ } for times without a date.

self { . } { ordinal } (

) -\textgreater{} \href{/docs/reference/foundations/none/}{none}
\href{/docs/reference/foundations/int/}{int}

\href{/docs/reference/foundations/content/}{\pandocbounded{\includesvg[keepaspectratio]{/assets/icons/16-arrow-right.svg}}}

{ Content } { Previous page }

\href{/docs/reference/foundations/decimal/}{\pandocbounded{\includesvg[keepaspectratio]{/assets/icons/16-arrow-right.svg}}}

{ Decimal } { Next page }


\title{typst.app/docs/reference/foundations/sys}

\begin{itemize}
\tightlist
\item
  \href{/docs}{\includesvg[width=0.16667in,height=0.16667in]{/assets/icons/16-docs-dark.svg}}
\item
  \includesvg[width=0.16667in,height=0.16667in]{/assets/icons/16-arrow-right.svg}
\item
  \href{/docs/reference/}{Reference}
\item
  \includesvg[width=0.16667in,height=0.16667in]{/assets/icons/16-arrow-right.svg}
\item
  \href{/docs/reference/foundations/}{Foundations}
\item
  \includesvg[width=0.16667in,height=0.16667in]{/assets/icons/16-arrow-right.svg}
\item
  \href{/docs/reference/foundations/sys}{System}
\end{itemize}

\section{System}\label{summary}

Module for system interactions.

This module defines the following items:

\begin{itemize}
\item
  The \texttt{\ sys.version\ } constant (of type
  \href{/docs/reference/foundations/version/}{\texttt{\ version\ }} )
  that specifies the currently active Typst compiler version.
\item
  The \texttt{\ sys.inputs\ }
  \href{/docs/reference/foundations/dictionary/}{dictionary} , which
  makes external inputs available to the project. An input specified in
  the command line as \texttt{\ -\/-input\ key=value\ } becomes
  available under \texttt{\ sys.inputs.key\ } as
  \texttt{\ }{\texttt{\ "value"\ }}\texttt{\ } . To include spaces in
  the value, it may be enclosed with single or double quotes.

  The value is always of type
  \href{/docs/reference/foundations/str/}{string} . More complex data
  may be parsed manually using functions like
  \href{/docs/reference/data-loading/json/\#definitions-decode}{\texttt{\ json.decode\ }}
  .
\end{itemize}

\href{/docs/reference/foundations/style/}{\pandocbounded{\includesvg[keepaspectratio]{/assets/icons/16-arrow-right.svg}}}

{ Style } { Previous page }

\href{/docs/reference/foundations/type/}{\pandocbounded{\includesvg[keepaspectratio]{/assets/icons/16-arrow-right.svg}}}

{ Type } { Next page }


\title{typst.app/docs/reference/foundations/style}

\begin{itemize}
\tightlist
\item
  \href{/docs}{\includesvg[width=0.16667in,height=0.16667in]{/assets/icons/16-docs-dark.svg}}
\item
  \includesvg[width=0.16667in,height=0.16667in]{/assets/icons/16-arrow-right.svg}
\item
  \href{/docs/reference/}{Reference}
\item
  \includesvg[width=0.16667in,height=0.16667in]{/assets/icons/16-arrow-right.svg}
\item
  \href{/docs/reference/foundations/}{Foundations}
\item
  \includesvg[width=0.16667in,height=0.16667in]{/assets/icons/16-arrow-right.svg}
\item
  \href{/docs/reference/foundations/style/}{Style}
\end{itemize}

\section{\texorpdfstring{\texttt{\ style\ }}{ style }}\label{summary}

Provides access to active styles.

\textbf{Deprecation planned.} Use
\href{/docs/reference/context/}{context} instead.

\begin{verbatim}
#let thing(body) = style(styles => {
  let size = measure(body, styles)
  [Width of "#body" is #size.width]
})

#thing[Hey] \
#thing[Welcome]
\end{verbatim}

\includegraphics[width=5in,height=\textheight,keepaspectratio]{/assets/docs/B9BBPtfgbYihKwTWFPTllQAAAAAAAAAA.png}

\subsection{\texorpdfstring{{ Parameters
}}{ Parameters }}\label{parameters}

\phantomsection\label{parameters-tooltip}
Parameters are the inputs to a function. They are specified in
parentheses after the function name.

{ style } (

{ \href{/docs/reference/foundations/function/}{function} }

) -\textgreater{} \href{/docs/reference/foundations/content/}{content}

\subsubsection{\texorpdfstring{\texttt{\ func\ }}{ func }}\label{parameters-func}

\href{/docs/reference/foundations/function/}{function}

{Required} {{ Positional }}

\phantomsection\label{parameters-func-positional-tooltip}
Positional parameters are specified in order, without names.

A function to call with the styles. Its return value is displayed in the
document.

This function is called once for each time the content returned by
\texttt{\ style\ } appears in the document. That makes it possible to
generate content that depends on the style context it appears in.

\href{/docs/reference/foundations/str/}{\pandocbounded{\includesvg[keepaspectratio]{/assets/icons/16-arrow-right.svg}}}

{ String } { Previous page }

\href{/docs/reference/foundations/sys/}{\pandocbounded{\includesvg[keepaspectratio]{/assets/icons/16-arrow-right.svg}}}

{ System } { Next page }


\title{typst.app/docs/reference/foundations/decimal}

\begin{itemize}
\tightlist
\item
  \href{/docs}{\includesvg[width=0.16667in,height=0.16667in]{/assets/icons/16-docs-dark.svg}}
\item
  \includesvg[width=0.16667in,height=0.16667in]{/assets/icons/16-arrow-right.svg}
\item
  \href{/docs/reference/}{Reference}
\item
  \includesvg[width=0.16667in,height=0.16667in]{/assets/icons/16-arrow-right.svg}
\item
  \href{/docs/reference/foundations/}{Foundations}
\item
  \includesvg[width=0.16667in,height=0.16667in]{/assets/icons/16-arrow-right.svg}
\item
  \href{/docs/reference/foundations/decimal/}{Decimal}
\end{itemize}

\section{\texorpdfstring{{ decimal }}{ decimal }}\label{summary}

A fixed-point decimal number type.

This type should be used for precise arithmetic operations on numbers
represented in base 10. A typical use case is representing currency.

\subsection{Example}\label{example}

\begin{verbatim}
Decimal: #(decimal("0.1") + decimal("0.2")) \
Float: #(0.1 + 0.2)
\end{verbatim}

\includegraphics[width=5in,height=\textheight,keepaspectratio]{/assets/docs/W31Kvh6BvfIgTgIeq2uIEQAAAAAAAAAA.png}

\subsection{Construction and casts}\label{construction-and-casts}

To create a decimal number, use the
\texttt{\ }{\texttt{\ decimal\ }}\texttt{\ }{\texttt{\ (\ }}\texttt{\ string\ }{\texttt{\ )\ }}\texttt{\ }
constructor, such as in
\texttt{\ }{\texttt{\ decimal\ }}\texttt{\ }{\texttt{\ (\ }}\texttt{\ }{\texttt{\ "3.141592653"\ }}\texttt{\ }{\texttt{\ )\ }}\texttt{\ }
\textbf{(note the double quotes!)} . This constructor preserves all
given fractional digits, provided they are representable as per the
limits specified below (otherwise, an error is raised).

You can also convert any
\href{/docs/reference/foundations/int/}{integer} to a decimal with the
\texttt{\ }{\texttt{\ decimal\ }}\texttt{\ }{\texttt{\ (\ }}\texttt{\ int\ }{\texttt{\ )\ }}\texttt{\ }
constructor, e.g.
\texttt{\ }{\texttt{\ decimal\ }}\texttt{\ }{\texttt{\ (\ }}\texttt{\ }{\texttt{\ 59\ }}\texttt{\ }{\texttt{\ )\ }}\texttt{\ }
. However, note that constructing a decimal from a
\href{/docs/reference/foundations/float/}{floating-point number} , while
supported, \textbf{is an imprecise conversion and therefore
discouraged.} A warning will be raised if Typst detects that there was
an accidental \texttt{\ float\ } to \texttt{\ decimal\ } cast through
its constructor, e.g. if writing
\texttt{\ }{\texttt{\ decimal\ }}\texttt{\ }{\texttt{\ (\ }}\texttt{\ }{\texttt{\ 3.14\ }}\texttt{\ }{\texttt{\ )\ }}\texttt{\ }
(note the lack of double quotes, indicating this is an accidental
\texttt{\ float\ } cast and therefore imprecise). It is recommended to
use strings for constant decimal values instead (e.g.
\texttt{\ }{\texttt{\ decimal\ }}\texttt{\ }{\texttt{\ (\ }}\texttt{\ }{\texttt{\ "3.14"\ }}\texttt{\ }{\texttt{\ )\ }}\texttt{\ }
).

The precision of a \texttt{\ float\ } to \texttt{\ decimal\ } cast can
be slightly improved by rounding the result to 15 digits with
\href{/docs/reference/foundations/calc/\#functions-round}{\texttt{\ calc.round\ }}
, but there are still no precision guarantees for that kind of
conversion.

\subsection{Operations}\label{operations}

Basic arithmetic operations are supported on two decimals and on pairs
of decimals and integers.

Built-in operations between \texttt{\ float\ } and \texttt{\ decimal\ }
are not supported in order to guard against accidental loss of
precision. They will raise an error instead.

Certain \texttt{\ calc\ } functions, such as trigonometric functions and
power between two real numbers, are also only supported for
\texttt{\ float\ } (although raising \texttt{\ decimal\ } to integer
exponents is supported). You can opt into potentially imprecise
operations with the
\texttt{\ }{\texttt{\ float\ }}\texttt{\ }{\texttt{\ (\ }}\texttt{\ decimal\ }{\texttt{\ )\ }}\texttt{\ }
constructor, which casts the \texttt{\ decimal\ } number into a
\texttt{\ float\ } , allowing for operations without precision
guarantees.

\subsection{Displaying decimals}\label{displaying-decimals}

To display a decimal, simply insert the value into the document. To only
display a certain number of digits,
\href{/docs/reference/foundations/calc/\#functions-round}{round} the
decimal first. Localized formatting of decimals and other numbers is not
yet supported, but planned for the future.

You can convert decimals to strings using the
\href{/docs/reference/foundations/str/}{\texttt{\ str\ }} constructor.
This way, you can post-process the displayed representation, e.g. to
replace the period with a comma (as a stand-in for proper built-in
localization to languages that use the comma).

\subsection{Precision and limits}\label{precision-and-limits}

A \texttt{\ decimal\ } number has a limit of 28 to 29 significant
base-10 digits. This includes the sum of digits before and after the
decimal point. As such, numbers with more fractional digits have a
smaller range. The maximum and minimum \texttt{\ decimal\ } numbers have
a value of
\texttt{\ }{\texttt{\ 79228162514264337593543950335\ }}\texttt{\ } and
\texttt{\ }{\texttt{\ -\ }}\texttt{\ }{\texttt{\ 79228162514264337593543950335\ }}\texttt{\ }
respectively. In contrast with
\href{/docs/reference/foundations/float/}{\texttt{\ float\ }} , this
type does not support infinity or NaN, so overflowing or underflowing
operations will raise an error.

Typical operations between \texttt{\ decimal\ } numbers, such as
addition, multiplication, and
\href{/docs/reference/foundations/calc/\#functions-pow}{power} to an
integer, will be highly precise due to their fixed-point representation.
Note, however, that multiplication and division may not preserve all
digits in some edge cases: while they are considered precise, digits
past the limits specified above are rounded off and lost, so some loss
of precision beyond the maximum representable digits is possible. Note
that this behavior can be observed not only when dividing, but also when
multiplying by numbers between 0 and 1, as both operations can push a
number\textquotesingle s fractional digits beyond the limits described
above, leading to rounding. When those two operations do not surpass the
digit limits, they are fully precise.

\subsection{\texorpdfstring{Constructor
{}}{Constructor }}\label{constructor}

\phantomsection\label{constructor-constructor-tooltip}
If a type has a constructor, you can call it like a function to create a
new value of the type.

Converts a value to a \texttt{\ decimal\ } .

It is recommended to use a string to construct the decimal number, or an
\href{/docs/reference/foundations/int/}{integer} (if desired). The
string must contain a number in the format
\texttt{\ }{\texttt{\ "3.14159"\ }}\texttt{\ } (or
\texttt{\ }{\texttt{\ "-3.141519"\ }}\texttt{\ } for negative numbers).
The fractional digits are fully preserved; if that\textquotesingle s not
possible due to the limit of significant digits (around 28 to 29) having
been reached, an error is raised as the given decimal number
wouldn\textquotesingle t be representable.

While this constructor can be used with
\href{/docs/reference/foundations/float/}{floating-point numbers} to
cast them to \texttt{\ decimal\ } , doing so is \textbf{discouraged} as
\textbf{this cast is inherently imprecise.} It is easy to accidentally
perform this cast by writing
\texttt{\ }{\texttt{\ decimal\ }}\texttt{\ }{\texttt{\ (\ }}\texttt{\ }{\texttt{\ 1.234\ }}\texttt{\ }{\texttt{\ )\ }}\texttt{\ }
(note the lack of double quotes), which is why Typst will emit a warning
in that case. Please write
\texttt{\ }{\texttt{\ decimal\ }}\texttt{\ }{\texttt{\ (\ }}\texttt{\ }{\texttt{\ "1.234"\ }}\texttt{\ }{\texttt{\ )\ }}\texttt{\ }
instead for that particular case (initialization of a constant decimal).
Also note that floats that are NaN or infinite cannot be cast to
decimals and will raise an error.

{ decimal } (

{ \href{/docs/reference/foundations/int/}{int}
\href{/docs/reference/foundations/float/}{float}
\href{/docs/reference/foundations/str/}{str} }

) -\textgreater{} \href{/docs/reference/foundations/decimal/}{decimal}

\begin{verbatim}
#decimal("1.222222222222222")
\end{verbatim}

\includegraphics[width=5in,height=\textheight,keepaspectratio]{/assets/docs/RfqlB85Q5lIVeebJq7RlmgAAAAAAAAAA.png}

\paragraph{\texorpdfstring{\texttt{\ value\ }}{ value }}\label{constructor-value}

\href{/docs/reference/foundations/int/}{int} {or}
\href{/docs/reference/foundations/float/}{float} {or}
\href{/docs/reference/foundations/str/}{str}

{Required} {{ Positional }}

\phantomsection\label{constructor-value-positional-tooltip}
Positional parameters are specified in order, without names.

The value that should be converted to a decimal.

\href{/docs/reference/foundations/datetime/}{\pandocbounded{\includesvg[keepaspectratio]{/assets/icons/16-arrow-right.svg}}}

{ Datetime } { Previous page }

\href{/docs/reference/foundations/dictionary/}{\pandocbounded{\includesvg[keepaspectratio]{/assets/icons/16-arrow-right.svg}}}

{ Dictionary } { Next page }


\title{typst.app/docs/reference/foundations/arguments}

\begin{itemize}
\tightlist
\item
  \href{/docs}{\includesvg[width=0.16667in,height=0.16667in]{/assets/icons/16-docs-dark.svg}}
\item
  \includesvg[width=0.16667in,height=0.16667in]{/assets/icons/16-arrow-right.svg}
\item
  \href{/docs/reference/}{Reference}
\item
  \includesvg[width=0.16667in,height=0.16667in]{/assets/icons/16-arrow-right.svg}
\item
  \href{/docs/reference/foundations/}{Foundations}
\item
  \includesvg[width=0.16667in,height=0.16667in]{/assets/icons/16-arrow-right.svg}
\item
  \href{/docs/reference/foundations/arguments/}{Arguments}
\end{itemize}

\section{\texorpdfstring{{ arguments }}{ arguments }}\label{summary}

Captured arguments to a function.

\subsection{Argument Sinks}\label{argument-sinks}

Like built-in functions, custom functions can also take a variable
number of arguments. You can specify an \emph{argument sink} which
collects all excess arguments as \texttt{\ ..sink\ } . The resulting
\texttt{\ sink\ } value is of the \texttt{\ arguments\ } type. It
exposes methods to access the positional and named arguments.

\begin{verbatim}
#let format(title, ..authors) = {
  let by = authors
    .pos()
    .join(", ", last: " and ")

  [*#title* \ _Written by #by;_]
}

#format("ArtosFlow", "Jane", "Joe")
\end{verbatim}

\includegraphics[width=5in,height=\textheight,keepaspectratio]{/assets/docs/DWzn69gGuCd1q_LVZvjEEgAAAAAAAAAA.png}

\subsection{Spreading}\label{spreading}

Inversely to an argument sink, you can \emph{spread} arguments, arrays
and dictionaries into a function call with the \texttt{\ ..spread\ }
operator:

\begin{verbatim}
#let array = (2, 3, 5)
#calc.min(..array)
#let dict = (fill: blue)
#text(..dict)[Hello]
\end{verbatim}

\includegraphics[width=5in,height=\textheight,keepaspectratio]{/assets/docs/kcmqtH9qxq6Bg8ZwwKnMCQAAAAAAAAAA.png}

\subsection{\texorpdfstring{Constructor
{}}{Constructor }}\label{constructor}

\phantomsection\label{constructor-constructor-tooltip}
If a type has a constructor, you can call it like a function to create a
new value of the type.

Construct spreadable arguments in place.

This function behaves like
\texttt{\ }{\texttt{\ let\ }}\texttt{\ }{\texttt{\ args\ }}\texttt{\ }{\texttt{\ (\ }}\texttt{\ }{\texttt{\ ..\ }}\texttt{\ sink\ }{\texttt{\ )\ }}\texttt{\ }{\texttt{\ =\ }}\texttt{\ sink\ }
.

{ arguments } (

{ \hyperref[constructor-parameters-arguments]{..} { any } }

) -\textgreater{}
\href{/docs/reference/foundations/arguments/}{arguments}

\begin{verbatim}
#let args = arguments(stroke: red, inset: 1em, [Body])
#box(..args)
\end{verbatim}

\includegraphics[width=5in,height=\textheight,keepaspectratio]{/assets/docs/JbzK099-rqq0pkW-oHCQsgAAAAAAAAAA.png}

\paragraph{\texorpdfstring{\texttt{\ arguments\ }}{ arguments }}\label{constructor-arguments}

{ any }

{Required} {{ Positional }}

\phantomsection\label{constructor-arguments-positional-tooltip}
Positional parameters are specified in order, without names.

{{ Variadic }}

\phantomsection\label{constructor-arguments-variadic-tooltip}
Variadic parameters can be specified multiple times.

The arguments to construct.

\subsection{\texorpdfstring{{ Definitions
}}{ Definitions }}\label{definitions}

\phantomsection\label{definitions-tooltip}
Functions and types and can have associated definitions. These are
accessed by specifying the function or type, followed by a period, and
then the definition\textquotesingle s name.

\subsubsection{\texorpdfstring{\texttt{\ at\ }}{ at }}\label{definitions-at}

Returns the positional argument at the specified index, or the named
argument with the specified name.

If the key is an \href{/docs/reference/foundations/int/}{integer} , this
is equivalent to first calling
\href{/docs/reference/foundations/arguments/\#definitions-pos}{\texttt{\ pos\ }}
and then
\href{/docs/reference/foundations/array/\#definitions-at}{\texttt{\ array.at\ }}
. If it is a \href{/docs/reference/foundations/str/}{string} , this is
equivalent to first calling
\href{/docs/reference/foundations/arguments/\#definitions-named}{\texttt{\ named\ }}
and then
\href{/docs/reference/foundations/dictionary/\#definitions-at}{\texttt{\ dictionary.at\ }}
.

self { . } { at } (

{ \href{/docs/reference/foundations/int/}{int}
\href{/docs/reference/foundations/str/}{str} , } {
\hyperref[definitions-at-parameters-default]{default :} { any } , }

) -\textgreater{} { any }

\paragraph{\texorpdfstring{\texttt{\ key\ }}{ key }}\label{definitions-at-key}

\href{/docs/reference/foundations/int/}{int} {or}
\href{/docs/reference/foundations/str/}{str}

{Required} {{ Positional }}

\phantomsection\label{definitions-at-key-positional-tooltip}
Positional parameters are specified in order, without names.

The index or name of the argument to get.

\paragraph{\texorpdfstring{\texttt{\ default\ }}{ default }}\label{definitions-at-default}

{ any }

A default value to return if the key is invalid.

\subsubsection{\texorpdfstring{\texttt{\ pos\ }}{ pos }}\label{definitions-pos}

Returns the captured positional arguments as an array.

self { . } { pos } (

) -\textgreater{} \href{/docs/reference/foundations/array/}{array}

\subsubsection{\texorpdfstring{\texttt{\ named\ }}{ named }}\label{definitions-named}

Returns the captured named arguments as a dictionary.

self { . } { named } (

) -\textgreater{}
\href{/docs/reference/foundations/dictionary/}{dictionary}

\href{/docs/reference/foundations/}{\pandocbounded{\includesvg[keepaspectratio]{/assets/icons/16-arrow-right.svg}}}

{ Foundations } { Previous page }

\href{/docs/reference/foundations/array/}{\pandocbounded{\includesvg[keepaspectratio]{/assets/icons/16-arrow-right.svg}}}

{ Array } { Next page }


\title{typst.app/docs/reference/foundations/duration}

\begin{itemize}
\tightlist
\item
  \href{/docs}{\includesvg[width=0.16667in,height=0.16667in]{/assets/icons/16-docs-dark.svg}}
\item
  \includesvg[width=0.16667in,height=0.16667in]{/assets/icons/16-arrow-right.svg}
\item
  \href{/docs/reference/}{Reference}
\item
  \includesvg[width=0.16667in,height=0.16667in]{/assets/icons/16-arrow-right.svg}
\item
  \href{/docs/reference/foundations/}{Foundations}
\item
  \includesvg[width=0.16667in,height=0.16667in]{/assets/icons/16-arrow-right.svg}
\item
  \href{/docs/reference/foundations/duration/}{Duration}
\end{itemize}

\section{\texorpdfstring{{ duration }}{ duration }}\label{summary}

Represents a positive or negative span of time.

\subsection{\texorpdfstring{Constructor
{}}{Constructor }}\label{constructor}

\phantomsection\label{constructor-constructor-tooltip}
If a type has a constructor, you can call it like a function to create a
new value of the type.

Creates a new duration.

You can specify the
\href{/docs/reference/foundations/duration/}{duration} using weeks,
days, hours, minutes and seconds. You can also get a duration by
subtracting two \href{/docs/reference/foundations/datetime/}{datetimes}
.

{ duration } (

{ \hyperref[constructor-parameters-seconds]{seconds :}
\href{/docs/reference/foundations/int/}{int} , } {
\hyperref[constructor-parameters-minutes]{minutes :}
\href{/docs/reference/foundations/int/}{int} , } {
\hyperref[constructor-parameters-hours]{hours :}
\href{/docs/reference/foundations/int/}{int} , } {
\hyperref[constructor-parameters-days]{days :}
\href{/docs/reference/foundations/int/}{int} , } {
\hyperref[constructor-parameters-weeks]{weeks :}
\href{/docs/reference/foundations/int/}{int} , }

) -\textgreater{} \href{/docs/reference/foundations/duration/}{duration}

\begin{verbatim}
#duration(
  days: 3,
  hours: 12,
).hours()
\end{verbatim}

\includegraphics[width=5in,height=\textheight,keepaspectratio]{/assets/docs/GmG9JKsQZEqcWXCc52iIiQAAAAAAAAAA.png}

\paragraph{\texorpdfstring{\texttt{\ seconds\ }}{ seconds }}\label{constructor-seconds}

\href{/docs/reference/foundations/int/}{int}

The number of seconds.

Default: \texttt{\ }{\texttt{\ 0\ }}\texttt{\ }

\paragraph{\texorpdfstring{\texttt{\ minutes\ }}{ minutes }}\label{constructor-minutes}

\href{/docs/reference/foundations/int/}{int}

The number of minutes.

Default: \texttt{\ }{\texttt{\ 0\ }}\texttt{\ }

\paragraph{\texorpdfstring{\texttt{\ hours\ }}{ hours }}\label{constructor-hours}

\href{/docs/reference/foundations/int/}{int}

The number of hours.

Default: \texttt{\ }{\texttt{\ 0\ }}\texttt{\ }

\paragraph{\texorpdfstring{\texttt{\ days\ }}{ days }}\label{constructor-days}

\href{/docs/reference/foundations/int/}{int}

The number of days.

Default: \texttt{\ }{\texttt{\ 0\ }}\texttt{\ }

\paragraph{\texorpdfstring{\texttt{\ weeks\ }}{ weeks }}\label{constructor-weeks}

\href{/docs/reference/foundations/int/}{int}

The number of weeks.

Default: \texttt{\ }{\texttt{\ 0\ }}\texttt{\ }

\subsection{\texorpdfstring{{ Definitions
}}{ Definitions }}\label{definitions}

\phantomsection\label{definitions-tooltip}
Functions and types and can have associated definitions. These are
accessed by specifying the function or type, followed by a period, and
then the definition\textquotesingle s name.

\subsubsection{\texorpdfstring{\texttt{\ seconds\ }}{ seconds }}\label{definitions-seconds}

The duration expressed in seconds.

This function returns the total duration represented in seconds as a
floating-point number rather than the second component of the duration.

self { . } { seconds } (

) -\textgreater{} \href{/docs/reference/foundations/float/}{float}

\subsubsection{\texorpdfstring{\texttt{\ minutes\ }}{ minutes }}\label{definitions-minutes}

The duration expressed in minutes.

This function returns the total duration represented in minutes as a
floating-point number rather than the second component of the duration.

self { . } { minutes } (

) -\textgreater{} \href{/docs/reference/foundations/float/}{float}

\subsubsection{\texorpdfstring{\texttt{\ hours\ }}{ hours }}\label{definitions-hours}

The duration expressed in hours.

This function returns the total duration represented in hours as a
floating-point number rather than the second component of the duration.

self { . } { hours } (

) -\textgreater{} \href{/docs/reference/foundations/float/}{float}

\subsubsection{\texorpdfstring{\texttt{\ days\ }}{ days }}\label{definitions-days}

The duration expressed in days.

This function returns the total duration represented in days as a
floating-point number rather than the second component of the duration.

self { . } { days } (

) -\textgreater{} \href{/docs/reference/foundations/float/}{float}

\subsubsection{\texorpdfstring{\texttt{\ weeks\ }}{ weeks }}\label{definitions-weeks}

The duration expressed in weeks.

This function returns the total duration represented in weeks as a
floating-point number rather than the second component of the duration.

self { . } { weeks } (

) -\textgreater{} \href{/docs/reference/foundations/float/}{float}

\href{/docs/reference/foundations/dictionary/}{\pandocbounded{\includesvg[keepaspectratio]{/assets/icons/16-arrow-right.svg}}}

{ Dictionary } { Previous page }

\href{/docs/reference/foundations/eval/}{\pandocbounded{\includesvg[keepaspectratio]{/assets/icons/16-arrow-right.svg}}}

{ Evaluate } { Next page }


\title{typst.app/docs/reference/foundations/none}

\begin{itemize}
\tightlist
\item
  \href{/docs}{\includesvg[width=0.16667in,height=0.16667in]{/assets/icons/16-docs-dark.svg}}
\item
  \includesvg[width=0.16667in,height=0.16667in]{/assets/icons/16-arrow-right.svg}
\item
  \href{/docs/reference/}{Reference}
\item
  \includesvg[width=0.16667in,height=0.16667in]{/assets/icons/16-arrow-right.svg}
\item
  \href{/docs/reference/foundations/}{Foundations}
\item
  \includesvg[width=0.16667in,height=0.16667in]{/assets/icons/16-arrow-right.svg}
\item
  \href{/docs/reference/foundations/none/}{None}
\end{itemize}

\section{\texorpdfstring{{ none }}{ none }}\label{summary}

A value that indicates the absence of any other value.

The none type has exactly one value:
\texttt{\ }{\texttt{\ none\ }}\texttt{\ } .

When inserted into the document, it is not visible. This is also the
value that is produced by empty code blocks. It can be
\href{/docs/reference/scripting/\#blocks}{joined} with any value,
yielding the other value.

\subsection{Example}\label{example}

\begin{verbatim}
Not visible: #none
\end{verbatim}

\includegraphics[width=5in,height=\textheight,keepaspectratio]{/assets/docs/bWChCwjCUgpluIjZfBh2dgAAAAAAAAAA.png}

\href{/docs/reference/foundations/module/}{\pandocbounded{\includesvg[keepaspectratio]{/assets/icons/16-arrow-right.svg}}}

{ Module } { Previous page }

\href{/docs/reference/foundations/panic/}{\pandocbounded{\includesvg[keepaspectratio]{/assets/icons/16-arrow-right.svg}}}

{ Panic } { Next page }


\title{typst.app/docs/reference/foundations/int}

\begin{itemize}
\tightlist
\item
  \href{/docs}{\includesvg[width=0.16667in,height=0.16667in]{/assets/icons/16-docs-dark.svg}}
\item
  \includesvg[width=0.16667in,height=0.16667in]{/assets/icons/16-arrow-right.svg}
\item
  \href{/docs/reference/}{Reference}
\item
  \includesvg[width=0.16667in,height=0.16667in]{/assets/icons/16-arrow-right.svg}
\item
  \href{/docs/reference/foundations/}{Foundations}
\item
  \includesvg[width=0.16667in,height=0.16667in]{/assets/icons/16-arrow-right.svg}
\item
  \href{/docs/reference/foundations/int/}{Integer}
\end{itemize}

\section{\texorpdfstring{{ int }}{ int }}\label{summary}

A whole number.

The number can be negative, zero, or positive. As Typst uses 64 bits to
store integers, integers cannot be smaller than
\texttt{\ }{\texttt{\ -\ }}\texttt{\ }{\texttt{\ 9223372036854775808\ }}\texttt{\ }
or larger than \texttt{\ }{\texttt{\ 9223372036854775807\ }}\texttt{\ }
.

The number can also be specified as hexadecimal, octal, or binary by
starting it with a zero followed by either \texttt{\ x\ } ,
\texttt{\ o\ } , or \texttt{\ b\ } .

You can convert a value to an integer with this type\textquotesingle s
constructor.

\subsection{Example}\label{example}

\begin{verbatim}
#(1 + 2) \
#(2 - 5) \
#(3 + 4 < 8)

#0xff \
#0o10 \
#0b1001
\end{verbatim}

\includegraphics[width=5in,height=\textheight,keepaspectratio]{/assets/docs/wfpxRJDZrNeGDA3RjEgFJgAAAAAAAAAA.png}

\subsection{\texorpdfstring{Constructor
{}}{Constructor }}\label{constructor}

\phantomsection\label{constructor-constructor-tooltip}
If a type has a constructor, you can call it like a function to create a
new value of the type.

Converts a value to an integer. Raises an error if there is an attempt
to produce an integer larger than the maximum 64-bit signed integer or
smaller than the minimum 64-bit signed integer.

\begin{itemize}
\tightlist
\item
  Booleans are converted to \texttt{\ 0\ } or \texttt{\ 1\ } .
\item
  Floats and decimals are truncated to the next 64-bit integer.
\item
  Strings are parsed in base 10.
\end{itemize}

{ int } (

{ \href{/docs/reference/foundations/bool/}{bool}
\href{/docs/reference/foundations/int/}{int}
\href{/docs/reference/foundations/float/}{float}
\href{/docs/reference/foundations/str/}{str}
\href{/docs/reference/foundations/decimal/}{decimal} }

) -\textgreater{} \href{/docs/reference/foundations/int/}{int}

\begin{verbatim}
#int(false) \
#int(true) \
#int(2.7) \
#int(decimal("3.8")) \
#(int("27") + int("4"))
\end{verbatim}

\includegraphics[width=5in,height=\textheight,keepaspectratio]{/assets/docs/4vDM_wHvGAGqziHd9y2LQQAAAAAAAAAA.png}

\paragraph{\texorpdfstring{\texttt{\ value\ }}{ value }}\label{constructor-value}

\href{/docs/reference/foundations/bool/}{bool} {or}
\href{/docs/reference/foundations/int/}{int} {or}
\href{/docs/reference/foundations/float/}{float} {or}
\href{/docs/reference/foundations/str/}{str} {or}
\href{/docs/reference/foundations/decimal/}{decimal}

{Required} {{ Positional }}

\phantomsection\label{constructor-value-positional-tooltip}
Positional parameters are specified in order, without names.

The value that should be converted to an integer.

\subsection{\texorpdfstring{{ Definitions
}}{ Definitions }}\label{definitions}

\phantomsection\label{definitions-tooltip}
Functions and types and can have associated definitions. These are
accessed by specifying the function or type, followed by a period, and
then the definition\textquotesingle s name.

\subsubsection{\texorpdfstring{\texttt{\ signum\ }}{ signum }}\label{definitions-signum}

Calculates the sign of an integer.

\begin{itemize}
\tightlist
\item
  If the number is positive, returns
  \texttt{\ }{\texttt{\ 1\ }}\texttt{\ } .
\item
  If the number is negative, returns
  \texttt{\ }{\texttt{\ -\ }}\texttt{\ }{\texttt{\ 1\ }}\texttt{\ } .
\item
  If the number is zero, returns \texttt{\ }{\texttt{\ 0\ }}\texttt{\ }
  .
\end{itemize}

self { . } { signum } (

) -\textgreater{} \href{/docs/reference/foundations/int/}{int}

\begin{verbatim}
#(5).signum() \
#(-5).signum() \
#(0).signum()
\end{verbatim}

\includegraphics[width=5in,height=\textheight,keepaspectratio]{/assets/docs/Vicm2VF6Z98sgjNZQYlaBgAAAAAAAAAA.png}

\subsubsection{\texorpdfstring{\texttt{\ bit-not\ }}{ bit-not }}\label{definitions-bit-not}

Calculates the bitwise NOT of an integer.

For the purposes of this function, the operand is treated as a signed
integer of 64 bits.

self { . } { bit-not } (

) -\textgreater{} \href{/docs/reference/foundations/int/}{int}

\begin{verbatim}
#4.bit-not() \
#(-1).bit-not()
\end{verbatim}

\includegraphics[width=5in,height=\textheight,keepaspectratio]{/assets/docs/3AYO-p6E-z3VLH4vNyWEKgAAAAAAAAAA.png}

\subsubsection{\texorpdfstring{\texttt{\ bit-and\ }}{ bit-and }}\label{definitions-bit-and}

Calculates the bitwise AND between two integers.

For the purposes of this function, the operands are treated as signed
integers of 64 bits.

self { . } { bit-and } (

{ \href{/docs/reference/foundations/int/}{int} }

) -\textgreater{} \href{/docs/reference/foundations/int/}{int}

\begin{verbatim}
#128.bit-and(192)
\end{verbatim}

\includegraphics[width=5in,height=\textheight,keepaspectratio]{/assets/docs/knwTrW-Xbj5sqbcdza7ewgAAAAAAAAAA.png}

\paragraph{\texorpdfstring{\texttt{\ rhs\ }}{ rhs }}\label{definitions-bit-and-rhs}

\href{/docs/reference/foundations/int/}{int}

{Required} {{ Positional }}

\phantomsection\label{definitions-bit-and-rhs-positional-tooltip}
Positional parameters are specified in order, without names.

The right-hand operand of the bitwise AND.

\subsubsection{\texorpdfstring{\texttt{\ bit-or\ }}{ bit-or }}\label{definitions-bit-or}

Calculates the bitwise OR between two integers.

For the purposes of this function, the operands are treated as signed
integers of 64 bits.

self { . } { bit-or } (

{ \href{/docs/reference/foundations/int/}{int} }

) -\textgreater{} \href{/docs/reference/foundations/int/}{int}

\begin{verbatim}
#64.bit-or(32)
\end{verbatim}

\includegraphics[width=5in,height=\textheight,keepaspectratio]{/assets/docs/zaVKMztfj-8VIfbLXeJFUAAAAAAAAAAA.png}

\paragraph{\texorpdfstring{\texttt{\ rhs\ }}{ rhs }}\label{definitions-bit-or-rhs}

\href{/docs/reference/foundations/int/}{int}

{Required} {{ Positional }}

\phantomsection\label{definitions-bit-or-rhs-positional-tooltip}
Positional parameters are specified in order, without names.

The right-hand operand of the bitwise OR.

\subsubsection{\texorpdfstring{\texttt{\ bit-xor\ }}{ bit-xor }}\label{definitions-bit-xor}

Calculates the bitwise XOR between two integers.

For the purposes of this function, the operands are treated as signed
integers of 64 bits.

self { . } { bit-xor } (

{ \href{/docs/reference/foundations/int/}{int} }

) -\textgreater{} \href{/docs/reference/foundations/int/}{int}

\begin{verbatim}
#64.bit-xor(96)
\end{verbatim}

\includegraphics[width=5in,height=\textheight,keepaspectratio]{/assets/docs/KUPqsOL5IXWcGSfAhFpL6wAAAAAAAAAA.png}

\paragraph{\texorpdfstring{\texttt{\ rhs\ }}{ rhs }}\label{definitions-bit-xor-rhs}

\href{/docs/reference/foundations/int/}{int}

{Required} {{ Positional }}

\phantomsection\label{definitions-bit-xor-rhs-positional-tooltip}
Positional parameters are specified in order, without names.

The right-hand operand of the bitwise XOR.

\subsubsection{\texorpdfstring{\texttt{\ bit-lshift\ }}{ bit-lshift }}\label{definitions-bit-lshift}

Shifts the operand\textquotesingle s bits to the left by the specified
amount.

For the purposes of this function, the operand is treated as a signed
integer of 64 bits. An error will occur if the result is too large to
fit in a 64-bit integer.

self { . } { bit-lshift } (

{ \href{/docs/reference/foundations/int/}{int} }

) -\textgreater{} \href{/docs/reference/foundations/int/}{int}

\begin{verbatim}
#33.bit-lshift(2) \
#(-1).bit-lshift(3)
\end{verbatim}

\includegraphics[width=5in,height=\textheight,keepaspectratio]{/assets/docs/kIVISyJsbGpK3k_fu59O2AAAAAAAAAAA.png}

\paragraph{\texorpdfstring{\texttt{\ shift\ }}{ shift }}\label{definitions-bit-lshift-shift}

\href{/docs/reference/foundations/int/}{int}

{Required} {{ Positional }}

\phantomsection\label{definitions-bit-lshift-shift-positional-tooltip}
Positional parameters are specified in order, without names.

The amount of bits to shift. Must not be negative.

\subsubsection{\texorpdfstring{\texttt{\ bit-rshift\ }}{ bit-rshift }}\label{definitions-bit-rshift}

Shifts the operand\textquotesingle s bits to the right by the specified
amount. Performs an arithmetic shift by default (extends the sign bit to
the left, such that negative numbers stay negative), but that can be
changed by the \texttt{\ logical\ } parameter.

For the purposes of this function, the operand is treated as a signed
integer of 64 bits.

self { . } { bit-rshift } (

{ \href{/docs/reference/foundations/int/}{int} , } {
\hyperref[definitions-bit-rshift-parameters-logical]{logical :}
\href{/docs/reference/foundations/bool/}{bool} , }

) -\textgreater{} \href{/docs/reference/foundations/int/}{int}

\begin{verbatim}
#64.bit-rshift(2) \
#(-8).bit-rshift(2) \
#(-8).bit-rshift(2, logical: true)
\end{verbatim}

\includegraphics[width=5in,height=\textheight,keepaspectratio]{/assets/docs/gebaB-CZOzDtnvfjDfjtTgAAAAAAAAAA.png}

\paragraph{\texorpdfstring{\texttt{\ shift\ }}{ shift }}\label{definitions-bit-rshift-shift}

\href{/docs/reference/foundations/int/}{int}

{Required} {{ Positional }}

\phantomsection\label{definitions-bit-rshift-shift-positional-tooltip}
Positional parameters are specified in order, without names.

The amount of bits to shift. Must not be negative.

Shifts larger than 63 are allowed and will cause the return value to
saturate. For non-negative numbers, the return value saturates at
\texttt{\ }{\texttt{\ 0\ }}\texttt{\ } , while, for negative numbers, it
saturates at
\texttt{\ }{\texttt{\ -\ }}\texttt{\ }{\texttt{\ 1\ }}\texttt{\ } if
\texttt{\ logical\ } is set to
\texttt{\ }{\texttt{\ false\ }}\texttt{\ } , or
\texttt{\ }{\texttt{\ 0\ }}\texttt{\ } if it is
\texttt{\ }{\texttt{\ true\ }}\texttt{\ } . This behavior is consistent
with just applying this operation multiple times. Therefore, the shift
will always succeed.

\paragraph{\texorpdfstring{\texttt{\ logical\ }}{ logical }}\label{definitions-bit-rshift-logical}

\href{/docs/reference/foundations/bool/}{bool}

Toggles whether a logical (unsigned) right shift should be performed
instead of arithmetic right shift. If this is
\texttt{\ }{\texttt{\ true\ }}\texttt{\ } , negative operands will not
preserve their sign bit, and bits which appear to the left after the
shift will be \texttt{\ }{\texttt{\ 0\ }}\texttt{\ } . This parameter
has no effect on non-negative operands.

Default: \texttt{\ }{\texttt{\ false\ }}\texttt{\ }

\subsubsection{\texorpdfstring{\texttt{\ from-bytes\ }}{ from-bytes }}\label{definitions-from-bytes}

Converts bytes to an integer.

int { . } { from-bytes } (

{ \href{/docs/reference/foundations/bytes/}{bytes} , } {
\hyperref[definitions-from-bytes-parameters-endian]{endian :}
\href{/docs/reference/foundations/str/}{str} , } {
\hyperref[definitions-from-bytes-parameters-signed]{signed :}
\href{/docs/reference/foundations/bool/}{bool} , }

) -\textgreater{} \href{/docs/reference/foundations/int/}{int}

\begin{verbatim}
#int.from-bytes(bytes((0, 0, 0, 0, 0, 0, 0, 1))) \
#int.from-bytes(bytes((1, 0, 0, 0, 0, 0, 0, 0)), endian: "big")
\end{verbatim}

\includegraphics[width=5in,height=\textheight,keepaspectratio]{/assets/docs/I0LPQ0WUii0fthcD20cosAAAAAAAAAAA.png}

\paragraph{\texorpdfstring{\texttt{\ bytes\ }}{ bytes }}\label{definitions-from-bytes-bytes}

\href{/docs/reference/foundations/bytes/}{bytes}

{Required} {{ Positional }}

\phantomsection\label{definitions-from-bytes-bytes-positional-tooltip}
Positional parameters are specified in order, without names.

The bytes that should be converted to an integer.

Must be of length at most 8 so that the result fits into a 64-bit signed
integer.

\paragraph{\texorpdfstring{\texttt{\ endian\ }}{ endian }}\label{definitions-from-bytes-endian}

\href{/docs/reference/foundations/str/}{str}

The endianness of the conversion.

\begin{longtable}[]{@{}ll@{}}
\toprule\noalign{}
Variant & Details \\
\midrule\noalign{}
\endhead
\bottomrule\noalign{}
\endlastfoot
\texttt{\ "\ big\ "\ } & Big-endian byte order: The highest-value byte
is at the beginning of the bytes. \\
\texttt{\ "\ little\ "\ } & Little-endian byte order: The lowest-value
byte is at the beginning of the bytes. \\
\end{longtable}

Default: \texttt{\ }{\texttt{\ "little"\ }}\texttt{\ }

\paragraph{\texorpdfstring{\texttt{\ signed\ }}{ signed }}\label{definitions-from-bytes-signed}

\href{/docs/reference/foundations/bool/}{bool}

Whether the bytes should be treated as a signed integer. If this is
\texttt{\ }{\texttt{\ true\ }}\texttt{\ } and the most significant bit
is set, the resulting number will negative.

Default: \texttt{\ }{\texttt{\ true\ }}\texttt{\ }

\subsubsection{\texorpdfstring{\texttt{\ to-bytes\ }}{ to-bytes }}\label{definitions-to-bytes}

Converts an integer to bytes.

self { . } { to-bytes } (

{ \hyperref[definitions-to-bytes-parameters-endian]{endian :}
\href{/docs/reference/foundations/str/}{str} , } {
\hyperref[definitions-to-bytes-parameters-size]{size :}
\href{/docs/reference/foundations/int/}{int} , }

) -\textgreater{} \href{/docs/reference/foundations/bytes/}{bytes}

\begin{verbatim}
#array(10000.to-bytes(endian: "big")) \
#array(10000.to-bytes(size: 4))
\end{verbatim}

\includegraphics[width=5in,height=\textheight,keepaspectratio]{/assets/docs/FF7gGW4eVOEhYjIZXy8BIgAAAAAAAAAA.png}

\paragraph{\texorpdfstring{\texttt{\ endian\ }}{ endian }}\label{definitions-to-bytes-endian}

\href{/docs/reference/foundations/str/}{str}

The endianness of the conversion.

\begin{longtable}[]{@{}ll@{}}
\toprule\noalign{}
Variant & Details \\
\midrule\noalign{}
\endhead
\bottomrule\noalign{}
\endlastfoot
\texttt{\ "\ big\ "\ } & Big-endian byte order: The highest-value byte
is at the beginning of the bytes. \\
\texttt{\ "\ little\ "\ } & Little-endian byte order: The lowest-value
byte is at the beginning of the bytes. \\
\end{longtable}

Default: \texttt{\ }{\texttt{\ "little"\ }}\texttt{\ }

\paragraph{\texorpdfstring{\texttt{\ size\ }}{ size }}\label{definitions-to-bytes-size}

\href{/docs/reference/foundations/int/}{int}

The size in bytes of the resulting bytes (must be at least zero). If the
integer is too large to fit in the specified size, the conversion will
truncate the remaining bytes based on the endianness. To keep the same
resulting value, if the endianness is big-endian, the truncation will
happen at the rightmost bytes. Otherwise, if the endianness is
little-endian, the truncation will happen at the leftmost bytes.

Be aware that if the integer is negative and the size is not enough to
make the number fit, when passing the resulting bytes to
\texttt{\ int.from-bytes\ } , the resulting number might be positive, as
the most significant bit might not be set to 1.

Default: \texttt{\ }{\texttt{\ 8\ }}\texttt{\ }

\href{/docs/reference/foundations/function/}{\pandocbounded{\includesvg[keepaspectratio]{/assets/icons/16-arrow-right.svg}}}

{ Function } { Previous page }

\href{/docs/reference/foundations/label/}{\pandocbounded{\includesvg[keepaspectratio]{/assets/icons/16-arrow-right.svg}}}

{ Label } { Next page }


\title{typst.app/docs/reference/foundations/bytes}

\begin{itemize}
\tightlist
\item
  \href{/docs}{\includesvg[width=0.16667in,height=0.16667in]{/assets/icons/16-docs-dark.svg}}
\item
  \includesvg[width=0.16667in,height=0.16667in]{/assets/icons/16-arrow-right.svg}
\item
  \href{/docs/reference/}{Reference}
\item
  \includesvg[width=0.16667in,height=0.16667in]{/assets/icons/16-arrow-right.svg}
\item
  \href{/docs/reference/foundations/}{Foundations}
\item
  \includesvg[width=0.16667in,height=0.16667in]{/assets/icons/16-arrow-right.svg}
\item
  \href{/docs/reference/foundations/bytes/}{Bytes}
\end{itemize}

\section{\texorpdfstring{{ bytes }}{ bytes }}\label{summary}

A sequence of bytes.

This is conceptually similar to an array of
\href{/docs/reference/foundations/int/}{integers} between
\texttt{\ }{\texttt{\ 0\ }}\texttt{\ } and
\texttt{\ }{\texttt{\ 255\ }}\texttt{\ } , but represented much more
efficiently. You can iterate over it using a
\href{/docs/reference/scripting/\#loops}{for loop} .

You can convert

\begin{itemize}
\tightlist
\item
  a \href{/docs/reference/foundations/str/}{string} or an
  \href{/docs/reference/foundations/array/}{array} of integers to bytes
  with the \href{/docs/reference/foundations/bytes/}{\texttt{\ bytes\ }}
  constructor
\item
  bytes to a string with the
  \href{/docs/reference/foundations/str/}{\texttt{\ str\ }} constructor,
  with UTF-8 encoding
\item
  bytes to an array of integers with the
  \href{/docs/reference/foundations/array/}{\texttt{\ array\ }}
  constructor
\end{itemize}

When \href{/docs/reference/data-loading/read/}{reading} data from a
file, you can decide whether to load it as a string or as raw bytes.

\begin{verbatim}
#bytes((123, 160, 22, 0)) \
#bytes("Hello 😃")

#let data = read(
  "rhino.png",
  encoding: none,
)

// Magic bytes.
#array(data.slice(0, 4)) \
#str(data.slice(1, 4))
\end{verbatim}

\includegraphics[width=5in,height=\textheight,keepaspectratio]{/assets/docs/sJtYFgVyQkDZELEHje5ywwAAAAAAAAAA.png}

\subsection{\texorpdfstring{Constructor
{}}{Constructor }}\label{constructor}

\phantomsection\label{constructor-constructor-tooltip}
If a type has a constructor, you can call it like a function to create a
new value of the type.

Converts a value to bytes.

\begin{itemize}
\tightlist
\item
  Strings are encoded in UTF-8.
\item
  Arrays of integers between \texttt{\ }{\texttt{\ 0\ }}\texttt{\ } and
  \texttt{\ }{\texttt{\ 255\ }}\texttt{\ } are converted directly. The
  dedicated byte representation is much more efficient than the array
  representation and thus typically used for large byte buffers (e.g.
  image data).
\end{itemize}

{ bytes } (

{ \href{/docs/reference/foundations/str/}{str}
\href{/docs/reference/foundations/bytes/}{bytes}
\href{/docs/reference/foundations/array/}{array} }

) -\textgreater{} \href{/docs/reference/foundations/bytes/}{bytes}

\begin{verbatim}
#bytes("Hello 😃") \
#bytes((123, 160, 22, 0))
\end{verbatim}

\includegraphics[width=5in,height=\textheight,keepaspectratio]{/assets/docs/PlfVajGmfDLMY6p8X4S3BwAAAAAAAAAA.png}

\paragraph{\texorpdfstring{\texttt{\ value\ }}{ value }}\label{constructor-value}

\href{/docs/reference/foundations/str/}{str} {or}
\href{/docs/reference/foundations/bytes/}{bytes} {or}
\href{/docs/reference/foundations/array/}{array}

{Required} {{ Positional }}

\phantomsection\label{constructor-value-positional-tooltip}
Positional parameters are specified in order, without names.

The value that should be converted to bytes.

\subsection{\texorpdfstring{{ Definitions
}}{ Definitions }}\label{definitions}

\phantomsection\label{definitions-tooltip}
Functions and types and can have associated definitions. These are
accessed by specifying the function or type, followed by a period, and
then the definition\textquotesingle s name.

\subsubsection{\texorpdfstring{\texttt{\ len\ }}{ len }}\label{definitions-len}

The length in bytes.

self { . } { len } (

) -\textgreater{} \href{/docs/reference/foundations/int/}{int}

\subsubsection{\texorpdfstring{\texttt{\ at\ }}{ at }}\label{definitions-at}

Returns the byte at the specified index. Returns the default value if
the index is out of bounds or fails with an error if no default value
was specified.

self { . } { at } (

{ \href{/docs/reference/foundations/int/}{int} , } {
\hyperref[definitions-at-parameters-default]{default :} { any } , }

) -\textgreater{} { any }

\paragraph{\texorpdfstring{\texttt{\ index\ }}{ index }}\label{definitions-at-index}

\href{/docs/reference/foundations/int/}{int}

{Required} {{ Positional }}

\phantomsection\label{definitions-at-index-positional-tooltip}
Positional parameters are specified in order, without names.

The index at which to retrieve the byte.

\paragraph{\texorpdfstring{\texttt{\ default\ }}{ default }}\label{definitions-at-default}

{ any }

A default value to return if the index is out of bounds.

\subsubsection{\texorpdfstring{\texttt{\ slice\ }}{ slice }}\label{definitions-slice}

Extracts a subslice of the bytes. Fails with an error if the start or
end index is out of bounds.

self { . } { slice } (

{ \href{/docs/reference/foundations/int/}{int} , } {
\href{/docs/reference/foundations/none/}{none}
\href{/docs/reference/foundations/int/}{int} , } {
\hyperref[definitions-slice-parameters-count]{count :}
\href{/docs/reference/foundations/int/}{int} , }

) -\textgreater{} \href{/docs/reference/foundations/bytes/}{bytes}

\paragraph{\texorpdfstring{\texttt{\ start\ }}{ start }}\label{definitions-slice-start}

\href{/docs/reference/foundations/int/}{int}

{Required} {{ Positional }}

\phantomsection\label{definitions-slice-start-positional-tooltip}
Positional parameters are specified in order, without names.

The start index (inclusive).

\paragraph{\texorpdfstring{\texttt{\ end\ }}{ end }}\label{definitions-slice-end}

\href{/docs/reference/foundations/none/}{none} {or}
\href{/docs/reference/foundations/int/}{int}

{{ Positional }}

\phantomsection\label{definitions-slice-end-positional-tooltip}
Positional parameters are specified in order, without names.

The end index (exclusive). If omitted, the whole slice until the end is
extracted.

Default: \texttt{\ }{\texttt{\ none\ }}\texttt{\ }

\paragraph{\texorpdfstring{\texttt{\ count\ }}{ count }}\label{definitions-slice-count}

\href{/docs/reference/foundations/int/}{int}

The number of items to extract. This is equivalent to passing
\texttt{\ start\ +\ count\ } as the \texttt{\ end\ } position. Mutually
exclusive with \texttt{\ end\ } .

\href{/docs/reference/foundations/bool/}{\pandocbounded{\includesvg[keepaspectratio]{/assets/icons/16-arrow-right.svg}}}

{ Boolean } { Previous page }

\href{/docs/reference/foundations/calc/}{\pandocbounded{\includesvg[keepaspectratio]{/assets/icons/16-arrow-right.svg}}}

{ Calculation } { Next page }


\title{typst.app/docs/reference/foundations/assert}

\begin{itemize}
\tightlist
\item
  \href{/docs}{\includesvg[width=0.16667in,height=0.16667in]{/assets/icons/16-docs-dark.svg}}
\item
  \includesvg[width=0.16667in,height=0.16667in]{/assets/icons/16-arrow-right.svg}
\item
  \href{/docs/reference/}{Reference}
\item
  \includesvg[width=0.16667in,height=0.16667in]{/assets/icons/16-arrow-right.svg}
\item
  \href{/docs/reference/foundations/}{Foundations}
\item
  \includesvg[width=0.16667in,height=0.16667in]{/assets/icons/16-arrow-right.svg}
\item
  \href{/docs/reference/foundations/assert/}{Assert}
\end{itemize}

\section{\texorpdfstring{\texttt{\ assert\ }}{ assert }}\label{summary}

Ensures that a condition is fulfilled.

Fails with an error if the condition is not fulfilled. Does not produce
any output in the document.

If you wish to test equality between two values, see
\href{/docs/reference/foundations/assert/\#definitions-eq}{\texttt{\ assert.eq\ }}
and
\href{/docs/reference/foundations/assert/\#definitions-ne}{\texttt{\ assert.ne\ }}
.

\subsection{Example}\label{example}

\begin{verbatim}
#assert(1 < 2, message: "math broke")
\end{verbatim}

\subsection{\texorpdfstring{{ Parameters
}}{ Parameters }}\label{parameters}

\phantomsection\label{parameters-tooltip}
Parameters are the inputs to a function. They are specified in
parentheses after the function name.

{ assert } (

{ \href{/docs/reference/foundations/bool/}{bool} , } {
\hyperref[parameters-message]{message :}
\href{/docs/reference/foundations/str/}{str} , }

)

\subsubsection{\texorpdfstring{\texttt{\ condition\ }}{ condition }}\label{parameters-condition}

\href{/docs/reference/foundations/bool/}{bool}

{Required} {{ Positional }}

\phantomsection\label{parameters-condition-positional-tooltip}
Positional parameters are specified in order, without names.

The condition that must be true for the assertion to pass.

\subsubsection{\texorpdfstring{\texttt{\ message\ }}{ message }}\label{parameters-message}

\href{/docs/reference/foundations/str/}{str}

The error message when the assertion fails.

\subsection{\texorpdfstring{{ Definitions
}}{ Definitions }}\label{definitions}

\phantomsection\label{definitions-tooltip}
Functions and types and can have associated definitions. These are
accessed by specifying the function or type, followed by a period, and
then the definition\textquotesingle s name.

\subsubsection{\texorpdfstring{\texttt{\ eq\ }}{ eq }}\label{definitions-eq}

Ensures that two values are equal.

Fails with an error if the first value is not equal to the second. Does
not produce any output in the document.

assert { . } { eq } (

{ { any } , } { { any } , } {
\hyperref[definitions-eq-parameters-message]{message :}
\href{/docs/reference/foundations/str/}{str} , }

)

\includesvg[width=0.16667in,height=0.16667in]{/assets/icons/16-arrow-right.svg}
View example

\begin{verbatim}
#assert.eq(10, 10)
\end{verbatim}

\paragraph{\texorpdfstring{\texttt{\ left\ }}{ left }}\label{definitions-eq-left}

{ any }

{Required} {{ Positional }}

\phantomsection\label{definitions-eq-left-positional-tooltip}
Positional parameters are specified in order, without names.

The first value to compare.

\paragraph{\texorpdfstring{\texttt{\ right\ }}{ right }}\label{definitions-eq-right}

{ any }

{Required} {{ Positional }}

\phantomsection\label{definitions-eq-right-positional-tooltip}
Positional parameters are specified in order, without names.

The second value to compare.

\paragraph{\texorpdfstring{\texttt{\ message\ }}{ message }}\label{definitions-eq-message}

\href{/docs/reference/foundations/str/}{str}

An optional message to display on error instead of the representations
of the compared values.

\subsubsection{\texorpdfstring{\texttt{\ ne\ }}{ ne }}\label{definitions-ne}

Ensures that two values are not equal.

Fails with an error if the first value is equal to the second. Does not
produce any output in the document.

assert { . } { ne } (

{ { any } , } { { any } , } {
\hyperref[definitions-ne-parameters-message]{message :}
\href{/docs/reference/foundations/str/}{str} , }

)

\includesvg[width=0.16667in,height=0.16667in]{/assets/icons/16-arrow-right.svg}
View example

\begin{verbatim}
#assert.ne(3, 4)
\end{verbatim}

\paragraph{\texorpdfstring{\texttt{\ left\ }}{ left }}\label{definitions-ne-left}

{ any }

{Required} {{ Positional }}

\phantomsection\label{definitions-ne-left-positional-tooltip}
Positional parameters are specified in order, without names.

The first value to compare.

\paragraph{\texorpdfstring{\texttt{\ right\ }}{ right }}\label{definitions-ne-right}

{ any }

{Required} {{ Positional }}

\phantomsection\label{definitions-ne-right-positional-tooltip}
Positional parameters are specified in order, without names.

The second value to compare.

\paragraph{\texorpdfstring{\texttt{\ message\ }}{ message }}\label{definitions-ne-message}

\href{/docs/reference/foundations/str/}{str}

An optional message to display on error instead of the representations
of the compared values.

\href{/docs/reference/foundations/array/}{\pandocbounded{\includesvg[keepaspectratio]{/assets/icons/16-arrow-right.svg}}}

{ Array } { Previous page }

\href{/docs/reference/foundations/auto/}{\pandocbounded{\includesvg[keepaspectratio]{/assets/icons/16-arrow-right.svg}}}

{ Auto } { Next page }


\title{typst.app/docs/reference/foundations/version}

\begin{itemize}
\tightlist
\item
  \href{/docs}{\includesvg[width=0.16667in,height=0.16667in]{/assets/icons/16-docs-dark.svg}}
\item
  \includesvg[width=0.16667in,height=0.16667in]{/assets/icons/16-arrow-right.svg}
\item
  \href{/docs/reference/}{Reference}
\item
  \includesvg[width=0.16667in,height=0.16667in]{/assets/icons/16-arrow-right.svg}
\item
  \href{/docs/reference/foundations/}{Foundations}
\item
  \includesvg[width=0.16667in,height=0.16667in]{/assets/icons/16-arrow-right.svg}
\item
  \href{/docs/reference/foundations/version/}{Version}
\end{itemize}

\section{\texorpdfstring{{ version }}{ version }}\label{summary}

A version with an arbitrary number of components.

The first three components have names that can be used as fields:
\texttt{\ major\ } , \texttt{\ minor\ } , \texttt{\ patch\ } . All
following components do not have names.

The list of components is semantically extended by an infinite list of
zeros. This means that, for example, \texttt{\ 0.8\ } is the same as
\texttt{\ 0.8.0\ } . As a special case, the empty version (that has no
components at all) is the same as \texttt{\ 0\ } , \texttt{\ 0.0\ } ,
\texttt{\ 0.0.0\ } , and so on.

The current version of the Typst compiler is available as
\texttt{\ sys.version\ } .

You can convert a version to an array of explicitly given components
using the \href{/docs/reference/foundations/array/}{\texttt{\ array\ }}
constructor.

\subsection{\texorpdfstring{Constructor
{}}{Constructor }}\label{constructor}

\phantomsection\label{constructor-constructor-tooltip}
If a type has a constructor, you can call it like a function to create a
new value of the type.

Creates a new version.

It can have any number of components (even zero).

{ version } (

{ \hyperref[constructor-parameters-components]{..}
\href{/docs/reference/foundations/int/}{int}
\href{/docs/reference/foundations/array/}{array} }

) -\textgreater{} \href{/docs/reference/foundations/version/}{version}

\begin{verbatim}
#version() \
#version(1) \
#version(1, 2, 3, 4) \
#version((1, 2, 3, 4)) \
#version((1, 2), 3)
\end{verbatim}

\includegraphics[width=5in,height=\textheight,keepaspectratio]{/assets/docs/Fx1_6ds8kbJ35Werk0qIqQAAAAAAAAAA.png}

\paragraph{\texorpdfstring{\texttt{\ components\ }}{ components }}\label{constructor-components}

\href{/docs/reference/foundations/int/}{int} {or}
\href{/docs/reference/foundations/array/}{array}

{Required} {{ Positional }}

\phantomsection\label{constructor-components-positional-tooltip}
Positional parameters are specified in order, without names.

{{ Variadic }}

\phantomsection\label{constructor-components-variadic-tooltip}
Variadic parameters can be specified multiple times.

The components of the version (array arguments are flattened)

\subsection{\texorpdfstring{{ Definitions
}}{ Definitions }}\label{definitions}

\phantomsection\label{definitions-tooltip}
Functions and types and can have associated definitions. These are
accessed by specifying the function or type, followed by a period, and
then the definition\textquotesingle s name.

\subsubsection{\texorpdfstring{\texttt{\ at\ }}{ at }}\label{definitions-at}

Retrieves a component of a version.

The returned integer is always non-negative. Returns \texttt{\ 0\ } if
the version isn\textquotesingle t specified to the necessary length.

self { . } { at } (

{ \href{/docs/reference/foundations/int/}{int} }

) -\textgreater{} \href{/docs/reference/foundations/int/}{int}

\paragraph{\texorpdfstring{\texttt{\ index\ }}{ index }}\label{definitions-at-index}

\href{/docs/reference/foundations/int/}{int}

{Required} {{ Positional }}

\phantomsection\label{definitions-at-index-positional-tooltip}
Positional parameters are specified in order, without names.

The index at which to retrieve the component. If negative, indexes from
the back of the explicitly given components.

\href{/docs/reference/foundations/type/}{\pandocbounded{\includesvg[keepaspectratio]{/assets/icons/16-arrow-right.svg}}}

{ Type } { Previous page }

\href{/docs/reference/model/}{\pandocbounded{\includesvg[keepaspectratio]{/assets/icons/16-arrow-right.svg}}}

{ Model } { Next page }


\title{typst.app/docs/reference/foundations/module}

\begin{itemize}
\tightlist
\item
  \href{/docs}{\includesvg[width=0.16667in,height=0.16667in]{/assets/icons/16-docs-dark.svg}}
\item
  \includesvg[width=0.16667in,height=0.16667in]{/assets/icons/16-arrow-right.svg}
\item
  \href{/docs/reference/}{Reference}
\item
  \includesvg[width=0.16667in,height=0.16667in]{/assets/icons/16-arrow-right.svg}
\item
  \href{/docs/reference/foundations/}{Foundations}
\item
  \includesvg[width=0.16667in,height=0.16667in]{/assets/icons/16-arrow-right.svg}
\item
  \href{/docs/reference/foundations/module/}{Module}
\end{itemize}

\section{\texorpdfstring{{ module }}{ module }}\label{summary}

An evaluated module, either built-in or resulting from a file.

You can access definitions from the module using
\href{/docs/reference/scripting/\#fields}{field access notation} and
interact with it using the
\href{/docs/reference/scripting/\#modules}{import and include syntaxes}
. Alternatively, it is possible to convert a module to a dictionary, and
therefore access its contents dynamically, using the
\href{/docs/reference/foundations/dictionary/\#constructor}{dictionary
constructor} .

\subsection{Example}\label{example}

\begin{verbatim}
#import "utils.typ"
#utils.add(2, 5)

#import utils: sub
#sub(1, 4)
\end{verbatim}

\includegraphics[width=5in,height=\textheight,keepaspectratio]{/assets/docs/itOPaialNOb62A81RHFv_wAAAAAAAAAA.png}

\href{/docs/reference/foundations/label/}{\pandocbounded{\includesvg[keepaspectratio]{/assets/icons/16-arrow-right.svg}}}

{ Label } { Previous page }

\href{/docs/reference/foundations/none/}{\pandocbounded{\includesvg[keepaspectratio]{/assets/icons/16-arrow-right.svg}}}

{ None } { Next page }


