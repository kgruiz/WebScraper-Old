\title{sitandr.github.io/typst-examples-book/book/basics/must_know/project_struct}

\section{\texorpdfstring{\hyperref[project-structure]{Project
structure}}{Project structure}}\label{project-structure}

\subsection{\texorpdfstring{\hyperref[large-document]{Large
document}}{Large document}}\label{large-document}

Once the document becomes large enough, it becomes harder to navigate
it. If you haven\textquotesingle t reached that size yet, you can ignore
that section.

For managing that I would recommend splitting your document into
\emph{chapters} . It is just a way to work with this, but once you
understand how it works, you can do anything you want.

Let\textquotesingle s say you have two chapters, then the recommended
structure will look like this:

\begin{verbatim}
#import "@preview/treet:0.1.1": *

#show list: tree-list
#set par(leading: 0.8em)
#show list: set text(font: "DejaVu Sans Mono", size: 0.8em)
- chapters/
  - chapter_1.typ
  - chapter_2.typ
- main.typ 👁 #text(gray)[← document entry point]
- template.typ
\end{verbatim}

\pandocbounded{\includesvg[keepaspectratio]{typst-img/291489e71b40beea77872ad05adb609349872e9a11fc3a9c3f2008c88e37c9d5-1.svg}}

The exact file names are up to you.

Let\textquotesingle s see what to put in each of these files.

\subsubsection{\texorpdfstring{\hyperref[template]{Template}}{Template}}\label{template}

In the "template" file goes \emph{all useful functions and variables}
you will use across the chapters. If you have your own template or want
to write one, you can write it there.

\begin{verbatim}
// template.typ

#let template = doc => {
    set page(header: "My super document")
    show "physics": "magic"
    doc
}

#let info-block = block.with(stroke: blue, fill: blue.lighten(70%))
#let author = "@sitandr"
\end{verbatim}

\subsubsection{\texorpdfstring{\hyperref[main]{Main}}{Main}}\label{main}

\textbf{This file should be compiled} to get the whole compiled
document.

\begin{verbatim}
// main.typ

#import "template.typ": *
// if you have a template
#show: template

= This is the document title

// some additional formatting

#show emph: set text(blue)

// but don't define functions or variables there!
// chapters will not see it

// Now the chapters themselves as some Typst content
#include("chapters/chapter_1.typ")
#include("chapters/chapter_1.typ")
\end{verbatim}

\subsubsection{\texorpdfstring{\hyperref[chapter]{Chapter}}{Chapter}}\label{chapter}

\begin{verbatim}
// chapter_1.typ

#import "../template.typ": *

That's just content with _styling_ and blocks:

#infoblock[Some information].

// just any content you want to include in the document
\end{verbatim}

\subsection{\texorpdfstring{\hyperref[notes]{Notes}}{Notes}}\label{notes}

Note that modules in Typst can see only what they created themselves or
imported. Anything else is invisible for them. That\textquotesingle s
why you need \texttt{\ }{\texttt{\ template.typ\ }}\texttt{\ } file to
define all functions within.

That means chapters \emph{don\textquotesingle t see each other either} ,
only what is in the template.

\subsection{\texorpdfstring{\hyperref[cyclic-imports]{Cyclic
imports}}{Cyclic imports}}\label{cyclic-imports}

\textbf{Important:} Typst \emph{forbids} cyclic imports. That means you
can\textquotesingle t import
\texttt{\ }{\texttt{\ chapter\_1\ }}\texttt{\ } from
\texttt{\ }{\texttt{\ chapter\_2\ }}\texttt{\ } and
\texttt{\ }{\texttt{\ chapter\_2\ }}\texttt{\ } from
\texttt{\ }{\texttt{\ chapter\_1\ }}\texttt{\ } at the same time!

But the good news is that you can always create some other file to
import variable from.
