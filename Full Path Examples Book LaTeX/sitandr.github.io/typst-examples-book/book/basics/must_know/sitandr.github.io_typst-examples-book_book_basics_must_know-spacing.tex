\title{sitandr.github.io/typst-examples-book/book/basics/must_know/spacing}

\section{\texorpdfstring{\hyperref[using-spacing]{Using
spacing}}{Using spacing}}\label{using-spacing}

Most time you will pass spacing into functions. There are special
function fields that take only \emph{size} . They are usually called
like
\texttt{\ }{\texttt{\ width,\ length,\ in(out)set,\ spacing\ }}\texttt{\ }
and so on.

Like in CSS, one of the ways to set up spacing in Typst is setting
margins and padding of elements. However, you can also insert spacing
directly using functions \texttt{\ }{\texttt{\ h\ }}\texttt{\ }
(horizontal spacing) and \texttt{\ }{\texttt{\ v\ }}\texttt{\ }
(vertical spacing).

\begin{quote}
Links to reference: \href{https://typst.app/docs/reference/layout/h/}{h}
, \href{https://typst.app/docs/reference/layout/v/}{v} .
\end{quote}

\begin{verbatim}
Horizontal #h(1cm) spacing.
#v(1cm)
And some vertical too!
\end{verbatim}

\pandocbounded{\includesvg[keepaspectratio]{typst-img/47b3ea7d16575780e489790177df9a624ad3c6c669594baa4127c1db516ebc94-1.svg}}

\section{\texorpdfstring{\hyperref[absolute-length-units]{Absolute
length units}}{Absolute length units}}\label{absolute-length-units}

\begin{quote}
Link to
\href{https://typst.app/docs/reference/layout/length/}{reference}
\end{quote}

Absolute length (aka just "length") units are not affected by outer
content and size of parent.

\begin{verbatim}
#set rect(height: 1em)
#table(
  columns: 2,
  [Points], rect(width: 72pt),
  [Millimeters], rect(width: 25.4mm),
  [Centimeters], rect(width: 2.54cm),
  [Inches], rect(width: 1in),
)
\end{verbatim}

\pandocbounded{\includesvg[keepaspectratio]{typst-img/073ad26fe313743ab62dca82f30208dbf2d57ff354d5c37f0b6d4c063dc37d76-1.svg}}

\subsection{\texorpdfstring{\hyperref[relative-to-current-font-size]{Relative
to current font
size}}{Relative to current font size}}\label{relative-to-current-font-size}

\texttt{\ }{\texttt{\ 1em\ =\ 1\ current\ font\ size\ }}\texttt{\ } :

\begin{verbatim}
#set rect(height: 1em)
#table(
  columns: 2,
  [Centimeters], rect(width: 2.54cm),
  [Relative to font size], rect(width: 6.5em)
)

Double font size: #box(stroke: red, baseline: 40%, height: 2em, width: 2em)
\end{verbatim}

\pandocbounded{\includesvg[keepaspectratio]{typst-img/7d62c9e2540f8bce40d8a3fc65a2779b161eb6b5b5682cf87247fee7f14145c2-1.svg}}

It is a very convenient unit, so it is used a lot in Typst.

\subsection{\texorpdfstring{\hyperref[combined]{Combined}}{Combined}}\label{combined}

\begin{verbatim}
Combined: #box(rect(height: 5pt + 1em))

#(5pt + 1em).abs
#(5pt + 1em).em
\end{verbatim}

\pandocbounded{\includesvg[keepaspectratio]{typst-img/c8a0cae6047f35c85c41ac44ff2a6b0d28a28d0e097ca61b367202f9a361136e-1.svg}}

\section{\texorpdfstring{\hyperref[ratio-length]{Ratio
length}}{Ratio length}}\label{ratio-length}

\begin{quote}
Link to \href{https://typst.app/docs/reference/layout/ratio/}{reference}
\end{quote}

\texttt{\ }{\texttt{\ 1\%\ =\ 1\%\ from\ parent\ size\ in\ that\ dimension\ }}\texttt{\ }

\begin{verbatim}
This line width is 50% of available page size (without margins):

#line(length: 50%)

This line width is 50% of the box width: #box(stroke: red, width: 4em, inset: (y: 0.5em), line(length: 50%))
\end{verbatim}

\pandocbounded{\includesvg[keepaspectratio]{typst-img/d478cb8be0a049380479b634cae709dc1e1ed406d323ecb1edbca1e582d7eafe-1.svg}}

\section{\texorpdfstring{\hyperref[relative-length]{Relative
length}}{Relative length}}\label{relative-length}

\begin{quote}
Link to
\href{https://typst.app/docs/reference/layout/relative/}{reference}
\end{quote}

You can \emph{combine} absolute and ratio lengths into \emph{relative
length} :

\begin{verbatim}
#rect(width: 100% - 50pt)

#(100% - 50pt).length \
#(100% - 50pt).ratio
\end{verbatim}

\pandocbounded{\includesvg[keepaspectratio]{typst-img/6b72620a1972e758e55ef1ecf49d3e843095037399ed4dd2dfcd262ebbbe803f-1.svg}}

\section{\texorpdfstring{\hyperref[fractional-length]{Fractional
length}}{Fractional length}}\label{fractional-length}

\begin{quote}
Link to
\href{https://typst.app/docs/reference/layout/fraction/}{reference}
\end{quote}

Single fraction length just takes \emph{maximum size possible} to fill
the parent:

\begin{verbatim}
Left #h(1fr) Right

#rect(height: 1em)[
  #h(1fr)
]
\end{verbatim}

\pandocbounded{\includesvg[keepaspectratio]{typst-img/b9c91f53b684699fff70c6889c8a47fccc57c5c540d7629b93c51a797eb2ef3c-1.svg}}

There are not many places you can use fractions, mainly those are
\texttt{\ }{\texttt{\ h\ }}\texttt{\ } and
\texttt{\ }{\texttt{\ v\ }}\texttt{\ } .

\subsection{\texorpdfstring{\hyperref[several-fractions]{Several
fractions}}{Several fractions}}\label{several-fractions}

If you use several fractions inside one parent, they will take all
remaining space \emph{proportional to their number} :

\begin{verbatim}
Left #h(1fr) Left-ish #h(2fr) Right
\end{verbatim}

\pandocbounded{\includesvg[keepaspectratio]{typst-img/45182cbcecf395256d133af78fccacd9d48e29073672317744cb17340d0bafd8-1.svg}}

\subsection{\texorpdfstring{\hyperref[nested-layout]{Nested
layout}}{Nested layout}}\label{nested-layout}

Remember that fractions work in parent only, don\textquotesingle t
\emph{rely on them in nested layout} :

\begin{verbatim}
Word: #h(1fr) #box(height: 1em, stroke: red)[
  #h(2fr)
]
\end{verbatim}

\pandocbounded{\includesvg[keepaspectratio]{typst-img/0c7ed8b25ea7e39a0907b1105b82027a0fb8b921b28978f30692f6c693bea5f7-1.svg}}
