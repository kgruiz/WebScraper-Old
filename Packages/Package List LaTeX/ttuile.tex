\title{typst.app/universe/package/ttuile}

\phantomsection\label{banner}
\phantomsection\label{template-thumbnail}
\pandocbounded{\includegraphics[keepaspectratio]{https://packages.typst.org/preview/thumbnails/ttuile-0.1.1-small.webp}}

\section{ttuile}\label{ttuile}

{ 0.1.1 }

A template for students\textquotesingle{} lab reports at INSA Lyon, a
french engineering school.

\href{/app?template=ttuile&version=0.1.1}{Create project in app}

\phantomsection\label{readme}
\href{https://typst.app/}{\pandocbounded{\includegraphics[keepaspectratio]{https://img.shields.io/badge/Typst-\%232f90ba.svg?&logo=Typst&logoColor=white}}}
\href{https://github.com/vitto4/ttuile/blob/main/LICENSE}{\pandocbounded{\includegraphics[keepaspectratio]{https://img.shields.io/github/license/vitto4/ttuile}}}
\href{https://github.com/vitto4/ttuile/releases}{\pandocbounded{\includegraphics[keepaspectratio]{https://img.shields.io/github/v/release/vitto4/ttuile}}}

\emph{A \textbf{Typst} template for lab reports at
\href{https://en.wikipedia.org/wiki/Institut_national_des_sciences_appliqu\%C3\%A9es_de_Lyon}{INSA
Lyon} .}

\href{https://github.com/vitto4/ttuile/blob/main/template/main.pdf}{\pandocbounded{\includegraphics[keepaspectratio]{https://raw.githubusercontent.com/vitto4/ttuile/main/assets/ttuile-banner.png?raw=true}}}

\begin{quote}
\textbf{Note :} Voir aussi le
\href{https://github.com/vitto4/ttuile/blob/main/README.FR.md}{README.FR.md}
en français.
\end{quote}

\subsection{🧭 Table of contents}\label{uxf0uxff-table-of-contents}

\begin{enumerate}
\tightlist
\item
  \href{https://github.com/typst/packages/raw/main/packages/preview/ttuile/0.1.1/\#-usage}{Usage}
\item
  \href{https://github.com/typst/packages/raw/main/packages/preview/ttuile/0.1.1/\#-documentation}{Documentation}
\item
  \href{https://github.com/typst/packages/raw/main/packages/preview/ttuile/0.1.1/\#-notes}{Notes}
\item
  \href{https://github.com/typst/packages/raw/main/packages/preview/ttuile/0.1.1/\#-contributing}{Contributing}
\end{enumerate}

\subsection{ðŸ``Ž Usage}\label{uxf0uxffux17e-usage}

This template targets french students, thus labels will be in french,
see
\href{https://github.com/typst/packages/raw/main/packages/preview/ttuile/0.1.1/\#-notes}{Notes}
.

It is available on \emph{Typst Universe} :
\href{https://typst.app/universe/package/ttuile}{\texttt{\ @preview/ttuile:0.1.1\ }}
.

If you wish to use it in a fully local manner, you’ll need to either
manually include \texttt{\ ttuile.typ\ } and
\texttt{\ logo-insa-lyon.png\ } in your project’s root directory ; or
upload them to the \emph{Typst web app} if that’s what you use.

You’ll find these files in the
\href{https://github.com/vitto4/ttuile/releases}{releases} section.

Your folder structure should then look something like this :

\begin{verbatim}
.
├── ttuile.typ
├── logo-insa-lyon.png
└── main.typ
\end{verbatim}

The template is now ready to be used, and can be called supplying the
following arguments. \texttt{\ ?\ } means the argument can be null if
not applicable.

\begin{longtable}[]{@{}cccl@{}}
\toprule\noalign{}
Argument & Default value & Type & Description \\
\midrule\noalign{}
\endhead
\bottomrule\noalign{}
\endlastfoot
\texttt{\ titre\ } & \texttt{\ none\ } & \texttt{\ content?\ } & The
title of your report. \\
\texttt{\ auteurs\ } & \texttt{\ none\ } &
\texttt{\ array\textless{}str\textgreater{}\ \textbar{}\ content?\ } &
One or multiple authors to be credited in the report. \\
\texttt{\ groupe\ } & \texttt{\ none\ } & \texttt{\ content?\ } & Your
class number/letter/identifier. Will be displayed right after the
author(s). \\
\texttt{\ numero-tp\ } & \texttt{\ none\ } & \texttt{\ content?\ } & The
number/identifier of the lab work/practical you’re writing this report
for. \\
\texttt{\ numero-poste\ } & \texttt{\ none\ } & \texttt{\ content?\ } &
Number of your lab bench. \\
\texttt{\ date\ } & \texttt{\ none\ } &
\texttt{\ datetime\ \textbar{}\ content?\ } & Date at which the lab
work/practical was carried out. \\
\texttt{\ sommaire\ } & \texttt{\ true\ } & \texttt{\ bool\ } & Display
the table of contents ? \\
\texttt{\ logo\ } & \texttt{\ image("logo-insa-lyon.png")\ } &
\texttt{\ image?\ } & University logo to use. \\
\texttt{\ point-legende\ } & \texttt{\ false\ } & \texttt{\ bool\ } &
Enable automatic enforcement of full stops at the end of figures’
captions. (still somewhat experimental). \\
\end{longtable}

A single positional argument is accepted, being the report’s body.

You can call the template using the following syntax :

\begin{Shaded}
\begin{Highlighting}[]
\NormalTok{// Local import}
\NormalTok{// \#import "ttuile.typ": *}

\NormalTok{// Universe import}
\NormalTok{\#import "@preview/ttuile:0.1.1": *}

\NormalTok{\#show: ttuile.with(}
\NormalTok{  titre: [« \#lorem(8) »],}
\NormalTok{  auteurs: (}
\NormalTok{      "Theresa Tungsten",}
\NormalTok{      "Jean Dupont",}
\NormalTok{      "Eugene Deklan",}
\NormalTok{  ),}
\NormalTok{  groupe: "TD0",}
\NormalTok{  numero{-}tp: 0,}
\NormalTok{  numero{-}poste: "0",}
\NormalTok{  date: datetime.today(),}
\NormalTok{  // sommaire: false,}
\NormalTok{  // logo: image("path\_to/logo.png"),}
\NormalTok{  // point{-}legende: true,}
\NormalTok{)}
\end{Highlighting}
\end{Shaded}

\subsection{ðŸ``š Documentation}\label{uxf0uxffux161-documentation}

The package \texttt{\ ttuile.typ\ } exposes multiple functions, find out
more about them in the \emph{documentation} .

\href{https://github.com/vitto4/ttuile/blob/main/DOC.EN.md}{To the
documentation}

An example file is also available in
\href{https://github.com/vitto4/ttuile/blob/main/template/main.typ}{\texttt{\ template/main.typ\ }}

\subsection{ðŸ''-- Notes}\label{uxf0uxff-notes}

\begin{itemize}
\item
  Beware, all of the labels will be in french (authors != auteurs,
  appendix != annexe, …)
\item
  If you really want to use this template despite not being an INSA
  student, you can probably figure out what to change in the code
  (namely labels mentioned above). You can remove the INSA logo by
  setting \texttt{\ logo:\ none\ }

  Should you still need help, no worries, feel free to reach out !
\item
  The code - variable names and comments - is all in french. That’s on
  me, I didn’t really think it through when first writing the template
  haha. I might consider translating sometime in the future.
\item
  The MIT license doesn’t apply to the file
  \texttt{\ logo-insa-lyon.png\ } , it was retrieved from
  \href{https://www.insa-lyon.fr/fr/elements-graphiques}{INSA Lyon -
  éléments graphiques} . It doesn’t apply either to the “INSA�
  branding.
\end{itemize}

\subsection{🧩 Contributing}\label{uxf0uxff-contributing}

Contributions are welcome ! Parts of the template are very much
spaghetti code, especially where the spacing between different headings
is handled (seriously, it’s pretty bad).

If you know the proper way of doing this, an issue or PR would be
greatly appreciated :)

\href{/app?template=ttuile&version=0.1.1}{Create project in app}

\subsubsection{How to use}\label{how-to-use}

Click the button above to create a new project using this template in
the Typst app.

You can also use the Typst CLI to start a new project on your computer
using this command:

\begin{verbatim}
typst init @preview/ttuile:0.1.1
\end{verbatim}

\includesvg[width=0.16667in,height=0.16667in]{/assets/icons/16-copy.svg}

\subsubsection{About}\label{about}

\begin{description}
\tightlist
\item[Author :]
\href{https://github.com/vitto4}{vitto}
\item[License:]
MIT
\item[Current version:]
0.1.1
\item[Last updated:]
May 6, 2024
\item[First released:]
May 3, 2024
\item[Archive size:]
46.8 kB
\href{https://packages.typst.org/preview/ttuile-0.1.1.tar.gz}{\pandocbounded{\includesvg[keepaspectratio]{/assets/icons/16-download.svg}}}
\item[Repository:]
\href{https://github.com/vitto4/ttuile}{GitHub}
\item[Discipline :]
\begin{itemize}
\tightlist
\item[]
\item
  \href{https://typst.app/universe/search/?discipline=engineering}{Engineering}
\end{itemize}
\item[Categor y :]
\begin{itemize}
\tightlist
\item[]
\item
  \pandocbounded{\includesvg[keepaspectratio]{/assets/icons/16-speak.svg}}
  \href{https://typst.app/universe/search/?category=report}{Report}
\end{itemize}
\end{description}

\subsubsection{Where to report issues?}\label{where-to-report-issues}

This template is a project of vitto . Report issues on
\href{https://github.com/vitto4/ttuile}{their repository} . You can also
try to ask for help with this template on the
\href{https://forum.typst.app}{Forum} .

Please report this template to the Typst team using the
\href{https://typst.app/contact}{contact form} if you believe it is a
safety hazard or infringes upon your rights.

\phantomsection\label{versions}
\subsubsection{Version history}\label{version-history}

\begin{longtable}[]{@{}ll@{}}
\toprule\noalign{}
Version & Release Date \\
\midrule\noalign{}
\endhead
\bottomrule\noalign{}
\endlastfoot
0.1.1 & May 6, 2024 \\
\href{https://typst.app/universe/package/ttuile/0.1.0/}{0.1.0} & May 3,
2024 \\
\end{longtable}

Typst GmbH did not create this template and cannot guarantee correct
functionality of this template or compatibility with any version of the
Typst compiler or app.
