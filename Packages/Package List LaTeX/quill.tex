\title{typst.app/universe/package/quill}

\phantomsection\label{banner}
\section{quill}\label{quill}

{ 0.5.0 }

Effortlessly create quantum circuit diagrams.

{ } Featured Package

\phantomsection\label{readme}
\pandocbounded{\includegraphics[keepaspectratio]{https://github.com/user-attachments/assets/bf6bfe99-6667-41b0-9b45-13c73ed18590}}

\href{https://typst.app/universe/package/quill}{\pandocbounded{\includegraphics[keepaspectratio]{https://img.shields.io/badge/dynamic/toml?url=https\%3A\%2F\%2Fraw.githubusercontent.com\%2FMc-Zen\%2Fquill\%2Fv0.5.0\%2Ftypst.toml&query=\%24.package.version&prefix=v&logo=typst&label=package&color=239DAD}}}
\href{https://github.com/Mc-Zen/quill/actions/workflows/run_tests.yml}{\pandocbounded{\includesvg[keepaspectratio]{https://github.com/Mc-Zen/quill/actions/workflows/run_tests.yml/badge.svg}}}
\href{https://github.com/Mc-Zen/quill/blob/main/LICENSE}{\pandocbounded{\includegraphics[keepaspectratio]{https://img.shields.io/badge/license-MIT-blue}}}
\href{https://github.com/Mc-Zen/quill/releases/download/v0.5.0/quill-guide.pdf}{\pandocbounded{\includegraphics[keepaspectratio]{https://img.shields.io/badge/manual-.pdf-purple}}}

\textbf{Quill} is a package for creating quantum circuit diagrams in
\href{https://typst.app/}{Typst} .

\emph{Note, that this package is in beta and may still be undergoing
breaking changes. As new features like data types and scoped functions
will be added to Typst, this package will be adapted to profit from the
new paradigms.}

\emph{Meanwhile, we suggest importing everything from the package in a
local scope to avoid polluting the global namespace (see example
below).}

\begin{itemize}
\tightlist
\item
  \href{https://github.com/typst/packages/raw/main/packages/preview/quill/0.5.0/\#basic-usage}{\textbf{Usage}}
  \emph{quick introduction}
\item
  \href{https://github.com/typst/packages/raw/main/packages/preview/quill/0.5.0/\#cheat-sheet}{\textbf{Cheat
  sheet}} \emph{gallery for quickly viewing all kinds of gates}
\item
  \href{https://github.com/typst/packages/raw/main/packages/preview/quill/0.5.0/\#tequila}{\textbf{Tequila}}
  \emph{building (sub-)circuits in a way similar to QASM or Qiskit}
\item
  \href{https://github.com/typst/packages/raw/main/packages/preview/quill/0.5.0/\#examples}{\textbf{Examples}}
\item
  \href{https://github.com/typst/packages/raw/main/packages/preview/quill/0.5.0/\#changelog}{\textbf{Changelog}}
\end{itemize}

\subsection{Basic usage}\label{basic-usage}

The function \texttt{\ quantum-circuit()\ } takes any number of
positional gates and works somewhat similarly to the built-int Typst
functions \texttt{\ table()\ } or \texttt{\ grid()\ } . A variety of
different gate and instruction commands are available for adding
elements and integers can be used to produce any number of empty cells
(filled with the current wire style). A new wire is started by adding a
\texttt{\ {[}\textbackslash{}\ {]}\ } item.

\begin{Shaded}
\begin{Highlighting}[]
\NormalTok{\#\{}
\NormalTok{  import "@preview/quill:0.5.0": *}

\NormalTok{  quantum{-}circuit(}
\NormalTok{    lstick($|0〉$), $H$, ctrl(1), rstick($(|00〉+|11〉)/√2$, n: 2), [\textbackslash{} ],}
\NormalTok{    lstick($|0〉$), 1, targ(), 1}
\NormalTok{  )}
\NormalTok{\}}
\end{Highlighting}
\end{Shaded}

\pandocbounded{\includegraphics[keepaspectratio]{https://github.com/user-attachments/assets/53d0c581-ab62-44e3-abf5-5497993d7aba}}

Plain quantum gates â€'' such as a Hadamard gate â€'' can be written
with the shorthand notation \texttt{\ \$H\$\ } instead of the more
lengthy \texttt{\ gate(\$H\$)\ } . The latter offers more options,
however.

Refer to the
\href{https://github.com/Mc-Zen/quill/releases/download/v0.5.0/quill-guide.pdf}{user
guide} for a full documentation of this package. You can also look up
the documentation of any function by calling the help module, e.g.,
\texttt{\ help("gate")\ } in order to print the signature and
description of the \texttt{\ gate\ } command, just where you are
currently typing (powered by \href{https://github.com/Mc-Zen/tidy}{tidy}
).

\subsection{Cheat Sheet}\label{cheat-sheet}

Instead of listing every featured gate (as is done in the
\href{https://github.com/Mc-Zen/quill/releases/download/v0.5.0/quill-guide.pdf}{user
guide} ), this gallery quickly showcases a large selection of possible
gates and decorations that can be added to any quantum circuit.

\pandocbounded{\includegraphics[keepaspectratio]{https://github.com/user-attachments/assets/29987e5b-c373-4cd6-86a0-58e27d778fb1}}

\subsection{Tequila}\label{tequila}

\emph{Tequila} is a submodule that adds a completely different way of
building circuits.

\begin{Shaded}
\begin{Highlighting}[]
\NormalTok{\#import "@preview/quill:0.5.0" as quill: tequila as tq}

\NormalTok{\#quill.quantum{-}circuit(}
\NormalTok{  ..tq.build(}
\NormalTok{    tq.h(0),}
\NormalTok{    tq.cx(0, 1),}
\NormalTok{    tq.cx(0, 2),}
\NormalTok{  ),}
\NormalTok{  quill.gategroup(x: 2, y: 0, 3, 2)}
\NormalTok{)}
\end{Highlighting}
\end{Shaded}

This is similar to how \emph{QASM} and \emph{Qiskit} work: gates are
successively applied to the circuit which is then layed out
automatically by packing gates as tightly as possible. We start by
calling the \texttt{\ tq.build()\ } function and filling it with quantum
operations. This returns a collection of gates which we expand into the
circuit with the \texttt{\ ..\ } syntax. Now, we still have the option
to add annotations, groups, slices, or even more gates via manual
placement.

The syntax works analog to Qiskit. Available gates are \texttt{\ x\ } ,
\texttt{\ y\ } , \texttt{\ z\ } , \texttt{\ h\ } , \texttt{\ s\ } ,
\texttt{\ sdg\ } , \texttt{\ sx\ } , \texttt{\ sxdg\ } , \texttt{\ t\ }
, \texttt{\ tdg\ } , \texttt{\ p\ } , \texttt{\ rx\ } , \texttt{\ ry\ }
, \texttt{\ rz\ } , \texttt{\ u\ } , \texttt{\ cx\ } , \texttt{\ cz\ } ,
and \texttt{\ swap\ } . With \texttt{\ barrier\ } , an invisible barrier
can be inserted to prevent gates on different qubits to be packed
tightly. Finally, with \texttt{\ tq.gate\ } and \texttt{\ tq.mqgate\ } ,
a generic gate can be created. These two accept the same styling
arguments as the normal \texttt{\ gate\ } (or \texttt{\ mqgate\ } ).

Also like Qiskit, all qubit arguments support ranges, e.g.,
\texttt{\ tq.h(range(5))\ } adds a Hadamard gate on the first five
qubits and \texttt{\ tq.cx((0,\ 1),\ (1,\ 2))\ } adds two CX gates: one
from qubit 0 to 1 and one from qubit 1 to 2.

With Tequila, it is easy to build templates for quantum circuits and to
compose circuits of various building blocks. For this purpose,
\texttt{\ tq.build()\ } and the built-in templates all feature optional
\texttt{\ x\ } and \texttt{\ y\ } arguments to allow placing a
subcircuit at an arbitrary position of the circuit. As an example,
Tequila provides a \texttt{\ tq.graph-state()\ } template for quickly
drawing graph state preparation circuits.

The following example demonstrates how to compose multiple subcircuits.

\begin{Shaded}
\begin{Highlighting}[]
\NormalTok{\#import tequila as tq}

\NormalTok{\#quantum{-}circuit(}
\NormalTok{  ..tq.graph{-}state((0, 1), (1, 2)),}
\NormalTok{  ..tq.build(y: 3, }
\NormalTok{      tq.p($pi$, 0), }
\NormalTok{      tq.cx(0, (1, 2)), }
\NormalTok{    ),}
\NormalTok{  ..tq.graph{-}state(x: 6, y: 2, invert: true, (0, 1), (0, 2)),}
\NormalTok{  gategroup(x: 1, 3, 3),}
\NormalTok{  gategroup(x: 1, y: 3, 3, 3),}
\NormalTok{  gategroup(x: 6, y: 2, 3, 3),}
\NormalTok{  slice(x: 5)}
\NormalTok{)}
\end{Highlighting}
\end{Shaded}

\pandocbounded{\includegraphics[keepaspectratio]{https://github.com/user-attachments/assets/41c8d60a-1a5e-4d0b-a7f4-82756423f5a8}}

\subsection{Examples}\label{examples}

Some show-off examples, loosely replicating figures from
\href{https://www.cambridge.org/highereducation/books/quantum-computation-and-quantum-information/01E10196D0A682A6AEFFEA52D53BE9AE\#overview}{Quantum
Computation and Quantum Information by M. Nielsen and I. Chuang} . The
code for these examples can be found in the
\href{https://github.com/Mc-Zen/quill/tree/v0.5.0/examples}{example
folder} or in the
\href{https://github.com/Mc-Zen/quill/releases/download/v0.5.0/quill-guide.pdf}{user
guide} .

\pandocbounded{\includegraphics[keepaspectratio]{https://github.com/user-attachments/assets/f38abeb9-fc2f-4be4-9592-7932e07af764}}

\pandocbounded{\includegraphics[keepaspectratio]{https://github.com/user-attachments/assets/6e947f71-67dc-4522-936e-6d9b795a1bba}}

\pandocbounded{\includegraphics[keepaspectratio]{https://github.com/user-attachments/assets/3fc92cd0-915e-4c5e-893d-63dac6f83ade}}

\subsection{Contribution}\label{contribution}

If you spot an issue or have a suggestion, you are invited to
\href{https://github.com/Mc-Zen/quill/issues}{post it} or to contribute.
In
\href{https://github.com/Mc-Zen/quill/tree/v0.5.0/docs/architecture.md}{architecture.md}
, you can also find a description of the algorithm that forms the base
of \texttt{\ quantum-circuit()\ } .

\subsection{Tests}\label{tests}

This package uses
\href{https://github.com/tingerrr/typst-test/}{typst-test} for running
\href{https://github.com/Mc-Zen/quill/tree/v0.5.0/tests/}{tests} .

\subsection{Changelog}\label{changelog}

\subsubsection{v0.5.0}\label{v0.5.0}

\begin{itemize}
\tightlist
\item
  Added support for multi-controlled gates with Tequila.
\item
  Switched to using \texttt{\ context\ } instead of the now deprecated
  \texttt{\ style()\ } for measurement. Note: Starting with this
  version, Typst 0.11.0 or higher is required.
\end{itemize}

\subsubsection{v0.4.0}\label{v0.4.0}

\begin{itemize}
\tightlist
\item
  Alternative model for creating and composing circuits:
  \href{https://github.com/typst/packages/raw/main/packages/preview/quill/0.5.0/\#tequila}{Tequila}
  .
\end{itemize}

\subsubsection{v0.3.0}\label{v0.3.0}

\begin{itemize}
\tightlist
\item
  New features

  \begin{itemize}
  \tightlist
  \item
    Enable manual placement of gates,
    \texttt{\ gate(\$X\$,\ x:\ 3,\ y:\ 1)\ } , similar to built-in
    \texttt{\ table()\ } in addition to automatic placement. This works
    for most elements, not only gates.
  \item
    Add parameter \texttt{\ pad\ } to \texttt{\ lstick()\ } and
    \texttt{\ rstick()\ } .
  \item
    Add parameter \texttt{\ fill-wires\ } to
    \texttt{\ quantum-circuit()\ } . All wires are filled unto the end
    (determined by the longest wire) by default (breaking change
    âš~ï¸?). This behavior can be reverted by setting
    \texttt{\ fill-wires:\ false\ } .
  \item
    \texttt{\ gategroup()\ } \texttt{\ slice()\ } and
    \texttt{\ annotate()\ } can now be placed above or below the circuit
    with \texttt{\ z:\ "above"\ } and \texttt{\ z:\ "below"\ } .
  \item
    \texttt{\ help()\ } command for quickly displaying the documentation
    of a given function, e.g., \texttt{\ help("gate")\ } . Powered by
    \href{https://github.com/Mc-Zen/tidy}{tidy} .
  \end{itemize}
\item
  Improvements:

  \begin{itemize}
  \tightlist
  \item
    Complete rework of circuit layout implementation

    \begin{itemize}
    \tightlist
    \item
      allows transparent gates since wires are not drawn through gates
      anymore. The default fill is now \texttt{\ auto\ } and using
      \texttt{\ none\ } sets the background to transparent.
    \item
      \texttt{\ midstick\ } is now transparent by default.
    \end{itemize}
  \item
    \texttt{\ setwire()\ } can now be used to override only partial wire
    settings, such as wire color \texttt{\ setwire(1,\ stroke:\ blue)\ }
    , width \texttt{\ setwire(1,\ stroke:\ 1pt)\ } or wire distance, all
    separately. Before, some settings were reset.
  \end{itemize}
\item
  Fixes:

  \begin{itemize}
  \tightlist
  \item
    Fixed \texttt{\ lstick\ } / \texttt{\ rstick\ } when equation
    numbering is turned on.
  \end{itemize}
\item
  Removed:

  \begin{itemize}
  \tightlist
  \item
    The already deprecated \texttt{\ scale-factor\ } (use
    \texttt{\ scale\ } instead)
  \end{itemize}
\end{itemize}

\subsubsection{v0.2.1}\label{v0.2.1}

\begin{itemize}
\tightlist
\item
  Improvements:

  \begin{itemize}
  \tightlist
  \item
    Add \texttt{\ fill\ } parameter to \texttt{\ midstick()\ } .
  \item
    Add \texttt{\ bend\ } parameter to \texttt{\ permute()\ } .
  \item
    Add \texttt{\ separation\ } parameter to \texttt{\ permute()\ } .
  \end{itemize}
\item
  Fixes:

  \begin{itemize}
  \tightlist
  \item
    With Typst 0.11.0, \texttt{\ scale()\ } now takes into account outer
    alignment. This broke the positioning of centered/right-aligned
    circuits, e.g., ones put into a \texttt{\ figure()\ } .
  \item
    Change wires to be drawn all through \texttt{\ ctrl()\ } , making it
    consistent to \texttt{\ swap()\ } and \texttt{\ targ()\ } .
  \end{itemize}
\end{itemize}

\subsubsection{v0.2.0}\label{v0.2.0}

\begin{itemize}
\tightlist
\item
  New features:

  \begin{itemize}
  \tightlist
  \item
    Add arbitrary labels to any \texttt{\ gate\ } (also derived gates
    such as \texttt{\ meter\ } , \texttt{\ ctrl\ } , …),
    \texttt{\ gategroup\ } or \texttt{\ slice\ } that can be anchored to
    any of the nine 2d alignments.
  \item
    Add optional gate inputs and outputs for multi-qubit gates (see
    gallery).
  \item
    Implicit gates (breaking change âš~ï¸?): a content item
    automatically becomes a gate, so you can just type
    \texttt{\ \$H\$\ } instead of \texttt{\ gate(\$H\$)\ } (of course,
    the \texttt{\ gate()\ } function is still important in order to use
    the many available options).
  \end{itemize}
\item
  Other breaking changes âš~ï¸?:

  \begin{itemize}
  \tightlist
  \item
    \texttt{\ slice()\ } has no \texttt{\ dx\ } and \texttt{\ dy\ }
    parameters anymore. Instead, labels are handled through
    \texttt{\ label\ } exactly as in \texttt{\ gate()\ } . Also the
    \texttt{\ wires\ } parameter is replaced with \texttt{\ n\ } for
    consistency with other multi-qubit gates.
  \item
    Swap order of row and column parameters in \texttt{\ annotate()\ }
    to make it consistent with built-in Typst functions.
  \end{itemize}
\item
  Improvements:

  \begin{itemize}
  \tightlist
  \item
    Improve layout (allow row/column spacing and min lengths to be
    specified in em-lengths).
  \item
    Automatic bounds computation, even for labels.
  \item
    Improve meter (allow multi-qubit gate meters and respect global
    (per-circuit) gate padding).d
  \end{itemize}
\item
  Fixes:

  \begin{itemize}
  \tightlist
  \item
    \texttt{\ lstick\ } / \texttt{\ rstick\ } braces broke with Typst
    0.7.0.
  \item
    \texttt{\ lstick\ } / \texttt{\ rstick\ } bounds.
  \end{itemize}
\item
  Documentation

  \begin{itemize}
  \tightlist
  \item
    Add section on creating custom gates.
  \item
    Add section on using labels.
  \item
    Explain usage of \texttt{\ slice()\ } and \texttt{\ gategroup()\ } .
  \end{itemize}
\end{itemize}

\subsubsection{v0.1.0}\label{v0.1.0}

Initial Release

\subsubsection{How to add}\label{how-to-add}

Copy this into your project and use the import as \texttt{\ quill\ }

\begin{verbatim}
#import "@preview/quill:0.5.0"
\end{verbatim}

\includesvg[width=0.16667in,height=0.16667in]{/assets/icons/16-copy.svg}

Check the docs for
\href{https://typst.app/docs/reference/scripting/\#packages}{more
information on how to import packages} .

\subsubsection{About}\label{about}

\begin{description}
\tightlist
\item[Author :]
\href{https://github.com/Mc-Zen}{Mc-Zen}
\item[License:]
MIT
\item[Current version:]
0.5.0
\item[Last updated:]
October 24, 2024
\item[First released:]
June 28, 2023
\item[Minimum Typst version:]
0.11.0
\item[Archive size:]
24.9 kB
\href{https://packages.typst.org/preview/quill-0.5.0.tar.gz}{\pandocbounded{\includesvg[keepaspectratio]{/assets/icons/16-download.svg}}}
\item[Repository:]
\href{https://github.com/Mc-Zen/quill}{GitHub}
\item[Discipline s :]
\begin{itemize}
\tightlist
\item[]
\item
  \href{https://typst.app/universe/search/?discipline=physics}{Physics}
\item
  \href{https://typst.app/universe/search/?discipline=computer-science}{Computer
  Science}
\end{itemize}
\item[Categor y :]
\begin{itemize}
\tightlist
\item[]
\item
  \pandocbounded{\includesvg[keepaspectratio]{/assets/icons/16-chart.svg}}
  \href{https://typst.app/universe/search/?category=visualization}{Visualization}
\end{itemize}
\end{description}

\subsubsection{Where to report issues?}\label{where-to-report-issues}

This package is a project of Mc-Zen . Report issues on
\href{https://github.com/Mc-Zen/quill}{their repository} . You can also
try to ask for help with this package on the
\href{https://forum.typst.app}{Forum} .

Please report this package to the Typst team using the
\href{https://typst.app/contact}{contact form} if you believe it is a
safety hazard or infringes upon your rights.

\phantomsection\label{versions}
\subsubsection{Version history}\label{version-history}

\begin{longtable}[]{@{}ll@{}}
\toprule\noalign{}
Version & Release Date \\
\midrule\noalign{}
\endhead
\bottomrule\noalign{}
\endlastfoot
0.5.0 & October 24, 2024 \\
\href{https://typst.app/universe/package/quill/0.4.0/}{0.4.0} &
September 16, 2024 \\
\href{https://typst.app/universe/package/quill/0.3.0/}{0.3.0} & May 22,
2024 \\
\href{https://typst.app/universe/package/quill/0.2.1/}{0.2.1} & March
11, 2024 \\
\href{https://typst.app/universe/package/quill/0.2.0/}{0.2.0} &
September 28, 2023 \\
\href{https://typst.app/universe/package/quill/0.1.0/}{0.1.0} & June 28,
2023 \\
\end{longtable}

Typst GmbH did not create this package and cannot guarantee correct
functionality of this package or compatibility with any version of the
Typst compiler or app.
