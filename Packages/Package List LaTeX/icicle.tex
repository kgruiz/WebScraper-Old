\title{typst.app/universe/package/icicle}

\phantomsection\label{banner}
\phantomsection\label{template-thumbnail}
\pandocbounded{\includegraphics[keepaspectratio]{https://packages.typst.org/preview/thumbnails/icicle-0.1.0-small.webp}}

\section{icicle}\label{icicle}

{ 0.1.0 }

Help the Typst Guys reach the helicopter pad and save Christmas!

\href{/app?template=icicle&version=0.1.0}{Create project in app}

\phantomsection\label{readme}
Help the Typst Guys reach the helicopter pad and save Christmas!
Navigate them with the WASD keys and solve puzzles with snowballs to
make way for the Typst Guys.

This small Christmas-themed game is playable in the Typst editor and
best enjoyed with the web app or \texttt{\ typst\ watch\ } . It was
first released for the 24 Days to Christmas campaign in winter of 2023.

\subsection{Usage}\label{usage}

You can use this template in the Typst web app by clicking “Start from
template� on the dashboard and searching for \texttt{\ icicle\ } .

Alternatively, you can use the CLI to kick this project off using the
command

\begin{verbatim}
typst init @preview/icicle
\end{verbatim}

Typst will create a new directory with all the files needed to get you
started.

\subsection{Configuration}\label{configuration}

This template exports the \texttt{\ game\ } function, which accepts a
positional argument for the game input.

The template will initialize your package with a sample call to the
\texttt{\ game\ } function in a show rule. If you want to change an
existing project to use this template, you can add a show rule like this
at the top of your file:

\begin{Shaded}
\begin{Highlighting}[]
\NormalTok{\#import "@preview/icicle:0.1.0": game}
\NormalTok{\#show: game}

\NormalTok{// Move with WASD.}
\end{Highlighting}
\end{Shaded}

You can also add your own levels by adding an array of level definition
strings in the \texttt{\ game\ } function’s named \texttt{\ levels\ }
argument. Each level file must conform to the following format:

\begin{itemize}
\tightlist
\item
  First, a line with two comma separated integers indicating the
  player’s starting position.
\item
  Then, a matrix with the characters f (floor), x (wall), w (water), or
  g (goal).
\item
  Finally, a matrix with the characters b (snowball) or \_ (nothing).
\end{itemize}

The three arguments must be separated by double newlines. Additionally,
each row in the matrices space-separates its values. Newlines terminate
the rows. Comments can be added with a double slash. Find an example for
a valid level string below:

\begin{verbatim}
// The starting position
0, 0

// The back layer
f f f w f f f
f f f w f f f
f f x w f f f
f f f w f f f
f f f w f x x
x x x g x x x

// The front layer.
_ _ b _ _ _ _
_ _ b _ _ _ _
_ _ _ _ b _ _
_ _ _ _ b _ _
_ _ _ _ _ _ _
_ _ _ _ _ _ _
\end{verbatim}

It’s best to put levels into separate files and load them with the
\texttt{\ read\ } function.

\href{/app?template=icicle&version=0.1.0}{Create project in app}

\subsubsection{How to use}\label{how-to-use}

Click the button above to create a new project using this template in
the Typst app.

You can also use the Typst CLI to start a new project on your computer
using this command:

\begin{verbatim}
typst init @preview/icicle:0.1.0
\end{verbatim}

\includesvg[width=0.16667in,height=0.16667in]{/assets/icons/16-copy.svg}

\subsubsection{About}\label{about}

\begin{description}
\tightlist
\item[Author :]
\href{https://typst.app}{Typst GmbH}
\item[License:]
MIT-0
\item[Current version:]
0.1.0
\item[Last updated:]
March 6, 2024
\item[First released:]
March 6, 2024
\item[Minimum Typst version:]
0.8.0
\item[Archive size:]
143 kB
\href{https://packages.typst.org/preview/icicle-0.1.0.tar.gz}{\pandocbounded{\includesvg[keepaspectratio]{/assets/icons/16-download.svg}}}
\item[Repository:]
\href{https://github.com/typst/templates}{GitHub}
\item[Categor y :]
\begin{itemize}
\tightlist
\item[]
\item
  \pandocbounded{\includesvg[keepaspectratio]{/assets/icons/16-smile.svg}}
  \href{https://typst.app/universe/search/?category=fun}{Fun}
\end{itemize}
\end{description}

\subsubsection{Where to report issues?}\label{where-to-report-issues}

This template is a project of Typst GmbH . Report issues on
\href{https://github.com/typst/templates}{their repository} . You can
also try to ask for help with this template on the
\href{https://forum.typst.app}{Forum} .

\phantomsection\label{versions}
\subsubsection{Version history}\label{version-history}

\begin{longtable}[]{@{}ll@{}}
\toprule\noalign{}
Version & Release Date \\
\midrule\noalign{}
\endhead
\bottomrule\noalign{}
\endlastfoot
0.1.0 & March 6, 2024 \\
\end{longtable}
