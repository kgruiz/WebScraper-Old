\title{typst.app/universe/package/supercharged-dhbw}

\phantomsection\label{banner}
\phantomsection\label{template-thumbnail}
\pandocbounded{\includegraphics[keepaspectratio]{https://packages.typst.org/preview/thumbnails/supercharged-dhbw-3.3.2-small.webp}}

\section{supercharged-dhbw}\label{supercharged-dhbw}

{ 3.3.2 }

Unofficial Template for DHBW

\href{/app?template=supercharged-dhbw&version=3.3.2}{Create project in
app}

\phantomsection\label{readme}
Unofficial \href{https://typst.app/}{Typst} template for DHBW students.

You can see an example PDF of how the template looks
\href{https://github.com/DannySeidel/typst-dhbw-template/blob/main/examples/example.pdf}{here}
.

To see an example of how you can use this template, check out the
\texttt{\ main.typ\ } file. More examples can be found in the
\href{https://github.com/DannySeidel/typst-dhbw-template/blob/main/examples}{examples
directory} of the GitHub repository.

\subsection{Usage}\label{usage}

You can use this template in the Typst web app by clicking “Start from
template� on the dashboard and searching for
\texttt{\ supercharged-dhbw\ } .

Alternatively, you can use the CLI to kick this project off using the
command

\begin{Shaded}
\begin{Highlighting}[]
\NormalTok{typst init @preview/supercharged{-}dhbw}
\end{Highlighting}
\end{Shaded}

Typst will create a new directory with all the files needed to get you
started.

\subsection{Fonts}\label{fonts}

This template uses the following fonts:

\begin{itemize}
\tightlist
\item
  \href{https://fonts.google.com/specimen/Montserrat}{Montserrat}
\item
  \href{https://fonts.google.com/specimen/Open+Sans}{Open Sans}
\end{itemize}

If you want to use typst locally, you can download the fonts from the
links above and install them on your system. Otherwise, when using the
web version add the fonts to your project.

For further information on how to add fonts to your project, please
refer to the
\href{https://typst.app/docs/reference/text/text/\#parameters-font}{Typst
documentation} .

\subsection{Used Packages}\label{used-packages}

This template uses the following packages:

\begin{itemize}
\tightlist
\item
  \href{https://typst.app/universe/package/codelst}{codelst} : To create
  code snippets
\end{itemize}

Insert code snippets using the following syntax:

\begin{Shaded}
\begin{Highlighting}[]
\NormalTok{\#figure(caption: "Codeblock Example", sourcecode[\textasciigrave{}\textasciigrave{}\textasciigrave{}ts}
\NormalTok{const ReactComponent = () =\textgreater{} \{}
\NormalTok{  return (}
\NormalTok{    \textless{}div\textgreater{}}
\NormalTok{      \textless{}h1\textgreater{}Hello World\textless{}/h1\textgreater{}}
\NormalTok{    \textless{}/div\textgreater{}}
\NormalTok{  );}
\NormalTok{\};}

\NormalTok{export default ReactComponent;}
\NormalTok{\textasciigrave{}\textasciigrave{}\textasciigrave{}])}
\end{Highlighting}
\end{Shaded}

\subsection{Configuration}\label{configuration}

This template exports the \texttt{\ supercharged-dhbw\ } function with
the following named arguments:

\texttt{\ title\ (str*)\ } : Title of the document

\texttt{\ authors\ (dictionary*)\ } : List of authors with the following
named arguments (max. 6 authors when in the company or 8 authors when at
DHBW):

\begin{itemize}
\tightlist
\item
  name (str*): Name of the author
\item
  student-id (str*): Student ID of the author
\item
  course (str*): Course of the author
\item
  course-of-studies (str*): Course of studies of the author
\item
  company (dictionary): Company of the author (only needed when
  \texttt{\ at-university\ } is \texttt{\ false\ } ) with the following
  named arguments:

  \begin{itemize}
  \tightlist
  \item
    name (str*): Name of the company
  \item
    post-code (str): Post code of the company
  \item
    city (str*): City of the company
  \item
    country (str): Country of the company
  \end{itemize}
\end{itemize}

\texttt{\ abstract\ (content)\ } : Content of the abstract, it is
recommended that you pass a variable containing the content or a
function that returns the content

\texttt{\ acronym-spacing\ (length)\ } : Spacing between the acronym and
its long form (check the
\href{https://typst.app/docs/reference/layout/length/}{Typst
documentation} for examples on how to provide parameters of type
length), default is \texttt{\ 5em\ }

\texttt{\ acronyms\ (dictionary)\ } : Pass a dictionary containing the
acronyms and their long forms (See the example in the
\texttt{\ acronyms.typ\ } file)

\texttt{\ appendix\ (content)\ } : Content of the appendix, it is
recommended that you pass a variable containing the content or a
function that returns the content

\texttt{\ at-university\ (bool*)\ } : Whether the document is written at
university or not, default is \texttt{\ false\ }

\texttt{\ bibliography\ (content)\ } : Path to the bibliography file

\texttt{\ bib-style\ (str)\ } : Style of the bibliography, default is
\texttt{\ ieee\ }

\texttt{\ city\ (str)\ } : City of the author (only needed when
\texttt{\ at-university\ } is \texttt{\ true\ } )

\texttt{\ confidentiality-marker:\ (dictionary)\ } : Configure the
confidentially marker (red or green circle) on the title page (using
this option reduces the maximum number of authors by 2 to 4 authors when
in the company or 6 authors when at DHBW)

\begin{itemize}
\tightlist
\item
  display (bool*): Whether the confidentiality marker should be shown,
  default is \texttt{\ false\ }
\item
  offset-x (length): Horizontal offset of the confidentiality marker,
  default is \texttt{\ 0pt\ }
\item
  offset-y (length): Vertical offset of the confidentiality marker,
  default is \texttt{\ 0pt\ }
\item
  size (length): Size of the confidentiality marker, default is
  \texttt{\ 7em\ }
\item
  title-spacing (length): Adds space below the title to make room for
  the confidentiality marker, default is \texttt{\ 2em\ }
\end{itemize}

\texttt{\ confidentiality-statement-content\ (content)\ } : Provide a
custom confidentiality statement

\texttt{\ date\ (datetime*\ \textbar{}\ array*)\ } : Provide a datetime
object to display one date (e.g. submission date) or a array containing
two datetime objects to display a date range (e.g. start and end date of
the project), default is \texttt{\ datetime.today()\ }

\texttt{\ date-format\ (str)\ } : Format of the displayed dates, default
is \texttt{\ "{[}day{]}.{[}month{]}.{[}year{]}"\ } (for more information
on possible formats check the
\href{https://typst.app/docs/reference/foundations/datetime/\#format}{Typst
documentation} )

\texttt{\ declaration-of-authorship-content\ (content)\ } : Provide a
custom declaration of authorship

\texttt{\ glossary\ (dictionary)\ } : Pass a dictionary containing the
glossary terms and their definitions (See the example in the
\texttt{\ glossary.typ\ } file)

\texttt{\ glossary-spacing\ (length)\ } : Spacing between the glossary
term and its definition (check the
\href{https://typst.app/docs/reference/layout/length/}{Typst
documentation} for examples on how to provide parameters of type
length), default is \texttt{\ 1.5em\ }

\texttt{\ header\ (dictionary)\ } : Configure the header of the document

\begin{itemize}
\tightlist
\item
  display (bool): Whether the header should be shown, default is
  \texttt{\ true\ }
\item
  show-chapter (bool): Whether the current chapter should be shown in
  the header, default is \texttt{\ true\ }
\item
  show-left-logo (bool): Whether the left logo should be shown in the
  header, default is \texttt{\ true\ }
\item
  show-right-logo (bool): Whether the right logo should be shown in the
  header, default is \texttt{\ true\ }
\item
  show-divider (bool): Whether the header divider should be shown,
  default is \texttt{\ true\ }
\item
  content (content): Content for a custom header, it is recommended that
  you pass a variable containing the content or a function that returns
  the content
\end{itemize}

\texttt{\ heading-numering\ (str)\ } : Numbering style of the headings,
default is \texttt{\ "1.1"\ } (for more information on possible
numbering formats check the
\href{https://typst.app/docs/reference/model/numbering}{Typst
documentation} )

\texttt{\ ignored-link-label-keys-for-highlighting\ (array)\ } : List of
keys of labels that should be ignored when highlighting links in the
document, default is \texttt{\ ()\ }

\texttt{\ language\ (str*)\ } : Language of the document which is either
\texttt{\ en\ } or \texttt{\ de\ } , default is \texttt{\ en\ }

\texttt{\ logo-left\ (content)\ } : Path to the logo on the left side of
the title page (usage: image(“path/to/image.png�)), default is the
\texttt{\ DHBW\ logo\ }

\texttt{\ logo-right\ (content)\ } : Path to the logo on the right side
of the title page (usage: image(“path/to/image.png�)), default is
\texttt{\ no\ logo\ }

\texttt{\ logo-size-ratio\ (str)\ } : Ratio between the right logo and
the left logo height (left-logo:right-logo), default is
\texttt{\ "1:1"\ }

\texttt{\ math-numbering\ (str)\ } : Numbering style of the math
equations, set to \texttt{\ none\ } to turn off equation numbering,
default is \texttt{\ "(1)"\ } (for more information on possible
numbering formats check the
\href{https://typst.app/docs/reference/model/numbering}{Typst
documentation} )

\texttt{\ numbering-alignment\ (alignment)\ } : Alignment of the page
numbering (for possible options check the
\href{https://typst.app/docs/reference/layout/alignment/}{Typst
documentation} ), default is \texttt{\ center\ }

\texttt{\ show-abstract\ (bool)\ } : Whether the abstract should be
shown, default is \texttt{\ true\ }

\texttt{\ show-acronyms\ (bool)\ } : Whether the list of acronyms should
be shown, default is \texttt{\ true\ }

\texttt{\ show-code-snippets\ (bool)\ } : Whether the code snippets
should be shown, default is \texttt{\ true\ }

\texttt{\ show-confidentiality-statement\ (bool)\ } : Whether the
confidentiality statement should be shown, default is \texttt{\ true\ }

\texttt{\ show-declaration-of-authorship\ (bool)\ } : Whether the
declaration of authorship should be shown, default is \texttt{\ true\ }

\texttt{\ show-list-of-figures\ (bool)\ } : Whether the list of figures
should be shown, default is \texttt{\ true\ }

\texttt{\ show-list-of-tables\ (bool)\ } : Whether the list of tables
should be shown, default is \texttt{\ true\ }

\texttt{\ show-table-of-contents\ (bool)\ } : Whether the table of
contents should be shown, default is \texttt{\ true\ }

\texttt{\ supervisor\ (dictionary*)\ } : Name of the supervisor at the
university and/or company (e.g. supervisor: (company: “John Doe�,
university: “Jane Doe�))

\begin{itemize}
\tightlist
\item
  company (str): Name of the supervisor at the company (note while the
  argument is optional at least one of the two arguments must be
  provided)
\item
  university (str): Name of the supervisor at the university (note while
  the argument is optional at least one of the two arguments must be
  provided)
\end{itemize}

\texttt{\ titlepage-content\ (content)\ } : Provide a custom title page

\texttt{\ toc-depth\ (int)\ } : Depth of the table of contents, default
is \texttt{\ 3\ }

\texttt{\ type-of-thesis\ (str)\ } : Type of the thesis, default is
\texttt{\ none\ } (using this option reduces the maximum number of
authors by 2 to 4 authors when in the company or 6 authors when at DHBW)

\texttt{\ type-of-degree\ (str)\ } : Type of the degree, default is
\texttt{\ none\ } (using this option reduces the maximum number of
authors by 2 to 4 authors when in the company or 6 authors when at DHBW)

\texttt{\ university\ (str*)\ } : Name of the university

\texttt{\ university-location\ (str*)\ } : Campus or city of the
university

\texttt{\ university-short\ (str*)\ } : Short name of the university
(e.g. DHBW), displayed for the university supervisor

Behind the arguments the type of the value is given in parentheses. All
arguments marked with \texttt{\ *\ } are required.

\subsection{Acronyms}\label{acronyms}

This template provides functions to reference acronyms in the text. To
use these functions, you need to define the acronyms in the
\texttt{\ acronyms\ } attribute of the template. The acronyms referenced
with the functions below will be linked to their definition in the list
of acronyms.

\subsubsection{Functions}\label{functions}

This template provides the following functions to reference acronyms:

\texttt{\ acr\ } : Reference an acronym in the text (e.g.
\texttt{\ acr("API")\ } -\textgreater{}
\texttt{\ Application\ Programming\ Interface\ (API)\ } or
\texttt{\ API\ } )

\texttt{\ acrpl\ } : Reference an acronym in the text in plural form
(e.g. \texttt{\ acrpl("API")\ } -\textgreater{}
\texttt{\ Application\ Programming\ Interfaces\ (API)\ } or
\texttt{\ APIs\ } )

\texttt{\ acrs\ } : Reference an acronym in the text in short form (e.g.
\texttt{\ acrs("API")\ } -\textgreater{} \texttt{\ API\ } )

\texttt{\ acrspl\ } : Reference an acronym in the text in short form in
plural form (e.g. \texttt{\ acrpl("API")\ } -\textgreater{}
\texttt{\ APIs\ } )

\texttt{\ acrl\ } : Reference an acronym in the text in long form (e.g.
\texttt{\ acrl("API")\ } -\textgreater{}
\texttt{\ Application\ Programming\ Interface\ } )

\texttt{\ acrlpl\ } : Reference an acronym in the text in long form in
plural form (e.g. \texttt{\ acrlpl("API")\ } -\textgreater{}
\texttt{\ Application\ Programming\ Interfaces\ } )

\texttt{\ acrf\ } : Reference an acronym in the text in full form (e.g.
\texttt{\ acrf("API")\ } -\textgreater{}
\texttt{\ Application\ Programming\ Interface\ (API)\ } )

\texttt{\ acrfpl\ } : Reference an acronym in the text in full form in
plural form (e.g. \texttt{\ acrfpl("API")\ } -\textgreater{}
\texttt{\ Application\ Programming\ Interfaces\ (API)\ } )

\subsubsection{Definition}\label{definition}

To define acronyms use a dictionary and pass it to the acronyms
attribute of the template. The dictionary should contain the acronyms as
keys and their long forms as values.

\begin{Shaded}
\begin{Highlighting}[]
\NormalTok{\#let acronyms = (}
\NormalTok{  API: "Application Programming Interface",}
\NormalTok{  HTTP: "Hypertext Transfer Protocol",}
\NormalTok{  REST: "Representational State Transfer",}
\NormalTok{)}
\end{Highlighting}
\end{Shaded}

To define the plural form of an acronym use a array as value with the
first element being the singular form and the second element being the
plural form. If you don’t define the plural form, the template will
automatically add an “s� to the singular form.

\begin{Shaded}
\begin{Highlighting}[]
\NormalTok{\#let acronyms = (}
\NormalTok{  API: ("Application Programming Interface", "Application Programming Interfaces"),}
\NormalTok{  HTTP: ("Hypertext Transfer Protocol", "Hypertext Transfer Protocols"),}
\NormalTok{  REST: ("Representational State Transfer", "Representational State Transfers"),}
\NormalTok{)}
\end{Highlighting}
\end{Shaded}

\subsection{Glossary}\label{glossary}

Similar to the acronyms, this template provides a function to reference
glossary terms in the text. To use the function, you need to define the
glossary terms in the \texttt{\ glossary\ } attribute of the template.
The glossary terms referenced with the function below will be linked to
their definition in the list of glossary terms.

\subsubsection{Reference}\label{reference}

\texttt{\ gls\ } : Reference a glossary term in the text (e.g.
\texttt{\ gls("Vulnerability")\ } -\textgreater{} link to the definition
of “Vulnerability� in the glossary)

\subsubsection{Definition}\label{definition-1}

The definition works analogously to the acronyms. Define the glossary
terms in a dictionary and pass it to the glossary attribute of the
template. The dictionary should contain the glossary terms as keys and
their definitions as values.

\begin{Shaded}
\begin{Highlighting}[]
\NormalTok{\#let glossary = (}
\NormalTok{  Vulnerability: "A Vulnerability is a flaw in a computer system that weakens the overall security of the system.",}
\NormalTok{  Patch: "A patch is data that is intended to be used to modify an existing software resource such as a program or a file, often to fix bugs and security vulnerabilities.",}
\NormalTok{  Exploit: "An exploit is a method or piece of code that takes advantage of vulnerabilities in software, applications, networks, operating systems, or hardware, typically for malicious purposes.",}
\NormalTok{)}
\end{Highlighting}
\end{Shaded}

\href{/app?template=supercharged-dhbw&version=3.3.2}{Create project in
app}

\subsubsection{How to use}\label{how-to-use}

Click the button above to create a new project using this template in
the Typst app.

You can also use the Typst CLI to start a new project on your computer
using this command:

\begin{verbatim}
typst init @preview/supercharged-dhbw:3.3.2
\end{verbatim}

\includesvg[width=0.16667in,height=0.16667in]{/assets/icons/16-copy.svg}

\subsubsection{About}\label{about}

\begin{description}
\tightlist
\item[Author :]
\href{https://github.com/DannySeidel}{Danny Seidel}
\item[License:]
MIT
\item[Current version:]
3.3.2
\item[Last updated:]
November 4, 2024
\item[First released:]
May 14, 2024
\item[Archive size:]
26.9 kB
\href{https://packages.typst.org/preview/supercharged-dhbw-3.3.2.tar.gz}{\pandocbounded{\includesvg[keepaspectratio]{/assets/icons/16-download.svg}}}
\item[Repository:]
\href{https://github.com/DannySeidel/typst-dhbw-template}{GitHub}
\item[Categor ies :]
\begin{itemize}
\tightlist
\item[]
\item
  \pandocbounded{\includesvg[keepaspectratio]{/assets/icons/16-atom.svg}}
  \href{https://typst.app/universe/search/?category=paper}{Paper}
\item
  \pandocbounded{\includesvg[keepaspectratio]{/assets/icons/16-mortarboard.svg}}
  \href{https://typst.app/universe/search/?category=thesis}{Thesis}
\item
  \pandocbounded{\includesvg[keepaspectratio]{/assets/icons/16-speak.svg}}
  \href{https://typst.app/universe/search/?category=report}{Report}
\end{itemize}
\end{description}

\subsubsection{Where to report issues?}\label{where-to-report-issues}

This template is a project of Danny Seidel . Report issues on
\href{https://github.com/DannySeidel/typst-dhbw-template}{their
repository} . You can also try to ask for help with this template on the
\href{https://forum.typst.app}{Forum} .

Please report this template to the Typst team using the
\href{https://typst.app/contact}{contact form} if you believe it is a
safety hazard or infringes upon your rights.

\phantomsection\label{versions}
\subsubsection{Version history}\label{version-history}

\begin{longtable}[]{@{}ll@{}}
\toprule\noalign{}
Version & Release Date \\
\midrule\noalign{}
\endhead
\bottomrule\noalign{}
\endlastfoot
3.3.2 & November 4, 2024 \\
\href{https://typst.app/universe/package/supercharged-dhbw/3.3.1/}{3.3.1}
& October 3, 2024 \\
\href{https://typst.app/universe/package/supercharged-dhbw/3.3.0/}{3.3.0}
& September 22, 2024 \\
\href{https://typst.app/universe/package/supercharged-dhbw/3.2.0/}{3.2.0}
& September 17, 2024 \\
\href{https://typst.app/universe/package/supercharged-dhbw/3.1.1/}{3.1.1}
& August 26, 2024 \\
\href{https://typst.app/universe/package/supercharged-dhbw/3.1.0/}{3.1.0}
& August 21, 2024 \\
\href{https://typst.app/universe/package/supercharged-dhbw/3.0.0/}{3.0.0}
& August 8, 2024 \\
\href{https://typst.app/universe/package/supercharged-dhbw/2.2.0/}{2.2.0}
& July 29, 2024 \\
\href{https://typst.app/universe/package/supercharged-dhbw/2.1.0/}{2.1.0}
& July 19, 2024 \\
\href{https://typst.app/universe/package/supercharged-dhbw/2.0.2/}{2.0.2}
& July 4, 2024 \\
\href{https://typst.app/universe/package/supercharged-dhbw/2.0.1/}{2.0.1}
& July 4, 2024 \\
\href{https://typst.app/universe/package/supercharged-dhbw/2.0.0/}{2.0.0}
& July 2, 2024 \\
\href{https://typst.app/universe/package/supercharged-dhbw/1.5.0/}{1.5.0}
& June 24, 2024 \\
\href{https://typst.app/universe/package/supercharged-dhbw/1.4.0/}{1.4.0}
& June 10, 2024 \\
\href{https://typst.app/universe/package/supercharged-dhbw/1.3.1/}{1.3.1}
& May 27, 2024 \\
\href{https://typst.app/universe/package/supercharged-dhbw/1.3.0/}{1.3.0}
& May 23, 2024 \\
\href{https://typst.app/universe/package/supercharged-dhbw/1.2.0/}{1.2.0}
& May 16, 2024 \\
\href{https://typst.app/universe/package/supercharged-dhbw/1.1.0/}{1.1.0}
& May 16, 2024 \\
\href{https://typst.app/universe/package/supercharged-dhbw/1.0.0/}{1.0.0}
& May 14, 2024 \\
\end{longtable}

Typst GmbH did not create this template and cannot guarantee correct
functionality of this template or compatibility with any version of the
Typst compiler or app.
