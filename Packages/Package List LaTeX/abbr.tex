\title{typst.app/universe/package/abbr}

\phantomsection\label{banner}
\section{abbr}\label{abbr}

{ 0.1.0 }

An Abbreviations package.

\phantomsection\label{readme}
Short package for making the handling of abbreviations, acronyms, and
initialisms \emph{easy} .

Declare your abbreviations anywhere, use everywhere â€`` they get picked
up automatically.

\subsection{Features}\label{features}

\begin{itemize}
\tightlist
\item
  Automatic plurals, with optional overrides.
\item
  Automatic 1- or 2-column sorted list of abbreviations
\item
  Automatic links to list of abbreviations, if included.
\item
  styling configuration
\end{itemize}

\subsection{Getting started}\label{getting-started}

\begin{Shaded}
\begin{Highlighting}[]
\NormalTok{\#import "@preview/abbr:0.1.0"}

\NormalTok{\#abbr.list()}
\NormalTok{\#abbr.make(}
\NormalTok{  ("PDE", "Partial Differential Equation"),}
\NormalTok{  ("BC", "Boundary Condition"),}
\NormalTok{  ("DOF", "Degree of Freedom", "Degrees of Freedom"),}
\NormalTok{)}

\NormalTok{= Constrained Equations}

\NormalTok{\#abbr.pla[BC] constrain the \#abbr.pla[DOF] of the \#abbr.pla[PDE] they act on.\textbackslash{}}
\NormalTok{\#abbr.pla[BC] constrain the \#abbr.pla[DOF] of the \#abbr.pla[PDE] they act on.}

\NormalTok{\#abbr.add("MOL", "Method of Lines")}
\NormalTok{The \#abbr.a[MOL] is a procedure to solve \#abbr.pla[PDE] in time.}
\end{Highlighting}
\end{Shaded}

\pandocbounded{\includegraphics[keepaspectratio]{https://github.com/typst/packages/raw/main/packages/preview/abbr/0.1.0/example.png}}

\subsection{API Reference}\label{api-reference}

\subsubsection{Configuration}\label{configuration}

\begin{itemize}
\tightlist
\item
  \textbf{style} \texttt{\ (func)\ }\\
  Set a callable for styling the short version in the text.
\end{itemize}

\subsubsection{Creation}\label{creation}

\begin{itemize}
\item
  \textbf{add} \texttt{\ (short,\ long,\ long-plural)\ }\\
  Add single entry to use later.\\
  \texttt{\ long-plural\ } is \emph{optional} , if not given but used,
  an \texttt{\ s\ } is appended to create a plural.
\item
  \textbf{make} \texttt{\ (list,\ of,\ entries)\ }\\
  Add multiple entries, each of the form
  \texttt{\ (short,\ long,\ long-plural)\ } .
\end{itemize}

\subsubsection{Listing}\label{listing}

\begin{itemize}
\tightlist
\item
  \textbf{list} \texttt{\ (title)\ }\\
  Create an outline with all abbreviations in short and expanded form
\end{itemize}

\subsubsection{Usage}\label{usage}

\begin{itemize}
\tightlist
\item
  \textbf{s} \texttt{\ ()\ } - short\\
  force short form of abbreviation
\item
  \textbf{l} \texttt{\ ()\ } - long\\
  force long form of abbreviation
\item
  \textbf{a} \texttt{\ ()\ } - auto\\
  first occurence will be long form, the rest short
\item
  \textbf{pls} \texttt{\ ()\ } - plural short\\
  plural short form
\item
  \textbf{pll} \texttt{\ ()\ } - plural long\\
  plural long form
\item
  \textbf{pl} \texttt{\ ()\ } - plural automatic\\
  plural. first occurence long form, then short
\end{itemize}

\subsection{Why yet another Abbreviations
package?}\label{why-yet-another-abbreviations-package}

This mostly exists because I started working on it before checking if
somebody already made a package for it. After I saw that e.g.
\texttt{\ acrotastic\ } exists, I kept convincing myself a new package
still makes sense for the following reasons:

\begin{itemize}
\tightlist
\item
  Getting to know Typst
\item
  More automatic handling than other packages
\item
  Ability to keep keys as {[}Content{]} instead of having to stringify
  everything
\end{itemize}

Especially the last part seems to lower the friction of writing for me.
It seems silly, I know.

\subsection{Contributing}\label{contributing}

Please head over to the \href{https://sr.ht/~slowjo/typst-packages}{hub}
to find the mailing list and ticket tracker.

Or simply reach out on IRC (
\href{https://web.libera.chat/gamja/?autojoin=\#typst}{\#typst on
libera.chat} )!

\subsubsection{How to add}\label{how-to-add}

Copy this into your project and use the import as \texttt{\ abbr\ }

\begin{verbatim}
#import "@preview/abbr:0.1.0"
\end{verbatim}

\includesvg[width=0.16667in,height=0.16667in]{/assets/icons/16-copy.svg}

Check the docs for
\href{https://typst.app/docs/reference/scripting/\#packages}{more
information on how to import packages} .

\subsubsection{About}\label{about}

\begin{description}
\tightlist
\item[Author :]
\href{mailto:slowjo@halmen.xyz}{Jonathan Halmen}
\item[License:]
MIT
\item[Current version:]
0.1.0
\item[Last updated:]
November 5, 2024
\item[First released:]
November 5, 2024
\item[Archive size:]
3.40 kB
\href{https://packages.typst.org/preview/abbr-0.1.0.tar.gz}{\pandocbounded{\includesvg[keepaspectratio]{/assets/icons/16-download.svg}}}
\item[Repository:]
\href{https://git.sr.ht/~slowjo/typst-abbr}{git.sr.ht}
\item[Categor y :]
\begin{itemize}
\tightlist
\item[]
\item
  \pandocbounded{\includesvg[keepaspectratio]{/assets/icons/16-list-unordered.svg}}
  \href{https://typst.app/universe/search/?category=model}{Model}
\end{itemize}
\end{description}

\subsubsection{Where to report issues?}\label{where-to-report-issues}

This package is a project of Jonathan Halmen . Report issues on
\href{https://git.sr.ht/~slowjo/typst-abbr}{their repository} . You can
also try to ask for help with this package on the
\href{https://forum.typst.app}{Forum} .

Please report this package to the Typst team using the
\href{https://typst.app/contact}{contact form} if you believe it is a
safety hazard or infringes upon your rights.

\phantomsection\label{versions}
\subsubsection{Version history}\label{version-history}

\begin{longtable}[]{@{}ll@{}}
\toprule\noalign{}
Version & Release Date \\
\midrule\noalign{}
\endhead
\bottomrule\noalign{}
\endlastfoot
0.1.0 & November 5, 2024 \\
\end{longtable}

Typst GmbH did not create this package and cannot guarantee correct
functionality of this package or compatibility with any version of the
Typst compiler or app.
