\title{typst.app/universe/package/ofbnote}

\phantomsection\label{banner}
\phantomsection\label{template-thumbnail}
\pandocbounded{\includegraphics[keepaspectratio]{https://packages.typst.org/preview/thumbnails/ofbnote-0.2.0-small.webp}}

\section{ofbnote}\label{ofbnote}

{ 0.2.0 }

A document template using French Office for biodiversity design
guidelines

\href{/app?template=ofbnote&version=0.2.0}{Create project in app}

\phantomsection\label{readme}
This is a Typst template to help formatting documents according to the
French office for biodiversity design guidelines.

\subsection{Usage}\label{usage}

You can use the CLI to kick this project off using the command

\begin{verbatim}
typst init @preview/ofbnote
\end{verbatim}

Typst will create a new directory with all the files needed to get you
started.

\subsection{Configuration}\label{configuration}

This template exports the \texttt{\ ofbnote\ } function with one named
argument called \texttt{\ meta\ } which should be a dictionary of
metadata for the document. The \texttt{\ meta\ } dictionary can contain
the following fields:

\begin{itemize}
\tightlist
\item
  \texttt{\ title\ } : The document’s title as a string or content.
\item
  \texttt{\ authors\ } : The document’s author(s) as a string.
\item
  \texttt{\ date\ } : The document’s date as a string or content.
\item
  \texttt{\ version\ } : The document’s version as a string.
\end{itemize}

It may contains other values, but they have no effect on the final
document.

The function also accepts a single, positional argument for the body of
the paper.

The template will initialize your package with a sample call to the
\texttt{\ ofbnote\ } function in a show rule. If you want to change an
existing project to use this template, you can add a show rule like this
at the top of your file:

\begin{Shaded}
\begin{Highlighting}[]
\NormalTok{\#import "@preview/ofbnote:0.2.0": *}

\NormalTok{\#show: ofbnote.with( meta:(}
\NormalTok{  title: "My note",}
\NormalTok{  authors: "Me",}
\NormalTok{  date: "March 23rd, 2023",}
\NormalTok{  version: "1.0"}
\NormalTok{))}

\NormalTok{// Your content goes below.}
\end{Highlighting}
\end{Shaded}

\href{/app?template=ofbnote&version=0.2.0}{Create project in app}

\subsubsection{How to use}\label{how-to-use}

Click the button above to create a new project using this template in
the Typst app.

You can also use the Typst CLI to start a new project on your computer
using this command:

\begin{verbatim}
typst init @preview/ofbnote:0.2.0
\end{verbatim}

\includesvg[width=0.16667in,height=0.16667in]{/assets/icons/16-copy.svg}

\subsubsection{About}\label{about}

\begin{description}
\tightlist
\item[Author :]
François Hissel
\item[License:]
MIT-0
\item[Current version:]
0.2.0
\item[Last updated:]
August 12, 2024
\item[First released:]
August 12, 2024
\item[Minimum Typst version:]
0.11.0
\item[Archive size:]
9.14 kB
\href{https://packages.typst.org/preview/ofbnote-0.2.0.tar.gz}{\pandocbounded{\includesvg[keepaspectratio]{/assets/icons/16-download.svg}}}
\item[Categor ies :]
\begin{itemize}
\tightlist
\item[]
\item
  \pandocbounded{\includesvg[keepaspectratio]{/assets/icons/16-envelope.svg}}
  \href{https://typst.app/universe/search/?category=office}{Office}
\item
  \pandocbounded{\includesvg[keepaspectratio]{/assets/icons/16-speak.svg}}
  \href{https://typst.app/universe/search/?category=report}{Report}
\end{itemize}
\end{description}

\subsubsection{Where to report issues?}\label{where-to-report-issues}

This template is a project of François Hissel . You can also try to ask
for help with this template on the \href{https://forum.typst.app}{Forum}
.

Please report this template to the Typst team using the
\href{https://typst.app/contact}{contact form} if you believe it is a
safety hazard or infringes upon your rights.

\phantomsection\label{versions}
\subsubsection{Version history}\label{version-history}

\begin{longtable}[]{@{}ll@{}}
\toprule\noalign{}
Version & Release Date \\
\midrule\noalign{}
\endhead
\bottomrule\noalign{}
\endlastfoot
0.2.0 & August 12, 2024 \\
\end{longtable}

Typst GmbH did not create this template and cannot guarantee correct
functionality of this template or compatibility with any version of the
Typst compiler or app.


\title{typst.app/universe/package/quill}

\phantomsection\label{banner}
\section{quill}\label{quill}

{ 0.5.0 }

Effortlessly create quantum circuit diagrams.

{ } Featured Package

\phantomsection\label{readme}
\pandocbounded{\includegraphics[keepaspectratio]{https://github.com/user-attachments/assets/bf6bfe99-6667-41b0-9b45-13c73ed18590}}

\href{https://typst.app/universe/package/quill}{\pandocbounded{\includegraphics[keepaspectratio]{https://img.shields.io/badge/dynamic/toml?url=https\%3A\%2F\%2Fraw.githubusercontent.com\%2FMc-Zen\%2Fquill\%2Fv0.5.0\%2Ftypst.toml&query=\%24.package.version&prefix=v&logo=typst&label=package&color=239DAD}}}
\href{https://github.com/Mc-Zen/quill/actions/workflows/run_tests.yml}{\pandocbounded{\includesvg[keepaspectratio]{https://github.com/Mc-Zen/quill/actions/workflows/run_tests.yml/badge.svg}}}
\href{https://github.com/Mc-Zen/quill/blob/main/LICENSE}{\pandocbounded{\includegraphics[keepaspectratio]{https://img.shields.io/badge/license-MIT-blue}}}
\href{https://github.com/Mc-Zen/quill/releases/download/v0.5.0/quill-guide.pdf}{\pandocbounded{\includegraphics[keepaspectratio]{https://img.shields.io/badge/manual-.pdf-purple}}}

\textbf{Quill} is a package for creating quantum circuit diagrams in
\href{https://typst.app/}{Typst} .

\emph{Note, that this package is in beta and may still be undergoing
breaking changes. As new features like data types and scoped functions
will be added to Typst, this package will be adapted to profit from the
new paradigms.}

\emph{Meanwhile, we suggest importing everything from the package in a
local scope to avoid polluting the global namespace (see example
below).}

\begin{itemize}
\tightlist
\item
  \href{https://github.com/typst/packages/raw/main/packages/preview/quill/0.5.0/\#basic-usage}{\textbf{Usage}}
  \emph{quick introduction}
\item
  \href{https://github.com/typst/packages/raw/main/packages/preview/quill/0.5.0/\#cheat-sheet}{\textbf{Cheat
  sheet}} \emph{gallery for quickly viewing all kinds of gates}
\item
  \href{https://github.com/typst/packages/raw/main/packages/preview/quill/0.5.0/\#tequila}{\textbf{Tequila}}
  \emph{building (sub-)circuits in a way similar to QASM or Qiskit}
\item
  \href{https://github.com/typst/packages/raw/main/packages/preview/quill/0.5.0/\#examples}{\textbf{Examples}}
\item
  \href{https://github.com/typst/packages/raw/main/packages/preview/quill/0.5.0/\#changelog}{\textbf{Changelog}}
\end{itemize}

\subsection{Basic usage}\label{basic-usage}

The function \texttt{\ quantum-circuit()\ } takes any number of
positional gates and works somewhat similarly to the built-int Typst
functions \texttt{\ table()\ } or \texttt{\ grid()\ } . A variety of
different gate and instruction commands are available for adding
elements and integers can be used to produce any number of empty cells
(filled with the current wire style). A new wire is started by adding a
\texttt{\ {[}\textbackslash{}\ {]}\ } item.

\begin{Shaded}
\begin{Highlighting}[]
\NormalTok{\#\{}
\NormalTok{  import "@preview/quill:0.5.0": *}

\NormalTok{  quantum{-}circuit(}
\NormalTok{    lstick($|0〉$), $H$, ctrl(1), rstick($(|00〉+|11〉)/√2$, n: 2), [\textbackslash{} ],}
\NormalTok{    lstick($|0〉$), 1, targ(), 1}
\NormalTok{  )}
\NormalTok{\}}
\end{Highlighting}
\end{Shaded}

\pandocbounded{\includegraphics[keepaspectratio]{https://github.com/user-attachments/assets/53d0c581-ab62-44e3-abf5-5497993d7aba}}

Plain quantum gates â€'' such as a Hadamard gate â€'' can be written
with the shorthand notation \texttt{\ \$H\$\ } instead of the more
lengthy \texttt{\ gate(\$H\$)\ } . The latter offers more options,
however.

Refer to the
\href{https://github.com/Mc-Zen/quill/releases/download/v0.5.0/quill-guide.pdf}{user
guide} for a full documentation of this package. You can also look up
the documentation of any function by calling the help module, e.g.,
\texttt{\ help("gate")\ } in order to print the signature and
description of the \texttt{\ gate\ } command, just where you are
currently typing (powered by \href{https://github.com/Mc-Zen/tidy}{tidy}
).

\subsection{Cheat Sheet}\label{cheat-sheet}

Instead of listing every featured gate (as is done in the
\href{https://github.com/Mc-Zen/quill/releases/download/v0.5.0/quill-guide.pdf}{user
guide} ), this gallery quickly showcases a large selection of possible
gates and decorations that can be added to any quantum circuit.

\pandocbounded{\includegraphics[keepaspectratio]{https://github.com/user-attachments/assets/29987e5b-c373-4cd6-86a0-58e27d778fb1}}

\subsection{Tequila}\label{tequila}

\emph{Tequila} is a submodule that adds a completely different way of
building circuits.

\begin{Shaded}
\begin{Highlighting}[]
\NormalTok{\#import "@preview/quill:0.5.0" as quill: tequila as tq}

\NormalTok{\#quill.quantum{-}circuit(}
\NormalTok{  ..tq.build(}
\NormalTok{    tq.h(0),}
\NormalTok{    tq.cx(0, 1),}
\NormalTok{    tq.cx(0, 2),}
\NormalTok{  ),}
\NormalTok{  quill.gategroup(x: 2, y: 0, 3, 2)}
\NormalTok{)}
\end{Highlighting}
\end{Shaded}

This is similar to how \emph{QASM} and \emph{Qiskit} work: gates are
successively applied to the circuit which is then layed out
automatically by packing gates as tightly as possible. We start by
calling the \texttt{\ tq.build()\ } function and filling it with quantum
operations. This returns a collection of gates which we expand into the
circuit with the \texttt{\ ..\ } syntax. Now, we still have the option
to add annotations, groups, slices, or even more gates via manual
placement.

The syntax works analog to Qiskit. Available gates are \texttt{\ x\ } ,
\texttt{\ y\ } , \texttt{\ z\ } , \texttt{\ h\ } , \texttt{\ s\ } ,
\texttt{\ sdg\ } , \texttt{\ sx\ } , \texttt{\ sxdg\ } , \texttt{\ t\ }
, \texttt{\ tdg\ } , \texttt{\ p\ } , \texttt{\ rx\ } , \texttt{\ ry\ }
, \texttt{\ rz\ } , \texttt{\ u\ } , \texttt{\ cx\ } , \texttt{\ cz\ } ,
and \texttt{\ swap\ } . With \texttt{\ barrier\ } , an invisible barrier
can be inserted to prevent gates on different qubits to be packed
tightly. Finally, with \texttt{\ tq.gate\ } and \texttt{\ tq.mqgate\ } ,
a generic gate can be created. These two accept the same styling
arguments as the normal \texttt{\ gate\ } (or \texttt{\ mqgate\ } ).

Also like Qiskit, all qubit arguments support ranges, e.g.,
\texttt{\ tq.h(range(5))\ } adds a Hadamard gate on the first five
qubits and \texttt{\ tq.cx((0,\ 1),\ (1,\ 2))\ } adds two CX gates: one
from qubit 0 to 1 and one from qubit 1 to 2.

With Tequila, it is easy to build templates for quantum circuits and to
compose circuits of various building blocks. For this purpose,
\texttt{\ tq.build()\ } and the built-in templates all feature optional
\texttt{\ x\ } and \texttt{\ y\ } arguments to allow placing a
subcircuit at an arbitrary position of the circuit. As an example,
Tequila provides a \texttt{\ tq.graph-state()\ } template for quickly
drawing graph state preparation circuits.

The following example demonstrates how to compose multiple subcircuits.

\begin{Shaded}
\begin{Highlighting}[]
\NormalTok{\#import tequila as tq}

\NormalTok{\#quantum{-}circuit(}
\NormalTok{  ..tq.graph{-}state((0, 1), (1, 2)),}
\NormalTok{  ..tq.build(y: 3, }
\NormalTok{      tq.p($pi$, 0), }
\NormalTok{      tq.cx(0, (1, 2)), }
\NormalTok{    ),}
\NormalTok{  ..tq.graph{-}state(x: 6, y: 2, invert: true, (0, 1), (0, 2)),}
\NormalTok{  gategroup(x: 1, 3, 3),}
\NormalTok{  gategroup(x: 1, y: 3, 3, 3),}
\NormalTok{  gategroup(x: 6, y: 2, 3, 3),}
\NormalTok{  slice(x: 5)}
\NormalTok{)}
\end{Highlighting}
\end{Shaded}

\pandocbounded{\includegraphics[keepaspectratio]{https://github.com/user-attachments/assets/41c8d60a-1a5e-4d0b-a7f4-82756423f5a8}}

\subsection{Examples}\label{examples}

Some show-off examples, loosely replicating figures from
\href{https://www.cambridge.org/highereducation/books/quantum-computation-and-quantum-information/01E10196D0A682A6AEFFEA52D53BE9AE\#overview}{Quantum
Computation and Quantum Information by M. Nielsen and I. Chuang} . The
code for these examples can be found in the
\href{https://github.com/Mc-Zen/quill/tree/v0.5.0/examples}{example
folder} or in the
\href{https://github.com/Mc-Zen/quill/releases/download/v0.5.0/quill-guide.pdf}{user
guide} .

\pandocbounded{\includegraphics[keepaspectratio]{https://github.com/user-attachments/assets/f38abeb9-fc2f-4be4-9592-7932e07af764}}

\pandocbounded{\includegraphics[keepaspectratio]{https://github.com/user-attachments/assets/6e947f71-67dc-4522-936e-6d9b795a1bba}}

\pandocbounded{\includegraphics[keepaspectratio]{https://github.com/user-attachments/assets/3fc92cd0-915e-4c5e-893d-63dac6f83ade}}

\subsection{Contribution}\label{contribution}

If you spot an issue or have a suggestion, you are invited to
\href{https://github.com/Mc-Zen/quill/issues}{post it} or to contribute.
In
\href{https://github.com/Mc-Zen/quill/tree/v0.5.0/docs/architecture.md}{architecture.md}
, you can also find a description of the algorithm that forms the base
of \texttt{\ quantum-circuit()\ } .

\subsection{Tests}\label{tests}

This package uses
\href{https://github.com/tingerrr/typst-test/}{typst-test} for running
\href{https://github.com/Mc-Zen/quill/tree/v0.5.0/tests/}{tests} .

\subsection{Changelog}\label{changelog}

\subsubsection{v0.5.0}\label{v0.5.0}

\begin{itemize}
\tightlist
\item
  Added support for multi-controlled gates with Tequila.
\item
  Switched to using \texttt{\ context\ } instead of the now deprecated
  \texttt{\ style()\ } for measurement. Note: Starting with this
  version, Typst 0.11.0 or higher is required.
\end{itemize}

\subsubsection{v0.4.0}\label{v0.4.0}

\begin{itemize}
\tightlist
\item
  Alternative model for creating and composing circuits:
  \href{https://github.com/typst/packages/raw/main/packages/preview/quill/0.5.0/\#tequila}{Tequila}
  .
\end{itemize}

\subsubsection{v0.3.0}\label{v0.3.0}

\begin{itemize}
\tightlist
\item
  New features

  \begin{itemize}
  \tightlist
  \item
    Enable manual placement of gates,
    \texttt{\ gate(\$X\$,\ x:\ 3,\ y:\ 1)\ } , similar to built-in
    \texttt{\ table()\ } in addition to automatic placement. This works
    for most elements, not only gates.
  \item
    Add parameter \texttt{\ pad\ } to \texttt{\ lstick()\ } and
    \texttt{\ rstick()\ } .
  \item
    Add parameter \texttt{\ fill-wires\ } to
    \texttt{\ quantum-circuit()\ } . All wires are filled unto the end
    (determined by the longest wire) by default (breaking change
    âš~ï¸?). This behavior can be reverted by setting
    \texttt{\ fill-wires:\ false\ } .
  \item
    \texttt{\ gategroup()\ } \texttt{\ slice()\ } and
    \texttt{\ annotate()\ } can now be placed above or below the circuit
    with \texttt{\ z:\ "above"\ } and \texttt{\ z:\ "below"\ } .
  \item
    \texttt{\ help()\ } command for quickly displaying the documentation
    of a given function, e.g., \texttt{\ help("gate")\ } . Powered by
    \href{https://github.com/Mc-Zen/tidy}{tidy} .
  \end{itemize}
\item
  Improvements:

  \begin{itemize}
  \tightlist
  \item
    Complete rework of circuit layout implementation

    \begin{itemize}
    \tightlist
    \item
      allows transparent gates since wires are not drawn through gates
      anymore. The default fill is now \texttt{\ auto\ } and using
      \texttt{\ none\ } sets the background to transparent.
    \item
      \texttt{\ midstick\ } is now transparent by default.
    \end{itemize}
  \item
    \texttt{\ setwire()\ } can now be used to override only partial wire
    settings, such as wire color \texttt{\ setwire(1,\ stroke:\ blue)\ }
    , width \texttt{\ setwire(1,\ stroke:\ 1pt)\ } or wire distance, all
    separately. Before, some settings were reset.
  \end{itemize}
\item
  Fixes:

  \begin{itemize}
  \tightlist
  \item
    Fixed \texttt{\ lstick\ } / \texttt{\ rstick\ } when equation
    numbering is turned on.
  \end{itemize}
\item
  Removed:

  \begin{itemize}
  \tightlist
  \item
    The already deprecated \texttt{\ scale-factor\ } (use
    \texttt{\ scale\ } instead)
  \end{itemize}
\end{itemize}

\subsubsection{v0.2.1}\label{v0.2.1}

\begin{itemize}
\tightlist
\item
  Improvements:

  \begin{itemize}
  \tightlist
  \item
    Add \texttt{\ fill\ } parameter to \texttt{\ midstick()\ } .
  \item
    Add \texttt{\ bend\ } parameter to \texttt{\ permute()\ } .
  \item
    Add \texttt{\ separation\ } parameter to \texttt{\ permute()\ } .
  \end{itemize}
\item
  Fixes:

  \begin{itemize}
  \tightlist
  \item
    With Typst 0.11.0, \texttt{\ scale()\ } now takes into account outer
    alignment. This broke the positioning of centered/right-aligned
    circuits, e.g., ones put into a \texttt{\ figure()\ } .
  \item
    Change wires to be drawn all through \texttt{\ ctrl()\ } , making it
    consistent to \texttt{\ swap()\ } and \texttt{\ targ()\ } .
  \end{itemize}
\end{itemize}

\subsubsection{v0.2.0}\label{v0.2.0}

\begin{itemize}
\tightlist
\item
  New features:

  \begin{itemize}
  \tightlist
  \item
    Add arbitrary labels to any \texttt{\ gate\ } (also derived gates
    such as \texttt{\ meter\ } , \texttt{\ ctrl\ } , …),
    \texttt{\ gategroup\ } or \texttt{\ slice\ } that can be anchored to
    any of the nine 2d alignments.
  \item
    Add optional gate inputs and outputs for multi-qubit gates (see
    gallery).
  \item
    Implicit gates (breaking change âš~ï¸?): a content item
    automatically becomes a gate, so you can just type
    \texttt{\ \$H\$\ } instead of \texttt{\ gate(\$H\$)\ } (of course,
    the \texttt{\ gate()\ } function is still important in order to use
    the many available options).
  \end{itemize}
\item
  Other breaking changes âš~ï¸?:

  \begin{itemize}
  \tightlist
  \item
    \texttt{\ slice()\ } has no \texttt{\ dx\ } and \texttt{\ dy\ }
    parameters anymore. Instead, labels are handled through
    \texttt{\ label\ } exactly as in \texttt{\ gate()\ } . Also the
    \texttt{\ wires\ } parameter is replaced with \texttt{\ n\ } for
    consistency with other multi-qubit gates.
  \item
    Swap order of row and column parameters in \texttt{\ annotate()\ }
    to make it consistent with built-in Typst functions.
  \end{itemize}
\item
  Improvements:

  \begin{itemize}
  \tightlist
  \item
    Improve layout (allow row/column spacing and min lengths to be
    specified in em-lengths).
  \item
    Automatic bounds computation, even for labels.
  \item
    Improve meter (allow multi-qubit gate meters and respect global
    (per-circuit) gate padding).d
  \end{itemize}
\item
  Fixes:

  \begin{itemize}
  \tightlist
  \item
    \texttt{\ lstick\ } / \texttt{\ rstick\ } braces broke with Typst
    0.7.0.
  \item
    \texttt{\ lstick\ } / \texttt{\ rstick\ } bounds.
  \end{itemize}
\item
  Documentation

  \begin{itemize}
  \tightlist
  \item
    Add section on creating custom gates.
  \item
    Add section on using labels.
  \item
    Explain usage of \texttt{\ slice()\ } and \texttt{\ gategroup()\ } .
  \end{itemize}
\end{itemize}

\subsubsection{v0.1.0}\label{v0.1.0}

Initial Release

\subsubsection{How to add}\label{how-to-add}

Copy this into your project and use the import as \texttt{\ quill\ }

\begin{verbatim}
#import "@preview/quill:0.5.0"
\end{verbatim}

\includesvg[width=0.16667in,height=0.16667in]{/assets/icons/16-copy.svg}

Check the docs for
\href{https://typst.app/docs/reference/scripting/\#packages}{more
information on how to import packages} .

\subsubsection{About}\label{about}

\begin{description}
\tightlist
\item[Author :]
\href{https://github.com/Mc-Zen}{Mc-Zen}
\item[License:]
MIT
\item[Current version:]
0.5.0
\item[Last updated:]
October 24, 2024
\item[First released:]
June 28, 2023
\item[Minimum Typst version:]
0.11.0
\item[Archive size:]
24.9 kB
\href{https://packages.typst.org/preview/quill-0.5.0.tar.gz}{\pandocbounded{\includesvg[keepaspectratio]{/assets/icons/16-download.svg}}}
\item[Repository:]
\href{https://github.com/Mc-Zen/quill}{GitHub}
\item[Discipline s :]
\begin{itemize}
\tightlist
\item[]
\item
  \href{https://typst.app/universe/search/?discipline=physics}{Physics}
\item
  \href{https://typst.app/universe/search/?discipline=computer-science}{Computer
  Science}
\end{itemize}
\item[Categor y :]
\begin{itemize}
\tightlist
\item[]
\item
  \pandocbounded{\includesvg[keepaspectratio]{/assets/icons/16-chart.svg}}
  \href{https://typst.app/universe/search/?category=visualization}{Visualization}
\end{itemize}
\end{description}

\subsubsection{Where to report issues?}\label{where-to-report-issues}

This package is a project of Mc-Zen . Report issues on
\href{https://github.com/Mc-Zen/quill}{their repository} . You can also
try to ask for help with this package on the
\href{https://forum.typst.app}{Forum} .

Please report this package to the Typst team using the
\href{https://typst.app/contact}{contact form} if you believe it is a
safety hazard or infringes upon your rights.

\phantomsection\label{versions}
\subsubsection{Version history}\label{version-history}

\begin{longtable}[]{@{}ll@{}}
\toprule\noalign{}
Version & Release Date \\
\midrule\noalign{}
\endhead
\bottomrule\noalign{}
\endlastfoot
0.5.0 & October 24, 2024 \\
\href{https://typst.app/universe/package/quill/0.4.0/}{0.4.0} &
September 16, 2024 \\
\href{https://typst.app/universe/package/quill/0.3.0/}{0.3.0} & May 22,
2024 \\
\href{https://typst.app/universe/package/quill/0.2.1/}{0.2.1} & March
11, 2024 \\
\href{https://typst.app/universe/package/quill/0.2.0/}{0.2.0} &
September 28, 2023 \\
\href{https://typst.app/universe/package/quill/0.1.0/}{0.1.0} & June 28,
2023 \\
\end{longtable}

Typst GmbH did not create this package and cannot guarantee correct
functionality of this package or compatibility with any version of the
Typst compiler or app.


\title{typst.app/universe/package/metalogo}

\phantomsection\label{banner}
\section{metalogo}\label{metalogo}

{ 1.0.2 }

Typeset various LaTeX logos

\phantomsection\label{readme}
Typeset LaTeX compiler logos in
\href{https://github.com/typst/typst}{typst} .

\subsection{usage}\label{usage}

From
\href{https://github.com/typst/packages/raw/main/packages/preview/metalogo/1.0.2/demo.typ}{./demo.typ}
:

\begin{Shaded}
\begin{Highlighting}[]
\NormalTok{\#import "@preview/metalogo:1.0.2": TeX, LaTeX, XeLaTeX, XeTeX, LuaLaTeX}

\NormalTok{\#LaTeX is a typestting program based on \#TeX. Some people use \#XeLaTeX}
\NormalTok{(sometimes \#XeTeX), or \#LuaLaTeX to typeset their documents.}

\NormalTok{People who are afraid of \#LaTeX and its complex macro system may use typst}
\NormalTok{instead.}
\end{Highlighting}
\end{Shaded}

Output:

\pandocbounded{\includesvg[keepaspectratio]{https://github.com/typst/packages/raw/main/packages/preview/metalogo/1.0.2/demo.svg}}

\subsubsection{How to add}\label{how-to-add}

Copy this into your project and use the import as \texttt{\ metalogo\ }

\begin{verbatim}
#import "@preview/metalogo:1.0.2"
\end{verbatim}

\includesvg[width=0.16667in,height=0.16667in]{/assets/icons/16-copy.svg}

Check the docs for
\href{https://typst.app/docs/reference/scripting/\#packages}{more
information on how to import packages} .

\subsubsection{About}\label{about}

\begin{description}
\tightlist
\item[Author :]
\href{mailto:loek@pipeframe.xyz}{Loek Le Blansch}
\item[License:]
MIT
\item[Current version:]
1.0.2
\item[Last updated:]
August 26, 2024
\item[First released:]
August 26, 2024
\item[Minimum Typst version:]
0.10.0
\item[Archive size:]
1.57 kB
\href{https://packages.typst.org/preview/metalogo-1.0.2.tar.gz}{\pandocbounded{\includesvg[keepaspectratio]{/assets/icons/16-download.svg}}}
\item[Repository:]
\href{https://github.com/lonkaars/typst-metalogo.git}{GitHub}
\end{description}

\subsubsection{Where to report issues?}\label{where-to-report-issues}

This package is a project of Loek Le Blansch . Report issues on
\href{https://github.com/lonkaars/typst-metalogo.git}{their repository}
. You can also try to ask for help with this package on the
\href{https://forum.typst.app}{Forum} .

Please report this package to the Typst team using the
\href{https://typst.app/contact}{contact form} if you believe it is a
safety hazard or infringes upon your rights.

\phantomsection\label{versions}
\subsubsection{Version history}\label{version-history}

\begin{longtable}[]{@{}ll@{}}
\toprule\noalign{}
Version & Release Date \\
\midrule\noalign{}
\endhead
\bottomrule\noalign{}
\endlastfoot
1.0.2 & August 26, 2024 \\
\end{longtable}

Typst GmbH did not create this package and cannot guarantee correct
functionality of this package or compatibility with any version of the
Typst compiler or app.


\title{typst.app/universe/package/blind-cvpr}

\phantomsection\label{banner}
\phantomsection\label{template-thumbnail}
\pandocbounded{\includegraphics[keepaspectratio]{https://packages.typst.org/preview/thumbnails/blind-cvpr-0.5.0-small.webp}}

\section{blind-cvpr}\label{blind-cvpr}

{ 0.5.0 }

CVPR-style paper template to publish at the Computer Vision and Pattern
Recognition (CVPR) conferences.

\href{/app?template=blind-cvpr&version=0.5.0}{Create project in app}

\phantomsection\label{readme}
\subsection{Usage}\label{usage}

You can use this template in the Typst web app by clicking \emph{Start
from template} on the dashboard and searching for
\texttt{\ blind-cvpr\ } .

Alternatively, you can use the CLI to kick this project off using the
command

\begin{Shaded}
\begin{Highlighting}[]
\NormalTok{typst init @preview/blind{-}cvpr}
\end{Highlighting}
\end{Shaded}

Typst will create a new directory with all the files needed to get you
started.

\subsection{Configuration}\label{configuration}

This template exports the \texttt{\ cvpr2022\ } and
\texttt{\ cvpr2025\ } styling rule with the following named arguments.

\begin{itemize}
\tightlist
\item
  \texttt{\ title\ } : The paper’s title as content.
\item
  \texttt{\ authors\ } : An array of author dictionaries. Each of the
  author dictionaries must have a name key and can have the keys
  department, organization, location, and email.
\item
  \texttt{\ keywords\ } : Publication keywords (used in PDF metadata).
\item
  \texttt{\ date\ } : Creation date (used in PDF metadata).
\item
  \texttt{\ abstract\ } : The content of a brief summary of the paper or
  none. Appears at the top under the title.
\item
  \texttt{\ bibliography\ } : The result of a call to the bibliography
  function or none. The function also accepts a single, positional
  argument for the body of the paper.
\item
  \texttt{\ appendix\ } : Content to append after bibliography section.
\item
  \texttt{\ accepted\ } : If this is set to \texttt{\ false\ } then
  anonymized ready for submission document is produced;
  \texttt{\ accepted:\ true\ } produces camera-redy version. If the
  argument is set to \texttt{\ none\ } then preprint version is produced
  (can be uploaded to arXiv).
\item
  \texttt{\ id\ } : Identifier of a submission.
\end{itemize}

The template will initialize your package with a sample call to the
\texttt{\ cvpr2025\ } function in a show rule. If you want to change an
existing project to use this template, you can add a show rule at the
top of your file.

\begin{Shaded}
\begin{Highlighting}[]
\NormalTok{\#import "@preview/blind{-}cvpr:0.5.0": cvpr2025}

\NormalTok{\#show: cvpr2025.with(}
\NormalTok{  title: [LaTeX Author Guidelines for CVPR Proceedings],}
\NormalTok{  authors: (authors, affls),}
\NormalTok{  keywords: (),}
\NormalTok{  abstract: [}
\NormalTok{    The ABSTRACT is to be in fully justified italicized text, at the top of the}
\NormalTok{    left{-}hand column, below the author and affiliation information. Use the}
\NormalTok{    word "Abstract" as the title, in 12{-}point Times, boldface type, centered}
\NormalTok{    relative to the column, initially capitalized. The abstract is to be in}
\NormalTok{    10{-}point, single{-}spaced type. Leave two blank lines after the Abstract,}
\NormalTok{    then begin the main text. Look at previous CVPR abstracts to get a feel for}
\NormalTok{    style and length.}
\NormalTok{  ],}
\NormalTok{  bibliography: bibliography("main.bib"),}
\NormalTok{  accepted: false,}
\NormalTok{  id: none,}
\NormalTok{)}
\end{Highlighting}
\end{Shaded}

\subsection{Issues}\label{issues}

\begin{itemize}
\item
  In case of US Letter, column sizes + gap does not equals to text width
  (2 * 3.25 + 5/16 != 6 + 7/8). It seems that correct gap should be 3/8.
\item
  At the moment of Typst v0.11.0, it is impossible to indent the first
  paragraph in a section (see
  \href{https://github.com/typst/typst/issues/311}{typst/typst\#311} ).
  The workaround is to add indentation manually as follows.

\begin{Shaded}
\begin{Highlighting}[]
\NormalTok{== H2}

\NormalTok{\#h(12pt)  Manually as space for the first paragraph.}
\NormalTok{Lorem ipsum dolor sit amet, consectetur adipiscing elit, sed do.}

\NormalTok{// The second one is just fine.}
\NormalTok{Lorem ipsum dolor sit amet, consectetur adipiscing elit, sed do.}
\end{Highlighting}
\end{Shaded}

  Also, we add \texttt{\ indent\ } constant as a shortcut for
  \texttt{\ h(12pt)\ } .

  This issue is relevant to CVPR 2022. In the 2025 template there is no
  indentaino of the first paragraph in section.
\item
  At the moment Typst v0.11.0 does not allow flexible customization of
  citation styles. Specifically, CVPR 2022 citation lookes like
  \texttt{\ {[}42{]}\ } where number is colored hyperlink. In order to
  achive this, we shouuld provide custom CSL-style and then colorize
  number and put it into square parenthesis in typst markup.
\item
  CVPR 2022 requires simple ruler which enumerates lines in regular
  intervals whilst CVPR2025 already requires a ruler which add line
  numers per line in paragraph or heading. Thus we need the next major
  Typst release v0.12.0 for ruler. With the next Typst release, we can
  do the following.

\begin{Shaded}
\begin{Highlighting}[]
\NormalTok{set par.line(numbering: "1")}
\NormalTok{show figure: set par.line(numbering: none)}
\end{Highlighting}
\end{Shaded}

  For implementation details see
  \href{https://github.com/typst/typst/pull/4516}{typst/typst\#4516} .
\item
  CVPR 2022 and 2025 requires IEEE-like bibliography style but does not
  follow its guidelines closely. Since writing CSL-style files is
  tedious task, we adopt close enough bibliography style from Zotero.
\end{itemize}

\subsection{References}\label{references}

\begin{itemize}
\tightlist
\item
  CVPR 2022 conference
  \href{https://cvpr2022.thecvf.com/author-guidelines\#dates}{web site}
  .
\item
  CVPR 2025 conference
  \href{https://cvpr.thecvf.com/Conferences/2025}{web site} .
\end{itemize}

\href{/app?template=blind-cvpr&version=0.5.0}{Create project in app}

\subsubsection{How to use}\label{how-to-use}

Click the button above to create a new project using this template in
the Typst app.

You can also use the Typst CLI to start a new project on your computer
using this command:

\begin{verbatim}
typst init @preview/blind-cvpr:0.5.0
\end{verbatim}

\includesvg[width=0.16667in,height=0.16667in]{/assets/icons/16-copy.svg}

\subsubsection{About}\label{about}

\begin{description}
\tightlist
\item[Author :]
daskol
\item[License:]
MIT
\item[Current version:]
0.5.0
\item[Last updated:]
September 22, 2024
\item[First released:]
September 22, 2024
\item[Minimum Typst version:]
0.11.1
\item[Archive size:]
18.3 kB
\href{https://packages.typst.org/preview/blind-cvpr-0.5.0.tar.gz}{\pandocbounded{\includesvg[keepaspectratio]{/assets/icons/16-download.svg}}}
\item[Repository:]
\href{https://github.com/daskol/typst-templates}{GitHub}
\item[Discipline s :]
\begin{itemize}
\tightlist
\item[]
\item
  \href{https://typst.app/universe/search/?discipline=computer-science}{Computer
  Science}
\item
  \href{https://typst.app/universe/search/?discipline=mathematics}{Mathematics}
\end{itemize}
\item[Categor y :]
\begin{itemize}
\tightlist
\item[]
\item
  \pandocbounded{\includesvg[keepaspectratio]{/assets/icons/16-atom.svg}}
  \href{https://typst.app/universe/search/?category=paper}{Paper}
\end{itemize}
\end{description}

\subsubsection{Where to report issues?}\label{where-to-report-issues}

This template is a project of daskol . Report issues on
\href{https://github.com/daskol/typst-templates}{their repository} . You
can also try to ask for help with this template on the
\href{https://forum.typst.app}{Forum} .

Please report this template to the Typst team using the
\href{https://typst.app/contact}{contact form} if you believe it is a
safety hazard or infringes upon your rights.

\phantomsection\label{versions}
\subsubsection{Version history}\label{version-history}

\begin{longtable}[]{@{}ll@{}}
\toprule\noalign{}
Version & Release Date \\
\midrule\noalign{}
\endhead
\bottomrule\noalign{}
\endlastfoot
0.5.0 & September 22, 2024 \\
\end{longtable}

Typst GmbH did not create this template and cannot guarantee correct
functionality of this template or compatibility with any version of the
Typst compiler or app.


\title{typst.app/universe/package/simple-preavis}

\phantomsection\label{banner}
\phantomsection\label{template-thumbnail}
\pandocbounded{\includegraphics[keepaspectratio]{https://packages.typst.org/preview/thumbnails/simple-preavis-0.1.0-small.webp}}

\section{simple-preavis}\label{simple-preavis}

{ 0.1.0 }

ðŸ``-- a french move out letter

\href{/app?template=simple-preavis&version=0.1.0}{Create project in app}

\phantomsection\label{readme}
\textbf{simple-preavis} est un template typst pour écrire une lettre de
préavis d’état des lieux Ã~ son propriétaire.

Il est fortement inspiré de cet
\href{https://www.service-public.fr/simulateur/calcul/CongeLogement}{outil}
réalisés par les services publics �.

\subsection{Utilisation}\label{utilisation}

\subsubsection{Exemple d’utilisation}\label{exemple-duxe2utilisation}

\begin{Shaded}
\begin{Highlighting}[]
\NormalTok{\#import "@preview/simple{-}preavis:0.1.0":*}
\NormalTok{\#lettre{-}preavis(}
\NormalTok{  locataire: locataire(}
\NormalTok{     "Dupont locataire",}
\NormalTok{     "Jean",}
\NormalTok{     adresse(}
\NormalTok{       "123 rue de la Paix",}
\NormalTok{       "75000",}
\NormalTok{       "Paris",}
\NormalTok{       complement: "Appartement 2"}
\NormalTok{    )}
\NormalTok{  ),}
\NormalTok{  proprietaire: proprietaire(}
\NormalTok{     "Martin proprietaire",}
\NormalTok{    "Sophie",}
\NormalTok{     adresse(}
\NormalTok{       "456 avenue des Champs{-}Élysées",}
\NormalTok{       "75008",}
\NormalTok{       "Paris"}
\NormalTok{    ),}
\NormalTok{    "Madame"}
\NormalTok{  ),}
\NormalTok{  date{-}etat{-}des{-}lieux: datetime(year:2024, month:9, day:21)}
\NormalTok{)}
\end{Highlighting}
\end{Shaded}

\subsection{TODO}\label{todo}

\begin{itemize}
\tightlist
\item
  {[} {]} Supporter plusieurs locataires
\item
  {[} {]} Supporter la législation zone tendu en fonction du code
  postal
\item
  {[} {]} Améliorer la documentation des fonctions
\item
  {[} {]} Séparer en une librairie et un template pour que cela
  ressemble plus aux autres template types
\end{itemize}

\subsection{Mention license}\label{mention-license}

Conformément Ã~ la license
\href{https://github.com/etalab/licence-ouverte/blob/master/LO.md}{etalab}

\begin{itemize}
\tightlist
\item
  Condérant : Direction de l’information légale et administrative
  (Premier ministre)
\item
  Date de mise Ã~ jour : Vérifié le 23 Avril 2024
\end{itemize}

\subsection{License}\label{license}

\href{https://github.com/typst/packages/raw/main/packages/preview/simple-preavis/0.1.0/LICENSE}{license
MIT}

\href{/app?template=simple-preavis&version=0.1.0}{Create project in app}

\subsubsection{How to use}\label{how-to-use}

Click the button above to create a new project using this template in
the Typst app.

You can also use the Typst CLI to start a new project on your computer
using this command:

\begin{verbatim}
typst init @preview/simple-preavis:0.1.0
\end{verbatim}

\includesvg[width=0.16667in,height=0.16667in]{/assets/icons/16-copy.svg}

\subsubsection{About}\label{about}

\begin{description}
\tightlist
\item[Author :]
\href{https://github.com/mathias-aparicio/}{Mathias APARICIO}
\item[License:]
MIT
\item[Current version:]
0.1.0
\item[Last updated:]
July 23, 2024
\item[First released:]
July 23, 2024
\item[Archive size:]
3.09 kB
\href{https://packages.typst.org/preview/simple-preavis-0.1.0.tar.gz}{\pandocbounded{\includesvg[keepaspectratio]{/assets/icons/16-download.svg}}}
\item[Repository:]
\href{https://github.com/mathias-aparicio/simple-preavis}{GitHub}
\item[Categor y :]
\begin{itemize}
\tightlist
\item[]
\item
  \pandocbounded{\includesvg[keepaspectratio]{/assets/icons/16-envelope.svg}}
  \href{https://typst.app/universe/search/?category=office}{Office}
\end{itemize}
\end{description}

\subsubsection{Where to report issues?}\label{where-to-report-issues}

This template is a project of Mathias APARICIO . Report issues on
\href{https://github.com/mathias-aparicio/simple-preavis}{their
repository} . You can also try to ask for help with this template on the
\href{https://forum.typst.app}{Forum} .

Please report this template to the Typst team using the
\href{https://typst.app/contact}{contact form} if you believe it is a
safety hazard or infringes upon your rights.

\phantomsection\label{versions}
\subsubsection{Version history}\label{version-history}

\begin{longtable}[]{@{}ll@{}}
\toprule\noalign{}
Version & Release Date \\
\midrule\noalign{}
\endhead
\bottomrule\noalign{}
\endlastfoot
0.1.0 & July 23, 2024 \\
\end{longtable}

Typst GmbH did not create this template and cannot guarantee correct
functionality of this template or compatibility with any version of the
Typst compiler or app.


\title{typst.app/universe/package/bamdone-aiaa}

\phantomsection\label{banner}
\phantomsection\label{template-thumbnail}
\pandocbounded{\includegraphics[keepaspectratio]{https://packages.typst.org/preview/thumbnails/bamdone-aiaa-0.1.1-small.webp}}

\section{bamdone-aiaa}\label{bamdone-aiaa}

{ 0.1.1 }

An American Institute of Aeronautics and Astronautics (AIAA) template
for conferences.

\href{/app?template=bamdone-aiaa&version=0.1.1}{Create project in app}

\phantomsection\label{readme}
This is a Typst template for a one-column paper from the proceedings of
the American Institute of Aeronautics and Astronautics. The paper is
tightly spaced, fits a lot of content and comes preconfigured for
numeric citations from BibLaTeX or Hayagriva files.

\subsection{Usage}\label{usage}

You can use this template in the Typst web app by clicking “Start from
template� on the dashboard and searching for \texttt{\ bamdone-aiaa\ }
.

Alternatively, you can use the CLI to kick this project off using the
command

\begin{verbatim}
typst init @preview/bamdone-aiaa
\end{verbatim}

Typst will create a new directory with all the files needed to get you
started.

\subsection{Configuration}\label{configuration}

This template exports the \texttt{\ aiaa\ } function with the following
named arguments:

\begin{itemize}
\tightlist
\item
  \texttt{\ title\ } : The paper’s title as content.
\item
  \texttt{\ authors-and-affiliations\ } : An array of author
  dictionaries and affiliation dictionaries. Author dictionaries must
  have a \texttt{\ name\ } key and can have the keys \texttt{\ job\ } ,
  \texttt{\ department\ } , \texttt{\ aiaa\ } is optional. Affiliation
  dictionaries must have the keys \texttt{\ institution\ } ,
  \texttt{\ city\ } , \texttt{\ state\ } , \texttt{\ zip\ } , and
  \texttt{\ country\ } .
\item
  \texttt{\ abstract\ } : The content of a brief summary of the paper or
  \texttt{\ none\ } . Appears at the top of the first column in
  boldface. Shall be \texttt{\ content\ } .
\item
  \texttt{\ paper-size\ } : Defaults to \texttt{\ us-letter\ } . Specify
  a
  \href{https://typst.app/docs/reference/layout/page/\#parameters-paper}{paper
  size string} to change the page format.
\item
  \texttt{\ bibliography\ } : The result of a call to the
  \texttt{\ bibliography\ } function or \texttt{\ none\ } . Specifying
  this will configure numeric, AIAA-style citations.
\end{itemize}

The function also accepts a single, positional argument for the body of
the paper.

The template will initialize your package with a sample call to the
\texttt{\ aiaa\ } function in a show rule. If you want to change an
existing project to use this template, you can add a show rule like this
at the top of your file:

\begin{Shaded}
\begin{Highlighting}[]
\NormalTok{\#import "@preview/bamdone{-}aiaa:0.1.0": aiaa}

\NormalTok{\#show: aiaa.with(}
\NormalTok{  title: [A typesetting system to untangle the scientific writing process],}
\NormalTok{  abstract: [}
\NormalTok{    These instructions give you guidelines for preparing papers for AIAA Technical Papers. Use this document as a template if you are using Typst. Otherwise, use this document as an instruction set. Define all symbols used in the abstract. Do not cite references in the abstract. The footnote on the first page should list the Job Title and AIAA Member Grade for each author, if known. Authors do not have to be AIAA members.}
\NormalTok{  ],}
\NormalTok{  authors: (}
\NormalTok{      (}
\NormalTok{        name:"First A. Author",}
\NormalTok{        job:"Insert Job Title",}
\NormalTok{        department:"Department Name",}
\NormalTok{        aiaa:"and AIAA Member Grade (if any) for first author"}
\NormalTok{      ),}
\NormalTok{      (}
\NormalTok{        institution:"Business or Academic Affiliation\textquotesingle{}s Full Name 1",}
\NormalTok{        city:"City",}
\NormalTok{        state:"State",}
\NormalTok{        zip:"Zip Code",}
\NormalTok{        country:"Country"}
\NormalTok{      ),}
\NormalTok{  ),}
\NormalTok{  bibliography: bibliography("refs.bib"),}
\NormalTok{)}

\NormalTok{// Your content goes below.}
\end{Highlighting}
\end{Shaded}

\href{/app?template=bamdone-aiaa&version=0.1.1}{Create project in app}

\subsubsection{How to use}\label{how-to-use}

Click the button above to create a new project using this template in
the Typst app.

You can also use the Typst CLI to start a new project on your computer
using this command:

\begin{verbatim}
typst init @preview/bamdone-aiaa:0.1.1
\end{verbatim}

\includesvg[width=0.16667in,height=0.16667in]{/assets/icons/16-copy.svg}

\subsubsection{About}\label{about}

\begin{description}
\tightlist
\item[Author s :]
\href{https://www.isaacew.com/}{Isaac Weintraub} \&
\href{https://avonmoll.github.io/}{Alexander Von Moll}
\item[License:]
MIT-0
\item[Current version:]
0.1.1
\item[Last updated:]
May 14, 2024
\item[First released:]
March 23, 2024
\item[Minimum Typst version:]
0.11.0
\item[Archive size:]
14.4 kB
\href{https://packages.typst.org/preview/bamdone-aiaa-0.1.1.tar.gz}{\pandocbounded{\includesvg[keepaspectratio]{/assets/icons/16-download.svg}}}
\item[Repository:]
\href{https://github.com/isaacew/aiaa-typst}{GitHub}
\item[Discipline s :]
\begin{itemize}
\tightlist
\item[]
\item
  \href{https://typst.app/universe/search/?discipline=engineering}{Engineering}
\item
  \href{https://typst.app/universe/search/?discipline=computer-science}{Computer
  Science}
\item
  \href{https://typst.app/universe/search/?discipline=mathematics}{Mathematics}
\item
  \href{https://typst.app/universe/search/?discipline=communication}{Communication}
\item
  \href{https://typst.app/universe/search/?discipline=transportation}{Transportation}
\item
  \href{https://typst.app/universe/search/?discipline=education}{Education}
\end{itemize}
\item[Categor y :]
\begin{itemize}
\tightlist
\item[]
\item
  \pandocbounded{\includesvg[keepaspectratio]{/assets/icons/16-atom.svg}}
  \href{https://typst.app/universe/search/?category=paper}{Paper}
\end{itemize}
\end{description}

\subsubsection{Where to report issues?}\label{where-to-report-issues}

This template is a project of Isaac Weintraub and Alexander Von Moll .
Report issues on \href{https://github.com/isaacew/aiaa-typst}{their
repository} . You can also try to ask for help with this template on the
\href{https://forum.typst.app}{Forum} .

Please report this template to the Typst team using the
\href{https://typst.app/contact}{contact form} if you believe it is a
safety hazard or infringes upon your rights.

\phantomsection\label{versions}
\subsubsection{Version history}\label{version-history}

\begin{longtable}[]{@{}ll@{}}
\toprule\noalign{}
Version & Release Date \\
\midrule\noalign{}
\endhead
\bottomrule\noalign{}
\endlastfoot
0.1.1 & May 14, 2024 \\
\href{https://typst.app/universe/package/bamdone-aiaa/0.1.0/}{0.1.0} &
March 23, 2024 \\
\end{longtable}

Typst GmbH did not create this template and cannot guarantee correct
functionality of this template or compatibility with any version of the
Typst compiler or app.


\title{typst.app/universe/package/suiji}

\phantomsection\label{banner}
\section{suiji}\label{suiji}

{ 0.3.0 }

A highly efficient random number generator for Typst

{ } Featured Package

\phantomsection\label{readme}
\href{https://github.com/liuguangxi/suiji}{Suiji} (�机 in Chinese,
/suíjī/, meaning random) is a high efficient random number generator in
Typst. Partial algorithm is inherited from
\href{https://www.gnu.org/software/gsl}{GSL} and most APIs are similar
to
\href{https://numpy.org/doc/stable/reference/random/generator.html}{NumPy
Random Generator} . It provides pure function implementation and does
not rely on any global state variables, resulting in better performance
and independency.

\subsection{Features}\label{features}

\begin{itemize}
\tightlist
\item
  All functions are immutable, which means results of random are
  completely deterministic.
\item
  Core random engine chooses “Maximally equidistributed combined
  Tausworthe generator� and “LCG�.
\item
  Generate random integers or floats from various distribution.
\item
  Randomly shuffle an array of objects.
\item
  Randomly sample from an array of objects.
\item
  Generate blind text of Simplified Chinese.
\end{itemize}

\subsection{Examples}\label{examples}

The example below uses \texttt{\ suiji\ } and \texttt{\ cetz\ } packages
to create a trajectory of a random walk.

\begin{Shaded}
\begin{Highlighting}[]
\NormalTok{\#import "@preview/suiji:0.3.0": *}
\NormalTok{\#import "@preview/cetz:0.2.2"}

\NormalTok{\#set page(width: auto, height: auto, margin: 0.5cm)}

\NormalTok{\#cetz.canvas(length: 5pt, \{}
\NormalTok{  import cetz.draw: *}

\NormalTok{  let n = 2000}
\NormalTok{  let (x, y) = (0, 0)}
\NormalTok{  let (x{-}new, y{-}new) = (0, 0)}
\NormalTok{  let rng = gen{-}rng(42)}
\NormalTok{  let v = ()}

\NormalTok{  for i in range(n) \{}
\NormalTok{    (rng, v) = uniform(rng, low: {-}2.0, high: 2.0, size: 2)}
\NormalTok{    (x{-}new, y{-}new) = (x {-} v.at(1), y {-} v.at(0))}
\NormalTok{    let col = color.mix((blue.transparentize(20\%), 1{-}i/n), (green.transparentize(20\%), i/n))}
\NormalTok{    line(stroke: (paint: col, cap: "round", thickness: 2pt),}
\NormalTok{      (x, y), (x{-}new, y{-}new)}
\NormalTok{    )}
\NormalTok{    (x, y) = (x{-}new, y{-}new)}
\NormalTok{  \}}
\NormalTok{\})}
\end{Highlighting}
\end{Shaded}

\pandocbounded{\includegraphics[keepaspectratio]{https://github.com/typst/packages/raw/main/packages/preview/suiji/0.3.0/examples/random-walk.png}}

Another example is drawing the the famous \textbf{Matrix} rain effect of
falling green characters in a terminal.

\begin{Shaded}
\begin{Highlighting}[]
\NormalTok{\#import "@preview/suiji:0.3.0": *}
\NormalTok{\#import "@preview/cetz:0.2.2"}

\NormalTok{\#set page(width: auto, height: auto, margin: 0pt)}

\NormalTok{\#cetz.canvas(length: 1pt, \{}
\NormalTok{  import cetz.draw: *}

\NormalTok{  let font{-}size = 10}
\NormalTok{  let num{-}col = 80}
\NormalTok{  let num{-}row = 32}
\NormalTok{  let text{-}len = 16}
\NormalTok{  let seq = "abcdefghijklmnopqrstuvwxyz!@\#$\%\^{}\&*".split("").slice(1, 35).map(it =\textgreater{} raw(it))}
\NormalTok{  let rng = gen{-}rng(42)}
\NormalTok{  let num{-}cnt = 0}
\NormalTok{  let val = 0}
\NormalTok{  let chars = ()}

\NormalTok{  rect(({-}10, {-}10), (font{-}size * (num{-}col {-} 1) * 0.6 + 10, font{-}size * (num{-}row {-} 1) + 10), fill: black)}

\NormalTok{  for c in range(num{-}col) \{}
\NormalTok{    (rng, num{-}cnt) = integers(rng, low: 1, high: 3)}
\NormalTok{    for cnt in range(num{-}cnt) \{}
\NormalTok{      (rng, val) = integers(rng, low: {-}10, high: num{-}row {-} 2)}
\NormalTok{      (rng, chars) = choice(rng, seq, size: text{-}len)}
\NormalTok{      for i in range(text{-}len) \{}
\NormalTok{        let y = i + val}
\NormalTok{        if y \textgreater{}= 0 and y \textless{} num{-}row \{}
\NormalTok{          let col = green.transparentize((i / text{-}len) * 100\%)}
\NormalTok{          content(}
\NormalTok{            (c * font{-}size * 0.6, y * font{-}size),}
\NormalTok{            text(size: font{-}size * 1pt, fill:col, stroke: (text{-}len {-} i) * 0.04pt + col, chars.at(i))}
\NormalTok{          )}
\NormalTok{        \}}
\NormalTok{      \}}
\NormalTok{    \}}
\NormalTok{  \}}
\NormalTok{\})}
\end{Highlighting}
\end{Shaded}

\pandocbounded{\includegraphics[keepaspectratio]{https://github.com/typst/packages/raw/main/packages/preview/suiji/0.3.0/examples/matrix-rain.png}}

\subsection{Usage}\label{usage}

Import \texttt{\ suiji\ } module first before use any random functions
from it.

\begin{Shaded}
\begin{Highlighting}[]
\NormalTok{\#import "@preview/suiji:0.3.0": *}
\end{Highlighting}
\end{Shaded}

For functions that generate various random numbers or randomly shuffle,
a random number generator object ( \textbf{rng} ) is required as both
input and output arguments. And the original \textbf{rng} should be
created by function \texttt{\ gen-rng\ } , with an integer as the
argument of seed. This calling style seems to be a little inconvenient,
as it is limited by the programming paradigm. For function
\texttt{\ discrete\ } , the given probalilities of the discrete events
should be preprocessed by function \texttt{\ discrete-preproc\ } , whose
output serves as an input argument of \texttt{\ discrete\ } .

Another set of functions with the same functionality provides higher
performance (about 3 times faster) and has the suffix \texttt{\ -f\ } in
their names. For example, \texttt{\ gen-rng-f\ } and
\texttt{\ integers-f\ } are the fast versions of \texttt{\ gen-rng\ }
and \texttt{\ integers\ } , respectively.

The function \texttt{\ rand-sc\ } creates blind text of Simplified
Chinese. This function yields a Chinese-like Lorem Ipsum blind text with
the given number of words, where punctuations are optional.

The code below generates several random permutations of 0 to 9. Each
time after function \texttt{\ shuffle-f\ } is called, the value of
variable \texttt{\ rng\ } is updated, so generated permutations are
different.

\begin{Shaded}
\begin{Highlighting}[]
\NormalTok{\#\{}
\NormalTok{  let rng = gen{-}rng{-}f(42)}
\NormalTok{  let a = ()}
\NormalTok{  for i in range(5) \{}
\NormalTok{    (rng, a) = shuffle{-}f(rng, range(10))}
\NormalTok{    [\#(a.map(it =\textgreater{} str(it)).join("  ")) \textbackslash{} ]}
\NormalTok{  \}}
\NormalTok{\}}
\end{Highlighting}
\end{Shaded}

\pandocbounded{\includegraphics[keepaspectratio]{https://github.com/typst/packages/raw/main/packages/preview/suiji/0.3.0/examples/random-permutation.png}}

For more codes with these functions see
\href{https://github.com/typst/packages/raw/main/packages/preview/suiji/0.3.0/tests}{tests}
.

\subsection{Reference}\label{reference}

\subsubsection{\texorpdfstring{\texttt{\ gen-rng\ } /
\texttt{\ gen-rng-f\ }}{ gen-rng  /  gen-rng-f }}\label{gen-rng-gen-rng-f}

Construct a new random number generator with a seed.

\begin{Shaded}
\begin{Highlighting}[]
\NormalTok{\#let gen{-}rng(seed) = \{...\}}
\end{Highlighting}
\end{Shaded}

\begin{itemize}
\item
  \textbf{Input Arguments}

  \begin{itemize}
  \tightlist
  \item
    \texttt{\ seed\ } : {[} \texttt{\ int\ } {]} value of seed.
  \end{itemize}
\item
  \textbf{Output Arguments}

  \begin{itemize}
  \tightlist
  \item
    \texttt{\ rng\ } : {[} \texttt{\ object\ } {]} generated object of
    random number generator.
  \end{itemize}
\end{itemize}

\subsubsection{\texorpdfstring{\texttt{\ randi-f\ }}{ randi-f }}\label{randi-f}

Return a raw random integer from {[}0, 2\^{}31).

\begin{Shaded}
\begin{Highlighting}[]
\NormalTok{\#let randi{-}f(rng) = \{...\}}
\end{Highlighting}
\end{Shaded}

\begin{itemize}
\item
  \textbf{Input Arguments}

  \begin{itemize}
  \tightlist
  \item
    \texttt{\ rng\ } : {[} \texttt{\ object\ } \textbar{}
    \texttt{\ int\ } {]} object of random number generator (generated by
    function \texttt{\ *-f\ } ).
  \end{itemize}
\item
  \textbf{Output Arguments}

  \begin{itemize}
  \tightlist
  \item
    \texttt{\ rng-out\ } : {[} \texttt{\ object\ } \textbar{}
    \texttt{\ int\ } {]} updated object of random number generator
    (random integer from the interval {[}0, 2\^{}31-1{]}).
  \end{itemize}
\end{itemize}

\subsubsection{\texorpdfstring{\texttt{\ integers\ } /
\texttt{\ integers-f\ }}{ integers  /  integers-f }}\label{integers-integers-f}

Return random integers from \texttt{\ low\ } (inclusive) to
\texttt{\ high\ } (exclusive).

\begin{Shaded}
\begin{Highlighting}[]
\NormalTok{\#let integers(rng, low: 0, high: 100, size: none, endpoint: false) = \{...\}}
\end{Highlighting}
\end{Shaded}

\begin{itemize}
\item
  \textbf{Input Arguments}

  \begin{itemize}
  \tightlist
  \item
    \texttt{\ rng\ } : {[} \texttt{\ object\ } {]} object of random
    number generator.
  \item
    \texttt{\ low\ } : {[} \texttt{\ int\ } {]} lowest (signed) integers
    to be drawn from the distribution, optional.
  \item
    \texttt{\ high\ } : {[} \texttt{\ int\ } {]} one above the largest
    (signed) integer to be drawn from the distribution, optional.
  \item
    \texttt{\ size\ } : {[} \texttt{\ none\ } or \texttt{\ int\ } {]}
    returned array size, must be none or non-negative integer, optional.
  \item
    \texttt{\ endpoint\ } : {[} \texttt{\ bool\ } {]} if true, sample
    from the interval {[} \texttt{\ low\ } , \texttt{\ high\ } {]}
    instead of the default {[} \texttt{\ low\ } , \texttt{\ high\ } ),
    optional.
  \end{itemize}
\item
  \textbf{Output Arguments}

  \begin{itemize}
  \tightlist
  \item
    {[} \texttt{\ array\ } {]} : ( \texttt{\ rng-out\ } ,
    \texttt{\ arr-out\ } )

    \begin{itemize}
    \tightlist
    \item
      \texttt{\ rng-out\ } : {[} \texttt{\ object\ } {]} updated object
      of random number generator.
    \item
      \texttt{\ arr-out\ } : {[} \texttt{\ int\ } \textbar{}
      \texttt{\ array\ } of \texttt{\ int\ } {]} array of random
      numbers.
    \end{itemize}
  \end{itemize}
\end{itemize}

\subsubsection{\texorpdfstring{\texttt{\ random\ } /
\texttt{\ random-f\ }}{ random  /  random-f }}\label{random-random-f}

Return random floats in the half-open interval {[}0.0, 1.0).

\begin{Shaded}
\begin{Highlighting}[]
\NormalTok{\#let random(rng, size: none) = \{...\}}
\end{Highlighting}
\end{Shaded}

\begin{itemize}
\item
  \textbf{Input Arguments}

  \begin{itemize}
  \tightlist
  \item
    \texttt{\ rng\ } : {[} \texttt{\ object\ } {]} object of random
    number generator.
  \item
    \texttt{\ size\ } : {[} \texttt{\ none\ } or \texttt{\ int\ } {]}
    returned array size, must be none or non-negative integer, optional.
  \end{itemize}
\item
  \textbf{Output Arguments}

  \begin{itemize}
  \tightlist
  \item
    {[} \texttt{\ array\ } {]} : ( \texttt{\ rng-out\ } ,
    \texttt{\ arr-out\ } )

    \begin{itemize}
    \tightlist
    \item
      \texttt{\ rng-out\ } : {[} \texttt{\ object\ } {]} updated object
      of random number generator.
    \item
      \texttt{\ arr-out\ } : {[} \texttt{\ float\ } \textbar{}
      \texttt{\ array\ } of \texttt{\ float\ } {]} array of random
      numbers.
    \end{itemize}
  \end{itemize}
\end{itemize}

\subsubsection{\texorpdfstring{\texttt{\ uniform\ } /
\texttt{\ uniform-f\ }}{ uniform  /  uniform-f }}\label{uniform-uniform-f}

Draw samples from a uniform distribution. Samples are uniformly
distributed over the half-open interval {[} \texttt{\ low\ } ,
\texttt{\ high\ } ) (includes \texttt{\ low\ } , but excludes
\texttt{\ high\ } ).

\begin{Shaded}
\begin{Highlighting}[]
\NormalTok{\#let uniform(rng, low: 0.0, high: 1.0, size: none) = \{...\}}
\end{Highlighting}
\end{Shaded}

\begin{itemize}
\item
  \textbf{Input Arguments}

  \begin{itemize}
  \tightlist
  \item
    \texttt{\ rng\ } : {[} \texttt{\ object\ } {]} object of random
    number generator.
  \item
    \texttt{\ low\ } : {[} \texttt{\ float\ } {]} lower boundary of the
    output interval, optional.
  \item
    \texttt{\ high\ } : {[} \texttt{\ float\ } {]} upper boundary of the
    output interval, optional.
  \item
    \texttt{\ size\ } : {[} \texttt{\ none\ } or \texttt{\ int\ } {]}
    returned array size, must be none or non-negative integer, optional.
  \end{itemize}
\item
  \textbf{Output Arguments}

  \begin{itemize}
  \tightlist
  \item
    {[} \texttt{\ array\ } {]} : ( \texttt{\ rng-out\ } ,
    \texttt{\ arr-out\ } )

    \begin{itemize}
    \tightlist
    \item
      \texttt{\ rng-out\ } : {[} \texttt{\ object\ } {]} updated object
      of random number generator.
    \item
      \texttt{\ arr-out\ } : {[} \texttt{\ float\ } \textbar{}
      \texttt{\ array\ } of \texttt{\ float\ } {]} array of random
      numbers.
    \end{itemize}
  \end{itemize}
\end{itemize}

\subsubsection{\texorpdfstring{\texttt{\ normal\ } /
\texttt{\ normal-f\ }}{ normal  /  normal-f }}\label{normal-normal-f}

Draw random samples from a normal (Gaussian) distribution.

\begin{Shaded}
\begin{Highlighting}[]
\NormalTok{\#let normal(rng, loc: 0.0, scale: 1.0, size: none) = \{...\}}
\end{Highlighting}
\end{Shaded}

\begin{itemize}
\item
  \textbf{Input Arguments}

  \begin{itemize}
  \tightlist
  \item
    \texttt{\ rng\ } : {[} \texttt{\ object\ } {]} object of random
    number generator.
  \item
    \texttt{\ loc\ } : {[} \texttt{\ float\ } {]} mean (centre) of the
    distribution, optional.
  \item
    \texttt{\ scale\ } : {[} \texttt{\ float\ } {]} standard deviation
    (spread or width) of the distribution, must be non-negative,
    optional.
  \item
    \texttt{\ size\ } : {[} \texttt{\ none\ } or \texttt{\ int\ } {]}
    returned array size, must be none or non-negative integer, optional.
  \end{itemize}
\item
  \textbf{Output Arguments}

  \begin{itemize}
  \tightlist
  \item
    {[} \texttt{\ array\ } {]} : ( \texttt{\ rng-out\ } ,
    \texttt{\ arr-out\ } )

    \begin{itemize}
    \tightlist
    \item
      \texttt{\ rng-out\ } : {[} \texttt{\ object\ } {]} updated object
      of random number generator.
    \item
      \texttt{\ arr-out\ } : {[} \texttt{\ float\ } \textbar{}
      \texttt{\ array\ } of \texttt{\ float\ } {]} array of random
      numbers.
    \end{itemize}
  \end{itemize}
\end{itemize}

\subsubsection{\texorpdfstring{\texttt{\ discrete-preproc\ } and
\texttt{\ discrete\ } / \texttt{\ discrete-preproc-f\ } and
\texttt{\ discrete-f\ }}{ discrete-preproc  and  discrete  /  discrete-preproc-f  and  discrete-f }}\label{discrete-preproc-and-discrete-discrete-preproc-f-and-discrete-f}

Return random indices from the given probalilities of the discrete
events.

\begin{Shaded}
\begin{Highlighting}[]
\NormalTok{\#let discrete{-}preproc(p) = \{...\}}
\end{Highlighting}
\end{Shaded}

\begin{itemize}
\item
  \textbf{Input Arguments}

  \begin{itemize}
  \tightlist
  \item
    \texttt{\ p\ } : {[} \texttt{\ array\ } of \texttt{\ int\ } or
    \texttt{\ float\ } {]} the array of probalilities of the discrete
    events, probalilities must be non-negative.
  \end{itemize}
\item
  \textbf{Output Arguments}

  \begin{itemize}
  \tightlist
  \item
    \texttt{\ g\ } : {[} \texttt{\ object\ } {]} generated object that
    contains the lookup table.
  \end{itemize}
\end{itemize}

\begin{Shaded}
\begin{Highlighting}[]
\NormalTok{\#let discrete(rng, g, size: none) = \{...\}}
\end{Highlighting}
\end{Shaded}

\begin{itemize}
\item
  \textbf{Input Arguments}

  \begin{itemize}
  \tightlist
  \item
    \texttt{\ rng\ } : {[} \texttt{\ object\ } {]} object of random
    number generator.
  \item
    \texttt{\ g\ } : {[} \texttt{\ object\ } {]} generated object that
    contains the lookup table by \texttt{\ discrete-preproc\ } function.
  \item
    \texttt{\ size\ } : {[} \texttt{\ none\ } or \texttt{\ int\ } {]}
    returned array size, must be none or non-negative integer, optional.
  \end{itemize}
\item
  \textbf{Output Arguments}

  \begin{itemize}
  \tightlist
  \item
    {[} \texttt{\ array\ } {]} : ( \texttt{\ rng-out\ } ,
    \texttt{\ arr-out\ } )

    \begin{itemize}
    \tightlist
    \item
      \texttt{\ rng-out\ } : {[} \texttt{\ object\ } {]} updated object
      of random number generator.
    \item
      \texttt{\ arr-out\ } : {[} \texttt{\ int\ } \textbar{}
      \texttt{\ array\ } of \texttt{\ int\ } {]} array of random
      indices.
    \end{itemize}
  \end{itemize}
\end{itemize}

\subsubsection{\texorpdfstring{\texttt{\ shuffle\ } /
\texttt{\ shuffle-f\ }}{ shuffle  /  shuffle-f }}\label{shuffle-shuffle-f}

Randomly shuffle a given array.

\begin{Shaded}
\begin{Highlighting}[]
\NormalTok{\#let shuffle(rng, arr) = \{...\}}
\end{Highlighting}
\end{Shaded}

\begin{itemize}
\item
  \textbf{Input Arguments}

  \begin{itemize}
  \tightlist
  \item
    \texttt{\ rng\ } : {[} \texttt{\ object\ } {]} object of random
    number generator.
  \item
    \texttt{\ arr\ } : {[} \texttt{\ array\ } {]} the array to be
    shuffled.
  \end{itemize}
\item
  \textbf{Output Arguments}

  \begin{itemize}
  \tightlist
  \item
    {[} \texttt{\ array\ } {]} : ( \texttt{\ rng-out\ } ,
    \texttt{\ arr-out\ } )

    \begin{itemize}
    \tightlist
    \item
      \texttt{\ rng-out\ } : {[} \texttt{\ object\ } {]} updated object
      of random number generator.
    \item
      \texttt{\ arr-out\ } : {[} \texttt{\ array\ } {]} shuffled array.
    \end{itemize}
  \end{itemize}
\end{itemize}

\subsubsection{\texorpdfstring{\texttt{\ choice\ } /
\texttt{\ choice-f\ }}{ choice  /  choice-f }}\label{choice-choice-f}

Generate random samples from a given array.

\begin{Shaded}
\begin{Highlighting}[]
\NormalTok{\#let choice(rng, arr, size: none, replacement: true, permutation: true) = \{...\}}
\end{Highlighting}
\end{Shaded}

\begin{itemize}
\item
  \textbf{Input Arguments}

  \begin{itemize}
  \tightlist
  \item
    \texttt{\ rng\ } : {[} \texttt{\ object\ } {]} object of random
    number generator.
  \item
    \texttt{\ arr\ } : {[} \texttt{\ array\ } {]} the array to be
    sampled.
  \item
    \texttt{\ size\ } : {[} \texttt{\ none\ } or \texttt{\ int\ } {]}
    returned array size, must be none or non-negative integer, optional.
  \item
    \texttt{\ replacement\ } : {[} \texttt{\ bool\ } {]} whether the
    sample is with or without replacement, optional; default is true,
    meaning that a value of \texttt{\ arr\ } can be selected multiple
    times.
  \item
    \texttt{\ permutation\ } : {[} \texttt{\ bool\ } {]} whether the
    sample is permuted when sampling without replacement, optional;
    default is true, false provides a speedup.
  \end{itemize}
\item
  \textbf{Output Arguments}

  \begin{itemize}
  \tightlist
  \item
    {[} \texttt{\ array\ } {]} : ( \texttt{\ rng-out\ } ,
    \texttt{\ arr-out\ } )

    \begin{itemize}
    \tightlist
    \item
      \texttt{\ rng-out\ } : {[} \texttt{\ object\ } {]} updated object
      of random number generator.
    \item
      \texttt{\ arr-out\ } : {[} \texttt{\ array\ } {]} generated random
      samples.
    \end{itemize}
  \end{itemize}
\end{itemize}

\subsubsection{\texorpdfstring{\texttt{\ rand-sc\ }}{ rand-sc }}\label{rand-sc}

Generate blind text of Simplified Chinese.

\begin{Shaded}
\begin{Highlighting}[]
\NormalTok{\#let rand{-}sc(words, seed: 42, punctuation: false, gap: 10) = \{...\}}
\end{Highlighting}
\end{Shaded}

\begin{itemize}
\item
  \textbf{Input Arguments}

  \begin{itemize}
  \tightlist
  \item
    \texttt{\ words\ } : {[} \texttt{\ int\ } {]} the length of the
    blind text in pure words.
  \item
    \texttt{\ seed\ } : {[} \texttt{\ int\ } {]} value of seed,
    optional.
  \item
    \texttt{\ punctuation\ } : {[} \texttt{\ bool\ } {]} if true, insert
    punctuations in generated words, optional.
  \item
    \texttt{\ gap\ } : {[} \texttt{\ int\ } {]} average gap between
    punctuations, optional.
  \end{itemize}
\item
  \textbf{Output Arguments}

  \begin{itemize}
  \tightlist
  \item
    {[} \texttt{\ str\ } {]} : generated blind text of Simplified
    Chinese.
  \end{itemize}
\end{itemize}

\subsubsection{How to add}\label{how-to-add}

Copy this into your project and use the import as \texttt{\ suiji\ }

\begin{verbatim}
#import "@preview/suiji:0.3.0"
\end{verbatim}

\includesvg[width=0.16667in,height=0.16667in]{/assets/icons/16-copy.svg}

Check the docs for
\href{https://typst.app/docs/reference/scripting/\#packages}{more
information on how to import packages} .

\subsubsection{About}\label{about}

\begin{description}
\tightlist
\item[Author :]
\href{https://github.com/liuguangxi}{Guangxi Liu}
\item[License:]
MIT
\item[Current version:]
0.3.0
\item[Last updated:]
April 16, 2024
\item[First released:]
March 19, 2024
\item[Minimum Typst version:]
0.11.0
\item[Archive size:]
17.4 kB
\href{https://packages.typst.org/preview/suiji-0.3.0.tar.gz}{\pandocbounded{\includesvg[keepaspectratio]{/assets/icons/16-download.svg}}}
\item[Repository:]
\href{https://github.com/liuguangxi/suiji}{GitHub}
\item[Categor y :]
\begin{itemize}
\tightlist
\item[]
\item
  \pandocbounded{\includesvg[keepaspectratio]{/assets/icons/16-hammer.svg}}
  \href{https://typst.app/universe/search/?category=utility}{Utility}
\end{itemize}
\end{description}

\subsubsection{Where to report issues?}\label{where-to-report-issues}

This package is a project of Guangxi Liu . Report issues on
\href{https://github.com/liuguangxi/suiji}{their repository} . You can
also try to ask for help with this package on the
\href{https://forum.typst.app}{Forum} .

Please report this package to the Typst team using the
\href{https://typst.app/contact}{contact form} if you believe it is a
safety hazard or infringes upon your rights.

\phantomsection\label{versions}
\subsubsection{Version history}\label{version-history}

\begin{longtable}[]{@{}ll@{}}
\toprule\noalign{}
Version & Release Date \\
\midrule\noalign{}
\endhead
\bottomrule\noalign{}
\endlastfoot
0.3.0 & April 16, 2024 \\
\href{https://typst.app/universe/package/suiji/0.2.2/}{0.2.2} & April 9,
2024 \\
\href{https://typst.app/universe/package/suiji/0.2.1/}{0.2.1} & March
28, 2024 \\
\href{https://typst.app/universe/package/suiji/0.2.0/}{0.2.0} & March
22, 2024 \\
\href{https://typst.app/universe/package/suiji/0.1.0/}{0.1.0} & March
19, 2024 \\
\end{longtable}

Typst GmbH did not create this package and cannot guarantee correct
functionality of this package or compatibility with any version of the
Typst compiler or app.


\title{typst.app/universe/package/zen-zine}

\phantomsection\label{banner}
\phantomsection\label{template-thumbnail}
\pandocbounded{\includegraphics[keepaspectratio]{https://packages.typst.org/preview/thumbnails/zen-zine-0.1.0-small.webp}}

\section{zen-zine}\label{zen-zine}

{ 0.1.0 }

Excellently type-set a fun little zine!

\href{/app?template=zen-zine&version=0.1.0}{Create project in app}

\phantomsection\label{readme}
Excellently type-set a cute little zine about your favorite topic!

Providing your eight pages in order will produce a US-Letter page with
the content in a layout ready to be folded into a zine! The content is
wrapped before movement so that padding and alignment are respected.

Here is the template and its preview:

\begin{Shaded}
\begin{Highlighting}[]
\NormalTok{\#import "@preview/zen{-}zine:0.1.0": zine}

\NormalTok{\#set document(author: "Tom", title: "Zen Zine Example")}
\NormalTok{\#set text(font: "Linux Libertine", lang: "en")}

\NormalTok{\#let my\_eight\_pages = (}
\NormalTok{  range(8).map(}
\NormalTok{    number =\textgreater{} [}
\NormalTok{      \#pad(2em, text(10em, align(center, str(number))))}
\NormalTok{    ]}
\NormalTok{  )}
\NormalTok{)}

\NormalTok{// provide your content pages in order and they}
\NormalTok{// are placed into the zine template positions.}
\NormalTok{// the content is wrapped before movement so that}
\NormalTok{// padding and alignment are respected.}
\NormalTok{\#zine(}
\NormalTok{  // draw\_border: true,}
\NormalTok{  // zine\_page\_margin: 5pt,}
\NormalTok{  contents: my\_eight\_pages}
\NormalTok{)}
\end{Highlighting}
\end{Shaded}

\pandocbounded{\includegraphics[keepaspectratio]{https://github.com/typst/packages/raw/main/packages/preview/zen-zine/0.1.0/template/preview.png}}

\subsection{Improvement Ideas}\label{improvement-ideas}

Roughly in order of priority.

\begin{itemize}
\tightlist
\item
  Write documentation and generate a manual
\item
  Deduce \texttt{\ page\ } properties so that user can change the page
  they wish to use.

  \begin{itemize}
  \tightlist
  \item
    Make sure the page is \texttt{\ flipped\ } and deduce the zine page
    width and height from the full page width and height (and the zine
    margin)
  \item
    I’m currently struggling with finding out the page properties
    (what’s the \texttt{\ \#get\ } equivalent to \texttt{\ \#set\ } ?)
  \end{itemize}
\item
  Add other zine sizes (there is a 16 page one I believe?)
\item
  Digital mode where zine pages are separate pages (of the same size)
  rather than ‘sub pages’ of a printer page
\end{itemize}

\href{/app?template=zen-zine&version=0.1.0}{Create project in app}

\subsubsection{How to use}\label{how-to-use}

Click the button above to create a new project using this template in
the Typst app.

You can also use the Typst CLI to start a new project on your computer
using this command:

\begin{verbatim}
typst init @preview/zen-zine:0.1.0
\end{verbatim}

\includesvg[width=0.16667in,height=0.16667in]{/assets/icons/16-copy.svg}

\subsubsection{About}\label{about}

\begin{description}
\tightlist
\item[Author :]
\href{https://github.com/tomeichlersmith}{Tom Eichlersmith}
\item[License:]
MIT
\item[Current version:]
0.1.0
\item[Last updated:]
April 4, 2024
\item[First released:]
April 4, 2024
\item[Minimum Typst version:]
0.11.0
\item[Archive size:]
2.34 kB
\href{https://packages.typst.org/preview/zen-zine-0.1.0.tar.gz}{\pandocbounded{\includesvg[keepaspectratio]{/assets/icons/16-download.svg}}}
\item[Repository:]
\href{https://github.com/tomeichlersmith/zen-zine}{GitHub}
\item[Categor ies :]
\begin{itemize}
\tightlist
\item[]
\item
  \pandocbounded{\includesvg[keepaspectratio]{/assets/icons/16-smile.svg}}
  \href{https://typst.app/universe/search/?category=fun}{Fun}
\item
  \pandocbounded{\includesvg[keepaspectratio]{/assets/icons/16-layout.svg}}
  \href{https://typst.app/universe/search/?category=layout}{Layout}
\end{itemize}
\end{description}

\subsubsection{Where to report issues?}\label{where-to-report-issues}

This template is a project of Tom Eichlersmith . Report issues on
\href{https://github.com/tomeichlersmith/zen-zine}{their repository} .
You can also try to ask for help with this template on the
\href{https://forum.typst.app}{Forum} .

Please report this template to the Typst team using the
\href{https://typst.app/contact}{contact form} if you believe it is a
safety hazard or infringes upon your rights.

\phantomsection\label{versions}
\subsubsection{Version history}\label{version-history}

\begin{longtable}[]{@{}ll@{}}
\toprule\noalign{}
Version & Release Date \\
\midrule\noalign{}
\endhead
\bottomrule\noalign{}
\endlastfoot
0.1.0 & April 4, 2024 \\
\end{longtable}

Typst GmbH did not create this template and cannot guarantee correct
functionality of this template or compatibility with any version of the
Typst compiler or app.


\title{typst.app/universe/package/paddling-tongji-thesis}

\phantomsection\label{banner}
\phantomsection\label{template-thumbnail}
\pandocbounded{\includegraphics[keepaspectratio]{https://packages.typst.org/preview/thumbnails/paddling-tongji-thesis-0.1.1-small.webp}}

\section{paddling-tongji-thesis}\label{paddling-tongji-thesis}

{ 0.1.1 }

å?ŒæµŽå¤§å­¦æœ¬ç§`ç''Ÿæ¯•ä¸šè®¾è®¡è®ºæ--‡æ¨¡æ?¿ \textbar{} Tongji
University Undergraduate Thesis Template

\href{/app?template=paddling-tongji-thesis&version=0.1.1}{Create project
in app}

\phantomsection\label{readme}
中æ--‡ \textbar{}
\href{https://github.com/typst/packages/raw/main/packages/preview/paddling-tongji-thesis/0.1.1/README-EN.md}{English}

\begin{quote}
{[}!CAUTION{]} ç''±äºŽ Typst
项目ä»?处于密集å?{}`展阶段,ä¸''对æŸ?些功能的æ''¯æŒ?ä¸?完å--„,å›~此本模æ?¿å?¯èƒ½å­˜åœ¨ä¸€äº›é---®é¢˜ã€‚如果您在使ç''¨è¿‡ç¨‹ä¸­é?‡åˆ°äº†é---®é¢˜ï¼Œæ¬¢è¿Žæ??交
issue æˆ-- PR,æˆ`们会尽力解决。

在此期é---´ï¼Œæ¬¢è¿Žå¤§å®¶ä½¿ç''¨
\href{https://github.com/TJ-CSCCG/tongji-undergrad-thesis}{æˆ`们的
LaTeX 模�} 。
\end{quote}

\subsection{æ~·ä¾‹å±•ç¤º}\label{uxe6-uxe4uxbeuxe5uxe7uxba}

以下�次展示
“å°?é?¢â€?ã€?“中æ--‡æ`˜è¦?â€?ã€?“目录â€?ã€?“主è¦?å†\ldots 容â€?ã€?“å?‚考æ--‡çŒ®â€?
与 “谢辞�。

\includegraphics[width=0.3\linewidth,height=\textheight,keepaspectratio]{https://media.githubusercontent.com/media/TJ-CSCCG/TJCS-Images/tongji-undergrad-thesis-typst/preview/main_page-0001.jpg}
\includegraphics[width=0.3\linewidth,height=\textheight,keepaspectratio]{https://media.githubusercontent.com/media/TJ-CSCCG/TJCS-Images/tongji-undergrad-thesis-typst/preview/main_page-0002.jpg}
\includegraphics[width=0.3\linewidth,height=\textheight,keepaspectratio]{https://media.githubusercontent.com/media/TJ-CSCCG/TJCS-Images/tongji-undergrad-thesis-typst/preview/main_page-0004.jpg}
\includegraphics[width=0.3\linewidth,height=\textheight,keepaspectratio]{https://media.githubusercontent.com/media/TJ-CSCCG/TJCS-Images/tongji-undergrad-thesis-typst/preview/main_page-0005.jpg}
\includegraphics[width=0.3\linewidth,height=\textheight,keepaspectratio]{https://media.githubusercontent.com/media/TJ-CSCCG/TJCS-Images/tongji-undergrad-thesis-typst/preview/main_page-0019.jpg}
\includegraphics[width=0.3\linewidth,height=\textheight,keepaspectratio]{https://media.githubusercontent.com/media/TJ-CSCCG/TJCS-Images/tongji-undergrad-thesis-typst/preview/main_page-0020.jpg}

\subsection{使ç''¨æ--¹æ³•}\label{uxe4uxbduxe7uxe6uxb9uxe6uxb3}

\subsubsection{在线 Web App}\label{uxe5ux153uxe7uxba-web-app}

\paragraph{创建项目}\label{uxe5ux2c6uxe5uxbauxe9uxb9uxe7}

\begin{itemize}
\item
  æ‰``å¼€ Typst Universe 中的
  \href{https://www.overleaf.com/latex/templates/tongji-university-undergraduate-thesis-template/tfvdvyggqybn}{\pandocbounded{\includegraphics[keepaspectratio]{https://img.shields.io/badge/Typst-paddling--tongji--thesis-239dae}}}
  并点击 \texttt{\ Create\ project\ in\ app\ } 。
\item
  æˆ--在 \href{https://typst.app/}{Typst Web App} 中选择
  \texttt{\ Start\ from\ a\ template\ } ,然�选择
  \texttt{\ paddling-tongji-thesis\ } 。
\end{itemize}

\paragraph{上ä¼~å­---ä½``}\label{uxe4ux161uxe4uxbc-uxe5uxe4uxbd}

从
\href{https://github.com/TJ-CSCCG/tongji-undergrad-thesis-typst/tree/fonts}{\texttt{\ fonts\ }
分æ''¯}
下载所有å­---ä½``æ--‡ä»¶ï¼Œä¸Šä¼~到该项目的æ~¹ç›®å½•ã€‚å?³å?¯å¼€å§‹ä½¿ç''¨ã€‚

\subsubsection{本地使ç''¨}\label{uxe6ux153uxe5ux153uxe4uxbduxe7}

\paragraph{1. 安è£\ldots{} Typst}\label{uxe5uxe8-typst}

å?‚ç\ldots§
\href{https://github.com/typst/typst?tab=readme-ov-file\#installation}{Typst}
官æ--¹æ--‡æ¡£å®‰è£\ldots{} Typst。

\paragraph{2. 下载å­---ä½``}\label{uxe4uxe8uxbduxbduxe5uxe4uxbd}

从
\href{https://github.com/TJ-CSCCG/tongji-undergrad-thesis-typst/tree/fonts}{\texttt{\ fonts\ }
分æ''¯} 下载所有å­---ä½``æ--‡ä»¶ï¼Œå¹¶
\textbf{安è£\ldots 到系统中} 。

\paragraph{\texorpdfstring{使ç''¨ \texttt{\ typst\ }
åˆ?始åŒ--}{使ç''¨  typst  åˆ?始åŒ--}}\label{uxe4uxbduxe7-typst-uxe5ux2c6uxe5uxe5ux153}

\subparagraph{åˆ?始åŒ--项目}\label{uxe5ux2c6uxe5uxe5ux153uxe9uxb9uxe7}

\begin{Shaded}
\begin{Highlighting}[]
\ExtensionTok{typst}\NormalTok{ init @preview/paddling{-}tongji{-}thesis}
\end{Highlighting}
\end{Shaded}

\subparagraph{ç¼--è¯`}\label{uxe7uxbcuxe8}

\begin{Shaded}
\begin{Highlighting}[]
\ExtensionTok{typst}\NormalTok{ compile main.typ}
\end{Highlighting}
\end{Shaded}

\paragraph{\texorpdfstring{使ç''¨ \texttt{\ git\ clone\ }
åˆ?始åŒ--}{使ç''¨  git clone  åˆ?始åŒ--}}\label{uxe4uxbduxe7-git-clone-uxe5ux2c6uxe5uxe5ux153}

\subparagraph{Git Clone 项目}\label{git-clone-uxe9uxb9uxe7}

\begin{Shaded}
\begin{Highlighting}[]
\FunctionTok{git}\NormalTok{ clone https://github.com/TJ{-}CSCCG/tongji{-}undergrad{-}thesis{-}typst.git}
\BuiltInTok{cd}\NormalTok{ tongji{-}undergrad{-}thesis{-}typst}
\end{Highlighting}
\end{Shaded}

\subparagraph{ç¼--è¯`}\label{uxe7uxbcuxe8-1}

\begin{Shaded}
\begin{Highlighting}[]
\ExtensionTok{typst}\NormalTok{ compile init{-}files/main.typ }\AttributeTok{{-}{-}root}\NormalTok{ .}
\end{Highlighting}
\end{Shaded}

\begin{quote}
{[}!TIP{]}
è‹¥ä½~ä¸?想把项目使ç''¨çš„å­---ä½``安è£\ldots 到系统中,å?¯ä»¥åœ¨ç¼--è¯`æ---¶æŒ‡å®šå­---ä½``路径,例如:

\begin{Shaded}
\begin{Highlighting}[]
\ExtensionTok{typst}\NormalTok{ compile init{-}files/main.typ }\AttributeTok{{-}{-}root}\NormalTok{ . }\AttributeTok{{-}{-}font{-}path}\NormalTok{ \{YOUR\_FONT\_PATH\}}
\end{Highlighting}
\end{Shaded}
\end{quote}

\subsection{如何为该项目贡献代ç~??}\label{uxe5uxe4uxbduxe4uxbauxe8uxe9uxb9uxe7uxe8uxe7ux153uxe4uxe7-uxefuxbcuxff}

还请查看
\href{https://github.com/typst/packages/raw/main/packages/preview/paddling-tongji-thesis/0.1.1/CONTRIBUTING.md/\#how-to-pull-request}{How
to pull request} 。

\subsection{å¼€æº?å??è®®}\label{uxe5uxbcuxe6uxbauxe5uxe8}

该项目使ç''¨
\href{https://github.com/typst/packages/raw/main/packages/preview/paddling-tongji-thesis/0.1.1/LICENSE}{MIT
License} å¼€æº?å??议。

\subsubsection{å\ldots?责声明}\label{uxe5uxe8uxe5uxe6ux17e}

本项目使ç''¨äº†æ--¹æ­£å­---åº``中的å­---ä½``,版æ?ƒå½'æ--¹æ­£å­---åº``所有。本项目ä»\ldots ç''¨äºŽå­¦ä¹~交æµ?,ä¸?å¾---ç''¨äºŽå•†ä¸šç''¨é€''。

\subsection{有å\ldots³çª?出贡献的说明}\label{uxe6ux153uxe5uxb3uxe7uxaauxe5uxbauxe8uxe7ux153uxe7ux161uxe8uxe6ux17e}

\begin{itemize}
\tightlist
\item
  该项目起�于 \href{https://github.com/seashell11234455}{FeO3}
  的�始版本项目
  \href{https://github.com/TJ-CSCCG/tongji-undergrad-thesis-typst/tree/lky}{tongji-undergrad-thesis-typst}
  。
\item
  å?Žæ?¥ \href{https://github.com/RizhongLin}{RizhongLin}
  对模æ?¿è¿›è¡Œäº†å®Œå--„,使å\ldots¶æ›´åŠ~符å?ˆå?ŒæµŽå¤§å­¦æœ¬ç§`ç''Ÿæ¯•ä¸šè®¾è®¡è®ºæ--‡çš„è¦?求,并增åŠ~了é'ˆå¯¹
  Typst 的基础教程。
\end{itemize}

æˆ`们é?žå¸¸æ„Ÿè°¢ä»¥ä¸Šè´¡çŒ®è€\ldots 的付出,ä»--们的工作为更多å?Œå­¦æ??供了æ--¹ä¾¿å'Œå¸®åŠ©ã€‚

在使ç''¨æœ¬æ¨¡æ?¿æ---¶ï¼Œå¦‚果您觉å¾---本项目对您的毕业设计æˆ--论æ--‡æœ‰æ‰€å¸®åŠ©ï¼Œæˆ`们希望您å?¯ä»¥åœ¨æ‚¨çš„致谢部分感谢并致以敬æ„?。

\subsection{致谢}\label{uxe8uxe8}

æˆ`们从顶å°--高æ~¡çš„优秀开æº?项目中学到了很多:

\begin{itemize}
\tightlist
\item
  \href{https://github.com/lucifer1004/pkuthss-typst}{lucifer1004/pkuthss-typst}
\item
  \href{https://github.com/werifu/HUST-typst-template}{werifu/HUST-typst-template}
\end{itemize}

\subsection{è?''ç³»æ--¹å¼?}\label{uxe8uxe7uxb3uxe6uxb9uxe5uxbc}

\begin{Shaded}
\begin{Highlighting}[]
\CommentTok{\# Python}
\NormalTok{[}
    \StringTok{\textquotesingle{}rizhonglin@$.\%\textquotesingle{}}\NormalTok{.replace(}\StringTok{\textquotesingle{}$\textquotesingle{}}\NormalTok{, }\StringTok{\textquotesingle{}epfl\textquotesingle{}}\NormalTok{).replace(}\StringTok{\textquotesingle{}\%\textquotesingle{}}\NormalTok{, }\StringTok{\textquotesingle{}ch\textquotesingle{}}\NormalTok{),}
\NormalTok{]}
\end{Highlighting}
\end{Shaded}

\subsubsection{QQ 群}\label{qq-uxe7uxbe}

\begin{itemize}
\tightlist
\item
  TJ-CSCCG 交�群: \texttt{\ 1013806782\ }
\end{itemize}

\href{/app?template=paddling-tongji-thesis&version=0.1.1}{Create project
in app}

\subsubsection{How to use}\label{how-to-use}

Click the button above to create a new project using this template in
the Typst app.

You can also use the Typst CLI to start a new project on your computer
using this command:

\begin{verbatim}
typst init @preview/paddling-tongji-thesis:0.1.1
\end{verbatim}

\includesvg[width=0.16667in,height=0.16667in]{/assets/icons/16-copy.svg}

\subsubsection{About}\label{about}

\begin{description}
\tightlist
\item[Author :]
\href{https://github.com/TJ-CSCCG}{TJ-CSCCG}
\item[License:]
MIT
\item[Current version:]
0.1.1
\item[Last updated:]
July 1, 2024
\item[First released:]
July 1, 2024
\item[Archive size:]
32.7 kB
\href{https://packages.typst.org/preview/paddling-tongji-thesis-0.1.1.tar.gz}{\pandocbounded{\includesvg[keepaspectratio]{/assets/icons/16-download.svg}}}
\item[Repository:]
\href{https://github.com/TJ-CSCCG/tongji-undergrad-thesis-typst.git}{GitHub}
\item[Categor y :]
\begin{itemize}
\tightlist
\item[]
\item
  \pandocbounded{\includesvg[keepaspectratio]{/assets/icons/16-mortarboard.svg}}
  \href{https://typst.app/universe/search/?category=thesis}{Thesis}
\end{itemize}
\end{description}

\subsubsection{Where to report issues?}\label{where-to-report-issues}

This template is a project of TJ-CSCCG . Report issues on
\href{https://github.com/TJ-CSCCG/tongji-undergrad-thesis-typst.git}{their
repository} . You can also try to ask for help with this template on the
\href{https://forum.typst.app}{Forum} .

Please report this template to the Typst team using the
\href{https://typst.app/contact}{contact form} if you believe it is a
safety hazard or infringes upon your rights.

\phantomsection\label{versions}
\subsubsection{Version history}\label{version-history}

\begin{longtable}[]{@{}ll@{}}
\toprule\noalign{}
Version & Release Date \\
\midrule\noalign{}
\endhead
\bottomrule\noalign{}
\endlastfoot
0.1.1 & July 1, 2024 \\
\end{longtable}

Typst GmbH did not create this template and cannot guarantee correct
functionality of this template or compatibility with any version of the
Typst compiler or app.


\title{typst.app/universe/package/aio-studi-and-thesis}

\phantomsection\label{banner}
\phantomsection\label{template-thumbnail}
\pandocbounded{\includegraphics[keepaspectratio]{https://packages.typst.org/preview/thumbnails/aio-studi-and-thesis-0.1.0-small.webp}}

\section{aio-studi-and-thesis}\label{aio-studi-and-thesis}

{ 0.1.0 }

All-in-one template for students and theses

\href{/app?template=aio-studi-and-thesis&version=0.1.0}{Create project
in app}

\phantomsection\label{readme}
\href{https://github.com/fuchs-fabian/typst-template-aio-studi-and-thesis/blob/main/docs/manual-de.pdf}{\pandocbounded{\includegraphics[keepaspectratio]{https://img.shields.io/website?down_message=offline&label=manual\%20de&up_color=007aff&up_message=online&url=https\%3A\%2F\%2Fgithub.com\%2Ffuchs-fabian\%2Ftypst-template-aio-studi-and-thesis\%2Fblob\%2Fmain\%2Fdocs\%2Fmanual-de.pdf}}}
\href{https://github.com/fuchs-fabian/typst-template-aio-studi-and-thesis/blob/main/docs/manual-en.pdf}{\pandocbounded{\includegraphics[keepaspectratio]{https://img.shields.io/website?down_message=offline&label=manual\%20en&up_color=007aff&up_message=online&url=https\%3A\%2F\%2Fgithub.com\%2Ffuchs-fabian\%2Ftypst-template-aio-studi-and-thesis\%2Fblob\%2Fmain\%2Fdocs\%2Fmanual-en.pdf}}}
\href{https://github.com/fuchs-fabian/typst-template-aio-studi-and-thesis/blob/main/docs/example-de-thesis.pdf}{\pandocbounded{\includegraphics[keepaspectratio]{https://img.shields.io/website?down_message=offline&label=example\%20de&up_color=007aff&up_message=online&url=https\%3A\%2F\%2Fgithub.com\%2Ffuchs-fabian\%2Ftypst-template-aio-studi-and-thesis\%2Fblob\%2Fmain\%2Fdocs\%2Fexample-de-thesis.pdf}}}
\href{https://github.com/fuchs-fabian/typst-template-aio-studi-and-thesis/blob/main/docs/example-en-thesis.pdf}{\pandocbounded{\includegraphics[keepaspectratio]{https://img.shields.io/website?down_message=offline&label=example\%20en&up_color=007aff&up_message=online&url=https\%3A\%2F\%2Fgithub.com\%2Ffuchs-fabian\%2Ftypst-template-aio-studi-and-thesis\%2Fblob\%2Fmain\%2Fdocs\%2Fexample-en-thesis.pdf}}}
\href{https://github.com/fuchs-fabian/typst-template-aio-studi-and-thesis/blob/main/LICENSE}{\pandocbounded{\includegraphics[keepaspectratio]{https://img.shields.io/badge/license-MIT-brightgreen}}}

This template can be used for extensive documentation as well as for
final theses such as bachelor theses.

It is characterised by the fact that it is highly customisable despite
the predefined design.

Initially, all template parameters are optional by default. It is then
suitable for documentation. To make it suitable for theses, only one
parameter needs to be changed.

\subsection{\texorpdfstring{âš~ï¸? \textbf{Disclaimer -
Important!}}{âš~ï¸? Disclaimer - Important!}}\label{uxe2ux161-uxef-disclaimer---important}

\begin{itemize}
\tightlist
\item
  It is a template and does not have to meet the exact requirements of
  your university
\item
  It is only supported in German and English (Default setting: German)
\end{itemize}

\subsection{Getting Started}\label{getting-started}

You can use this template in the Typst web app by clicking “Start from
template� on the dashboard and searching for
\texttt{\ aio-studi-and-thesis\ } .

Alternatively, you can use the CLI to kick this project off using the
command

\begin{Shaded}
\begin{Highlighting}[]
\ExtensionTok{typst}\NormalTok{ init @preview/aio{-}studi{-}and{-}thesis}
\end{Highlighting}
\end{Shaded}

Typst will create a new directory with all the files needed to get you
started.

\subsection{Usage}\label{usage}

The template (
\href{https://github.com/typst/packages/raw/main/packages/preview/aio-studi-and-thesis/0.1.0/docs/example-de-thesis.pdf}{rendered
PDF (DE)} ) contains thesis writing advice (in German) as example
content.

If you are looking for the details of this template package’s
function, take a look at the
\href{https://github.com/typst/packages/raw/main/packages/preview/aio-studi-and-thesis/0.1.0/docs/manual-de.pdf}{german
manual} or the
\href{https://github.com/typst/packages/raw/main/packages/preview/aio-studi-and-thesis/0.1.0/docs/manual-en.pdf}{english
manual} .

\begin{quote}
Roboto is used as the default font. Please note accordingly if you want
to use exactly this font.
\end{quote}

\subsection{Example configuration}\label{example-configuration}

\begin{Shaded}
\begin{Highlighting}[]
\NormalTok{\#import "@preview/aio{-}studi{-}and{-}thesis:0.1.0": *}

\NormalTok{\#show: project.with(}
\NormalTok{  lang: "de",}
\NormalTok{  authors: (}
\NormalTok{    (name: "Firstname Lastname"),}
\NormalTok{  ),}
\NormalTok{  title: "Title",}
\NormalTok{  subtitle: "Subtitle",}
\NormalTok{  cover{-}sheet: (}
\NormalTok{    cover{-}image: none,}
\NormalTok{    description: []}
\NormalTok{  )}
\NormalTok{)}
\end{Highlighting}
\end{Shaded}

\subsection{\texorpdfstring{Donate with
\href{https://www.paypal.com/donate/?hosted_button_id=4G9X8TDNYYNKG}{PayPal}}{Donate with PayPal}}\label{donate-with-paypal}

If you think this template is useful and saves you a lot of work and
nerves (Word and LaTex can be very tiring) and lets you sleep better,
then a small donation would be very nice.

\href{https://www.paypal.com/donate/?hosted_button_id=4G9X8TDNYYNKG}{\pandocbounded{\includegraphics[keepaspectratio]{https://www.paypalobjects.com/de_DE/i/btn/btn_donateCC_LG.gif}}}

\href{/app?template=aio-studi-and-thesis&version=0.1.0}{Create project
in app}

\subsubsection{How to use}\label{how-to-use}

Click the button above to create a new project using this template in
the Typst app.

You can also use the Typst CLI to start a new project on your computer
using this command:

\begin{verbatim}
typst init @preview/aio-studi-and-thesis:0.1.0
\end{verbatim}

\includesvg[width=0.16667in,height=0.16667in]{/assets/icons/16-copy.svg}

\subsubsection{About}\label{about}

\begin{description}
\tightlist
\item[Author :]
\href{https://github.com/fuchs-fabian/}{Fabian Fuchs}
\item[License:]
MIT
\item[Current version:]
0.1.0
\item[Last updated:]
July 31, 2024
\item[First released:]
July 31, 2024
\item[Minimum Typst version:]
0.11.1
\item[Archive size:]
352 kB
\href{https://packages.typst.org/preview/aio-studi-and-thesis-0.1.0.tar.gz}{\pandocbounded{\includesvg[keepaspectratio]{/assets/icons/16-download.svg}}}
\item[Repository:]
\href{https://github.com/fuchs-fabian/typst-template-aio-studi-and-thesis}{GitHub}
\item[Discipline :]
\begin{itemize}
\tightlist
\item[]
\item
  \href{https://typst.app/universe/search/?discipline=computer-science}{Computer
  Science}
\end{itemize}
\item[Categor y :]
\begin{itemize}
\tightlist
\item[]
\item
  \pandocbounded{\includesvg[keepaspectratio]{/assets/icons/16-mortarboard.svg}}
  \href{https://typst.app/universe/search/?category=thesis}{Thesis}
\end{itemize}
\end{description}

\subsubsection{Where to report issues?}\label{where-to-report-issues}

This template is a project of Fabian Fuchs . Report issues on
\href{https://github.com/fuchs-fabian/typst-template-aio-studi-and-thesis}{their
repository} . You can also try to ask for help with this template on the
\href{https://forum.typst.app}{Forum} .

Please report this template to the Typst team using the
\href{https://typst.app/contact}{contact form} if you believe it is a
safety hazard or infringes upon your rights.

\phantomsection\label{versions}
\subsubsection{Version history}\label{version-history}

\begin{longtable}[]{@{}ll@{}}
\toprule\noalign{}
Version & Release Date \\
\midrule\noalign{}
\endhead
\bottomrule\noalign{}
\endlastfoot
0.1.0 & July 31, 2024 \\
\end{longtable}

Typst GmbH did not create this template and cannot guarantee correct
functionality of this template or compatibility with any version of the
Typst compiler or app.


\title{typst.app/universe/package/qcm}

\phantomsection\label{banner}
\section{qcm}\label{qcm}

{ 0.1.0 }

Qualitative Colormaps

\phantomsection\label{readme}
Qualitative Colormaps for Typst

Qualitative colormaps contain a fixed number of distinct and easily
differentiable colors. They are suitable to use for e.g. categorical
data visualization.

\subsection{Source}\label{source}

The following colormaps are available:

\begin{itemize}
\tightlist
\item
  all \href{https://github.com/axismaps/colorbrewer/}{colorbrew}
  qualitive colormaps, for discovery and as documentation visit
  \href{https://colorbrewer2.org/}{colorbrewer2.org}
\end{itemize}

\subsection{Usage}\label{usage}

Usage is very simple:

\begin{Shaded}
\begin{Highlighting}[]
\NormalTok{\#import "@preview/qcm:0.1.0": colormap}

\NormalTok{\#colormap("Set1", 5)}
\end{Highlighting}
\end{Shaded}

\subsubsection{How to add}\label{how-to-add}

Copy this into your project and use the import as \texttt{\ qcm\ }

\begin{verbatim}
#import "@preview/qcm:0.1.0"
\end{verbatim}

\includesvg[width=0.16667in,height=0.16667in]{/assets/icons/16-copy.svg}

Check the docs for
\href{https://typst.app/docs/reference/scripting/\#packages}{more
information on how to import packages} .

\subsubsection{About}\label{about}

\begin{description}
\tightlist
\item[Author :]
Ludwig Austermann
\item[License:]
MIT
\item[Current version:]
0.1.0
\item[Last updated:]
April 12, 2024
\item[First released:]
April 12, 2024
\item[Minimum Typst version:]
0.10.0
\item[Archive size:]
2.37 kB
\href{https://packages.typst.org/preview/qcm-0.1.0.tar.gz}{\pandocbounded{\includesvg[keepaspectratio]{/assets/icons/16-download.svg}}}
\item[Repository:]
\href{https://github.com/ludwig-austermann/qcm}{GitHub}
\end{description}

\subsubsection{Where to report issues?}\label{where-to-report-issues}

This package is a project of Ludwig Austermann . Report issues on
\href{https://github.com/ludwig-austermann/qcm}{their repository} . You
can also try to ask for help with this package on the
\href{https://forum.typst.app}{Forum} .

Please report this package to the Typst team using the
\href{https://typst.app/contact}{contact form} if you believe it is a
safety hazard or infringes upon your rights.

\phantomsection\label{versions}
\subsubsection{Version history}\label{version-history}

\begin{longtable}[]{@{}ll@{}}
\toprule\noalign{}
Version & Release Date \\
\midrule\noalign{}
\endhead
\bottomrule\noalign{}
\endlastfoot
0.1.0 & April 12, 2024 \\
\end{longtable}

Typst GmbH did not create this package and cannot guarantee correct
functionality of this package or compatibility with any version of the
Typst compiler or app.


\title{typst.app/universe/package/leipzig-glossing}

\phantomsection\label{banner}
\section{leipzig-glossing}\label{leipzig-glossing}

{ 0.4.0 }

Linguistic interlinear glosses according to the Leipzig Glossing rules

\phantomsection\label{readme}
\texttt{\ leipzig-glossing\ } is a
\href{https://github.com/typst/typst}{Typst} library for creating
interlinear morpheme-by-morpheme glosses according to the
\href{https://www.eva.mpg.de/lingua/pdf/Glossing-Rules.pdf}{Leipzig
glossing rules} .

Run \texttt{\ typst\ compile\ documentation.typ\ } in the root of the
repository to generate a pdf file with examples and documentation. This
command is also codified in the accompanying
\href{https://github.com/casey/just}{justfile} as
\texttt{\ just\ build-doc\ } .

The definitions intended for use by end users are the \texttt{\ gloss\ }
and \texttt{\ numbered-gloss\ } functions, and the
\texttt{\ abbreviations\ } submodule.

\subsection{Repositories}\label{repositories}

The canonical repository for this project is on the
\href{https://code.everydayimshuflin.com/greg/typst-lepizig-glossing}{Gitea
instance} .

There is also a
\href{https://github.com/neunenak/typst-leipzig-glossing/}{Github
mirror} , and a \href{https://radicle.xyz/}{Radicle} mirror available at
{rad://z2j7vQLS3EtQbPkrzi7Tn2XR7YWLw} .

Bug reports and code contributions are welcome from all users.

\subsection{License}\label{license}

This library uses the MIT license; see \texttt{\ LICENSE.txt\ } .

\subsection{Contributors}\label{contributors}

Thanks to \href{https://github.com/betoma}{Bethany E. Toma} for a number
of suggestions and improvements.

Thanks to \href{https://github.com/rwmpelstilzchen}{Maja
Abramski-Kronenberg} for the labeling functionality.

\subsubsection{How to add}\label{how-to-add}

Copy this into your project and use the import as
\texttt{\ leipzig-glossing\ }

\begin{verbatim}
#import "@preview/leipzig-glossing:0.4.0"
\end{verbatim}

\includesvg[width=0.16667in,height=0.16667in]{/assets/icons/16-copy.svg}

Check the docs for
\href{https://typst.app/docs/reference/scripting/\#packages}{more
information on how to import packages} .

\subsubsection{About}\label{about}

\begin{description}
\tightlist
\item[Author s :]
\href{mailto:greg@everydayimshuflin.com}{Greg Shuflin} \& Other
open-source contributors
\item[License:]
MIT
\item[Current version:]
0.4.0
\item[Last updated:]
November 12, 2024
\item[First released:]
July 6, 2023
\item[Minimum Typst version:]
0.12.0
\item[Archive size:]
5.26 kB
\href{https://packages.typst.org/preview/leipzig-glossing-0.4.0.tar.gz}{\pandocbounded{\includesvg[keepaspectratio]{/assets/icons/16-download.svg}}}
\item[Repository:]
\href{https://code.everydayimshuflin.com/greg/typst-lepizig-glossing}{code.everydayimshuflin.com}
\item[Discipline :]
\begin{itemize}
\tightlist
\item[]
\item
  \href{https://typst.app/universe/search/?discipline=linguistics}{Linguistics}
\end{itemize}
\item[Categor y :]
\begin{itemize}
\tightlist
\item[]
\item
  \pandocbounded{\includesvg[keepaspectratio]{/assets/icons/16-atom.svg}}
  \href{https://typst.app/universe/search/?category=paper}{Paper}
\end{itemize}
\end{description}

\subsubsection{Where to report issues?}\label{where-to-report-issues}

This package is a project of Greg Shuflin and Other open-source
contributors . Report issues on
\href{https://code.everydayimshuflin.com/greg/typst-lepizig-glossing}{their
repository} . You can also try to ask for help with this package on the
\href{https://forum.typst.app}{Forum} .

Please report this package to the Typst team using the
\href{https://typst.app/contact}{contact form} if you believe it is a
safety hazard or infringes upon your rights.

\phantomsection\label{versions}
\subsubsection{Version history}\label{version-history}

\begin{longtable}[]{@{}ll@{}}
\toprule\noalign{}
Version & Release Date \\
\midrule\noalign{}
\endhead
\bottomrule\noalign{}
\endlastfoot
0.4.0 & November 12, 2024 \\
\href{https://typst.app/universe/package/leipzig-glossing/0.3.0/}{0.3.0}
& August 21, 2024 \\
\href{https://typst.app/universe/package/leipzig-glossing/0.2.0/}{0.2.0}
& October 4, 2023 \\
\href{https://typst.app/universe/package/leipzig-glossing/0.1.0/}{0.1.0}
& July 6, 2023 \\
\end{longtable}

Typst GmbH did not create this package and cannot guarantee correct
functionality of this package or compatibility with any version of the
Typst compiler or app.


\title{typst.app/universe/package/blinky}

\phantomsection\label{banner}
\section{blinky}\label{blinky}

{ 0.1.0 }

Typesets paper titles in bibliographies as hyperlinks.

\phantomsection\label{readme}
This package permits the creation of Typst bibliographies in which paper
titles are typeset as hyperlinks. Here’s an example (with links
typeset in blue):

\includegraphics[width=0.8\linewidth,height=\textheight,keepaspectratio]{https://raw.githubusercontent.com/alexanderkoller/typst-blinky/main/examples/screenshot.png}

The bibliography is generated from a Bibtex file, and citations are done
with the usual Typst mechanisms. The hyperlinks are specified through
DOI or URL fields in the Bibtex entries; if such a field is present, the
title of the entry will be automatically typeset as a hyperlink.

See
\href{https://github.com/alexanderkoller/typst-blinky/tree/main/examples}{here}
for a full example.

\subsection{Usage}\label{usage}

Adding hyperlinks to your bibliography is a two-step process: (a) use a
CSL style with magic symbols (explained below), and (b) enclose the
\texttt{\ bibliography\ } command with the \texttt{\ link-bib-urls\ }
function:

\begin{verbatim}
#import "@preview/blinky:0.1.0": link-bib-urls

... @cite something ... @cite more ...

#let bibsrc = read("custom.bib")
#link-bib-urls(bibsrc)[
  #bibliography("custom.bib", style: "./association-for-computational-linguistics-blinky.csl")
]
\end{verbatim}

Observe that the Bibtex file \texttt{\ custom.bib\ } is loaded twice:
once to load into \texttt{\ link-bib-urls\ } and once in the standard
Typst \texttt{\ bibliography\ } command. Obviously, this needs to be the
same file twice. See under “Alternative solutions� below why this
can’t be simplified further at the moment.

If a Bibtex entry contains a DOI field, the title will become a
hyperlink to the DOI. Otherwise, if the Bibtex entry contains a URL
field, the title will become a hyperlink to this URL. Otherwise, the
title will be shown as normal, without a link.

\subsection{CSL with magic symbols}\label{csl-with-magic-symbols}

Blinky generates the hyperlinked titles through a regex show rule that
replaces a “magic symbol� with a
\href{https://typst.app/docs/reference/model/link/}{link} command. This
“magic symbol� is a string of the form
\texttt{\ !!BIBENTRY!\textless{}key\textgreater{}!!\ } , where
\texttt{\ \textless{}key\textgreater{}\ } is the Bibtex citation key of
the reference.

You will therefore need to tweak your CSL style to use it with Blinky.
Specifically, in every place where you would usually have the paper
title, i.e.

\begin{verbatim}
\end{verbatim}

or similar, your CSL file now instead needs to print a decorated version
of the paper’s citation-key (= Bibtex key):

\begin{verbatim}
\end{verbatim}

You can have more prefix before and suffix after the
\texttt{\ !!BIBENTRY!\ } and \texttt{\ !!\ } , as in the example, but
these magic symbols need to be there so Blinky can find the places in
the document where the hyperlinked title needs to be inserted.

You can check the
\href{https://github.com/alexanderkoller/typst-blinky/blob/main/examples/association-for-computational-linguistics-blinky.csl}{example
CSL file} to see what this looks like in practice; compare to
\href{https://github.com/citation-style-language/styles/blob/master/association-for-computational-linguistics.csl}{the
unmodified original} .

\subsection{Alternative solutions}\label{alternative-solutions}

The current mechanism in Blinky is somewhat heavy-handed: a Typst plugin
uses the \href{https://github.com/typst/biblatex}{biblatex} crate to
parse the Bibtex file (independently of the normal operations of the
\texttt{\ bibliography\ } command), and then all occurrences of the
magic symbol in the Typst bibliography are replaced by the hyperlinked
titles.

It would be great to replace this mechanism by something simpler, but it
is actually remarkably tricky to make bibliography titles hyperlinks
with the current version of Typst (0.11.1). All the alternatives that I
could think of don’t work. Here are some of them:

\begin{itemize}
\tightlist
\item
  Print the URL/DOI using the CSL style, and then use a regex show rule
  to convert it into a \texttt{\ link\ } around the title somehow. This
  does not work because most URLs contain a colon character (:), and
  these \href{https://github.com/typst/typst/issues/86}{cause trouble
  with Typst regexes} .
\item
  Make the CSL style output text of the form
  \texttt{\ \#link(url){[}title{]}\ } . This does not work because the
  content generated by CSL is not evaluated further by Typst. Also,
  Typst \href{https://github.com/typst/typst/issues/942}{does not
  support show rules for the individual bibliography items} , which
  makes it tricky to call
  \href{https://typst.app/docs/reference/foundations/eval/}{eval} on
  them.
\item
  Create a show rule for \texttt{\ link\ } . Some CSL styles already
  generate \texttt{\ link\ } elements if a URL/DOI is present in the bib
  entry - one could consider replacing it with a \texttt{\ link\ } whose
  URL is the same as before, but the text is a link symbol or some such.
  However, a show rule for a link that generates another link runs into
  an infinite recursion; Typst made
  \href{https://github.com/typst/typst/pull/3327}{the deliberate
  decision} to handle such recursions only for \texttt{\ text\ } show
  rules.
\item
  The best solution would be to simply use an unmodified CSL file, but
  it is not clear to me how one would pick out the paper title from the
  bibliography in a general way. I’m afraid that any solution that
  hyperlinks titles will require modifications to the CSL style.
\end{itemize}

It would furthermore be desirable to hide the fact that we are reading
the same Bibtex file twice behind a single function call. However, code
in a Typst package
\href{https://github.com/typst/typst/issues/2126}{resolves all filenames
relative to the package directory} , which means that the package cannot
access a bibliography file outside of the package directory. We may be
able to simplify this once
\href{https://github.com/typst/typst/issues/971}{\#971} gets addressed.

\subsubsection{How to add}\label{how-to-add}

Copy this into your project and use the import as \texttt{\ blinky\ }

\begin{verbatim}
#import "@preview/blinky:0.1.0"
\end{verbatim}

\includesvg[width=0.16667in,height=0.16667in]{/assets/icons/16-copy.svg}

Check the docs for
\href{https://typst.app/docs/reference/scripting/\#packages}{more
information on how to import packages} .

\subsubsection{About}\label{about}

\begin{description}
\tightlist
\item[Author :]
\href{mailto:akoller@gmail.com}{Alexander Koller}
\item[License:]
MIT
\item[Current version:]
0.1.0
\item[Last updated:]
August 7, 2024
\item[First released:]
August 7, 2024
\item[Archive size:]
75.1 kB
\href{https://packages.typst.org/preview/blinky-0.1.0.tar.gz}{\pandocbounded{\includesvg[keepaspectratio]{/assets/icons/16-download.svg}}}
\item[Repository:]
\href{https://github.com/alexanderkoller/typst-blinky}{GitHub}
\end{description}

\subsubsection{Where to report issues?}\label{where-to-report-issues}

This package is a project of Alexander Koller . Report issues on
\href{https://github.com/alexanderkoller/typst-blinky}{their repository}
. You can also try to ask for help with this package on the
\href{https://forum.typst.app}{Forum} .

Please report this package to the Typst team using the
\href{https://typst.app/contact}{contact form} if you believe it is a
safety hazard or infringes upon your rights.

\phantomsection\label{versions}
\subsubsection{Version history}\label{version-history}

\begin{longtable}[]{@{}ll@{}}
\toprule\noalign{}
Version & Release Date \\
\midrule\noalign{}
\endhead
\bottomrule\noalign{}
\endlastfoot
0.1.0 & August 7, 2024 \\
\end{longtable}

Typst GmbH did not create this package and cannot guarantee correct
functionality of this package or compatibility with any version of the
Typst compiler or app.


\title{typst.app/universe/package/nifty-ntnu-thesis}

\phantomsection\label{banner}
\phantomsection\label{template-thumbnail}
\pandocbounded{\includegraphics[keepaspectratio]{https://packages.typst.org/preview/thumbnails/nifty-ntnu-thesis-0.1.1-small.webp}}

\section{nifty-ntnu-thesis}\label{nifty-ntnu-thesis}

{ 0.1.1 }

An NTNU thesis template

\href{/app?template=nifty-ntnu-thesis&version=0.1.1}{Create project in
app}

\phantomsection\label{readme}
Port of \href{https://github.com/COPCSE-NTNU/thesis-NTNU}{thesis-NTNU}
template to Typst.
\href{https://github.com/saimnaveediqbal/thesis-NTNU-typst/blob/main/template/main.typ}{main.pdf}
contains a full usage example, see
\href{https://github.com/saimnaveediqbal/thesis-NTNU-typst/blob/main/example.pdf}{example.pdf}
for a rendered pdf.

To use this template you need to import it at the beginning of your
document:

\begin{Shaded}
\begin{Highlighting}[]
\NormalTok{\#import "@preview/nifty{-}ntnu{-}thesis:0.1.0": *}
\end{Highlighting}
\end{Shaded}

The template has many arguments you can specify:

\begin{longtable}[]{@{}llll@{}}
\toprule\noalign{}
Argument & Default Value & Type & Description \\
\midrule\noalign{}
\endhead
\bottomrule\noalign{}
\endlastfoot
\texttt{\ title\ } & \texttt{\ Title\ } & {[}content{]} & The title of
the thesis. \\
\texttt{\ short-title\ } & \texttt{\ Title\ } & {[}content{]} & Short
form of the title. If specified, will show up in the header \\
\texttt{\ author\ } & \texttt{\ Author\ } & {[}array{]} & An array of
authors \\
\texttt{\ short-author\ } & `` & {[}string{]} & Short form version of
the authors. If specified, will show up in header \\
\texttt{\ font\ } & \texttt{\ Charter\ } & {[}string{]} & Main font of
template \\
\texttt{\ raw-font\ } & \texttt{\ DejaVu\ Sans\ Mono\ } & {[}string{]} &
Font used for code listings \\
\texttt{\ paper-size\ } & \texttt{\ a4\ } & {[}string{]} & Specify a
{[}paper size string{]} to change the page size. \\
\texttt{\ date\ } & \texttt{\ datetime.today()\ } & {[}datetime{]} & The
date that will be displayed on the cover page. \\
\texttt{\ date-format\ } &
\texttt{\ {[}day\ padding:zero{]}/{[}month\ repr:numerical{]}/{[}year\ repr:full{]}\ }
& {[}string{]} & The format for the date that will be displayed on the
cover page. By default, the date will be displayed as
\texttt{\ DD/MM/YYYY\ } . \\
\texttt{\ abstract-en\ } & \texttt{\ none\ } & {[}content{]} & English
abstract shown before main content. \\
\texttt{\ abstract-no\ } & \texttt{\ none\ } & {[}content{]} & Norwegian
abstract shown before main content. \\
\texttt{\ preface\ } & \texttt{\ none\ } & {[}content{]} & The preface
for your work. The preface content is shown on its own separate page
after the abstracts. \\
\texttt{\ table-of-contents\ } & \texttt{\ outline()\ } & {[}content{]}
& The table of contents. Setting this to \texttt{\ none\ } will disable
the table of contents. \\
\texttt{\ titlepage\ } & \texttt{\ false\ } & {[}bool{]} & Whether to
display the titlepage or not. \\
\texttt{\ bibliography\ } & \texttt{\ none\ } & {[}content{]} & The
bibliography function or none. Specifying this will configure numeric,
IEEE-style citations. \\
\texttt{\ chapter-pagebreak\ } & \texttt{\ true\ } & {[}bool{]} &
Setting this to \texttt{\ false\ } will prevent chapters from starting
on a new page. \\
\texttt{\ chapters-on-odd\ } & \texttt{\ false\ } & {[}bool{]} & Setting
this to \texttt{\ false\ } will prevent chapters from only starting on
an odd page. \\
\texttt{\ figure-index\ } &
\texttt{\ (enabled:\ true,\ title:\ "Figures")\ } & {[}dictionary{]} &
Setting this to \texttt{\ true\ } will display a index of image figures
at the end of the document. \\
\texttt{\ table-index\ } &
\texttt{\ (enabled:\ true,\ title:\ "Tables")\ } & {[}dictionary{]} &
Setting this to \texttt{\ true\ } will display a index of table figures
at the end of the document. \\
\texttt{\ listing-index\ } &
\texttt{\ (enabled:\ true,\ title:\ "Listings")\ } & {[}dictionary{]} &
Setting this to \texttt{\ true\ } will display a index of listing (code
block) figures at the end of the document. \\
\end{longtable}

\begin{itemize}
\tightlist
\item
  {[} {]} Styling for:

  \begin{itemize}
  \tightlist
  \item
    {[}x{]} Code blocks
  \item
    {[}x{]} Tables
  \item
    {[}x{]} Header and footer
  \item
    {[}x{]} Lists
  \end{itemize}
\item
  {[}x{]} Subfigures
\item
  {[}x{]} Abstract
\item
  {[}x{]} Preface
\item
  {[}x{]} Code block captions
\item
  {[}x{]} Bibliography
\item
  {[} {]} Norwegian language support
\item
  {[}x{]} Proper figure numbering
\item
  {[}x{]} Short title in header
\item
  {[}x{]} Multiple authors
\item
  {[}x{]} Start chapters on only odd pages
\end{itemize}

Thanks to:

\begin{itemize}
\tightlist
\item
  The creator of the
  \href{https://github.com/talal/ilm/blob/main/lib.typ}{ILM template}
  which I used as the basis for this.
\item
  The creators of the original
  \href{https://github.com/COPCSE-NTNU/thesis-NTNU}{NTNU thesis
  template}
\item
  The creators of the
  \href{https://github.com/maucejo/elsearticle}{elsearticle template}
  for their implementation of subfigures and appendix environment
\end{itemize}

\href{/app?template=nifty-ntnu-thesis&version=0.1.1}{Create project in
app}

\subsubsection{How to use}\label{how-to-use}

Click the button above to create a new project using this template in
the Typst app.

You can also use the Typst CLI to start a new project on your computer
using this command:

\begin{verbatim}
typst init @preview/nifty-ntnu-thesis:0.1.1
\end{verbatim}

\includesvg[width=0.16667in,height=0.16667in]{/assets/icons/16-copy.svg}

\subsubsection{About}\label{about}

\begin{description}
\tightlist
\item[Author :]
Saim Iqbal
\item[License:]
MIT
\item[Current version:]
0.1.1
\item[Last updated:]
November 6, 2024
\item[First released:]
August 29, 2024
\item[Archive size:]
856 kB
\href{https://packages.typst.org/preview/nifty-ntnu-thesis-0.1.1.tar.gz}{\pandocbounded{\includesvg[keepaspectratio]{/assets/icons/16-download.svg}}}
\item[Repository:]
\href{https://github.com/saimnaveediqbal/thesis-NTNU-typst}{GitHub}
\item[Categor y :]
\begin{itemize}
\tightlist
\item[]
\item
  \pandocbounded{\includesvg[keepaspectratio]{/assets/icons/16-mortarboard.svg}}
  \href{https://typst.app/universe/search/?category=thesis}{Thesis}
\end{itemize}
\end{description}

\subsubsection{Where to report issues?}\label{where-to-report-issues}

This template is a project of Saim Iqbal . Report issues on
\href{https://github.com/saimnaveediqbal/thesis-NTNU-typst}{their
repository} . You can also try to ask for help with this template on the
\href{https://forum.typst.app}{Forum} .

Please report this template to the Typst team using the
\href{https://typst.app/contact}{contact form} if you believe it is a
safety hazard or infringes upon your rights.

\phantomsection\label{versions}
\subsubsection{Version history}\label{version-history}

\begin{longtable}[]{@{}ll@{}}
\toprule\noalign{}
Version & Release Date \\
\midrule\noalign{}
\endhead
\bottomrule\noalign{}
\endlastfoot
0.1.1 & November 6, 2024 \\
\href{https://typst.app/universe/package/nifty-ntnu-thesis/0.1.0/}{0.1.0}
& August 29, 2024 \\
\end{longtable}

Typst GmbH did not create this template and cannot guarantee correct
functionality of this template or compatibility with any version of the
Typst compiler or app.


\title{typst.app/universe/package/badgery}

\phantomsection\label{banner}
\section{badgery}\label{badgery}

{ 0.1.1 }

Adds styled badges, boxes and menu actions.

\phantomsection\label{readme}
This package defines some colourful badges and boxes around text that
represent user interface actions such as a click or following a menu.

For examples have a look at the example
\href{https://github.com/typst/packages/raw/main/packages/preview/badgery/0.1.1/example/main.typ}{main.typ}
,
\href{https://github.com/typst/packages/raw/main/packages/preview/badgery/0.1.1/exmaple/main.pdf}{main.pdf}
.

\pandocbounded{\includegraphics[keepaspectratio]{https://github.com/typst/packages/raw/main/packages/preview/badgery/0.1.1/demo.png}}

\subsection{Badges}\label{badges}

\begin{Shaded}
\begin{Highlighting}[]
\NormalTok{\#badge{-}gray("Gray badge"),}
\NormalTok{\#badge{-}red("Red badge"),}
\NormalTok{\#badge{-}yellow("Yellow badge"),}
\NormalTok{\#badge{-}green("Green badge"),}
\NormalTok{\#badge{-}blue("Blue badge"),}
\NormalTok{\#badge{-}purple("Purple badge")}
\end{Highlighting}
\end{Shaded}

\subsection{User interface actions}\label{user-interface-actions}

This is a user interface action (ie. a click):

\begin{Shaded}
\begin{Highlighting}[]
\NormalTok{\#ui{-}action("Click X")}
\end{Highlighting}
\end{Shaded}

This is an action to follow a user interface menu (2 steps):

\begin{Shaded}
\begin{Highlighting}[]
\NormalTok{\#menu(("File", "New File..."))}
\end{Highlighting}
\end{Shaded}

This is a menu action with multiple steps:

\begin{Shaded}
\begin{Highlighting}[]
\NormalTok{\#menu(("Menu", "Sub{-}menu", "Sub{-}sub menu", "Action"))}
\end{Highlighting}
\end{Shaded}

\subsubsection{How to add}\label{how-to-add}

Copy this into your project and use the import as \texttt{\ badgery\ }

\begin{verbatim}
#import "@preview/badgery:0.1.1"
\end{verbatim}

\includesvg[width=0.16667in,height=0.16667in]{/assets/icons/16-copy.svg}

Check the docs for
\href{https://typst.app/docs/reference/scripting/\#packages}{more
information on how to import packages} .

\subsubsection{About}\label{about}

\begin{description}
\tightlist
\item[Author :]
dogezen
\item[License:]
MIT
\item[Current version:]
0.1.1
\item[Last updated:]
March 20, 2024
\item[First released:]
March 19, 2024
\item[Minimum Typst version:]
0.11.0
\item[Archive size:]
2.50 kB
\href{https://packages.typst.org/preview/badgery-0.1.1.tar.gz}{\pandocbounded{\includesvg[keepaspectratio]{/assets/icons/16-download.svg}}}
\item[Repository:]
\href{https://github.com/dogezen/badgery}{GitHub}
\item[Discipline s :]
\begin{itemize}
\tightlist
\item[]
\item
  \href{https://typst.app/universe/search/?discipline=computer-science}{Computer
  Science}
\item
  \href{https://typst.app/universe/search/?discipline=engineering}{Engineering}
\item
  \href{https://typst.app/universe/search/?discipline=business}{Business}
\item
  \href{https://typst.app/universe/search/?discipline=communication}{Communication}
\end{itemize}
\item[Categor y :]
\begin{itemize}
\tightlist
\item[]
\item
  \pandocbounded{\includesvg[keepaspectratio]{/assets/icons/16-package.svg}}
  \href{https://typst.app/universe/search/?category=components}{Components}
\end{itemize}
\end{description}

\subsubsection{Where to report issues?}\label{where-to-report-issues}

This package is a project of dogezen . Report issues on
\href{https://github.com/dogezen/badgery}{their repository} . You can
also try to ask for help with this package on the
\href{https://forum.typst.app}{Forum} .

Please report this package to the Typst team using the
\href{https://typst.app/contact}{contact form} if you believe it is a
safety hazard or infringes upon your rights.

\phantomsection\label{versions}
\subsubsection{Version history}\label{version-history}

\begin{longtable}[]{@{}ll@{}}
\toprule\noalign{}
Version & Release Date \\
\midrule\noalign{}
\endhead
\bottomrule\noalign{}
\endlastfoot
0.1.1 & March 20, 2024 \\
\href{https://typst.app/universe/package/badgery/0.1.0/}{0.1.0} & March
19, 2024 \\
\end{longtable}

Typst GmbH did not create this package and cannot guarantee correct
functionality of this package or compatibility with any version of the
Typst compiler or app.


\title{typst.app/universe/package/haw-hamburg-bachelor-thesis}

\phantomsection\label{banner}
\phantomsection\label{template-thumbnail}
\pandocbounded{\includegraphics[keepaspectratio]{https://packages.typst.org/preview/thumbnails/haw-hamburg-bachelor-thesis-0.3.1-small.webp}}

\section{haw-hamburg-bachelor-thesis}\label{haw-hamburg-bachelor-thesis}

{ 0.3.1 }

Unofficial template for writing a bachelor-thesis in the HAW Hamburg
department of Computer Science design.

\href{/app?template=haw-hamburg-bachelor-thesis&version=0.3.1}{Create
project in app}

\phantomsection\label{readme}
This is an \textbf{\texttt{\ unofficial\ }} template for writing a
bachelor thesis in the \texttt{\ HAW\ Hamburg\ } department of
\texttt{\ Computer\ Science\ } design using
\href{https://github.com/typst/typst}{Typst} .

\subsection{Required Fonts}\label{required-fonts}

To correctly render this template please make sure that the
\texttt{\ New\ Computer\ Modern\ } font is installed on your system.

\subsection{Usage}\label{usage}

To use this package just add the following code to your
\href{https://github.com/typst/typst}{Typst} document:

\begin{Shaded}
\begin{Highlighting}[]
\NormalTok{\#import "@preview/haw{-}hamburg:0.3.1": bachelor{-}thesis}

\NormalTok{\#show: bachelor{-}thesis.with(}
\NormalTok{  language: "en",}

\NormalTok{  title{-}de: "Beispiel Titel",}
\NormalTok{  keywords{-}de: ("Stichwort", "Wichtig", "Super"),}
\NormalTok{  abstract{-}de: "Beispiel Zusammenfassung",}

\NormalTok{  title{-}en: "Example title",}
\NormalTok{  keywords{-}en:  ("Keyword", "Important", "Super"),}
\NormalTok{  abstract{-}en: "Example abstract",}

\NormalTok{  author: "Example author",}
\NormalTok{  faculty: "Engineering and Computer Science",}
\NormalTok{  department: "Computer Science",}
\NormalTok{  study{-}course: "Bachelor of Science Informatik Technischer Systeme",}
\NormalTok{  supervisors: ("Prof. Dr. Example", "Prof. Dr. Example"),}
\NormalTok{  submission{-}date: datetime(year: 1948, month: 12, day: 10),}
\NormalTok{  include{-}declaration{-}of{-}independent{-}processing: true,}
\NormalTok{)}
\end{Highlighting}
\end{Shaded}

\subsection{How to Compile}\label{how-to-compile}

This project contains an example setup that splits individual chapters
into different files.\\
This can cause problems when using references etc.\\
These problems can be avoided by following these steps:

\begin{itemize}
\tightlist
\item
  Make sure to always compile your \texttt{\ main.typ\ } file which
  includes all of your chapters for references to work correctly.
\item
  VSCode:

  \begin{itemize}
  \tightlist
  \item
    Install the
    \href{https://marketplace.visualstudio.com/items?itemName=myriad-dreamin.tinymist}{Tinymist
    Typst} extension.
  \item
    Make sure to start the \texttt{\ PDF\ } or
    \texttt{\ Live\ Preview\ } only from within your
    \texttt{\ main.typ\ } file.
  \item
    If problems occur it usually helps to close the preview and restart
    it from your \texttt{\ main.typ\ } file.
  \end{itemize}
\end{itemize}

\href{/app?template=haw-hamburg-bachelor-thesis&version=0.3.1}{Create
project in app}

\subsubsection{How to use}\label{how-to-use}

Click the button above to create a new project using this template in
the Typst app.

You can also use the Typst CLI to start a new project on your computer
using this command:

\begin{verbatim}
typst init @preview/haw-hamburg-bachelor-thesis:0.3.1
\end{verbatim}

\includesvg[width=0.16667in,height=0.16667in]{/assets/icons/16-copy.svg}

\subsubsection{About}\label{about}

\begin{description}
\tightlist
\item[Author :]
Lasse Rosenow
\item[License:]
MIT
\item[Current version:]
0.3.1
\item[Last updated:]
November 13, 2024
\item[First released:]
October 14, 2024
\item[Archive size:]
6.57 kB
\href{https://packages.typst.org/preview/haw-hamburg-bachelor-thesis-0.3.1.tar.gz}{\pandocbounded{\includesvg[keepaspectratio]{/assets/icons/16-download.svg}}}
\item[Repository:]
\href{https://github.com/LasseRosenow/HAW-Hamburg-Typst-Template}{GitHub}
\item[Categor y :]
\begin{itemize}
\tightlist
\item[]
\item
  \pandocbounded{\includesvg[keepaspectratio]{/assets/icons/16-mortarboard.svg}}
  \href{https://typst.app/universe/search/?category=thesis}{Thesis}
\end{itemize}
\end{description}

\subsubsection{Where to report issues?}\label{where-to-report-issues}

This template is a project of Lasse Rosenow . Report issues on
\href{https://github.com/LasseRosenow/HAW-Hamburg-Typst-Template}{their
repository} . You can also try to ask for help with this template on the
\href{https://forum.typst.app}{Forum} .

Please report this template to the Typst team using the
\href{https://typst.app/contact}{contact form} if you believe it is a
safety hazard or infringes upon your rights.

\phantomsection\label{versions}
\subsubsection{Version history}\label{version-history}

\begin{longtable}[]{@{}ll@{}}
\toprule\noalign{}
Version & Release Date \\
\midrule\noalign{}
\endhead
\bottomrule\noalign{}
\endlastfoot
0.3.1 & November 13, 2024 \\
\href{https://typst.app/universe/package/haw-hamburg-bachelor-thesis/0.3.0/}{0.3.0}
& October 14, 2024 \\
\end{longtable}

Typst GmbH did not create this template and cannot guarantee correct
functionality of this template or compatibility with any version of the
Typst compiler or app.


\title{typst.app/universe/package/modpattern}

\phantomsection\label{banner}
\section{modpattern}\label{modpattern}

{ 0.1.0 }

Easily create patterns in typst.

\phantomsection\label{readme}
This package provides exactly one function: \texttt{\ modpattern\ }

It’s primary goal is to create make this:
\pandocbounded{\includegraphics[keepaspectratio]{https://github.com/typst/packages/raw/main/packages/preview/modpattern/0.1.0/examples/comparison.png}}

\subsection{modpattern function}\label{modpattern-function}

The function with the signature
\texttt{\ modpattern(size,\ body,\ dx:\ 0pt,\ dy:\ 0pt,\ background:\ none)\ }
has the following parameters:

\begin{itemize}
\tightlist
\item
  \texttt{\ size\ } is as size for patterns
\item
  \texttt{\ body\ } is the inside/body/content of the pattern
\item
  \texttt{\ dx\ } , \texttt{\ dy\ } allow for translations
\item
  \texttt{\ background\ } allows any type allowed in the box fill
  argument. It gets applied first
\end{itemize}

Notice that, in contrast to typst patterns, size is a positional
argument.

Take a look at the example directory, to understand how to use this and
to see the reasoning behind the \texttt{\ background\ } argument.

\subsubsection{How to add}\label{how-to-add}

Copy this into your project and use the import as
\texttt{\ modpattern\ }

\begin{verbatim}
#import "@preview/modpattern:0.1.0"
\end{verbatim}

\includesvg[width=0.16667in,height=0.16667in]{/assets/icons/16-copy.svg}

Check the docs for
\href{https://typst.app/docs/reference/scripting/\#packages}{more
information on how to import packages} .

\subsubsection{About}\label{about}

\begin{description}
\tightlist
\item[Author :]
Ludwig Austermann
\item[License:]
MIT
\item[Current version:]
0.1.0
\item[Last updated:]
April 12, 2024
\item[First released:]
April 12, 2024
\item[Minimum Typst version:]
0.11.0
\item[Archive size:]
1.62 kB
\href{https://packages.typst.org/preview/modpattern-0.1.0.tar.gz}{\pandocbounded{\includesvg[keepaspectratio]{/assets/icons/16-download.svg}}}
\item[Repository:]
\href{https://github.com/ludwig-austermann/modpattern}{GitHub}
\end{description}

\subsubsection{Where to report issues?}\label{where-to-report-issues}

This package is a project of Ludwig Austermann . Report issues on
\href{https://github.com/ludwig-austermann/modpattern}{their repository}
. You can also try to ask for help with this package on the
\href{https://forum.typst.app}{Forum} .

Please report this package to the Typst team using the
\href{https://typst.app/contact}{contact form} if you believe it is a
safety hazard or infringes upon your rights.

\phantomsection\label{versions}
\subsubsection{Version history}\label{version-history}

\begin{longtable}[]{@{}ll@{}}
\toprule\noalign{}
Version & Release Date \\
\midrule\noalign{}
\endhead
\bottomrule\noalign{}
\endlastfoot
0.1.0 & April 12, 2024 \\
\end{longtable}

Typst GmbH did not create this package and cannot guarantee correct
functionality of this package or compatibility with any version of the
Typst compiler or app.


\title{typst.app/universe/package/starter-journal-article}

\phantomsection\label{banner}
\phantomsection\label{template-thumbnail}
\pandocbounded{\includegraphics[keepaspectratio]{https://packages.typst.org/preview/thumbnails/starter-journal-article-0.3.1-small.webp}}

\section{starter-journal-article}\label{starter-journal-article}

{ 0.3.1 }

A starter template for journal articles.

\href{/app?template=starter-journal-article&version=0.3.1}{Create
project in app}

\phantomsection\label{readme}
This package provides a template for writing journal articles to
organise authors, institutions, and information of corresponding
authors.

\subsection{Usage}\label{usage}

Run the following command to use this template

\begin{Shaded}
\begin{Highlighting}[]
\NormalTok{typst init @preview/starter{-}journal{-}article}
\end{Highlighting}
\end{Shaded}

\subsection{Documentation}\label{documentation}

\subsubsection{\texorpdfstring{\texttt{\ article\ }}{ article }}\label{article}

The template for creating journal articles. It needs the following
arguments.

Arguments:

\begin{itemize}
\tightlist
\item
  \texttt{\ title\ } : The title of this article. Default:
  \texttt{\ "Article\ Title"\ } .
\item
  \texttt{\ authors\ } : A dictionary of authors. Dictionary keys are
  authors’ names. Dictionary values are meta data of every author,
  including label(s) of affiliation(s), email, contact address, or a
  self-defined name (to avoid name conflicts). The label(s) of
  affiliation(s) must be those claimed in the argument
  \texttt{\ affiliations\ } . Once the email or address exists, the
  author(s) will be labelled as the corresponding author(s), and their
  address will show in footnotes. Function \texttt{\ author-meta()\ } is
  useful in creating information for each author. Default:
  \texttt{\ ("Author\ Name":\ author-meta("affiliation-label"))\ } .
\item
  \texttt{\ affiliations\ } : A dictionary of affiliation. Dictionary
  keys are affiliations’ labels. These labels show be constent with
  those used in authors’ meta data. Dictionary values are addresses of
  every affiliation. Default:
  \texttt{\ ("affiliation-label":\ "Affiliation\ address")\ } .
\item
  \texttt{\ abstract\ } : The paper’s abstract. Default:
  \texttt{\ {[}{]}\ } .
\item
  \texttt{\ keywords\ } : The paper’s keywords. Default:
  \texttt{\ {[}{]}\ } .
\item
  \texttt{\ bib\ } : The bibliography. Accept value from the built-in
  \texttt{\ bibliography\ } function. Default: \texttt{\ none\ } .
\item
  \texttt{\ template\ } : Templates for the following parts. Please see
  below for more informations

  \begin{itemize}
  \tightlist
  \item
    \texttt{\ title:\ (title)\ =\textgreater{}\ \{\}\ } : how to show
    the title of this article.
  \item
    \texttt{\ author-info:\ (authors,\ affiliations)\ =\textgreater{}\ \{\}\ }
    : how to show each author’s information.
  \item
    \texttt{\ abstract:\ (abstract,\ keywords)\ =\textgreater{}\ \{\}\ }
    : how to show the abstract and keywords.
  \item
    \texttt{\ bibliography:\ (bib)\ =\textgreater{}\ \{\}\ } : how to
    show the bibliography.
  \item
    \texttt{\ body:\ (body)\ =\textgreater{}\ \{\}\ } : how to show main
    text.
  \end{itemize}
\end{itemize}

\subsubsection{\texorpdfstring{\texttt{\ author-meta\ }}{ author-meta }}\label{author-meta}

A helper to create meta information for an author.

Arguments:

\begin{itemize}
\tightlist
\item
  \texttt{\ ..affiliation\ } : Capture the positioned arguments as
  label(s) of affiliation(s). Mandatory.
\item
  \texttt{\ email\ } : The email address of the author. Default:
  \texttt{\ none\ } .
\item
  \texttt{\ alias\ } : The display name of the author. Default:
  \texttt{\ none\ } .
\item
  \texttt{\ address\ } : The address of the author. Default:
  \texttt{\ none\ } .
\item
  \texttt{\ cofirst\ } : Whether the author is the co-first author.
  Default: \texttt{\ false\ } .
\end{itemize}

\subsection{Default templates}\label{default-templates}

The following code shows how the default templates are defined. You can
override any of the by setting the \texttt{\ template\ } argument in the
\texttt{\ article()\ } function to customise the template.

\begin{Shaded}
\begin{Highlighting}[]
\NormalTok{\#let default{-}title(title) = \{}
\NormalTok{  show: block.with(width: 100\%)}
\NormalTok{  set align(center)}
\NormalTok{  set text(size: 1.75em, weight: "bold")}
\NormalTok{  title}
\NormalTok{\}}

\NormalTok{\#let default{-}author(author) = \{}
\NormalTok{  text(author.name)}
\NormalTok{  super(author.insts.map(it =\textgreater{} str(it)).join(","))}
\NormalTok{  if author.corresponding \{}
\NormalTok{    footnote[}
\NormalTok{      Corresponding author. Address: \#author.address.}
\NormalTok{      \#if author.email != none \{}
\NormalTok{        [Email: \#underline(author.email).]}
\NormalTok{      \}}
\NormalTok{    ]}
\NormalTok{  \}}
\NormalTok{  if author.cofirst == "thefirst" \{}
\NormalTok{    footnote("cofirst{-}author{-}mark")}
\NormalTok{  \} else if author.cofirst == "cofirst" \{}
\NormalTok{    locate(loc =\textgreater{} query(footnote.where(body: [cofirst{-}author{-}mark]), loc).last())}
\NormalTok{  \}}
\NormalTok{\}}

\NormalTok{\#let default{-}affiliation(id, address) = \{}
\NormalTok{  set text(size: 0.8em)}
\NormalTok{  super([\#(id+1)])}
\NormalTok{  address}
\NormalTok{\}}

\NormalTok{\#let default{-}author{-}info(authors, affiliations) = \{}
\NormalTok{  \{}
\NormalTok{    show: block.with(width: 100\%)}
\NormalTok{    authors.map(it =\textgreater{} default{-}author(it)).join(", ")}
\NormalTok{  \}}
\NormalTok{  \{}
\NormalTok{    show: block.with(width: 100\%)}
\NormalTok{    set par(leading: 0.4em)}
\NormalTok{    affiliations.keys().enumerate().map(((ik, key)) =\textgreater{} \{}
\NormalTok{      default{-}affiliation(ik, affiliations.at(key))}
\NormalTok{    \}).join(linebreak())}
\NormalTok{  \}}
\NormalTok{\}}

\NormalTok{\#let default{-}abstract(abstract, keywords) = \{}
\NormalTok{  // Abstract and keyword block}
\NormalTok{  if abstract != [] \{}
\NormalTok{    stack(}
\NormalTok{      dir: ttb,}
\NormalTok{      spacing: 1em,}
\NormalTok{      ..([}
\NormalTok{        \#heading([Abstract])}
\NormalTok{        \#abstract}
\NormalTok{      ], if keywords.len() \textgreater{} 0 \{}
\NormalTok{        text(weight: "bold", [Key words: ])}
\NormalTok{        text([\#keywords.join([; ]).])}
\NormalTok{      \} else \{none\} )}
\NormalTok{    )}
\NormalTok{  \}}
\NormalTok{\}}

\NormalTok{\#let default{-}bibliography(bib) = \{}
\NormalTok{  show bibliography: set text(1em)}
\NormalTok{  show bibliography: set par(first{-}line{-}indent: 0em)}
\NormalTok{  set bibliography(title: [References], style: "apa")}
\NormalTok{  bib}
\NormalTok{\}}

\NormalTok{\#let default{-}body(body) = \{}
\NormalTok{  show heading.where(level: 1): it =\textgreater{} block(above: 1.5em, below: 1.5em)[}
\NormalTok{    \#set pad(bottom: 2em, top: 1em)}
\NormalTok{    \#it.body}
\NormalTok{  ]}
\NormalTok{  set par(first{-}line{-}indent: 2em)}
\NormalTok{  set footnote(numbering: "1")}
\NormalTok{  body}
\NormalTok{\}}
\end{Highlighting}
\end{Shaded}

\subsection{Example}\label{example}

See
\href{https://github.com/typst/packages/raw/main/packages/preview/starter-journal-article/0.3.1/template/main.typ}{the
template} for full example.

\begin{Shaded}
\begin{Highlighting}[]
\NormalTok{\#show: article.with(}
\NormalTok{  title: "Artile Title",}
\NormalTok{  authors: (}
\NormalTok{    "Author One": author{-}meta(}
\NormalTok{      "UCL", "TSU",}
\NormalTok{      email: "author.one@inst.ac.uk",}
\NormalTok{    ),}
\NormalTok{    "Author Two": author{-}meta(}
\NormalTok{      "TSU",}
\NormalTok{      cofirst: true}
\NormalTok{    ),}
\NormalTok{    "Author Three": author{-}meta(}
\NormalTok{      "TSU"}
\NormalTok{    )}
\NormalTok{  ),}
\NormalTok{  affiliations: (}
\NormalTok{    "UCL": "UCL Centre for Advanced Spatial Analysis, First Floor, 90 Tottenham Court Road, London W1T 4TJ, United Kingdom",}
\NormalTok{    "TSU": "Haidian  District, Beijing, 100084, P. R. China"}
\NormalTok{  ),}
\NormalTok{  abstract: [\#lorem(100)],}
\NormalTok{  keywords: ("Typst", "Template", "Journal Article"),}
\NormalTok{  bib: bibliography("./ref.bib")}
\NormalTok{)}
\end{Highlighting}
\end{Shaded}

\pandocbounded{\includegraphics[keepaspectratio]{https://github.com/typst/packages/raw/main/packages/preview/starter-journal-article/0.3.1/assets/basic.png}}

\subsubsection{Custom title}\label{custom-title}

\begin{Shaded}
\begin{Highlighting}[]
\NormalTok{\#show: article.with(}
\NormalTok{  title: "Artile Title",}
\NormalTok{  authors: (}
\NormalTok{    "Author One": author{-}meta(}
\NormalTok{      "UCL", "TSU",}
\NormalTok{      email: "author.one@inst.ac.uk",}
\NormalTok{    ),}
\NormalTok{    "Author Two": author{-}meta(}
\NormalTok{      "TSU",}
\NormalTok{      cofirst: true}
\NormalTok{    ),}
\NormalTok{    "Author Three": author{-}meta(}
\NormalTok{      "TSU"}
\NormalTok{    )}
\NormalTok{  ),}
\NormalTok{  affiliations: (}
\NormalTok{    "UCL": "UCL Centre for Advanced Spatial Analysis, First Floor, 90 Tottenham Court Road, London W1T 4TJ, United Kingdom",}
\NormalTok{    "TSU": "Haidian  District, Beijing, 100084, P. R. China"}
\NormalTok{  ),}
\NormalTok{  abstract: [\#lorem(100)],}
\NormalTok{  keywords: ("Typst", "Template", "Journal Article"),}
\NormalTok{  bib: bibliography("./ref.bib"),}
\NormalTok{  template: (}
\NormalTok{    title: (title) =\textgreater{} \{}
\NormalTok{      set align(left)}
\NormalTok{      set text(size: 1.5em, weight: "bold", style: "italic")}
\NormalTok{      title}
\NormalTok{    \}}
\NormalTok{  )}
\NormalTok{)}
\end{Highlighting}
\end{Shaded}

\pandocbounded{\includegraphics[keepaspectratio]{https://github.com/typst/packages/raw/main/packages/preview/starter-journal-article/0.3.1/assets/custom-title.png}}

\subsubsection{Custom author infomation}\label{custom-author-infomation}

\begin{Shaded}
\begin{Highlighting}[]
\NormalTok{\#show: article.with(}
\NormalTok{  title: "Artile Title",}
\NormalTok{  authors: (}
\NormalTok{    "Author One": author{-}meta("UCL", email: "author.one@inst.ac.uk"),}
\NormalTok{    "Author Two": author{-}meta("TSU")}
\NormalTok{  ),}
\NormalTok{  affiliations: (}
\NormalTok{    "UCL": "UCL Centre for Advanced Spatial Analysis, First Floor, 90 Tottenham Court Road, London W1T 4TJ, United Kingdom",}
\NormalTok{    "TSU": "Haidian  District, Beijing, 100084, P. R. China"}
\NormalTok{  ),}
\NormalTok{  abstract: [\#lorem(20)],}
\NormalTok{  keywords: ("Typst", "Template", "Journal Article"),}
\NormalTok{  template: (}
\NormalTok{    author{-}info: (authors, affiliations) =\textgreater{} \{}
\NormalTok{      set align(center)}
\NormalTok{      show: block.with(width: 100\%, above: 2em, below: 2em)}
\NormalTok{      let first\_insts = authors.map(it =\textgreater{} it.insts.at(0)).dedup()}
\NormalTok{      stack(}
\NormalTok{        dir: ttb,}
\NormalTok{        spacing: 1em,}
\NormalTok{        ..first\_insts.map(inst\_id =\textgreater{} \{}
\NormalTok{          let inst\_authors = authors.filter(it =\textgreater{} it.insts.at(0) == inst\_id)}
\NormalTok{          stack(}
\NormalTok{            dir: ttb,}
\NormalTok{            spacing: 1em,}
\NormalTok{            \{}
\NormalTok{              inst\_authors.map(it =\textgreater{} it.name).join(", ")}
\NormalTok{            \},}
\NormalTok{            \{}
\NormalTok{              set text(0.8em, style: "italic")}
\NormalTok{              affiliations.values().at(inst\_id)}
\NormalTok{            \}}
\NormalTok{          )}
\NormalTok{        \})}
\NormalTok{      )}
\NormalTok{    \}}
\NormalTok{  )}
\NormalTok{)}
\end{Highlighting}
\end{Shaded}

\pandocbounded{\includegraphics[keepaspectratio]{https://github.com/typst/packages/raw/main/packages/preview/starter-journal-article/0.3.1/assets/custom-author-info.png}}

\subsubsection{Custom abstract}\label{custom-abstract}

\begin{Shaded}
\begin{Highlighting}[]
\NormalTok{\#show: article.with(}
\NormalTok{  title: "Artile Title",}
\NormalTok{  authors: (}
\NormalTok{    "Author One": author{-}meta("UCL", email: "author.one@inst.ac.uk"),}
\NormalTok{    "Author Two": author{-}meta("TSU")}
\NormalTok{  ),}
\NormalTok{  affiliations: (}
\NormalTok{    "UCL": "UCL Centre for Advanced Spatial Analysis, First Floor, 90 Tottenham Court Road, London W1T 4TJ, United Kingdom",}
\NormalTok{    "TSU": "Haidian  District, Beijing, 100084, P. R. China"}
\NormalTok{  ),}
\NormalTok{  abstract: [\#lorem(20)],}
\NormalTok{  keywords: ("Typst", "Template", "Journal Article"),}
\NormalTok{  template: (}
\NormalTok{    abstract: (abstract, keywords) =\textgreater{} \{}
\NormalTok{      show: block.with(}
\NormalTok{        width: 100\%,}
\NormalTok{        stroke: (y: 0.5pt + black),}
\NormalTok{        inset: (y: 1em)}
\NormalTok{      )}
\NormalTok{      show heading: set text(size: 12pt)}
\NormalTok{      heading(numbering: none, outlined: false, bookmarked: false, [Abstract])}
\NormalTok{      par(abstract)}
\NormalTok{      stack(}
\NormalTok{        dir: ltr,}
\NormalTok{        spacing: 4pt,}
\NormalTok{        strong([Keywords]),}
\NormalTok{        keywords.join(", ")}
\NormalTok{      )}
\NormalTok{    \}}
\NormalTok{  )}
\NormalTok{)}
\end{Highlighting}
\end{Shaded}

\pandocbounded{\includegraphics[keepaspectratio]{https://github.com/typst/packages/raw/main/packages/preview/starter-journal-article/0.3.1/assets/custom-abstract.png}}

\href{/app?template=starter-journal-article&version=0.3.1}{Create
project in app}

\subsubsection{How to use}\label{how-to-use}

Click the button above to create a new project using this template in
the Typst app.

You can also use the Typst CLI to start a new project on your computer
using this command:

\begin{verbatim}
typst init @preview/starter-journal-article:0.3.1
\end{verbatim}

\includesvg[width=0.16667in,height=0.16667in]{/assets/icons/16-copy.svg}

\subsubsection{About}\label{about}

\begin{description}
\tightlist
\item[Author :]
\href{https://github.com/HPDell}{HPDell}
\item[License:]
MIT
\item[Current version:]
0.3.1
\item[Last updated:]
August 19, 2024
\item[First released:]
March 26, 2024
\item[Archive size:]
5.29 kB
\href{https://packages.typst.org/preview/starter-journal-article-0.3.1.tar.gz}{\pandocbounded{\includesvg[keepaspectratio]{/assets/icons/16-download.svg}}}
\item[Repository:]
\href{https://github.com/HPDell/typst-starter-journal-article}{GitHub}
\item[Categor y :]
\begin{itemize}
\tightlist
\item[]
\item
  \pandocbounded{\includesvg[keepaspectratio]{/assets/icons/16-atom.svg}}
  \href{https://typst.app/universe/search/?category=paper}{Paper}
\end{itemize}
\end{description}

\subsubsection{Where to report issues?}\label{where-to-report-issues}

This template is a project of HPDell . Report issues on
\href{https://github.com/HPDell/typst-starter-journal-article}{their
repository} . You can also try to ask for help with this template on the
\href{https://forum.typst.app}{Forum} .

Please report this template to the Typst team using the
\href{https://typst.app/contact}{contact form} if you believe it is a
safety hazard or infringes upon your rights.

\phantomsection\label{versions}
\subsubsection{Version history}\label{version-history}

\begin{longtable}[]{@{}ll@{}}
\toprule\noalign{}
Version & Release Date \\
\midrule\noalign{}
\endhead
\bottomrule\noalign{}
\endlastfoot
0.3.1 & August 19, 2024 \\
\href{https://typst.app/universe/package/starter-journal-article/0.3.0/}{0.3.0}
& April 8, 2024 \\
\href{https://typst.app/universe/package/starter-journal-article/0.2.0/}{0.2.0}
& April 2, 2024 \\
\href{https://typst.app/universe/package/starter-journal-article/0.1.1/}{0.1.1}
& March 26, 2024 \\
\end{longtable}

Typst GmbH did not create this template and cannot guarantee correct
functionality of this template or compatibility with any version of the
Typst compiler or app.


\title{typst.app/universe/package/outline-summaryst}

\phantomsection\label{banner}
\section{outline-summaryst}\label{outline-summaryst}

{ 0.1.0 }

A basic template for including a summary for each entry in the table of
contents. Useful for writing books.

\phantomsection\label{readme}
\subsection{Description}\label{description}

\texttt{\ outline-summaryst\ } is a basic package designed for including
a summary for each entry in the table of contents, particularly useful
for writing books. It provides a simple structure for organizing content
and generating formatted documents with summary sections.

\subsection{Features}\label{features}

\begin{itemize}
\tightlist
\item
  A template for the outline, which styles both the heading and their
  summaries.
\item
  A macro for creating new headings and a summary for each heading.
\end{itemize}

\subsection{Note:}\label{note}

Because of the way the project is implemented, only the headings created
with the provided \texttt{\ make-heading("heading\ name",\ "summary")\ }
are shown in in the outline. Headings created with the default
\texttt{\ =\ Heading\ } syntax will not show in said outline (though
they will show up in the document itself).

\subsection{Example Usage:}\label{example-usage}

\begin{verbatim}
#import "@preview/outline-summaryst:0.1.0": style-outline, make-heading


// you can set `outline-title: none` in order not to display any title
#show outline: style-outline.with(outline-title: "Table of Contents")

#outline()


#make-heading("Part One", "This is the summary for part one")
#lorem(500)

#make-heading("Chapter One", "Summary for chapter one in part one", level: 2)
#lorem(300)

#make-heading("Chapter Two", "This is the summary for chapter two in part one", level: 2)
#lorem(300)

#make-heading("Part Two", "And here we have the summary for part two")
#lorem(500)

#make-heading("Chapter One", "Summary for chapter one in part two", level: 2)
#lorem(300)

#make-heading("Chapter Two", "Summary for chapter two in part two", level: 2)
#lorem(300)
\end{verbatim}

\subsection{Known limitations}\label{known-limitations}

\begin{itemize}
\tightlist
\item
  Currently, we do not provide a way for styling the table of contents
  or headings
\end{itemize}

\subsection{License:}\label{license}

This project is licensed under the MIT License. See the LICENSE file for
details.

\subsection{Contribution:}\label{contribution}

Contributions are welcome! Feel free to open an issue or submit a pull
request on GitHub.

\subsubsection{How to add}\label{how-to-add}

Copy this into your project and use the import as
\texttt{\ outline-summaryst\ }

\begin{verbatim}
#import "@preview/outline-summaryst:0.1.0"
\end{verbatim}

\includesvg[width=0.16667in,height=0.16667in]{/assets/icons/16-copy.svg}

Check the docs for
\href{https://typst.app/docs/reference/scripting/\#packages}{more
information on how to import packages} .

\subsubsection{About}\label{about}

\begin{description}
\tightlist
\item[Author :]
@aarneng
\item[License:]
MIT
\item[Current version:]
0.1.0
\item[Last updated:]
April 15, 2024
\item[First released:]
April 15, 2024
\item[Archive size:]
2.97 kB
\href{https://packages.typst.org/preview/outline-summaryst-0.1.0.tar.gz}{\pandocbounded{\includesvg[keepaspectratio]{/assets/icons/16-download.svg}}}
\item[Repository:]
\href{https://github.com/aarneng/Outline-Summary}{GitHub}
\item[Discipline :]
\begin{itemize}
\tightlist
\item[]
\item
  \href{https://typst.app/universe/search/?discipline=literature}{Literature}
\end{itemize}
\item[Categor y :]
\begin{itemize}
\tightlist
\item[]
\item
  \pandocbounded{\includesvg[keepaspectratio]{/assets/icons/16-layout.svg}}
  \href{https://typst.app/universe/search/?category=layout}{Layout}
\end{itemize}
\end{description}

\subsubsection{Where to report issues?}\label{where-to-report-issues}

This package is a project of @aarneng . Report issues on
\href{https://github.com/aarneng/Outline-Summary}{their repository} .
You can also try to ask for help with this package on the
\href{https://forum.typst.app}{Forum} .

Please report this package to the Typst team using the
\href{https://typst.app/contact}{contact form} if you believe it is a
safety hazard or infringes upon your rights.

\phantomsection\label{versions}
\subsubsection{Version history}\label{version-history}

\begin{longtable}[]{@{}ll@{}}
\toprule\noalign{}
Version & Release Date \\
\midrule\noalign{}
\endhead
\bottomrule\noalign{}
\endlastfoot
0.1.0 & April 15, 2024 \\
\end{longtable}

Typst GmbH did not create this package and cannot guarantee correct
functionality of this package or compatibility with any version of the
Typst compiler or app.


\title{typst.app/universe/package/titleize}

\phantomsection\label{banner}
\section{titleize}\label{titleize}

{ 0.1.1 }

Turn strings into title case

\phantomsection\label{readme}
Small wrapper around the
\href{https://crates.io/crates/titlecase}{titlecase} library to convert
text to title case. It follows the
\href{https://daringfireball.net/2008/05/title_case}{rules defined by
John Gruber} . For more details, refer to the library.

\texttt{\ titlecase\ } applies a show rule, that by default transforms
every string of at least four characters. This limit can be changed with
the \texttt{\ limit\ } parameter. Especially with equations, the results
can be a bit unpredictable, so proceed with care.

\begin{Shaded}
\begin{Highlighting}[]
\NormalTok{\#import "@preview/titleize:0.1.1": titlecase}

\NormalTok{\#for s in (}
\NormalTok{  "Being productive on linux",}
\NormalTok{  "Finding an alternative to Mac OS X — part 2",}
\NormalTok{  [an example with small words and sub{-}phrases: "the example"],}
\NormalTok{) [}
\NormalTok{  \#s =\textgreater{} \#titlecase(s) \textbackslash{}}
\NormalTok{]}

\NormalTok{\#let debug{-}print(x) = \{}
\NormalTok{  if type(x) == content \{}
\NormalTok{    let fields = x.fields()}
\NormalTok{    let func = x.func()}
\NormalTok{    [}
\NormalTok{      \#repr(func)}
\NormalTok{      \#for (k, v) in fields [}
\NormalTok{        {-} \#k: \#debug{-}print(v)}
\NormalTok{      ]}
\NormalTok{    ]}
\NormalTok{  \} else \{}
\NormalTok{    if type(x) == array [}
\NormalTok{      array}
\NormalTok{      \#for y in x [}
\NormalTok{        {-} \#debug{-}print(y)}
\NormalTok{      ]}
\NormalTok{    ] else [}
\NormalTok{      \#repr(type(x)) (\#repr(x))}
\NormalTok{    ]}
\NormalTok{  \}}
\NormalTok{\}}

\NormalTok{\#show: titlecase}

\NormalTok{= This is a test, even with math $a = b + c$}

\NormalTok{In math, text can appear in various places:}

\NormalTok{$}
\NormalTok{  a\_"for example in a subscript" \&= "or in a longer text" \textbackslash{}}
\NormalTok{  f(x) \&= sin(x)}
\NormalTok{$}
\end{Highlighting}
\end{Shaded}

\pandocbounded{\includegraphics[keepaspectratio]{https://github.com/typst/packages/raw/main/packages/preview/titleize/0.1.1/example.png}}

\subsubsection{How to add}\label{how-to-add}

Copy this into your project and use the import as \texttt{\ titleize\ }

\begin{verbatim}
#import "@preview/titleize:0.1.1"
\end{verbatim}

\includesvg[width=0.16667in,height=0.16667in]{/assets/icons/16-copy.svg}

Check the docs for
\href{https://typst.app/docs/reference/scripting/\#packages}{more
information on how to import packages} .

\subsubsection{About}\label{about}

\begin{description}
\tightlist
\item[Author :]
\href{mailto:mail@solidtux.de}{Daniel Hauck}
\item[License:]
MIT
\item[Current version:]
0.1.1
\item[Last updated:]
October 15, 2024
\item[First released:]
October 7, 2024
\item[Archive size:]
253 kB
\href{https://packages.typst.org/preview/titleize-0.1.1.tar.gz}{\pandocbounded{\includesvg[keepaspectratio]{/assets/icons/16-download.svg}}}
\item[Repository:]
\href{https://gitlab.com/SolidTux/titleize}{GitLab}
\item[Categor ies :]
\begin{itemize}
\tightlist
\item[]
\item
  \pandocbounded{\includesvg[keepaspectratio]{/assets/icons/16-text.svg}}
  \href{https://typst.app/universe/search/?category=text}{Text}
\item
  \pandocbounded{\includesvg[keepaspectratio]{/assets/icons/16-code.svg}}
  \href{https://typst.app/universe/search/?category=scripting}{Scripting}
\item
  \pandocbounded{\includesvg[keepaspectratio]{/assets/icons/16-hammer.svg}}
  \href{https://typst.app/universe/search/?category=utility}{Utility}
\end{itemize}
\end{description}

\subsubsection{Where to report issues?}\label{where-to-report-issues}

This package is a project of Daniel Hauck . Report issues on
\href{https://gitlab.com/SolidTux/titleize}{their repository} . You can
also try to ask for help with this package on the
\href{https://forum.typst.app}{Forum} .

Please report this package to the Typst team using the
\href{https://typst.app/contact}{contact form} if you believe it is a
safety hazard or infringes upon your rights.

\phantomsection\label{versions}
\subsubsection{Version history}\label{version-history}

\begin{longtable}[]{@{}ll@{}}
\toprule\noalign{}
Version & Release Date \\
\midrule\noalign{}
\endhead
\bottomrule\noalign{}
\endlastfoot
0.1.1 & October 15, 2024 \\
\href{https://typst.app/universe/package/titleize/0.1.0/}{0.1.0} &
October 7, 2024 \\
\end{longtable}

Typst GmbH did not create this package and cannot guarantee correct
functionality of this package or compatibility with any version of the
Typst compiler or app.


\title{typst.app/universe/package/clean-math-paper}

\phantomsection\label{banner}
\phantomsection\label{template-thumbnail}
\pandocbounded{\includegraphics[keepaspectratio]{https://packages.typst.org/preview/thumbnails/clean-math-paper-0.1.0-small.webp}}

\section{clean-math-paper}\label{clean-math-paper}

{ 0.1.0 }

A simple and good looking template for mathematical papers

\href{/app?template=clean-math-paper&version=0.1.0}{Create project in
app}

\phantomsection\label{readme}
\href{https://github.com/JoshuaLampert/clean-math-paper/actions/workflows/build.yml}{\pandocbounded{\includesvg[keepaspectratio]{https://github.com/JoshuaLampert/clean-math-paper/actions/workflows/build.yml/badge.svg}}}
\href{https://github.com/JoshuaLampert/clean-math-paper}{\pandocbounded{\includegraphics[keepaspectratio]{https://img.shields.io/badge/GitHub-repo-blue}}}
\href{https://opensource.org/licenses/MIT}{\pandocbounded{\includesvg[keepaspectratio]{https://img.shields.io/badge/License-MIT-success.svg}}}

\href{https://typst.app/home/}{Typst} paper template for mathematical
papers built for simple, efficient use and a clean look. Of course, it
can also be used for other subjects, but the following math-specific
features are already contained in the template:

\begin{itemize}
\tightlist
\item
  theorems, lemmas, corollaries, proofs etc. prepared using
  \href{https://typst.app/universe/package/great-theorems}{great-theorems}
\item
  equation settings
\end{itemize}

\subsection{Set-Up}\label{set-up}

The template is already filled with dummy data, to give users an
impression how it looks like. The paper is obtained by compiling
\texttt{\ main.typ\ } .

\begin{itemize}
\tightlist
\item
  after
  \href{https://github.com/typst/typst?tab=readme-ov-file\#installation}{installing
  Typst} you can conveniently use the following to create a new folder
  containing this project.
\end{itemize}

\begin{Shaded}
\begin{Highlighting}[]
\ExtensionTok{typst}\NormalTok{ init @preview/clean{-}math{-}paper:0.1.0}
\end{Highlighting}
\end{Shaded}

\begin{itemize}
\tightlist
\item
  edit the data in \texttt{\ main.typ\ } â†'
  \texttt{\ \#show\ template.with({[}your\ data{]})\ }
\end{itemize}

\subsubsection{Parameters of the
Template}\label{parameters-of-the-template}

\begin{itemize}
\tightlist
\item
  \texttt{\ title\ } : Title of the paper.
\item
  \texttt{\ authors\ } : List of names of the authors of the paper. Each
  entry of the list is a dictionary with the following keys:

  \begin{itemize}
  \tightlist
  \item
    \texttt{\ name\ } : Name of the author.
  \item
    \texttt{\ affiliation-id\ } : The ID of the affiliation in
    \texttt{\ affiliations\ } , see below.
  \item
    optionally \texttt{\ orcid\ } : The \href{https://orcid.org/}{ORCID}
    of the author. If provided, the author’s name will be linked to
    their ORCID profile.
  \end{itemize}
\item
  \texttt{\ affiliations\ } : List of affiliations of the authors. Each
  entry of the list is a dictionary with the following keys:

  \begin{itemize}
  \tightlist
  \item
    \texttt{\ id\ } : ID of the affiliation, which is used to link the
    authors to the affiliation, see above.
  \item
    \texttt{\ name\ } : Name of the affiliation.
  \end{itemize}
\item
  \texttt{\ date\ } : Date of the paper.
\item
  \texttt{\ heading-color\ } : Color of the headings including the
  title.
\item
  \texttt{\ link-color\ } : Color of the links.
\item
  \texttt{\ abstract\ } : Abstract of the paper.
\item
  \texttt{\ keywords\ } : List of keywords of the paper. If not
  provided, nothing will be shown.
\item
  \texttt{\ AMS\ } : List of AMS subject classifications of the paper.
  If not provided, nothing will be shown.
\end{itemize}

\subsection{Acknowledgements}\label{acknowledgements}

Some parts of this template are based on the
\href{https://github.com/mgoulao/arkheion}{arkheion} template.

\subsection{Feedback \& Improvements}\label{feedback-improvements}

If you encounter problems, please open issues. In case you found useful
extensions or improved anything We are also very happy to accept pull
requests.

\href{/app?template=clean-math-paper&version=0.1.0}{Create project in
app}

\subsubsection{How to use}\label{how-to-use}

Click the button above to create a new project using this template in
the Typst app.

You can also use the Typst CLI to start a new project on your computer
using this command:

\begin{verbatim}
typst init @preview/clean-math-paper:0.1.0
\end{verbatim}

\includesvg[width=0.16667in,height=0.16667in]{/assets/icons/16-copy.svg}

\subsubsection{About}\label{about}

\begin{description}
\tightlist
\item[Author :]
\href{https://github.com/JoshuaLampert}{Joshua Lampert}
\item[License:]
MIT
\item[Current version:]
0.1.0
\item[Last updated:]
November 21, 2024
\item[First released:]
November 21, 2024
\item[Minimum Typst version:]
0.12.0
\item[Archive size:]
5.95 kB
\href{https://packages.typst.org/preview/clean-math-paper-0.1.0.tar.gz}{\pandocbounded{\includesvg[keepaspectratio]{/assets/icons/16-download.svg}}}
\item[Repository:]
\href{https://github.com/JoshuaLampert/clean-math-paper}{GitHub}
\item[Discipline s :]
\begin{itemize}
\tightlist
\item[]
\item
  \href{https://typst.app/universe/search/?discipline=mathematics}{Mathematics}
\item
  \href{https://typst.app/universe/search/?discipline=engineering}{Engineering}
\item
  \href{https://typst.app/universe/search/?discipline=computer-science}{Computer
  Science}
\end{itemize}
\item[Categor y :]
\begin{itemize}
\tightlist
\item[]
\item
  \pandocbounded{\includesvg[keepaspectratio]{/assets/icons/16-atom.svg}}
  \href{https://typst.app/universe/search/?category=paper}{Paper}
\end{itemize}
\end{description}

\subsubsection{Where to report issues?}\label{where-to-report-issues}

This template is a project of Joshua Lampert . Report issues on
\href{https://github.com/JoshuaLampert/clean-math-paper}{their
repository} . You can also try to ask for help with this template on the
\href{https://forum.typst.app}{Forum} .

Please report this template to the Typst team using the
\href{https://typst.app/contact}{contact form} if you believe it is a
safety hazard or infringes upon your rights.

\phantomsection\label{versions}
\subsubsection{Version history}\label{version-history}

\begin{longtable}[]{@{}ll@{}}
\toprule\noalign{}
Version & Release Date \\
\midrule\noalign{}
\endhead
\bottomrule\noalign{}
\endlastfoot
0.1.0 & November 21, 2024 \\
\end{longtable}

Typst GmbH did not create this template and cannot guarantee correct
functionality of this template or compatibility with any version of the
Typst compiler or app.


