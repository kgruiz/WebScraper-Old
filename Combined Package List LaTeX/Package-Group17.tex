\title{typst.app/universe/package/lemmify}

\phantomsection\label{banner}
\section{lemmify}\label{lemmify}

{ 0.1.6 }

Theorem typesetting library.

\phantomsection\label{readme}
Lemmify is a library for typesetting mathematical theorems in typst. It
aims to be easy to use while trying to be as flexible and idiomatic as
possible. This means that the interface might change with updates to
typst (for example if user-defined element functions are introduced).
But no functionality should be lost.

\subsection{Basic Usage}\label{basic-usage}

To get started with Lemmify, follow these steps:

\begin{enumerate}
\tightlist
\item
  Import the Lemmify library:
\end{enumerate}

\begin{Shaded}
\begin{Highlighting}[]
\NormalTok{\#import "@preview/lemmify:0.1.6": *}
\end{Highlighting}
\end{Shaded}

\begin{enumerate}
\setcounter{enumi}{1}
\tightlist
\item
  Define the default styling for a few default theorem types:
\end{enumerate}

\begin{Shaded}
\begin{Highlighting}[]
\NormalTok{\#let (}
\NormalTok{  theorem, lemma, corollary,}
\NormalTok{  remark, proposition, example,}
\NormalTok{  proof, rules: thm{-}rules}
\NormalTok{) = default{-}theorems("thm{-}group", lang: "en")}
\end{Highlighting}
\end{Shaded}

\begin{enumerate}
\setcounter{enumi}{2}
\tightlist
\item
  Apply the generated styling:
\end{enumerate}

\begin{Shaded}
\begin{Highlighting}[]
\NormalTok{\#show: thm{-}rules}
\end{Highlighting}
\end{Shaded}

\begin{enumerate}
\setcounter{enumi}{3}
\tightlist
\item
  Create theorems, lemmas, and proofs using the defined styling:
\end{enumerate}

\begin{Shaded}
\begin{Highlighting}[]
\NormalTok{\#theorem(name: "Some theorem")[}
\NormalTok{  Theorem content goes here.}
\NormalTok{]\textless{}thm\textgreater{}}

\NormalTok{\#proof[}
\NormalTok{  Complicated proof.}
\NormalTok{]\textless{}proof\textgreater{}}

\NormalTok{@proof and @thm[theorem]}
\end{Highlighting}
\end{Shaded}

\begin{enumerate}
\setcounter{enumi}{4}
\tightlist
\item
  Customize the styling further using show rules. For example, to add a
  red box around proofs:
\end{enumerate}

\begin{Shaded}
\begin{Highlighting}[]
\NormalTok{\#show thm{-}selector("thm{-}group", subgroup: "proof"): it =\textgreater{} box(}
\NormalTok{  it,}
\NormalTok{  stroke: red + 1pt,}
\NormalTok{  inset: 1em}
\NormalTok{)}
\end{Highlighting}
\end{Shaded}

The result should now look something like this:

\pandocbounded{\includegraphics[keepaspectratio]{https://github.com/Marmare314/lemmify/assets/49279081/ba5e7ed4-336d-4b1a-8470-99fa23bf5119}}

\subsection{Useful examples}\label{useful-examples}

If you do not want to reset the theorem counter on headings you can use
the \texttt{\ max-reset-level\ } parameter:

\begin{Shaded}
\begin{Highlighting}[]
\NormalTok{default{-}theorems("thm{-}group", max{-}reset{-}level: 0)}
\end{Highlighting}
\end{Shaded}

It specifies the highest level at which the counter is reset. To
manually reset the counter you can use the
\texttt{\ thm-reset-counter\ } function.

\begin{center}\rule{0.5\linewidth}{0.5pt}\end{center}

By specifying \texttt{\ numbering:\ none\ } you can create unnumbered
theorems.

\begin{Shaded}
\begin{Highlighting}[]
\NormalTok{\#example(numbering: none)[}
\NormalTok{  Some example.}
\NormalTok{]}
\end{Highlighting}
\end{Shaded}

To make all examples unnumbered you could use the following code:

\begin{Shaded}
\begin{Highlighting}[]
\NormalTok{\#let example = example.with(numbering: none)}
\end{Highlighting}
\end{Shaded}

\begin{center}\rule{0.5\linewidth}{0.5pt}\end{center}

To create other types (or subgroups) of theorems you can use the
\texttt{\ new-theorems\ } function.

\begin{Shaded}
\begin{Highlighting}[]
\NormalTok{\#let (note, rules) = new{-}theorems("thm{-}group", ("note": text(red)[Note]))}
\NormalTok{\#show: rules}
\end{Highlighting}
\end{Shaded}

If you have already defined custom styling you will notice that the
newly created theorem does not use it. You can create a dictionary to
make applying it again easier.

\begin{Shaded}
\begin{Highlighting}[]
\NormalTok{\#let my{-}styling = (}
\NormalTok{  thm{-}styling: thm{-}styling{-}simple,}
\NormalTok{  thm{-}numbering: ..,}
\NormalTok{  ref{-}styling: ..}
\NormalTok{)}

\NormalTok{\#let (note, rules) = new{-}theorems("thm{-}group", ("note": "Note), ..my{-}styling)}
\end{Highlighting}
\end{Shaded}

\begin{center}\rule{0.5\linewidth}{0.5pt}\end{center}

By varying the \texttt{\ group\ } parameter you can create independently
numbered theorems:

\begin{Shaded}
\begin{Highlighting}[]
\NormalTok{\#let (}
\NormalTok{  theorem, proof,}
\NormalTok{  rules: thm{-}rules{-}a}
\NormalTok{) = default{-}theorems("thm{-}group{-}a")}
\NormalTok{\#let (}
\NormalTok{  definition,}
\NormalTok{  rules: thm{-}rules{-}b}
\NormalTok{) = default{-}theorems("thm{-}group{-}b")}

\NormalTok{\#show: thm{-}rules{-}a}
\NormalTok{\#show: thm{-}rules{-}b}
\end{Highlighting}
\end{Shaded}

\begin{center}\rule{0.5\linewidth}{0.5pt}\end{center}

To specify parameters of the
\href{https://github.com/typst/packages/raw/main/packages/preview/lemmify/0.1.6/\#styling-parameters}{styling}
functions the \texttt{\ .with\ } function is used.

\begin{Shaded}
\begin{Highlighting}[]
\NormalTok{\#let (}
\NormalTok{  theorem,}
\NormalTok{  rules: thm{-}rules}
\NormalTok{) = default{-}theorems(}
\NormalTok{  "thm{-}group",}
\NormalTok{  thm{-}numbering: thm{-}numbering{-}heading.with(max{-}heading{-}level: 2)}
\NormalTok{)}
\end{Highlighting}
\end{Shaded}

\subsection{Example}\label{example}

\begin{Shaded}
\begin{Highlighting}[]
\NormalTok{\#import "@preview/lemmify:0.1.6": *}

\NormalTok{\#let my{-}thm{-}style(}
\NormalTok{  thm{-}type, name, number, body}
\NormalTok{) = grid(}
\NormalTok{  columns: (1fr, 3fr),}
\NormalTok{  column{-}gutter: 1em,}
\NormalTok{  stack(spacing: .5em, strong(thm{-}type), number, emph(name)),}
\NormalTok{  body}
\NormalTok{)}

\NormalTok{\#let my{-}styling = (}
\NormalTok{  thm{-}styling: my{-}thm{-}style}
\NormalTok{)}

\NormalTok{\#let (}
\NormalTok{  theorem, rules}
\NormalTok{) = default{-}theorems("thm{-}group", lang: "en", ..my{-}styling)}
\NormalTok{\#show: rules}
\NormalTok{\#show thm{-}selector("thm{-}group"): box.with(inset: 1em)}

\NormalTok{\#lorem(20)}
\NormalTok{\#theorem[}
\NormalTok{  \#lorem(40)}
\NormalTok{]}
\NormalTok{\#lorem(20)}
\NormalTok{\#theorem(name: "Some theorem")[}
\NormalTok{  \#lorem(30)}
\NormalTok{]}
\end{Highlighting}
\end{Shaded}

\pandocbounded{\includegraphics[keepaspectratio]{https://github.com/Marmare314/lemmify/assets/49279081/b3c72b3e-7e21-4acd-82bb-3d63f87ec84b}}

\subsection{Documentation}\label{documentation}

The two most important functions are:

\texttt{\ default-theorems\ } : Create a default set of theorems based
on the given language and styling.

\begin{itemize}
\tightlist
\item
  \texttt{\ group\ } : The group id.
\item
  \texttt{\ lang\ } : The language to which the theorems are adapted.
\item
  \texttt{\ thm-styling\ } , \texttt{\ thm-numbering\ } ,
  \texttt{\ ref-styling\ } : Styling parameters are explained in further
  detail in the
  \href{https://github.com/typst/packages/raw/main/packages/preview/lemmify/0.1.6/\#styling-parameters}{Styling}
  section.
\item
  \texttt{\ proof-styling\ } : Styling which is only applied to proofs.
\item
  \texttt{\ max-reset-level\ } : The highest heading level on which
  theorems are still reset.
\end{itemize}

\texttt{\ new-theorems\ } : Create custom sets of theorems with the
given styling.

\begin{itemize}
\tightlist
\item
  \texttt{\ group\ } : The group id.
\item
  \texttt{\ subgroup-map\ } : Mapping from group id to some argument.
  The simple styles use \texttt{\ thm-type\ } as the argument (ie
  “Beispiel� or “Example� for group id “example�)
\item
  \texttt{\ thm-styling\ } , \texttt{\ thm-numbering\ } ,
  \texttt{\ ref-styling\ } , \texttt{\ ref-numbering\ } : Styling which
  to apply to all subgroups.
\end{itemize}

\begin{center}\rule{0.5\linewidth}{0.5pt}\end{center}

\texttt{\ use-proof-numbering\ } : Decreases the numbering of a theorem
function by one. See
\href{https://github.com/typst/packages/raw/main/packages/preview/lemmify/0.1.6/\#styling}{Styling}
for more information.

\begin{center}\rule{0.5\linewidth}{0.5pt}\end{center}

\texttt{\ thm-selector\ } : Returns a selector for all theorems of the
specified group. If subgroup is specified, only the theorems belonging
to it will be selected.

\begin{center}\rule{0.5\linewidth}{0.5pt}\end{center}

There are also a few functions to help with resetting counters.

\texttt{\ thm-reset-counter\ } : Reset theorem group counter manually.
Returned content needs to added to the document.

\texttt{\ thm-reset-counter-heading-at\ } : Reset theorem group counter
at headings of the specified level. Returns a rule that needs to be
shown.

\texttt{\ thm-reset-counter-heading\ } : Reset theorem group counter at
headings of at most the specified level. Returns a rule that needs to be
shown.

\subsubsection{Styling parameters}\label{styling-parameters}

If possible the best way to adapt the look of theorems is to use show
rules as shown
\href{https://github.com/typst/packages/raw/main/packages/preview/lemmify/0.1.6/\#basic-usage}{above}
, but this is not always possible. For example if we wanted theorems to
start with \texttt{\ 1.1\ Theorem\ } instead of
\texttt{\ Theorem\ 1.1\ } . You can provide the following functions to
adapt the look of the theorems.

\begin{center}\rule{0.5\linewidth}{0.5pt}\end{center}

\texttt{\ thm-styling\ } : A function:
\texttt{\ (arg,\ name,\ number,\ body)\ -\textgreater{}\ content\ } ,
that allows you to define the styling for different types of theorems.
Below only the \texttt{\ arg\ } will be specified.

Pre-defined functions

\begin{itemize}
\tightlist
\item
  \texttt{\ thm-style-simple(thm-type)\ } : \textbf{thm-type num}
  \emph{(name)} body
\item
  \texttt{\ thm-style-proof(thm-type)\ } : \textbf{thm-type num}
  \emph{(name)} body â--¡
\item
  \texttt{\ thm-style-reversed(thm-type)\ } : \textbf{num thm-type}
  \emph{(name)} body
\end{itemize}

\begin{center}\rule{0.5\linewidth}{0.5pt}\end{center}

\texttt{\ thm-numbering\ } : A function:
\texttt{\ figure\ -\textgreater{}\ content\ } , that determines how
theorems are numbered.

Pre-defined functions: (Assume heading is 1.1 and theorem count is 2)

\begin{itemize}
\tightlist
\item
  \texttt{\ thm-numbering-heading\ } : 1.1.2

  \begin{itemize}
  \tightlist
  \item
    \texttt{\ max-heading-level\ } : only use the a limited number of
    subheadings. In this case if it is set to \texttt{\ 1\ } the result
    would be \texttt{\ 1.2\ } instead.
  \end{itemize}
\item
  \texttt{\ thm-numbering-linear\ } : 2
\item
  \texttt{\ thm-numbering-proof\ } : No visible content is returned, but
  the counter is reduced by 1 (so that the proof keeps the same count as
  the theorem). Useful in combination with
  \texttt{\ use-proof-numbering\ } to create theorems that reference the
  previous theorem (like proofs).
\end{itemize}

\begin{center}\rule{0.5\linewidth}{0.5pt}\end{center}

\texttt{\ ref-styling\ } : A function:
\texttt{\ (arg,\ thm-numbering,\ ref)\ -\textgreater{}\ content\ } , to
style theorem references.

Pre-defined functions:

\begin{itemize}
\tightlist
\item
  \texttt{\ thm-ref-style-simple(thm-type)\ }

  \begin{itemize}
  \tightlist
  \item
    \texttt{\ @thm\ -\textgreater{}\ thm-type\ 1.1\ }
  \item
    \texttt{\ @thm{[}custom{]}\ -\textgreater{}\ custom\ 1.1\ }
  \end{itemize}
\end{itemize}

\begin{center}\rule{0.5\linewidth}{0.5pt}\end{center}

\texttt{\ ref-numbering\ } : Same as \texttt{\ thm-numbering\ } but only
applies to the references.

\subsection{Roadmap}\label{roadmap}

\begin{itemize}
\tightlist
\item
  More pre-defined styles.

  \begin{itemize}
  \tightlist
  \item
    Referencing theorems by name.
  \end{itemize}
\item
  Support more languages.
\item
  Better documentation.
\item
  Outlining theorems.
\end{itemize}

If you are encountering any bugs, have questions or are missing
features, feel free to open an issue on
\href{https://github.com/Marmare314/lemmify}{Github} .

\subsection{Changelog}\label{changelog}

\begin{itemize}
\item
  Version 0.1.6

  \begin{itemize}
  \tightlist
  \item
    Add Portuguese translation (
    \href{https://github.com/PgBiel}{@PgBiel} )
  \item
    Add Catalan translation (
    \href{https://github.com/Eloitor}{@Eloitor} )
  \item
    Add Spanish translation (
    \href{https://github.com/mismorgano}{@mismorgano} )
  \item
    Remove extra space before empty supplements (
    \href{https://github.com/PgBiel}{@PgBiel} )
  \item
    Use ref-styling parameter of default-theorems
  \end{itemize}
\item
  Version 0.1.5

  \begin{itemize}
  \tightlist
  \item
    Add Russian translation (
    \href{https://github.com/WeetHet}{@WeetHet} )
  \end{itemize}
\item
  Version 0.1.4

  \begin{itemize}
  \tightlist
  \item
    Fix error on unnamed theorems
  \end{itemize}
\item
  Version 0.1.3

  \begin{itemize}
  \tightlist
  \item
    Allow “1.1.� numbering style by default
  \item
    Ignore unnumbered subheadings
  \item
    Add max-heading-level parameter to thm-numbering-heading
  \item
    Adapt lemmify to typst version 0.8.0
  \end{itemize}
\item
  Version 0.1.2

  \begin{itemize}
  \tightlist
  \item
    Better error message on unnumbered headings
  \item
    Add Italian translations (
    \href{https://github.com/porcaror}{@porcaror} )
  \end{itemize}
\item
  Version 0.1.1

  \begin{itemize}
  \tightlist
  \item
    Add Dutch translations (
    \href{https://github.com/BroodjeKroepoek}{@BroodjeKroepoek} )
  \item
    Add French translations ( \href{https://github.com/MDLC01}{@MDLC01}
    )
  \item
    Fix size of default styles and make them breakable
  \end{itemize}
\end{itemize}

\subsubsection{How to add}\label{how-to-add}

Copy this into your project and use the import as \texttt{\ lemmify\ }

\begin{verbatim}
#import "@preview/lemmify:0.1.6"
\end{verbatim}

\includesvg[width=0.16667in,height=0.16667in]{/assets/icons/16-copy.svg}

Check the docs for
\href{https://typst.app/docs/reference/scripting/\#packages}{more
information on how to import packages} .

\subsubsection{About}\label{about}

\begin{description}
\tightlist
\item[Author :]
Marmare314
\item[License:]
GPL-3.0-only
\item[Current version:]
0.1.6
\item[Last updated:]
August 29, 2024
\item[First released:]
July 2, 2023
\item[Archive size:]
18.2 kB
\href{https://packages.typst.org/preview/lemmify-0.1.6.tar.gz}{\pandocbounded{\includesvg[keepaspectratio]{/assets/icons/16-download.svg}}}
\item[Repository:]
\href{https://github.com/Marmare314/lemmify}{GitHub}
\end{description}

\subsubsection{Where to report issues?}\label{where-to-report-issues}

This package is a project of Marmare314 . Report issues on
\href{https://github.com/Marmare314/lemmify}{their repository} . You can
also try to ask for help with this package on the
\href{https://forum.typst.app}{Forum} .

Please report this package to the Typst team using the
\href{https://typst.app/contact}{contact form} if you believe it is a
safety hazard or infringes upon your rights.

\phantomsection\label{versions}
\subsubsection{Version history}\label{version-history}

\begin{longtable}[]{@{}ll@{}}
\toprule\noalign{}
Version & Release Date \\
\midrule\noalign{}
\endhead
\bottomrule\noalign{}
\endlastfoot
0.1.6 & August 29, 2024 \\
\href{https://typst.app/universe/package/lemmify/0.1.5/}{0.1.5} &
December 3, 2023 \\
\href{https://typst.app/universe/package/lemmify/0.1.4/}{0.1.4} &
September 26, 2023 \\
\href{https://typst.app/universe/package/lemmify/0.1.3/}{0.1.3} &
September 22, 2023 \\
\href{https://typst.app/universe/package/lemmify/0.1.2/}{0.1.2} & July
24, 2023 \\
\href{https://typst.app/universe/package/lemmify/0.1.1/}{0.1.1} & July
8, 2023 \\
\href{https://typst.app/universe/package/lemmify/0.1.0/}{0.1.0} & July
2, 2023 \\
\end{longtable}

Typst GmbH did not create this package and cannot guarantee correct
functionality of this package or compatibility with any version of the
Typst compiler or app.


\title{typst.app/universe/package/nulite}

\phantomsection\label{banner}
\section{nulite}\label{nulite}

{ 0.1.0 }

Generate charts with vegalite.

\phantomsection\label{readme}
A typst plugin to generate charts using
\href{https://vega.github.io/vega-lite/}{vegalite}

\subsection{Usage}\label{usage}

\begin{Shaded}
\begin{Highlighting}[]
\NormalTok{\#import "@preview/nulite:0.1.0" as vegalite}

\NormalTok{\#vegalite.render(}
\NormalTok{  width: 100\%,}
\NormalTok{  height: 100\%,}
\NormalTok{  zoom: 1,}
\NormalTok{  json("spec.json")}
\NormalTok{  )}
\end{Highlighting}
\end{Shaded}

\pandocbounded{\includegraphics[keepaspectratio]{https://github.com/typst/packages/raw/main/packages/preview/nulite/0.1.0/examples/image.png}}

The module exports a single function, \texttt{\ render\ } with four
arguments

\begin{itemize}
\tightlist
\item
  \texttt{\ width\ } : Width of the chart in percent of the
  container’s width
\item
  \texttt{\ height\ } : Height of the chart in percent of the
  container’s height
\item
  \texttt{\ zoom\ } : Zoom factor applied to the SVG. This mainly
  affects the sizing of text in relation to the graphical elements.
\item
  \texttt{\ spec\ } :
  \href{https://vega.github.io/vega-lite/docs/spec.html}{Vegalite
  specification}
\end{itemize}

\subsection{Compatibility}\label{compatibility}

This plugin uses vegalite v5.21 and vega v5.30.

The following features of vegalite are \textbf{not supported} :

\begin{itemize}
\tightlist
\item
  Setting \texttt{\ width\ } and \texttt{\ height\ } in the spec. These
  values should be provided as arguments to \texttt{\ render\ } . If
  \texttt{\ width\ } or \texttt{\ height\ } are included in the spec
  then they will be ignored.
\item
  Loading data with the \texttt{\ url\ } property. Attempting to do this
  will result in an error while trying to compile the \texttt{\ typst\ }
  document. All data should be provided as part of the spec itself
  (inline).
\item
  Interactive charts and tooltips.
\end{itemize}

\subsubsection{How to add}\label{how-to-add}

Copy this into your project and use the import as \texttt{\ nulite\ }

\begin{verbatim}
#import "@preview/nulite:0.1.0"
\end{verbatim}

\includesvg[width=0.16667in,height=0.16667in]{/assets/icons/16-copy.svg}

Check the docs for
\href{https://typst.app/docs/reference/scripting/\#packages}{more
information on how to import packages} .

\subsubsection{About}\label{about}

\begin{description}
\tightlist
\item[Author :]
j-mueller
\item[License:]
MIT
\item[Current version:]
0.1.0
\item[Last updated:]
September 30, 2024
\item[First released:]
September 30, 2024
\item[Minimum Typst version:]
0.11.1
\item[Archive size:]
686 kB
\href{https://packages.typst.org/preview/nulite-0.1.0.tar.gz}{\pandocbounded{\includesvg[keepaspectratio]{/assets/icons/16-download.svg}}}
\item[Repository:]
\href{https://github.com/j-mueller/typst-vegalite}{GitHub}
\item[Discipline :]
\begin{itemize}
\tightlist
\item[]
\item
  \href{https://typst.app/universe/search/?discipline=mathematics}{Mathematics}
\end{itemize}
\item[Categor ies :]
\begin{itemize}
\tightlist
\item[]
\item
  \pandocbounded{\includesvg[keepaspectratio]{/assets/icons/16-chart.svg}}
  \href{https://typst.app/universe/search/?category=visualization}{Visualization}
\item
  \pandocbounded{\includesvg[keepaspectratio]{/assets/icons/16-integration.svg}}
  \href{https://typst.app/universe/search/?category=integration}{Integration}
\end{itemize}
\end{description}

\subsubsection{Where to report issues?}\label{where-to-report-issues}

This package is a project of j-mueller . Report issues on
\href{https://github.com/j-mueller/typst-vegalite}{their repository} .
You can also try to ask for help with this package on the
\href{https://forum.typst.app}{Forum} .

Please report this package to the Typst team using the
\href{https://typst.app/contact}{contact form} if you believe it is a
safety hazard or infringes upon your rights.

\phantomsection\label{versions}
\subsubsection{Version history}\label{version-history}

\begin{longtable}[]{@{}ll@{}}
\toprule\noalign{}
Version & Release Date \\
\midrule\noalign{}
\endhead
\bottomrule\noalign{}
\endlastfoot
0.1.0 & September 30, 2024 \\
\end{longtable}

Typst GmbH did not create this package and cannot guarantee correct
functionality of this package or compatibility with any version of the
Typst compiler or app.


\title{typst.app/universe/package/acrostiche}

\phantomsection\label{banner}
\section{acrostiche}\label{acrostiche}

{ 0.4.1 }

Manage acronyms and their definitions in Typst.

\phantomsection\label{readme}
Manages acronyms so you don’t have to.

\subsection{Quick Start}\label{quick-start}

\begin{verbatim}
#import "@preview/acrostiche:0.4.0": *

#init-acronyms((
  "WTP": ("Wonderful Typst Package","Wonderful Typst Packages"),
))

Acrostiche is a #acr("WTP")! This #acr("WTP") enables easy acronyms manipulation.

Its main features are auto-expansion of the first occurence, global or selective expansion reset #reset-all-acronyms(), implicit or manual plural form support (there may be multiple #acrpl("WTP")), and customizable index printing. Have Fun!
\end{verbatim}

\subsection{Usage}\label{usage}

The main goal of Acrostiche is to keep track of which acronyms to
define.

\subsubsection{Define acronyms}\label{define-acronyms}

All acronyms used with Acrostiche must be defined in a dictionary passed
to the \texttt{\ init-acronyms\ } function. There are two possible forms
for the definition, depending on the required features.

\paragraph{Simple Definitions}\label{simple-definitions}

For a quick and easy definion, you can use the acronym to display as the
key and an array of one or more strings as the singular and plural
versions of the expanded meaning of the acronym.

\begin{verbatim}
#init-acronyms((
  "SDA": ("Simply Defined Acronym","Simply Defined Acronyms"),
  "ASDA": ("Another Simply Defined Acronym","Another Simply Defined Acronyms"),
))
\end{verbatim}

If there is only a singular version of the definition, the array
contains only one value. If there are both singular and plural versions,
define the definition as an array where the first item is the singular
definition and the second item is the plural.

\paragraph{Advanced Definitions}\label{advanced-definitions}

If you find yourself needing more flexibility when defining the
acronyms, you can pass a dictionary for each acronym. The expected keys
are: \texttt{\ short\ } the singular short form to display,
\texttt{\ short-pl\ } the plural short form, \texttt{\ long\ } singular
long (expanded) form to display, and \texttt{\ long-pl\ } the plural
long form. The main benefit of this definition is to use a separate key
for calling the acronym, useful when acronyms are long and tedious to
write.

\begin{verbatim}
#init-acronyms((
  "la": (
short: "LATATW",
long: "Long And Tedious Acronym To Write",
short-pl: "LATAsTW",
long-pl: "Long And Tedious Acronyms To Write"),
))
\end{verbatim}

Any other keys than these will be discarded.

\subsubsection{Call Acrostiche
functions}\label{call-acrostiche-functions}

Once the acronyms are defined, you can use them in the text with the
\texttt{\ \#acr(...)\ } function. The argument is the acronym as a
string (for example, “BIOS�). On the first call of the function, it
prints the acronym with its definition (for example, “Basic
Input/Output System (BIOS)�). On the next calls, it prints only the
acronym.

To get the plural version of the acronym, you can use the
\texttt{\ \#acrpl(...)\ } function that adds an ‘s’ after the
acronym. If a plural version of the definition is provided, it will be
used if the first use of the acronym is plural. Otherwise, the singular
version is used, and a trailing ‘s’ is added.

To intentionally print the full version of the acronym (definition +
acronym, as for the first instance), without affecting the state, you
can use the \texttt{\ \#acrfull(...)\ } function. For the plural
version, use the \texttt{\ \#acrfullpl(...)\ } function. Both functions
have shortcuts with \texttt{\ \#acrf(...)\ } and
\texttt{\ \#acrfpl(...)\ } .

At any point in the document, you can reset acronyms with the functions
\texttt{\ \#reset-acronym(...)\ } (for a single acronym) or
\texttt{\ reset-all-acronyms()\ } (to reset all acronyms). After a
reset, the next use of the acronym is expanded. Both functions have
shortcuts with \texttt{\ \#racr(...)\ } and \texttt{\ \#raacr(...)\ } .

You can also print an index of all acronyms used in the document with
the \texttt{\ \#print-index()\ } function. The index is printed as a
section for which you can choose the heading level, the numbering, and
the outline parameters (with respectively the \texttt{\ level:\ int\ } ,
\texttt{\ numbering:\ none\ \textbar{}\ string\ \textbar{}\ function\ }
, and \texttt{\ outlined:\ bool\ } parameters). You can also choose
their order with the \texttt{\ sorted:\ string\ } parameter that accepts
either an empty string (print in the order they are defined), “up�
(print in ascending alphabetical order), or “down� (print in
descending alphabetical order). By default, the index contains all the
acronyms you defined. You can choose to only display acronyms that are
actually used in the document by passing \texttt{\ used-only:\ true\ }
to the function. Warning, the detection of used acronym uses the states
at the end of the document. Thus, if you reset an acronym and do not use
it again until the end, it will not appear in the index. You can use the
\texttt{\ title:\ string\ } parameter to change the name of the heading
for the index section. The default value is “Acronyms Index�.
Passing an empty string for \texttt{\ title\ } results in the index
having no heading (i.e., no section for the index). You can customize
the string displayed after the acronym in the list with the
\texttt{\ delimiter:\ ":"\ } parameter. To adjust the spacing between
the acronyms adjust the
\texttt{\ row-gutter:\ auto\ \textbar{}\ int\ \textbar{}\ relative\ \textbar{}\ fraction\ \textbar{}\ array\ }
parameter, the default is \texttt{\ 2pt\ } .

Finally, you can call the \texttt{\ \#display-def(...)\ } function to
display the definition of an acronym. Set the \texttt{\ plural\ }
parameter to true to get the plural version.

\subsubsection{Functions Summary:}\label{functions-summary}

\begin{longtable}[]{@{}ll@{}}
\toprule\noalign{}
\textbf{Function} & \textbf{Description} \\
\midrule\noalign{}
\endhead
\bottomrule\noalign{}
\endlastfoot
\textbf{\#init-acronyms(…)} & Initializes the acronyms by defining
them in a dictionary where the keys are acronyms and the values are
definitions. \\
\textbf{\#acr(…)} & Prints the acronym with its definition on the
first call, then just the acronym in subsequent calls. \\
\textbf{\#acrpl(…)} & Prints the plural version of the acronym. Uses
plural definition if available, otherwise adds an ‘s’ to the
acronym. \\
\textbf{\#acrfull(…)} & Displays the full (long) version of the
acronym without affecting the state or tracking its usage. \\
\textbf{\#acrfullpl(…)} & Displays the full plural version of the
acronym without affecting the state or tracking its usage. \\
\textbf{\#reset-acronym(…)} & Resets a single acronym so the next
usage will include its definition again. \\
\textbf{\#reset-all-acronyms()} & Resets all acronyms so the next usage
will include their definitions again. \\
\textbf{\#print-index(…)} & Prints an index of all acronyms used, with
customizable heading level, order, and display parameters. \\
\textbf{\#display-def(…)} & Displays the definition of an acronym. Use
\texttt{\ plural:\ true\ } to display the plural version of the
definition. \\
\textbf{racr, raacr, acrf, acrfpl} & Shortcuts names for respectively
\texttt{\ reset-acronym\ } , \texttt{\ reset-all-acronyms\ } ,
\texttt{\ acrfull\ } , and \texttt{\ acrfullpl\ } . \\
\end{longtable}

\subsection{Advanced Definitions}\label{advanced-definitions-1}

This is a bit of a hacky feature coming from pure serendipity. There is
no enforcement of the type of the definitions. Most users would
naturally use strings as definitions, but any other content is
acceptable. For example, you set your definition to a content block with
rainbow-fille text, or even an image. The rainbow text is kinda cool
because the gradient depend on the position in the page so depending on
the position of first use the acronym will have a pseudo-random color.

If you use anything else than string for the definition, do not forget
the trailing comma to force the definition to be an array (an array of a
single element is not an array in Typst at the time of writing this). I
cannot guarantee that arbitrary content will remain available in future
versions but I will do my best to keep it as it is kinda cool. If you
find cool uses, please reach out to show me!

\begin{quote}
Yes it is posible to build nest/recursive acronyms definitions. If you
point to another acronym, it all works fine. If you point to the same
acronym, you obviously create a recursive situation, and it fails. It
will not converge, and the compiler will warn you and will panic. Be
nice to the compiler, don\textquotesingle t throw recursive traps.
\end{quote}

Here is a minimal working example of funky acronyms:

\begin{verbatim}
#import "@preview/acrostiche:0.4.0": *                                                           
#init-acronyms((
  "RFA": ([#text(fill: gradient.linear(..color.map.rainbow))[Rainbow Filled Acronym]],),                                                             
  "NA": ([Nested #acr("RFA") Acronym],)
))
#acr("NA")
\end{verbatim}

\subsection{Possible Errors:}\label{possible-errors}

\begin{itemize}
\tightlist
\item
  If an acronym is not defined, an error will tell you which one is
  causing the error. Simply add it to the dictionary or check the
  spelling.
\item
  \texttt{\ display-def\ } leverages the state \texttt{\ display\ }
  function and only works if the return value is actually printed in the
  document. For more information on states, see the Typst documentation
  on states.
\item
  Acrostiche uses a state named \texttt{\ acronyms\ } to keep track of
  the definitions and usage. If you redefined this state or use it
  manually in your document, unexpacted behaviour might happen.
\end{itemize}

Thank you to the contributors: \textbf{caemor} , \textbf{AurelWeinhold}
, \textbf{daniel-eder} , \textbf{iostapyshyn} .

If you notice any bug or want to contribute a new feature, please open
an issue or a merge request on the fork
\href{https://github.com/Grisely/packages}{Grisely/packages}

\subsubsection{How to add}\label{how-to-add}

Copy this into your project and use the import as
\texttt{\ acrostiche\ }

\begin{verbatim}
#import "@preview/acrostiche:0.4.1"
\end{verbatim}

\includesvg[width=0.16667in,height=0.16667in]{/assets/icons/16-copy.svg}

Check the docs for
\href{https://typst.app/docs/reference/scripting/\#packages}{more
information on how to import packages} .

\subsubsection{About}\label{about}

\begin{description}
\tightlist
\item[Author :]
Grizzly
\item[License:]
MIT
\item[Current version:]
0.4.1
\item[Last updated:]
November 21, 2024
\item[First released:]
July 6, 2023
\item[Archive size:]
6.52 kB
\href{https://packages.typst.org/preview/acrostiche-0.4.1.tar.gz}{\pandocbounded{\includesvg[keepaspectratio]{/assets/icons/16-download.svg}}}
\item[Repository:]
\href{https://github.com/Grisely/packages}{GitHub}
\item[Categor ies :]
\begin{itemize}
\tightlist
\item[]
\item
  \pandocbounded{\includesvg[keepaspectratio]{/assets/icons/16-hammer.svg}}
  \href{https://typst.app/universe/search/?category=utility}{Utility}
\item
  \pandocbounded{\includesvg[keepaspectratio]{/assets/icons/16-list-unordered.svg}}
  \href{https://typst.app/universe/search/?category=model}{Model}
\end{itemize}
\end{description}

\subsubsection{Where to report issues?}\label{where-to-report-issues}

This package is a project of Grizzly . Report issues on
\href{https://github.com/Grisely/packages}{their repository} . You can
also try to ask for help with this package on the
\href{https://forum.typst.app}{Forum} .

Please report this package to the Typst team using the
\href{https://typst.app/contact}{contact form} if you believe it is a
safety hazard or infringes upon your rights.

\phantomsection\label{versions}
\subsubsection{Version history}\label{version-history}

\begin{longtable}[]{@{}ll@{}}
\toprule\noalign{}
Version & Release Date \\
\midrule\noalign{}
\endhead
\bottomrule\noalign{}
\endlastfoot
0.4.1 & November 21, 2024 \\
\href{https://typst.app/universe/package/acrostiche/0.4.0/}{0.4.0} &
October 31, 2024 \\
\href{https://typst.app/universe/package/acrostiche/0.3.5/}{0.3.5} &
October 28, 2024 \\
\href{https://typst.app/universe/package/acrostiche/0.3.4/}{0.3.4} &
October 22, 2024 \\
\href{https://typst.app/universe/package/acrostiche/0.3.3/}{0.3.3} &
September 24, 2024 \\
\href{https://typst.app/universe/package/acrostiche/0.3.2/}{0.3.2} &
July 10, 2024 \\
\href{https://typst.app/universe/package/acrostiche/0.3.1/}{0.3.1} &
January 6, 2024 \\
\href{https://typst.app/universe/package/acrostiche/0.3.0/}{0.3.0} &
August 22, 2023 \\
\href{https://typst.app/universe/package/acrostiche/0.2.0/}{0.2.0} &
July 8, 2023 \\
\href{https://typst.app/universe/package/acrostiche/0.1.0/}{0.1.0} &
July 6, 2023 \\
\end{longtable}

Typst GmbH did not create this package and cannot guarantee correct
functionality of this package or compatibility with any version of the
Typst compiler or app.


\title{typst.app/universe/package/haw-hamburg}

\phantomsection\label{banner}
\section{haw-hamburg}\label{haw-hamburg}

{ 0.3.1 }

Unofficial template for writing a report or thesis in the HAW Hamburg
department of Computer Science design.

\phantomsection\label{readme}
This is an \textbf{\texttt{\ unofficial\ }} template for writing a
report or thesis in the \texttt{\ HAW\ Hamburg\ } department of
\texttt{\ Computer\ Science\ } design using
\href{https://github.com/typst/typst}{Typst} .

\subsection{Required Fonts}\label{required-fonts}

To correctly render this template please make sure that the
\texttt{\ New\ Computer\ Modern\ } font is installed on your system.

\subsection{Usage}\label{usage}

To use this package just add the following code to your
\href{https://github.com/typst/typst}{Typst} document:

\subsubsection{Report}\label{report}

\begin{Shaded}
\begin{Highlighting}[]
\NormalTok{\#import "@preview/haw{-}hamburg:0.3.0": report}

\NormalTok{\#show: report.with(}
\NormalTok{  language: "en",}
\NormalTok{  title: "Example title",}
\NormalTok{  author:"Example author",}
\NormalTok{  faculty: "Engineering and Computer Science",}
\NormalTok{  department: "Computer Science",}
\NormalTok{  include{-}declaration{-}of{-}independent{-}processing: true,}
\NormalTok{)}
\end{Highlighting}
\end{Shaded}

\subsubsection{Bachelor Thesis}\label{bachelor-thesis}

\begin{Shaded}
\begin{Highlighting}[]
\NormalTok{\#import "@preview/haw{-}hamburg:0.3.0": bachelor{-}thesis}

\NormalTok{\#show: bachelor{-}thesis.with(}
\NormalTok{  language: "en",}

\NormalTok{  title{-}de: "Beispiel Titel",}
\NormalTok{  keywords{-}de: ("Stichwort", "Wichtig", "Super"),}
\NormalTok{  abstract{-}de: "Beispiel Zusammenfassung",}

\NormalTok{  title{-}en: "Example title",}
\NormalTok{  keywords{-}en:  ("Keyword", "Important", "Super"),}
\NormalTok{  abstract{-}en: "Example abstract",}

\NormalTok{  author: "Example author",}
\NormalTok{  faculty: "Engineering and Computer Science",}
\NormalTok{  department: "Computer Science",}
\NormalTok{  study{-}course: "Bachelor of Science Informatik Technischer Systeme",}
\NormalTok{  supervisors: ("Prof. Dr. Example", "Prof. Dr. Example"),}
\NormalTok{  submission{-}date: datetime(year: 1948, month: 12, day: 10),}
\NormalTok{  include{-}declaration{-}of{-}independent{-}processing: true,}
\NormalTok{)}
\end{Highlighting}
\end{Shaded}

\subsubsection{Master Thesis}\label{master-thesis}

\begin{Shaded}
\begin{Highlighting}[]
\NormalTok{\#import "@preview/haw{-}hamburg:0.3.0": master{-}thesis}

\NormalTok{\#show: master{-}thesis.with(}
\NormalTok{  language: "en",}

\NormalTok{  title{-}de: "Beispiel Titel",}
\NormalTok{  keywords{-}de: ("Stichwort", "Wichtig", "Super"),}
\NormalTok{  abstract{-}de: "Beispiel Zusammenfassung",}

\NormalTok{  title{-}en: "Example title",}
\NormalTok{  keywords{-}en:  ("Keyword", "Important", "Super"),}
\NormalTok{  abstract{-}en: "Example abstract",}

\NormalTok{  author: "The Computer",}
\NormalTok{  faculty: "Engineering and Computer Science",}
\NormalTok{  department: "Computer Science",}
\NormalTok{  study{-}course: "Master of Science Computer Science",}
\NormalTok{  supervisors: ("Prof. Dr. Example", "Prof. Dr. Example"),}
\NormalTok{  submission{-}date: datetime(year: 1948, month: 12, day: 10),}
\NormalTok{  include{-}declaration{-}of{-}independent{-}processing: true,}
\NormalTok{)}
\end{Highlighting}
\end{Shaded}

\subsection{How to Compile}\label{how-to-compile}

This project contains an example setup that splits individual chapters
into different files.\\
This can cause problems when using references etc.\\
These problems can be avoided by following these steps:

\begin{itemize}
\tightlist
\item
  Make sure to always compile your \texttt{\ main.typ\ } file which
  includes all of your chapters for references to work correctly.
\item
  VSCode:

  \begin{itemize}
  \tightlist
  \item
    Install the
    \href{https://marketplace.visualstudio.com/items?itemName=myriad-dreamin.tinymist}{Tinymist
    Typst} extension.
  \item
    Make sure to start the \texttt{\ PDF\ } or
    \texttt{\ Live\ Preview\ } only from within your
    \texttt{\ main.typ\ } file.
  \item
    If problems occur it usually helps to close the preview and restart
    it from your \texttt{\ main.typ\ } file.
  \end{itemize}
\end{itemize}

\subsection{Examples}\label{examples}

Examples can be found inside of the
\href{https://github.com/LasseRosenow/HAW-Hamburg-Typst-Template/tree/main/examples}{examples}
directory

\begin{itemize}
\tightlist
\item
  For Bachelor theses see:
  \href{https://github.com/LasseRosenow/HAW-Hamburg-Typst-Template/tree/main/examples/bachelor-thesis}{Bachelor
  thesis example}
\item
  For Master theses see:
  \href{https://github.com/LasseRosenow/HAW-Hamburg-Typst-Template/tree/main/examples/master-thesis}{Master
  thesis example}
\item
  For reports see:
  \href{https://github.com/LasseRosenow/HAW-Hamburg-Typst-Template/tree/main/examples/report}{Report
  example}
\end{itemize}

\subsubsection{How to add}\label{how-to-add}

Copy this into your project and use the import as
\texttt{\ haw-hamburg\ }

\begin{verbatim}
#import "@preview/haw-hamburg:0.3.1"
\end{verbatim}

\includesvg[width=0.16667in,height=0.16667in]{/assets/icons/16-copy.svg}

Check the docs for
\href{https://typst.app/docs/reference/scripting/\#packages}{more
information on how to import packages} .

\subsubsection{About}\label{about}

\begin{description}
\tightlist
\item[Author :]
Lasse Rosenow
\item[License:]
MIT
\item[Current version:]
0.3.1
\item[Last updated:]
November 13, 2024
\item[First released:]
September 26, 2024
\item[Archive size:]
12.2 kB
\href{https://packages.typst.org/preview/haw-hamburg-0.3.1.tar.gz}{\pandocbounded{\includesvg[keepaspectratio]{/assets/icons/16-download.svg}}}
\item[Repository:]
\href{https://github.com/LasseRosenow/HAW-Hamburg-Typst-Template}{GitHub}
\item[Categor ies :]
\begin{itemize}
\tightlist
\item[]
\item
  \pandocbounded{\includesvg[keepaspectratio]{/assets/icons/16-speak.svg}}
  \href{https://typst.app/universe/search/?category=report}{Report}
\item
  \pandocbounded{\includesvg[keepaspectratio]{/assets/icons/16-mortarboard.svg}}
  \href{https://typst.app/universe/search/?category=thesis}{Thesis}
\end{itemize}
\end{description}

\subsubsection{Where to report issues?}\label{where-to-report-issues}

This package is a project of Lasse Rosenow . Report issues on
\href{https://github.com/LasseRosenow/HAW-Hamburg-Typst-Template}{their
repository} . You can also try to ask for help with this package on the
\href{https://forum.typst.app}{Forum} .

Please report this package to the Typst team using the
\href{https://typst.app/contact}{contact form} if you believe it is a
safety hazard or infringes upon your rights.

\phantomsection\label{versions}
\subsubsection{Version history}\label{version-history}

\begin{longtable}[]{@{}ll@{}}
\toprule\noalign{}
Version & Release Date \\
\midrule\noalign{}
\endhead
\bottomrule\noalign{}
\endlastfoot
0.3.1 & November 13, 2024 \\
\href{https://typst.app/universe/package/haw-hamburg/0.3.0/}{0.3.0} &
October 14, 2024 \\
\href{https://typst.app/universe/package/haw-hamburg/0.2.0/}{0.2.0} &
October 9, 2024 \\
\href{https://typst.app/universe/package/haw-hamburg/0.1.0/}{0.1.0} &
September 26, 2024 \\
\end{longtable}

Typst GmbH did not create this package and cannot guarantee correct
functionality of this package or compatibility with any version of the
Typst compiler or app.


\title{typst.app/universe/package/light-cv}

\phantomsection\label{banner}
\phantomsection\label{template-thumbnail}
\pandocbounded{\includegraphics[keepaspectratio]{https://packages.typst.org/preview/thumbnails/light-cv-0.1.1-small.webp}}

\section{light-cv}\label{light-cv}

{ 0.1.1 }

Minimalistic CV template for your own CV. Please install the font
awesome fonts on your system before using the template.

\href{/app?template=light-cv&version=0.1.1}{Create project in app}

\phantomsection\label{readme}
This is my CV template written in Typst. You can find a two example CVs
in this repository as PDFs:

\begin{itemize}
\tightlist
\item
  \href{https://github.com/AnsgarLichter/light-cv/blob/main/cv-de.pdf}{German
  CV}
\item
  \href{https://github.com/AnsgarLichter/light-cv/blob/main/cv-en.pdf}{English
  CV}
\end{itemize}

\subsection{Setup}\label{setup}

To use the CV you have to install the font awesome fonts for the icons
to work. Please refer to the intstructons of the font awesome package
itself. You can find these on: -
\href{https://typst.app/universe/package/fontawesome}{Typst Universe} -
\href{https://github.com/duskmoon314/typst-fontawesome}{GitHub} .

\subsection{Functions}\label{functions}

\begin{enumerate}
\item
  \texttt{\ header\ } : Creates a page haeder including your name,
  current job title or any other sub title, socials and profile picture

  \begin{itemize}
  \tightlist
  \item
    \texttt{\ full-name\ } : your full name
  \item
    \texttt{\ job-title\ } : your current job title rendered below your
    name
  \item
    \texttt{\ socials\ } : array containing all socials. Every social
    must have the following properties: \texttt{\ icon\ } ,
    \texttt{\ link\ } and \texttt{\ text\ }
  \item
    \texttt{\ profile-picture\ } : path to your profile picture
  \end{itemize}
\item
  \texttt{\ section\ } : Creates a new section, e. g.
  \texttt{\ Professional\ Experience\ } or \texttt{\ Education\ }

  \begin{itemize}
  \tightlist
  \item
    \texttt{\ title\ } : section’s title
  \end{itemize}
\item
  \texttt{\ entry\ } : Adds an entry / item to the current section

  \begin{itemize}
  \tightlist
  \item
    \texttt{\ title\ } : the entry’s title, e. g. your job title
  \item
    \texttt{\ company-or-university\ } : the name of the institution
    which you were at, e. g. company or university
  \item
    \texttt{\ date\ } : start and end date of this entry, e. g. 2020 -
    2022
  \item
    \texttt{\ location\ } : describes where you worked, e. g. London, UK
  \item
    \texttt{\ logo\ } : path to the logo of this entry
  \item
    ``description`: description what you have done - normally supplied
    as a list
  \end{itemize}
\end{enumerate}

\subsection{Customization}\label{customization}

In the folder \texttt{\ settings\ } you will a file
\texttt{\ styles.typ\ } which includes all used styles. You can
customize them as you want to.

\subsection{Multi Language Support}\label{multi-language-support}

If you want to add a new language, copy the \texttt{\ cv-en.typ\ } and
rename it. Afterwards you can adapt the text correspondingly. Maybe I
will introduce i18n in the future.

\subsection{Insipration}\label{insipration}

A big thanks to
\href{https://github.com/mintyfrankie/brilliant-CV}{brilliant-CV} as
this project was an inspiraton for my CV and for the scripting.

\subsection{Questions \& Issues}\label{questions-issues}

If you have questions, plase create a
\href{https://github.com/AnsgarLichter/light-cv/discussions}{discussion}
. If you have an issue, please create an
\href{https://github.com/AnsgarLichter/light-cv/issues}{issue} .

\href{/app?template=light-cv&version=0.1.1}{Create project in app}

\subsubsection{How to use}\label{how-to-use}

Click the button above to create a new project using this template in
the Typst app.

You can also use the Typst CLI to start a new project on your computer
using this command:

\begin{verbatim}
typst init @preview/light-cv:0.1.1
\end{verbatim}

\includesvg[width=0.16667in,height=0.16667in]{/assets/icons/16-copy.svg}

\subsubsection{About}\label{about}

\begin{description}
\tightlist
\item[Author :]
Ansgar Lichter
\item[License:]
MIT
\item[Current version:]
0.1.1
\item[Last updated:]
May 6, 2024
\item[First released:]
April 17, 2024
\item[Archive size:]
414 kB
\href{https://packages.typst.org/preview/light-cv-0.1.1.tar.gz}{\pandocbounded{\includesvg[keepaspectratio]{/assets/icons/16-download.svg}}}
\item[Repository:]
\href{https://github.com/AnsgarLichter/cv-typst-template}{GitHub}
\item[Categor y :]
\begin{itemize}
\tightlist
\item[]
\item
  \pandocbounded{\includesvg[keepaspectratio]{/assets/icons/16-user.svg}}
  \href{https://typst.app/universe/search/?category=cv}{CV}
\end{itemize}
\end{description}

\subsubsection{Where to report issues?}\label{where-to-report-issues}

This template is a project of Ansgar Lichter . Report issues on
\href{https://github.com/AnsgarLichter/cv-typst-template}{their
repository} . You can also try to ask for help with this template on the
\href{https://forum.typst.app}{Forum} .

Please report this template to the Typst team using the
\href{https://typst.app/contact}{contact form} if you believe it is a
safety hazard or infringes upon your rights.

\phantomsection\label{versions}
\subsubsection{Version history}\label{version-history}

\begin{longtable}[]{@{}ll@{}}
\toprule\noalign{}
Version & Release Date \\
\midrule\noalign{}
\endhead
\bottomrule\noalign{}
\endlastfoot
0.1.1 & May 6, 2024 \\
\href{https://typst.app/universe/package/light-cv/0.1.0/}{0.1.0} & April
17, 2024 \\
\end{longtable}

Typst GmbH did not create this template and cannot guarantee correct
functionality of this template or compatibility with any version of the
Typst compiler or app.


\title{typst.app/universe/package/glossy}

\phantomsection\label{banner}
\section{glossy}\label{glossy}

{ 0.2.0 }

A very simple glossary system with easily customizable output.

\phantomsection\label{readme}
This package provides utilities to manage and render glossaries within
documents. It includes functions to define and use glossary terms, track
their usage, and generate a glossary list with references to where terms
are used in the document.

\pandocbounded{\includegraphics[keepaspectratio]{https://github.com/typst/packages/raw/main/packages/preview/glossy/0.2.0/thumbnail.png}}

\subsection{Motivation}\label{motivation}

Glossy is heavily inspired by
\href{https://typst.app/universe/package/glossarium}{glossarium} , with
a few key different goals:

\begin{enumerate}
\tightlist
\item
  Provide a simple interface which allows for complete control over
  glossary display. To do this, \texttt{\ glossy\ } ’s
  \texttt{\ \#glossary()\ } function accepts a theme parameter. The goal
  here is to separate presentation and logic.
\item
  Simplify the user interface as much as possible. Glossy has exactly
  two exports, \texttt{\ init-glossary\ } and \texttt{\ glossary\ } .
\item
  Double-down on \texttt{\ glossy\ } ’s excellent \texttt{\ @term\ }
  reference approach, completely eliminating the need to make any calls
  to \texttt{\ gls()\ } and friends.
\item
  Mimic established patterns and best practices. For example,
  \texttt{\ glossy\ } ’s \texttt{\ \#glossary()\ } function is
  intentionally similar (in naming and parameters) to \texttt{\ typst\ }
  ’s built-in \texttt{\ \#bibliography\ } , to the degree possible.
\item
  Simplify the implementation. The \texttt{\ glossy\ } code is
  significantly shorter and easier to understand.
\end{enumerate}

\subsection{Features}\label{features}

\begin{itemize}
\tightlist
\item
  Define glossary terms with short and long forms, descriptions, and
  grouping
\item
  Automatically tracks term usage in the document through
  \texttt{\ @labels\ }
\item
  Supports modifiers to adjust how terms are displayed (capitalize,
  pluralize, etc.)
\item
  Generates a formatted glossary section with backlinks to term
  occurrences
\item
  Customizable themes for rendering glossary sections, groups, and
  entries
\item
  Automatic pluralization of terms with custom override options
\item
  Page number references back to term usage locations
\end{itemize}

\subsection{Usage}\label{usage}

\subsubsection{Import the package}\label{import-the-package}

\begin{Shaded}
\begin{Highlighting}[]
\NormalTok{\#import "@preview/glossy:0.2.0": *}
\end{Highlighting}
\end{Shaded}

\subsubsection{Defining Glossary Terms}\label{defining-glossary-terms}

Use the \texttt{\ init-glossary\ } function to initialize glossary
entries:

\begin{Shaded}
\begin{Highlighting}[]
\NormalTok{\#let myGlossary = (}
\NormalTok{    html: (}
\NormalTok{      short: "HTML",}
\NormalTok{      long: "Hypertext Markup Language",}
\NormalTok{      description: "A standard language for creating web pages",}
\NormalTok{      group: "Web"}
\NormalTok{    ),}
\NormalTok{    css: (}
\NormalTok{      short: "CSS",}
\NormalTok{      long: "Cascading Style Sheets",}
\NormalTok{      description: "A stylesheet language used for describing the presentation of a document",}
\NormalTok{      group: "Web"}
\NormalTok{    ),}
\NormalTok{    tps: (}
\NormalTok{      short: "TPS",}
\NormalTok{      long: "test procedure specification",}
\NormalTok{      description: "A formal document describing test steps and expected results",}
\NormalTok{      // Optional: Override automatic pluralization}
\NormalTok{      plural: "TPSes",}
\NormalTok{      longplural: "test procedure specifications"}
\NormalTok{    )}
\NormalTok{)}

\NormalTok{\#show: init{-}glossary.with(myGlossary)}
\end{Highlighting}
\end{Shaded}

Each glossary entry supports the following fields:

\begin{itemize}
\tightlist
\item
  \texttt{\ short\ } (required): Short form of the term
\item
  \texttt{\ long\ } (optional): Long form of the term
\item
  \texttt{\ description\ } (optional): Term description (often a
  definition)
\item
  \texttt{\ group\ } (optional): Category grouping
\item
  \texttt{\ plural\ } (optional): Override automatic pluralization of
  short form
\item
  \texttt{\ longplural\ } (optional): Override automatic pluralization
  of long form
\end{itemize}

You can also load glossary entries from a data file using \#yaml(),
\#json(), or similar.

For example, the above glossary could be in this YAML file:

\begin{Shaded}
\begin{Highlighting}[]
\FunctionTok{html}\KeywordTok{:}
\AttributeTok{  }\FunctionTok{short}\KeywordTok{:}\AttributeTok{ HTML}
\AttributeTok{  }\FunctionTok{long}\KeywordTok{:}\AttributeTok{ Hypertext Markup Language}
\AttributeTok{  }\FunctionTok{description}\KeywordTok{:}\AttributeTok{ A standard language for creating web pages}
\AttributeTok{  }\FunctionTok{group}\KeywordTok{:}\AttributeTok{ Web}

\FunctionTok{css}\KeywordTok{:}
\AttributeTok{  }\FunctionTok{short}\KeywordTok{:}\AttributeTok{ CSS}
\AttributeTok{  }\FunctionTok{long}\KeywordTok{:}\AttributeTok{ Cascading Style Sheets}
\AttributeTok{  }\FunctionTok{description}\KeywordTok{:}\AttributeTok{ A stylesheet language used for describing the presentation of a document}
\AttributeTok{  }\FunctionTok{group}\KeywordTok{:}\AttributeTok{ Web}

\FunctionTok{tps}\KeywordTok{:}
\AttributeTok{  }\FunctionTok{short}\KeywordTok{:}\AttributeTok{ TPS}
\AttributeTok{  }\FunctionTok{long}\KeywordTok{:}\AttributeTok{ test procedure specification}
\AttributeTok{  }\FunctionTok{description}\KeywordTok{:}\AttributeTok{ A formal document describing test steps and expected results}
\AttributeTok{  }\FunctionTok{plural}\KeywordTok{:}\AttributeTok{ TPSes}
\AttributeTok{  }\FunctionTok{longplural}\KeywordTok{:}\AttributeTok{ test procedure specifications}
\end{Highlighting}
\end{Shaded}

And then loaded during initialization as follows:

\begin{Shaded}
\begin{Highlighting}[]
\NormalTok{\#show: init{-}glossary.with(yaml("glossary.yaml"))}
\end{Highlighting}
\end{Shaded}

\subsubsection{Using Glossary Terms}\label{using-glossary-terms}

Reference glossary terms using Typst’s \texttt{\ @reference\ } syntax:

\begin{Shaded}
\begin{Highlighting}[]
\NormalTok{In modern web development, languages like @html and @css are essential.}
\NormalTok{The @tps:pl need to be submitted by Friday.}
\end{Highlighting}
\end{Shaded}

Available modifiers:

\begin{itemize}
\tightlist
\item
  \textbf{cap} : Capitalizes the term
\item
  \textbf{pl} : Uses the plural form
\item
  \textbf{both} : Shows “Long Form (Short Form)�
\item
  \textbf{short} : Shows only short form
\item
  \textbf{long} : Shows only long form
\item
  \textbf{def} or \textbf{desc} : Shows the description
\end{itemize}

Modifiers can be combined with colons:

\begin{longtable}[]{@{}ll@{}}
\toprule\noalign{}
\textbf{Input} & \textbf{Output} \\
\midrule\noalign{}
\endhead
\bottomrule\noalign{}
\endlastfoot
\texttt{\ @tps\ } (first use) & “test procedure specification
(TPS)� \\
\texttt{\ @tps\ } (subsequent) & “TPS� \\
\texttt{\ @tps:short\ } & “TPS� \\
\texttt{\ @tps:long\ } & “test procedure specification� \\
\texttt{\ @tps:both\ } & “test procedure specification (TPS)� \\
\texttt{\ @tps:long:cap\ } & “Test procedure specification� \\
\texttt{\ @tps:long:pl\ } & “test procedure specifications� \\
\texttt{\ @tps:short:pl\ } & “TPSes� \\
\texttt{\ @tps:both:pl:cap\ } & “Technical procedure specifications
(TPSes)� \\
\texttt{\ @tps:def\ } & “A formal document describing test steps and
expected results� \\
\end{longtable}

\subsubsection{Generating the Glossary}\label{generating-the-glossary}

Display the glossary using the \texttt{\ glossary()\ } function:

\begin{Shaded}
\begin{Highlighting}[]
\NormalTok{\#glossary(}
\NormalTok{  title: "Web Development Glossary",}
\NormalTok{  theme: my{-}theme,}
\NormalTok{  groups: ("Web")  // Optional: Filter to specific groups}
\NormalTok{)}
\end{Highlighting}
\end{Shaded}

\subsubsection{Customizing Term Display}\label{customizing-term-display}

Control how terms appear in the document by providing a custom
\texttt{\ show-term\ } function:

\begin{Shaded}
\begin{Highlighting}[]
\NormalTok{\#let emph{-}term(term{-}body) = \{ emph(term{-}body) \}}

\NormalTok{\#show: init{-}glossary.with(}
\NormalTok{  myGlossary,}
\NormalTok{  show{-}term: emph{-}term}
\NormalTok{)}
\end{Highlighting}
\end{Shaded}

\subsubsection{Glossary Themes}\label{glossary-themes}

\paragraph{Included Themes}\label{included-themes}

Glossy comes with several built-in themes that can be used directly or
serve as examples for custom themes:

\subparagraph{theme-twocol}\label{theme-twocol}

A professional two-column layout ideal for technical documentation.
Features:

\begin{itemize}
\tightlist
\item
  Two-column layout for efficient space usage
\item
  Dotted leaders to page numbers
\item
  Clear hierarchy with optional group headings
\item
  Compact but readable formatting
\item
  Terms in regular weight with long forms and descriptions inline
\end{itemize}

\begin{Shaded}
\begin{Highlighting}[]
\NormalTok{\#glossary(theme: theme{-}twocol)}
\end{Highlighting}
\end{Shaded}

\subparagraph{theme-basic}\label{theme-basic}

A traditional single-column layout similar to book glossaries. Features:

\begin{itemize}
\tightlist
\item
  Bold terms with indented content
\item
  Clear separation between entries
\item
  Hanging indentation for wrapped lines
\item
  Parenthetical long forms
\item
  Page numbers with “pp.� prefix
\item
  Simple, clean design
\end{itemize}

\begin{Shaded}
\begin{Highlighting}[]
\NormalTok{\#glossary(theme: theme{-}basic)}
\end{Highlighting}
\end{Shaded}

\subparagraph{theme-compact}\label{theme-compact}

A space-efficient layout perfect for technical documents or appendices.
Features:

\begin{itemize}
\tightlist
\item
  Reduced vertical spacing
\item
  Smaller font sizes for secondary information
\item
  Color-coded term components
\item
  Grid-based alignment
\item
  Minimal decorative elements
\item
  Gray text for supplementary information
\item
  Bullet separators between components
\end{itemize}

\begin{Shaded}
\begin{Highlighting}[]
\NormalTok{\#glossary(theme: theme{-}compact)}
\end{Highlighting}
\end{Shaded}

\paragraph{Custom Themes}\label{custom-themes}

Customize glossary appearance by defining a theme with three functions:

\begin{Shaded}
\begin{Highlighting}[]
\NormalTok{\#let my{-}theme = (}
\NormalTok{  // Main glossary section}
\NormalTok{  section: (title, body) =\textgreater{} \{}
\NormalTok{    heading(level: 1, title)}
\NormalTok{    body}
\NormalTok{  \},}

\NormalTok{  // Group of related terms}
\NormalTok{  group: (name, index, total, body) =\textgreater{} \{}
\NormalTok{    // index = group index, total = total groups}
\NormalTok{    if name != "" and total \textgreater{} 1 \{}
\NormalTok{      heading(level: 2, name)}
\NormalTok{    \}}
\NormalTok{    body}
\NormalTok{  \},}

\NormalTok{  // Individual glossary entry}
\NormalTok{  entry: (entry, index, total) =\textgreater{} \{}
\NormalTok{    // index = entry index, total = total entries in group}
\NormalTok{    let output = [\#entry.short]}
\NormalTok{    if entry.long != none \{}
\NormalTok{      output = [\#output {-}{-} \#entry.long]}
\NormalTok{    \}}
\NormalTok{    if entry.description != none \{}
\NormalTok{      output = [\#output: \#entry.description]}
\NormalTok{    \}}
\NormalTok{    block(}
\NormalTok{      grid(}
\NormalTok{        columns: (auto, 1fr, auto),}
\NormalTok{        output,}
\NormalTok{        repeat([\#h(0.25em) . \#h(0.25em)]),}
\NormalTok{        entry.pages,}
\NormalTok{      )}
\NormalTok{    )}
\NormalTok{  \}}
\NormalTok{)}
\end{Highlighting}
\end{Shaded}

Entry fields available to themes:

\begin{itemize}
\tightlist
\item
  \texttt{\ short\ } : Short form (always present)
\item
  \texttt{\ long\ } : Long form (can be \texttt{\ none\ } )
\item
  \texttt{\ description\ } : Term description (can be \texttt{\ none\ }
  )
\item
  \texttt{\ label\ } : Term’s dictionary label
\item
  \texttt{\ pages\ } : Linked page numbers where term appears
\end{itemize}

\subsection{License}\label{license}

This project is licensed under the MIT License.

\subsubsection{How to add}\label{how-to-add}

Copy this into your project and use the import as \texttt{\ glossy\ }

\begin{verbatim}
#import "@preview/glossy:0.2.0"
\end{verbatim}

\includesvg[width=0.16667in,height=0.16667in]{/assets/icons/16-copy.svg}

Check the docs for
\href{https://typst.app/docs/reference/scripting/\#packages}{more
information on how to import packages} .

\subsubsection{About}\label{about}

\begin{description}
\tightlist
\item[Author :]
\href{mailto:steve@waits.net}{Stephen Waits}
\item[License:]
MIT
\item[Current version:]
0.2.0
\item[Last updated:]
November 26, 2024
\item[First released:]
October 23, 2024
\item[Archive size:]
10.2 kB
\href{https://packages.typst.org/preview/glossy-0.2.0.tar.gz}{\pandocbounded{\includesvg[keepaspectratio]{/assets/icons/16-download.svg}}}
\item[Repository:]
\href{https://github.com/swaits/typst-collection}{GitHub}
\item[Categor y :]
\begin{itemize}
\tightlist
\item[]
\item
  \pandocbounded{\includesvg[keepaspectratio]{/assets/icons/16-list-unordered.svg}}
  \href{https://typst.app/universe/search/?category=model}{Model}
\end{itemize}
\end{description}

\subsubsection{Where to report issues?}\label{where-to-report-issues}

This package is a project of Stephen Waits . Report issues on
\href{https://github.com/swaits/typst-collection}{their repository} .
You can also try to ask for help with this package on the
\href{https://forum.typst.app}{Forum} .

Please report this package to the Typst team using the
\href{https://typst.app/contact}{contact form} if you believe it is a
safety hazard or infringes upon your rights.

\phantomsection\label{versions}
\subsubsection{Version history}\label{version-history}

\begin{longtable}[]{@{}ll@{}}
\toprule\noalign{}
Version & Release Date \\
\midrule\noalign{}
\endhead
\bottomrule\noalign{}
\endlastfoot
0.2.0 & November 26, 2024 \\
\href{https://typst.app/universe/package/glossy/0.1.2/}{0.1.2} & October
31, 2024 \\
\href{https://typst.app/universe/package/glossy/0.1.1/}{0.1.1} & October
24, 2024 \\
\href{https://typst.app/universe/package/glossy/0.1.0/}{0.1.0} & October
23, 2024 \\
\end{longtable}

Typst GmbH did not create this package and cannot guarantee correct
functionality of this package or compatibility with any version of the
Typst compiler or app.


\title{typst.app/universe/package/minimal-cv}

\phantomsection\label{banner}
\phantomsection\label{template-thumbnail}
\pandocbounded{\includegraphics[keepaspectratio]{https://packages.typst.org/preview/thumbnails/minimal-cv-0.1.0-small.webp}}

\section{minimal-cv}\label{minimal-cv}

{ 0.1.0 }

A clean and customizable CV template

\href{/app?template=minimal-cv&version=0.1.0}{Create project in app}

\phantomsection\label{readme}
Yet another John Doe CV.

\href{https://github.com/typst/packages/raw/main/packages/preview/minimal-cv/0.1.0/thumbnail.png}{\includegraphics[width=3.125in,height=\textheight,keepaspectratio]{https://github.com/typst/packages/raw/main/packages/preview/minimal-cv/0.1.0/thumbnail.png}}

A Typst CV template that aims for :

\begin{itemize}
\tightlist
\item
  Clean aesthetics
\item
  Easy customizability
\end{itemize}

\subsection{Usage}\label{usage}

\subsubsection{From Typst app}\label{from-typst-app}

Create a new project based on the template
\href{https://typst.app/universe/package/minimal-cv}{minimal-cv} .

\subsubsection{Locally}\label{locally}

The default font is
\href{https://fonts.google.com/specimen/Inria+Sans}{“Inria Sans�} .
Make sure it is installed on your system, or change it in
\href{https://github.com/typst/packages/raw/main/packages/preview/minimal-cv/0.1.0/\#theme}{\#
Theme} .

Copy the
\href{https://raw.githubusercontent.com/lelimacon/typst-minimal-cv/main/template/cv.typ}{template}
to your Typst project.

\subsubsection{From a blank project}\label{from-a-blank-project}

Import the library :

\begin{Shaded}
\begin{Highlighting}[]
\NormalTok{\#import "@preview/minimal{-}cv:0.1.0": *}
\end{Highlighting}
\end{Shaded}

Show the root \texttt{\ cv\ } function :

\begin{Shaded}
\begin{Highlighting}[]
\NormalTok{\#show: cv.with(}
\NormalTok{  theme: (),}
\NormalTok{  title: "YOUR NAME",}
\NormalTok{  subtitle: "YOUR POSITION",}
\NormalTok{  aside: [}
\NormalTok{    ASIDE CONTENT}
\NormalTok{  ]}

\NormalTok{MAIN CONTENT}
\end{Highlighting}
\end{Shaded}

Several content functions are available.

\textbf{Section}

\begin{Shaded}
\begin{Highlighting}[]
\NormalTok{\#section(}
\NormalTok{  theme: (),}
\NormalTok{  "TITLE\_CONTENT",}
\NormalTok{  "BODY\_CONTENT",}
\NormalTok{)}
\end{Highlighting}
\end{Shaded}

\textbf{Entry}

\begin{Shaded}
\begin{Highlighting}[]
\NormalTok{\#entry(}
\NormalTok{  theme: (),}
\NormalTok{  right: "FLOATING\_CONTENT",}

\NormalTok{  "GUTTER\_CONTENT",}
\NormalTok{  "TITLE\_CONTENT",}
\NormalTok{  "BODY\_CONTENT",}
\NormalTok{)}
\end{Highlighting}
\end{Shaded}

\textbf{Progress bar}

\begin{Shaded}
\begin{Highlighting}[]
\NormalTok{\#progress{-}bar(50\%)}
\end{Highlighting}
\end{Shaded}

\subsection{Theme}\label{theme}

Customize the theme by specifying the \texttt{\ theme\ } parameter and
overriding 1 or more keys.

\subsubsection{\texorpdfstring{Function
\texttt{\ cv\ }}{Function  cv }}\label{function-cv}

\begin{longtable}[]{@{}lll@{}}
\toprule\noalign{}
Key & Type & Default \\
\midrule\noalign{}
\endhead
\bottomrule\noalign{}
\endlastfoot
\texttt{\ margin\ } & relative & \texttt{\ 22pt\ } \\
\texttt{\ font\ } & relative & \texttt{\ "Inria\ Sans"\ } \\
\texttt{\ font-size\ } & relative & \texttt{\ 11pt\ } \\
\texttt{\ accent-color\ } & color & \texttt{\ blue\ } \\
\texttt{\ body-color\ } & color & \texttt{\ rgb("222")\ } \\
\texttt{\ header-accent-color\ } & color & inherit \\
\texttt{\ header-body-color\ } & color & inherit \\
\texttt{\ main-accent-color\ } & color & inherit \\
\texttt{\ main-body-color\ } & color & inherit \\
\texttt{\ main-width\ } & relative & \texttt{\ 5fr\ } \\
\texttt{\ main-gutter-width\ } & relative & \texttt{\ 64pt\ } \\
\texttt{\ aside-accent-color\ } & color & inherit \\
\texttt{\ aside-body-color\ } & color & inherit \\
\texttt{\ aside-width\ } & relative & \texttt{\ 3fr\ } \\
\texttt{\ aside-gutter-width\ } & relative & \texttt{\ 48pt\ } \\
\end{longtable}

\subsubsection{\texorpdfstring{Function
\texttt{\ section\ }}{Function  section }}\label{function-section}

\begin{longtable}[]{@{}lll@{}}
\toprule\noalign{}
Key & Type & Default \\
\midrule\noalign{}
\endhead
\bottomrule\noalign{}
\endlastfoot
\texttt{\ gutter-size\ } & color & inherit \\
\texttt{\ accent-color\ } & color & inherit \\
\texttt{\ body-color\ } & color & inherit \\
\end{longtable}

\subsubsection{\texorpdfstring{Function
\texttt{\ entry\ }}{Function  entry }}\label{function-entry}

\begin{longtable}[]{@{}lll@{}}
\toprule\noalign{}
Key & Type & Default \\
\midrule\noalign{}
\endhead
\bottomrule\noalign{}
\endlastfoot
\texttt{\ gutter-size\ } & color & inherit \\
\texttt{\ accent-color\ } & color & inherit \\
\texttt{\ body-color\ } & color & inherit \\
\end{longtable}

\href{/app?template=minimal-cv&version=0.1.0}{Create project in app}

\subsubsection{How to use}\label{how-to-use}

Click the button above to create a new project using this template in
the Typst app.

You can also use the Typst CLI to start a new project on your computer
using this command:

\begin{verbatim}
typst init @preview/minimal-cv:0.1.0
\end{verbatim}

\includesvg[width=0.16667in,height=0.16667in]{/assets/icons/16-copy.svg}

\subsubsection{About}\label{about}

\begin{description}
\tightlist
\item[Author :]
\href{https://github.com/lelimacon}{lelimacon}
\item[License:]
MIT
\item[Current version:]
0.1.0
\item[Last updated:]
June 12, 2024
\item[First released:]
June 12, 2024
\item[Archive size:]
4.43 kB
\href{https://packages.typst.org/preview/minimal-cv-0.1.0.tar.gz}{\pandocbounded{\includesvg[keepaspectratio]{/assets/icons/16-download.svg}}}
\item[Repository:]
\href{https://github.com/lelimacon/typst-minimal-cv}{GitHub}
\item[Categor y :]
\begin{itemize}
\tightlist
\item[]
\item
  \pandocbounded{\includesvg[keepaspectratio]{/assets/icons/16-user.svg}}
  \href{https://typst.app/universe/search/?category=cv}{CV}
\end{itemize}
\end{description}

\subsubsection{Where to report issues?}\label{where-to-report-issues}

This template is a project of lelimacon . Report issues on
\href{https://github.com/lelimacon/typst-minimal-cv}{their repository} .
You can also try to ask for help with this template on the
\href{https://forum.typst.app}{Forum} .

Please report this template to the Typst team using the
\href{https://typst.app/contact}{contact form} if you believe it is a
safety hazard or infringes upon your rights.

\phantomsection\label{versions}
\subsubsection{Version history}\label{version-history}

\begin{longtable}[]{@{}ll@{}}
\toprule\noalign{}
Version & Release Date \\
\midrule\noalign{}
\endhead
\bottomrule\noalign{}
\endlastfoot
0.1.0 & June 12, 2024 \\
\end{longtable}

Typst GmbH did not create this template and cannot guarantee correct
functionality of this template or compatibility with any version of the
Typst compiler or app.


\title{typst.app/universe/package/mitex}

\phantomsection\label{banner}
\section{mitex}\label{mitex}

{ 0.2.4 }

LaTeX support for Typst, powered by Rust and WASM.

\phantomsection\label{readme}
\textbf{\href{https://www.latex-project.org/}{LaTeX} support for
\href{https://typst.app/}{Typst} , powered by
\href{https://www.rust-lang.org/}{Rust} and
\href{https://webassembly.org/}{WASM} .}

\href{https://github.com/mitex-rs/mitex}{MiTeX} processes LaTeX code
into an abstract syntax tree (AST). Then it transforms the AST into
Typst code and evaluates code into Typst content by \texttt{\ eval\ }
function.

MiTeX has been proved to be practical on a large project. It has already
correctly converted 32.5k equations from
\href{https://github.com/OI-wiki/OI-wiki}{OI Wiki} . Compared to
\href{https://github.com/jgm/texmath}{texmath} , MiTeX has a better
display effect and performance in that wiki project. It is also more
easy to use, since importing MiTeX to Typst is just one line of code,
while texmath is an external program.

In addition, MiTeX is not only \textbf{SMALL} but also \textbf{FAST} !
MiTeX has a size of just about 185 KB, comparing that texmath has a size
of 17 MB. A not strict but intuitive comparison is shown below. To
convert 32.5k equations from OI Wiki, texmath takes about 109s, while
MiTeX WASM takes only 2.28s and MiTeX x86 takes merely 0.085s.

Thanks to \href{https://github.com/Myriad-Dreamin}{@Myriad-Dreamin} , he
completed the most complex development work: developing the parser for
generating AST.

\subsection{MiTeX as a Typst Package}\label{mitex-as-a-typst-package}

\begin{itemize}
\tightlist
\item
  Use \texttt{\ mitex-convert\ } to convert LaTeX code into Typst code
  in string.
\item
  Use \texttt{\ mi\ } to render an inline LaTeX equation in Typst.
\item
  Use \texttt{\ mitex(numbering:\ none,\ supplement:\ auto,\ ..)\ } or
  \texttt{\ mimath\ } to render a block LaTeX equation in Typst.
\item
  Use \texttt{\ mitext\ } to render a LaTeX text in Typst.
\end{itemize}

PS: \texttt{\ \#set\ math.equation(numbering:\ "(1)")\ } is also valid
for MiTeX.

Following is
\href{https://github.com/mitex-rs/mitex/blob/main/packages/mitex/examples/example.typ}{a
simple example} of using MiTeX in Typst:

\begin{Shaded}
\begin{Highlighting}[]
\NormalTok{\#import "@preview/mitex:0.2.4": *}

\NormalTok{\#assert.eq(mitex{-}convert("\textbackslash{}alpha x"), "alpha  x ")}

\NormalTok{Write inline equations like \#mi("x") or \#mi[y].}

\NormalTok{Also block equations (this case is from \#text(blue.lighten(20\%), link("https://katex.org/")[katex.org])):}

\NormalTok{\#mitex(\textasciigrave{}}
\NormalTok{  \textbackslash{}newcommand\{\textbackslash{}f\}[2]\{\#1f(\#2)\}}
\NormalTok{  \textbackslash{}f\textbackslash{}relax\{x\} = \textbackslash{}int\_\{{-}\textbackslash{}infty\}\^{}\textbackslash{}infty}
\NormalTok{    \textbackslash{}f\textbackslash{}hat\textbackslash{}xi\textbackslash{},e\^{}\{2 \textbackslash{}pi i \textbackslash{}xi x\}}
\NormalTok{    \textbackslash{},d\textbackslash{}xi}
\NormalTok{\textasciigrave{})}

\NormalTok{We also support text mode (in development):}

\NormalTok{\#mitext(\textasciigrave{}}
\NormalTok{  \textbackslash{}iftypst}
\NormalTok{    \#set math.equation(numbering: "(1)", supplement: "equation")}
\NormalTok{  \textbackslash{}fi}

\NormalTok{  \textbackslash{}section\{Title\}}

\NormalTok{  A \textbackslash{}textbf\{strong\} text, a \textbackslash{}emph\{emph\} text and inline equation $x + y$.}

\NormalTok{  Also block \textbackslash{}eqref\{eq:pythagoras\}.}

\NormalTok{  \textbackslash{}begin\{equation\}}
\NormalTok{    a\^{}2 + b\^{}2 = c\^{}2 \textbackslash{}label\{eq:pythagoras\}}
\NormalTok{  \textbackslash{}end\{equation\}}
\NormalTok{\textasciigrave{})}
\end{Highlighting}
\end{Shaded}

\pandocbounded{\includegraphics[keepaspectratio]{https://github.com/typst/packages/raw/main/packages/preview/mitex/0.2.4/examples/example.png}}

\subsection{MiTeX as a CLI Tool}\label{mitex-as-a-cli-tool}

\subsubsection{Installation}\label{installation}

Install latest nightly version by
\texttt{\ cargo\ install\ -\/-git\ https://github.com/mitex-rs/mitex\ mitex-cli\ }
.

\subsubsection{Usage}\label{usage}

\begin{Shaded}
\begin{Highlighting}[]
\ExtensionTok{mitex}\NormalTok{ compile main.tex}
\CommentTok{\# or (same as above)}
\ExtensionTok{mitex}\NormalTok{ compile main.tex mitex.typ}
\end{Highlighting}
\end{Shaded}

\subsection{MiTeX as a Web App}\label{mitex-as-a-web-app}

\subsubsection{MiTeX Online Math
Converter}\label{mitex-online-math-converter}

We can convert LaTeX equations to Typst equations in web by wasm.
\url{https://mitex-rs.github.io/mitex/}

\subsubsection{Underleaf}\label{underleaf}

We made a proof of concept online tex editor to show our conversion
speed in text mode. The PoC loads files from a git repository and then
runs typst compile in browser. As you see, each keystroking get response
in preview quickly.

\url{https://mitex-rs.github.io/mitex/tools/underleaf.html}

\url{https://github.com/mitex-rs/mitex/assets/34951714/0ce77a2c-0a7d-445f-b26d-e139f3038f83}

\subsection{Implemented Features}\label{implemented-features}

\begin{itemize}
\tightlist
\item
  {[}x{]} User-defined TeX (macro) commands, such as
  \texttt{\ \textbackslash{}newcommand\{\textbackslash{}mysym\}\{\textbackslash{}alpha\}\ }
  .
\item
  {[}x{]} LaTeX equations support.

  \begin{itemize}
  \tightlist
  \item
    {[}x{]} Coloring commands (
    \texttt{\ \textbackslash{}color\{red\}\ text\ } ,
    \texttt{\ \textbackslash{}textcolor\{red\}\{text\}\ } ).
  \item
    {[}x{]} Support for various environments, such as aligned, matrix,
    cases.
  \end{itemize}
\item
  {[}x{]} Basic text mode support, you can use it to write LaTeX drafts.

  \begin{itemize}
  \tightlist
  \item
    {[}x{]} \texttt{\ \textbackslash{}section\ } ,
    \texttt{\ \textbackslash{}textbf\ } ,
    \texttt{\ \textbackslash{}emph\ } .
  \item
    {[}x{]} Inline and block math equations.
  \item
    {[}x{]} \texttt{\ \textbackslash{}ref\ } ,
    \texttt{\ \textbackslash{}eqref\ } and
    \texttt{\ \textbackslash{}label\ } .
  \item
    {[}x{]} \texttt{\ itemize\ } and \texttt{\ enumerate\ }
    environments.
  \end{itemize}
\end{itemize}

\subsection{Features to Implement}\label{features-to-implement}

\begin{itemize}
\tightlist
\item
  {[} {]} Pass command specification to MiTeX plugin dynamically. With
  that you can define a typst function
  \texttt{\ let\ myop(it)\ =\ op(upright(it))\ } and then use it by
  \texttt{\ \textbackslash{}myop\{it\}\ } .
\item
  {[} {]} Package support, which means that you can change set of
  commands by telling MiTeX to use a list of packages.
\item
  {[} {]} Better text mode support, such as figure, algorithm and
  description environments.
\end{itemize}

To achieve the latter two goals, we need a well-structured architecture
for the text mode, along with intricate work. Currently, we don’t have
very clear ideas yet. If you are willing to contribute by discussing in
the issues or even submitting pull requests, your contribution is highly
welcome.

\subsection{Differences between MiTeX and other
solutions}\label{differences-between-mitex-and-other-solutions}

MiTeX has different objectives compared to
\href{https://github.com/jgm/texmath}{texmath} (a.k.a.
\href{https://pandoc.org/}{pandoc} ):

\begin{itemize}
\tightlist
\item
  MiTeX focuses on rendering LaTeX content correctly within Typst,
  leveraging the powerful programming capabilities of WASM and typst to
  achieve results that are essentially consistent with LaTeX display.
\item
  texmath aims to be general-purpose converters and generate strings
  that are more human-readable.
\end{itemize}

For example, MiTeX transforms
\texttt{\ \textbackslash{}frac\{1\}\{2\}\_3\ } into
\texttt{\ frac(1,\ 2)\_3\ } , while texmath converts it into
\texttt{\ 1\ /\ 2\_3\ } . The latter’s display is not entirely
correct, whereas the former ensures consistency in display.

Another example is that MiTeX transforms
\texttt{\ (\textbackslash{}frac\{1\}\{2\})\ } into
\texttt{\ \textbackslash{}(frac(1,\ 2)\textbackslash{})\ } instead of
\texttt{\ (frac(1,\ 2))\ } , avoiding the use of automatic Left/Right to
achieve consistency with LaTeX rendering.

\textbf{Certainly, the greatest advantage is that you can directly write
LaTeX content in Typst without the need for manual conversion!}

\subsection{Submitting Issues}\label{submitting-issues}

If you find missing commands or bugs of MiTeX, please feel free to
submit an issue \href{https://github.com/mitex-rs/mitex/issues}{here} .

\subsection{Contributing to MiTeX}\label{contributing-to-mitex}

Currently, MiTeX maintains following three parts of code:

\begin{itemize}
\tightlist
\item
  A TeX parser library written in \textbf{Rust} , see
  \href{https://github.com/mitex-rs/mitex/tree/main/crates/mitex-lexer}{mitex-lexer}
  and
  \href{https://github.com/mitex-rs/mitex/tree/main/crates/mitex-parser}{mitex-parser}
  .
\item
  A TeX to Typst converter library written in \textbf{Rust} , see
  \href{https://github.com/mitex-rs/mitex/tree/main/crates/mitex}{mitex}
  .
\item
  A list of TeX packages and comamnds written in \textbf{Typst} , which
  then used by the typst package, see
  \href{https://github.com/mitex-rs/mitex/tree/main/packages/mitex/specs}{MiTeX
  Command Specification} .
\end{itemize}

For a translation process, for example, we have:

\begin{verbatim}
\frac{1}{2}

===[parser]===> AST ===[converter]===>

#eval("$frac(1, 2)$", scope: (frac: (num, den) => $(num)/(den)$))
\end{verbatim}

You can use the \texttt{\ \#mitex-convert()\ } function to get the Typst
Code generated from LaTeX Code.

\subsubsection{Add missing TeX commands}\label{add-missing-tex-commands}

Even if you don’t know Rust at all, you can still add missing TeX
commands to MiTeX by modifying
\href{https://github.com/mitex-rs/mitex/tree/main/packages/mitex/specs}{specification
files} , since they are written in typst! You can open an issue to
acquire the commands you want to add, or you can edit the files and
submit a pull request.

In the future, we will provide the ability to customize TeX commands,
which will make it easier for you to use the commands you create for
yourself.

\subsubsection{Develop the parser and the
converter}\label{develop-the-parser-and-the-converter}

See
\href{https://github.com/mitex-rs/mitex/blob/main/CONTRIBUTING.md}{CONTRIBUTING.md}
.

\subsubsection{How to add}\label{how-to-add}

Copy this into your project and use the import as \texttt{\ mitex\ }

\begin{verbatim}
#import "@preview/mitex:0.2.4"
\end{verbatim}

\includesvg[width=0.16667in,height=0.16667in]{/assets/icons/16-copy.svg}

Check the docs for
\href{https://typst.app/docs/reference/scripting/\#packages}{more
information on how to import packages} .

\subsubsection{About}\label{about}

\begin{description}
\tightlist
\item[Author s :]
Myriad-Dreamin , OrangeX4 , \& Enter-tainer
\item[License:]
Apache-2.0
\item[Current version:]
0.2.4
\item[Last updated:]
May 13, 2024
\item[First released:]
December 23, 2023
\item[Archive size:]
109 kB
\href{https://packages.typst.org/preview/mitex-0.2.4.tar.gz}{\pandocbounded{\includesvg[keepaspectratio]{/assets/icons/16-download.svg}}}
\item[Repository:]
\href{https://github.com/mitex-rs/mitex}{GitHub}
\item[Categor y :]
\begin{itemize}
\tightlist
\item[]
\item
  \pandocbounded{\includesvg[keepaspectratio]{/assets/icons/16-hammer.svg}}
  \href{https://typst.app/universe/search/?category=utility}{Utility}
\end{itemize}
\end{description}

\subsubsection{Where to report issues?}\label{where-to-report-issues}

This package is a project of Myriad-Dreamin, OrangeX4, and Enter-tainer
. Report issues on \href{https://github.com/mitex-rs/mitex}{their
repository} . You can also try to ask for help with this package on the
\href{https://forum.typst.app}{Forum} .

Please report this package to the Typst team using the
\href{https://typst.app/contact}{contact form} if you believe it is a
safety hazard or infringes upon your rights.

\phantomsection\label{versions}
\subsubsection{Version history}\label{version-history}

\begin{longtable}[]{@{}ll@{}}
\toprule\noalign{}
Version & Release Date \\
\midrule\noalign{}
\endhead
\bottomrule\noalign{}
\endlastfoot
0.2.4 & May 13, 2024 \\
\href{https://typst.app/universe/package/mitex/0.2.3/}{0.2.3} & April 1,
2024 \\
\href{https://typst.app/universe/package/mitex/0.2.2/}{0.2.2} & March
10, 2024 \\
\href{https://typst.app/universe/package/mitex/0.2.1/}{0.2.1} & January
15, 2024 \\
\href{https://typst.app/universe/package/mitex/0.2.0/}{0.2.0} & January
1, 2024 \\
\href{https://typst.app/universe/package/mitex/0.1.0/}{0.1.0} & December
23, 2023 \\
\end{longtable}

Typst GmbH did not create this package and cannot guarantee correct
functionality of this package or compatibility with any version of the
Typst compiler or app.


\title{typst.app/universe/package/cetz}

\phantomsection\label{banner}
\section{cetz}\label{cetz}

{ 0.3.1 }

Drawing with Typst made easy, providing an API inspired by TikZ and
Processing. Includes modules for plotting, charts and tree layout.

{ } Featured Package

\phantomsection\label{readme}
CeTZ (CeTZ, ein Typst Zeichenpaket) is a library for drawing with
\href{https://typst.app/}{Typst} with an API inspired by TikZ and
\href{https://processing.org/}{Processing} .

\subsection{Examples}\label{examples}

\begin{longtable}[]{@{}lll@{}}
\toprule\noalign{}
\endhead
\bottomrule\noalign{}
\endlastfoot
\href{https://github.com/typst/packages/raw/main/packages/preview/cetz/0.3.1/gallery/karls-picture.typ}{\includegraphics[width=2.60417in,height=\textheight,keepaspectratio]{https://github.com/typst/packages/raw/main/packages/preview/cetz/0.3.1/gallery/karls-picture.png}}
&
\href{https://github.com/typst/packages/raw/main/packages/preview/cetz/0.3.1/gallery/tree.typ}{\includegraphics[width=2.60417in,height=\textheight,keepaspectratio]{https://github.com/typst/packages/raw/main/packages/preview/cetz/0.3.1/gallery/tree.png}}
&
\href{https://github.com/typst/packages/raw/main/packages/preview/cetz/0.3.1/gallery/waves.typ}{\includegraphics[width=2.60417in,height=\textheight,keepaspectratio]{https://github.com/typst/packages/raw/main/packages/preview/cetz/0.3.1/gallery/waves.png}} \\
Karl\textquotesingle s Picture & Tree Layout & Waves \\
\end{longtable}

\emph{Click on the example image to jump to the code.}

\subsection{Usage}\label{usage}

For information, see the
\href{https://cetz-package.github.io/docs}{online manual} .

To use this package, simply add the following code to your document:

\begin{verbatim}
#import "@preview/cetz:0.3.1"

#cetz.canvas({
  import cetz.draw: *
  // Your drawing code goes here
})
\end{verbatim}

\subsection{CeTZ Libraries}\label{cetz-libraries}

\begin{itemize}
\tightlist
\item
  \href{https://github.com/cetz-package/cetz-plot}{cetz-plot - Plotting
  and Charts Library}
\item
  \href{https://github.com/cetz-package/cetz-venn}{cetz-venn - Simple
  two- or three-set Venn diagrams}
\end{itemize}

\subsection{Installing}\label{installing}

To install the CeTZ package under
\href{https://github.com/typst/packages?tab=readme-ov-file\#local-packages}{your
local typst package dir} you can use the \texttt{\ install\ } script
from the repository.

\begin{Shaded}
\begin{Highlighting}[]
\ExtensionTok{just}\NormalTok{ install}
\end{Highlighting}
\end{Shaded}

The installed version can be imported by prefixing the package name with
\texttt{\ @local\ } .

\begin{Shaded}
\begin{Highlighting}[]
\NormalTok{\#import "@local/cetz:0.3.1"}

\NormalTok{\#cetz.canvas(\{}
\NormalTok{  import cetz.draw: *}
\NormalTok{  // Your drawing code goes here}
\NormalTok{\})}
\end{Highlighting}
\end{Shaded}

\subsubsection{Just}\label{just}

This project uses \href{https://github.com/casey/just}{just} , a handy
command runner.

You can run all commands without having \texttt{\ just\ } installed,
just have a look into the \texttt{\ justfile\ } . To install
\texttt{\ just\ } on your system, use your systems package manager. On
Windows, \href{https://doc.rust-lang.org/cargo/}{Cargo} (
\texttt{\ cargo\ install\ just\ } ),
\href{https://chocolatey.org/}{Chocolatey} (
\texttt{\ choco\ install\ just\ } ) and
\href{https://just.systems/man/en/chapter_4.html}{some other sources}
can be used. You need to run it from a \texttt{\ sh\ } compatible shell
on Windows (e.g git-bash).

\subsection{Testing}\label{testing}

This package comes with some unit tests under the \texttt{\ tests\ }
directory. To run all tests you can run the \texttt{\ just\ test\ }
target. You need to have
\href{https://github.com/tingerrr/typst-test/}{\texttt{\ typst-test\ }}
in your \texttt{\ PATH\ } :
\texttt{\ cargo\ install\ typst-test\ -\/-git\ https://github.com/tingerrr/typst-test\ }
.

\subsection{Projects using CeTZ}\label{projects-using-cetz}

\begin{itemize}
\tightlist
\item
  \href{https://github.com/fenjalien/cirCeTZ}{cirCeTZ} A port of
  \href{https://github.com/circuitikz/circuitikz}{circuitikz} to Typst.
\item
  \href{https://github.com/sitandr/conchord}{conchord} Package for
  writing lyrics with chords that generates fretboard diagrams using
  CeTZ.
\item
  \href{https://github.com/jneug/typst-finite}{finite} Finite is a Typst
  package for rendering finite automata.
\item
  \href{https://github.com/Jollywatt/typst-fletcher}{fletcher} Package
  for drawing commutative diagrams and figures with arrows.
\item
  \href{https://github.com/ThatOneCalculator/riesketcher}{riesketcher}
  Package for drawing Riemann sums.
\end{itemize}

\subsubsection{How to add}\label{how-to-add}

Copy this into your project and use the import as \texttt{\ cetz\ }

\begin{verbatim}
#import "@preview/cetz:0.3.1"
\end{verbatim}

\includesvg[width=0.16667in,height=0.16667in]{/assets/icons/16-copy.svg}

Check the docs for
\href{https://typst.app/docs/reference/scripting/\#packages}{more
information on how to import packages} .

\subsubsection{About}\label{about}

\begin{description}
\tightlist
\item[Author s :]
\href{https://github.com/johannes-wolf}{Johannes Wolf} \&
\href{https://github.com/fenjalien}{fenjalien}
\item[License:]
LGPL-3.0-or-later
\item[Current version:]
0.3.1
\item[Last updated:]
October 21, 2024
\item[First released:]
July 8, 2023
\item[Minimum Typst version:]
0.12.0
\item[Archive size:]
74.3 kB
\href{https://packages.typst.org/preview/cetz-0.3.1.tar.gz}{\pandocbounded{\includesvg[keepaspectratio]{/assets/icons/16-download.svg}}}
\item[Repository:]
\href{https://github.com/cetz-package/cetz}{GitHub}
\item[Categor y :]
\begin{itemize}
\tightlist
\item[]
\item
  \pandocbounded{\includesvg[keepaspectratio]{/assets/icons/16-chart.svg}}
  \href{https://typst.app/universe/search/?category=visualization}{Visualization}
\end{itemize}
\end{description}

\subsubsection{Where to report issues?}\label{where-to-report-issues}

This package is a project of Johannes Wolf and fenjalien . Report issues
on \href{https://github.com/cetz-package/cetz}{their repository} . You
can also try to ask for help with this package on the
\href{https://forum.typst.app}{Forum} .

Please report this package to the Typst team using the
\href{https://typst.app/contact}{contact form} if you believe it is a
safety hazard or infringes upon your rights.

\phantomsection\label{versions}
\subsubsection{Version history}\label{version-history}

\begin{longtable}[]{@{}ll@{}}
\toprule\noalign{}
Version & Release Date \\
\midrule\noalign{}
\endhead
\bottomrule\noalign{}
\endlastfoot
0.3.1 & October 21, 2024 \\
\href{https://typst.app/universe/package/cetz/0.3.0/}{0.3.0} & October
15, 2024 \\
\href{https://typst.app/universe/package/cetz/0.2.2/}{0.2.2} & March 18,
2024 \\
\href{https://typst.app/universe/package/cetz/0.2.1/}{0.2.1} & February
23, 2024 \\
\href{https://typst.app/universe/package/cetz/0.2.0/}{0.2.0} & January
16, 2024 \\
\href{https://typst.app/universe/package/cetz/0.1.2/}{0.1.2} & October
1, 2023 \\
\href{https://typst.app/universe/package/cetz/0.1.1/}{0.1.1} & September
11, 2023 \\
\href{https://typst.app/universe/package/cetz/0.1.0/}{0.1.0} & August
19, 2023 \\
\href{https://typst.app/universe/package/cetz/0.0.2/}{0.0.2} & July 31,
2023 \\
\href{https://typst.app/universe/package/cetz/0.0.1/}{0.0.1} & July 8,
2023 \\
\end{longtable}

Typst GmbH did not create this package and cannot guarantee correct
functionality of this package or compatibility with any version of the
Typst compiler or app.


\title{typst.app/universe/package/touying-simpl-hkustgz}

\phantomsection\label{banner}
\phantomsection\label{template-thumbnail}
\pandocbounded{\includegraphics[keepaspectratio]{https://packages.typst.org/preview/thumbnails/touying-simpl-hkustgz-0.1.1-small.webp}}

\section{touying-simpl-hkustgz}\label{touying-simpl-hkustgz}

{ 0.1.1 }

Touying Slide Theme for HKUST(GZ)

\href{/app?template=touying-simpl-hkustgz&version=0.1.1}{Create project
in app}

\phantomsection\label{readme}
Inspired by \href{https://github.com/Coekjan/touying-buaa}{Touying Slide
Theme for Beihang University}

\subsection{Use as Typst Template
Package}\label{use-as-typst-template-package}

Use \texttt{\ typst\ init\ @preview/touying-simpl-hkustgz\ } to create a
new project with this theme.

\begin{Shaded}
\begin{Highlighting}[]
\NormalTok{$ typst init @preview/touying{-}simpl{-}hkustgz}
\NormalTok{Successfully created new project from @preview/touying{-}simpl{-}hkustgz:}
\NormalTok{To start writing, run:}
\NormalTok{\textgreater{} cd touying{-}simpl{-}hkustgz}
\NormalTok{\textgreater{} typst watch main.typ}
\end{Highlighting}
\end{Shaded}

\subsection{Examples}\label{examples}

See
\href{https://github.com/typst/packages/raw/main/packages/preview/touying-simpl-hkustgz/0.1.1/examples}{examples}
and \href{https://exaclior.github.io/touying-simpl-hkustgz}{Github
Pages} for more details.

You can compile the examples by yourself.

\begin{Shaded}
\begin{Highlighting}[]
\NormalTok{$ typst compile ./examples/main.typ {-}{-}root .}
\end{Highlighting}
\end{Shaded}

And the PDF file \texttt{\ ./examples/main.pdf\ } will be generated.

\subsection{License}\label{license}

Licensed under the
\href{https://github.com/typst/packages/raw/main/packages/preview/touying-simpl-hkustgz/0.1.1/LICENSE}{MIT
License} .

\href{/app?template=touying-simpl-hkustgz&version=0.1.1}{Create project
in app}

\subsubsection{How to use}\label{how-to-use}

Click the button above to create a new project using this template in
the Typst app.

You can also use the Typst CLI to start a new project on your computer
using this command:

\begin{verbatim}
typst init @preview/touying-simpl-hkustgz:0.1.1
\end{verbatim}

\includesvg[width=0.16667in,height=0.16667in]{/assets/icons/16-copy.svg}

\subsubsection{About}\label{about}

\begin{description}
\tightlist
\item[Author :]
\href{mailto:yushengzhao2020@outlook.com}{Yusheng Zhao}
\item[License:]
MIT
\item[Current version:]
0.1.1
\item[Last updated:]
November 12, 2024
\item[First released:]
August 28, 2024
\item[Archive size:]
9.83 kB
\href{https://packages.typst.org/preview/touying-simpl-hkustgz-0.1.1.tar.gz}{\pandocbounded{\includesvg[keepaspectratio]{/assets/icons/16-download.svg}}}
\item[Repository:]
\href{https://github.com/exAClior/touying-simpl-hkustgz}{GitHub}
\item[Categor y :]
\begin{itemize}
\tightlist
\item[]
\item
  \pandocbounded{\includesvg[keepaspectratio]{/assets/icons/16-presentation.svg}}
  \href{https://typst.app/universe/search/?category=presentation}{Presentation}
\end{itemize}
\end{description}

\subsubsection{Where to report issues?}\label{where-to-report-issues}

This template is a project of Yusheng Zhao . Report issues on
\href{https://github.com/exAClior/touying-simpl-hkustgz}{their
repository} . You can also try to ask for help with this template on the
\href{https://forum.typst.app}{Forum} .

Please report this template to the Typst team using the
\href{https://typst.app/contact}{contact form} if you believe it is a
safety hazard or infringes upon your rights.

\phantomsection\label{versions}
\subsubsection{Version history}\label{version-history}

\begin{longtable}[]{@{}ll@{}}
\toprule\noalign{}
Version & Release Date \\
\midrule\noalign{}
\endhead
\bottomrule\noalign{}
\endlastfoot
0.1.1 & November 12, 2024 \\
\href{https://typst.app/universe/package/touying-simpl-hkustgz/0.1.0/}{0.1.0}
& August 28, 2024 \\
\end{longtable}

Typst GmbH did not create this template and cannot guarantee correct
functionality of this template or compatibility with any version of the
Typst compiler or app.


\title{typst.app/universe/package/orange-book}

\phantomsection\label{banner}
\phantomsection\label{template-thumbnail}
\pandocbounded{\includegraphics[keepaspectratio]{https://packages.typst.org/preview/thumbnails/orange-book-0.4.0-small.webp}}

\section{orange-book}\label{orange-book}

{ 0.4.0 }

A book template inspired by The Legrand Orange Book of Mathias Legrand
and Vel

\href{/app?template=orange-book&version=0.4.0}{Create project in app}

\phantomsection\label{readme}
A book template inspired by The Legrand Orange Book of Mathias Legrand
and Vel
\url{https://www.latextemplates.com/template/legrand-orange-book} .

\subsection{Usage}\label{usage}

You can use this template in the Typst web app by clicking “Start from
template� on the dashboard and searching for \texttt{\ orange-book\ }
.

Alternatively, you can use the CLI to kick this project off using the
command

\begin{verbatim}
typst init @preview/orange-book
\end{verbatim}

Typst will create a new directory with all the files needed to get you
started.

\subsection{Configuration}\label{configuration}

This template exports the \texttt{\ book\ } function with the following
named arguments:

\begin{itemize}
\tightlist
\item
  \texttt{\ title\ } : The book’s title as content.
\item
  \texttt{\ subtitle\ } : The book’s subtitle as content.
\item
  \texttt{\ author\ } : Content or an array of content to specify the
  author.
\item
  \texttt{\ paper-size\ } : Defaults to \texttt{\ a4\ } . Specify a
  \href{https://typst.app/docs/reference/layout/page/\#parameters-paper}{paper
  size string} to change the page format.
\item
  \texttt{\ copyright\ } : Details about the copyright or
  \texttt{\ none\ } .
\item
  \texttt{\ lowercase-references\ } : True to have references in
  lowercase (Eg. table 1.1)
\end{itemize}

The function also accepts a single, positional argument for the body of
the book.

The template will initialize your package with a sample call to the
\texttt{\ book\ } function in a show rule. If you, however, want to
change an existing project to use this template, you can add a show rule
like this at the top of your file:

\begin{Shaded}
\begin{Highlighting}[]
\NormalTok{\#import "@preview/orange{-}book:0.1.0": book}

\NormalTok{\#show: book.with(}
\NormalTok{  title: "Exploring the Physical Manifestation of Humanity’s Subconscious Desires",}
\NormalTok{  subtitle: "A Practical Guide",}
\NormalTok{  date: "Anno scolastico 2023{-}2024",}
\NormalTok{  author: "Goro Akechi",}
\NormalTok{  mainColor: rgb("\#F36619"),}
\NormalTok{  lang: "en",}
\NormalTok{  cover: image("./background.svg"),}
\NormalTok{  imageIndex: image("./orange1.jpg"),}
\NormalTok{  listOfFigureTitle: "List of Figures",}
\NormalTok{  listOfTableTitle: "List of Tables",}
\NormalTok{  supplementChapter: "Chapter",}
\NormalTok{  supplementPart: "Part",}
\NormalTok{  part\_style: 0,}
\NormalTok{  copyright: [}
\NormalTok{    Copyright © 2023 Flavio Barisi}

\NormalTok{    PUBLISHED BY PUBLISHER}

\NormalTok{    \#link("https://github.com/flavio20002/typst{-}orange{-}template", "TEMPLATE{-}WEBSITE")}

\NormalTok{    Licensed under the Apache 2.0 License (the “License”).}
\NormalTok{    You may not use this file except in compliance with the License. You may obtain a copy of}
\NormalTok{    the License at https://www.apache.org/licenses/LICENSE{-}2.0. Unless required by}
\NormalTok{    applicable law or agreed to in writing, software distributed under the License is distributed on an}
\NormalTok{    “AS IS” BASIS, WITHOUT WARRANTIES OR CONDITIONS OF ANY KIND, either express or implied.}
\NormalTok{    See the License for the specific language governing permissions and limitations under the License.}

\NormalTok{    \_First printing, July 2023\_}
\NormalTok{  ],}
\NormalTok{  lowercase{-}references: false}
\NormalTok{)}

\NormalTok{// Your content goes below.}
\end{Highlighting}
\end{Shaded}

\href{/app?template=orange-book&version=0.4.0}{Create project in app}

\subsubsection{How to use}\label{how-to-use}

Click the button above to create a new project using this template in
the Typst app.

You can also use the Typst CLI to start a new project on your computer
using this command:

\begin{verbatim}
typst init @preview/orange-book:0.4.0
\end{verbatim}

\includesvg[width=0.16667in,height=0.16667in]{/assets/icons/16-copy.svg}

\subsubsection{About}\label{about}

\begin{description}
\tightlist
\item[Author :]
Flavio Barisi
\item[License:]
MIT-0
\item[Current version:]
0.4.0
\item[Last updated:]
November 4, 2024
\item[First released:]
August 26, 2024
\item[Minimum Typst version:]
0.12.0
\item[Archive size:]
661 kB
\href{https://packages.typst.org/preview/orange-book-0.4.0.tar.gz}{\pandocbounded{\includesvg[keepaspectratio]{/assets/icons/16-download.svg}}}
\item[Repository:]
\href{https://github.com/flavio20002/typst-orange-template}{GitHub}
\item[Categor y :]
\begin{itemize}
\tightlist
\item[]
\item
  \pandocbounded{\includesvg[keepaspectratio]{/assets/icons/16-docs.svg}}
  \href{https://typst.app/universe/search/?category=book}{Book}
\end{itemize}
\end{description}

\subsubsection{Where to report issues?}\label{where-to-report-issues}

This template is a project of Flavio Barisi . Report issues on
\href{https://github.com/flavio20002/typst-orange-template}{their
repository} . You can also try to ask for help with this template on the
\href{https://forum.typst.app}{Forum} .

Please report this template to the Typst team using the
\href{https://typst.app/contact}{contact form} if you believe it is a
safety hazard or infringes upon your rights.

\phantomsection\label{versions}
\subsubsection{Version history}\label{version-history}

\begin{longtable}[]{@{}ll@{}}
\toprule\noalign{}
Version & Release Date \\
\midrule\noalign{}
\endhead
\bottomrule\noalign{}
\endlastfoot
0.4.0 & November 4, 2024 \\
\href{https://typst.app/universe/package/orange-book/0.3.0/}{0.3.0} &
October 22, 2024 \\
\href{https://typst.app/universe/package/orange-book/0.2.0/}{0.2.0} &
October 3, 2024 \\
\href{https://typst.app/universe/package/orange-book/0.1.0/}{0.1.0} &
August 26, 2024 \\
\end{longtable}

Typst GmbH did not create this template and cannot guarantee correct
functionality of this template or compatibility with any version of the
Typst compiler or app.


\title{typst.app/universe/package/riesketcher}

\phantomsection\label{banner}
\section{riesketcher}\label{riesketcher}

{ 0.2.1 }

A package to draw Riemann sums (and their plots) of a function with
CeTZ.

\phantomsection\label{readme}
A package to draw Riemann sums (and their plots) of a function with
CeTZ.

Usage example and docs:
\href{https://github.com/ThatOneCalculator/riesketcher/blob/main/manual.pdf}{manual.pdf}

\begin{Shaded}
\begin{Highlighting}[]
\NormalTok{\#import "@preview/riesketcher:0.2.1": riesketcher}
\end{Highlighting}
\end{Shaded}

\pandocbounded{\includegraphics[keepaspectratio]{https://github.com/ThatOneCalculator/riesketcher/assets/44733677/4f87b750-e4be-4698-b650-74f4fe56789d}}

\subsubsection{How to add}\label{how-to-add}

Copy this into your project and use the import as
\texttt{\ riesketcher\ }

\begin{verbatim}
#import "@preview/riesketcher:0.2.1"
\end{verbatim}

\includesvg[width=0.16667in,height=0.16667in]{/assets/icons/16-copy.svg}

Check the docs for
\href{https://typst.app/docs/reference/scripting/\#packages}{more
information on how to import packages} .

\subsubsection{About}\label{about}

\begin{description}
\tightlist
\item[Author :]
Kainoa Kanter
\item[License:]
MIT
\item[Current version:]
0.2.1
\item[Last updated:]
May 22, 2024
\item[First released:]
December 19, 2023
\item[Archive size:]
2.40 kB
\href{https://packages.typst.org/preview/riesketcher-0.2.1.tar.gz}{\pandocbounded{\includesvg[keepaspectratio]{/assets/icons/16-download.svg}}}
\item[Repository:]
\href{https://github.com/ThatOneCalculator/riesketcher}{GitHub}
\end{description}

\subsubsection{Where to report issues?}\label{where-to-report-issues}

This package is a project of Kainoa Kanter . Report issues on
\href{https://github.com/ThatOneCalculator/riesketcher}{their
repository} . You can also try to ask for help with this package on the
\href{https://forum.typst.app}{Forum} .

Please report this package to the Typst team using the
\href{https://typst.app/contact}{contact form} if you believe it is a
safety hazard or infringes upon your rights.

\phantomsection\label{versions}
\subsubsection{Version history}\label{version-history}

\begin{longtable}[]{@{}ll@{}}
\toprule\noalign{}
Version & Release Date \\
\midrule\noalign{}
\endhead
\bottomrule\noalign{}
\endlastfoot
0.2.1 & May 22, 2024 \\
\href{https://typst.app/universe/package/riesketcher/0.2.0/}{0.2.0} &
January 17, 2024 \\
\href{https://typst.app/universe/package/riesketcher/0.1.0/}{0.1.0} &
December 19, 2023 \\
\end{longtable}

Typst GmbH did not create this package and cannot guarantee correct
functionality of this package or compatibility with any version of the
Typst compiler or app.


\title{typst.app/universe/package/outrageous}

\phantomsection\label{banner}
\section{outrageous}\label{outrageous}

{ 0.3.0 }

Easier customization of outline entries.

\phantomsection\label{readme}
Easier customization of outline entries.

\subsection{Examples}\label{examples}

For the full source see
\href{https://github.com/typst/packages/raw/main/packages/preview/outrageous/0.3.0/examples/basic.typ}{\texttt{\ examples/basic.typ\ }}
and for more examples see the
\href{https://github.com/typst/packages/raw/main/packages/preview/outrageous/0.3.0/examples}{\texttt{\ examples\ }
directory} .

\subsubsection{Default Style}\label{default-style}

\pandocbounded{\includegraphics[keepaspectratio]{https://github.com/typst/packages/raw/main/packages/preview/outrageous/0.3.0/example-default.png}}

\begin{Shaded}
\begin{Highlighting}[]
\NormalTok{\#import "@preview/outrageous:0.1.0"}
\NormalTok{\#show outline.entry: outrageous.show{-}entry}
\end{Highlighting}
\end{Shaded}

\subsubsection{Custom Settings}\label{custom-settings}

\pandocbounded{\includegraphics[keepaspectratio]{https://github.com/typst/packages/raw/main/packages/preview/outrageous/0.3.0/example-custom.png}}

\begin{Shaded}
\begin{Highlighting}[]
\NormalTok{\#import "@preview/outrageous:0.1.0"}
\NormalTok{\#show outline.entry: outrageous.show{-}entry.with(}
\NormalTok{  // the typst preset retains the normal Typst appearance}
\NormalTok{  ..outrageous.presets.typst,}
\NormalTok{  // we only override a few things:}
\NormalTok{  // level{-}1 entries are italic, all others keep their font style}
\NormalTok{  font{-}style: ("italic", auto),}
\NormalTok{  // no fill for level{-}1 entries, a thin gray line for all deeper levels}
\NormalTok{  fill: (none, line(length: 100\%, stroke: gray + .5pt)),}
\NormalTok{)}
\end{Highlighting}
\end{Shaded}

\subsection{Usage}\label{usage}

\subsubsection{\texorpdfstring{\texttt{\ show-entry\ }}{ show-entry }}\label{show-entry}

Show the given outline entry with the provided styling. Should be used
in a show rule like
\texttt{\ \#show\ outline.entry:\ outrageous.show-entry\ } .

\begin{Shaded}
\begin{Highlighting}[]
\NormalTok{\#let show{-}entry(}
\NormalTok{  entry,}
\NormalTok{  font{-}weight: presets.outrageous{-}toc.font{-}weight,}
\NormalTok{  font{-}style: presets.outrageous{-}toc.font{-}style,}
\NormalTok{  vspace: presets.outrageous{-}toc.vspace,}
\NormalTok{  font: presets.outrageous{-}toc.font,}
\NormalTok{  fill: presets.outrageous{-}toc.fill,}
\NormalTok{  fill{-}right{-}pad: presets.outrageous{-}toc.fill{-}right{-}pad,}
\NormalTok{  fill{-}align: presets.outrageous{-}toc.fill{-}align,}
\NormalTok{  body{-}transform: presets.outrageous{-}toc.body{-}transform,}
\NormalTok{  label: \textless{}outrageous{-}modified{-}entry\textgreater{},}
\NormalTok{  state{-}key: "outline{-}page{-}number{-}max{-}width",}
\NormalTok{) = \{ .. \}}
\end{Highlighting}
\end{Shaded}

\textbf{Arguments:}

For all the arguments that take arrays, the array’s first item
specifies the value for all level-one entries, the second item for
level-two, and so on. The array’s last item will be used for all
deeper/following levels as well.

\begin{itemize}
\tightlist
\item
  \texttt{\ entry\ } :
  \href{https://typst.app/docs/reference/foundations/content/}{\texttt{\ content\ }}
  â€'' The
  \href{https://typst.app/docs/reference/model/outline/\#definitions-entry}{\texttt{\ outline.entry\ }}
  element from the show rule.
\item
  \texttt{\ font-weight\ } :
  \href{https://typst.app/docs/reference/foundations/array/}{\texttt{\ array\ }}
  of (
  \href{https://typst.app/docs/reference/foundations/str/}{\texttt{\ str\ }}
  or
  \href{https://typst.app/docs/reference/foundations/int/}{\texttt{\ int\ }}
  or \texttt{\ auto\ } or \texttt{\ none\ } ) â€'' The entry’s font
  weight. Setting to \texttt{\ auto\ } or \texttt{\ none\ } keeps the
  context’s current style.
\item
  \texttt{\ font-style\ } :
  \href{https://typst.app/docs/reference/foundations/array/}{\texttt{\ array\ }}
  of (
  \href{https://typst.app/docs/reference/foundations/str/}{\texttt{\ str\ }}
  or \texttt{\ auto\ } or \texttt{\ none\ } ) â€'' The entry’s font
  style. Setting to \texttt{\ auto\ } or \texttt{\ none\ } keeps the
  context’s current style.
\item
  \texttt{\ vspace\ } :
  \href{https://typst.app/docs/reference/foundations/array/}{\texttt{\ array\ }}
  of (
  \href{https://typst.app/docs/reference/layout/relative/}{\texttt{\ relative\ }}
  or
  \href{https://typst.app/docs/reference/layout/fraction/}{\texttt{\ fraction\ }}
  or \texttt{\ none\ } ) â€'' Vertical spacing to add above the entry.
  Setting to \texttt{\ none\ } adds no space.
\item
  \texttt{\ font\ } :
  \href{https://typst.app/docs/reference/foundations/array/}{\texttt{\ array\ }}
  of (
  \href{https://typst.app/docs/reference/foundations/str/}{\texttt{\ str\ }}
  or
  \href{https://typst.app/docs/reference/foundations/array/}{\texttt{\ array\ }}
  or \texttt{\ auto\ } or \texttt{\ none\ } ) â€'' The entry’s font.
  Setting to \texttt{\ auto\ } or \texttt{\ none\ } keeps the
  context’s current font.
\item
  \texttt{\ fill\ } :
  \href{https://typst.app/docs/reference/foundations/array/}{\texttt{\ array\ }}
  of (
  \href{https://typst.app/docs/reference/foundations/content/}{\texttt{\ content\ }}
  or \texttt{\ auto\ } or \texttt{\ none\ } ) â€'' The entry’s fill.
  Setting to \texttt{\ auto\ } keeps the context’s current setting.
\item
  \texttt{\ fill-right-pad\ } :
  \href{https://typst.app/docs/reference/layout/relative/}{\texttt{\ relative\ }}
  or \texttt{\ none\ } â€'' Horizontal space to put between the fill and
  page number.
\item
  \texttt{\ fill-align\ } :
  \href{https://typst.app/docs/reference/foundations/bool/}{\texttt{\ bool\ }}
  â€'' Whether \texttt{\ fill-right-pad\ } should be relative to the
  current page number or the widest page number. Setting this to
  \texttt{\ true\ } has the effect of all fills ending on the same
  vertical line.
\item
  \texttt{\ body-transform\ } :
  \href{https://typst.app/docs/reference/foundations/function/}{\texttt{\ function\ }}
  or \texttt{\ none\ } â€'' Callback for custom edits to the entry’s
  body. It gets passed the entry’s level (
  \href{https://typst.app/docs/reference/foundations/int/}{\texttt{\ int\ }}
  ) and body (
  \href{https://typst.app/docs/reference/foundations/content/}{\texttt{\ content\ }}
  ) and should return
  \href{https://typst.app/docs/reference/foundations/content/}{\texttt{\ content\ }}
  or \texttt{\ none\ } . If \texttt{\ none\ } is returned, no
  modifications are made.
\item
  \texttt{\ page-transform\ } :
  \href{https://typst.app/docs/reference/foundations/function/}{\texttt{\ function\ }}
  or \texttt{\ none\ } â€'' Callback for custom edits to the entry’s
  page number. It gets passed the entry’s level (
  \href{https://typst.app/docs/reference/foundations/int/}{\texttt{\ int\ }}
  ) and page number (
  \href{https://typst.app/docs/reference/foundations/content/}{\texttt{\ content\ }}
  ) and should return
  \href{https://typst.app/docs/reference/foundations/content/}{\texttt{\ content\ }}
  or \texttt{\ none\ } . If \texttt{\ none\ } is returned, no
  modifications are made.
\item
  \texttt{\ label\ } :
  \href{https://typst.app/docs/reference/foundations/label/}{\texttt{\ label\ }}
  â€'' The label to internally use for tracking recursion. This should
  not need to be modified.
\item
  \texttt{\ state-key\ } :
  \href{https://typst.app/docs/reference/foundations/str/}{\texttt{\ str\ }}
  â€'' The key to use for the internal state which tracks the maximum
  page number width. The state is global for the entire document and
  thus applies to all outlines. If you wish to re-calculate the max page
  number width for \texttt{\ fill-align\ } , then you must provide a
  different key for each outline.
\end{itemize}

\textbf{Returns:}
\href{https://typst.app/docs/reference/foundations/content/}{\texttt{\ content\ }}

\subsubsection{\texorpdfstring{\texttt{\ presets\ }}{ presets }}\label{presets}

Presets for the arguments for
\href{https://github.com/typst/packages/raw/main/packages/preview/outrageous/0.3.0/\#show-entry}{\texttt{\ show-entry()\ }}
. You can use them in your show rule with
\texttt{\ \#show\ outline.entry:\ outrageous.show-entry.with(..outrageous.presets.outrageous-figures)\ }
.

\begin{Shaded}
\begin{Highlighting}[]
\NormalTok{\#let presets = (}
\NormalTok{  // outrageous preset for a Table of Contents}
\NormalTok{  outrageous{-}toc: (}
\NormalTok{    // ...}
\NormalTok{  ),}
\NormalTok{  // outrageous preset for List of Figures/Tables/Listings}
\NormalTok{  outrageous{-}figures: (}
\NormalTok{    // ...}
\NormalTok{  ),}
\NormalTok{  // preset without any style changes}
\NormalTok{  typst: (}
\NormalTok{    // ...}
\NormalTok{  ),}
\NormalTok{)}
\end{Highlighting}
\end{Shaded}

\subsubsection{\texorpdfstring{\texttt{\ repeat\ }}{ repeat }}\label{repeat}

Utility function to repeat content to fill space with a fixed size gap.

\begin{Shaded}
\begin{Highlighting}[]
\NormalTok{\#let repeat(gap: none, justify: false, body) = \{ .. \}}
\end{Highlighting}
\end{Shaded}

\textbf{Arguments:}

\begin{itemize}
\tightlist
\item
  \texttt{\ gap\ } :
  \href{https://typst.app/docs/reference/layout/length/}{\texttt{\ length\ }}
  or \texttt{\ none\ } â€'' The gap between repeated items.
\item
  \texttt{\ justify\ } :
  \href{https://typst.app/docs/reference/foundations/bool/}{\texttt{\ bool\ }}
  â€'' Whether to increase the gap to justify the items.
\item
  \texttt{\ body\ } :
  \href{https://typst.app/docs/reference/foundations/content/}{\texttt{\ content\ }}
  â€'' The content to repeat.
\end{itemize}

\textbf{Returns:}
\href{https://typst.app/docs/reference/foundations/content/}{\texttt{\ content\ }}

\subsubsection{\texorpdfstring{\texttt{\ align-helper\ }}{ align-helper }}\label{align-helper}

Utility function to help with aligning multiple items.

\begin{Shaded}
\begin{Highlighting}[]
\NormalTok{\#let align{-}helper(state{-}key, what{-}to{-}measure, display) = \{ .. \}}
\end{Highlighting}
\end{Shaded}

\textbf{Arguments:}

\begin{itemize}
\tightlist
\item
  \texttt{\ state-key\ } :
  \href{https://typst.app/docs/reference/foundations/str/}{\texttt{\ str\ }}
  â€'' The key to use for the
  \href{https://typst.app/docs/reference/introspection/state/}{\texttt{\ state\ }}
  that keeps track of the maximum encountered width.
\item
  \texttt{\ what-to-measure\ } :
  \href{https://typst.app/docs/reference/foundations/content/}{\texttt{\ content\ }}
  â€'' The content to measure at this call.
\item
  \texttt{\ display\ } :
  \href{https://typst.app/docs/reference/foundations/function/}{\texttt{\ function\ }}
  â€'' A callback which gets passed the maximum encountered width and
  the width of the current item (what was given to
  \texttt{\ what-to-measure\ } ), both as
  \href{https://typst.app/docs/reference/layout/length/}{\texttt{\ length\ }}
  , and should return
  \href{https://typst.app/docs/reference/foundations/content/}{\texttt{\ content\ }}
  which can make use of these widths for alignment.
\end{itemize}

\textbf{Returns:}
\href{https://typst.app/docs/reference/foundations/content/}{\texttt{\ content\ }}

\subsubsection{How to add}\label{how-to-add}

Copy this into your project and use the import as
\texttt{\ outrageous\ }

\begin{verbatim}
#import "@preview/outrageous:0.3.0"
\end{verbatim}

\includesvg[width=0.16667in,height=0.16667in]{/assets/icons/16-copy.svg}

Check the docs for
\href{https://typst.app/docs/reference/scripting/\#packages}{more
information on how to import packages} .

\subsubsection{About}\label{about}

\begin{description}
\tightlist
\item[Author :]
RubixDev
\item[License:]
GPL-3.0-only
\item[Current version:]
0.3.0
\item[Last updated:]
October 21, 2024
\item[First released:]
October 9, 2023
\item[Minimum Typst version:]
0.11.0
\item[Archive size:]
15.8 kB
\href{https://packages.typst.org/preview/outrageous-0.3.0.tar.gz}{\pandocbounded{\includesvg[keepaspectratio]{/assets/icons/16-download.svg}}}
\item[Repository:]
\href{https://github.com/RubixDev/typst-outrageous}{GitHub}
\end{description}

\subsubsection{Where to report issues?}\label{where-to-report-issues}

This package is a project of RubixDev . Report issues on
\href{https://github.com/RubixDev/typst-outrageous}{their repository} .
You can also try to ask for help with this package on the
\href{https://forum.typst.app}{Forum} .

Please report this package to the Typst team using the
\href{https://typst.app/contact}{contact form} if you believe it is a
safety hazard or infringes upon your rights.

\phantomsection\label{versions}
\subsubsection{Version history}\label{version-history}

\begin{longtable}[]{@{}ll@{}}
\toprule\noalign{}
Version & Release Date \\
\midrule\noalign{}
\endhead
\bottomrule\noalign{}
\endlastfoot
0.3.0 & October 21, 2024 \\
\href{https://typst.app/universe/package/outrageous/0.2.0/}{0.2.0} &
September 14, 2024 \\
\href{https://typst.app/universe/package/outrageous/0.1.0/}{0.1.0} &
October 9, 2023 \\
\end{longtable}

Typst GmbH did not create this package and cannot guarantee correct
functionality of this package or compatibility with any version of the
Typst compiler or app.


\title{typst.app/universe/package/glossarium}

\phantomsection\label{banner}
\section{glossarium}\label{glossarium}

{ 0.5.1 }

Glossarium is a simple, easily customizable typst glossary.

{ } Featured Package

\phantomsection\label{readme}
\begin{quote}
{[}!TIP{]} Glossarium is based in great part of the work of
\href{https://github.com/Dherse}{Sébastien d’Herbais de Thun} from
his master thesis available at:
\url{https://github.com/Dherse/masterproef} . His glossary is available
under the MIT license
\href{https://github.com/Dherse/masterproef/blob/main/elems/acronyms.typ}{here}
.
\end{quote}

Glossarium is a simple, easily customizable typst glossary inspired by
\href{https://www.ctan.org/pkg/glossaries}{LaTeX glossaries package} .
You can see various examples showcasing the different features in the
\texttt{\ examples\ } folder.

\pandocbounded{\includegraphics[keepaspectratio]{https://github.com/typst/packages/raw/main/packages/preview/glossarium/0.5.1/.github/example.png}}

\subsection{Manual}\label{manual}

\subsection{Fast start}\label{fast-start}

\begin{Shaded}
\begin{Highlighting}[]
\NormalTok{\#import "@preview/glossarium:0.5.1": make{-}glossary, register{-}glossary, print{-}glossary, gls, glspl}
\NormalTok{\#show: make{-}glossary}
\NormalTok{\#let entry{-}list = (}
\NormalTok{  (}
\NormalTok{    key: "kuleuven",}
\NormalTok{    short: "KU Leuven",}
\NormalTok{    long: "Katholieke Universiteit Leuven",}
\NormalTok{    description: "A university in Belgium.",}
\NormalTok{  ),}
\NormalTok{  // Add more terms}
\NormalTok{)}
\NormalTok{\#register{-}glossary(entry{-}list)}
\NormalTok{// Your document body}
\NormalTok{\#print{-}glossary(}
\NormalTok{ entry{-}list}
\NormalTok{)}
\end{Highlighting}
\end{Shaded}

\subsubsection{Import and setup}\label{import-and-setup}

This manual assume you have a good enough understanding of typst markup
and scripting.

For Typst 0.6.0 or later import the package from the typst preview
repository:

\begin{Shaded}
\begin{Highlighting}[]
\NormalTok{\#import "@preview/glossarium:0.5.1": make{-}glossary, register{-}glossary, print{-}glossary, gls, glspl}
\end{Highlighting}
\end{Shaded}

For Typst before 0.6.0 or to use \textbf{glossarium} as a local module,
download the package files into your project folder and import
\texttt{\ glossarium.typ\ } :

\begin{Shaded}
\begin{Highlighting}[]
\NormalTok{\#import "glossarium.typ": make{-}glossary, register{-}glossary, print{-}glossary, gls, glspl}
\end{Highlighting}
\end{Shaded}

After importing the package and before making any calls to
\texttt{\ gls\ } , \texttt{\ print-glossary\ } or \texttt{\ glspl\ } ,
please \emph{\textbf{MAKE SURE}} you add this line

\begin{Shaded}
\begin{Highlighting}[]
\NormalTok{\#show: make{-}glossary}
\end{Highlighting}
\end{Shaded}

\begin{quote}
\emph{WHY DO WE NEED THAT ?} : In order to be able to create references
to the terms in your glossary using typst ref syntax \texttt{\ @key\ }
glossarium needs to setup some
\href{https://typst.app/docs/tutorial/advanced-styling/}{show rules}
before any references exist. This is due to the way typst works, there
is no workaround.

Therefore I recommend that you always put the \texttt{\ \#show:\ ...\ }
statement on the line just below the \texttt{\ \#import\ } statement.
\end{quote}

\subsubsection{Registering the glossary}\label{registering-the-glossary}

First we have to define the terms. A term is a
\href{https://typst.app/docs/reference/types/dictionary/}{dictionary} as
follows:

\begin{longtable}[]{@{}llll@{}}
\toprule\noalign{}
Key & Type & Required/Optional & Description \\
\midrule\noalign{}
\endhead
\bottomrule\noalign{}
\endlastfoot
\texttt{\ key\ } & string & required & Case-sensitive, unique identifier
used to reference the term. \\
\texttt{\ short\ } & string & semi-optional & The short form of the term
replacing the term citation. \\
\texttt{\ long\ } & string or content & semi-optional & The long form of
the term, displayed in the glossary and on the first citation of the
term. \\
\texttt{\ description\ } & string or content & optional & The
description of the term. \\
\texttt{\ plural\ } & string or content & optional & The pluralized
short form of the term. \\
\texttt{\ longplural\ } & string or content & optional & The pluralized
long form of the term. \\
\texttt{\ group\ } & string & optional & Case-sensitive group the term
belongs to. The terms are displayed by groups in the glossary. \\
\end{longtable}

\begin{Shaded}
\begin{Highlighting}[]
\NormalTok{\#let entry{-}list = (}
\NormalTok{  // minimal term}
\NormalTok{  (}
\NormalTok{    key: "kuleuven",}
\NormalTok{    short: "KU Leuven"}
\NormalTok{  ),}
\NormalTok{  // a term with a long form and a group}
\NormalTok{  (}
\NormalTok{    key: "unamur",}
\NormalTok{    short: "UNamur",}
\NormalTok{    long: "Namur University",}
\NormalTok{    group: "Universities"}
\NormalTok{  ),}
\NormalTok{  // a term with a markup description}
\NormalTok{  (}
\NormalTok{    key: "oidc",}
\NormalTok{    short: "OIDC",}
\NormalTok{    long: "OpenID Connect",}
\NormalTok{    description: [}
\NormalTok{      OpenID is an open standard and decentralized authentication protocol promoted by the non{-}profit}
\NormalTok{      \#link("https://en.wikipedia.org/wiki/OpenID\#OpenID\_Foundation")[OpenID Foundation].}
\NormalTok{    ],}
\NormalTok{    group: "Acronyms",}
\NormalTok{  ),}
\NormalTok{  // a term with a short plural}
\NormalTok{  (}
\NormalTok{    key: "potato",}
\NormalTok{    short: "potato",}
\NormalTok{    // "plural" will be used when "short" should be pluralized}
\NormalTok{    plural: "potatoes",}
\NormalTok{    description: [\#lorem(10)],}
\NormalTok{  ),}
\NormalTok{  // a term with a long plural}
\NormalTok{  (}
\NormalTok{    key: "dm",}
\NormalTok{    short: "DM",}
\NormalTok{    long: "diagonal matrix",}
\NormalTok{    // "longplural" will be used when "long" should be pluralized}
\NormalTok{    longplural: "diagonal matrices",}
\NormalTok{    description: "Probably some math stuff idk",}
\NormalTok{  ),}
\NormalTok{)}
\end{Highlighting}
\end{Shaded}

Then the terms are passed as a list to \texttt{\ register-glossary\ }

\begin{Shaded}
\begin{Highlighting}[]
\NormalTok{\#register{-}glossary(entry{-}list)}
\end{Highlighting}
\end{Shaded}

\subsubsection{Printing the glossary}\label{printing-the-glossary}

Now, you can display the glossary using the \texttt{\ print-glossary\ }
function.

\begin{Shaded}
\begin{Highlighting}[]
\NormalTok{\#print{-}glossary(entry{-}list)}
\end{Highlighting}
\end{Shaded}

By default, the terms that are not referenced in the document are not
shown in the glossary, you can force their appearance by setting the
\texttt{\ show-all\ } argument to true.

You can also disable the back-references by setting the parameter
\texttt{\ disable-back-references\ } to \texttt{\ true\ } .

By default, group breaks use \texttt{\ linebreaks\ } . This behaviour
can be changed by setting the \texttt{\ user-group-break\ } parameter to
\texttt{\ pagebreak()\ } , or \texttt{\ colbreak()\ } , or any other
function that returns the \texttt{\ content\ } you want.

You can call this function from anywhere in your document.

\subsubsection{Referencing terms.}\label{referencing-terms.}

Referencing terms is done using the key of the terms using the
\texttt{\ gls\ } function or the reference syntax.

\begin{Shaded}
\begin{Highlighting}[]
\NormalTok{// referencing the OIDC term using gls}
\NormalTok{\#gls("oidc")}
\NormalTok{// displaying the long form forcibly}
\NormalTok{\#gls("oidc", long: true)}

\NormalTok{// referencing the OIDC term using the reference syntax}
\NormalTok{@oidc}
\end{Highlighting}
\end{Shaded}

\paragraph{Handling plurals}\label{handling-plurals}

You can use the \texttt{\ glspl\ } function and the references
supplements to pluralize terms. The \texttt{\ plural\ } key will be used
when \texttt{\ short\ } should be pluralized and \texttt{\ longplural\ }
will be used when \texttt{\ long\ } should be pluralized. If the
\texttt{\ plural\ } key is missing then glossarium will add an ‘s’
at the end of the short form as a fallback.

\begin{Shaded}
\begin{Highlighting}[]
\NormalTok{\#glspl("potato")}
\end{Highlighting}
\end{Shaded}

Please look at the examples regarding plurals.

\paragraph{Overriding the text shown}\label{overriding-the-text-shown}

You can also override the text displayed by setting the
\texttt{\ display\ } argument.

\begin{Shaded}
\begin{Highlighting}[]
\NormalTok{\#gls("oidc", display: "whatever you want")}
\end{Highlighting}
\end{Shaded}

\subsection{Final tips}\label{final-tips}

I recommend setting a show rule for the links to that your readers
understand that they can click on the references to go to the term in
the glossary.

\begin{Shaded}
\begin{Highlighting}[]
\NormalTok{\#show link: set text(fill: blue.darken(60\%))}
\NormalTok{// links are now blue !}
\end{Highlighting}
\end{Shaded}

\subsubsection{How to add}\label{how-to-add}

Copy this into your project and use the import as
\texttt{\ glossarium\ }

\begin{verbatim}
#import "@preview/glossarium:0.5.1"
\end{verbatim}

\includesvg[width=0.16667in,height=0.16667in]{/assets/icons/16-copy.svg}

Check the docs for
\href{https://typst.app/docs/reference/scripting/\#packages}{more
information on how to import packages} .

\subsubsection{About}\label{about}

\begin{description}
\tightlist
\item[Author s :]
slashformotion \& Dherse
\item[License:]
MIT
\item[Current version:]
0.5.1
\item[Last updated:]
October 28, 2024
\item[First released:]
July 31, 2023
\item[Archive size:]
10.5 kB
\href{https://packages.typst.org/preview/glossarium-0.5.1.tar.gz}{\pandocbounded{\includesvg[keepaspectratio]{/assets/icons/16-download.svg}}}
\item[Repository:]
\href{https://github.com/typst-community/glossarium}{GitHub}
\end{description}

\subsubsection{Where to report issues?}\label{where-to-report-issues}

This package is a project of slashformotion and Dherse . Report issues
on \href{https://github.com/typst-community/glossarium}{their
repository} . You can also try to ask for help with this package on the
\href{https://forum.typst.app}{Forum} .

Please report this package to the Typst team using the
\href{https://typst.app/contact}{contact form} if you believe it is a
safety hazard or infringes upon your rights.

\phantomsection\label{versions}
\subsubsection{Version history}\label{version-history}

\begin{longtable}[]{@{}ll@{}}
\toprule\noalign{}
Version & Release Date \\
\midrule\noalign{}
\endhead
\bottomrule\noalign{}
\endlastfoot
0.5.1 & October 28, 2024 \\
\href{https://typst.app/universe/package/glossarium/0.5.0/}{0.5.0} &
October 14, 2024 \\
\href{https://typst.app/universe/package/glossarium/0.4.2/}{0.4.2} &
October 7, 2024 \\
\href{https://typst.app/universe/package/glossarium/0.4.1/}{0.4.1} & May
29, 2024 \\
\href{https://typst.app/universe/package/glossarium/0.4.0/}{0.4.0} &
April 29, 2024 \\
\href{https://typst.app/universe/package/glossarium/0.3.0/}{0.3.0} &
April 8, 2024 \\
\href{https://typst.app/universe/package/glossarium/0.2.6/}{0.2.6} &
January 29, 2024 \\
\href{https://typst.app/universe/package/glossarium/0.2.5/}{0.2.5} &
December 3, 2023 \\
\href{https://typst.app/universe/package/glossarium/0.2.4/}{0.2.4} &
November 16, 2023 \\
\href{https://typst.app/universe/package/glossarium/0.2.3/}{0.2.3} &
October 30, 2023 \\
\href{https://typst.app/universe/package/glossarium/0.2.2/}{0.2.2} &
September 16, 2023 \\
\href{https://typst.app/universe/package/glossarium/0.2.1/}{0.2.1} &
September 3, 2023 \\
\href{https://typst.app/universe/package/glossarium/0.2.0/}{0.2.0} &
August 19, 2023 \\
\href{https://typst.app/universe/package/glossarium/0.1.0/}{0.1.0} &
July 31, 2023 \\
\end{longtable}

Typst GmbH did not create this package and cannot guarantee correct
functionality of this package or compatibility with any version of the
Typst compiler or app.


\title{typst.app/universe/package/stv-vub-huisstijl}

\phantomsection\label{banner}
\phantomsection\label{template-thumbnail}
\pandocbounded{\includegraphics[keepaspectratio]{https://packages.typst.org/preview/thumbnails/stv-vub-huisstijl-0.1.0-small.webp}}

\section{stv-vub-huisstijl}\label{stv-vub-huisstijl}

{ 0.1.0 }

An unofficial template to get the look of the Vrije Universiteit Brussel
(VUB) huisstijl in Typst.

\href{/app?template=stv-vub-huisstijl&version=0.1.0}{Create project in
app}

\phantomsection\label{readme}
An unofficial template to get the look of the
\href{https://www.vub.be/}{Vrije Universiteit Brussel (VUB)} huisstijl
in Typst based on \href{https://gitlab.com/rubdos/texlive-vub}{this
LaTeX template}

\subsection{Getting Started}\label{getting-started}

You can choose “Start from template� in the web app, and search for
\texttt{\ vub-huisstijl\ } .

If you are running Typst locally, you can use the following command to
initialize the template:

\begin{Shaded}
\begin{Highlighting}[]
\NormalTok{typst init @preview/stv{-}vub{-}huisstijl:0.1.0}
\end{Highlighting}
\end{Shaded}

\subsubsection{Fonts}\label{fonts}

The package makes use of the “TeX Gyre Adventor� font, with
“Roboto� as a fallback. These should be installed for the title page
to look right. They are available for free, and also come bundled with
texlive.

\subsection{Note}\label{note}

This only provides a template for a thesis title page, not for slides.
That can be added in the future.

\subsection{About the name}\label{about-the-name}

St V ( \href{https://en.wikipedia.org/wiki/Saint_Verhaegen}{Saint
Verhaegen} ) is an important part of the folklore of the VUB and the
ULB.

\href{/app?template=stv-vub-huisstijl&version=0.1.0}{Create project in
app}

\subsubsection{How to use}\label{how-to-use}

Click the button above to create a new project using this template in
the Typst app.

You can also use the Typst CLI to start a new project on your computer
using this command:

\begin{verbatim}
typst init @preview/stv-vub-huisstijl:0.1.0
\end{verbatim}

\includesvg[width=0.16667in,height=0.16667in]{/assets/icons/16-copy.svg}

\subsubsection{About}\label{about}

\begin{description}
\tightlist
\item[Author :]
\href{https://github.com/WannesMalfait}{Wannes Malfait}
\item[License:]
MIT
\item[Current version:]
0.1.0
\item[Last updated:]
October 21, 2024
\item[First released:]
October 21, 2024
\item[Archive size:]
8.02 kB
\href{https://packages.typst.org/preview/stv-vub-huisstijl-0.1.0.tar.gz}{\pandocbounded{\includesvg[keepaspectratio]{/assets/icons/16-download.svg}}}
\item[Repository:]
\href{https://github.com/WannesMalfait/vub-huisstijl-typst/}{GitHub}
\item[Categor y :]
\begin{itemize}
\tightlist
\item[]
\item
  \pandocbounded{\includesvg[keepaspectratio]{/assets/icons/16-mortarboard.svg}}
  \href{https://typst.app/universe/search/?category=thesis}{Thesis}
\end{itemize}
\end{description}

\subsubsection{Where to report issues?}\label{where-to-report-issues}

This template is a project of Wannes Malfait . Report issues on
\href{https://github.com/WannesMalfait/vub-huisstijl-typst/}{their
repository} . You can also try to ask for help with this template on the
\href{https://forum.typst.app}{Forum} .

Please report this template to the Typst team using the
\href{https://typst.app/contact}{contact form} if you believe it is a
safety hazard or infringes upon your rights.

\phantomsection\label{versions}
\subsubsection{Version history}\label{version-history}

\begin{longtable}[]{@{}ll@{}}
\toprule\noalign{}
Version & Release Date \\
\midrule\noalign{}
\endhead
\bottomrule\noalign{}
\endlastfoot
0.1.0 & October 21, 2024 \\
\end{longtable}

Typst GmbH did not create this template and cannot guarantee correct
functionality of this template or compatibility with any version of the
Typst compiler or app.


\title{typst.app/universe/package/aero-check}

\phantomsection\label{banner}
\phantomsection\label{template-thumbnail}
\pandocbounded{\includegraphics[keepaspectratio]{https://packages.typst.org/preview/thumbnails/aero-check-0.1.1-small.webp}}

\section{aero-check}\label{aero-check}

{ 0.1.1 }

A simple template to create checklists with an aviation inspired style.

{ } Featured Template

\href{/app?template=aero-check&version=0.1.1}{Create project in app}

\phantomsection\label{readme}
\pandocbounded{\includegraphics[keepaspectratio]{https://img.shields.io/github/v/release/TomVer99/Typst-checklist-template?style=flat-square}}
\pandocbounded{\includegraphics[keepaspectratio]{https://img.shields.io/github/stars/TomVer99/Typst-checklist-template?style=flat-square}}

\pandocbounded{\includegraphics[keepaspectratio]{https://img.shields.io/maintenance/Yes/2024?style=flat-square}}

This template is meant to create checklists with a style inspired by
aviation checklists.

It includes 2 different styles!

\subsection{Usage}\label{usage}

Start your checklist with the following code:

\begin{Shaded}
\begin{Highlighting}[]
\NormalTok{\#import "@preview/aero{-}check:0.1.1": *}

\NormalTok{\#show: checklist.with(}
\NormalTok{  title: "Title",}
\NormalTok{  // disclaimer: "",}
\NormalTok{  // 0 or 1}
\NormalTok{  // style: 0,}
\NormalTok{)}
\end{Highlighting}
\end{Shaded}

You can then add items to your checklist with the following code:

\begin{Shaded}
\begin{Highlighting}[]
\NormalTok{\#topic("Topic")[}
\NormalTok{  \#section("Section")[}
\NormalTok{    \#step("Step", "Check")}
\NormalTok{    \#step("Step", "Check")}
\NormalTok{    \#step("Step", "Check")}
\NormalTok{    \#step("Step", "Check")}
\NormalTok{  ]}
\NormalTok{\#section("Section")[}
\NormalTok{    \#step("Step", "Check")}
\NormalTok{    \#step("Step", "Check")}
\NormalTok{    \#step("Step", "Check")}
\NormalTok{    \#step("Step", "Check")}
\NormalTok{  ]}
\NormalTok{]}
\end{Highlighting}
\end{Shaded}

And you can use \texttt{\ \#colbreak()\ } to add a column break.

\href{/app?template=aero-check&version=0.1.1}{Create project in app}

\subsubsection{How to use}\label{how-to-use}

Click the button above to create a new project using this template in
the Typst app.

You can also use the Typst CLI to start a new project on your computer
using this command:

\begin{verbatim}
typst init @preview/aero-check:0.1.1
\end{verbatim}

\includesvg[width=0.16667in,height=0.16667in]{/assets/icons/16-copy.svg}

\subsubsection{About}\label{about}

\begin{description}
\tightlist
\item[Author :]
TomVer99
\item[License:]
MIT
\item[Current version:]
0.1.1
\item[Last updated:]
September 14, 2024
\item[First released:]
May 13, 2024
\item[Minimum Typst version:]
0.11.0
\item[Archive size:]
2.59 kB
\href{https://packages.typst.org/preview/aero-check-0.1.1.tar.gz}{\pandocbounded{\includesvg[keepaspectratio]{/assets/icons/16-download.svg}}}
\item[Repository:]
\href{https://github.com/TomVer99/Typst-checklist-template}{GitHub}
\item[Categor y :]
\begin{itemize}
\tightlist
\item[]
\item
  \pandocbounded{\includesvg[keepaspectratio]{/assets/icons/16-hammer.svg}}
  \href{https://typst.app/universe/search/?category=utility}{Utility}
\end{itemize}
\end{description}

\subsubsection{Where to report issues?}\label{where-to-report-issues}

This template is a project of TomVer99 . Report issues on
\href{https://github.com/TomVer99/Typst-checklist-template}{their
repository} . You can also try to ask for help with this template on the
\href{https://forum.typst.app}{Forum} .

Please report this template to the Typst team using the
\href{https://typst.app/contact}{contact form} if you believe it is a
safety hazard or infringes upon your rights.

\phantomsection\label{versions}
\subsubsection{Version history}\label{version-history}

\begin{longtable}[]{@{}ll@{}}
\toprule\noalign{}
Version & Release Date \\
\midrule\noalign{}
\endhead
\bottomrule\noalign{}
\endlastfoot
0.1.1 & September 14, 2024 \\
\href{https://typst.app/universe/package/aero-check/0.1.0/}{0.1.0} & May
13, 2024 \\
\end{longtable}

Typst GmbH did not create this template and cannot guarantee correct
functionality of this template or compatibility with any version of the
Typst compiler or app.


\title{typst.app/universe/package/sweet-graduate-resume}

\phantomsection\label{banner}
\phantomsection\label{template-thumbnail}
\pandocbounded{\includegraphics[keepaspectratio]{https://packages.typst.org/preview/thumbnails/sweet-graduate-resume-0.1.0-small.webp}}

\section{sweet-graduate-resume}\label{sweet-graduate-resume}

{ 0.1.0 }

A simple graduate student resume template

\href{/app?template=sweet-graduate-resume&version=0.1.0}{Create project
in app}

\phantomsection\label{readme}
A basic resume template in typst.

To compile/watch, make sure you pass the argument
\texttt{\ -\/-font-path\ ./fonts/\ } to typst-cli.

If you use the helix editor, a configuration has been given in the
repository. This auto-compiles your document on saving and you can
preview the results in real time using a pdf viewer. Also provides
autocomplete, code renaming, and other cool (LSP) features.

Make sure you have tinymist and typstyle installed before using the
helix config.

\subsection{Preview}\label{preview}

\pandocbounded{\includegraphics[keepaspectratio]{https://github.com/typst/packages/raw/main/packages/preview/sweet-graduate-resume/0.1.0/screenshot.png}}

\subsection{LICENSE}\label{license}

The FontAwesome Free/Brand fonts are licensed under
\href{https://github.com/FortAwesome/Font-Awesome?tab=License-1-ov-file}{Font
Awesome Free License}

The Codeberg SVG in the svg directory are licensed under
\href{https://codeberg.org/Codeberg/Design/src/commit/ac514aa9aaa2457d4af3c3e13df3ab136d22a49a/LICENSE}{Creative
Commons CC0}

For the rest, see
\href{https://github.com/typst/packages/raw/main/packages/preview/sweet-graduate-resume/0.1.0/LICENSE}{LICENSE}
.

\subsection{Fonts}\label{fonts}

For fonts, please install the fonts from the repository of the project
in codeberg.

\href{/app?template=sweet-graduate-resume&version=0.1.0}{Create project
in app}

\subsubsection{How to use}\label{how-to-use}

Click the button above to create a new project using this template in
the Typst app.

You can also use the Typst CLI to start a new project on your computer
using this command:

\begin{verbatim}
typst init @preview/sweet-graduate-resume:0.1.0
\end{verbatim}

\includesvg[width=0.16667in,height=0.16667in]{/assets/icons/16-copy.svg}

\subsubsection{About}\label{about}

\begin{description}
\tightlist
\item[Author :]
\href{https://innocent_zero.codeberg.page}{innocentzero (Md Isfarul
Haque)}
\item[License:]
MIT
\item[Current version:]
0.1.0
\item[Last updated:]
July 4, 2024
\item[First released:]
July 4, 2024
\item[Archive size:]
47.6 kB
\href{https://packages.typst.org/preview/sweet-graduate-resume-0.1.0.tar.gz}{\pandocbounded{\includesvg[keepaspectratio]{/assets/icons/16-download.svg}}}
\item[Repository:]
\href{https://codeberg.org/innocent_zero/typst-resume}{Codeberg}
\item[Discipline :]
\begin{itemize}
\tightlist
\item[]
\item
  \href{https://typst.app/universe/search/?discipline=engineering}{Engineering}
\end{itemize}
\item[Categor y :]
\begin{itemize}
\tightlist
\item[]
\item
  \pandocbounded{\includesvg[keepaspectratio]{/assets/icons/16-user.svg}}
  \href{https://typst.app/universe/search/?category=cv}{CV}
\end{itemize}
\end{description}

\subsubsection{Where to report issues?}\label{where-to-report-issues}

This template is a project of innocentzero (Md Isfarul Haque) . Report
issues on \href{https://codeberg.org/innocent_zero/typst-resume}{their
repository} . You can also try to ask for help with this template on the
\href{https://forum.typst.app}{Forum} .

Please report this template to the Typst team using the
\href{https://typst.app/contact}{contact form} if you believe it is a
safety hazard or infringes upon your rights.

\phantomsection\label{versions}
\subsubsection{Version history}\label{version-history}

\begin{longtable}[]{@{}ll@{}}
\toprule\noalign{}
Version & Release Date \\
\midrule\noalign{}
\endhead
\bottomrule\noalign{}
\endlastfoot
0.1.0 & July 4, 2024 \\
\end{longtable}

Typst GmbH did not create this template and cannot guarantee correct
functionality of this template or compatibility with any version of the
Typst compiler or app.


