\title{typst.app/universe/package/abbr}

\phantomsection\label{banner}
\section{abbr}\label{abbr}

{ 0.1.0 }

An Abbreviations package.

\phantomsection\label{readme}
Short package for making the handling of abbreviations, acronyms, and
initialisms \emph{easy} .

Declare your abbreviations anywhere, use everywhere â€`` they get picked
up automatically.

\subsection{Features}\label{features}

\begin{itemize}
\tightlist
\item
  Automatic plurals, with optional overrides.
\item
  Automatic 1- or 2-column sorted list of abbreviations
\item
  Automatic links to list of abbreviations, if included.
\item
  styling configuration
\end{itemize}

\subsection{Getting started}\label{getting-started}

\begin{Shaded}
\begin{Highlighting}[]
\NormalTok{\#import "@preview/abbr:0.1.0"}

\NormalTok{\#abbr.list()}
\NormalTok{\#abbr.make(}
\NormalTok{  ("PDE", "Partial Differential Equation"),}
\NormalTok{  ("BC", "Boundary Condition"),}
\NormalTok{  ("DOF", "Degree of Freedom", "Degrees of Freedom"),}
\NormalTok{)}

\NormalTok{= Constrained Equations}

\NormalTok{\#abbr.pla[BC] constrain the \#abbr.pla[DOF] of the \#abbr.pla[PDE] they act on.\textbackslash{}}
\NormalTok{\#abbr.pla[BC] constrain the \#abbr.pla[DOF] of the \#abbr.pla[PDE] they act on.}

\NormalTok{\#abbr.add("MOL", "Method of Lines")}
\NormalTok{The \#abbr.a[MOL] is a procedure to solve \#abbr.pla[PDE] in time.}
\end{Highlighting}
\end{Shaded}

\pandocbounded{\includegraphics[keepaspectratio]{https://github.com/typst/packages/raw/main/packages/preview/abbr/0.1.0/example.png}}

\subsection{API Reference}\label{api-reference}

\subsubsection{Configuration}\label{configuration}

\begin{itemize}
\tightlist
\item
  \textbf{style} \texttt{\ (func)\ }\\
  Set a callable for styling the short version in the text.
\end{itemize}

\subsubsection{Creation}\label{creation}

\begin{itemize}
\item
  \textbf{add} \texttt{\ (short,\ long,\ long-plural)\ }\\
  Add single entry to use later.\\
  \texttt{\ long-plural\ } is \emph{optional} , if not given but used,
  an \texttt{\ s\ } is appended to create a plural.
\item
  \textbf{make} \texttt{\ (list,\ of,\ entries)\ }\\
  Add multiple entries, each of the form
  \texttt{\ (short,\ long,\ long-plural)\ } .
\end{itemize}

\subsubsection{Listing}\label{listing}

\begin{itemize}
\tightlist
\item
  \textbf{list} \texttt{\ (title)\ }\\
  Create an outline with all abbreviations in short and expanded form
\end{itemize}

\subsubsection{Usage}\label{usage}

\begin{itemize}
\tightlist
\item
  \textbf{s} \texttt{\ ()\ } - short\\
  force short form of abbreviation
\item
  \textbf{l} \texttt{\ ()\ } - long\\
  force long form of abbreviation
\item
  \textbf{a} \texttt{\ ()\ } - auto\\
  first occurence will be long form, the rest short
\item
  \textbf{pls} \texttt{\ ()\ } - plural short\\
  plural short form
\item
  \textbf{pll} \texttt{\ ()\ } - plural long\\
  plural long form
\item
  \textbf{pl} \texttt{\ ()\ } - plural automatic\\
  plural. first occurence long form, then short
\end{itemize}

\subsection{Why yet another Abbreviations
package?}\label{why-yet-another-abbreviations-package}

This mostly exists because I started working on it before checking if
somebody already made a package for it. After I saw that e.g.
\texttt{\ acrotastic\ } exists, I kept convincing myself a new package
still makes sense for the following reasons:

\begin{itemize}
\tightlist
\item
  Getting to know Typst
\item
  More automatic handling than other packages
\item
  Ability to keep keys as {[}Content{]} instead of having to stringify
  everything
\end{itemize}

Especially the last part seems to lower the friction of writing for me.
It seems silly, I know.

\subsection{Contributing}\label{contributing}

Please head over to the \href{https://sr.ht/~slowjo/typst-packages}{hub}
to find the mailing list and ticket tracker.

Or simply reach out on IRC (
\href{https://web.libera.chat/gamja/?autojoin=\#typst}{\#typst on
libera.chat} )!

\subsubsection{How to add}\label{how-to-add}

Copy this into your project and use the import as \texttt{\ abbr\ }

\begin{verbatim}
#import "@preview/abbr:0.1.0"
\end{verbatim}

\includesvg[width=0.16667in,height=0.16667in]{/assets/icons/16-copy.svg}

Check the docs for
\href{https://typst.app/docs/reference/scripting/\#packages}{more
information on how to import packages} .

\subsubsection{About}\label{about}

\begin{description}
\tightlist
\item[Author :]
\href{mailto:slowjo@halmen.xyz}{Jonathan Halmen}
\item[License:]
MIT
\item[Current version:]
0.1.0
\item[Last updated:]
November 5, 2024
\item[First released:]
November 5, 2024
\item[Archive size:]
3.40 kB
\href{https://packages.typst.org/preview/abbr-0.1.0.tar.gz}{\pandocbounded{\includesvg[keepaspectratio]{/assets/icons/16-download.svg}}}
\item[Repository:]
\href{https://git.sr.ht/~slowjo/typst-abbr}{git.sr.ht}
\item[Categor y :]
\begin{itemize}
\tightlist
\item[]
\item
  \pandocbounded{\includesvg[keepaspectratio]{/assets/icons/16-list-unordered.svg}}
  \href{https://typst.app/universe/search/?category=model}{Model}
\end{itemize}
\end{description}

\subsubsection{Where to report issues?}\label{where-to-report-issues}

This package is a project of Jonathan Halmen . Report issues on
\href{https://git.sr.ht/~slowjo/typst-abbr}{their repository} . You can
also try to ask for help with this package on the
\href{https://forum.typst.app}{Forum} .

Please report this package to the Typst team using the
\href{https://typst.app/contact}{contact form} if you believe it is a
safety hazard or infringes upon your rights.

\phantomsection\label{versions}
\subsubsection{Version history}\label{version-history}

\begin{longtable}[]{@{}ll@{}}
\toprule\noalign{}
Version & Release Date \\
\midrule\noalign{}
\endhead
\bottomrule\noalign{}
\endlastfoot
0.1.0 & November 5, 2024 \\
\end{longtable}

Typst GmbH did not create this package and cannot guarantee correct
functionality of this package or compatibility with any version of the
Typst compiler or app.


\title{typst.app/universe/package/kouhu}

\phantomsection\label{banner}
\section{kouhu}\label{kouhu}

{ 0.1.1 }

Chinese lipsum text generator; 中æ--‡ä¹±æ•°å?‡æ--‡ï¼ˆLorem
Ipsum)ç''Ÿæˆ?器

\phantomsection\label{readme}
\texttt{\ kouhu\ } is a Chinese lipsum text generator for
\href{https://typst.app/}{Typst} . It provides a set of built-in text
samples containing both Simplified and Traditional Chinese characters.
You can choose from generated fake text, classic or modern Chinese
literature, or specify your own text.

\texttt{\ kouhu\ } is inspired by
\href{https://ctan.org/pkg/zhlipsum}{zhlipsum} LaTeX package and
\href{https://typst.app/universe/package/roremu}{roremu} Typst package.

All
\href{https://github.com/typst/packages/raw/main/packages/preview/kouhu/0.1.1/data/zhlipsum.json}{sample
text} is excerpted from \texttt{\ zhlipsum\ } LaTeX package (see
Appendix for details).

\subsection{Usage}\label{usage}

\begin{Shaded}
\begin{Highlighting}[]
\NormalTok{\#import "@preview/kouhu:0.1.0": kouhu}

\NormalTok{\#kouhu(indicies: range(1, 3)) // select the first 3 paragraphs from default text}

\NormalTok{\#kouhu(builtin{-}text: "zhufu", offset: 5, length: 100) // select 100 characters from the 5th paragraph of "zhufu" text}

\NormalTok{\#kouhu(custom{-}text: ("Foo", "Bar")) // provide your own text}
\end{Highlighting}
\end{Shaded}

See
\href{https://github.com/Harry-Chen/kouhu/blob/master/doc/manual.pdf}{manual}
for more details.

\subsection{\texorpdfstring{What does \texttt{\ kouhu\ }
mean?}{What does  kouhu  mean?}}\label{what-does-kouhu-mean}

GitHub Copilot says:

\begin{quote}
\texttt{\ kouhu\ } (�胡) is a Chinese term for reading aloud without
understanding the meaning. It is often used in the context of learning
Chinese language or reciting Chinese literature.
\end{quote}

\subsection{Changelog}\label{changelog}

\subsubsection{0.1.1}\label{section}

\begin{itemize}
\tightlist
\item
  Fix some wrong paths in \texttt{\ README.md\ } .
\item
  Fix genearation of \texttt{\ indicies\ } when not specified by user.
\item
  Add repetition of text until \texttt{\ length\ } is reached.
\end{itemize}

\subsubsection{0.1.0}\label{section-1}

\begin{itemize}
\tightlist
\item
  Initial release.
\end{itemize}

\subsection{Appendix}\label{appendix}

\subsubsection{\texorpdfstring{Generating
\texttt{\ zhlipsum.json\ }}{Generating  zhlipsum.json }}\label{generating-zhlipsum.json}

First download the \texttt{\ zhlipsum-text.dtx\ } from
\href{https://ctan.org/pkg/zhlipsum}{CTAN} or from local TeX Live (
\texttt{\ kpsewhich\ zhlipsum-text.dtx\ } ). Then run:

\begin{Shaded}
\begin{Highlighting}[]
\ExtensionTok{python3}\NormalTok{ utils/generate\_zhlipsum.py /path/to/zhlipsum{-}text.dtx src/zhlipsum.json}
\end{Highlighting}
\end{Shaded}

\subsubsection{How to add}\label{how-to-add}

Copy this into your project and use the import as \texttt{\ kouhu\ }

\begin{verbatim}
#import "@preview/kouhu:0.1.1"
\end{verbatim}

\includesvg[width=0.16667in,height=0.16667in]{/assets/icons/16-copy.svg}

Check the docs for
\href{https://typst.app/docs/reference/scripting/\#packages}{more
information on how to import packages} .

\subsubsection{About}\label{about}

\begin{description}
\tightlist
\item[Author :]
\href{mailto:harry-chen@outlook.com}{Shengqi Chen}
\item[License:]
MIT
\item[Current version:]
0.1.1
\item[Last updated:]
September 30, 2024
\item[First released:]
September 27, 2024
\item[Archive size:]
905 kB
\href{https://packages.typst.org/preview/kouhu-0.1.1.tar.gz}{\pandocbounded{\includesvg[keepaspectratio]{/assets/icons/16-download.svg}}}
\item[Repository:]
\href{https://github.com/Harry-Chen/kouhu}{GitHub}
\item[Categor y :]
\begin{itemize}
\tightlist
\item[]
\item
  \pandocbounded{\includesvg[keepaspectratio]{/assets/icons/16-hammer.svg}}
  \href{https://typst.app/universe/search/?category=utility}{Utility}
\end{itemize}
\end{description}

\subsubsection{Where to report issues?}\label{where-to-report-issues}

This package is a project of Shengqi Chen . Report issues on
\href{https://github.com/Harry-Chen/kouhu}{their repository} . You can
also try to ask for help with this package on the
\href{https://forum.typst.app}{Forum} .

Please report this package to the Typst team using the
\href{https://typst.app/contact}{contact form} if you believe it is a
safety hazard or infringes upon your rights.

\phantomsection\label{versions}
\subsubsection{Version history}\label{version-history}

\begin{longtable}[]{@{}ll@{}}
\toprule\noalign{}
Version & Release Date \\
\midrule\noalign{}
\endhead
\bottomrule\noalign{}
\endlastfoot
0.1.1 & September 30, 2024 \\
\href{https://typst.app/universe/package/kouhu/0.1.0/}{0.1.0} &
September 27, 2024 \\
\end{longtable}

Typst GmbH did not create this package and cannot guarantee correct
functionality of this package or compatibility with any version of the
Typst compiler or app.


\title{typst.app/universe/package/ttt-exam}

\phantomsection\label{banner}
\phantomsection\label{template-thumbnail}
\pandocbounded{\includegraphics[keepaspectratio]{https://packages.typst.org/preview/thumbnails/ttt-exam-0.1.2-small.webp}}

\section{ttt-exam}\label{ttt-exam}

{ 0.1.2 }

A collection of tools to make a teachers life easier (german).

\href{/app?template=ttt-exam&version=0.1.2}{Create project in app}

\phantomsection\label{readme}
\texttt{\ ttt-exam\ } is a \emph{template} to create exams and belongs
to the
\href{https://github.com/jomaway/typst-teacher-templates}{typst-teacher-tools-collection}
.

\subsection{Usage}\label{usage}

Run this command inside your terminal to init a new exam.

\begin{Shaded}
\begin{Highlighting}[]
\ExtensionTok{typst}\NormalTok{ init @preview/ttt{-}exam my{-}exam}
\end{Highlighting}
\end{Shaded}

This will scaffold the following folder structure.

\begin{Shaded}
\begin{Highlighting}[]
\NormalTok{my{-}exam/}
\NormalTok{├─ meta.toml}
\NormalTok{├─ exam.typ}
\NormalTok{├─ eval.typ}
\NormalTok{├─ justfile}
\NormalTok{└─ logo.jpg}
\end{Highlighting}
\end{Shaded}

Replace the \texttt{\ logo.jpg\ } with your schools, university, …
logo or remove it. Then edit the \texttt{\ meta.toml\ } . Edit the
\texttt{\ exam.typ\ } and replace the questions with your own. If you
like you can also remove the \texttt{\ meta.toml\ } file and specify the
values directly inside \texttt{\ exam.typ\ }

If you have installed \href{https://just.systems/}{just} you can use it
to build a \emph{student} and \emph{teacher} version of your exam by
running \texttt{\ just\ build\ } .

Here you can see an example with both versions. On the left the student
version and on the right the teachers version.

\pandocbounded{\includegraphics[keepaspectratio]{https://raw.githubusercontent.com/jomaway/typst-teacher-templates/main/ttt-exam/thumbnail.png}}

The \texttt{\ eval.typ\ } is a template for generating grade lists. You
need to add your students to \texttt{\ meta.toml\ } and add the total
amount of points.

\subsection{Features}\label{features}

You can pass the following arguments to \texttt{\ exam\ }

\begin{Shaded}
\begin{Highlighting}[]
\NormalTok{\#let exam(}
\NormalTok{  // metadata }
\NormalTok{  logo: none, // none | image}
\NormalTok{  title: "exam", // the title of the exam}
\NormalTok{  subtitle: none, // is shown below the title}
\NormalTok{  date: none,     // date of the exam, preferred type of datetime.}
\NormalTok{  class: "",      }
\NormalTok{  subject: "" ,}
\NormalTok{  authors: "",  // string | array}
\NormalTok{  // config}
\NormalTok{  solution: auto,  // if solutions are displayed can also be specified with \textasciigrave{}{-}{-}input solution=true\textasciigrave{} on the cli.}
\NormalTok{   cover: true, // true | false}
\NormalTok{   header: auto, // true | false | auto}
\NormalTok{   eval{-}table: false,  // true | false}
\NormalTok{   appendix: none, // content | none}
\NormalTok{)}
\end{Highlighting}
\end{Shaded}

\href{/app?template=ttt-exam&version=0.1.2}{Create project in app}

\subsubsection{How to use}\label{how-to-use}

Click the button above to create a new project using this template in
the Typst app.

You can also use the Typst CLI to start a new project on your computer
using this command:

\begin{verbatim}
typst init @preview/ttt-exam:0.1.2
\end{verbatim}

\includesvg[width=0.16667in,height=0.16667in]{/assets/icons/16-copy.svg}

\subsubsection{About}\label{about}

\begin{description}
\tightlist
\item[Author :]
\href{https://github.com/jomaway}{Jomaway}
\item[License:]
MIT
\item[Current version:]
0.1.2
\item[Last updated:]
May 23, 2024
\item[First released:]
April 2, 2024
\item[Minimum Typst version:]
0.11.0
\item[Archive size:]
152 kB
\href{https://packages.typst.org/preview/ttt-exam-0.1.2.tar.gz}{\pandocbounded{\includesvg[keepaspectratio]{/assets/icons/16-download.svg}}}
\item[Repository:]
\href{https://github.com/jomaway/typst-teacher-templates}{GitHub}
\item[Discipline :]
\begin{itemize}
\tightlist
\item[]
\item
  \href{https://typst.app/universe/search/?discipline=education}{Education}
\end{itemize}
\item[Categor y :]
\begin{itemize}
\tightlist
\item[]
\item
  \pandocbounded{\includesvg[keepaspectratio]{/assets/icons/16-envelope.svg}}
  \href{https://typst.app/universe/search/?category=office}{Office}
\end{itemize}
\end{description}

\subsubsection{Where to report issues?}\label{where-to-report-issues}

This template is a project of Jomaway . Report issues on
\href{https://github.com/jomaway/typst-teacher-templates}{their
repository} . You can also try to ask for help with this template on the
\href{https://forum.typst.app}{Forum} .

Please report this template to the Typst team using the
\href{https://typst.app/contact}{contact form} if you believe it is a
safety hazard or infringes upon your rights.

\phantomsection\label{versions}
\subsubsection{Version history}\label{version-history}

\begin{longtable}[]{@{}ll@{}}
\toprule\noalign{}
Version & Release Date \\
\midrule\noalign{}
\endhead
\bottomrule\noalign{}
\endlastfoot
0.1.2 & May 23, 2024 \\
\href{https://typst.app/universe/package/ttt-exam/0.1.0/}{0.1.0} & April
2, 2024 \\
\end{longtable}

Typst GmbH did not create this template and cannot guarantee correct
functionality of this template or compatibility with any version of the
Typst compiler or app.


\title{typst.app/universe/package/chic-hdr}

\phantomsection\label{banner}
\section{chic-hdr}\label{chic-hdr}

{ 0.4.0 }

Typst package for creating elegant headers and footers

\phantomsection\label{readme}
\textbf{Chic-header} (chic-hdr) is a Typst package for creating elegant
headers and footers

\subsection{Usage}\label{usage}

To use this library through the Typst package manager (for Typst 0.6.0
or greater), write \texttt{\ \#import\ "@preview/chic-hdr:0.4.0":\ *\ }
at the beginning of your Typst file. Once imported, you can start using
the package by writing the instruction \texttt{\ \#show:\ chic.with()\ }
and giving any of the chic functions inside the parenthesis
\texttt{\ ()\ } .

\emph{\textbf{Important: If you are using a custom template that also
needs the \texttt{\ \#show\ } instruction to be applied, prefer to use
\texttt{\ \#show:\ chic()\ } after the template’s \texttt{\ \#show\ }
.}}

For example, the code below…

\begin{Shaded}
\begin{Highlighting}[]
\NormalTok{\#import "@preview/chic{-}hdr:0.4.0": *}

\NormalTok{\#set page(paper: "a7")}

\NormalTok{\#show: chic.with(}
\NormalTok{  chic{-}footer(}
\NormalTok{    left{-}side: strong(}
\NormalTok{        link("mailto:admin@chic.hdr", "admin@chic.hdr")}
\NormalTok{    ),}
\NormalTok{    right{-}side: chic{-}page{-}number()}
\NormalTok{  ),}
\NormalTok{  chic{-}header(}
\NormalTok{    left{-}side: emph(chic{-}heading{-}name(fill: true)),}
\NormalTok{    right{-}side: smallcaps("Example")}
\NormalTok{  ),}
\NormalTok{  chic{-}separator(1pt),}
\NormalTok{  chic{-}offset(7pt),}
\NormalTok{  chic{-}height(1.5cm)}
\NormalTok{)}

\NormalTok{= Introduction}
\NormalTok{\#lorem(30)}

\NormalTok{== Details}
\NormalTok{\#lorem(70)}
\end{Highlighting}
\end{Shaded}

…will look like this:

\subsubsection{\texorpdfstring{\protect\pandocbounded{\includegraphics[keepaspectratio]{https://github.com/typst/packages/raw/main/packages/preview/chic-hdr/0.4.0/assets/usage.png}}}{Usage example}}\label{usage-example}

\subsection{Reference}\label{reference}

\emph{Note: For a detailed explanation of the functions and parameters,
see Chic-header’s Manual.pdf.}

While using \texttt{\ \#show:\ chic.with()\ } , you can give the
following parameters inside the parenthesis:

\begin{itemize}
\tightlist
\item
  \texttt{\ width\ } : Indicates the with of headers and footers in all
  the document (default is \texttt{\ 100\%\ } ).
\item
  \texttt{\ skip\ } : Which pages must be skipped for setting its header
  and footer. Other properties changed with \texttt{\ chic-height()\ }
  or \texttt{\ chic-offset()\ } are preserved. Giving a negative index
  causes a skip of the last pages using last page as index -1(default is
  \texttt{\ ()\ } ).
\item
  \texttt{\ even\ } : Header and footer for even pages. Here, only
  \texttt{\ chic-header()\ } , \texttt{\ chic-footer()\ } and
  \texttt{\ chic-separator()\ } functions will take effect. Other
  functions must be given as an argument of \texttt{\ chic()\ } .
\item
  \texttt{\ odd\ } : Sets the header and footer for odd pages. Here,
  only \texttt{\ chic-header()\ } , \texttt{\ chic-footer()\ } and
  \texttt{\ chic-separator()\ } functions will take effect. Other
  functions must be given as an argument of \texttt{\ chic()\ } .
\item
  \texttt{\ ..functions()\ } : These are a variable number of arguments
  that corresponds to Chic-header’s style functions.
\end{itemize}

\subsubsection{Functions}\label{functions}

\begin{enumerate}
\tightlist
\item
  \texttt{\ chic-header()\ } - Sets the header content.

  \begin{itemize}
  \tightlist
  \item
    \texttt{\ v-center\ } : Whether to vertically align the header
    content, or not (default is \texttt{\ false\ } ).
  \item
    \texttt{\ side-width\ } : Custom width for the sides. It can be an
    3-element-array, length or relative length (default is
    \texttt{\ none\ } and widths are set to \texttt{\ 1fr\ } if a side
    is present).
  \item
    \texttt{\ left-side\ } : Content displayed in the left side of the
    header (default is \texttt{\ none\ } ).
  \item
    \texttt{\ center-side\ } : Content displayed in the center of the
    header (default is \texttt{\ none\ } ).
  \item
    \texttt{\ right-side\ } : Content displayed in the right side of the
    header (default is \texttt{\ none\ } ).
  \end{itemize}
\item
  \texttt{\ chic-footer()\ } - Sets the footer content.

  \begin{itemize}
  \tightlist
  \item
    \texttt{\ v-center\ } : Whether to vertically align the header
    content, or not (default is \texttt{\ false\ } ).
  \item
    \texttt{\ side-width\ } : Custom width for the sides. It can be an
    3-element-array, length or relative length (default is
    \texttt{\ none\ } and widths are set to \texttt{\ 1fr\ } if a side
    is present).
  \item
    \texttt{\ left-side\ } : Content displayed in the left side of the
    footer (default is \texttt{\ none\ } ).
  \item
    \texttt{\ center-side\ } : Content displayed in the center of the
    footer (default is \texttt{\ none\ } ).
  \item
    \texttt{\ right-side\ } : Content displayed in the right side of the
    footer (default is \texttt{\ none\ } ).
  \end{itemize}
\item
  \texttt{\ chic-separator()\ } - Sets the separator for either the
  header, the footer or both.

  \begin{itemize}
  \tightlist
  \item
    \texttt{\ on\ } : Where to apply the separator. It can be
    \texttt{\ "header"\ } , \texttt{\ "footer"\ } or \texttt{\ "both"\ }
    (default is \texttt{\ "both"\ } ).
  \item
    \texttt{\ outset\ } : Space around the separator beyond the page
    margins (default is \texttt{\ 0pt\ } ).
  \item
    \texttt{\ gutter\ } : How much spacing insert around the separator
    (default is \texttt{\ 0.65em\ } ).
  \item
    (unnamed): A length for a \texttt{\ line()\ } , a stroke for a
    \texttt{\ line()\ } , or a custom content element.
  \end{itemize}
\item
  \texttt{\ chic-styled-separator()\ } - Returns a pre-made custom
  separator for using it in \texttt{\ chic-separator()\ }

  \begin{itemize}
  \tightlist
  \item
    \texttt{\ color\ } : Separator’s color (default is
    \texttt{\ black\ } ).
  \item
    (unnamed): A string indicating the separator’s style. It can be
    \texttt{\ "double-line"\ } , \texttt{\ "center-dot"\ } ,
    \texttt{\ "bold-center"\ } , or \texttt{\ "flower-end"\ } .
  \end{itemize}
\item
  \texttt{\ chic-height()\ } - Sets the height of either the header, the
  footer or both.

  \begin{itemize}
  \tightlist
  \item
    \texttt{\ on\ } : Where to change the height. It can be
    \texttt{\ "header"\ } , \texttt{\ "footer"\ } or \texttt{\ "both"\ }
    (default is \texttt{\ "both"\ } ).
  \item
    (unnamed): A relative length (the new height value).
  \end{itemize}
\item
  \texttt{\ chic-offset()\ } - Sets the offset of either the header, the
  footer or both (relative to the page content).

  \begin{itemize}
  \tightlist
  \item
    \texttt{\ on\ } : Where to change the offset It can be
    \texttt{\ "header"\ } , \texttt{\ "footer"\ } or \texttt{\ "both"\ }
    (default is \texttt{\ "both\ } ).
  \item
    (unnamed): A relative length (the new offset value).
  \end{itemize}
\item
  \texttt{\ chic-page-number()\ } - Returns the current page number.
  Useful for header and footer \texttt{\ sides\ } . It doesn’t take
  any parameters.
\item
  \texttt{\ chic-heading-name()\ } - Returns the next heading name in
  the \texttt{\ dir\ } direction. The heading must have a lower or equal
  level than \texttt{\ level\ } . If there’re no more headings in that
  direction, and \texttt{\ fill\ } is \texttt{\ true\ } , then headings
  are sought in the other direction.

  \begin{itemize}
  \tightlist
  \item
    \texttt{\ dir\ } : Direction for searching the next heading:
    \texttt{\ "next"\ } (from the current page, get the next heading) or
    \texttt{\ "prev"\ } (from the current page, get the previous
    heading). Default is \texttt{\ "next"\ } .
  \item
    \texttt{\ fill\ } : If there’s no more headings in the
    \texttt{\ dir\ } direction, indicates whether to try to get a
    heading in the opposite direction (default is \texttt{\ false\ } ).
  \item
    \texttt{\ level\ } : Up to what level of headings should this
    function search (default is \texttt{\ 2\ } ).
  \end{itemize}
\end{enumerate}

\subsection{Gallery}\label{gallery}

\subsubsection{\texorpdfstring{\protect\pandocbounded{\includegraphics[keepaspectratio]{https://github.com/typst/packages/raw/main/packages/preview/chic-hdr/0.4.0/assets/example-1.png}}}{Example 1}}\label{example-1}

\emph{Header with \texttt{\ chic-heading-name()\ } at left, and
\texttt{\ chic-page-number()\ } at right. There’s a
\texttt{\ chic-separator()\ } of \texttt{\ 1pt\ } only for the header.}

\subsubsection{\texorpdfstring{\protect\pandocbounded{\includegraphics[keepaspectratio]{https://github.com/typst/packages/raw/main/packages/preview/chic-hdr/0.4.0/assets/example-2.png}}}{Example 2}}\label{example-2}

\emph{Footer with \texttt{\ chic-page-number()\ } at right, and a custom
\texttt{\ chic-separator()\ } showing “end of page (No. page)�
between 9 \texttt{\ \textasciitilde{}\ } symbols at each side.}

\subsection{Changelog}\label{changelog}

\subsubsection{Version 0.1.0}\label{version-0.1.0}

\begin{itemize}
\tightlist
\item
  Initial release
\item
  Implemented \texttt{\ chic-header()\ } , \texttt{\ chic-footer()\ } ,
  \texttt{\ chic-separator()\ } , \texttt{\ chic-height()\ } ,
  \texttt{\ chic-offset()\ } , \texttt{\ chic-page-number()\ } , and
  \texttt{\ chic-heading-name()\ } functions
\end{itemize}

\subsubsection{Version 0.2.0}\label{version-0.2.0}

\emph{Thanks to Slashformotion ( \url{https://github.com/slashformotion}
) for noticing this version bugs, and suggesting a vertical alignment
for headers.}

\begin{itemize}
\tightlist
\item
  Fix alignment error in \texttt{\ chic-header()\ } and
  \texttt{\ chic-footer()\ }
\item
  Add \texttt{\ v-center\ } option for \texttt{\ chic-header()\ } and
  \texttt{\ chic-footer()\ }
\item
  Add \texttt{\ outset\ } option for \texttt{\ chic-separator()\ }
\item
  Add \texttt{\ chic-styled-separator()\ } function
\end{itemize}

\subsubsection{Version 0.3.0}\label{version-0.3.0}

\begin{itemize}
\tightlist
\item
  Add \texttt{\ side-width\ } option for \texttt{\ chic-header()\ } and
  \texttt{\ chic-footer()\ }
\end{itemize}

\subsubsection{Version 0.4.0}\label{version-0.4.0}

\emph{Thanks to David ( \url{https://github.com/davidleejy} ) for being
interested in the package and giving feedback and ideas for new
parameters}

\begin{itemize}
\tightlist
\item
  Update \texttt{\ type()\ } conditionals to met Typst 0.8.0 standards
\item
  Add \texttt{\ dir\ } , \texttt{\ fill\ } , and \texttt{\ level\ }
  parameters to \texttt{\ chic-heading-name()\ }
\item
  Allow negative indexes for skipping final pages while using
  \texttt{\ skip\ }
\item
  Include some panic alerts for types mismatch
\item
  Upload manual code in the package repository
\end{itemize}

\subsubsection{How to add}\label{how-to-add}

Copy this into your project and use the import as \texttt{\ chic-hdr\ }

\begin{verbatim}
#import "@preview/chic-hdr:0.4.0"
\end{verbatim}

\includesvg[width=0.16667in,height=0.16667in]{/assets/icons/16-copy.svg}

Check the docs for
\href{https://typst.app/docs/reference/scripting/\#packages}{more
information on how to import packages} .

\subsubsection{About}\label{about}

\begin{description}
\tightlist
\item[Author s :]
Pablo González Calderón \& Chic-hdr Contributors
\item[License:]
MIT
\item[Current version:]
0.4.0
\item[Last updated:]
December 28, 2023
\item[First released:]
August 19, 2023
\item[Archive size:]
7.64 kB
\href{https://packages.typst.org/preview/chic-hdr-0.4.0.tar.gz}{\pandocbounded{\includesvg[keepaspectratio]{/assets/icons/16-download.svg}}}
\item[Repository:]
\href{https://github.com/Pablo-Gonzalez-Calderon/chic-header-package}{GitHub}
\end{description}

\subsubsection{Where to report issues?}\label{where-to-report-issues}

This package is a project of Pablo González Calderón and Chic-hdr
Contributors . Report issues on
\href{https://github.com/Pablo-Gonzalez-Calderon/chic-header-package}{their
repository} . You can also try to ask for help with this package on the
\href{https://forum.typst.app}{Forum} .

Please report this package to the Typst team using the
\href{https://typst.app/contact}{contact form} if you believe it is a
safety hazard or infringes upon your rights.

\phantomsection\label{versions}
\subsubsection{Version history}\label{version-history}

\begin{longtable}[]{@{}ll@{}}
\toprule\noalign{}
Version & Release Date \\
\midrule\noalign{}
\endhead
\bottomrule\noalign{}
\endlastfoot
0.4.0 & December 28, 2023 \\
\href{https://typst.app/universe/package/chic-hdr/0.3.0/}{0.3.0} &
September 11, 2023 \\
\href{https://typst.app/universe/package/chic-hdr/0.2.0/}{0.2.0} &
August 19, 2023 \\
\href{https://typst.app/universe/package/chic-hdr/0.1.0/}{0.1.0} &
August 19, 2023 \\
\end{longtable}

Typst GmbH did not create this package and cannot guarantee correct
functionality of this package or compatibility with any version of the
Typst compiler or app.


\title{typst.app/universe/package/cereal-words}

\phantomsection\label{banner}
\phantomsection\label{template-thumbnail}
\pandocbounded{\includegraphics[keepaspectratio]{https://packages.typst.org/preview/thumbnails/cereal-words-0.1.0-small.webp}}

\section{cereal-words}\label{cereal-words}

{ 0.1.0 }

Time to kill? Search for words in a box of letters!

\href{/app?template=cereal-words&version=0.1.0}{Create project in app}

\phantomsection\label{readme}
Oh no, the Typst guys jumbled the letters! Bring order into this mess by
finding the hidden words.

This small game is playable in the Typst editor and best enjoyed with
the web app or \texttt{\ typst\ watch\ } . It was first released for the
24 Days to Christmas campaign in winter of 2023.

\subsection{Usage}\label{usage}

You can use this template in the Typst web app by clicking “Start from
template� on the dashboard and searching for \texttt{\ cereal-words\ }
.

Alternatively, you can use the CLI to kick this project off using the
command

\begin{verbatim}
typst init @preview/cereal-words
\end{verbatim}

Typst will create a new directory with all the files needed to get you
started.

\subsection{Configuration}\label{configuration}

This template exports the \texttt{\ game\ } function, which accepts a
single positional argument for the game input.

The template will initialize your package with a sample call to the
\texttt{\ game\ } function in a show rule. If you want to change an
existing project to use this template, you can add a show rule like this
at the top of your file:

\begin{Shaded}
\begin{Highlighting}[]
\NormalTok{\#import "@preview/cereal{-}words:0.1.0": game}
\NormalTok{\#show: game}

\NormalTok{// Type the words here}
\end{Highlighting}
\end{Shaded}

\href{/app?template=cereal-words&version=0.1.0}{Create project in app}

\subsubsection{How to use}\label{how-to-use}

Click the button above to create a new project using this template in
the Typst app.

You can also use the Typst CLI to start a new project on your computer
using this command:

\begin{verbatim}
typst init @preview/cereal-words:0.1.0
\end{verbatim}

\includesvg[width=0.16667in,height=0.16667in]{/assets/icons/16-copy.svg}

\subsubsection{About}\label{about}

\begin{description}
\tightlist
\item[Author :]
\href{https://typst.app}{Typst GmbH}
\item[License:]
MIT-0
\item[Current version:]
0.1.0
\item[Last updated:]
March 6, 2024
\item[First released:]
March 6, 2024
\item[Minimum Typst version:]
0.8.0
\item[Archive size:]
3.29 kB
\href{https://packages.typst.org/preview/cereal-words-0.1.0.tar.gz}{\pandocbounded{\includesvg[keepaspectratio]{/assets/icons/16-download.svg}}}
\item[Repository:]
\href{https://github.com/typst/templates}{GitHub}
\item[Categor y :]
\begin{itemize}
\tightlist
\item[]
\item
  \pandocbounded{\includesvg[keepaspectratio]{/assets/icons/16-smile.svg}}
  \href{https://typst.app/universe/search/?category=fun}{Fun}
\end{itemize}
\end{description}

\subsubsection{Where to report issues?}\label{where-to-report-issues}

This template is a project of Typst GmbH . Report issues on
\href{https://github.com/typst/templates}{their repository} . You can
also try to ask for help with this template on the
\href{https://forum.typst.app}{Forum} .

\phantomsection\label{versions}
\subsubsection{Version history}\label{version-history}

\begin{longtable}[]{@{}ll@{}}
\toprule\noalign{}
Version & Release Date \\
\midrule\noalign{}
\endhead
\bottomrule\noalign{}
\endlastfoot
0.1.0 & March 6, 2024 \\
\end{longtable}


\title{typst.app/universe/package/sigfig}

\phantomsection\label{banner}
\section{sigfig}\label{sigfig}

{ 0.1.0 }

Typst library for rounding numbers with significant figures and
measurement uncertainty.

\phantomsection\label{readme}
\texttt{\ sigfig\ } is a \href{https://typst.app/}{Typst} package for
rounding numbers with
\href{https://en.wikipedia.org/wiki/Significant_figures}{significant
figures} and
\href{https://en.wikipedia.org/wiki/Measurement_uncertainty}{measurement
uncertainty} .

\subsection{Overview}\label{overview}

\begin{Shaded}
\begin{Highlighting}[]
\NormalTok{\#import "@preview/sigfig:0.1.0": round, urounds}
\NormalTok{\#import "@preview/unify:0.5.0": num}

\NormalTok{$ \#num(round(98654, 3)) $}
\NormalTok{$ \#num(round(2.8977729e{-}3, 4)) $}
\NormalTok{$ \#num(round({-}.0999, 2)) $}
\NormalTok{$ \#num(urounds(114514.19, 1.98)) $}
\NormalTok{$ \#num(urounds(1234.5678, 0.096)) $}
\end{Highlighting}
\end{Shaded}

yields

\includegraphics[width=2.5in,height=\textheight,keepaspectratio]{https://github.com/typst/packages/assets/20166026/f3d69c3c-bc67-484f-81f9-80a10913fd11}

\subsection{Documentation}\label{documentation}

\subsubsection{\texorpdfstring{\texttt{\ round\ }}{ round }}\label{round}

\texttt{\ round\ } is similar to JavaScript’s
\texttt{\ Number.prototype.toPrecision()\ } (
\href{https://tc39.es/ecma262/multipage/numbers-and-dates.html\#sec-number.prototype.toprecision}{ES
spec} ).

\begin{Shaded}
\begin{Highlighting}[]
\NormalTok{\#assert(round(114514, 3) == "1.15e5")}
\NormalTok{\#assert(round(1, 5) == "1.0000")}
\NormalTok{\#assert(round(12345, 10) == "12345.00000")}
\NormalTok{\#assert(round(.00000002468, 3) == "2.47e{-}8")}
\end{Highlighting}
\end{Shaded}

Note that what is different from the ES spec is that there will be no
sign (\$+\$) if the exponent after \texttt{\ e\ } is positive.

\subsubsection{\texorpdfstring{\texttt{\ uround\ }}{ uround }}\label{uround}

\texttt{\ uround\ } rounds a number with its uncertainty, and returns a
string of both.

\begin{Shaded}
\begin{Highlighting}[]
\NormalTok{\#assert(uround(114514, 1919) == "1.15e5+{-}2e3")}
\NormalTok{\#assert(uround(114514.0, 1.9) == "114514+{-}2")}
\end{Highlighting}
\end{Shaded}

\subsubsection{\texorpdfstring{\texttt{\ urounds\ }}{ urounds }}\label{urounds}

\texttt{\ uround\ } rounds a number with its uncertainty, and returns a
string of both with the same exponent, if any.

You can use \texttt{\ num\ } in \texttt{\ unify\ } to display the
result.

\subsection{License}\label{license}

MIT © 2024 OverflowCat (
\href{https://about.overflow.cat/}{overflow.cat} ).

\subsubsection{How to add}\label{how-to-add}

Copy this into your project and use the import as \texttt{\ sigfig\ }

\begin{verbatim}
#import "@preview/sigfig:0.1.0"
\end{verbatim}

\includesvg[width=0.16667in,height=0.16667in]{/assets/icons/16-copy.svg}

Check the docs for
\href{https://typst.app/docs/reference/scripting/\#packages}{more
information on how to import packages} .

\subsubsection{About}\label{about}

\begin{description}
\tightlist
\item[Author :]
OverflowCat
\item[License:]
MIT
\item[Current version:]
0.1.0
\item[Last updated:]
June 17, 2024
\item[First released:]
June 17, 2024
\item[Archive size:]
3.73 kB
\href{https://packages.typst.org/preview/sigfig-0.1.0.tar.gz}{\pandocbounded{\includesvg[keepaspectratio]{/assets/icons/16-download.svg}}}
\item[Repository:]
\href{https://github.com/OverflowCat/sigfig}{GitHub}
\item[Discipline :]
\begin{itemize}
\tightlist
\item[]
\item
  \href{https://typst.app/universe/search/?discipline=engineering}{Engineering}
\end{itemize}
\end{description}

\subsubsection{Where to report issues?}\label{where-to-report-issues}

This package is a project of OverflowCat . Report issues on
\href{https://github.com/OverflowCat/sigfig}{their repository} . You can
also try to ask for help with this package on the
\href{https://forum.typst.app}{Forum} .

Please report this package to the Typst team using the
\href{https://typst.app/contact}{contact form} if you believe it is a
safety hazard or infringes upon your rights.

\phantomsection\label{versions}
\subsubsection{Version history}\label{version-history}

\begin{longtable}[]{@{}ll@{}}
\toprule\noalign{}
Version & Release Date \\
\midrule\noalign{}
\endhead
\bottomrule\noalign{}
\endlastfoot
0.1.0 & June 17, 2024 \\
\end{longtable}

Typst GmbH did not create this package and cannot guarantee correct
functionality of this package or compatibility with any version of the
Typst compiler or app.


\title{typst.app/universe/package/k-mapper}

\phantomsection\label{banner}
\section{k-mapper}\label{k-mapper}

{ 1.1.0 }

A package to add Karnaugh maps into Typst projects.

\phantomsection\label{readme}
ðŸ``-- See the \texttt{\ k-mapper\ } Manual
\href{https://github.com/derekchai/k-mapper/blob/1f334d9e0f02cc656c01835302474bf728db9f80/manual.pdf}{here}
! The Manual features much more documentation, and is typeset using
Typst.

This is a package for adding Karnaugh maps into your Typst projects.

See the changelog for the package
\href{https://github.com/derekchai/k-mapper/blob/698e8554ce67e3a61dd30319ab8f712a6a6b8daa/changelog.md}{here}
.

\subsection{Features}\label{features}

\begin{itemize}
\tightlist
\item
  2-variable (2 by 2) Karnaugh maps
\item
  3-variable (2 by 4) Karnaugh maps
\item
  4-variable (4 by 4) Karnaugh maps
\end{itemize}

\subsection{Getting Started}\label{getting-started}

Simply import \texttt{\ k-mapper\ } using the Typst package manager to
begin using \texttt{\ k-mapper\ } within your Typst documents.

\begin{Shaded}
\begin{Highlighting}[]
\NormalTok{\#import "@preview/k{-}mapper:1.1.0": *}
\end{Highlighting}
\end{Shaded}

\subsection{Example}\label{example}

\begin{Shaded}
\begin{Highlighting}[]
\NormalTok{  \#karnaugh(}
\NormalTok{    16,}
\NormalTok{    x{-}label: $C D$,}
\NormalTok{    y{-}label: $A B$,}
\NormalTok{    manual{-}terms: (}
\NormalTok{      0, 1, 2, 3, 4, 5, 6, 7, 8, }
\NormalTok{      9, 10, 11, 12, 13, 14, 15}
\NormalTok{    ),}
\NormalTok{    implicants: ((5, 7), (5, 13), (15, 15)),}
\NormalTok{    vertical{-}implicants: ((1, 11), ),}
\NormalTok{    horizontal{-}implicants: ((4, 14), ),}
\NormalTok{    corner{-}implicants: true,}
\NormalTok{  )}
\end{Highlighting}
\end{Shaded}

\pandocbounded{\includegraphics[keepaspectratio]{https://raw.githubusercontent.com/derekchai/k-mapper/005cb0a839499d0dfa90ee48d2128d41e582d755/readme-example.png}}

For more detailed documentation and examples, including function
parameters, see the Manual
\href{https://github.com/derekchai/k-mapper/blob/1f334d9e0f02cc656c01835302474bf728db9f80/manual.pdf}{PDF}
and
\href{https://github.com/derekchai/k-mapper/blob/1f334d9e0f02cc656c01835302474bf728db9f80/manual.typ}{Typst
file} in the
\href{https://github.com/derekchai/typst-karnaugh-map}{Github repo} .

\subsubsection{How to add}\label{how-to-add}

Copy this into your project and use the import as \texttt{\ k-mapper\ }

\begin{verbatim}
#import "@preview/k-mapper:1.1.0"
\end{verbatim}

\includesvg[width=0.16667in,height=0.16667in]{/assets/icons/16-copy.svg}

Check the docs for
\href{https://typst.app/docs/reference/scripting/\#packages}{more
information on how to import packages} .

\subsubsection{About}\label{about}

\begin{description}
\tightlist
\item[Author :]
Derek Chai
\item[License:]
MIT
\item[Current version:]
1.1.0
\item[Last updated:]
May 14, 2024
\item[First released:]
May 13, 2024
\item[Archive size:]
4.52 kB
\href{https://packages.typst.org/preview/k-mapper-1.1.0.tar.gz}{\pandocbounded{\includesvg[keepaspectratio]{/assets/icons/16-download.svg}}}
\item[Repository:]
\href{https://github.com/derekchai/typst-karnaugh-map}{GitHub}
\item[Categor y :]
\begin{itemize}
\tightlist
\item[]
\item
  \pandocbounded{\includesvg[keepaspectratio]{/assets/icons/16-chart.svg}}
  \href{https://typst.app/universe/search/?category=visualization}{Visualization}
\end{itemize}
\end{description}

\subsubsection{Where to report issues?}\label{where-to-report-issues}

This package is a project of Derek Chai . Report issues on
\href{https://github.com/derekchai/typst-karnaugh-map}{their repository}
. You can also try to ask for help with this package on the
\href{https://forum.typst.app}{Forum} .

Please report this package to the Typst team using the
\href{https://typst.app/contact}{contact form} if you believe it is a
safety hazard or infringes upon your rights.

\phantomsection\label{versions}
\subsubsection{Version history}\label{version-history}

\begin{longtable}[]{@{}ll@{}}
\toprule\noalign{}
Version & Release Date \\
\midrule\noalign{}
\endhead
\bottomrule\noalign{}
\endlastfoot
1.1.0 & May 14, 2024 \\
\href{https://typst.app/universe/package/k-mapper/1.0.0/}{1.0.0} & May
13, 2024 \\
\end{longtable}

Typst GmbH did not create this package and cannot guarantee correct
functionality of this package or compatibility with any version of the
Typst compiler or app.


\title{typst.app/universe/package/edgeframe}

\phantomsection\label{banner}
\section{edgeframe}\label{edgeframe}

{ 0.1.0 }

For quick paper setups.

\phantomsection\label{readme}
Custom margins and other components for page setup or layout.

\subsection{Usage}\label{usage}

Add the package with the following code. Remember to add the asterisk
\texttt{\ :\ *\ } at the end.

\begin{Shaded}
\begin{Highlighting}[]
\NormalTok{\#include "@preview/edgeframe:0.1.0": *}
\end{Highlighting}
\end{Shaded}

\begin{Shaded}
\begin{Highlighting}[]
\NormalTok{\#set page(margin: margin{-}normal)}
\end{Highlighting}
\end{Shaded}

\subsection{List of parameters}\label{list-of-parameters}

\begin{itemize}
\tightlist
\item
  margin-normal
\item
  margin-narrow
\item
  margin-moderate-x
\item
  margin-moderate-y
\item
  margin-wide-x
\item
  margin-wide-y
\item
  margin-a5-x
\item
  margin-a5-y
\end{itemize}

\begin{quote}
Parameters with \texttt{\ x\ } and \texttt{\ y\ } should to be used
together

\begin{verbatim}
#set page(margin: (x: margin-moderate-x, y: margin-moderate-y))
\end{verbatim}
\end{quote}

\subsubsection{How to add}\label{how-to-add}

Copy this into your project and use the import as \texttt{\ edgeframe\ }

\begin{verbatim}
#import "@preview/edgeframe:0.1.0"
\end{verbatim}

\includesvg[width=0.16667in,height=0.16667in]{/assets/icons/16-copy.svg}

Check the docs for
\href{https://typst.app/docs/reference/scripting/\#packages}{more
information on how to import packages} .

\subsubsection{About}\label{about}

\begin{description}
\tightlist
\item[Author :]
\href{https://github.com/neuralpain}{neuralpain}
\item[License:]
MIT
\item[Current version:]
0.1.0
\item[Last updated:]
November 29, 2024
\item[First released:]
November 29, 2024
\item[Archive size:]
1.44 kB
\href{https://packages.typst.org/preview/edgeframe-0.1.0.tar.gz}{\pandocbounded{\includesvg[keepaspectratio]{/assets/icons/16-download.svg}}}
\item[Repository:]
\href{https://github.com/neuralpain/edgeframe}{GitHub}
\item[Categor ies :]
\begin{itemize}
\tightlist
\item[]
\item
  \pandocbounded{\includesvg[keepaspectratio]{/assets/icons/16-envelope.svg}}
  \href{https://typst.app/universe/search/?category=office}{Office}
\item
  \pandocbounded{\includesvg[keepaspectratio]{/assets/icons/16-layout.svg}}
  \href{https://typst.app/universe/search/?category=layout}{Layout}
\end{itemize}
\end{description}

\subsubsection{Where to report issues?}\label{where-to-report-issues}

This package is a project of neuralpain . Report issues on
\href{https://github.com/neuralpain/edgeframe}{their repository} . You
can also try to ask for help with this package on the
\href{https://forum.typst.app}{Forum} .

Please report this package to the Typst team using the
\href{https://typst.app/contact}{contact form} if you believe it is a
safety hazard or infringes upon your rights.

\phantomsection\label{versions}
\subsubsection{Version history}\label{version-history}

\begin{longtable}[]{@{}ll@{}}
\toprule\noalign{}
Version & Release Date \\
\midrule\noalign{}
\endhead
\bottomrule\noalign{}
\endlastfoot
0.1.0 & November 29, 2024 \\
\end{longtable}

Typst GmbH did not create this package and cannot guarantee correct
functionality of this package or compatibility with any version of the
Typst compiler or app.


\title{typst.app/universe/package/definitely-not-tuw-thesis}

\phantomsection\label{banner}
\phantomsection\label{template-thumbnail}
\pandocbounded{\includegraphics[keepaspectratio]{https://packages.typst.org/preview/thumbnails/definitely-not-tuw-thesis-0.1.0-small.webp}}

\section{definitely-not-tuw-thesis}\label{definitely-not-tuw-thesis}

{ 0.1.0 }

An unofficial template for a thesis at the TU Wien informatics
institute.

\href{/app?template=definitely-not-tuw-thesis&version=0.1.0}{Create
project in app}

\phantomsection\label{readme}
An example thesis can be viewed here:
\url{https://otto-aa.github.io/definitely-not-tuw-thesis/thesis.pdf}

\subsection{Usage}\label{usage}

You can download the template with:

\begin{Shaded}
\begin{Highlighting}[]
\ExtensionTok{typst}\NormalTok{ init @preview/definitely{-}not{-}tuw{-}thesis}
\end{Highlighting}
\end{Shaded}

\subsubsection{Template overview}\label{template-overview}

After setting up the template, you will have the following files:

\begin{itemize}
\tightlist
\item
  \texttt{\ thesis.typ\ } : overall structure and styling, configuration
  for the cover pages and PDF metadata
\item
  \texttt{\ content/front-matter.typ\ } : acknowledgments and abstract
\item
  \texttt{\ content/main.typ\ } : all your chapters
\item
  \texttt{\ content/appendix.typ\ } : AI tools acknowledgment and other
  appendices
\item
  \texttt{\ refs.bib\ } : references
\end{itemize}

Then copy the values you get from compiling the
\href{https://gitlab.com/ThomasAUZINGER/vutinfth}{official template} ,
and paste them in \texttt{\ thesis.typ\ } . Remove all unused fields
and, finally, compare if it is close enough to the official template. If
not, please open an issue or PR to fix it.

\subsubsection{Styling}\label{styling}

If you want to adapt the styling, you can remove the
\texttt{\ show:\ ...\ } commands in the \texttt{\ thesis.typ\ } and
replace them with your own, or simply extend them with your own
\texttt{\ show:\ ...\ } commands.

\subsection{Contributing}\label{contributing}

I guess there are many ways to improve this template, feel free to do so
and submit issues and PRs! More information at
\href{https://github.com/Otto-AA/unofficial-tu-wien-thesis-template/blob/main/CONTRIBUTING.md}{CONTRIBUTING.md}

\subsection{License}\label{license}

The code is licensed under MIT-0. The ‘TU Wien Informatics’ logo and
signet are copyright of the TU Wien.

\subsection{Acknowledgments}\label{acknowledgments}

This work is based on the
\href{https://gitlab.com/ThomasAUZINGER/vutinfth}{official template}
maintained by Thomas Auzinger. The repository structure is based on
\href{https://github.com/typst-community/typst-package-template}{typst-package-template}
.

\href{/app?template=definitely-not-tuw-thesis&version=0.1.0}{Create
project in app}

\subsubsection{How to use}\label{how-to-use}

Click the button above to create a new project using this template in
the Typst app.

You can also use the Typst CLI to start a new project on your computer
using this command:

\begin{verbatim}
typst init @preview/definitely-not-tuw-thesis:0.1.0
\end{verbatim}

\includesvg[width=0.16667in,height=0.16667in]{/assets/icons/16-copy.svg}

\subsubsection{About}\label{about}

\begin{description}
\tightlist
\item[Author :]
Othmar Lechner @Otto-AA
\item[License:]
MIT-0
\item[Current version:]
0.1.0
\item[Last updated:]
July 29, 2024
\item[First released:]
July 29, 2024
\item[Archive size:]
364 kB
\href{https://packages.typst.org/preview/definitely-not-tuw-thesis-0.1.0.tar.gz}{\pandocbounded{\includesvg[keepaspectratio]{/assets/icons/16-download.svg}}}
\item[Repository:]
\href{https://github.com/Otto-AA/definitely-not-tuw-thesis}{GitHub}
\item[Categor y :]
\begin{itemize}
\tightlist
\item[]
\item
  \pandocbounded{\includesvg[keepaspectratio]{/assets/icons/16-mortarboard.svg}}
  \href{https://typst.app/universe/search/?category=thesis}{Thesis}
\end{itemize}
\end{description}

\subsubsection{Where to report issues?}\label{where-to-report-issues}

This template is a project of Othmar Lechner @Otto-AA . Report issues on
\href{https://github.com/Otto-AA/definitely-not-tuw-thesis}{their
repository} . You can also try to ask for help with this template on the
\href{https://forum.typst.app}{Forum} .

Please report this template to the Typst team using the
\href{https://typst.app/contact}{contact form} if you believe it is a
safety hazard or infringes upon your rights.

\phantomsection\label{versions}
\subsubsection{Version history}\label{version-history}

\begin{longtable}[]{@{}ll@{}}
\toprule\noalign{}
Version & Release Date \\
\midrule\noalign{}
\endhead
\bottomrule\noalign{}
\endlastfoot
0.1.0 & July 29, 2024 \\
\end{longtable}

Typst GmbH did not create this template and cannot guarantee correct
functionality of this template or compatibility with any version of the
Typst compiler or app.


\title{typst.app/universe/package/a2c-nums}

\phantomsection\label{banner}
\section{a2c-nums}\label{a2c-nums}

{ 0.0.1 }

Convert a number to Chinese

\phantomsection\label{readme}
Convert Arabic numbers to Chinese characters.

\subsection{usage}\label{usage}

\begin{Shaded}
\begin{Highlighting}[]
\NormalTok{\#import "@preview/a2c{-}nums:0.0.1": int{-}to{-}cn{-}num, int{-}to{-}cn{-}ancient{-}num, int{-}to{-}cn{-}simple{-}num, num{-}to{-}cn{-}currency}

\NormalTok{\#int{-}to{-}cn{-}num(1234567890)}

\NormalTok{\#int{-}to{-}cn{-}ancient{-}num(1234567890)}

\NormalTok{\#int{-}to{-}cn{-}simple{-}num(2024)}

\NormalTok{\#num{-}to{-}cn{-}currency(1234567890.12)}
\end{Highlighting}
\end{Shaded}

\subsection{Functions}\label{functions}

\subsubsection{int-to-cn-num}\label{int-to-cn-num}

Convert an integer to Chinese number. ex:
\texttt{\ \#int-to-cn-num(123)\ } will be \texttt{\ 一百二å??三\ }

\subsubsection{int-to-cn-ancient-num}\label{int-to-cn-ancient-num}

Convert an integer to ancient Chinese number. ex:
\texttt{\ \#int-to-cn-ancient-num(123)\ } will be
\texttt{\ 壹佰贰拾å??\ }

\subsubsection{int-to-cn-simple-num}\label{int-to-cn-simple-num}

Convert an integer to simpple Chinese number. ex:
\texttt{\ \#int-to-cn-simple-num(2024)\ } will be
\texttt{\ 二〇二四\ }

\subsubsection{num-to-cn-currency}\label{num-to-cn-currency}

Convert a number to Chinese currency. ex:
\texttt{\ \#int-to-cn-simple-num(1234.56)\ } will be
\texttt{\ 壹仟贰佰å??拾肆元ä¼?角陆分\ }

\subsubsection{more details}\label{more-details}

Reference
\href{https://github.com/typst/packages/raw/main/packages/preview/a2c-nums/0.0.1/demo.typ}{demo.typ}
for more details please.

\subsubsection{How to add}\label{how-to-add}

Copy this into your project and use the import as \texttt{\ a2c-nums\ }

\begin{verbatim}
#import "@preview/a2c-nums:0.0.1"
\end{verbatim}

\includesvg[width=0.16667in,height=0.16667in]{/assets/icons/16-copy.svg}

Check the docs for
\href{https://typst.app/docs/reference/scripting/\#packages}{more
information on how to import packages} .

\subsubsection{About}\label{about}

\begin{description}
\tightlist
\item[Author :]
\href{mailto:soarowl@yeah.net}{Zhuo Nengwen}
\item[License:]
MIT
\item[Current version:]
0.0.1
\item[Last updated:]
January 8, 2024
\item[First released:]
January 8, 2024
\item[Minimum Typst version:]
0.10.0
\item[Archive size:]
2.54 kB
\href{https://packages.typst.org/preview/a2c-nums-0.0.1.tar.gz}{\pandocbounded{\includesvg[keepaspectratio]{/assets/icons/16-download.svg}}}
\item[Repository:]
\href{https://github.com/soarowl/a2c-nums.git}{GitHub}
\end{description}

\subsubsection{Where to report issues?}\label{where-to-report-issues}

This package is a project of Zhuo Nengwen . Report issues on
\href{https://github.com/soarowl/a2c-nums.git}{their repository} . You
can also try to ask for help with this package on the
\href{https://forum.typst.app}{Forum} .

Please report this package to the Typst team using the
\href{https://typst.app/contact}{contact form} if you believe it is a
safety hazard or infringes upon your rights.

\phantomsection\label{versions}
\subsubsection{Version history}\label{version-history}

\begin{longtable}[]{@{}ll@{}}
\toprule\noalign{}
Version & Release Date \\
\midrule\noalign{}
\endhead
\bottomrule\noalign{}
\endlastfoot
0.0.1 & January 8, 2024 \\
\end{longtable}

Typst GmbH did not create this package and cannot guarantee correct
functionality of this package or compatibility with any version of the
Typst compiler or app.


\title{typst.app/universe/package/showman}

\phantomsection\label{banner}
\section{showman}\label{showman}

{ 0.1.2 }

Eval \& show typst code outputs inline with their source

\phantomsection\label{readme}
\pandocbounded{\includegraphics[keepaspectratio]{https://www.github.com/ntjess/showman/raw/v0.1.0/showman.jpg}}

\begin{center}\rule{0.5\linewidth}{0.5pt}\end{center}

Automagic tools to smooth the package documentation \& development
process.

\begin{itemize}
\item
  Package your files for local typst installation or PR submission in
  one command
\item
  Provide your readme in typst format with code block examples, and let
  \texttt{\ showman\ } do the rest! In one command, it will the readme
  to markdown and render code block outputs as included images.

  \begin{itemize}
  \tightlist
  \item
    Bonus points â€`` let \texttt{\ showman\ } know your repository path
    and it will ensure images will properly appear in your readme even
    after your package has been distributed through typst’s registry.
  \end{itemize}
\item
  Execute non-typst code blocks and render their outputs
\end{itemize}

\textbf{Prerequisites} : Make sure you have \texttt{\ typst\ } and
\texttt{\ pandoc\ } available from the command line. Then, in a python
virtual environment, run:

\begin{Shaded}
\begin{Highlighting}[]
\ExtensionTok{pip}\NormalTok{ install showman}
\end{Highlighting}
\end{Shaded}

Create a typst file with
\texttt{\ \textasciigrave{}\textasciigrave{}\textasciigrave{}example\ }
code blocks that show the output you want to include in your readme. For
instance:

\begin{Shaded}
\begin{Highlighting}[]
\NormalTok{\#import "@preview/cetz:0.1.2"}
\NormalTok{// Just to avoid showing this heading in the readme itself}
\NormalTok{\#set heading(outlined: false)}

\NormalTok{= Hello, world!}
\NormalTok{Let\textquotesingle{}s do something complicated:}

\NormalTok{\#cetz.canvas(\{}
\NormalTok{  import cetz.plot}
\NormalTok{  import cetz.palette}
\NormalTok{  cetz.draw.set{-}style(axes: (tick: (length: {-}.05)))}
\NormalTok{  // Plot something}
\NormalTok{  plot.plot(size: (3,3), x{-}tick{-}step: 1, axis{-}style: "left", \{}
\NormalTok{      for i in range(0, 3) \{}
\NormalTok{      plot.add(domain: ({-}4, 2),}
\NormalTok{      x =\textgreater{} calc.exp({-}(calc.pow(x + i, 2))),}
\NormalTok{      fill: true, style: palette.tango)}
\NormalTok{    \}}
\NormalTok{  \})}
\NormalTok{\})}
\end{Highlighting}
\end{Shaded}

\pandocbounded{\includegraphics[keepaspectratio]{https://www.github.com/ntjess/showman/raw/v0.1.0/assets/example-1.png}}

Then, run the following command:

\begin{Shaded}
\begin{Highlighting}[]
\ExtensionTok{showman}\NormalTok{ md }
\end{Highlighting}
\end{Shaded}

Congrats, you now have a readme with inline images 🎉

You can optionally specify your workspace root, output file name, image
folder, etc. These options are visible under
\texttt{\ showman\ md\ -\/-help\ } .

\textbf{Note} : You can customize the appearance of these images by
providing \texttt{\ showman\ } the template to use when creating them.
In your file to be rendered, create a variable called
\texttt{\ showman-config\ } at the outermost scope:

\begin{Shaded}
\begin{Highlighting}[]
\NormalTok{// Render images with a black background and red text}
\NormalTok{\#let showman{-}config = (}
\NormalTok{  template: it =\textgreater{} \{}
\NormalTok{    set text(fill: red)}
\NormalTok{    rect(fill: black, it)}
\NormalTok{  \}}
\NormalTok{)}
\end{Highlighting}
\end{Shaded}

Behind the scenes, showman imports your file as a module and looks for
this variable. If it is found, your template and a few other options are
injected into the example rendering process.

\textbf{Note} : If every example has the same setup (package imports,
etc.), and you don’t want the text to be included in your examples,
you can pass \texttt{\ eval-kwargs\ } in this config as well to specify
a string that gets prefixed to every example. Alternatively, pass
variables in a scope directly:

\begin{Shaded}
\begin{Highlighting}[]
\NormalTok{// Setup either through providing scope or import prefixes}
\NormalTok{\#let my{-}variable = 5}
\NormalTok{\#let showman{-}config = (}
\NormalTok{  eval{-}kwargs: (}
\NormalTok{    prefix: "\#import \textbackslash{}"@preview/cetz:0.1.2\textbackslash{}"}
\NormalTok{  ),}
\NormalTok{  // Now you can use \textasciigrave{}my{-}var\textasciigrave{} in your examples}
\NormalTok{  scope: (my{-}var: my{-}variable)}
\NormalTok{)}
\end{Highlighting}
\end{Shaded}

\subsection{Caveats}\label{caveats}

\begin{itemize}
\item
  \texttt{\ showman\ } uses the beautiful \texttt{\ pandoc\ } to do most
  of the markdown conversion heavy lifting. So, if your document can’t
  be processed by pandoc, you may need to adjust your syntax until
  pandoc is happy making a markdown document.
\item
  Typst doesn’t allow page styling inside containers. Since
  \texttt{\ showman\ } must use containers to extract each rendered
  example, you can’t use \texttt{\ \#set\ page(...)\ } or
  \texttt{\ \#pagebreak()\ } inside your examples.
\end{itemize}

If you don’t care about converting your readme to markdown, it’s
even easier to have example rendered alongside their code. Simply add
the following preamble to your file:

\begin{Shaded}
\begin{Highlighting}[]
\NormalTok{\#import "@preview/showman:0.1.1"}
\NormalTok{\#show: showman.formatter.template}

\NormalTok{The code below will be rendered side by side with its output:}

\NormalTok{\textasciigrave{}\textasciigrave{}\textasciigrave{} typst}
\NormalTok{= Hello world!}
\NormalTok{\textasciigrave{}\textasciigrave{}\textasciigrave{}}
\NormalTok{![Example 2](https://www.github.com/ntjess/showman/raw/v0.1.0/assets/example{-}2.png)}

\NormalTok{Several keywords can be privded to customize appearance and more. See \textasciigrave{}showman.formatter.template\textasciigrave{} for more details.}
\end{Highlighting}
\end{Shaded}

You’ve done the hard work of creating a beautiful, well-documented
manual. Now it’s time to share it with the world. \texttt{\ showman\ }
can help you package your files for distribution in one command, after
some minimal setup.

\begin{enumerate}
\item
  Make sure you have a \texttt{\ typst.toml\ } file that follows typst
  \href{https://github.com/typst/packages}{packaging guidelines}
\item
  Add a new block to your toml file as follows:
\end{enumerate}

\begin{Shaded}
\begin{Highlighting}[]
\KeywordTok{[tool.packager]}
\DataTypeTok{paths} \OperatorTok{=} \OperatorTok{[}\ErrorTok{...}\OperatorTok{]}
\end{Highlighting}
\end{Shaded}

Where \texttt{\ paths\ } is a list of files and directories you want to
include in your package.

\begin{enumerate}
\setcounter{enumi}{2}
\tightlist
\item
  Run the following command from the root of your repository:
\end{enumerate}

\begin{Shaded}
\begin{Highlighting}[]
\ExtensionTok{showman}\NormalTok{ package }
\end{Highlighting}
\end{Shaded}

\begin{enumerate}
\setcounter{enumi}{3}
\item
  Without any other arguments, you’ve just installed your package in
  your system’s local typst packages folder. Now you can import it
  with
  \texttt{\ typst\ \#import\ "@local/mypackage:\textless{}version\textgreater{}"\ }
  .

  \begin{itemize}
  \tightlist
  \item
    You can alternatively specify the path to your fork of
    \texttt{\ typst/packages\ } to prep your files for a PR, or specify
    a \texttt{\ -\/-namespace\ } other than \texttt{\ local\ } .
  \end{itemize}
\end{enumerate}

\textbf{Note} : You can see the full list of command options with
\texttt{\ showman\ package\ -\/-help\ } .

This package also executes non-typst code (currently bash on
non-windows, python, and c++). You can use
\texttt{\ showman\ execute\ ./path/to/file.typ\ } to execute code blocks
in these languages, and the output will be captured in a
\texttt{\ .coderunner.json\ } file in the root directory you specified.
To enable this feature, you need to add the following preamble to your
file:

\begin{Shaded}
\begin{Highlighting}[]
\NormalTok{\#import "@preview/showman:0.1.1": runner}

\NormalTok{\#let cache = json("/.coderunner.json").at("path/to/file.typ", default: (:))}
\NormalTok{\#let show{-}rule = runner.external{-}code.with(result{-}cache: cache)}

\NormalTok{// Now, apply the show rule to languages that have a \textasciigrave{}showman execute\textasciigrave{} result:}
\NormalTok{\#show raw.where(lang: "python"): show{-}rule}
\end{Highlighting}
\end{Shaded}

You can optionally style
\texttt{\ \textless{}example-input\textgreater{}\ } and
\texttt{\ \textless{}example-output\textgreater{}\ } labels to customize
how input and output blocks appear. For even deeper customization, you
can specify the \texttt{\ container\ } that displays the input and
output blocks that accepts a keyword \texttt{\ direction\ } and
positional \texttt{\ input\ } and \texttt{\ output\ } .

\subsubsection{How to add}\label{how-to-add}

Copy this into your project and use the import as \texttt{\ showman\ }

\begin{verbatim}
#import "@preview/showman:0.1.2"
\end{verbatim}

\includesvg[width=0.16667in,height=0.16667in]{/assets/icons/16-copy.svg}

Check the docs for
\href{https://typst.app/docs/reference/scripting/\#packages}{more
information on how to import packages} .

\subsubsection{About}\label{about}

\begin{description}
\tightlist
\item[Author :]
Nathan Jessurun
\item[License:]
Unlicense
\item[Current version:]
0.1.2
\item[Last updated:]
November 28, 2024
\item[First released:]
January 15, 2024
\item[Minimum Typst version:]
0.12.0
\item[Archive size:]
6.00 kB
\href{https://packages.typst.org/preview/showman-0.1.2.tar.gz}{\pandocbounded{\includesvg[keepaspectratio]{/assets/icons/16-download.svg}}}
\item[Repository:]
\href{https://github.com/ntjess/showman}{GitHub}
\item[Categor ies :]
\begin{itemize}
\tightlist
\item[]
\item
  \pandocbounded{\includesvg[keepaspectratio]{/assets/icons/16-code.svg}}
  \href{https://typst.app/universe/search/?category=scripting}{Scripting}
\item
  \pandocbounded{\includesvg[keepaspectratio]{/assets/icons/16-hammer.svg}}
  \href{https://typst.app/universe/search/?category=utility}{Utility}
\end{itemize}
\end{description}

\subsubsection{Where to report issues?}\label{where-to-report-issues}

This package is a project of Nathan Jessurun . Report issues on
\href{https://github.com/ntjess/showman}{their repository} . You can
also try to ask for help with this package on the
\href{https://forum.typst.app}{Forum} .

Please report this package to the Typst team using the
\href{https://typst.app/contact}{contact form} if you believe it is a
safety hazard or infringes upon your rights.

\phantomsection\label{versions}
\subsubsection{Version history}\label{version-history}

\begin{longtable}[]{@{}ll@{}}
\toprule\noalign{}
Version & Release Date \\
\midrule\noalign{}
\endhead
\bottomrule\noalign{}
\endlastfoot
0.1.2 & November 28, 2024 \\
\href{https://typst.app/universe/package/showman/0.1.1/}{0.1.1} & March
16, 2024 \\
\href{https://typst.app/universe/package/showman/0.1.0/}{0.1.0} &
January 15, 2024 \\
\end{longtable}

Typst GmbH did not create this package and cannot guarantee correct
functionality of this package or compatibility with any version of the
Typst compiler or app.


\title{typst.app/universe/package/fontawesome}

\phantomsection\label{banner}
\section{fontawesome}\label{fontawesome}

{ 0.5.0 }

A Typst library for Font Awesome icons through the desktop fonts.

\phantomsection\label{readme}
A Typst library for Font Awesome icons through the desktop fonts.

p.s. The library is based on the Font Awesome 6 desktop fonts (v6.6.0)

\subsection{Usage}\label{usage}

\subsubsection{Install the fonts}\label{install-the-fonts}

You can download the fonts from the official website:
\url{https://fontawesome.com/download}

After downloading the zip file, you can install the fonts depending on
your OS.

\paragraph{Typst web app}\label{typst-web-app}

You can simply upload the \texttt{\ otf\ } files to the web app and use
them with this package.

\paragraph{Mac}\label{mac}

You can double click the \texttt{\ otf\ } files to install them.

\paragraph{Windows}\label{windows}

You can right-click the \texttt{\ otf\ } files and select
\texttt{\ Install\ } .

\paragraph{Some notes}\label{some-notes}

This library is tested with the otf files of the Font Awesome Free set.
TrueType fonts may not work as expected. (Though I am not sure whether
Font Awesome provides TrueType fonts, some issue is reported with
TrueType fonts.)

\subsubsection{Import the library}\label{import-the-library}

\paragraph{Using the typst packages}\label{using-the-typst-packages}

You can install the library using the typst packages:

\texttt{\ \#import\ "@preview/fontawesome:0.5.0":\ *\ }

\paragraph{Manually install}\label{manually-install}

Copy all files start with \texttt{\ lib\ } to your project and import
the library:

\texttt{\ \#import\ "lib.typ":\ *\ }

There are three files:

\begin{itemize}
\tightlist
\item
  \texttt{\ lib.typ\ } : The main entrypoint of the library.
\item
  \texttt{\ lib-impl.typ\ } : The implementation of \texttt{\ fa-icon\ }
  .
\item
  \texttt{\ lib-gen.typ\ } : The generated icon map and functions.
\end{itemize}

I recommend renaming these files to avoid conflicts with other
libraries.

\subsubsection{Use the icons}\label{use-the-icons}

You can use the \texttt{\ fa-icon\ } function to create an icon with its
name:

\texttt{\ \#fa-icon("chess-queen")\ }

Or you can use the \texttt{\ fa-\ } prefix to create an icon with its
name:

\texttt{\ \#fa-chess-queen()\ } (This is equivalent to
\texttt{\ \#fa-icon().with("chess-queen")\ } )

You can also set \texttt{\ solid\ } to \texttt{\ true\ } to use the
solid version of the icon:

\texttt{\ \#fa-icon("chess-queen",\ solid:\ true)\ }

Some icons only have the solid version in the Free set, so you need to
set \texttt{\ solid\ } to \texttt{\ true\ } to use them if you are using
the Free set. Otherwise, you may not get the expected glyph.

\paragraph{Full list of icons}\label{full-list-of-icons}

You can find all icons on the
\href{https://fontawesome.com/search}{official website}

\paragraph{Different sets}\label{different-sets}

By default, the library supports \texttt{\ Free\ } , \texttt{\ Brands\ }
, \texttt{\ Pro\ } , \texttt{\ Duotone\ } and \texttt{\ Sharp\ } sets.
(See
\href{https://github.com/typst/packages/raw/main/packages/preview/fontawesome/0.5.0/\#enable-pro-sets}{Enable
Pro sets} for enabling Pro sets.)

But only \texttt{\ Free\ } and \texttt{\ Brands\ } are tested by me.
That is, three font files are used to test:

\begin{itemize}
\tightlist
\item
  Font Awesome 6 Free (Also named as \emph{Font Awesome 6 Free Regular}
  )
\item
  Font Awesome 6 Free Solid
\item
  Font Awesome 6 Brands
\end{itemize}

Due to some limitations of typst 0.12.0, the regular and solid versions
are treated as different fonts. In this library, \texttt{\ solid\ } is
used to switch between the regular and solid versions.

To use other sets or specify one set, you can pass the \texttt{\ font\ }
parameter to the inner \texttt{\ text\ } function:\\
\texttt{\ fa-icon("github",\ font:\ "Font\ Awesome\ 6\ Pro\ Solid")\ }

If you have Font Awesome Pro, please help me test the library with the
Pro set. Any feedback is appreciated.

\subparagraph{Enable Pro sets}\label{enable-pro-sets}

Typst 0.12.0 raise a warning when the font is not found. To use the Pro
set, \texttt{\ \#fa-use-pro()\ } should be called before any
\texttt{\ fa-*\ } functions.

\begin{Shaded}
\begin{Highlighting}[]
\NormalTok{\#fa{-}use{-}pro()                 // Enable Pro sets}

\NormalTok{\#fa{-}icon("chess{-}queen{-}piece") // Use icons from Pro sets}
\end{Highlighting}
\end{Shaded}

\paragraph{Customization}\label{customization}

The \texttt{\ fa-icon\ } function passes args to \texttt{\ text\ } , so
you can customize the icon by passing parameters to it:

\texttt{\ \#fa-icon("chess-queen",\ fill:\ blue)\ }

\paragraph{Stacking icons}\label{stacking-icons}

The \texttt{\ fa-stack\ } function can be used to create stacked icons:

\texttt{\ \#fa-stack(fa-icon-args:\ (solid:\ true),\ "square",\ ("chess-queen",\ (fill:\ white,\ size:\ 5.5pt)))\ }

Declaration is
\texttt{\ fa-stack(box-args:\ (:),\ grid-args:\ (:),\ fa-icon-args:\ (:),\ ..icons)\ }

\begin{itemize}
\tightlist
\item
  The order of the icons is from the bottom to the top.
\item
  \texttt{\ fa-icon-args\ } is used to set the default args for all
  icons.
\item
  You can also control the internal \texttt{\ box\ } and
  \texttt{\ grid\ } by passing the \texttt{\ box-args\ } and
  \texttt{\ grid-args\ } to the \texttt{\ fa-stack\ } function.
\item
  Currently, four types of icons are supported. The first three types
  leverage the \texttt{\ fa-icon\ } function, and the last type is just
  a content you want to put in the stack.

  \begin{itemize}
  \tightlist
  \item
    \texttt{\ str\ } , e.g., \texttt{\ "square"\ }
  \item
    \texttt{\ array\ } , e.g.,
    \texttt{\ ("chess-queen",\ (fill:\ white,\ size:\ 5.5pt))\ }
  \item
    \texttt{\ arguments\ } , e.g.
    \texttt{\ arguments("chess-queen",\ solid:\ true,\ fill:\ white)\ }
  \item
    \texttt{\ content\ } , e.g.
    \texttt{\ fa-chess-queen(solid:\ true,\ fill:\ white)\ }
  \end{itemize}
\end{itemize}

\paragraph{Known Issues}\label{known-issues}

\begin{itemize}
\item
  \href{https://github.com/typst/typst/issues/2578}{typst\#2578}
  \href{https://github.com/duskmoon314/typst-fontawesome/issues/2}{typst-fontawesome\#2}

  This is a known issue that the ligatures may not work in headings,
  list items, grid items, and other elements. You can use the Unicode
  from the \href{https://fontawesome.com/}{official website} to avoid
  this issue when using Pro sets.

  For most icons, Unicode is used implicitly. So I assume we usually
  don’t need to worry about this.

  Any help on this issue is appreciated.
\end{itemize}

\subsection{Example}\label{example}

See the
\href{https://typst.app/project/rQwGUWt5p33vrsb_uNPR9F}{\texttt{\ example.typ\ }}
file for a complete example.

\subsection{Contribution}\label{contribution}

Feel free to open an issue or a pull request if you find any problems or
have any suggestions.

\subsubsection{Python helper}\label{python-helper}

The \texttt{\ helper.py\ } script is used to get metadata via the
GraphQL API and generate typst code. I aim only to use standard python
libraries, so running it on any platform with python installed should be
easy.

\subsubsection{Repo structure}\label{repo-structure}

\begin{itemize}
\tightlist
\item
  \texttt{\ helper.py\ } : The helper script to get metadata and
  generate typst code.
\item
  \texttt{\ lib.typ\ } : The main entrypoint of the library.
\item
  \texttt{\ lib-impl.typ\ } : The implementation of \texttt{\ fa-icon\ }
  .
\item
  \texttt{\ lib-gen.typ\ } : The generated functions of icons.
\item
  \texttt{\ example.typ\ } : An example file to show how to use the
  library.
\item
  \texttt{\ gallery.typ\ } : The generated gallery of icons. It is used
  in the example file.
\end{itemize}

\subsection{License}\label{license}

This library is licensed under the MIT license. Feel free to use it in
your project.

\subsubsection{How to add}\label{how-to-add}

Copy this into your project and use the import as
\texttt{\ fontawesome\ }

\begin{verbatim}
#import "@preview/fontawesome:0.5.0"
\end{verbatim}

\includesvg[width=0.16667in,height=0.16667in]{/assets/icons/16-copy.svg}

Check the docs for
\href{https://typst.app/docs/reference/scripting/\#packages}{more
information on how to import packages} .

\subsubsection{About}\label{about}

\begin{description}
\tightlist
\item[Author :]
\href{mailto:kp.campbell.he@duskmoon314.com}{duskmoon (Campbell He)}
\item[License:]
MIT
\item[Current version:]
0.5.0
\item[Last updated:]
October 21, 2024
\item[First released:]
July 3, 2023
\item[Archive size:]
74.7 kB
\href{https://packages.typst.org/preview/fontawesome-0.5.0.tar.gz}{\pandocbounded{\includesvg[keepaspectratio]{/assets/icons/16-download.svg}}}
\item[Repository:]
\href{https://github.com/duskmoon314/typst-fontawesome}{GitHub}
\end{description}

\subsubsection{Where to report issues?}\label{where-to-report-issues}

This package is a project of duskmoon (Campbell He) . Report issues on
\href{https://github.com/duskmoon314/typst-fontawesome}{their
repository} . You can also try to ask for help with this package on the
\href{https://forum.typst.app}{Forum} .

Please report this package to the Typst team using the
\href{https://typst.app/contact}{contact form} if you believe it is a
safety hazard or infringes upon your rights.

\phantomsection\label{versions}
\subsubsection{Version history}\label{version-history}

\begin{longtable}[]{@{}ll@{}}
\toprule\noalign{}
Version & Release Date \\
\midrule\noalign{}
\endhead
\bottomrule\noalign{}
\endlastfoot
0.5.0 & October 21, 2024 \\
\href{https://typst.app/universe/package/fontawesome/0.4.0/}{0.4.0} &
August 1, 2024 \\
\href{https://typst.app/universe/package/fontawesome/0.3.0/}{0.3.0} &
July 23, 2024 \\
\href{https://typst.app/universe/package/fontawesome/0.2.1/}{0.2.1} &
June 17, 2024 \\
\href{https://typst.app/universe/package/fontawesome/0.2.0/}{0.2.0} &
April 19, 2024 \\
\href{https://typst.app/universe/package/fontawesome/0.1.1/}{0.1.1} &
April 1, 2024 \\
\href{https://typst.app/universe/package/fontawesome/0.1.0/}{0.1.0} &
July 3, 2023 \\
\end{longtable}

Typst GmbH did not create this package and cannot guarantee correct
functionality of this package or compatibility with any version of the
Typst compiler or app.


\title{typst.app/universe/package/wordometer}

\phantomsection\label{banner}
\section{wordometer}\label{wordometer}

{ 0.1.4 }

Word counts and document statistics.

{ } Featured Package

\phantomsection\label{readme}
\href{https://github.com/typst/packages/raw/main/packages/preview/wordometer/0.1.4/docs/manual.pdf}{\pandocbounded{\includegraphics[keepaspectratio]{https://img.shields.io/badge/docs-manual.pdf-green}}}
\pandocbounded{\includegraphics[keepaspectratio]{https://img.shields.io/badge/dynamic/toml?url=https\%3A\%2F\%2Fgithub.com\%2FJollywatt\%2Ftypst-wordometer\%2Fraw\%2Fmaster\%2Ftypst.toml&query=package.version&label=latest\%20version}}
\href{https://github.com/Jollywatt/typst-wordometer}{\pandocbounded{\includegraphics[keepaspectratio]{https://img.shields.io/badge/GitHub-repo-blue}}}

A small
\href{https://github.com/typst/packages/raw/main/packages/preview/wordometer/0.1.4/\%22https://typst.app/\%22}{Typst}
package for quick and easy in-document word counts.

\subsection{Basic usage}\label{basic-usage}

\begin{Shaded}
\begin{Highlighting}[]
\NormalTok{\#import "@preview/wordometer:0.1.3": word{-}count, total{-}words}

\NormalTok{\#show: word{-}count}

\NormalTok{In this document, there are \#total{-}words words all up.}

\NormalTok{\#word{-}count(total =\textgreater{} [}
\NormalTok{  The number of words in this block is \#total.words}
\NormalTok{  and there are \#total.characters letters.}
\NormalTok{])}
\end{Highlighting}
\end{Shaded}

\subsection{Excluding elements}\label{excluding-elements}

You can exclude elements by name (e.g., \texttt{\ "caption"\ } ),
function (e.g., \texttt{\ figure.caption\ } ), where-selector (e.g.,
\texttt{\ raw.where(block:\ true)\ } ), or label (e.g.,
\texttt{\ \textless{}no-wc\textgreater{}\ } ).

\begin{Shaded}
\begin{Highlighting}[]
\NormalTok{\#show: word{-}count.with(exclude: (heading.where(level: 1), strike))}

\NormalTok{= This Heading Doesn\textquotesingle{}t Count}
\NormalTok{== But I do!}

\NormalTok{In this document \#strike[(excluding me)], there are \#total{-}words words all up.}

\NormalTok{\#word{-}count(total =\textgreater{} [}
\NormalTok{  You can exclude elements by label, too.}
\NormalTok{  \#[That was \#total.words, excluding this sentence!] \textless{}no{-}wc\textgreater{}}
\NormalTok{], exclude: \textless{}no{-}wc\textgreater{})}
\end{Highlighting}
\end{Shaded}

\subsection{Changelog}\label{changelog}

\subsubsection{v0.1.4}\label{v0.1.4}

\begin{itemize}
\tightlist
\item
  Fix deprecated use of \texttt{\ locate()\ } for Typst
  \texttt{\ \textgreater{}=0.12.0\ }
\end{itemize}

\subsubsection{v0.1.3}\label{v0.1.3}

(No changes 🤡)

\subsubsection{v0.1.2}\label{v0.1.2}

\begin{itemize}
\tightlist
\item
  Fix bugs when using labels and where-selectors to exclude elements
\end{itemize}

\subsubsection{v0.1.1}\label{v0.1.1}

\begin{itemize}
\tightlist
\item
  Allow excluding elements by passing their element functions
\item
  Add basic \texttt{\ element.where(..)\ } selectors
\item
  Fix crash for figures without captions
\end{itemize}

\subsubsection{v0.1.0}\label{v0.1.0}

\begin{itemize}
\tightlist
\item
  Initial version
\end{itemize}

\subsubsection{How to add}\label{how-to-add}

Copy this into your project and use the import as
\texttt{\ wordometer\ }

\begin{verbatim}
#import "@preview/wordometer:0.1.4"
\end{verbatim}

\includesvg[width=0.16667in,height=0.16667in]{/assets/icons/16-copy.svg}

Check the docs for
\href{https://typst.app/docs/reference/scripting/\#packages}{more
information on how to import packages} .

\subsubsection{About}\label{about}

\begin{description}
\tightlist
\item[Author :]
Joseph Wilson (Jollywatt)
\item[License:]
MIT
\item[Current version:]
0.1.4
\item[Last updated:]
October 29, 2024
\item[First released:]
January 29, 2024
\item[Archive size:]
7.49 kB
\href{https://packages.typst.org/preview/wordometer-0.1.4.tar.gz}{\pandocbounded{\includesvg[keepaspectratio]{/assets/icons/16-download.svg}}}
\item[Repository:]
\href{https://github.com/Jollywatt/typst-wordometer}{GitHub}
\end{description}

\subsubsection{Where to report issues?}\label{where-to-report-issues}

This package is a project of Joseph Wilson (Jollywatt) . Report issues
on \href{https://github.com/Jollywatt/typst-wordometer}{their
repository} . You can also try to ask for help with this package on the
\href{https://forum.typst.app}{Forum} .

Please report this package to the Typst team using the
\href{https://typst.app/contact}{contact form} if you believe it is a
safety hazard or infringes upon your rights.

\phantomsection\label{versions}
\subsubsection{Version history}\label{version-history}

\begin{longtable}[]{@{}ll@{}}
\toprule\noalign{}
Version & Release Date \\
\midrule\noalign{}
\endhead
\bottomrule\noalign{}
\endlastfoot
0.1.4 & October 29, 2024 \\
\href{https://typst.app/universe/package/wordometer/0.1.3/}{0.1.3} &
October 8, 2024 \\
\href{https://typst.app/universe/package/wordometer/0.1.2/}{0.1.2} &
April 29, 2024 \\
\href{https://typst.app/universe/package/wordometer/0.1.1/}{0.1.1} &
March 3, 2024 \\
\href{https://typst.app/universe/package/wordometer/0.1.0/}{0.1.0} &
January 29, 2024 \\
\end{longtable}

Typst GmbH did not create this package and cannot guarantee correct
functionality of this package or compatibility with any version of the
Typst compiler or app.


\title{typst.app/universe/package/touying-brandred-uobristol}

\phantomsection\label{banner}
\phantomsection\label{template-thumbnail}
\pandocbounded{\includegraphics[keepaspectratio]{https://packages.typst.org/preview/thumbnails/touying-brandred-uobristol-0.1.2-small.webp}}

\section{touying-brandred-uobristol}\label{touying-brandred-uobristol}

{ 0.1.2 }

Touying Slide Theme for University of Bristol

\href{/app?template=touying-brandred-uobristol&version=0.1.2}{Create
project in app}

\phantomsection\label{readme}
Inspired by the brand guidelines of University of Bristol and modified
from the Metropolis theme.

\subsection{Use as Typst Template
Package}\label{use-as-typst-template-package}

Use the following command to create a new project with this theme.

\begin{Shaded}
\begin{Highlighting}[]
\ExtensionTok{typst}\NormalTok{ init @preview/touying{-}uobristol}
\end{Highlighting}
\end{Shaded}

\subsection{Examples}\label{examples}

See the
\href{https://github.com/typst/packages/raw/main/packages/preview/touying-brandred-uobristol/0.1.2/examples/example.typ}{examples}
and
\href{https://github.com/typst/packages/raw/main/packages/preview/touying-brandred-uobristol/0.1.2/examples/example.pdf}{output}
for more details.

Licensed under the
\href{https://github.com/typst/packages/raw/main/packages/preview/touying-brandred-uobristol/0.1.2/LICENSE}{MIT
License} .

\href{/app?template=touying-brandred-uobristol&version=0.1.2}{Create
project in app}

\subsubsection{How to use}\label{how-to-use}

Click the button above to create a new project using this template in
the Typst app.

You can also use the Typst CLI to start a new project on your computer
using this command:

\begin{verbatim}
typst init @preview/touying-brandred-uobristol:0.1.2
\end{verbatim}

\includesvg[width=0.16667in,height=0.16667in]{/assets/icons/16-copy.svg}

\subsubsection{About}\label{about}

\begin{description}
\tightlist
\item[Author :]
\href{mailto:huyg0180110559@outlook.com}{HPDell}
\item[License:]
MIT
\item[Current version:]
0.1.2
\item[Last updated:]
November 18, 2024
\item[First released:]
September 18, 2024
\item[Archive size:]
12.0 kB
\href{https://packages.typst.org/preview/touying-brandred-uobristol-0.1.2.tar.gz}{\pandocbounded{\includesvg[keepaspectratio]{/assets/icons/16-download.svg}}}
\item[Repository:]
\href{https://github.com/HPDell/touying-brandred-uobristol}{GitHub}
\item[Categor y :]
\begin{itemize}
\tightlist
\item[]
\item
  \pandocbounded{\includesvg[keepaspectratio]{/assets/icons/16-presentation.svg}}
  \href{https://typst.app/universe/search/?category=presentation}{Presentation}
\end{itemize}
\end{description}

\subsubsection{Where to report issues?}\label{where-to-report-issues}

This template is a project of HPDell . Report issues on
\href{https://github.com/HPDell/touying-brandred-uobristol}{their
repository} . You can also try to ask for help with this template on the
\href{https://forum.typst.app}{Forum} .

Please report this template to the Typst team using the
\href{https://typst.app/contact}{contact form} if you believe it is a
safety hazard or infringes upon your rights.

\phantomsection\label{versions}
\subsubsection{Version history}\label{version-history}

\begin{longtable}[]{@{}ll@{}}
\toprule\noalign{}
Version & Release Date \\
\midrule\noalign{}
\endhead
\bottomrule\noalign{}
\endlastfoot
0.1.2 & November 18, 2024 \\
\href{https://typst.app/universe/package/touying-brandred-uobristol/0.1.1/}{0.1.1}
& October 9, 2024 \\
\href{https://typst.app/universe/package/touying-brandred-uobristol/0.1.0/}{0.1.0}
& September 18, 2024 \\
\end{longtable}

Typst GmbH did not create this template and cannot guarantee correct
functionality of this template or compatibility with any version of the
Typst compiler or app.


\title{typst.app/universe/package/fireside}

\phantomsection\label{banner}
\phantomsection\label{template-thumbnail}
\pandocbounded{\includegraphics[keepaspectratio]{https://packages.typst.org/preview/thumbnails/fireside-1.0.0-small.webp}}

\section{fireside}\label{fireside}

{ 1.0.0 }

A simple letter template with a touch of warmth

{ } Featured Template

\href{/app?template=fireside&version=1.0.0}{Create project in app}

\phantomsection\label{readme}
A Typst theme for a nice and simple looking letter that’s not
completely black and white. Inspired by a Canva theme.

Features:

\begin{itemize}
\tightlist
\item
  A neutral-warm beige background that feels cosier and softer to the
  eyes than pure white, while still looking kinda white-ish
\item
  Short content is vertically padded to look a bit more centered
\item
  Long content overflows gracefully on as many pages as necessary
\end{itemize}

\begin{longtable}[]{@{}lll@{}}
\toprule\noalign{}
Basic example & Short text (vertically centered) & Multi-page
overflowing text \\
\midrule\noalign{}
\endhead
\bottomrule\noalign{}
\endlastfoot
\href{https://github.com/typst/packages/raw/main/packages/preview/fireside/1.0.0/.rendered/demo_medium.pdf}{\texttt{\ .rendered/demo\_medium.pdf\ }}
&
\href{https://github.com/typst/packages/raw/main/packages/preview/fireside/1.0.0/.rendered/demo_short.pdf}{\texttt{\ .rendered/demo\_short.pdf\ }}
&
\href{https://github.com/typst/packages/raw/main/packages/preview/fireside/1.0.0/.rendered/demo_long.pdf}{\texttt{\ .rendered/demo\_long.pdf\ }} \\
\end{longtable}

\begin{itemize}
\item
  If using Typst locally, install the
  \href{https://fonts.google.com/specimen/Hanken+Grotesk}{HK Grotesk}
  font

  \begin{itemize}
  \tightlist
  \item
    \emph{Note: it is already installed on the \url{https://typst.app/}
    IDE}
  \end{itemize}
\item
  Insert the setup \texttt{\ show\ } statement

\begin{Shaded}
\begin{Highlighting}[]
\NormalTok{\#import "@preview/fireside:1.0.0": *}

\NormalTok{\#show: project.with(}
\NormalTok{  title: [Anakin \textbackslash{} Skywalker],}
\NormalTok{  from{-}details: [}
\NormalTok{    Appt. x, \textbackslash{}}
\NormalTok{    Mos Espa, \textbackslash{}}
\NormalTok{    Tatooine \textbackslash{}}
\NormalTok{    anakin\textbackslash{}@example.com \textbackslash{} +999 xxxx xxx}
\NormalTok{  ],}
\NormalTok{  to{-}details: [}
\NormalTok{    Sheev Palpatine \textbackslash{}}
\NormalTok{    500 Republica, \textbackslash{}}
\NormalTok{    Ambassadorial Sector, Senate District, \textbackslash{}}
\NormalTok{    Galactic City, \textbackslash{} Coruscant}
\NormalTok{  ],}
\NormalTok{)}

\NormalTok{Dear Emperor, ...}
\end{Highlighting}
\end{Shaded}
\item
  If your text overflows on multiple pages, you might want to add
  \href{https://typst.app/docs/reference/layout/page/\#parameters-numbering}{page
  numbering} , as shown in
  \href{https://github.com/typst/packages/raw/main/packages/preview/fireside/1.0.0/.demo/demo_long.typ}{\texttt{\ .demo/demo\_long.typ\ }}
  (line 3)
\end{itemize}

\begin{Shaded}
\begin{Highlighting}[]
\NormalTok{  background: rgb("f4f1eb"), \# Override the background color}
\NormalTok{  title: "",                 \# Set the top{-}left title. It looks best on two lines}
\NormalTok{  from{-}details: none,        \# Letter sender (you) details}
\NormalTok{  to{-}details: none,          \# Letter receiver details}
\NormalTok{  margin: 2.1cm,             \# Page margin}
\NormalTok{  vertical{-}center{-}level: 2,  \# When the content is small, it is vertically centered a bit, but still kept closer to the top. This controls how much. Setting to none will disable centering.}
\NormalTok{  body}
\end{Highlighting}
\end{Shaded}

\begin{itemize}
\tightlist
\item
  \texttt{\ lib.typ\ } is licensed as MIT (
  \url{https://opensource.org/license/mit} )
\item
  The demo/template files are licensed as CC0 (
  \url{https://creativecommons.org/publicdomain/zero/1.0/legalcode} )
\item
  Any document fully or partially generated using this template may be
  licensed however you wish
\end{itemize}

\href{/app?template=fireside&version=1.0.0}{Create project in app}

\subsubsection{How to use}\label{how-to-use}

Click the button above to create a new project using this template in
the Typst app.

You can also use the Typst CLI to start a new project on your computer
using this command:

\begin{verbatim}
typst init @preview/fireside:1.0.0
\end{verbatim}

\includesvg[width=0.16667in,height=0.16667in]{/assets/icons/16-copy.svg}

\subsubsection{About}\label{about}

\begin{description}
\tightlist
\item[Author :]
\href{https://edgar.bzh/}{Edgar Onghena}
\item[License:]
MIT
\item[Current version:]
1.0.0
\item[Last updated:]
May 8, 2024
\item[First released:]
May 8, 2024
\item[Minimum Typst version:]
0.11.0
\item[Archive size:]
2.96 kB
\href{https://packages.typst.org/preview/fireside-1.0.0.tar.gz}{\pandocbounded{\includesvg[keepaspectratio]{/assets/icons/16-download.svg}}}
\item[Categor y :]
\begin{itemize}
\tightlist
\item[]
\item
  \pandocbounded{\includesvg[keepaspectratio]{/assets/icons/16-envelope.svg}}
  \href{https://typst.app/universe/search/?category=office}{Office}
\end{itemize}
\end{description}

\subsubsection{Where to report issues?}\label{where-to-report-issues}

This template is a project of Edgar Onghena . You can also try to ask
for help with this template on the \href{https://forum.typst.app}{Forum}
.

Please report this template to the Typst team using the
\href{https://typst.app/contact}{contact form} if you believe it is a
safety hazard or infringes upon your rights.

\phantomsection\label{versions}
\subsubsection{Version history}\label{version-history}

\begin{longtable}[]{@{}ll@{}}
\toprule\noalign{}
Version & Release Date \\
\midrule\noalign{}
\endhead
\bottomrule\noalign{}
\endlastfoot
1.0.0 & May 8, 2024 \\
\end{longtable}

Typst GmbH did not create this template and cannot guarantee correct
functionality of this template or compatibility with any version of the
Typst compiler or app.


\title{typst.app/universe/package/salsa-dip}

\phantomsection\label{banner}
\section{salsa-dip}\label{salsa-dip}

{ 0.1.0 }

DIP chip labels for Typst

\phantomsection\label{readme}
Salsa Dip is a library for making
\href{https://en.wikipedia.org/wiki/Dual_in-line_package}{DIP} chip
labels in Typst.

\begin{Shaded}
\begin{Highlighting}[]
\NormalTok{\#import "@preview/salsa{-}dip:0.1.0": dip{-}chip{-}label}

\NormalTok{\#set text(font: ("JetBrains Mono", "Fira Code", "DejaVu Sans Mono"), weight: "extrabold")}
\NormalTok{\#set page(width: auto, height: auto, margin: .125cm)}

\NormalTok{\#let z80{-}labels = ("A11", ..., "A9", "A10")}

\NormalTok{\#dip{-}chip{-}label(40, 0.54in, z80{-}labels, "Z80", settings: (pin{-}number{-}margin: 1pt, pin{-}number{-}size: 2.5pt, chip{-}label{-}size: 5pt))}
\end{Highlighting}
\end{Shaded}

\pandocbounded{\includegraphics[keepaspectratio]{https://github.com/typst/packages/raw/main/packages/preview/salsa-dip/0.1.0/examples/z80.png}}

\begin{Shaded}
\begin{Highlighting}[]
\NormalTok{\#import "@preview/salsa{-}dip:0.1.0": dip{-}chip{-}label}

\NormalTok{\#set text(font: ("JetBrains Mono", "Fira Code", "DejaVu Sans Mono"), weight: "extrabold")}
\NormalTok{\#set page(width: auto, height: auto, margin: .125cm)}

\NormalTok{\#let labels = ("1A", "1B", "1Y", "2A", "2B", "2Y", "GND", "3Y", "3A", "3B", "4Y", "4A", "4B", "VCC")}
\NormalTok{\#dip{-}chip{-}label(14, 0.24in, labels, "74LS00")}
\end{Highlighting}
\end{Shaded}

\pandocbounded{\includegraphics[keepaspectratio]{https://github.com/typst/packages/raw/main/packages/preview/salsa-dip/0.1.0/examples/74ls00.png}}

The \texttt{\ dip-chip-label\ } function is called with four parameters,
an integer number of pins for the chip, the width (usually
\texttt{\ 0.24in\ } or \texttt{\ 0.54in\ } ), the list of pin labels (if
no labels are desired, an empty array can be passed), and the chip
label.

There is an additional \texttt{\ settings\ } parameter which can be used
to fine tune the apperance of the chip labels. The \texttt{\ settings\ }
parameter is a dictionary optionally containing any of the setting keys:

\begin{itemize}
\tightlist
\item
  \texttt{\ chip-label-size\ } : Font size for the chip label
\item
  \texttt{\ pin-number-margin\ } : Margin to give next to pin numbers
\item
  \texttt{\ pin-number-size\ } : Font size for pin numbers
\item
  \texttt{\ pin-label-size\ } : Font size for pin labels
\item
  \texttt{\ include-numbers\ } : Boolean enabling pin numbers
\item
  \texttt{\ pin-spacing\ } : Spacing of pins
\item
  \texttt{\ vertical-margin\ } : Total margin to put into spacing above
  and below pin labels
\end{itemize}

\subsubsection{How to add}\label{how-to-add}

Copy this into your project and use the import as \texttt{\ salsa-dip\ }

\begin{verbatim}
#import "@preview/salsa-dip:0.1.0"
\end{verbatim}

\includesvg[width=0.16667in,height=0.16667in]{/assets/icons/16-copy.svg}

Check the docs for
\href{https://typst.app/docs/reference/scripting/\#packages}{more
information on how to import packages} .

\subsubsection{About}\label{about}

\begin{description}
\tightlist
\item[Author :]
\href{https://gitlab.com/users/ashplasek}{Ashlen Plasek}
\item[License:]
MIT
\item[Current version:]
0.1.0
\item[Last updated:]
June 17, 2024
\item[First released:]
June 17, 2024
\item[Archive size:]
2.49 kB
\href{https://packages.typst.org/preview/salsa-dip-0.1.0.tar.gz}{\pandocbounded{\includesvg[keepaspectratio]{/assets/icons/16-download.svg}}}
\item[Repository:]
\href{https://gitlab.com/ashplasek/salsa-dip}{GitLab}
\end{description}

\subsubsection{Where to report issues?}\label{where-to-report-issues}

This package is a project of Ashlen Plasek . Report issues on
\href{https://gitlab.com/ashplasek/salsa-dip}{their repository} . You
can also try to ask for help with this package on the
\href{https://forum.typst.app}{Forum} .

Please report this package to the Typst team using the
\href{https://typst.app/contact}{contact form} if you believe it is a
safety hazard or infringes upon your rights.

\phantomsection\label{versions}
\subsubsection{Version history}\label{version-history}

\begin{longtable}[]{@{}ll@{}}
\toprule\noalign{}
Version & Release Date \\
\midrule\noalign{}
\endhead
\bottomrule\noalign{}
\endlastfoot
0.1.0 & June 17, 2024 \\
\end{longtable}

Typst GmbH did not create this package and cannot guarantee correct
functionality of this package or compatibility with any version of the
Typst compiler or app.


\title{typst.app/universe/package/quick-sip}

\phantomsection\label{banner}
\phantomsection\label{template-thumbnail}
\pandocbounded{\includegraphics[keepaspectratio]{https://packages.typst.org/preview/thumbnails/quick-sip-0.1.0-small.webp}}

\section{quick-sip}\label{quick-sip}

{ 0.1.0 }

A template for creating quick reference handbook style checklists.

\href{/app?template=quick-sip&version=0.1.0}{Create project in app}

\phantomsection\label{readme}
Creates aviation style checklists like Quick Reference Handbooks.

\includegraphics[width=3.125in,height=\textheight,keepaspectratio]{https://github.com/typst/packages/raw/main/packages/preview/quick-sip/0.1.0/thumbnail.png}

\subsubsection{Features:}\label{features}

\begin{itemize}
\tightlist
\item
  Index
\item
  Section
\item
  Conditions
\item
  Objective
\item
  Step (When/If)
\item
  Sub Step
\item
  Caution
\item
  Note
\item
  Choose One
\item
  Go to step
\item
  End section now
\end{itemize}

\subsection{Start with}\label{start-with}

\begin{Shaded}
\begin{Highlighting}[]
\NormalTok{\#import "@preview/quick{-}sip:0.1.0": *}
\NormalTok{\#show: QRH.with(title: "Cup of Tea")}
\end{Highlighting}
\end{Shaded}

Then add a section:

\begin{Shaded}
\begin{Highlighting}[]
\NormalTok{\#section("Cup of Tea preparation")[}
\NormalTok{  \#step("KETTLE", "Filled to 1 CUP")}
\NormalTok{  \#step([*When* KETTLE boiled:], "")}
\NormalTok{  \#step([*If* sugar required], "")}
\NormalTok{    //.. Rest of section goes here}
\NormalTok{]}
\end{Highlighting}
\end{Shaded}

\paragraph{Index}\label{index}

An index with an entry for each section in the document.

\begin{Shaded}
\begin{Highlighting}[]
\NormalTok{\#index()}
\end{Highlighting}
\end{Shaded}

\paragraph{Section}\label{section}

A section title, forces capitalisation.

\begin{Shaded}
\begin{Highlighting}[]
\NormalTok{\#section("Cup of Tea preparation")[}
\NormalTok{    //.. Rest of section goes here}
\NormalTok{]}
\end{Highlighting}
\end{Shaded}

\paragraph{Conditions}\label{conditions}

Conditionals for this section.

\begin{Shaded}
\begin{Highlighting}[]
\NormalTok{\#condition[}
\NormalTok{    {-} Dehydration}
\NormalTok{    {-} Fatigue}
\NormalTok{    {-} Inability to Concentrate}
\NormalTok{]}
\end{Highlighting}
\end{Shaded}

\paragraph{Objective}\label{objective}

An objective for this section (optional).

\begin{Shaded}
\begin{Highlighting}[]
\NormalTok{\#objective[To replenish fluids.]}
\end{Highlighting}
\end{Shaded}

\paragraph{Step}\label{step}

A numbered step in the checklist. The first parameter is to the left of
the dotted line, the second is to the right. If the second parameter is
\texttt{\ ""\ } then there is no dotted line.

\begin{Shaded}
\begin{Highlighting}[]
\NormalTok{\#step("KETTLE", "Filled to 1 CUP")}
\NormalTok{\#step([*When* KETTLE boiled:], "")}
\NormalTok{\#step([*If* sugar required], "")}
\end{Highlighting}
\end{Shaded}

\paragraph{Tab}\label{tab}

Indents contents by one tab.

\begin{Shaded}
\begin{Highlighting}[]
\NormalTok{\#tab(goto("9"))}
\NormalTok{\#tab(tab("Large mugs may require more water."))}
\end{Highlighting}
\end{Shaded}

\paragraph{Caution}\label{caution}

Adds a caution element.

\begin{Shaded}
\begin{Highlighting}[]
\NormalTok{\#caution([HOT WATER \#linebreak()Adult supervision required.])}
\end{Highlighting}
\end{Shaded}

\paragraph{Note}\label{note}

Adds a note.

\begin{Shaded}
\begin{Highlighting}[]
\NormalTok{\#note("Stir after each step")}
\end{Highlighting}
\end{Shaded}

\paragraph{Choose One}\label{choose-one}

A numbered step with options.

\begin{Shaded}
\begin{Highlighting}[]
\NormalTok{ \#choose{-}one[}
\NormalTok{    \#option[Black tea *required:*]}
\NormalTok{    \#option[Tea with MILK *required:*]}
\NormalTok{  ]}
\end{Highlighting}
\end{Shaded}

\paragraph{Go to step}\label{go-to-step}

Two right facing arrow heads followed by Go to step
\texttt{\ step\ number\ } . Links to step in pdf.

\begin{Shaded}
\begin{Highlighting}[]
\NormalTok{\#goto("9")}
\end{Highlighting}
\end{Shaded}

\paragraph{End}\label{end}

Ends the section here with 4 dots.

\begin{Shaded}
\begin{Highlighting}[]
\NormalTok{\#end()}
\end{Highlighting}
\end{Shaded}

\paragraph{Wait}\label{wait}

Long small dotted line for waiting for a task to complete.

\begin{Shaded}
\begin{Highlighting}[]
\NormalTok{\#wait()}
\end{Highlighting}
\end{Shaded}

\href{/app?template=quick-sip&version=0.1.0}{Create project in app}

\subsubsection{How to use}\label{how-to-use}

Click the button above to create a new project using this template in
the Typst app.

You can also use the Typst CLI to start a new project on your computer
using this command:

\begin{verbatim}
typst init @preview/quick-sip:0.1.0
\end{verbatim}

\includesvg[width=0.16667in,height=0.16667in]{/assets/icons/16-copy.svg}

\subsubsection{About}\label{about}

\begin{description}
\tightlist
\item[Author :]
\href{https://github.com/artomweb}{Archie Webster}
\item[License:]
MIT
\item[Current version:]
0.1.0
\item[Last updated:]
October 16, 2024
\item[First released:]
October 16, 2024
\item[Archive size:]
4.28 kB
\href{https://packages.typst.org/preview/quick-sip-0.1.0.tar.gz}{\pandocbounded{\includesvg[keepaspectratio]{/assets/icons/16-download.svg}}}
\item[Repository:]
\href{https://github.com/artomweb/Quick-Sip-Typst-Template}{GitHub}
\item[Categor y :]
\begin{itemize}
\tightlist
\item[]
\item
  \pandocbounded{\includesvg[keepaspectratio]{/assets/icons/16-hammer.svg}}
  \href{https://typst.app/universe/search/?category=utility}{Utility}
\end{itemize}
\end{description}

\subsubsection{Where to report issues?}\label{where-to-report-issues}

This template is a project of Archie Webster . Report issues on
\href{https://github.com/artomweb/Quick-Sip-Typst-Template}{their
repository} . You can also try to ask for help with this template on the
\href{https://forum.typst.app}{Forum} .

Please report this template to the Typst team using the
\href{https://typst.app/contact}{contact form} if you believe it is a
safety hazard or infringes upon your rights.

\phantomsection\label{versions}
\subsubsection{Version history}\label{version-history}

\begin{longtable}[]{@{}ll@{}}
\toprule\noalign{}
Version & Release Date \\
\midrule\noalign{}
\endhead
\bottomrule\noalign{}
\endlastfoot
0.1.0 & October 16, 2024 \\
\end{longtable}

Typst GmbH did not create this template and cannot guarantee correct
functionality of this template or compatibility with any version of the
Typst compiler or app.


\title{typst.app/universe/package/frackable}

\phantomsection\label{banner}
\section{frackable}\label{frackable}

{ 0.2.0 }

Vulgar Fractions

\phantomsection\label{readme}
Version 0.2.0

Provides a function,
\texttt{\ frackable(numerator,\ denominator,\ whole:\ none)\ } , to
typeset vulgar and mixed fractions. Provides a second
\texttt{\ generator(...)\ } function that returns another having the
same signature as \texttt{\ frackable\ } to typeset arbitrary vulgar and
mixed fractions in fonts that do not support the \texttt{\ frac\ }
feature.

\begin{Shaded}
\begin{Highlighting}[]
\NormalTok{\#import "@preview/frackable:0.2.0": *}

\NormalTok{\#frackable(1, 2)}
\NormalTok{\#frackable(1, 3)}
\NormalTok{\#frackable(3, 4, whole: 9)}
\NormalTok{\#frackable(9, 16)}
\NormalTok{\#frackable(31, 32)}
\NormalTok{\#frackable(0, "000")}
\end{Highlighting}
\end{Shaded}

\pandocbounded{\includegraphics[keepaspectratio]{https://github.com/typst/packages/raw/main/packages/preview/frackable/0.2.0/example.png}}

\subsubsection{How to add}\label{how-to-add}

Copy this into your project and use the import as \texttt{\ frackable\ }

\begin{verbatim}
#import "@preview/frackable:0.2.0"
\end{verbatim}

\includesvg[width=0.16667in,height=0.16667in]{/assets/icons/16-copy.svg}

Check the docs for
\href{https://typst.app/docs/reference/scripting/\#packages}{more
information on how to import packages} .

\subsubsection{About}\label{about}

\begin{description}
\tightlist
\item[Author :]
James R. Swift
\item[License:]
Unlicense
\item[Current version:]
0.2.0
\item[Last updated:]
September 27, 2024
\item[First released:]
September 24, 2024
\item[Minimum Typst version:]
0.11.0
\item[Archive size:]
2.83 kB
\href{https://packages.typst.org/preview/frackable-0.2.0.tar.gz}{\pandocbounded{\includesvg[keepaspectratio]{/assets/icons/16-download.svg}}}
\item[Repository:]
\href{https://www.github.com/jamesrswift/frackable}{GitHub}
\item[Categor ies :]
\begin{itemize}
\tightlist
\item[]
\item
  \pandocbounded{\includesvg[keepaspectratio]{/assets/icons/16-package.svg}}
  \href{https://typst.app/universe/search/?category=components}{Components}
\item
  \pandocbounded{\includesvg[keepaspectratio]{/assets/icons/16-text.svg}}
  \href{https://typst.app/universe/search/?category=text}{Text}
\end{itemize}
\end{description}

\subsubsection{Where to report issues?}\label{where-to-report-issues}

This package is a project of James R. Swift . Report issues on
\href{https://www.github.com/jamesrswift/frackable}{their repository} .
You can also try to ask for help with this package on the
\href{https://forum.typst.app}{Forum} .

Please report this package to the Typst team using the
\href{https://typst.app/contact}{contact form} if you believe it is a
safety hazard or infringes upon your rights.

\phantomsection\label{versions}
\subsubsection{Version history}\label{version-history}

\begin{longtable}[]{@{}ll@{}}
\toprule\noalign{}
Version & Release Date \\
\midrule\noalign{}
\endhead
\bottomrule\noalign{}
\endlastfoot
0.2.0 & September 27, 2024 \\
\href{https://typst.app/universe/package/frackable/0.1.0/}{0.1.0} &
September 24, 2024 \\
\end{longtable}

Typst GmbH did not create this package and cannot guarantee correct
functionality of this package or compatibility with any version of the
Typst compiler or app.


\title{typst.app/universe/package/dashing-dept-news}

\phantomsection\label{banner}
\phantomsection\label{template-thumbnail}
\pandocbounded{\includegraphics[keepaspectratio]{https://packages.typst.org/preview/thumbnails/dashing-dept-news-0.1.1-small.webp}}

\section{dashing-dept-news}\label{dashing-dept-news}

{ 0.1.1 }

Share the news with bold graphic design and a modern layout

{ } Featured Template

\href{/app?template=dashing-dept-news&version=0.1.1}{Create project in
app}

\phantomsection\label{readme}
A fun newsletter layout for departmental news. The template contains a
hero image, a main column, and a margin with secondary articles.

Place content in the sidebar with the \texttt{\ article\ } function, and
use the cool customized \texttt{\ blockquote\ } s and
\texttt{\ figure\ } s!

\subsection{Usage}\label{usage}

You can use this template in the Typst web app by clicking “Start from
template� on the dashboard and searching for
\texttt{\ dashing-dept-news\ } .

Alternatively, you can use the CLI to kick this project off using the
command

\begin{verbatim}
typst init @preview/dashing-dept-news
\end{verbatim}

Typst will create a new directory with all the files needed to get you
started.

\subsection{Configuration}\label{configuration}

This template exports the \texttt{\ newsletter\ } function with the
following named arguments:

\begin{itemize}
\tightlist
\item
  \texttt{\ title\ } : The newsletter’s title as content.
\item
  \texttt{\ edition\ } : The edition of the newsletter as content or
  \texttt{\ none\ } . This is displayed at the top of the sidebar.
\item
  \texttt{\ hero-image\ } : A dictionary with the keys
  \texttt{\ image\ } and \texttt{\ caption\ } or \texttt{\ none\ } .
  Image is content with the hero image while \texttt{\ caption\ } is
  content that is displayed to the right of the image.
\item
  \texttt{\ publication-info\ } : More information about the publication
  as content or \texttt{\ none\ } . It is displayed at the end of the
  document.
\end{itemize}

The function also accepts a single, positional argument for the body of
the newsletter’s main column and exports the \texttt{\ article\ }
function accepting a single content argument to populate the sidebar.

The template will initialize your package with a sample call to the
\texttt{\ newsletter\ } function in a show rule. If you, however, want
to change an existing project to use this template, you can add a show
rule like this at the top of your file:

\begin{Shaded}
\begin{Highlighting}[]
\NormalTok{\#import "@preview/dashing{-}dept{-}news:0.1.1": newsletter, article}
\NormalTok{\#show: newsletter.with(}
\NormalTok{  title: [Chemistry Department],}
\NormalTok{  edition: [}
\NormalTok{    March 18th, 2023 \textbackslash{}}
\NormalTok{    Purview College}
\NormalTok{  ],}
\NormalTok{  hero{-}image: (}
\NormalTok{    image: image("newsletter{-}cover.jpg"),}
\NormalTok{    caption: [Award{-}wining science],}
\NormalTok{  ),}
\NormalTok{  publication{-}info: [}
\NormalTok{    The Dean of the Department of Chemistry. \textbackslash{}}
\NormalTok{    Purview College, 17 Earlmeyer D, Exampleville, TN 59341. \textbackslash{}}
\NormalTok{    \#link("mailto:newsletter@chem.purview.edu")}
\NormalTok{  ],}
\NormalTok{)}

\NormalTok{// Your content goes here. Use \textasciigrave{}article\textasciigrave{} to populate the sidebar and \textasciigrave{}blockquote\textasciigrave{} for cool pull quotes.}
\end{Highlighting}
\end{Shaded}

\href{/app?template=dashing-dept-news&version=0.1.1}{Create project in
app}

\subsubsection{How to use}\label{how-to-use}

Click the button above to create a new project using this template in
the Typst app.

You can also use the Typst CLI to start a new project on your computer
using this command:

\begin{verbatim}
typst init @preview/dashing-dept-news:0.1.1
\end{verbatim}

\includesvg[width=0.16667in,height=0.16667in]{/assets/icons/16-copy.svg}

\subsubsection{About}\label{about}

\begin{description}
\tightlist
\item[Author :]
\href{https://typst.app}{Typst GmbH}
\item[License:]
MIT-0
\item[Current version:]
0.1.1
\item[Last updated:]
October 29, 2024
\item[First released:]
March 6, 2024
\item[Minimum Typst version:]
0.11.0
\item[Archive size:]
125 kB
\href{https://packages.typst.org/preview/dashing-dept-news-0.1.1.tar.gz}{\pandocbounded{\includesvg[keepaspectratio]{/assets/icons/16-download.svg}}}
\item[Repository:]
\href{https://github.com/typst/templates}{GitHub}
\item[Categor y :]
\begin{itemize}
\tightlist
\item[]
\item
  \pandocbounded{\includesvg[keepaspectratio]{/assets/icons/16-envelope.svg}}
  \href{https://typst.app/universe/search/?category=office}{Office}
\end{itemize}
\end{description}

\subsubsection{Where to report issues?}\label{where-to-report-issues}

This template is a project of Typst GmbH . Report issues on
\href{https://github.com/typst/templates}{their repository} . You can
also try to ask for help with this template on the
\href{https://forum.typst.app}{Forum} .

\phantomsection\label{versions}
\subsubsection{Version history}\label{version-history}

\begin{longtable}[]{@{}ll@{}}
\toprule\noalign{}
Version & Release Date \\
\midrule\noalign{}
\endhead
\bottomrule\noalign{}
\endlastfoot
0.1.1 & October 29, 2024 \\
\href{https://typst.app/universe/package/dashing-dept-news/0.1.0/}{0.1.0}
& March 6, 2024 \\
\end{longtable}


\title{typst.app/universe/package/genealotree}

\phantomsection\label{banner}
\section{genealotree}\label{genealotree}

{ 0.1.0 }

A package to draw genealogical trees, based on CeTZ

\phantomsection\label{readme}
Genealotree is a typst package to draw genealogical trees. It is
developped at \url{https://codeberg.org/drloiseau/genealogy} . This is
the place you can get the developpement version and send issues and pull
requests.

\pandocbounded{\includegraphics[keepaspectratio]{https://github.com/typst/packages/raw/main/packages/preview/genealotree/0.1.0/examples/example.jpg}}

This package is based on
\href{https://github.com/typst/packages/raw/main/packages/preview/genealotree/0.1.0/\%22https://typst.app/universe/package/cetz/\%22}{CeTZ}
and it provides functions to draw genealogical trees. It is oriented
towards medical genealogy to study genetic disorders inheritance, but
you might be able to use it to draw your family tree.

\textbf{Features :}

\begin{itemize}
\tightlist
\item
  Draw an unlimited number of independant genealogical trees
\item
  Supports consanguinity and unions between different trees (see
  limitations)
\item
  Auto adjusts position of children to optimize spacing
\item
  Customize all lengths
\item
  Draw as much phenotypes as needed by coloring individuals
\item
  Print genotype and/or phenotype labels under individuals
\end{itemize}

\textbf{Limitations :}

\begin{itemize}
\tightlist
\item
  Must manually adjust individual position in the tree when drawing
  consanguinity and unions between trees to prevent overlapping of
  individuals.
\item
  No remarriages (might be added in a future version)
\item
  No union between individuals at different generations (might be added
  in a future version)
\end{itemize}

\textbf{To be implemented :}

\begin{itemize}
\tightlist
\item
  Allow to pass CeTZ arguments to every elements to cutomize their
  appearance
\item
  Draw optional legends for tree symbols and for phenotypes
\end{itemize}

See example.typ for a simple usage example, and the manual for precise
references at
\href{https://codeberg.org/attachments/cfdad2b7-52ae-4e18-8d7b-453fadc45532}{manual.pdf}

The steps to produce a tree are :

\begin{itemize}
\tightlist
\item
  Import the package and CeTZ
\end{itemize}

\begin{Shaded}
\begin{Highlighting}[]
\NormalTok{\#import "@preview/genealotree:0.1.0": *}
\NormalTok{\#import "@preview/cetz:0.2.2": canvas}
\end{Highlighting}
\end{Shaded}

\begin{itemize}
\tightlist
\item
  Create a genealogy object
\end{itemize}

\begin{Shaded}
\begin{Highlighting}[]
\NormalTok{let my{-}geneal = genealogy{-}init()}
\end{Highlighting}
\end{Shaded}

\begin{itemize}
\tightlist
\item
  Add persons to the object : pass a dictionary mapping a persons name
  with a dictionary describing its characteristics. See the manual for a
  full reference.
\end{itemize}

\begin{Shaded}
\begin{Highlighting}[]
\NormalTok{let my{-}geneal = add{-}persons(}
\NormalTok{  my{-}geneal,}
\NormalTok{  (}
\NormalTok{    "I1": (sex: "m"),}
\NormalTok{    "I2": (sex: "f"),}
\NormalTok{    "II1": (sex: "f"),}
\NormalTok{  )}
\NormalTok{)}
\end{Highlighting}
\end{Shaded}

\begin{itemize}
\tightlist
\item
  Set unions between persons : give the parents names as an array of 2
  strings, and the children names as an array of strings.
\end{itemize}

\begin{Shaded}
\begin{Highlighting}[]
\NormalTok{let my{-}geneal = add{-}unions(}
\NormalTok{  my{-}geneal,}
\NormalTok{  (("I1", "I2"), ("II1",))}
\NormalTok{)}
\end{Highlighting}
\end{Shaded}

\begin{itemize}
\tightlist
\item
  Open up a CeTZ canva and draw the tree
\end{itemize}

\begin{Shaded}
\begin{Highlighting}[]
\NormalTok{\#canvas(length: 0.4cm, \{}
\NormalTok{    draw{-}tree(my{-}geneal)}
\NormalTok{\})}
\end{Highlighting}
\end{Shaded}

\subsubsection{How to add}\label{how-to-add}

Copy this into your project and use the import as
\texttt{\ genealotree\ }

\begin{verbatim}
#import "@preview/genealotree:0.1.0"
\end{verbatim}

\includesvg[width=0.16667in,height=0.16667in]{/assets/icons/16-copy.svg}

Check the docs for
\href{https://typst.app/docs/reference/scripting/\#packages}{more
information on how to import packages} .

\subsubsection{About}\label{about}

\begin{description}
\tightlist
\item[Author :]
DrLoiseau
\item[License:]
GPL-3.0-only
\item[Current version:]
0.1.0
\item[Last updated:]
May 23, 2024
\item[First released:]
May 23, 2024
\item[Minimum Typst version:]
0.10.0
\item[Archive size:]
22.9 kB
\href{https://packages.typst.org/preview/genealotree-0.1.0.tar.gz}{\pandocbounded{\includesvg[keepaspectratio]{/assets/icons/16-download.svg}}}
\item[Repository:]
\href{https://codeberg.org/drloiseau/genealogy}{Codeberg}
\item[Discipline s :]
\begin{itemize}
\tightlist
\item[]
\item
  \href{https://typst.app/universe/search/?discipline=anthropology}{Anthropology}
\item
  \href{https://typst.app/universe/search/?discipline=biology}{Biology}
\item
  \href{https://typst.app/universe/search/?discipline=history}{History}
\item
  \href{https://typst.app/universe/search/?discipline=medicine}{Medicine}
\end{itemize}
\item[Categor y :]
\begin{itemize}
\tightlist
\item[]
\item
  \pandocbounded{\includesvg[keepaspectratio]{/assets/icons/16-chart.svg}}
  \href{https://typst.app/universe/search/?category=visualization}{Visualization}
\end{itemize}
\end{description}

\subsubsection{Where to report issues?}\label{where-to-report-issues}

This package is a project of DrLoiseau . Report issues on
\href{https://codeberg.org/drloiseau/genealogy}{their repository} . You
can also try to ask for help with this package on the
\href{https://forum.typst.app}{Forum} .

Please report this package to the Typst team using the
\href{https://typst.app/contact}{contact form} if you believe it is a
safety hazard or infringes upon your rights.

\phantomsection\label{versions}
\subsubsection{Version history}\label{version-history}

\begin{longtable}[]{@{}ll@{}}
\toprule\noalign{}
Version & Release Date \\
\midrule\noalign{}
\endhead
\bottomrule\noalign{}
\endlastfoot
0.1.0 & May 23, 2024 \\
\end{longtable}

Typst GmbH did not create this package and cannot guarantee correct
functionality of this package or compatibility with any version of the
Typst compiler or app.


\title{typst.app/universe/package/fauxreilly}

\phantomsection\label{banner}
\section{fauxreilly}\label{fauxreilly}

{ 0.1.0 }

A package for creating O\textquotesingle Rly- /
O\textquotesingle Reilly-type cover pages

\phantomsection\label{readme}
\href{https://forthebadge.com/}{\pandocbounded{\includesvg[keepaspectratio]{https://raw.githubusercontent.com/dei-layborer/o-rly-typst/refs/heads/main/made-with-(2s)-2\%2C6-diamino-n-\%5B(2s)-1-phenylpropan-2-yl\%5Dhexanamide-n-\%5B(2s)-1-phenyl-2-propanyl\%5D-l-lysinamide.svg}}}

\href{https://deilayborer.neocities.org/funding}{\includesvg[width=\linewidth,height=0.3125in,keepaspectratio]{https://raw.githubusercontent.com/dei-layborer/o-rly-typst/refs/heads/main/\%24\%24\%24-gimmie.svg}}

A \texttt{\ typst\ } package for creating \textbf{O’RLY?} -style cover
pages.

\subsection{Example}\label{example}

\begin{Shaded}
\begin{Highlighting}[]
\NormalTok{\#import }\StringTok{"@preview/o{-}rly{-}cover:0.1.0"}\OperatorTok{:} \OperatorTok{*}

\NormalTok{\#orly(}
\NormalTok{    color}\OperatorTok{:}\NormalTok{ rgb(}\StringTok{"\#85144b"}\NormalTok{)}\OperatorTok{,}
\NormalTok{    title}\OperatorTok{:} \StringTok{"Learn to Stop Worrying and Love Metathesis"}\OperatorTok{,}
\NormalTok{    top}\OperatorTok{{-}}\NormalTok{text}\OperatorTok{:} \StringTok{"Axe nat why (or do)"}\OperatorTok{,}
\NormalTok{    subtitle}\OperatorTok{:} \StringTok{"Free yourself from prescriptivism"}\OperatorTok{,}
\NormalTok{    pic}\OperatorTok{:} \StringTok{"chomskydoz.png"}\OperatorTok{,}
\NormalTok{    signature}\OperatorTok{:} \StringTok{"Dr. N. Supponent"}
\NormalTok{)}
\end{Highlighting}
\end{Shaded}

\pandocbounded{\includegraphics[keepaspectratio]{https://github.com/typst/packages/raw/main/packages/preview/fauxreilly/0.1.0/example.png}}

\subsection{Usage}\label{usage}

First, import the package at the top of your \texttt{\ typst\ } file:
\texttt{\ \#import\ "@preview/o-rly-cover:0.1.0":\ *\ }

Only one function is exposed, \texttt{\ \#orly()\ } . This will create
its own page in the document at whatever location you call the function.
In other words, any content in the \texttt{\ typst\ } document that
appears before \texttt{\ \#orly()\ } is called will be before the
O’Rly? page in the PDF that \texttt{\ typst\ } renders. Anything after
the function call will be on subsequent page(s).

All content for the title page is passed as options to
\texttt{\ \#orly()\ } . I included what I figured were the most likely
things you’d want to customize without having a million options.
Meanwhile, most of the layout parameters (font sizes, the heights of
individual pieces, etc.) are variables within the code, so hopefully
aren’t too hard to alter if need-be. None of the options are strictly
required, although the text fields are the only ones that can be left
empty without potentially breaking the layout. A few have defaults
instead, and those are listed below where applicable.

\subsubsection{Options}\label{options}

The order that the options appear in the table is the order they must be
sent to the function, unless you specify the option’s key along with
its value.

Data types listed are based on \texttt{\ typst\ } ’s internal types,
so are entered the same way as they would be in any other function that
takes that data type. For example, the data type needed for the
\texttt{\ font\ } option is the same as what is used for
\texttt{\ typst\ } ’s built-in \texttt{\ \#text()\ } function, which
is linked in the table below. (All links go to their specific usage in
the \texttt{\ typst\ } documentation.)

\begin{longtable}[]{@{}llcc@{}}
\toprule\noalign{}
Option & Description & Type & Default \\
\midrule\noalign{}
\endhead
\bottomrule\noalign{}
\endlastfoot
\texttt{\ font\ } & The font for all text except the “publisher� in
the bottom-left corner &
\href{https://typst.app/docs/reference/text/text/\#parameters-font}{\texttt{\ string(s)\ }}
& Whatever is set in the document context \\
\texttt{\ color\ } & Accent color. Used for the background of the title
block and of the colored bar at the very top. &
\href{https://typst.app/docs/reference/visualize/color/}{\texttt{\ color\ }}
& \texttt{\ blue\ } (typst built-in) \\
\texttt{\ top-text\ } & The text at the top, just under the color bar &
\href{https://typst.app/docs/reference/foundations/str/}{\texttt{\ string\ }}
& Empty \\
\texttt{\ pic\ } & Image to be used above the title block &
\href{https://typst.app/docs/reference/visualize/image/\#parameters-path}{\texttt{\ string\ }}
with path to the image & Empty \\
\texttt{\ title\ } & The title of the book &
\href{https://typst.app/docs/reference/foundations/str/}{\texttt{\ string\ }}
& Empty \\
\texttt{\ title-align\ } & How the text is aligned (horizontally) in the
title block &
\href{https://typst.app/docs/reference/layout/alignment/}{\texttt{\ alignment\ }}
& \texttt{\ left\ } \\
\texttt{\ subtitle\ } & Text that appears just below the title block &
\href{https://typst.app/docs/reference/foundations/str/}{\texttt{\ string\ }}
& Empty \\
\texttt{\ publisher\ } & The name of the “publisher� in the
bottom-left &
\href{https://typst.app/docs/reference/foundations/str/}{\texttt{\ string\ }}
& O RLY \textsuperscript{?} (see example above) \\
\texttt{\ publisher-font\ } & Font to be used for “publisher� name &
\href{https://typst.app/docs/reference/text/text/\#parameters-font}{\texttt{\ string(s)\ }}
& Noto Sans, Arial Rounded MT, document context (in that order) \\
\texttt{\ signature\ } & Text in the bottom-right corner &
\href{https://typst.app/docs/reference/foundations/str/}{\texttt{\ string\ }}
& Empty \\
\texttt{\ margin\ } & Page margins &
\href{https://typst.app/docs/reference/layout/page/\#parameters-margin}{\texttt{\ length\ }
or \texttt{\ dictionary\ }} & \texttt{\ top:\ 0\ } , all others will use
the document context \\
\end{longtable}

\subsubsection{Usage Notes}\label{usage-notes}

There are a couple quirks to data types and the like that may not be
obvious.

\begin{enumerate}
\tightlist
\item
  \texttt{\ string\ } s typically must be contained in quotation marks.
  But note that this will render quotation marks \emph{within} those
  strings without using
  \href{https://typst.app/docs/reference/text/smartquote/}{smartquotes}
  . To avoid this, you may use content mode instead (by enclosing the
  text in square brackets \texttt{\ {[}{]}\ } ). For example,
  \texttt{\ "Some\ title"\ } â†' \texttt{\ {[}Some\ title{]}\ }

  \begin{itemize}
  \tightlist
  \item
    Similarly, you can use this to toggle italics (e.g.
    \texttt{\ {[}Italic\ text,\ \_except\_\ this\ one{]}\ } ) or apply
    other formatting
  \end{itemize}
\item
  Other types may look like strings but do \textbf{not} take quotes,
  specifically \texttt{\ color\ } (including when using the built-in
  color names) and \texttt{\ alignment\ }
\item
  With the \texttt{\ margin\ } type, if a single value is entered
  (without quotes), that value is applied to all four sides. All other
  usage must be done as a dictionary (meaning enclosed in parentheses),
  even if you’re only specifying one side. For example, the default is
  written \texttt{\ (top:\ 0in)\ } .

  \begin{itemize}
  \tightlist
  \item
    If you’re going to pass any value other than the top as an option,
    you’ll likely want to add \texttt{\ top:\ 0in\ } back in to avoid
    a gap between the top of the page and the color bar
  \item
    Any values passed to the function (or the default value if none are)
    will override any margin(s) set earlier in the \texttt{\ typst\ }
    file. So you can use a \texttt{\ set\ } rule at the beginning of the
    document without affecting the cover page
  \end{itemize}
\end{enumerate}

\subsubsection{Images}\label{images}

The package uses \texttt{\ typst\ } ’s built-in image handling, which
means it only supports PNG, JPG, and SVG.

The image will be resized to as close to 100\% page width (inside the
margins) as possible while both maintaining proportions and avoiding any
cropping. The rest of the layout \emph{should} flow reasonably well
around any image height, but outliers may exist.

O’Reilly-style animals can be found in the
\href{https://etc.usf.edu/clipart/galleries/730-animals}{relevant
section} of the Florida Center for Instructional Technology’s
\href{https://etc.usf.edu/clipart/}{ClipArt ETC} project. Just be aware
that these are provided as GIFs(!), so conversion to one of
\texttt{\ typst\ } ’s supported formats will be required.

\subsection{Bugs \& Feature Requests}\label{bugs-feature-requests}

I put this whole thing together in an afternoon when I should’ve been
doing work for my day job. Granted I’d already done a basic version
for a seminary writing assignment (I love to spoof academic writing),
but either way, I’ve gotten this project to a basic level of
functionality and no further. I’m entirely open to suggestions for
additional functionality, however, so feel free to
\href{https://github.com/dei-layborer/o-rly-typst/issues}{create an
issue} if there’s something you’d like to see added.

It hopefully goes without saying that the same is true if something
breaks!

Tested on \texttt{\ typst\ } versions \texttt{\ 0.11.1\ } and
\texttt{\ 0.12.0-rc1\ } .

\subsection{Thanks \& Shout-Outs}\label{thanks-shout-outs}

Shout out to Arthur Beaulieu (
\href{https://github.com/ArthurBeaulieu}{@arthurbeaulieu} ), whose
\href{https://arthurbeaulieu.github.io/ORlyGenerator/}{web-based
generator} served as both inspiration and reference (I pretty much stole
his layout settings).

Significant thanks to the folks in the
\href{https://discord.gg/2uDybryKPe}{typst discord} who helped me sort
out some layout woes.

Extra double appreciation to Enivex on the discord for the name.

\subsubsection{How to add}\label{how-to-add}

Copy this into your project and use the import as
\texttt{\ fauxreilly\ }

\begin{verbatim}
#import "@preview/fauxreilly:0.1.0"
\end{verbatim}

\includesvg[width=0.16667in,height=0.16667in]{/assets/icons/16-copy.svg}

Check the docs for
\href{https://typst.app/docs/reference/scripting/\#packages}{more
information on how to import packages} .

\subsubsection{About}\label{about}

\begin{description}
\tightlist
\item[Author :]
Dei Layborer
\item[License:]
GPL-3.0
\item[Current version:]
0.1.0
\item[Last updated:]
October 16, 2024
\item[First released:]
October 16, 2024
\item[Minimum Typst version:]
0.11.1
\item[Archive size:]
4.48 kB
\href{https://packages.typst.org/preview/fauxreilly-0.1.0.tar.gz}{\pandocbounded{\includesvg[keepaspectratio]{/assets/icons/16-download.svg}}}
\item[Repository:]
\href{https://github.com/dei-layborer/o-rly-typst}{GitHub}
\item[Categor ies :]
\begin{itemize}
\tightlist
\item[]
\item
  \pandocbounded{\includesvg[keepaspectratio]{/assets/icons/16-package.svg}}
  \href{https://typst.app/universe/search/?category=components}{Components}
\item
  \pandocbounded{\includesvg[keepaspectratio]{/assets/icons/16-layout.svg}}
  \href{https://typst.app/universe/search/?category=layout}{Layout}
\item
  \pandocbounded{\includesvg[keepaspectratio]{/assets/icons/16-smile.svg}}
  \href{https://typst.app/universe/search/?category=fun}{Fun}
\end{itemize}
\end{description}

\subsubsection{Where to report issues?}\label{where-to-report-issues}

This package is a project of Dei Layborer . Report issues on
\href{https://github.com/dei-layborer/o-rly-typst}{their repository} .
You can also try to ask for help with this package on the
\href{https://forum.typst.app}{Forum} .

Please report this package to the Typst team using the
\href{https://typst.app/contact}{contact form} if you believe it is a
safety hazard or infringes upon your rights.

\phantomsection\label{versions}
\subsubsection{Version history}\label{version-history}

\begin{longtable}[]{@{}ll@{}}
\toprule\noalign{}
Version & Release Date \\
\midrule\noalign{}
\endhead
\bottomrule\noalign{}
\endlastfoot
0.1.0 & October 16, 2024 \\
\end{longtable}

Typst GmbH did not create this package and cannot guarantee correct
functionality of this package or compatibility with any version of the
Typst compiler or app.


\title{typst.app/universe/package/headcount}

\phantomsection\label{banner}
\section{headcount}\label{headcount}

{ 0.1.0 }

Make counters inherit from the heading counter.

\phantomsection\label{readme}
This package allows you to make \textbf{counters depend on the current
chapter/section number} .

This works for \textbf{figures, theorems, and any other counters} .

The advantage compared to
\href{https://typst.app/universe/package/rich-counters/}{rich-counters}
is that you stick with native \texttt{\ counter\ } s and you can
influence e.g. the \texttt{\ figure\ } counter directly without writing
a new \texttt{\ show\ } rule with a custom counter or so.

\subsection{Showcase}\label{showcase}

In the following example, we demonstrate how you can inherit 1 level of
the heading counter for figures and 2 levels for theorems.

\begin{Shaded}
\begin{Highlighting}[]
\NormalTok{\#import "@preview/headcount:0.1.0": *}
\NormalTok{\#import "@preview/great{-}theorems:0.1.0": *}

\NormalTok{\#show: great{-}theorems{-}init}

\NormalTok{\#set heading(numbering: "1.1")}

\NormalTok{// contruct theorem environment with counter that inherits 2 levels from heading}
\NormalTok{\#let thmcounter = counter("hello")}
\NormalTok{\#let theorem = mathblock(}
\NormalTok{  blocktitle: [Theorem],}
\NormalTok{  counter: thmcounter,}
\NormalTok{  numbering: dependent{-}numbering("1.1", levels: 2)}
\NormalTok{)}
\NormalTok{\#show heading: reset{-}counter(thmcounter, levels: 2)}

\NormalTok{// set figure counter so that it inherits 1 level from heading}
\NormalTok{\#set figure(numbering: dependent{-}numbering("1.1"))}
\NormalTok{\#show heading: reset{-}counter(counter(figure.where(kind: image)))}

\NormalTok{= First heading}

\NormalTok{The theorems inherit 2 levels from the headings and the figures inherit 1 level from the headings.}

\NormalTok{\#theorem[Some theorem.]}
\NormalTok{\#theorem[Some theorem.]}
\NormalTok{\#figure([SOME FIGURE], caption: [some figure])}
\NormalTok{\#figure([SOME FIGURE], caption: [some figure])}

\NormalTok{== Subheading}

\NormalTok{\#theorem[Some theorem.]}
\NormalTok{\#figure([SOME FIGURE], caption: [some figure])}
\NormalTok{\#figure([SOME FIGURE], caption: [some figure])}

\NormalTok{= Second heading}

\NormalTok{\#theorem[Some theorem.]}
\NormalTok{\#figure([SOME FIGURE], caption: [some figure])}
\NormalTok{\#theorem[Some theorem.]}
\end{Highlighting}
\end{Shaded}

\pandocbounded{\includegraphics[keepaspectratio]{https://github.com/typst/packages/raw/main/packages/preview/headcount/0.1.0/example.png}}

\subsection{Usage}\label{usage}

To make another \texttt{\ counter\ } inherit from the heading counter,
you have to do \textbf{two} things.

\begin{enumerate}
\item
  For the numbering of your counter, you have to use
  \texttt{\ dependent-numbering(...)\ } .

  \begin{itemize}
  \item
    \texttt{\ dependent-numbering(style,\ level:\ 1)\ } (needs
    \texttt{\ context\ } )

    Is a replacement for the \texttt{\ numbering\ } function, with the
    difference that it precedes any counter value with
    \texttt{\ level\ } many values of the heading counter.
  \end{itemize}

\begin{Shaded}
\begin{Highlighting}[]
\NormalTok{\#import "@preview/headcount:0.1.0": *}

\NormalTok{\#set heading(numbering: "1.1")}

\NormalTok{\#let mycounter = counter("hello")}

\NormalTok{= First heading}

\NormalTok{\#context mycounter.step()}
\NormalTok{\#context mycounter.display(dependent{-}numbering("1.1"))}

\NormalTok{= Second heading}

\NormalTok{\#context mycounter.step()}
\NormalTok{\#context mycounter.display(dependent{-}numbering("1.1"))}

\NormalTok{\#context mycounter.step()}
\NormalTok{\#context mycounter.display(dependent{-}numbering("1.1"))}
\end{Highlighting}
\end{Shaded}

  This displays the desired amount of levels of the heading counter in
  front of the actual counter. However, as you can see in the code
  above, our actual counter does not yet reset in each section.
\item
  For resetting the counter at the appropriate places, you need to equip
  \texttt{\ heading\ } with the \texttt{\ show\ } rule that
  \texttt{\ reset-counter(...)\ } returns.

  \begin{itemize}
  \item
    \texttt{\ reset-counter(counter,\ level:\ 1)\ } (needs
    \texttt{\ context\ } )

    Returns a function that should be used as a \texttt{\ show\ } rule
    for \texttt{\ heading\ } . It will reset \texttt{\ counter\ } if the
    level of the heading is less than or equal to \texttt{\ level\ } .
  \end{itemize}

  \textbf{Important:} This \texttt{\ show\ } rule should be placed as
  the \emph{last} \texttt{\ show\ } rule for \texttt{\ heading\ } , or
  at least after \texttt{\ show\ } rules for \texttt{\ heading\ } that
  employ a custom design, see
  \href{https://forum.typst.app/t/i-figured-broken-with-custom-template/1730/10?u=jbirnick}{here}
  for an explanation.

\begin{Shaded}
\begin{Highlighting}[]
\NormalTok{\#import "@preview/headcount:0.1.0": *}

\NormalTok{\#set heading(numbering: "1.1")}
\NormalTok{\#show heading: reset{-}counter(mycounter, levels: 1)}

\NormalTok{\#let mycounter = counter("hello")}

\NormalTok{= First heading}

\NormalTok{\#context mycounter.step()}
\NormalTok{\#context mycounter.display(dependent{-}numbering("1.1"))}

\NormalTok{= Second heading}

\NormalTok{\#context mycounter.step()}
\NormalTok{\#context mycounter.display(dependent{-}numbering("1.1"))}

\NormalTok{\#context mycounter.step()}
\NormalTok{\#context mycounter.display(dependent{-}numbering("1.1"))}
\end{Highlighting}
\end{Shaded}
\end{enumerate}

\textbf{Note:} The \texttt{\ level\ } that you pass to
\texttt{\ dependent-numbering(...)\ } and the \texttt{\ level\ } that
you pass to \texttt{\ reset-counter(...)\ } must be the \emph{same} .

\subsection{Limitations}\label{limitations}

Due to current Typst limitations, there is no way to detect manual
updates or steps of the heading counter, like
\texttt{\ counter(heading).update(...)\ } or
\texttt{\ counter(heading).step(...)\ } . Only occurrences of actual
\texttt{\ heading\ } s can be detected. So make sure that after you call
e.g. \texttt{\ counter(heading).update(...)\ } , you place a heading
directly after it, before you use any counters that depend on the
heading counter.

\subsubsection{How to add}\label{how-to-add}

Copy this into your project and use the import as \texttt{\ headcount\ }

\begin{verbatim}
#import "@preview/headcount:0.1.0"
\end{verbatim}

\includesvg[width=0.16667in,height=0.16667in]{/assets/icons/16-copy.svg}

Check the docs for
\href{https://typst.app/docs/reference/scripting/\#packages}{more
information on how to import packages} .

\subsubsection{About}\label{about}

\begin{description}
\tightlist
\item[Author :]
\href{https://jbirnick.net}{Johann Birnick}
\item[License:]
MIT
\item[Current version:]
0.1.0
\item[Last updated:]
October 16, 2024
\item[First released:]
October 16, 2024
\item[Archive size:]
2.67 kB
\href{https://packages.typst.org/preview/headcount-0.1.0.tar.gz}{\pandocbounded{\includesvg[keepaspectratio]{/assets/icons/16-download.svg}}}
\item[Repository:]
\href{https://github.com/jbirnick/typst-headcount}{GitHub}
\item[Categor ies :]
\begin{itemize}
\tightlist
\item[]
\item
  \pandocbounded{\includesvg[keepaspectratio]{/assets/icons/16-list-unordered.svg}}
  \href{https://typst.app/universe/search/?category=model}{Model}
\item
  \pandocbounded{\includesvg[keepaspectratio]{/assets/icons/16-code.svg}}
  \href{https://typst.app/universe/search/?category=scripting}{Scripting}
\item
  \pandocbounded{\includesvg[keepaspectratio]{/assets/icons/16-hammer.svg}}
  \href{https://typst.app/universe/search/?category=utility}{Utility}
\end{itemize}
\end{description}

\subsubsection{Where to report issues?}\label{where-to-report-issues}

This package is a project of Johann Birnick . Report issues on
\href{https://github.com/jbirnick/typst-headcount}{their repository} .
You can also try to ask for help with this package on the
\href{https://forum.typst.app}{Forum} .

Please report this package to the Typst team using the
\href{https://typst.app/contact}{contact form} if you believe it is a
safety hazard or infringes upon your rights.

\phantomsection\label{versions}
\subsubsection{Version history}\label{version-history}

\begin{longtable}[]{@{}ll@{}}
\toprule\noalign{}
Version & Release Date \\
\midrule\noalign{}
\endhead
\bottomrule\noalign{}
\endlastfoot
0.1.0 & October 16, 2024 \\
\end{longtable}

Typst GmbH did not create this package and cannot guarantee correct
functionality of this package or compatibility with any version of the
Typst compiler or app.


\title{typst.app/universe/package/alchemist}

\phantomsection\label{banner}
\section{alchemist}\label{alchemist}

{ 0.1.2 }

A package to render skeletal formulas using cetz

{ } Featured Package

\phantomsection\label{readme}
Alchemist is a typst package to draw skeletal formulae. It is based on
the \href{https://ctan.org/pkg/chemfig}{chemfig} package. The main goal
of alchemist is not to reproduce one-to-one chemfig. Instead, it aims to
provide an interface to achieve the same results in Typst.

\begin{Shaded}
\begin{Highlighting}[]
\NormalTok{\#skeletize(\{}
\NormalTok{  molecule(name: "A", "A")}
\NormalTok{  single()}
\NormalTok{  molecule("B")}
\NormalTok{  branch(\{}
\NormalTok{    single(angle: 1)}
\NormalTok{    molecule(}
\NormalTok{      "W",}
\NormalTok{      links: (}
\NormalTok{        "A": double(stroke: red),}
\NormalTok{      ),}
\NormalTok{    )}
\NormalTok{    single()}
\NormalTok{    molecule(name: "X", "X")}
\NormalTok{  \})}
\NormalTok{  branch(\{}
\NormalTok{    single(angle: {-}1)}
\NormalTok{    molecule("Y")}
\NormalTok{    single()}
\NormalTok{    molecule(}
\NormalTok{      name: "Z",}
\NormalTok{      "Z",}
\NormalTok{      links: (}
\NormalTok{        "X": single(stroke: black + 3pt),}
\NormalTok{      ),}
\NormalTok{    )}
\NormalTok{  \})}
\NormalTok{  single()}
\NormalTok{  molecule(}
\NormalTok{    "C",}
\NormalTok{    links: (}
\NormalTok{      "X": cram{-}filled{-}left(fill: blue),}
\NormalTok{      "Z": single(),}
\NormalTok{    ),}
\NormalTok{  )}
\NormalTok{\})}
\end{Highlighting}
\end{Shaded}

\pandocbounded{\includegraphics[keepaspectratio]{https://raw.githubusercontent.com/Robotechnic/alchemist/master/images/links1.png}}

Alchemist uses cetz to draw the molecules. This means that you can draw
cetz shapes in the same canvas as the molecules. Like this:

\begin{Shaded}
\begin{Highlighting}[]
\NormalTok{\#skeletize(\{}
\NormalTok{  import cetz.draw: *}
\NormalTok{  double(absolute: 30deg, name: "l1")}
\NormalTok{  single(absolute: {-}30deg, name: "l2")}
\NormalTok{  molecule("X", name: "X")}
\NormalTok{  hobby(}
\NormalTok{    "l1.50\%",}
\NormalTok{    ("l1.start", 0.5, 90deg, "l1.end"),}
\NormalTok{    "l1.start",}
\NormalTok{    stroke: (paint: red, dash: "dashed"),}
\NormalTok{    mark: (end: "\textgreater{}"),}
\NormalTok{  )}
\NormalTok{  hobby(}
\NormalTok{    (to: "X.north", rel: (0, 1pt)),}
\NormalTok{    ("l2.end", 0.4, {-}90deg, "l2.start"),}
\NormalTok{    "l2.50\%",}
\NormalTok{    mark: (end: "\textgreater{}"),}
\NormalTok{  )}
\NormalTok{\})}
\end{Highlighting}
\end{Shaded}

\pandocbounded{\includegraphics[keepaspectratio]{https://raw.githubusercontent.com/Robotechnic/alchemist/master/images/cetz1.png}}

\subsection{Usage}\label{usage}

To start using alchemist, just use the following code:

\begin{Shaded}
\begin{Highlighting}[]
\NormalTok{\#import "@preview/alchemist:0.1.2": *}

\NormalTok{\#skeletize(\{}
\NormalTok{  // Your molecule here}
\NormalTok{\})}
\end{Highlighting}
\end{Shaded}

For more information, check the
\href{https://raw.githubusercontent.com/Robotechnic/alchemist/master/doc/manual.pdf}{manual}
.

\subsection{Changelog}\label{changelog}

\subsubsection{0.1.2}\label{section}

\begin{itemize}
\tightlist
\item
  Added default values for link style properties.
\item
  Updated \texttt{\ cetz\ } to version 0.3.1.
\item
  Added a \texttt{\ tip-lenght\ } argument to dashed cram links.
\end{itemize}

\subsubsection{0.1.1}\label{section-1}

\begin{itemize}
\tightlist
\item
  Exposed the \texttt{\ draw-skeleton\ } function. This allows to draw
  in a cetz canvas directly.
\item
  Fixed multiples bugs that causes overdraws of links.
\end{itemize}

\subsubsection{0.1.0}\label{section-2}

\begin{itemize}
\tightlist
\item
  Initial release
\end{itemize}

\subsubsection{How to add}\label{how-to-add}

Copy this into your project and use the import as \texttt{\ alchemist\ }

\begin{verbatim}
#import "@preview/alchemist:0.1.2"
\end{verbatim}

\includesvg[width=0.16667in,height=0.16667in]{/assets/icons/16-copy.svg}

Check the docs for
\href{https://typst.app/docs/reference/scripting/\#packages}{more
information on how to import packages} .

\subsubsection{About}\label{about}

\begin{description}
\tightlist
\item[Author :]
\href{https://github.com/Robotechnic}{Robotechnic}
\item[License:]
MIT
\item[Current version:]
0.1.2
\item[Last updated:]
November 13, 2024
\item[First released:]
August 14, 2024
\item[Minimum Typst version:]
0.11.0
\item[Archive size:]
11.5 kB
\href{https://packages.typst.org/preview/alchemist-0.1.2.tar.gz}{\pandocbounded{\includesvg[keepaspectratio]{/assets/icons/16-download.svg}}}
\item[Repository:]
\href{https://github.com/Robotechnic/alchemist}{GitHub}
\item[Discipline s :]
\begin{itemize}
\tightlist
\item[]
\item
  \href{https://typst.app/universe/search/?discipline=chemistry}{Chemistry}
\item
  \href{https://typst.app/universe/search/?discipline=biology}{Biology}
\end{itemize}
\item[Categor y :]
\begin{itemize}
\tightlist
\item[]
\item
  \pandocbounded{\includesvg[keepaspectratio]{/assets/icons/16-chart.svg}}
  \href{https://typst.app/universe/search/?category=visualization}{Visualization}
\end{itemize}
\end{description}

\subsubsection{Where to report issues?}\label{where-to-report-issues}

This package is a project of Robotechnic . Report issues on
\href{https://github.com/Robotechnic/alchemist}{their repository} . You
can also try to ask for help with this package on the
\href{https://forum.typst.app}{Forum} .

Please report this package to the Typst team using the
\href{https://typst.app/contact}{contact form} if you believe it is a
safety hazard or infringes upon your rights.

\phantomsection\label{versions}
\subsubsection{Version history}\label{version-history}

\begin{longtable}[]{@{}ll@{}}
\toprule\noalign{}
Version & Release Date \\
\midrule\noalign{}
\endhead
\bottomrule\noalign{}
\endlastfoot
0.1.2 & November 13, 2024 \\
\href{https://typst.app/universe/package/alchemist/0.1.1/}{0.1.1} &
August 19, 2024 \\
\href{https://typst.app/universe/package/alchemist/0.1.0/}{0.1.0} &
August 14, 2024 \\
\end{longtable}

Typst GmbH did not create this package and cannot guarantee correct
functionality of this package or compatibility with any version of the
Typst compiler or app.


\title{typst.app/universe/package/miage-rapide-tp}

\phantomsection\label{banner}
\phantomsection\label{template-thumbnail}
\pandocbounded{\includegraphics[keepaspectratio]{https://packages.typst.org/preview/thumbnails/miage-rapide-tp-0.1.2-small.webp}}

\section{miage-rapide-tp}\label{miage-rapide-tp}

{ 0.1.2 }

Quickly generate a report for MIAGE practical work.

\href{/app?template=miage-rapide-tp&version=0.1.2}{Create project in
app}

\phantomsection\label{readme}
Typst template to generate a practical work report for students of the
MIAGE (Méthodes Informatiques Appliquées Ã~ la Gestion des
Entreprises).

\subsection{ðŸ§`â€?ðŸ'» Usage}\label{uxf0uxffuxe2uxf0uxff-usage}

\begin{itemize}
\item
  Directly from \href{https://typst.app/}{Typst web app} by clicking
  “Start from template� on the dashboard and searching for
  \texttt{\ miage-rapide-tp\ } .
\item
  With CLI:
\end{itemize}

\begin{verbatim}
typst init @preview/miage-rapide-tp:{version}
\end{verbatim}

\subsection{🚀 Features}\label{uxf0uxffux161-features}

\begin{itemize}
\tightlist
\item
  Cover page
\item
  Table of contents (optionnal)
\item
  \texttt{\ question\ } = automatically generates a question number
  (optionnal) with the content of the question
\item
  \texttt{\ code-block\ } = code block with syntax highlighting. You can
  pass a filepath or code directly to display in the block
\item
  \texttt{\ remarque\ } = a remark block with content and color
\end{itemize}

\subsubsection{Cover page}\label{cover-page}

The conf looks like this:

\begin{Shaded}
\begin{Highlighting}[]
\NormalTok{\#let conf(}
\NormalTok{  subtitle: none,}
\NormalTok{  authors: (),}
\NormalTok{  toc: true,}
\NormalTok{  lang: "fr",}
\NormalTok{  font: "Satoshi",}
\NormalTok{  date: none,}
\NormalTok{  years: (2024, 2025),}
\NormalTok{  years{-}label: "Année universitaire",}
\NormalTok{  title,}
\NormalTok{  doc,}
\NormalTok{)}
\end{Highlighting}
\end{Shaded}

\subsubsection{Question}\label{question}

A question can be added like this:

\begin{Shaded}
\begin{Highlighting}[]
\NormalTok{\#question("Une question avec numéro ?")}
\NormalTok{\#question("Une question sans numéro ?", counter: false)}
\end{Highlighting}
\end{Shaded}

The first argument is the question content, and the second (OPTIONAL) is
the counter. If \texttt{\ counter\ } is set to \texttt{\ false\ } , the
question will not be numbered.

\subsubsection{Code-block}\label{code-block}

To use a \texttt{\ code-block\ } , you can do as follows :

\begin{Shaded}
\begin{Highlighting}[]
\NormalTok{\#code{-}block(read("code/main.py"), "py")}
\NormalTok{\#code{-}block(read("code/example.sql"), "sql", title: "Classic SQL")}
\end{Highlighting}
\end{Shaded}

The first argument is the code to display, the second is the language of
the code, and the third is the title of the code block.

\subsubsection{Remarque}\label{remarque}

To use a \texttt{\ remarque\ } , you can do as follows :

\begin{Shaded}
\begin{Highlighting}[]
\NormalTok{\#remarque("Ceci est une remarque")}
\NormalTok{\#remarque("Remarque personnalisée", bg{-}color: olive, text{-}color: white)}
\end{Highlighting}
\end{Shaded}

You can change the bg-color and text-color of the remark block to match
your needs.

\subsection{ðŸ``? License}\label{uxf0uxff-license}

This is MIT licensed.

\begin{quote}
Rapide means fast in French. tp is the abbreviation of “travaux
pratiques� which means practical work. MIAGE is a French degree in
computer science applied to management.
\end{quote}

\href{/app?template=miage-rapide-tp&version=0.1.2}{Create project in
app}

\subsubsection{How to use}\label{how-to-use}

Click the button above to create a new project using this template in
the Typst app.

You can also use the Typst CLI to start a new project on your computer
using this command:

\begin{verbatim}
typst init @preview/miage-rapide-tp:0.1.2
\end{verbatim}

\includesvg[width=0.16667in,height=0.16667in]{/assets/icons/16-copy.svg}

\subsubsection{About}\label{about}

\begin{description}
\tightlist
\item[Author :]
Rémi Saurel
\item[License:]
MIT-0
\item[Current version:]
0.1.2
\item[Last updated:]
September 25, 2024
\item[First released:]
September 11, 2024
\item[Archive size:]
299 kB
\href{https://packages.typst.org/preview/miage-rapide-tp-0.1.2.tar.gz}{\pandocbounded{\includesvg[keepaspectratio]{/assets/icons/16-download.svg}}}
\item[Discipline s :]
\begin{itemize}
\tightlist
\item[]
\item
  \href{https://typst.app/universe/search/?discipline=education}{Education}
\item
  \href{https://typst.app/universe/search/?discipline=engineering}{Engineering}
\item
  \href{https://typst.app/universe/search/?discipline=computer-science}{Computer
  Science}
\end{itemize}
\item[Categor y :]
\begin{itemize}
\tightlist
\item[]
\item
  \pandocbounded{\includesvg[keepaspectratio]{/assets/icons/16-speak.svg}}
  \href{https://typst.app/universe/search/?category=report}{Report}
\end{itemize}
\end{description}

\subsubsection{Where to report issues?}\label{where-to-report-issues}

This template is a project of Rémi Saurel . You can also try to ask for
help with this template on the \href{https://forum.typst.app}{Forum} .

Please report this template to the Typst team using the
\href{https://typst.app/contact}{contact form} if you believe it is a
safety hazard or infringes upon your rights.

\phantomsection\label{versions}
\subsubsection{Version history}\label{version-history}

\begin{longtable}[]{@{}ll@{}}
\toprule\noalign{}
Version & Release Date \\
\midrule\noalign{}
\endhead
\bottomrule\noalign{}
\endlastfoot
0.1.2 & September 25, 2024 \\
\href{https://typst.app/universe/package/miage-rapide-tp/0.1.1/}{0.1.1}
& September 14, 2024 \\
\href{https://typst.app/universe/package/miage-rapide-tp/0.1.0/}{0.1.0}
& September 11, 2024 \\
\end{longtable}

Typst GmbH did not create this template and cannot guarantee correct
functionality of this template or compatibility with any version of the
Typst compiler or app.


