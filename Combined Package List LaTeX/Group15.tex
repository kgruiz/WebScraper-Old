\title{typst.app/universe/package/based}

\phantomsection\label{banner}
\section{based}\label{based}

{ 0.2.0 }

Encoder and decoder for base64, base32, and base16.

\phantomsection\label{readme}
A package for encoding and decoding in base64, base32, and base16.

\subsection{Usage}\label{usage}

The package comes with three submodules: \texttt{\ base64\ } ,
\texttt{\ base32\ } , and \texttt{\ base16\ } . All of them have an
\texttt{\ encode\ } and \texttt{\ decode\ } function. The package also
provides the function aliases

\begin{itemize}
\tightlist
\item
  \texttt{\ encode64\ } / \texttt{\ decode64\ } ,
\item
  \texttt{\ encode32\ } / \texttt{\ decode32\ } , and
\item
  \texttt{\ encode16\ } / \texttt{\ decode16\ } .
\end{itemize}

Both base64 and base32 allow you to choose whether to use padding for
encoding via the \texttt{\ pad\ } parameter, which is enabled by
default. Base64 also allows you to encode with the URL-safe alphabet by
enabling the \texttt{\ url\ } parameter, while base32 allows you to
encode or decode with the “extended hex� alphabet via the
\texttt{\ hex\ } parameter. Both options are disabled by default. The
base16 encoder uses lowercase letters, the decoder is case-insensitive.

You can encode strings, arrays and bytes. The \texttt{\ encode\ }
function will return a string, while the \texttt{\ decode\ } function
will return bytes.

\subsection{Example}\label{example}

\begin{Shaded}
\begin{Highlighting}[]
\NormalTok{\#import "@preview/based:0.2.0": base64, base32, base16}

\NormalTok{\#table(}
\NormalTok{  columns: 3,}
  
\NormalTok{  table.header[*Base64*][*Base32*][*Base16*],}

\NormalTok{  raw(base64.encode("Hello world!")),}
\NormalTok{  raw(base32.encode("Hello world!")),}
\NormalTok{  raw(base16.encode("Hello world!")),}

\NormalTok{  str(base64.decode("SGVsbG8gd29ybGQh")),}
\NormalTok{  str(base32.decode("JBSWY3DPEB3W64TMMQQQ====")),}
\NormalTok{  str(base16.decode("48656C6C6F20776F726C6421"))}
\NormalTok{)}
\end{Highlighting}
\end{Shaded}

\pandocbounded{\includesvg[keepaspectratio]{https://github.com/typst/packages/raw/main/packages/preview/based/0.2.0/assets/example.svg}}

\subsubsection{How to add}\label{how-to-add}

Copy this into your project and use the import as \texttt{\ based\ }

\begin{verbatim}
#import "@preview/based:0.2.0"
\end{verbatim}

\includesvg[width=0.16667in,height=0.16667in]{/assets/icons/16-copy.svg}

Check the docs for
\href{https://typst.app/docs/reference/scripting/\#packages}{more
information on how to import packages} .

\subsubsection{About}\label{about}

\begin{description}
\tightlist
\item[Author :]
Eric Biedert
\item[License:]
MIT
\item[Current version:]
0.2.0
\item[Last updated:]
November 6, 2024
\item[First released:]
July 5, 2024
\item[Archive size:]
21.1 kB
\href{https://packages.typst.org/preview/based-0.2.0.tar.gz}{\pandocbounded{\includesvg[keepaspectratio]{/assets/icons/16-download.svg}}}
\item[Repository:]
\href{https://github.com/EpicEricEE/typst-based}{GitHub}
\item[Categor y :]
\begin{itemize}
\tightlist
\item[]
\item
  \pandocbounded{\includesvg[keepaspectratio]{/assets/icons/16-code.svg}}
  \href{https://typst.app/universe/search/?category=scripting}{Scripting}
\end{itemize}
\end{description}

\subsubsection{Where to report issues?}\label{where-to-report-issues}

This package is a project of Eric Biedert . Report issues on
\href{https://github.com/EpicEricEE/typst-based}{their repository} . You
can also try to ask for help with this package on the
\href{https://forum.typst.app}{Forum} .

Please report this package to the Typst team using the
\href{https://typst.app/contact}{contact form} if you believe it is a
safety hazard or infringes upon your rights.

\phantomsection\label{versions}
\subsubsection{Version history}\label{version-history}

\begin{longtable}[]{@{}ll@{}}
\toprule\noalign{}
Version & Release Date \\
\midrule\noalign{}
\endhead
\bottomrule\noalign{}
\endlastfoot
0.2.0 & November 6, 2024 \\
\href{https://typst.app/universe/package/based/0.1.0/}{0.1.0} & July 5,
2024 \\
\end{longtable}

Typst GmbH did not create this package and cannot guarantee correct
functionality of this package or compatibility with any version of the
Typst compiler or app.


\title{typst.app/universe/package/cetz-venn}

\phantomsection\label{banner}
\section{cetz-venn}\label{cetz-venn}

{ 0.1.2 }

CeTZ library for drawing venn diagrams for two or three sets.

\phantomsection\label{readme}
A \href{https://github.com/cetz-package/cetz}{CeTZ} library for drawing
simple two- or three-set Venn diagrams.

\subsection{Examples}\label{examples}

\begin{longtable}[]{@{}ll@{}}
\toprule\noalign{}
\endhead
\bottomrule\noalign{}
\endlastfoot
\href{https://github.com/typst/packages/raw/main/packages/preview/cetz-venn/0.1.2/gallery/venn2.typ}{\includegraphics[width=2.60417in,height=\textheight,keepaspectratio]{https://github.com/typst/packages/raw/main/packages/preview/cetz-venn/0.1.2/gallery/venn2.png}}
&
\href{https://github.com/typst/packages/raw/main/packages/preview/cetz-venn/0.1.2/gallery/venn3.typ}{\includegraphics[width=2.60417in,height=\textheight,keepaspectratio]{https://github.com/typst/packages/raw/main/packages/preview/cetz-venn/0.1.2/gallery/venn3.png}} \\
Two set Venn diagram & Three set Venn diagram \\
\end{longtable}

\emph{Click on the example image to jump to the code.}

\subsection{Usage}\label{usage}

This package requires CeTZ version \textgreater= 0.3.1!

For information, see the
\href{https://github.com/cetz-package/cetz-venn/blob/stable/manual.pdf?raw=true}{manual
(stable)} .

To use this package, simply add the following code to your document:

\begin{verbatim}
#import "@preview/cetz:0.3.1"
#import "@preview/cetz-venn:0.1.1"

#cetz.canvas({
  cetz-venn.venn2()
})
\end{verbatim}

\subsubsection{How to add}\label{how-to-add}

Copy this into your project and use the import as \texttt{\ cetz-venn\ }

\begin{verbatim}
#import "@preview/cetz-venn:0.1.2"
\end{verbatim}

\includesvg[width=0.16667in,height=0.16667in]{/assets/icons/16-copy.svg}

Check the docs for
\href{https://typst.app/docs/reference/scripting/\#packages}{more
information on how to import packages} .

\subsubsection{About}\label{about}

\begin{description}
\tightlist
\item[Author :]
\href{https://github.com/johannes-wolf}{Johannes Wolf}
\item[License:]
Apache-2.0
\item[Current version:]
0.1.2
\item[Last updated:]
October 28, 2024
\item[First released:]
July 1, 2024
\item[Archive size:]
6.34 kB
\href{https://packages.typst.org/preview/cetz-venn-0.1.2.tar.gz}{\pandocbounded{\includesvg[keepaspectratio]{/assets/icons/16-download.svg}}}
\item[Repository:]
\href{https://github.com/johannes-wolf/cetz-venn}{GitHub}
\end{description}

\subsubsection{Where to report issues?}\label{where-to-report-issues}

This package is a project of Johannes Wolf . Report issues on
\href{https://github.com/johannes-wolf/cetz-venn}{their repository} .
You can also try to ask for help with this package on the
\href{https://forum.typst.app}{Forum} .

Please report this package to the Typst team using the
\href{https://typst.app/contact}{contact form} if you believe it is a
safety hazard or infringes upon your rights.

\phantomsection\label{versions}
\subsubsection{Version history}\label{version-history}

\begin{longtable}[]{@{}ll@{}}
\toprule\noalign{}
Version & Release Date \\
\midrule\noalign{}
\endhead
\bottomrule\noalign{}
\endlastfoot
0.1.2 & October 28, 2024 \\
\href{https://typst.app/universe/package/cetz-venn/0.1.1/}{0.1.1} & July
19, 2024 \\
\href{https://typst.app/universe/package/cetz-venn/0.1.0/}{0.1.0} & July
1, 2024 \\
\end{longtable}

Typst GmbH did not create this package and cannot guarantee correct
functionality of this package or compatibility with any version of the
Typst compiler or app.


\title{typst.app/universe/package/minideck}

\phantomsection\label{banner}
\section{minideck}\label{minideck}

{ 0.2.1 }

Simple dynamic slides.

\phantomsection\label{readme}
A small package for dynamic slides in typst.

minideck provides basic functionality for dynamic slides (
\texttt{\ pause\ } , \texttt{\ uncover\ } , …), integration with
\href{https://typst.app/universe/package/fletcher}{fletcher} and
\href{https://typst.app/universe/package/cetz/}{CetZ} , and some minimal
infrastructure for theming.

\subsection{Usage}\label{usage}

Call \texttt{\ minideck.config\ } to get the functions you want to use:

\begin{Shaded}
\begin{Highlighting}[]
\NormalTok{\#import "@preview/minideck:0.2.1"}

\NormalTok{\#let (template, slide, title{-}slide, pause, uncover, only) = minideck.config()}
\NormalTok{\#show: template}

\NormalTok{\#title{-}slide[}
\NormalTok{  = Slides with \textasciigrave{}minideck\textasciigrave{}}
\NormalTok{  == Some examples}
\NormalTok{  John Doe}

\NormalTok{  \#datetime.today().display()}
\NormalTok{]}

\NormalTok{\#slide[}
\NormalTok{  = Some title}

\NormalTok{  Some content}

\NormalTok{  \#show: pause}

\NormalTok{  More content}

\NormalTok{  1. One}
\NormalTok{  2. Two}
\NormalTok{  \#show: pause}
\NormalTok{  3. Three}
\NormalTok{]}
\end{Highlighting}
\end{Shaded}

This will show three subslides with progressively more content. (Note
that the default theme uses the font Libertinus Sans from the
\href{https://github.com/alerque/libertinus}{Libertinus} family, so you
may want to install it.)

Instead of \texttt{\ \#show:\ pause\ } , you can use
\texttt{\ \#uncover(2,3){[}...{]}\ } to make content visible only on
subslides 2 and 3, or \texttt{\ \#uncover(from:\ 2){[}...{]}\ } to have
it visible on subslide 2 and following.

The \texttt{\ only\ } function is similar to \texttt{\ uncover\ } , but
instead of hiding the content (without affecting the layout), it removes
it.

\begin{Shaded}
\begin{Highlighting}[]
\NormalTok{\#slide[}
\NormalTok{  = \textasciigrave{}uncover\textasciigrave{} and \textasciigrave{}only\textasciigrave{}}
  
\NormalTok{  \#uncover(1, from:3)[}
\NormalTok{    Content visible on subslides 1 and 3+}
\NormalTok{    (space reserved on 2).}
\NormalTok{  ]}

\NormalTok{  \#only(2,3)[}
\NormalTok{    Content included on subslides 2 and 3}
\NormalTok{    (no space reserved on 1).}
\NormalTok{  ]}

\NormalTok{  Content always visible.}
\NormalTok{]}
\end{Highlighting}
\end{Shaded}

Contrary to \texttt{\ pause\ } , the \texttt{\ uncover\ } and
\texttt{\ only\ } functions also work in math mode:

\begin{Shaded}
\begin{Highlighting}[]
\NormalTok{\#slide[}
\NormalTok{  = Dynamic equation}

\NormalTok{  $}
\NormalTok{    f(x) \&= x\^{}2 + 2x + 1  \textbackslash{}}
\NormalTok{         \#uncover(2, $\&= (x + 1)\^{}2$)}
\NormalTok{  $}
\NormalTok{]}
\end{Highlighting}
\end{Shaded}

When mixing \texttt{\ pause\ } with \texttt{\ uncover\ } /
\texttt{\ only\ } , the sequence of pauses should be taken as reference
for the meaning of subslide indices. For example content after the first
pause always appears on the second subslide, even if it’s preceded by
a call to \texttt{\ \#uncover(from:\ 3){[}...{]}\ } .

The package also works well with
\href{https://typst.app/universe/package/pinit}{pinit} :

\begin{Shaded}
\begin{Highlighting}[]
\NormalTok{\#import "@preview/pinit:0.1.4": *}

\NormalTok{\#slide[}
\NormalTok{  = Works well with \textasciigrave{}pinit\textasciigrave{}}

\NormalTok{  Pythagorean theorem:}

\NormalTok{  $ \#pin(1)a\^{}2\#pin(2) + \#pin(3)b\^{}2\#pin(4) = \#pin(5)c\^{}2\#pin(6) $}

\NormalTok{  \#show: pause}

\NormalTok{  $a\^{}2$ and $b\^{}2$ : squares of triangle legs}

\NormalTok{  \#only(2, \{}
\NormalTok{    pinit{-}highlight(1,2)}
\NormalTok{    pinit{-}highlight(3,4)}
\NormalTok{  \})}

\NormalTok{  \#show: pause}

\NormalTok{  $c\^{}2$ : square of hypotenuse}

\NormalTok{  \#pinit{-}highlight(5,6, fill: green.transparentize(80\%))}
\NormalTok{  \#pinit{-}point{-}from(6)[larger than $a\^{}2$ and $b\^{}2$]}
\NormalTok{]}
\end{Highlighting}
\end{Shaded}

\subsubsection{Handout mode}\label{handout-mode}

minideck can make a handout version of the document, in which dynamic
behavior is disabled: the content of all subslides is shown together in
a single slide.

To compile a handout version, pass \texttt{\ -\/-input\ handout=true\ }
in the command line:

\begin{Shaded}
\begin{Highlighting}[]
\ExtensionTok{typst}\NormalTok{ compile }\AttributeTok{{-}{-}input}\NormalTok{ handout=true myfile.typ}
\end{Highlighting}
\end{Shaded}

It is also possible to enable handout mode from within the document, as
shown in the next section.

\subsubsection{Configuration}\label{configuration}

The behavior of the minideck functions depends on the settings passed to
\texttt{\ minideck.config\ } . For example, handout mode can also be
enabled like this:

\begin{Shaded}
\begin{Highlighting}[]
\NormalTok{\#import "@preview/minideck:0.2.1"}

\NormalTok{\#let (template, slide, pause) = minideck.config(handout: true)}
\NormalTok{\#show: template}

\NormalTok{\#slide[}
\NormalTok{  = Slide title}
  
\NormalTok{  Some text}

\NormalTok{  \#show: pause}

\NormalTok{  More text}
\NormalTok{]}
\end{Highlighting}
\end{Shaded}

(The default value of \texttt{\ handout\ } is \texttt{\ auto\ } , in
which case minideck checks for a command line setting as in the previous
section.)

\texttt{\ minideck.config\ } accepts the following named arguments:

\begin{itemize}
\tightlist
\item
  \texttt{\ paper\ } : a string for one of the paper size names
  recognized by
  \href{https://typst.app/docs/reference/layout/page/\#parameters-paper}{\texttt{\ page.paper\ }}
  , or one of the shorthands \texttt{\ "16:9"\ } or \texttt{\ "4:3"\ } .
  Default: \texttt{\ "4:3"\ } .
\item
  \texttt{\ landscape\ } : use the paper size in landscape orientation.
  Default: \texttt{\ true\ } .
\item
  \texttt{\ width\ } : page width as an absolute length. Takes
  precedence over \texttt{\ paper\ } and \texttt{\ landscape\ } .
\item
  \texttt{\ height\ } : page height as an absolute length. Takes
  precedence over \texttt{\ paper\ } and \texttt{\ landscape\ } .
\item
  \texttt{\ handout\ } : whether to make a document for handout, with
  content of all subslides shown together in a single slide.
\item
  \texttt{\ theme\ } : the theme (see below).
\item
  \texttt{\ cetz\ } : the CeTZ module (see below).
\item
  \texttt{\ fletcher\ } : the fletcher module (see below).
\end{itemize}

For example to make slides with 16:9 aspect ratio, use
\texttt{\ minideck.config(paper:\ "16:9")\ } .

\subsubsection{Themes}\label{themes}

Use \texttt{\ minideck.config(theme:\ ...)\ } to select a theme.
Currently there is only one built-in:
\texttt{\ minideck.themes.simple\ } . However you can also pass a theme
implemented by yourself or from a third-party package. See the
\href{https://github.com/typst/packages/raw/main/packages/preview/minideck/0.2.1/themes/README.md}{theme
documentation} for how that works.

Themes are functions and can be configured using the standard
\href{https://typst.app/docs/reference/foundations/function/\#definitions-with}{\texttt{\ with\ }
method} :

\begin{itemize}
\tightlist
\item
  The \texttt{\ simple\ } theme has a \texttt{\ variant\ } setting with
  values “light� (default) and “dark�.
\end{itemize}

Here’s an example:

\begin{Shaded}
\begin{Highlighting}[]
\NormalTok{\#import "@preview/minideck:0.2.1"}

\NormalTok{\#let (template, slide) = minideck.config(}
\NormalTok{  theme: minideck.themes.simple.with(variant: "dark"),}
\NormalTok{)}
\NormalTok{\#show: template}

\NormalTok{\#slide[}
\NormalTok{  = Slide with dark theme}
  
\NormalTok{  Some text}
\NormalTok{]}
\end{Highlighting}
\end{Shaded}

Note that you can override part of a theme with show and set rules:

\begin{Shaded}
\begin{Highlighting}[]
\NormalTok{\#import "@preview/minideck:0.2.1"}

\NormalTok{\#let (template, slide) = minideck.config(}
\NormalTok{  theme: minideck.themes.simple.with(variant: "dark"),}
\NormalTok{)}
\NormalTok{\#show: template}

\NormalTok{\#set page(footer: none) // get rid of the page number}
\NormalTok{\#show heading: it =\textgreater{} text(style: "italic", it)}
\NormalTok{\#set text(red)}

\NormalTok{\#slide[}
\NormalTok{  = Slide with dark theme and red text}
  
\NormalTok{  Some text}
\NormalTok{]}
\end{Highlighting}
\end{Shaded}

\subsubsection{Integration with CeTZ}\label{integration-with-cetz}

You can use \texttt{\ uncover\ } and \texttt{\ only\ } (but not
\texttt{\ pause\ } ) in CeTZ figures, with the following extra steps:

\begin{itemize}
\item
  Get CeTZ-specific functions \texttt{\ cetz-uncover\ } and
  \texttt{\ cetz-only\ } by passing the CeTZ module to
  \texttt{\ minideck.config\ } (see example below).

  This ensures that minideck uses CeTZ functions from the correct
  version of CeTZ.
\item
  Add a \texttt{\ context\ } keyword outside the \texttt{\ canvas\ }
  call.

  This is required to access the minideck subslide state from within the
  canvas without making the content opaque (CeTZ needs to inspect the
  canvas content to make the drawing).
\end{itemize}

Example:

\begin{Shaded}
\begin{Highlighting}[]
\NormalTok{\#import "@preview/cetz:0.2.2" as cetz: *}
\NormalTok{\#import "@preview/minideck:0.2.1"}

\NormalTok{\#let (template, slide, only, cetz{-}uncover, cetz{-}only) = minideck.config(cetz: cetz)}
\NormalTok{\#show: template}

\NormalTok{\#slide[}
\NormalTok{  = In a CeTZ figure}

\NormalTok{  Above canvas}
\NormalTok{  \#context canvas(\{}
\NormalTok{    import draw: *}
\NormalTok{    cetz{-}only(3, rect((0,{-}2), (14,4), stroke: 3pt))}
\NormalTok{    cetz{-}uncover(from: 2, rect((0,{-}2), (16,2), stroke: blue+3pt))}
\NormalTok{    content((8,0), box(stroke: red+3pt, inset: 1em)[}
\NormalTok{      A typst box \#only(2)[on 2nd subslide]}
\NormalTok{    ])}
\NormalTok{  \})}
\NormalTok{  Below canvas}
\NormalTok{]}
\end{Highlighting}
\end{Shaded}

\subsubsection{Integration with
fletcher}\label{integration-with-fletcher}

The same steps are required as for CeTZ integration (passing the
fletcher module to get fletcher-specific functions), plus an additional
step:

\begin{itemize}
\item
  Give explicitly the number of subslides to the \texttt{\ slide\ }
  function.

  This is required because I could not find a reliable way to update a
  typst state from within a fletcher diagram, so I cannot rely on the
  state to keep track of the number of subslides.
\end{itemize}

Example:

\begin{Shaded}
\begin{Highlighting}[]
\NormalTok{\#import "@preview/fletcher:0.5.0" as fletcher: diagram, node, edge}
\NormalTok{\#import "@preview/minideck:0.2.1"}
\NormalTok{\#let (template, slide, fletcher{-}uncover) = minideck.config(fletcher: fletcher)}
\NormalTok{\#show: template}

\NormalTok{\#slide(steps: 2)[}
\NormalTok{  = In a fletcher diagram}

\NormalTok{  \#set align(center)}

\NormalTok{  Above diagram}

\NormalTok{  \#context diagram(}
\NormalTok{    node{-}stroke: 1pt,}
\NormalTok{    node((0,0), [Start], corner{-}radius: 2pt, extrude: (0, 3)),}
\NormalTok{    edge("{-}|\textgreater{}"),}
\NormalTok{    node((1,0), align(center)[A]),}
\NormalTok{    fletcher{-}uncover(from: 2, edge("d,r,u,l", "{-}|\textgreater{}", [x], label{-}pos: 0.1))}
\NormalTok{  )}
  
\NormalTok{  Below diagram}
\NormalTok{]}
\end{Highlighting}
\end{Shaded}

\subsection{Comparison with other slides
packages}\label{comparison-with-other-slides-packages}

Performance: minideck is currently faster than
\href{https://typst.app/universe/package/polylux/}{Polylux} when using
\texttt{\ pause\ } , especially for incremental compilation, but a bit
slower than \href{https://typst.app/universe/package/touying}{Touying} ,
according to my tests.

Features: Polylux and Touying have more themes and more features, for
example support for \href{https://pdfpc.github.io/}{pdfpc} which
provides speaker notes and more. Minideck allows using
\texttt{\ uncover\ } and \texttt{\ only\ } in CeTZ figures and fletcher
diagrams, which Polylux currently doesn’t support.

Syntax: package configuration is simpler in minideck than Touying but a
bit more involved than in Polylux. The minideck \texttt{\ pause\ } is
more cumbersome to use: one must write \texttt{\ \#show:\ pause\ }
instead of \texttt{\ \#pause\ } . On the other hand minideck’s
\texttt{\ uncover\ } and \texttt{\ only\ } can be used directly in
equations without requiring a special math environment as in Touying (I
think).

Other: minideck sometimes has better error messages than Touying due to
implementation differences: the minideck stack trace points back to the
user’s code while Touying errors sometimes point only to an output
page number.

\subsection{Limitations}\label{limitations}

\begin{itemize}
\tightlist
\item
  \texttt{\ pause\ } , \texttt{\ uncover\ } and \texttt{\ only\ } work
  in enumerations but they require explicit enum indices (
  \texttt{\ 1.\ ...\ } rather than \texttt{\ +\ ...\ } ) as they cause a
  reset of the list index.
\item
  Usage in a CeTZ canvas or fletcher diagram requires a
  \texttt{\ context\ } keyword in front of the \texttt{\ canvas\ } /
  \texttt{\ diagram\ } call (see above).
\item
  fletcher diagrams also require to specify the number of subslides
  explicitly (see above).
\item
  \texttt{\ pause\ } doesn’t work in CeTZ figures, fletcher diagrams
  and math mode.
\item
  \texttt{\ pause\ } requires writing \texttt{\ \#show:\ pause\ } and
  its effect is lost after the \texttt{\ \#show\ } call goes out of
  scope. For example this means that one can use \texttt{\ pause\ }
  inside of a grid, but further \texttt{\ pause\ } calls after the grid
  (in the same slide) won’t work as intended.
\end{itemize}

\subsection{Internals}\label{internals}

The package uses states with the following keys:

\texttt{\ \_\_minideck-subslide-count\ } : an array of two integers for
counting pauses and keeping track of the subslide number automatically.
The first value is the number of subslides so far referenced in current
slide. The second value is the number of pauses seen so far in the
current slide. Both values are kept in one state so that an update
function can update the number of subslides based on the number of
pauses, without requiring a context. This avoids problems with layout
convergence.

\texttt{\ \_\_minideck-subslide-step\ } : the current subslide being
generated while processing a slide.

\subsubsection{How to add}\label{how-to-add}

Copy this into your project and use the import as \texttt{\ minideck\ }

\begin{verbatim}
#import "@preview/minideck:0.2.1"
\end{verbatim}

\includesvg[width=0.16667in,height=0.16667in]{/assets/icons/16-copy.svg}

Check the docs for
\href{https://typst.app/docs/reference/scripting/\#packages}{more
information on how to import packages} .

\subsubsection{About}\label{about}

\begin{description}
\tightlist
\item[Author :]
\href{https://github.com/knuesel}{Jeremie Knuesel}
\item[License:]
MIT
\item[Current version:]
0.2.1
\item[Last updated:]
July 1, 2024
\item[First released:]
July 1, 2024
\item[Archive size:]
10.3 kB
\href{https://packages.typst.org/preview/minideck-0.2.1.tar.gz}{\pandocbounded{\includesvg[keepaspectratio]{/assets/icons/16-download.svg}}}
\item[Repository:]
\href{https://github.com/knuesel/typst-minideck}{GitHub}
\item[Categor y :]
\begin{itemize}
\tightlist
\item[]
\item
  \pandocbounded{\includesvg[keepaspectratio]{/assets/icons/16-presentation.svg}}
  \href{https://typst.app/universe/search/?category=presentation}{Presentation}
\end{itemize}
\end{description}

\subsubsection{Where to report issues?}\label{where-to-report-issues}

This package is a project of Jeremie Knuesel . Report issues on
\href{https://github.com/knuesel/typst-minideck}{their repository} . You
can also try to ask for help with this package on the
\href{https://forum.typst.app}{Forum} .

Please report this package to the Typst team using the
\href{https://typst.app/contact}{contact form} if you believe it is a
safety hazard or infringes upon your rights.

\phantomsection\label{versions}
\subsubsection{Version history}\label{version-history}

\begin{longtable}[]{@{}ll@{}}
\toprule\noalign{}
Version & Release Date \\
\midrule\noalign{}
\endhead
\bottomrule\noalign{}
\endlastfoot
0.2.1 & July 1, 2024 \\
\end{longtable}

Typst GmbH did not create this package and cannot guarantee correct
functionality of this package or compatibility with any version of the
Typst compiler or app.


\title{typst.app/universe/package/ccicons}

\phantomsection\label{banner}
\section{ccicons}\label{ccicons}

{ 1.0.0 }

A port of the ccicon LaTeX package for Typst.

\phantomsection\label{readme}
Creative Commons icons for your Typst documents

\begin{center}\rule{0.5\linewidth}{0.5pt}\end{center}

\begin{quote}
{[}!NOTE{]} \texttt{\ ccicons\ } is an adaption of the
\href{https://ctan.org/pkg/ccicons}{ccicons package} for LaTeX by
\href{https://github.com/ummels}{Michael Ummels} .
\end{quote}

\subsection{Getting Started}\label{getting-started}

Import the package into your document:

\begin{Shaded}
\begin{Highlighting}[]
\NormalTok{\#import "@preview/ccicons:1.0.0": *}
\end{Highlighting}
\end{Shaded}

Start using license icons:

\begin{Shaded}
\begin{Highlighting}[]
\NormalTok{\#cc{-}by{-}nc{-}sa}
\end{Highlighting}
\end{Shaded}

See the
\href{https://github.com/typst/packages/raw/main/packages/preview/ccicons/1.0.0/docs/ccicons-manual.pdf}{the
manual} for more details and an overview all available Creative Commons
icons.

Please note that all icons that can be typeset using this package are
trademarks of Creative Commons and are subject to the Creative Commons
trademark policy (see \url{http://creativecommons.org/policies} ).

The symbols in this font have been obtained from
\url{https://creativecommons.org/mission/downloads/} and released by
Creative Commons under a Creative Commons Attribution 4.0 International
license: \url{https://creativecommons.org/licenses/by/4.0/}

\subsubsection{How to add}\label{how-to-add}

Copy this into your project and use the import as \texttt{\ ccicons\ }

\begin{verbatim}
#import "@preview/ccicons:1.0.0"
\end{verbatim}

\includesvg[width=0.16667in,height=0.16667in]{/assets/icons/16-copy.svg}

Check the docs for
\href{https://typst.app/docs/reference/scripting/\#packages}{more
information on how to import packages} .

\subsubsection{About}\label{about}

\begin{description}
\tightlist
\item[Author :]
J. Neugebauer
\item[License:]
MIT
\item[Current version:]
1.0.0
\item[Last updated:]
June 17, 2024
\item[First released:]
June 17, 2024
\item[Archive size:]
197 kB
\href{https://packages.typst.org/preview/ccicons-1.0.0.tar.gz}{\pandocbounded{\includesvg[keepaspectratio]{/assets/icons/16-download.svg}}}
\item[Repository:]
\href{https://github.com/jneug/typst-ccicons}{GitHub}
\item[Categor y :]
\begin{itemize}
\tightlist
\item[]
\item
  \pandocbounded{\includesvg[keepaspectratio]{/assets/icons/16-package.svg}}
  \href{https://typst.app/universe/search/?category=components}{Components}
\end{itemize}
\end{description}

\subsubsection{Where to report issues?}\label{where-to-report-issues}

This package is a project of J. Neugebauer . Report issues on
\href{https://github.com/jneug/typst-ccicons}{their repository} . You
can also try to ask for help with this package on the
\href{https://forum.typst.app}{Forum} .

Please report this package to the Typst team using the
\href{https://typst.app/contact}{contact form} if you believe it is a
safety hazard or infringes upon your rights.

\phantomsection\label{versions}
\subsubsection{Version history}\label{version-history}

\begin{longtable}[]{@{}ll@{}}
\toprule\noalign{}
Version & Release Date \\
\midrule\noalign{}
\endhead
\bottomrule\noalign{}
\endlastfoot
1.0.0 & June 17, 2024 \\
\end{longtable}

Typst GmbH did not create this package and cannot guarantee correct
functionality of this package or compatibility with any version of the
Typst compiler or app.


\title{typst.app/universe/package/unequivocal-ams}

\phantomsection\label{banner}
\phantomsection\label{template-thumbnail}
\pandocbounded{\includegraphics[keepaspectratio]{https://packages.typst.org/preview/thumbnails/unequivocal-ams-0.1.2-small.webp}}

\section{unequivocal-ams}\label{unequivocal-ams}

{ 0.1.2 }

An AMS-style paper template to publish at conferences and journals for
mathematicians

{ } Featured Template

\href{/app?template=unequivocal-ams&version=0.1.2}{Create project in
app}

\phantomsection\label{readme}
A single-column paper for the American Mathematical Society. The
template comes with functions for theorems and proofs. It also is a nice
starting point for a classy tech report or thesis.

\subsection{Usage}\label{usage}

You can use this template in the Typst web app by clicking “Start from
template� on the dashboard and searching for
\texttt{\ unequivocal-ams\ } .

Alternatively, you can use the CLI to kick this project off using the
command

\begin{verbatim}
typst init @preview/unequivocal-ams
\end{verbatim}

Typst will create a new directory with all the files needed to get you
started.

\subsection{Configuration}\label{configuration}

This template exports the \texttt{\ ams-article\ } function with the
following named arguments:

\begin{itemize}
\tightlist
\item
  \texttt{\ title\ } : The paper’s title as content.
\item
  \texttt{\ authors\ } : An array of author dictionaries. Each of the
  author dictionaries must have a \texttt{\ name\ } key and can have the
  keys \texttt{\ department\ } , \texttt{\ organization\ } ,
  \texttt{\ location\ } , and \texttt{\ email\ } . All keys accept
  content.
\item
  \texttt{\ abstract\ } : The content of a brief summary of the paper or
  \texttt{\ none\ } . Appears at the top of the first column in
  boldface.
\item
  \texttt{\ paper-size\ } : Defaults to \texttt{\ us-letter\ } . Specify
  a
  \href{https://typst.app/docs/reference/layout/page/\#parameters-paper}{paper
  size string} to change the page format.
\item
  \texttt{\ bibliography\ } : The result of a call to the
  \texttt{\ bibliography\ } function or \texttt{\ none\ } . Specifying
  this will configure numeric, Springer MathPhys-style citations.
\end{itemize}

The function also accepts a single, positional argument for the body of
the paper.

The template will initialize your package with a sample call to the
\texttt{\ ams-article\ } function in a show rule. If you, however, want
to change an existing project to use this template, you can add a show
rule like this at the top of your file:

\begin{Shaded}
\begin{Highlighting}[]
\NormalTok{\#import "@preview/unequivocal{-}ams:0.1.2": ams{-}article, theorem, proof}

\NormalTok{\#show: ams{-}article.with(}
\NormalTok{  title: [Mathematical Theorems],}
\NormalTok{  authors: (}
\NormalTok{    (}
\NormalTok{      name: "Ralph Howard",}
\NormalTok{      department: [Department of Mathematics],}
\NormalTok{      organization: [University of South Carolina],}
\NormalTok{      location: [Columbia, SC 29208],}
\NormalTok{      email: "howard@math.sc.edu",}
\NormalTok{      url: "www.math.sc.edu/\textasciitilde{}howard"}
\NormalTok{    ),}
\NormalTok{  ),}
\NormalTok{  abstract: lorem(100),}
\NormalTok{  bibliography: bibliography("refs.bib"),}
\NormalTok{)}

\NormalTok{// Your content goes below.}
\end{Highlighting}
\end{Shaded}

\href{/app?template=unequivocal-ams&version=0.1.2}{Create project in
app}

\subsubsection{How to use}\label{how-to-use}

Click the button above to create a new project using this template in
the Typst app.

You can also use the Typst CLI to start a new project on your computer
using this command:

\begin{verbatim}
typst init @preview/unequivocal-ams:0.1.2
\end{verbatim}

\includesvg[width=0.16667in,height=0.16667in]{/assets/icons/16-copy.svg}

\subsubsection{About}\label{about}

\begin{description}
\tightlist
\item[Author :]
\href{https://typst.app}{Typst GmbH}
\item[License:]
MIT-0
\item[Current version:]
0.1.2
\item[Last updated:]
October 29, 2024
\item[First released:]
March 6, 2024
\item[Minimum Typst version:]
0.12.0
\item[Archive size:]
6.30 kB
\href{https://packages.typst.org/preview/unequivocal-ams-0.1.2.tar.gz}{\pandocbounded{\includesvg[keepaspectratio]{/assets/icons/16-download.svg}}}
\item[Repository:]
\href{https://github.com/typst/templates}{GitHub}
\item[Discipline :]
\begin{itemize}
\tightlist
\item[]
\item
  \href{https://typst.app/universe/search/?discipline=mathematics}{Mathematics}
\end{itemize}
\item[Categor y :]
\begin{itemize}
\tightlist
\item[]
\item
  \pandocbounded{\includesvg[keepaspectratio]{/assets/icons/16-atom.svg}}
  \href{https://typst.app/universe/search/?category=paper}{Paper}
\end{itemize}
\end{description}

\subsubsection{Where to report issues?}\label{where-to-report-issues}

This template is a project of Typst GmbH . Report issues on
\href{https://github.com/typst/templates}{their repository} . You can
also try to ask for help with this template on the
\href{https://forum.typst.app}{Forum} .

\phantomsection\label{versions}
\subsubsection{Version history}\label{version-history}

\begin{longtable}[]{@{}ll@{}}
\toprule\noalign{}
Version & Release Date \\
\midrule\noalign{}
\endhead
\bottomrule\noalign{}
\endlastfoot
0.1.2 & October 29, 2024 \\
\href{https://typst.app/universe/package/unequivocal-ams/0.1.1/}{0.1.1}
& August 8, 2024 \\
\href{https://typst.app/universe/package/unequivocal-ams/0.1.0/}{0.1.0}
& March 6, 2024 \\
\end{longtable}


\title{typst.app/universe/package/rfc-vibe}

\phantomsection\label{banner}
\section{rfc-vibe}\label{rfc-vibe}

{ 0.1.0 }

Bring RFC language into everyday docs

\phantomsection\label{readme}
Bring that RFC lingo to your everyday documents.

A \href{https://typst.app/}{Typst} package that makes it easy to use the
exact keywords, boilerplate, and definitions provided by BCP 14,
RFC2119, and RFC8174. See the end of this README for an example of the
output.

In the future, this package may include other RFC-related patterns which
are applicable to a wide variety of everyday documents.

\subsection{Usage}\label{usage}

Import the package in your Typst document:

\begin{Shaded}
\begin{Highlighting}[]
\NormalTok{\#import "@preview/rfc{-}vibe:0.1.0": *}
\end{Highlighting}
\end{Shaded}

\subsubsection{Keywords}\label{keywords}

Use the keywords according to these examples:

\begin{Shaded}
\begin{Highlighting}[]
\NormalTok{\#must              // renders as: MUST}
\NormalTok{\#must{-}not          // renders as: MUST NOT}
\NormalTok{\#required          // renders as: REQUIRED}
\NormalTok{\#shall             // renders as: SHALL}
\NormalTok{\#shall{-}not         // renders as: SHALL NOT}
\NormalTok{\#should            // renders as: SHOULD}
\NormalTok{\#should{-}not        // renders as: SHOULD NOT}
\NormalTok{\#recommended       // renders as: RECOMMENDED}
\NormalTok{\#not{-}recommended   // renders as: NOT RECOMMENDED}
\NormalTok{\#may               // renders as: MAY}
\NormalTok{\#optional          // renders as: OPTIONAL}
\end{Highlighting}
\end{Shaded}

For the rare situation when you want the keywords included in quotation
marks, use the \texttt{\ -quoted\ } versions:

\begin{Shaded}
\begin{Highlighting}[]
\NormalTok{\#must{-}quoted              // renders as: "MUST"}
\NormalTok{\#must{-}not{-}quoted          // renders as: "MUST NOT"}
\NormalTok{\#required{-}quoted          // renders as: "REQUIRED"}
\NormalTok{\#shall{-}quoted             // renders as: "SHALL"}
\NormalTok{\#shall{-}not{-}quoted         // renders as: "SHALL NOT"}
\NormalTok{\#should{-}quoted            // renders as: "SHOULD"}
\NormalTok{\#should{-}not{-}quoted        // renders as: "SHOULD NOT"}
\NormalTok{\#recommended{-}quoted       // renders as: "RECOMMENDED"}
\NormalTok{\#not{-}recommended{-}quoted   // renders as: "NOT RECOMMENDED"}
\NormalTok{\#may{-}quoted               // renders as: "MAY"}
\NormalTok{\#optional{-}quoted          // renders as: "OPTIONAL"}
\end{Highlighting}
\end{Shaded}

\subsubsection{Boilerplate}\label{boilerplate}

According to RFC8174, \emph{authors who follow these guidelines should
incorporate a specific phrase near the beginning of their document} .
Include this boilerplate text with:

\begin{Shaded}
\begin{Highlighting}[]
\NormalTok{\#rfc{-}keyword{-}boilerplate}
\end{Highlighting}
\end{Shaded}

This will render as:

\begin{Shaded}
\begin{Highlighting}[]
\NormalTok{The key words "MUST", "MUST NOT", "REQUIRED", "SHALL", "SHALL NOT", "SHOULD",}
\NormalTok{"SHOULD NOT", "RECOMMENDED", "NOT RECOMMENDED", "MAY", and "OPTIONAL" in this}
\NormalTok{document are to be interpreted as described in BCP 14 [RFC2119] [RFC8174] when,}
\NormalTok{and only when, they appear in all capitals, as shown here.}
\end{Highlighting}
\end{Shaded}

\subsubsection{Definitions}\label{definitions}

Although not required (and maybe discouraged), you can include
definitions of individual keywords in your document:

\begin{Shaded}
\begin{Highlighting}[]
\NormalTok{\#rfc{-}keyword{-}must{-}definition}
\NormalTok{\#rfc{-}keyword{-}must{-}not{-}definition}
\NormalTok{\#rfc{-}keyword{-}should{-}definition}
\NormalTok{\#rfc{-}keyword{-}should{-}not{-}definition}
\NormalTok{\#rfc{-}keyword{-}may{-}definition}
\end{Highlighting}
\end{Shaded}

Or include all keyword definitions at once with:

\begin{Shaded}
\begin{Highlighting}[]
\NormalTok{\#rfc{-}keyword{-}definitions}
\end{Highlighting}
\end{Shaded}

\subsection{Example Output}\label{example-output}

\pandocbounded{\includegraphics[keepaspectratio]{https://github.com/typst/packages/raw/main/packages/preview/rfc-vibe/0.1.0/thumbnail.png}}

\subsection{License}\label{license}

This project is licensed under the MIT License. See the
\href{https://github.com/typst/packages/raw/main/packages/preview/rfc-vibe/0.1.0/LICENSE}{LICENSE}
file for details.

\subsubsection{How to add}\label{how-to-add}

Copy this into your project and use the import as \texttt{\ rfc-vibe\ }

\begin{verbatim}
#import "@preview/rfc-vibe:0.1.0"
\end{verbatim}

\includesvg[width=0.16667in,height=0.16667in]{/assets/icons/16-copy.svg}

Check the docs for
\href{https://typst.app/docs/reference/scripting/\#packages}{more
information on how to import packages} .

\subsubsection{About}\label{about}

\begin{description}
\tightlist
\item[Author :]
\href{mailto:steve@waits.net}{Stephen Waits}
\item[License:]
MIT
\item[Current version:]
0.1.0
\item[Last updated:]
November 28, 2024
\item[First released:]
November 28, 2024
\item[Archive size:]
3.35 kB
\href{https://packages.typst.org/preview/rfc-vibe-0.1.0.tar.gz}{\pandocbounded{\includesvg[keepaspectratio]{/assets/icons/16-download.svg}}}
\item[Repository:]
\href{https://github.com/swaits/typst-collection}{GitHub}
\item[Categor y :]
\begin{itemize}
\tightlist
\item[]
\item
  \pandocbounded{\includesvg[keepaspectratio]{/assets/icons/16-hammer.svg}}
  \href{https://typst.app/universe/search/?category=utility}{Utility}
\end{itemize}
\end{description}

\subsubsection{Where to report issues?}\label{where-to-report-issues}

This package is a project of Stephen Waits . Report issues on
\href{https://github.com/swaits/typst-collection}{their repository} .
You can also try to ask for help with this package on the
\href{https://forum.typst.app}{Forum} .

Please report this package to the Typst team using the
\href{https://typst.app/contact}{contact form} if you believe it is a
safety hazard or infringes upon your rights.

\phantomsection\label{versions}
\subsubsection{Version history}\label{version-history}

\begin{longtable}[]{@{}ll@{}}
\toprule\noalign{}
Version & Release Date \\
\midrule\noalign{}
\endhead
\bottomrule\noalign{}
\endlastfoot
0.1.0 & November 28, 2024 \\
\end{longtable}

Typst GmbH did not create this package and cannot guarantee correct
functionality of this package or compatibility with any version of the
Typst compiler or app.


\title{typst.app/universe/package/tufte-memo}

\phantomsection\label{banner}
\phantomsection\label{template-thumbnail}
\pandocbounded{\includegraphics[keepaspectratio]{https://packages.typst.org/preview/thumbnails/tufte-memo-0.1.2-small.webp}}

\section{tufte-memo}\label{tufte-memo}

{ 0.1.2 }

A memo template inspired by the design of Edward Tufte\textquotesingle s
books

{ } Featured Template

\href{/app?template=tufte-memo&version=0.1.2}{Create project in app}

\phantomsection\label{readme}
A memo document template inspired by the design of Edward Tufte’s
books for the Typst typesetting program.

For usage, see the usage guide
\href{https://github.com/nogula/tufte-memo/blob/main/template/main.pdf}{here}
.

The template provides handy functions: \texttt{\ template\ } ,
\texttt{\ note\ } , and \texttt{\ wideblock\ } . To create a document
with this template, use:

\begin{Shaded}
\begin{Highlighting}[]
\NormalTok{\#import "@preview/tufte{-}memo:0.1.2": *}

\NormalTok{\#show: template.with(}
\NormalTok{    title: [Document Title],}
\NormalTok{    authors: (}
\NormalTok{        (}
\NormalTok{        name: "Author Name",}
\NormalTok{        role: "Optional Role Line",}
\NormalTok{        affiliation: "Optional Affiliation Line",}
\NormalTok{        email: "email@company.com"}
\NormalTok{        ),}
\NormalTok{    )}
\NormalTok{)}
\NormalTok{...}
\end{Highlighting}
\end{Shaded}

additional configuration information is available in the usage guide.

The \texttt{\ note()\ } function provides the ability to produce
sidenotes next to the main body content. It can be called simply with
\texttt{\ \#note{[}...{]}\ } . Additionally, \texttt{\ wideblock()\ }
expands the width of its content to fill the full 6.5-inch-wide space,
rather than be compressed in to a four-inch column. It is simply called
with \texttt{\ wideblock{[}...{]}\ } .

\href{/app?template=tufte-memo&version=0.1.2}{Create project in app}

\subsubsection{How to use}\label{how-to-use}

Click the button above to create a new project using this template in
the Typst app.

You can also use the Typst CLI to start a new project on your computer
using this command:

\begin{verbatim}
typst init @preview/tufte-memo:0.1.2
\end{verbatim}

\includesvg[width=0.16667in,height=0.16667in]{/assets/icons/16-copy.svg}

\subsubsection{About}\label{about}

\begin{description}
\tightlist
\item[Author :]
Noah Gula
\item[License:]
MIT
\item[Current version:]
0.1.2
\item[Last updated:]
August 12, 2024
\item[First released:]
June 3, 2024
\item[Archive size:]
9.31 kB
\href{https://packages.typst.org/preview/tufte-memo-0.1.2.tar.gz}{\pandocbounded{\includesvg[keepaspectratio]{/assets/icons/16-download.svg}}}
\item[Repository:]
\href{https://github.com/nogula/tufte-memo}{GitHub}
\item[Categor y :]
\begin{itemize}
\tightlist
\item[]
\item
  \pandocbounded{\includesvg[keepaspectratio]{/assets/icons/16-speak.svg}}
  \href{https://typst.app/universe/search/?category=report}{Report}
\end{itemize}
\end{description}

\subsubsection{Where to report issues?}\label{where-to-report-issues}

This template is a project of Noah Gula . Report issues on
\href{https://github.com/nogula/tufte-memo}{their repository} . You can
also try to ask for help with this template on the
\href{https://forum.typst.app}{Forum} .

Please report this template to the Typst team using the
\href{https://typst.app/contact}{contact form} if you believe it is a
safety hazard or infringes upon your rights.

\phantomsection\label{versions}
\subsubsection{Version history}\label{version-history}

\begin{longtable}[]{@{}ll@{}}
\toprule\noalign{}
Version & Release Date \\
\midrule\noalign{}
\endhead
\bottomrule\noalign{}
\endlastfoot
0.1.2 & August 12, 2024 \\
\href{https://typst.app/universe/package/tufte-memo/0.1.1/}{0.1.1} &
June 5, 2024 \\
\href{https://typst.app/universe/package/tufte-memo/0.1.0/}{0.1.0} &
June 3, 2024 \\
\end{longtable}

Typst GmbH did not create this template and cannot guarantee correct
functionality of this template or compatibility with any version of the
Typst compiler or app.


\title{typst.app/universe/package/backtrack}

\phantomsection\label{banner}
\section{backtrack}\label{backtrack}

{ 1.0.0 }

A version-agnostic library for checking the compiler version.

\phantomsection\label{readme}
Backtrack is a simple and performant Typst library that determines the
current compiler version and provides an API for comparing, displaying,
and observing versions.

Unlike the built-in
\href{https://github.com/typst/typst/pull/2016}{version API} which is
only available on Typst 0.9.0+, Backtrack works on any
\href{https://github.com/typst/packages/raw/main/packages/preview/backtrack/1.0.0/\#version-support}{*}
Typst version. It uses the built-in API when available so that it’ll
continue to work on all future Typst versions without modification.

Additionally, it:

\begin{itemize}
\tightlist
\item
  doesn’t noticeably impact compilation time. All version checks are
  extremely simple, and newer versions are checked first to avoid
  overhead from supporting old versions.
\item
  is automatically tested on \emph{every} supported Typst version to
  ensure reliability.
\item
  can be downloaded and installed manually in addition to being
  available as a package.
\end{itemize}

\subsection{Example}\label{example}

\begin{Shaded}
\begin{Highlighting}[]
\NormalTok{\#import "@preview/backtrack:1.0.0": current{-}version, versions}

\NormalTok{You are using Typst \#current{-}version.displayable!}
\NormalTok{\#\{}
\NormalTok{  if current{-}version.cmpable \textless{}= versions.v2023{-}03{-}28.cmpable [}
\NormalTok{    That is ancient.}
\NormalTok{  ] else if current{-}version.cmpable \textless{} versions.v0{-}5{-}0.cmpable [}
\NormalTok{    That is old.}
\NormalTok{  ] else [}
\NormalTok{    That is modern.}
\NormalTok{  ]}
\NormalTok{\}}
\end{Highlighting}
\end{Shaded}

\subsection{Installation}\label{installation}

There are two ways to install the library:

\begin{itemize}
\item
  Use the package on Typst 0.6.0+. This is as simple as adding the
  following line to your document:

\begin{Shaded}
\begin{Highlighting}[]
\NormalTok{\#import "@preview/backtrack:1.0.0"}
\end{Highlighting}
\end{Shaded}
\item
  Download and install the library manually:

  \begin{enumerate}
  \item
    Download and extract the latest
    \href{https://github.com/TheLukeGuy/backtrack/releases}{release} .
  \item
    Rename \texttt{\ src/lib.typ\ } to \texttt{\ src/backtrack.typ\ } .
  \item
    Move/copy \texttt{\ COPYING\ } into \texttt{\ src/\ } .
  \item
    Rename the \texttt{\ src/\ } directory to \texttt{\ backtrack/\ } .
  \item
    Move/copy the newly-renamed \texttt{\ backtrack/\ } directory into
    your project.
  \item
    Add the following line to your document:

\begin{Shaded}
\begin{Highlighting}[]
\NormalTok{\#import "[path/to/]backtrack/backtrack.typ"}
\end{Highlighting}
\end{Shaded}
  \end{enumerate}
\end{itemize}

\subsection{Documentation}\label{documentation}

See
\href{https://github.com/typst/packages/raw/main/packages/preview/backtrack/1.0.0/DOCS.md}{DOCS.md}
. It’s quite short. 😀

\subsection{Version Support}\label{version-support}

Backtrack compiles on and can detect these versions:

\begin{longtable}[]{@{}lcl@{}}
\toprule\noalign{}
Version & Status & Notes \\
\midrule\noalign{}
\endhead
\bottomrule\noalign{}
\endlastfoot
0.6.0+ & ✠& Also available as a package \\
March 28, 2023â€``0.5.0 & ✠& \\
March 21, 2023 & ✠& Initial public/standalone Typst release \\
February 25, 2023 & 🟡 & Detects as February 15, 2023 \\
February 12â€``15, 2023 & ✠& \\
February 2, 2023 & 🟡 & Detects as January 30, 2023 \\
January 30, 2023 & ✠& \\
\end{longtable}

The partially-supported versions \emph{may} be possible to detect, but
they’re tricky since most of their changes are content-related.
Content values were opaque up until the March 21, 2023 release, making
it difficult to automatically check for the presence of these changes.

\subsection{Copying}\label{copying}

Copyright © 2023 \href{https://github.com/TheLukeGuy}{Luke Chambers}

Backtrack is licensed under the Apache License, Version 2.0. You can
find a copy of the license in
\href{https://github.com/typst/packages/raw/main/packages/preview/backtrack/1.0.0/COPYING}{COPYING}
or online at \url{https://www.apache.org/licenses/LICENSE-2.0} .

\subsubsection{How to add}\label{how-to-add}

Copy this into your project and use the import as \texttt{\ backtrack\ }

\begin{verbatim}
#import "@preview/backtrack:1.0.0"
\end{verbatim}

\includesvg[width=0.16667in,height=0.16667in]{/assets/icons/16-copy.svg}

Check the docs for
\href{https://typst.app/docs/reference/scripting/\#packages}{more
information on how to import packages} .

\subsubsection{About}\label{about}

\begin{description}
\tightlist
\item[Author :]
TheLukeGuy
\item[License:]
Apache-2.0
\item[Current version:]
1.0.0
\item[Last updated:]
October 27, 2023
\item[First released:]
October 27, 2023
\item[Archive size:]
8.64 kB
\href{https://packages.typst.org/preview/backtrack-1.0.0.tar.gz}{\pandocbounded{\includesvg[keepaspectratio]{/assets/icons/16-download.svg}}}
\item[Repository:]
\href{https://github.com/TheLukeGuy/backtrack}{GitHub}
\end{description}

\subsubsection{Where to report issues?}\label{where-to-report-issues}

This package is a project of TheLukeGuy . Report issues on
\href{https://github.com/TheLukeGuy/backtrack}{their repository} . You
can also try to ask for help with this package on the
\href{https://forum.typst.app}{Forum} .

Please report this package to the Typst team using the
\href{https://typst.app/contact}{contact form} if you believe it is a
safety hazard or infringes upon your rights.

\phantomsection\label{versions}
\subsubsection{Version history}\label{version-history}

\begin{longtable}[]{@{}ll@{}}
\toprule\noalign{}
Version & Release Date \\
\midrule\noalign{}
\endhead
\bottomrule\noalign{}
\endlastfoot
1.0.0 & October 27, 2023 \\
\end{longtable}

Typst GmbH did not create this package and cannot guarantee correct
functionality of this package or compatibility with any version of the
Typst compiler or app.


\title{typst.app/universe/package/delegis}

\phantomsection\label{banner}
\phantomsection\label{template-thumbnail}
\pandocbounded{\includegraphics[keepaspectratio]{https://packages.typst.org/preview/thumbnails/delegis-0.3.0-small.webp}}

\section{delegis}\label{delegis}

{ 0.3.0 }

A package and template for drafting legislative content in a
German-style structuring, such as for bylaws, etc.

\href{/app?template=delegis&version=0.3.0}{Create project in app}

\phantomsection\label{readme}
\begin{longtable}[]{@{}lll@{}}
\toprule\noalign{}
\endhead
\bottomrule\noalign{}
\endlastfoot
\pandocbounded{\includegraphics[keepaspectratio]{https://github.com/typst/packages/raw/main/packages/preview/delegis/0.3.0/demo-1.png}}
&
\pandocbounded{\includegraphics[keepaspectratio]{https://github.com/typst/packages/raw/main/packages/preview/delegis/0.3.0/demo-2.png}}
&
\pandocbounded{\includegraphics[keepaspectratio]{https://github.com/typst/packages/raw/main/packages/preview/delegis/0.3.0/demo-3.png}} \\
\end{longtable}

A package and template for drafting legislative content in a
German-style structuring, such as for bylaws, etc.

While the template is designed to be used in German documents, all
strings are customizable. You can have a look at the
\texttt{\ delegis.typ\ } to see all available parameters.

\subsection{General Usage}\label{general-usage}

While this \texttt{\ README.md\ } gives you a brief overview of the
package’s usage, we recommend that you use the template (in the
\texttt{\ template\ } folder) as a starting point instead.

\subsubsection{Importing the Package}\label{importing-the-package}

\begin{Shaded}
\begin{Highlighting}[]
\NormalTok{\#import "@preview/delegis:0.3.0": *}
\end{Highlighting}
\end{Shaded}

\subsubsection{Initializing the
template}\label{initializing-the-template}

\begin{Shaded}
\begin{Highlighting}[]
\NormalTok{\#show: delegis.with(}
\NormalTok{  // Metadata}
\NormalTok{  title: "Vereinsordnung zu ABCDEF", // title of the law/bylaw/...}
\NormalTok{  abbreviation: "ABCDEFVO", // abbreviation of the law/bylaw/...}
\NormalTok{  resolution: "3. Beschluss des Vorstands vom 24.01.2024", // resolution number and date}
\NormalTok{  in{-}effect: "24.01.2024", // date when it comes into effect}
\NormalTok{  draft: false, // whether this is a draft}
\NormalTok{  // Template}
\NormalTok{  logo: image("wuespace.jpg", alt: "WüSpace e. V."), // logo of the organization, shown on the first page}
\NormalTok{)}
\end{Highlighting}
\end{Shaded}

\subsubsection{Sections}\label{sections}

Sections are auto-detected as long as they follow the pattern
\texttt{\ §\ 1\ ...\ } or \texttt{\ §\ 1a\ ...\ } in its own
paragraph:

\begin{Shaded}
\begin{Highlighting}[]
\NormalTok{§ 1 Geltungsbereich}

\NormalTok{(1) }
\NormalTok{Diese Ordnung gilt für alle Mitglieder des Vereins.}

\NormalTok{(2) }
\NormalTok{Sie regelt die Mitgliedschaft im Verein.}

\NormalTok{§ 2 Mitgliedschaft}

\NormalTok{(1) }
\NormalTok{Die Mitgliedschaft im Verein ist freiwillig.}

\NormalTok{(2) }
\NormalTok{Sie kann jederzeit gekündigt werden.}

\NormalTok{§ 2a Ehrenmitgliedschaft}

\NormalTok{(1) }
\NormalTok{Die Ehrenmitgliedschaft wird durch den Vorstand verliehen.}
\end{Highlighting}
\end{Shaded}

Alternatively (or if you want to use special characters otherwise not
supported, such as \texttt{\ *\ } ), you can also use the
\texttt{\ \#section{[}number{]}{[}title{]}\ } function:

\begin{Shaded}
\begin{Highlighting}[]
\NormalTok{\#section[§ 3][Administrator*innen]}
\end{Highlighting}
\end{Shaded}

\subsubsection{Hierarchical Divisions}\label{hierarchical-divisions}

If you want to add more structure to your sections, you can use normal
Typst headings. Note that only the level 6 headings are reserved for the
section numbers:

\begin{Shaded}
\begin{Highlighting}[]
\NormalTok{= Allgemeine Bestimmungen}

\NormalTok{§ 1 ABC}

\NormalTok{§ 2 DEF}

\NormalTok{= Besondere Bestimmungen}

\NormalTok{§ 3 GHI}

\NormalTok{§ 4 JKL}
\end{Highlighting}
\end{Shaded}

Delegis will automatically use a numbering scheme for the divisions that
is in line with the “Handbuch der Rechtsförmlichkeit�, Rn. 379 f.
If you want to customize the division titles, you can do so by setting
the \texttt{\ division-prefixes\ } parameter in the \texttt{\ delegis\ }
function:

\begin{Shaded}
\begin{Highlighting}[]
\NormalTok{\#show: delegis.with(}
\NormalTok{  division{-}prefixes: ("Teil", "Kapitel", "Abschnitt", "Unterabschnitt")}
\NormalTok{)}
\end{Highlighting}
\end{Shaded}

\subsubsection{Sentence Numbering}\label{sentence-numbering}

If a paragraph contains multiple sentences, you can number them by
adding a \texttt{\ \#s\textasciitilde{}\ } at the beginning of the
sentences:

\begin{Shaded}
\begin{Highlighting}[]
\NormalTok{§ 3 Mitgliedsbeiträge}

\NormalTok{\#s\textasciitilde{}Die Mitgliedsbeiträge sind monatlich zu entrichten.}
\NormalTok{\#s\textasciitilde{}Sie sind bis zum 5. des Folgemonats zu zahlen.}
\end{Highlighting}
\end{Shaded}

This automatically adds corresponding sentence numbers in superscript.

\subsubsection{Referencing other
Sections}\label{referencing-other-sections}

Referencing works manually by specifying the section number. While
automations would be feasible, we have found that in practice, they’re
not as useful as they might seem for legislative documents.

In some cases, referencing sections using \texttt{\ §\ X\ } could be
mis-interpreted as a new section. To avoid this, use the non-breaking
space character \texttt{\ \textasciitilde{}\ } between the
\texttt{\ §\ } and the number:

\begin{Shaded}
\begin{Highlighting}[]
\NormalTok{§ 5 Inkrafttreten}

\NormalTok{Diese Ordnung tritt am 24.01.2024 in Kraft. §\textasciitilde{}4 bleibt unberührt.}
\end{Highlighting}
\end{Shaded}

\subsection{Changelog}\label{changelog}

\subsubsection{v0.3.0}\label{v0.3.0}

\paragraph{Features}\label{features}

\begin{itemize}
\tightlist
\item
  Adjust numbered list / enumeration numbering to be in line with
  “Handbuch der Rechtsförmlichkeit�, Rn. 374
\item
  Make division titles (e.g., “Part�, “Chapter�, “Division�)
  customizable and conform to the “Handbuch der
  Rechtsförmlichkeit�, Rn. 379 f.
\end{itemize}

\subsubsection{v0.2.0}\label{v0.2.0}

\paragraph{Features}\label{features-1}

\begin{itemize}
\tightlist
\item
  Add \texttt{\ \#metadata\ } fields for usage with
  \texttt{\ typst\ query\ } . You can now use
  \texttt{\ typst\ query\ file.typ\ "\textless{}field\textgreater{}"\ -\/-field\ value\ -\/-one\ }
  with \texttt{\ \textless{}field\textgreater{}\ } being one of the
  following to query metadata fields in the command line:

  \begin{itemize}
  \tightlist
  \item
    \texttt{\ \textless{}title\textgreater{}\ }
  \item
    \texttt{\ \textless{}abbreviation\textgreater{}\ }
  \item
    \texttt{\ \textless{}resolution\textgreater{}\ }
  \item
    \texttt{\ \textless{}in-effect\textgreater{}\ }
  \end{itemize}
\item
  Add \texttt{\ \#section{[}§\ 1{]}{[}ABC{]}\ } function to enable
  previously unsupported special chars (such as \texttt{\ *\ } ) in
  section headings. Note that this was previously possible using
  \texttt{\ \#unnumbered{[}§\ 1\textbackslash{}\ ABC{]}\ } , but the
  new function adds a semantically better-fitting alternative to this
  fix.
\item
  Improve heading style rules. This also fixes an incompatibility with
  \texttt{\ pandoc\ } , meaning it’s now possible to use
  \texttt{\ pandoc\ } to convert delegis documents to HTML, etc.
\item
  Set the footnote numbering to \texttt{\ {[}1{]}\ } to not collide with
  sentence numbers.
\end{itemize}

\paragraph{Bug Fixes}\label{bug-fixes}

\begin{itemize}
\tightlist
\item
  Fix a typo in the \texttt{\ str-draft\ } variable name that lead to
  draft documents resulting in a syntax error.
\item
  Fix hyphenation issues with the abbreviation on the title page
  (hyphenation between the parentheses and the abbreviation itself)
\end{itemize}

\subsubsection{v0.1.0}\label{v0.1.0}

Initial Release

\href{/app?template=delegis&version=0.3.0}{Create project in app}

\subsubsection{How to use}\label{how-to-use}

Click the button above to create a new project using this template in
the Typst app.

You can also use the Typst CLI to start a new project on your computer
using this command:

\begin{verbatim}
typst init @preview/delegis:0.3.0
\end{verbatim}

\includesvg[width=0.16667in,height=0.16667in]{/assets/icons/16-copy.svg}

\subsubsection{About}\label{about}

\begin{description}
\tightlist
\item[Author :]
\href{https://github.com/wuespace}{WüSpace e. V.}
\item[License:]
MIT
\item[Current version:]
0.3.0
\item[Last updated:]
May 22, 2024
\item[First released:]
March 16, 2024
\item[Archive size:]
13.4 kB
\href{https://packages.typst.org/preview/delegis-0.3.0.tar.gz}{\pandocbounded{\includesvg[keepaspectratio]{/assets/icons/16-download.svg}}}
\item[Repository:]
\href{https://github.com/wuespace/delegis}{GitHub}
\item[Discipline :]
\begin{itemize}
\tightlist
\item[]
\item
  \href{https://typst.app/universe/search/?discipline=law}{Law}
\end{itemize}
\item[Categor y :]
\begin{itemize}
\tightlist
\item[]
\item
  \pandocbounded{\includesvg[keepaspectratio]{/assets/icons/16-envelope.svg}}
  \href{https://typst.app/universe/search/?category=office}{Office}
\end{itemize}
\end{description}

\subsubsection{Where to report issues?}\label{where-to-report-issues}

This template is a project of WüSpace e. V. . Report issues on
\href{https://github.com/wuespace/delegis}{their repository} . You can
also try to ask for help with this template on the
\href{https://forum.typst.app}{Forum} .

Please report this template to the Typst team using the
\href{https://typst.app/contact}{contact form} if you believe it is a
safety hazard or infringes upon your rights.

\phantomsection\label{versions}
\subsubsection{Version history}\label{version-history}

\begin{longtable}[]{@{}ll@{}}
\toprule\noalign{}
Version & Release Date \\
\midrule\noalign{}
\endhead
\bottomrule\noalign{}
\endlastfoot
0.3.0 & May 22, 2024 \\
\href{https://typst.app/universe/package/delegis/0.2.0/}{0.2.0} & May
17, 2024 \\
\href{https://typst.app/universe/package/delegis/0.1.0/}{0.1.0} & March
16, 2024 \\
\end{longtable}

Typst GmbH did not create this template and cannot guarantee correct
functionality of this template or compatibility with any version of the
Typst compiler or app.


\title{typst.app/universe/package/diverential}

\phantomsection\label{banner}
\section{diverential}\label{diverential}

{ 0.2.0 }

Format differentials conveniently.

\phantomsection\label{readme}
\texttt{\ diverential\ } is a
\href{https://github.com/typst/typst}{Typst} package simplifying the
typesetting of differentials. It is the equivalent to LaTeX’s
\texttt{\ diffcoeff\ } , though not as mature.

\subsection{Overview}\label{overview}

\texttt{\ diverential\ } allows normal, partial, compact, and separated
derivatives with smart degree calculations.

\begin{Shaded}
\begin{Highlighting}[]
\NormalTok{\#}\ImportTok{import} \StringTok{"@preview/diverential:0.2.0"}\OperatorTok{:} \OperatorTok{*}

\NormalTok{$ }\FunctionTok{dv}\NormalTok{(f}\OperatorTok{,}\NormalTok{ x}\OperatorTok{,}\NormalTok{ deg}\OperatorTok{:} \DecValTok{2}\OperatorTok{,}\NormalTok{ eval}\OperatorTok{:} \DecValTok{0}\NormalTok{) $}
\NormalTok{$ }\FunctionTok{dvp}\NormalTok{(f}\OperatorTok{,}\NormalTok{ x}\OperatorTok{,}\NormalTok{ y}\OperatorTok{,}\NormalTok{ eval}\OperatorTok{:} \DecValTok{0}\OperatorTok{,}\NormalTok{ evalsym}\OperatorTok{:} \StringTok{"["}\NormalTok{) $}
\NormalTok{$ }\FunctionTok{dvpc}\NormalTok{(f}\OperatorTok{,}\NormalTok{ x) $}
\NormalTok{$ }\FunctionTok{dvps}\NormalTok{(f}\OperatorTok{,}\NormalTok{ \#([x]}\OperatorTok{,} \DecValTok{2}\NormalTok{)}\OperatorTok{,}\NormalTok{ \#([y]}\OperatorTok{,}\NormalTok{ [n])}\OperatorTok{,}\NormalTok{ \#([z]}\OperatorTok{,}\NormalTok{ [m])}\OperatorTok{,}\NormalTok{ eval}\OperatorTok{:} \DecValTok{0}\NormalTok{) $}
\end{Highlighting}
\end{Shaded}

\includegraphics[width=1.5625in,height=\textheight,keepaspectratio]{https://github.com/typst/packages/raw/main/packages/preview/diverential/0.2.0/examples/overview.jpg}

\subsection{\texorpdfstring{\texttt{\ dv\ }}{ dv }}\label{dv}

\texttt{\ dv\ } is an ordinary derivative. It takes the function as its
first argument and the variable as its second one. A degree can be
specified with \texttt{\ deg\ } . The derivate can be specified to be
evaluated at a point with \texttt{\ eval\ } , the brackets of which can
be changed with \texttt{\ evalsym\ } . \texttt{\ space\ } influences the
space between derivative and evaluation bracket. Unless defined
otherwise, no space is set by default, except for
\texttt{\ \textbar{}\ } , where a small gap is introduced.

\subsubsection{\texorpdfstring{\texttt{\ dvs\ }}{ dvs }}\label{dvs}

Same as \texttt{\ dv\ } , but separates the function from the
derivative.\\
Example: \$\$
\textbackslash frac\{\textbackslash mathrm\{d\}\}\{\textbackslash mathrm\{d\}x\}
f \$\$

\subsubsection{\texorpdfstring{\texttt{\ dvc\ }}{ dvc }}\label{dvc}

Same as \texttt{\ dv\ } , but uses a compact derivative.\\
Example: \$\$ \textbackslash mathrm\{d\}\_x f \$\$

\subsection{\texorpdfstring{\texttt{\ dvp\ }}{ dvp }}\label{dvp}

\texttt{\ dv\ } is a partial derivative. It takes the function as its
first argument and the variable as the rest. For information on
\texttt{\ eval\ } , \texttt{\ evalsym\ } , and \texttt{\ space\ } , read
the description of \texttt{\ dv\ } .\\
The variable can be one of these options:

\begin{itemize}
\tightlist
\item
  plain variable, e.g. \texttt{\ x\ }
\item
  list of variables, e.g. \texttt{\ x,\ y\ }
\item
  list of variables with higher degrees (type
  \texttt{\ (content,\ integer)\ } ), e.g.
  \texttt{\ x,\ \#({[}y{]},\ 2)\ } The degree is smartly calculated: If
  all degrees of the variables are integers, the total degree is their
  sum. If some are content, the integer ones are summed (arithmetically)
  up and added to the visual sum of the content degrees. Example:
  \texttt{\ \#({[}x{]},\ n),\ \#({[}y{]},\ 2),\ z\ } â†'
  \$\textbackslash frac\{\textbackslash partial\^{}\{n+3\}\}\{\textbackslash partial
  x\^{}n,\textbackslash partial y\^{}2,\textbackslash partial z\}\$\\
  Specifying \texttt{\ deg\ } manually is always possible and might be
  required in more complicated cases.
\end{itemize}

\subsubsection{\texorpdfstring{\texttt{\ dvps\ }}{ dvps }}\label{dvps}

Same as \texttt{\ dvp\ } , but separates the function from the
derivative.\\
Example: \$\$
\textbackslash frac\{\textbackslash partial\}\{\textbackslash partial
x\} f \$\$

\subsubsection{\texorpdfstring{\texttt{\ dvpc\ }}{ dvpc }}\label{dvpc}

Same as \texttt{\ dvp\ } , but uses a compact derivative.\\
Note: If supplying multiple variables, \texttt{\ deg\ } is ignored.\\
Example: \$\$ \textbackslash partial\_x f \$\$

\subsubsection{How to add}\label{how-to-add}

Copy this into your project and use the import as
\texttt{\ diverential\ }

\begin{verbatim}
#import "@preview/diverential:0.2.0"
\end{verbatim}

\includesvg[width=0.16667in,height=0.16667in]{/assets/icons/16-copy.svg}

Check the docs for
\href{https://typst.app/docs/reference/scripting/\#packages}{more
information on how to import packages} .

\subsubsection{About}\label{about}

\begin{description}
\tightlist
\item[Author :]
Christopher Hecker
\item[License:]
MIT
\item[Current version:]
0.2.0
\item[Last updated:]
July 29, 2023
\item[First released:]
July 29, 2023
\item[Archive size:]
3.38 kB
\href{https://packages.typst.org/preview/diverential-0.2.0.tar.gz}{\pandocbounded{\includesvg[keepaspectratio]{/assets/icons/16-download.svg}}}
\end{description}

\subsubsection{Where to report issues?}\label{where-to-report-issues}

This package is a project of Christopher Hecker . You can also try to
ask for help with this package on the
\href{https://forum.typst.app}{Forum} .

Please report this package to the Typst team using the
\href{https://typst.app/contact}{contact form} if you believe it is a
safety hazard or infringes upon your rights.

\phantomsection\label{versions}
\subsubsection{Version history}\label{version-history}

\begin{longtable}[]{@{}ll@{}}
\toprule\noalign{}
Version & Release Date \\
\midrule\noalign{}
\endhead
\bottomrule\noalign{}
\endlastfoot
0.2.0 & July 29, 2023 \\
\end{longtable}

Typst GmbH did not create this package and cannot guarantee correct
functionality of this package or compatibility with any version of the
Typst compiler or app.


\title{typst.app/universe/package/indenta}

\phantomsection\label{banner}
\section{indenta}\label{indenta}

{ 0.0.3 }

Fix indent of first paragraph.

\phantomsection\label{readme}
An attempt to fix the indentation of the first paragraph in typst.

It works.

\subsection{Usage}\label{usage}

\begin{Shaded}
\begin{Highlighting}[]
\NormalTok{\#set par(first{-}line{-}indent: 2em)}
\NormalTok{\#import "@preview/indenta:0.0.3": fix{-}indent}
\NormalTok{\#show: fix{-}indent()}
\end{Highlighting}
\end{Shaded}

\subsection{Demo}\label{demo}

\pandocbounded{\includegraphics[keepaspectratio]{https://github.com/flaribbit/indenta/assets/24885181/874df696-3277-4103-9166-a24639b0c7c6}}

\subsection{Note}\label{note}

When you use \texttt{\ fix-indent()\ } with other show rules, make sure
to call \texttt{\ fix-indent()\ } \textbf{after other show rules} . For
example:

\begin{Shaded}
\begin{Highlighting}[]
\NormalTok{\#show heading.where(level: 1): set text(size: 20pt)}
\NormalTok{\#show: fix{-}indent()}
\end{Highlighting}
\end{Shaded}

If you want to process the content inside your custom block, you can
call \texttt{\ fix-indent\ } inside your block. For example:

\begin{Shaded}
\begin{Highlighting}[]
\NormalTok{\#block[\#set text(fill: red)}
\NormalTok{\#show: fix{-}indent()}

\NormalTok{Hello}

\NormalTok{\#table()[table]}

\NormalTok{World}
\NormalTok{]}
\end{Highlighting}
\end{Shaded}

This package is in a very early stage and may not work as expected in
some cases. Currently, there is no easy way to check if an element is
inlined or not. If you got an unexpected result, you can try
\texttt{\ fix-indent(unsafe:\ true)\ } to disable the check.

Minor fixes can be made at any time, but the package in typst universe
may not be updated immediately. You can check the latest version on
\href{https://github.com/flaribbit/indenta}{GitHub} then copy and paste
the code into your typst file.

If it still doesn’t work as expected, you can try another solution
(aka fake-par solution):

\begin{Shaded}
\begin{Highlighting}[]
\NormalTok{\#let fakepar=context\{box();v({-}measure(block()+block()).height)\}}
\NormalTok{\#show heading: it=\textgreater{}it+fakepar}
\NormalTok{\#show figure: it=\textgreater{}it+fakepar}
\NormalTok{\#show math.equation.where(block: true): it=\textgreater{}it+fakepar}
\NormalTok{// ... other elements}
\end{Highlighting}
\end{Shaded}

\subsubsection{How to add}\label{how-to-add}

Copy this into your project and use the import as \texttt{\ indenta\ }

\begin{verbatim}
#import "@preview/indenta:0.0.3"
\end{verbatim}

\includesvg[width=0.16667in,height=0.16667in]{/assets/icons/16-copy.svg}

Check the docs for
\href{https://typst.app/docs/reference/scripting/\#packages}{more
information on how to import packages} .

\subsubsection{About}\label{about}

\begin{description}
\tightlist
\item[Author :]
\href{https://github.com/flaribbit}{梦飞ç¿''}
\item[License:]
MIT
\item[Current version:]
0.0.3
\item[Last updated:]
June 10, 2024
\item[First released:]
May 24, 2024
\item[Archive size:]
2.35 kB
\href{https://packages.typst.org/preview/indenta-0.0.3.tar.gz}{\pandocbounded{\includesvg[keepaspectratio]{/assets/icons/16-download.svg}}}
\item[Repository:]
\href{https://github.com/flaribbit/indenta}{GitHub}
\item[Categor y :]
\begin{itemize}
\tightlist
\item[]
\item
  \pandocbounded{\includesvg[keepaspectratio]{/assets/icons/16-hammer.svg}}
  \href{https://typst.app/universe/search/?category=utility}{Utility}
\end{itemize}
\end{description}

\subsubsection{Where to report issues?}\label{where-to-report-issues}

This package is a project of 梦飞ç¿'' . Report issues on
\href{https://github.com/flaribbit/indenta}{their repository} . You can
also try to ask for help with this package on the
\href{https://forum.typst.app}{Forum} .

Please report this package to the Typst team using the
\href{https://typst.app/contact}{contact form} if you believe it is a
safety hazard or infringes upon your rights.

\phantomsection\label{versions}
\subsubsection{Version history}\label{version-history}

\begin{longtable}[]{@{}ll@{}}
\toprule\noalign{}
Version & Release Date \\
\midrule\noalign{}
\endhead
\bottomrule\noalign{}
\endlastfoot
0.0.3 & June 10, 2024 \\
\href{https://typst.app/universe/package/indenta/0.0.2/}{0.0.2} & May
27, 2024 \\
\href{https://typst.app/universe/package/indenta/0.0.1/}{0.0.1} & May
24, 2024 \\
\end{longtable}

Typst GmbH did not create this package and cannot guarantee correct
functionality of this package or compatibility with any version of the
Typst compiler or app.


\title{typst.app/universe/package/tlacuache-thesis-fc-unam}

\phantomsection\label{banner}
\phantomsection\label{template-thumbnail}
\pandocbounded{\includegraphics[keepaspectratio]{https://packages.typst.org/preview/thumbnails/tlacuache-thesis-fc-unam-0.1.1-small.webp}}

\section{tlacuache-thesis-fc-unam}\label{tlacuache-thesis-fc-unam}

{ 0.1.1 }

Template para escribir una tesis para la facultad de ciencias.

\href{/app?template=tlacuache-thesis-fc-unam&version=0.1.1}{Create
project in app}

\phantomsection\label{readme}
Este es un template para tesis de la facultad de ciencias, en la
Universidad Nacional Autónoma de México (UNAM).

This is a thesis template for the Science Faculty at Universidad
Nacional Autónoma de México (UNAM) based on my thesis.

\subsection{Uso/Usage}\label{usousage}

En la aplicación web de Typst da click en “Start from template� y
busca \texttt{\ tlacuache-thesis-fc-unam\ } .

In the Typst web app simply click “Start from template� on the
dashboard and search for \texttt{\ tlacuache-thesis-fc-unam\ } .

Si estas usando la versión de teminal usa el comando: From the CLI you
can initialize the project with the command:

\begin{Shaded}
\begin{Highlighting}[]
\ExtensionTok{typst}\NormalTok{ init @preview/tlacuache{-}thesis{-}fc{-}unam:0.1.1}
\end{Highlighting}
\end{Shaded}

\subsection{Configuración/Configuration}\label{configuraciuxe3uxb3nconfiguration}

Para configurar tu tesis puedes hacerlo con estas lineas al inicio de tu
archivo principal.

To set the thesis template, you can use the following lines in your main
file.

\begin{Shaded}
\begin{Highlighting}[]
\NormalTok{\#import "@preview/tlacuache{-}thesis{-}fc{-}unam:0.1.1": thesis}

\NormalTok{\#show: thesis.with(}
\NormalTok{  ttitulo: [Titulo],}
\NormalTok{  grado: [Licenciatura],}
\NormalTok{  autor: [Autor],}
\NormalTok{  asesor: [Asesor],}
\NormalTok{  lugar: [Ciudad de México, México],}
\NormalTok{  agno: [\#datetime.today().year()],}
\NormalTok{  bibliography: bibliography("references.bib"),}
\NormalTok{)}

\NormalTok{// Tu tesis va aquí}
\end{Highlighting}
\end{Shaded}

Tambien puedes utilizar estas lineas para crear capítulos con
bibliografía, si deseas crear un pdf solomente para el capítulo.

You could also create a pdf for just a chapter with bibliography, by
using the following lines.

\begin{Shaded}
\begin{Highlighting}[]
\NormalTok{\#import "@preview/tlacuache{-}thesis{-}fc{-}unam:0.1.1": chapter}

\NormalTok{// completamente opcional cargar la bibliografía, compilar el capítulo}
\NormalTok{\#show: chapter.with(bibliography: bibliography("references.bib"))}

\NormalTok{// Tu capítulo va aquí}
\end{Highlighting}
\end{Shaded}

Si quieres crear pdf aun mas cortos, puedes utilizar estas lineas para
crear un pdf solo para el sección de tu capítulo.

You could even create a pdf for just a section of a chapter.

\begin{Shaded}
\begin{Highlighting}[]
\NormalTok{\#import "@preview/tlacuache{-}thesis{-}fc{-}unam:0.1.1": section}

\NormalTok{// completamente opcional cargar la bibliografía, compilar el sección}
\NormalTok{\#show: section.with(bibliography: bibliography("references.bib"))}

\NormalTok{// Tu sección va aquí}
\end{Highlighting}
\end{Shaded}

\href{/app?template=tlacuache-thesis-fc-unam&version=0.1.1}{Create
project in app}

\subsubsection{How to use}\label{how-to-use}

Click the button above to create a new project using this template in
the Typst app.

You can also use the Typst CLI to start a new project on your computer
using this command:

\begin{verbatim}
typst init @preview/tlacuache-thesis-fc-unam:0.1.1
\end{verbatim}

\includesvg[width=0.16667in,height=0.16667in]{/assets/icons/16-copy.svg}

\subsubsection{About}\label{about}

\begin{description}
\tightlist
\item[Author :]
David Valencia, davidalencia@ciencias.unam.mx
\item[License:]
MIT
\item[Current version:]
0.1.1
\item[Last updated:]
April 9, 2024
\item[First released:]
April 9, 2024
\item[Archive size:]
3.14 MB
\href{https://packages.typst.org/preview/tlacuache-thesis-fc-unam-0.1.1.tar.gz}{\pandocbounded{\includesvg[keepaspectratio]{/assets/icons/16-download.svg}}}
\item[Categor y :]
\begin{itemize}
\tightlist
\item[]
\item
  \pandocbounded{\includesvg[keepaspectratio]{/assets/icons/16-mortarboard.svg}}
  \href{https://typst.app/universe/search/?category=thesis}{Thesis}
\end{itemize}
\end{description}

\subsubsection{Where to report issues?}\label{where-to-report-issues}

This template is a project of David Valencia,
davidalencia@ciencias.unam.mx . You can also try to ask for help with
this template on the \href{https://forum.typst.app}{Forum} .

Please report this template to the Typst team using the
\href{https://typst.app/contact}{contact form} if you believe it is a
safety hazard or infringes upon your rights.

\phantomsection\label{versions}
\subsubsection{Version history}\label{version-history}

\begin{longtable}[]{@{}ll@{}}
\toprule\noalign{}
Version & Release Date \\
\midrule\noalign{}
\endhead
\bottomrule\noalign{}
\endlastfoot
0.1.1 & April 9, 2024 \\
\end{longtable}

Typst GmbH did not create this template and cannot guarantee correct
functionality of this template or compatibility with any version of the
Typst compiler or app.


\title{typst.app/universe/package/delimitizer}

\phantomsection\label{banner}
\section{delimitizer}\label{delimitizer}

{ 0.1.0 }

Customize the size of delimiters. Like \textbackslash big,
\textbackslash Big, \textbackslash bigg, \textbackslash Bigg in LaTeX.

\phantomsection\label{readme}
This package lets you customize the size of delimiters in your math
equations. It is useful when you want to make your equations more
readable by increasing the size of certain delimiters. Just like
\texttt{\ \textbackslash{}big\ } , \texttt{\ \textbackslash{}Big\ } ,
\texttt{\ \textbackslash{}bigg\ } , and
\texttt{\ \textbackslash{}Bigg\ } in LaTeX, \texttt{\ delimitizer\ }
provides you with the same functionality in Typst.

\begin{itemize}
\tightlist
\item
  \texttt{\ big(delimiter)\ } : Makes the delimiters bigger than the
  default size.
\item
  \texttt{\ Big(delimiter)\ } : Makes the delimiters bigger than
  \texttt{\ big()\ } .
\item
  \texttt{\ bigg(delimiter)\ } : Makes the delimiters bigger than
  \texttt{\ Big()\ } .
\item
  \texttt{\ Bigg(delimiter)\ } : Makes the delimiters bigger than
  \texttt{\ bigg()\ } .
\item
  \texttt{\ scaled-delimiter(delimiter,\ size)\ } : Scales the
  delimiters by a factor of your choice.
\item
  \texttt{\ paired-delimiter(left,\ right)\ } : Make a short hand for
  paired delimiters. This function returns a closure
  \texttt{\ f(size\ =\ auto:\ auto\ \textbar{}\ none\ \textbar{}\ big\ \textbar{}\ Big\ \textbar{}\ bigg\ \textbar{}\ Bigg\ \textbar{}\ relative,\ content:\ content)\ }
  . The keyed argument \texttt{\ size\ } is optional and defaults to
  \texttt{\ auto\ } . The positional argument \texttt{\ content\ } is
  required.

  \begin{itemize}
  \tightlist
  \item
    when \texttt{\ size\ } is \texttt{\ auto\ } , the size of the
    delimiters is automatically determined.
  \item
    when \texttt{\ size\ } is \texttt{\ none\ } , the size of the
    delimiters is \texttt{\ 1em\ } .
  \item
    when \texttt{\ size\ } is \texttt{\ big\ } / \texttt{\ Big\ } /
    \texttt{\ bigg\ } / \texttt{\ Bigg\ } , the size of the delimiters
    is set to \texttt{\ big\ } / \texttt{\ Big\ } / \texttt{\ bigg\ } /
    \texttt{\ Bigg\ } respectively.
  \item
    when \texttt{\ size\ } is \texttt{\ relative\ } length like
    \texttt{\ 3em\ } or \texttt{\ 150\%\ } , the size of the delimiters
    is scaled by the factor you provide.
  \end{itemize}
\end{itemize}

Example:

\begin{Shaded}
\begin{Highlighting}[]
\NormalTok{\#let parn = paired{-}delimiter("(", ")")}

\NormalTok{$}
\NormalTok{parn(size: bigg,}
\NormalTok{  parn(size: big, (a+b)times (a{-}b))}
\NormalTok{div}
\NormalTok{  parn(size: big, (c+d)times (c{-}d))}
\NormalTok{) + d \textbackslash{} = (a\^{}2{-}b\^{}2) / (c\^{}2{-}d\^{}2)+d}
\NormalTok{$}
\end{Highlighting}
\end{Shaded}

\pandocbounded{\includesvg[keepaspectratio]{https://github.com/typst/packages/raw/main/packages/preview/delimitizer/0.1.0/demo.svg}}

\subsubsection{How to add}\label{how-to-add}

Copy this into your project and use the import as
\texttt{\ delimitizer\ }

\begin{verbatim}
#import "@preview/delimitizer:0.1.0"
\end{verbatim}

\includesvg[width=0.16667in,height=0.16667in]{/assets/icons/16-copy.svg}

Check the docs for
\href{https://typst.app/docs/reference/scripting/\#packages}{more
information on how to import packages} .

\subsubsection{About}\label{about}

\begin{description}
\tightlist
\item[Author :]
Wenzhuo Liu
\item[License:]
MIT
\item[Current version:]
0.1.0
\item[Last updated:]
May 1, 2024
\item[First released:]
May 1, 2024
\item[Archive size:]
2.10 kB
\href{https://packages.typst.org/preview/delimitizer-0.1.0.tar.gz}{\pandocbounded{\includesvg[keepaspectratio]{/assets/icons/16-download.svg}}}
\item[Repository:]
\href{https://github.com/Enter-tainer/delimitizer}{GitHub}
\end{description}

\subsubsection{Where to report issues?}\label{where-to-report-issues}

This package is a project of Wenzhuo Liu . Report issues on
\href{https://github.com/Enter-tainer/delimitizer}{their repository} .
You can also try to ask for help with this package on the
\href{https://forum.typst.app}{Forum} .

Please report this package to the Typst team using the
\href{https://typst.app/contact}{contact form} if you believe it is a
safety hazard or infringes upon your rights.

\phantomsection\label{versions}
\subsubsection{Version history}\label{version-history}

\begin{longtable}[]{@{}ll@{}}
\toprule\noalign{}
Version & Release Date \\
\midrule\noalign{}
\endhead
\bottomrule\noalign{}
\endlastfoot
0.1.0 & May 1, 2024 \\
\end{longtable}

Typst GmbH did not create this package and cannot guarantee correct
functionality of this package or compatibility with any version of the
Typst compiler or app.


\title{typst.app/universe/package/vienna-tech}

\phantomsection\label{banner}
\phantomsection\label{template-thumbnail}
\pandocbounded{\includegraphics[keepaspectratio]{https://packages.typst.org/preview/thumbnails/vienna-tech-0.1.2-small.webp}}

\section{vienna-tech}\label{vienna-tech}

{ 0.1.2 }

An unofficial template for writing thesis at the TU Wien civil- and
environmental engineering faculty.

\href{/app?template=vienna-tech&version=0.1.2}{Create project in app}

\phantomsection\label{readme}
Version 0.1.2

This is a template, modeled after the LaTeX template provided by the
Vienna University of Technology for Engineering Students. It is intended
to be used as a starting point for writing Bachelor’s or Master’s
theses, but can be adapted for other purposes as well. It shall be noted
that this template is not an official template provided by the Vienna
University of Technology, but rather a personal effort to provide a
similar template in a new typesetting system. If you want to checkout
the original templates visit the website of
\href{https://www.tuwien.at/cee/edvlabor/lehre/vorlagen}{TU Wien}

\subsection{Getting Started}\label{getting-started}

These instructions will help you set up the template on the typst web
app.

\begin{Shaded}
\begin{Highlighting}[]
\NormalTok{\#import "@preview/vienna{-}tech:0.1.2": *}

\NormalTok{// Useing the configuration}
\NormalTok{\#show: tuw{-}thesis.with(}
\NormalTok{  title: [Titel of the Thesis],}
\NormalTok{  thesis{-}type: [Bachelorthesis],}
\NormalTok{  lang: "de",}
\NormalTok{  authors: (}
\NormalTok{    (}
\NormalTok{      name: "Firstname Lastname", }
\NormalTok{      email: "email@email.com",}
\NormalTok{      matrnr: "12345678",}
\NormalTok{      date: datetime.today().display("[day] [month repr:long] [year]"),}
\NormalTok{    ),}
\NormalTok{  ),}
\NormalTok{  abstract: [The Abstract of the Thesis],}
\NormalTok{  bibliography: bibliography("bibliography.bib"), }
\NormalTok{  appendix: [The Appendix of the Thesis], }
\NormalTok{    )}
\end{Highlighting}
\end{Shaded}

\subsection{Options}\label{options}

All the available options that are available for the template are listed
below.

\begin{longtable}[]{@{}lll@{}}
\toprule\noalign{}
Parameter & Type & Description \\
\midrule\noalign{}
\endhead
\bottomrule\noalign{}
\endlastfoot
\texttt{\ title\ } & \texttt{\ content\ } & Title of the thesis. \\
\texttt{\ thesis-type\ } & \texttt{\ content\ } & Type of thesis (e.g.,
Bachelor’s thesis, Master’s thesis). \\
\texttt{\ authors\ } & \texttt{\ content\ } ; \texttt{\ string\ } ;
\texttt{\ array\ } & Name of the author(s) as text or array. \\
\texttt{\ abstract\ } & \texttt{\ content\ } & Abstract of the
thesis. \\
\texttt{\ papersize\ } & \texttt{\ string\ } & Paper size (e.g., A4,
Letter). \\
\texttt{\ bibliography\ } & \texttt{\ bibliography\ } & Bibliography
section. \\
\texttt{\ lang\ } & \texttt{\ string\ } & Language of the thesis (e.g.,
“en� for English, “de� for German). \\
\texttt{\ appendix\ } & \texttt{\ content\ } & Appendix of the
thesis. \\
\texttt{\ toc\ } & \texttt{\ bool\ } & Show table of contents (
\texttt{\ true\ } or \texttt{\ false\ } ). \\
\texttt{\ font-size\ } & \texttt{\ length\ } & Font size for the main
text. \\
\texttt{\ main-font\ } & \texttt{\ string\ } ; \texttt{\ array\ } & Main
font as a name or an array of font names. \\
\texttt{\ title-font\ } & \texttt{\ string\ } ; \texttt{\ array\ } &
Font for the title as a name or an array of font names. \\
\texttt{\ raw-font\ } & \texttt{\ string\ } ; \texttt{\ array\ } & Font
for specific text as a name or an array of font names. \\
\texttt{\ title-page\ } & \texttt{\ content\ } & Content of the title
page. \\
\texttt{\ paper-margins\ } & \texttt{\ auto\ } ; \texttt{\ relative\ } ;
\texttt{\ dictionary\ } & Margins of the document. Can be set as
automatic, relative, or defined by a dictionary. \\
\texttt{\ title-hyphenation\ } & \texttt{\ auto\ } ; \texttt{\ bool\ } &
Title hyphenation, either automatic ( \texttt{\ auto\ } ) or manual (
\texttt{\ true\ } or \texttt{\ false\ } ). \\
\end{longtable}

\subsection{Usage}\label{usage}

These instructions will get you a copy of the project up and running on
the typst web app.

\begin{Shaded}
\begin{Highlighting}[]
\ExtensionTok{typst}\NormalTok{ init @preview/vienna{-}tech:0.1.2}
\end{Highlighting}
\end{Shaded}

\subsubsection{Template overview}\label{template-overview}

After setting up the template, you will have the following files:

\begin{itemize}
\tightlist
\item
  \texttt{\ main.typ\ } : the file which is used to compile the document
\item
  \texttt{\ abstract.typ\ } : a file where you can put your abstract
  text
\item
  \texttt{\ appendix.typ\ } : a file where you can put your appendix
  text
\item
  \texttt{\ sections.typ\ } : a file which can include all your contents
\item
  \texttt{\ refs.bib\ } : references
\end{itemize}

\subsection{Contribute to the
template}\label{contribute-to-the-template}

Feel free to contribute to the template by opening a pull request. If
you have any questions, feel free to open an issue.

\href{/app?template=vienna-tech&version=0.1.2}{Create project in app}

\subsubsection{How to use}\label{how-to-use}

Click the button above to create a new project using this template in
the Typst app.

You can also use the Typst CLI to start a new project on your computer
using this command:

\begin{verbatim}
typst init @preview/vienna-tech:0.1.2
\end{verbatim}

\includesvg[width=0.16667in,height=0.16667in]{/assets/icons/16-copy.svg}

\subsubsection{About}\label{about}

\begin{description}
\tightlist
\item[Author :]
Niko Pikall
\item[License:]
Unlicense
\item[Current version:]
0.1.2
\item[Last updated:]
November 6, 2024
\item[First released:]
August 23, 2024
\item[Archive size:]
66.3 kB
\href{https://packages.typst.org/preview/vienna-tech-0.1.2.tar.gz}{\pandocbounded{\includesvg[keepaspectratio]{/assets/icons/16-download.svg}}}
\item[Repository:]
\href{https://github.com/npikall/vienna-tech.git}{GitHub}
\item[Categor y :]
\begin{itemize}
\tightlist
\item[]
\item
  \pandocbounded{\includesvg[keepaspectratio]{/assets/icons/16-mortarboard.svg}}
  \href{https://typst.app/universe/search/?category=thesis}{Thesis}
\end{itemize}
\end{description}

\subsubsection{Where to report issues?}\label{where-to-report-issues}

This template is a project of Niko Pikall . Report issues on
\href{https://github.com/npikall/vienna-tech.git}{their repository} .
You can also try to ask for help with this template on the
\href{https://forum.typst.app}{Forum} .

Please report this template to the Typst team using the
\href{https://typst.app/contact}{contact form} if you believe it is a
safety hazard or infringes upon your rights.

\phantomsection\label{versions}
\subsubsection{Version history}\label{version-history}

\begin{longtable}[]{@{}ll@{}}
\toprule\noalign{}
Version & Release Date \\
\midrule\noalign{}
\endhead
\bottomrule\noalign{}
\endlastfoot
0.1.2 & November 6, 2024 \\
\href{https://typst.app/universe/package/vienna-tech/0.1.1/}{0.1.1} &
August 27, 2024 \\
\href{https://typst.app/universe/package/vienna-tech/0.1.0/}{0.1.0} &
August 23, 2024 \\
\end{longtable}

Typst GmbH did not create this template and cannot guarantee correct
functionality of this template or compatibility with any version of the
Typst compiler or app.


\title{typst.app/universe/package/superb-pci}

\phantomsection\label{banner}
\phantomsection\label{template-thumbnail}
\pandocbounded{\includegraphics[keepaspectratio]{https://packages.typst.org/preview/thumbnails/superb-pci-0.1.0-small.webp}}

\section{superb-pci}\label{superb-pci}

{ 0.1.0 }

A Peer Community In (PCI) and Peer Community Journal (PCJ) template.

\href{/app?template=superb-pci&version=0.1.0}{Create project in app}

\phantomsection\label{readme}
Template for \href{https://peercommunityin.org/}{Peer Community In}
(PCI) submission and \href{https://peercommunityjournal.org/}{Peer
Community Journal} (PCJ) post-recommendation upload.

The template is as close as possible to the LaTeX one.

\subsection{Usage}\label{usage}

To use this template in Typst, simply import it at the top of your
document.

\begin{verbatim}
#import "@preview/superb-pci:0.1.0": *
\end{verbatim}

Alternatively, you can start using this template from the command line
with

\begin{verbatim}
typst init @preview/superb-pci:0.1.0 my-superb-manuscript-dir
\end{verbatim}

or directly in the web app by clicking “Start from template�.

Please see the main Readme about Typst packages
\url{https://github.com/typst/packages} .

\subsection{Configuration}\label{configuration}

This template exports the \texttt{\ pci\ } function with the following
named arguments:

\begin{itemize}
\tightlist
\item
  \texttt{\ title\ } : the paper title
\item
  \texttt{\ authors\ } : array of author dictionaries. Each author must
  have the \texttt{\ name\ } field, and can have the optional fields
  \texttt{\ orcid\ } , and \texttt{\ affiliations\ } .
\item
  \texttt{\ affiliations\ } : array of affiliation dictionaries, each
  with the keys \texttt{\ id\ } and \texttt{\ name\ } . All
  correspondence between authors and affiliations is done manually.
\item
  \texttt{\ abstract\ } : abstract of the paper as content
\item
  \texttt{\ doi\ } : DOI of the paper displayed on the front page
\item
  \texttt{\ keywords\ } : array of keywords displayed on the front page
\item
  \texttt{\ correspondence\ } : corresponding address displayed on the
  front page
\item
  \texttt{\ numbered\_sections\ } : boolean, whether sections should be
  numbered
\item
  \texttt{\ pcj\ } : boolean, provides a way to remove the front page
  and headers/footers for upload to the Peer Community Journal.
  \texttt{\ {[}default:\ false{]}\ }
\end{itemize}

The template will initialize your folder with a sample call to the
\texttt{\ pci\ } function in a show rule and dummy content as an
example. If you want to change an existing project to use this template,
you can add a show rule like this at the top of your file:

\begin{Shaded}
\begin{Highlighting}[]
\NormalTok{\#import "@preview/superb{-}pci:0.1.0": *}

\NormalTok{\#show: pci.with(}
\NormalTok{  title: [Sample for the template, with quite a very long title],}
\NormalTok{  abstract: lorem(200),}
\NormalTok{  authors: (}
\NormalTok{    (}
\NormalTok{      name: "Antoine Lavoisier",}
\NormalTok{      orcid: "0000{-}0000{-}0000{-}0001",}
\NormalTok{      affiliations: "\#,1"}
\NormalTok{    ),}
\NormalTok{    (}
\NormalTok{      name: "Mary P. Curry",}
\NormalTok{      orcid: "0000{-}0000{-}0000{-}0001",}
\NormalTok{      affiliations: "\#,2",}
\NormalTok{    ),}
\NormalTok{    (}
\NormalTok{      name: "Peter Curry",}
\NormalTok{      affiliations: "2",}
\NormalTok{    ),}
\NormalTok{    (}
\NormalTok{      name: "Dick Darlington",}
\NormalTok{      orcid: "0000{-}0000{-}0000{-}0001",}
\NormalTok{      affiliations: "1,3"}
\NormalTok{    ),}
\NormalTok{  ),}
\NormalTok{  affiliations: (}
\NormalTok{   (id: "1", name: "Rue sans aplomb, Paris, France"),}
\NormalTok{   (id: "2", name: "Center for spiced radium experiments, United Kingdom"),}
\NormalTok{   (id: "3", name: "Bruce\textquotesingle{}s Bar and Grill, London (near Susan\textquotesingle{}s)"),}
\NormalTok{   (id: "\#", name: "Equal contributions"),}
\NormalTok{  ),}
\NormalTok{  doi: "https://doi.org/10.5802/fake.doi",}
\NormalTok{  keywords: ("Scientific writing", "Typst", "PCI", "Example"),}
\NormalTok{  correspondence: "a{-}lavois@lead{-}free{-}univ.edu",}
\NormalTok{  numbered\_sections: false,}
\NormalTok{  pcj: false,}
\NormalTok{)}

\NormalTok{// Your content goes here}
\end{Highlighting}
\end{Shaded}

You might also need to use the \texttt{\ table\_note\ } function from
the template.

\subsection{To do}\label{to-do}

Some things that are not straightforward in Typst yet that need to be
added in the futures:

\begin{itemize}
\tightlist
\item
  {[} {]} line numbers
\item
  {[} {]} switch equation numbers to the left
\end{itemize}

\href{/app?template=superb-pci&version=0.1.0}{Create project in app}

\subsubsection{How to use}\label{how-to-use}

Click the button above to create a new project using this template in
the Typst app.

You can also use the Typst CLI to start a new project on your computer
using this command:

\begin{verbatim}
typst init @preview/superb-pci:0.1.0
\end{verbatim}

\includesvg[width=0.16667in,height=0.16667in]{/assets/icons/16-copy.svg}

\subsubsection{About}\label{about}

\begin{description}
\tightlist
\item[Author :]
Alexis Simon
\item[License:]
MIT-0
\item[Current version:]
0.1.0
\item[Last updated:]
April 15, 2024
\item[First released:]
April 15, 2024
\item[Minimum Typst version:]
0.11.0
\item[Archive size:]
170 kB
\href{https://packages.typst.org/preview/superb-pci-0.1.0.tar.gz}{\pandocbounded{\includesvg[keepaspectratio]{/assets/icons/16-download.svg}}}
\item[Repository:]
\href{https://codeberg.org/alxsim/superb-pci}{Codeberg}
\item[Discipline s :]
\begin{itemize}
\tightlist
\item[]
\item
  \href{https://typst.app/universe/search/?discipline=biology}{Biology}
\item
  \href{https://typst.app/universe/search/?discipline=archaeology}{Archaeology}
\end{itemize}
\item[Categor y :]
\begin{itemize}
\tightlist
\item[]
\item
  \pandocbounded{\includesvg[keepaspectratio]{/assets/icons/16-atom.svg}}
  \href{https://typst.app/universe/search/?category=paper}{Paper}
\end{itemize}
\end{description}

\subsubsection{Where to report issues?}\label{where-to-report-issues}

This template is a project of Alexis Simon . Report issues on
\href{https://codeberg.org/alxsim/superb-pci}{their repository} . You
can also try to ask for help with this template on the
\href{https://forum.typst.app}{Forum} .

Please report this template to the Typst team using the
\href{https://typst.app/contact}{contact form} if you believe it is a
safety hazard or infringes upon your rights.

\phantomsection\label{versions}
\subsubsection{Version history}\label{version-history}

\begin{longtable}[]{@{}ll@{}}
\toprule\noalign{}
Version & Release Date \\
\midrule\noalign{}
\endhead
\bottomrule\noalign{}
\endlastfoot
0.1.0 & April 15, 2024 \\
\end{longtable}

Typst GmbH did not create this template and cannot guarantee correct
functionality of this template or compatibility with any version of the
Typst compiler or app.


\title{typst.app/universe/package/tabut}

\phantomsection\label{banner}
\section{tabut}\label{tabut}

{ 1.0.2 }

Display data as tables.

\phantomsection\label{readme}
\emph{Powerful, Simple, Concise}

A Typst plugin for turning data into tables.

\subsection{Outline}\label{outline}

\begin{itemize}
\item
  \href{https://github.com/typst/packages/raw/main/packages/preview/tabut/1.0.2/\#examples}{Examples}

  \begin{itemize}
  \item
    \href{https://github.com/typst/packages/raw/main/packages/preview/tabut/1.0.2/\#input-format-and-creation}{Input
    Format and Creation}
  \item
    \href{https://github.com/typst/packages/raw/main/packages/preview/tabut/1.0.2/\#basic-table}{Basic
    Table}
  \item
    \href{https://github.com/typst/packages/raw/main/packages/preview/tabut/1.0.2/\#table-styling}{Table
    Styling}
  \item
    \href{https://github.com/typst/packages/raw/main/packages/preview/tabut/1.0.2/\#header-formatting}{Header
    Formatting}
  \item
    \href{https://github.com/typst/packages/raw/main/packages/preview/tabut/1.0.2/\#remove-headers}{Remove
    Headers}
  \item
    \href{https://github.com/typst/packages/raw/main/packages/preview/tabut/1.0.2/\#cell-expressions-and-formatting}{Cell
    Expressions and Formatting}
  \item
    \href{https://github.com/typst/packages/raw/main/packages/preview/tabut/1.0.2/\#index}{Index}
  \item
    \href{https://github.com/typst/packages/raw/main/packages/preview/tabut/1.0.2/\#transpose}{Transpose}
  \item
    \href{https://github.com/typst/packages/raw/main/packages/preview/tabut/1.0.2/\#alignment}{Alignment}
  \item
    \href{https://github.com/typst/packages/raw/main/packages/preview/tabut/1.0.2/\#column-width}{Column
    Width}
  \item
    \href{https://github.com/typst/packages/raw/main/packages/preview/tabut/1.0.2/\#get-cells-only}{Get
    Cells Only}
  \item
    \href{https://github.com/typst/packages/raw/main/packages/preview/tabut/1.0.2/\#use-with-tablex}{Use
    with Tablex}
  \end{itemize}
\item
  \href{https://github.com/typst/packages/raw/main/packages/preview/tabut/1.0.2/\#data-operation-examples}{Data
  Operation Examples}

  \begin{itemize}
  \item
    \href{https://github.com/typst/packages/raw/main/packages/preview/tabut/1.0.2/\#csv-data}{CSV
    Data}
  \item
    \href{https://github.com/typst/packages/raw/main/packages/preview/tabut/1.0.2/\#slice}{Slice}
  \item
    \href{https://github.com/typst/packages/raw/main/packages/preview/tabut/1.0.2/\#sorting-and-reversing}{Sorting
    and Reversing}
  \item
    \href{https://github.com/typst/packages/raw/main/packages/preview/tabut/1.0.2/\#filter}{Filter}
  \item
    \href{https://github.com/typst/packages/raw/main/packages/preview/tabut/1.0.2/\#aggregation-using-map-and-sum}{Aggregation
    using Map and Sum}
  \item
    \href{https://github.com/typst/packages/raw/main/packages/preview/tabut/1.0.2/\#grouping}{Grouping}
  \end{itemize}
\item
  \href{https://github.com/typst/packages/raw/main/packages/preview/tabut/1.0.2/\#function-definitions}{Function
  Definitions}

  \begin{itemize}
  \item
    \href{https://github.com/typst/packages/raw/main/packages/preview/tabut/1.0.2/\#tabut}{\texttt{\ tabut\ }}
  \item
    \href{https://github.com/typst/packages/raw/main/packages/preview/tabut/1.0.2/\#tabut-cells}{\texttt{\ tabut-cells\ }}
  \item
    \href{https://github.com/typst/packages/raw/main/packages/preview/tabut/1.0.2/\#rows-to-records}{\texttt{\ rows-to-records\ }}
  \item
    \href{https://github.com/typst/packages/raw/main/packages/preview/tabut/1.0.2/\#records-from-csv}{\texttt{\ records-from-csv\ }}
  \item
    \href{https://github.com/typst/packages/raw/main/packages/preview/tabut/1.0.2/\#group}{\texttt{\ group\ }}
  \end{itemize}
\end{itemize}

\subsection[Input Format and Creation ]{\texorpdfstring{Input Format and
Creation \protect\hypertarget{input-format-and-creation}{}{
}}{Input Format and Creation  }}\label{input-format-and-creation}

The \texttt{\ tabut\ } function takes input in “record� format, an
array of dictionaries, with each dictionary representing a single
“object� or “record�.

In the example below, each record is a listing for an office supply
product.

\begin{Shaded}
\begin{Highlighting}[]
\NormalTok{\#let supplies = (}
\NormalTok{  (name: "Notebook", price: 3.49, quantity: 5),}
\NormalTok{  (name: "Ballpoint Pens", price: 5.99, quantity: 2),}
\NormalTok{  (name: "Printer Paper", price: 6.99, quantity: 3),}
\NormalTok{)}
\end{Highlighting}
\end{Shaded}

\subsection[Basic Table ]{\texorpdfstring{Basic Table
\protect\hypertarget{basic-table}{}{
}}{Basic Table  }}\label{basic-table}

Now create a basic table from the data.

\begin{Shaded}
\begin{Highlighting}[]
\NormalTok{\#import "@preview/tabut:1.0.2": tabut}
\NormalTok{\#import "example{-}data/supplies.typ": supplies}

\NormalTok{\#tabut(}
\NormalTok{  supplies, // the source of the data used to generate the table}
\NormalTok{  ( // column definitions}
\NormalTok{    (}
\NormalTok{      header: [Name], // label, takes content.}
\NormalTok{      func: r =\textgreater{} r.name // generates the cell content.}
\NormalTok{    ), }
\NormalTok{    (header: [Price], func: r =\textgreater{} r.price), }
\NormalTok{    (header: [Quantity], func: r =\textgreater{} r.quantity), }
\NormalTok{  )}
\NormalTok{)}
\end{Highlighting}
\end{Shaded}

\pandocbounded{\includesvg[keepaspectratio]{https://github.com/typst/packages/raw/main/packages/preview/tabut/1.0.2/doc/compiled-snippets/basic.svg}}

\texttt{\ funct\ } takes a function which generates content for a given
cell corrosponding to the defined column for each record. \texttt{\ r\ }
is the record, so \texttt{\ r\ =\textgreater{}\ r.name\ } returns the
\texttt{\ name\ } property of each record in the input data if it has
one.

The philosphy of \texttt{\ tabut\ } is that the display of data should
be simple and clearly defined, therefore each column and it’s content
and formatting should be defined within a single clear column defintion.
One consequence is you can comment out, remove or move, any column
easily, for example:

\begin{Shaded}
\begin{Highlighting}[]
\NormalTok{\#import "@preview/tabut:1.0.2": tabut}
\NormalTok{\#import "example{-}data/supplies.typ": supplies}

\NormalTok{\#tabut(}
\NormalTok{  supplies,}
\NormalTok{  (}
\NormalTok{    (header: [Price], func: r =\textgreater{} r.price), // This column is moved to the front}
\NormalTok{    (header: [Name], func: r =\textgreater{} r.name), }
\NormalTok{    (header: [Name 2], func: r =\textgreater{} r.name), // copied}
\NormalTok{    // (header: [Quantity], func: r =\textgreater{} r.quantity), // removed via comment}
\NormalTok{  )}
\NormalTok{)}
\end{Highlighting}
\end{Shaded}

\pandocbounded{\includesvg[keepaspectratio]{https://github.com/typst/packages/raw/main/packages/preview/tabut/1.0.2/doc/compiled-snippets/rearrange.svg}}

\subsection[Table Styling ]{\texorpdfstring{Table Styling
\protect\hypertarget{table-styling}{}{
}}{Table Styling  }}\label{table-styling}

Any default Table style options can be tacked on and are passed to the
final table function.

\begin{Shaded}
\begin{Highlighting}[]
\NormalTok{\#import "@preview/tabut:1.0.2": tabut}
\NormalTok{\#import "example{-}data/supplies.typ": supplies}

\NormalTok{\#tabut(}
\NormalTok{  supplies,}
\NormalTok{  ( }
\NormalTok{    (header: [Name], func: r =\textgreater{} r.name), }
\NormalTok{    (header: [Price], func: r =\textgreater{} r.price), }
\NormalTok{    (header: [Quantity], func: r =\textgreater{} r.quantity),}
\NormalTok{  ),}
\NormalTok{  fill: (\_, row) =\textgreater{} if calc.odd(row) \{ luma(240) \} else \{ luma(220) \}, }
\NormalTok{  stroke: none}
\NormalTok{)}
\end{Highlighting}
\end{Shaded}

\pandocbounded{\includesvg[keepaspectratio]{https://github.com/typst/packages/raw/main/packages/preview/tabut/1.0.2/doc/compiled-snippets/styling.svg}}

\subsection[Header Formatting ]{\texorpdfstring{Header Formatting
\protect\hypertarget{header-formatting}{}{
}}{Header Formatting  }}\label{header-formatting}

You can pass any content or expression into the header property.

\begin{Shaded}
\begin{Highlighting}[]
\NormalTok{\#import "@preview/tabut:1.0.2": tabut}
\NormalTok{\#import "example{-}data/supplies.typ": supplies}

\NormalTok{\#let fmt(it) = \{}
\NormalTok{  heading(}
\NormalTok{    outlined: false,}
\NormalTok{    upper(it)}
\NormalTok{  )}
\NormalTok{\}}

\NormalTok{\#tabut(}
\NormalTok{  supplies,}
\NormalTok{  ( }
\NormalTok{    (header: fmt([Name]), func: r =\textgreater{} r.name ), }
\NormalTok{    (header: fmt([Price]), func: r =\textgreater{} r.price), }
\NormalTok{    (header: fmt([Quantity]), func: r =\textgreater{} r.quantity), }
\NormalTok{  ),}
\NormalTok{  fill: (\_, row) =\textgreater{} if calc.odd(row) \{ luma(240) \} else \{ luma(220) \}, }
\NormalTok{  stroke: none}
\NormalTok{)}
\end{Highlighting}
\end{Shaded}

\pandocbounded{\includesvg[keepaspectratio]{https://github.com/typst/packages/raw/main/packages/preview/tabut/1.0.2/doc/compiled-snippets/title.svg}}

\subsection[Remove Headers ]{\texorpdfstring{Remove Headers
\protect\hypertarget{remove-headers}{}{
}}{Remove Headers  }}\label{remove-headers}

You can prevent from being generated with the \texttt{\ headers\ }
paramater. This is useful with the \texttt{\ tabut-cells\ } function as
demonstrated in it’s section.

\begin{Shaded}
\begin{Highlighting}[]
\NormalTok{\#import "@preview/tabut:1.0.2": tabut}
\NormalTok{\#import "example{-}data/supplies.typ": supplies}

\NormalTok{\#tabut(}
\NormalTok{  supplies,}
\NormalTok{  (}
\NormalTok{    (header: [*Name*], func: r =\textgreater{} r.name), }
\NormalTok{    (header: [*Price*], func: r =\textgreater{} r.price), }
\NormalTok{    (header: [*Quantity*], func: r =\textgreater{} r.quantity), }
\NormalTok{  ),}
\NormalTok{  headers: false, // Prevents Headers from being generated}
\NormalTok{  fill: (\_, row) =\textgreater{} if calc.odd(row) \{ luma(240) \} else \{ luma(220) \}, }
\NormalTok{  stroke: none,}
\NormalTok{)}
\end{Highlighting}
\end{Shaded}

\pandocbounded{\includesvg[keepaspectratio]{https://github.com/typst/packages/raw/main/packages/preview/tabut/1.0.2/doc/compiled-snippets/no-headers.svg}}

\subsection[Cell Expressions and Formatting ]{\texorpdfstring{Cell
Expressions and Formatting
\protect\hypertarget{cell-expressions-and-formatting}{}{
}}{Cell Expressions and Formatting  }}\label{cell-expressions-and-formatting}

Just like the headers, cell contents can be modified and formatted like
any content in Typst.

\begin{Shaded}
\begin{Highlighting}[]
\NormalTok{\#import "@preview/tabut:1.0.2": tabut}
\NormalTok{\#import "usd.typ": usd}
\NormalTok{\#import "example{-}data/supplies.typ": supplies}

\NormalTok{\#tabut(}
\NormalTok{  supplies,}
\NormalTok{  ( }
\NormalTok{    (header: [*Name*], func: r =\textgreater{} r.name ), }
\NormalTok{    (header: [*Price*], func: r =\textgreater{} usd(r.price)), }
\NormalTok{  ),}
\NormalTok{  fill: (\_, row) =\textgreater{} if calc.odd(row) \{ luma(240) \} else \{ luma(220) \}, }
\NormalTok{  stroke: none}
\NormalTok{)}
\end{Highlighting}
\end{Shaded}

\pandocbounded{\includesvg[keepaspectratio]{https://github.com/typst/packages/raw/main/packages/preview/tabut/1.0.2/doc/compiled-snippets/format.svg}}

You can have the cell content function do calculations on a record
property.

\begin{Shaded}
\begin{Highlighting}[]
\NormalTok{\#import "@preview/tabut:1.0.2": tabut}
\NormalTok{\#import "usd.typ": usd}
\NormalTok{\#import "example{-}data/supplies.typ": supplies}

\NormalTok{\#tabut(}
\NormalTok{  supplies,}
\NormalTok{  ( }
\NormalTok{    (header: [*Name*], func: r =\textgreater{} r.name ), }
\NormalTok{    (header: [*Price*], func: r =\textgreater{} usd(r.price)), }
\NormalTok{    (header: [*Tax*], func: r =\textgreater{} usd(r.price * .2)), }
\NormalTok{    (header: [*Total*], func: r =\textgreater{} usd(r.price * 1.2)), }
\NormalTok{  ),}
\NormalTok{  fill: (\_, row) =\textgreater{} if calc.odd(row) \{ luma(240) \} else \{ luma(220) \}, }
\NormalTok{  stroke: none}
\NormalTok{)}
\end{Highlighting}
\end{Shaded}

\pandocbounded{\includesvg[keepaspectratio]{https://github.com/typst/packages/raw/main/packages/preview/tabut/1.0.2/doc/compiled-snippets/calculation.svg}}

Or even combine multiple record properties, go wild.

\begin{Shaded}
\begin{Highlighting}[]
\NormalTok{\#import "@preview/tabut:1.0.2": tabut}

\NormalTok{\#let employees = (}
\NormalTok{    (id: 3251, first: "Alice", last: "Smith", middle: "Jane"),}
\NormalTok{    (id: 4872, first: "Carlos", last: "Garcia", middle: "Luis"),}
\NormalTok{    (id: 5639, first: "Evelyn", last: "Chen", middle: "Ming")}
\NormalTok{);}

\NormalTok{\#tabut(}
\NormalTok{  employees,}
\NormalTok{  ( }
\NormalTok{    (header: [*ID*], func: r =\textgreater{} r.id ),}
\NormalTok{    (header: [*Full Name*], func: r =\textgreater{} [\#r.first \#r.middle.first(), \#r.last] ),}
\NormalTok{  ),}
\NormalTok{  fill: (\_, row) =\textgreater{} if calc.odd(row) \{ luma(240) \} else \{ luma(220) \}, }
\NormalTok{  stroke: none}
\NormalTok{)}
\end{Highlighting}
\end{Shaded}

\pandocbounded{\includesvg[keepaspectratio]{https://github.com/typst/packages/raw/main/packages/preview/tabut/1.0.2/doc/compiled-snippets/combine.svg}}

\subsection[Index ]{\texorpdfstring{Index \protect\hypertarget{index}{}{
}}{Index  }}\label{index}

\texttt{\ tabut\ } automatically adds an \texttt{\ \_index\ } property
to each record.

\begin{Shaded}
\begin{Highlighting}[]
\NormalTok{\#import "@preview/tabut:1.0.2": tabut}
\NormalTok{\#import "example{-}data/supplies.typ": supplies}

\NormalTok{\#tabut(}
\NormalTok{  supplies,}
\NormalTok{  ( }
\NormalTok{    (header: [*\textbackslash{}\#*], func: r =\textgreater{} r.\_index),}
\NormalTok{    (header: [*Name*], func: r =\textgreater{} r.name ), }
\NormalTok{  ),}
\NormalTok{  fill: (\_, row) =\textgreater{} if calc.odd(row) \{ luma(240) \} else \{ luma(220) \}, }
\NormalTok{  stroke: none}
\NormalTok{)}
\end{Highlighting}
\end{Shaded}

\pandocbounded{\includesvg[keepaspectratio]{https://github.com/typst/packages/raw/main/packages/preview/tabut/1.0.2/doc/compiled-snippets/index.svg}}

You can also prevent the \texttt{\ index\ } property being generated by
setting it to \texttt{\ none\ } , or you can also set an alternate name
of the index property as shown below.

\begin{Shaded}
\begin{Highlighting}[]
\NormalTok{\#import "@preview/tabut:1.0.2": tabut}
\NormalTok{\#import "example{-}data/supplies.typ": supplies}

\NormalTok{\#tabut(}
\NormalTok{  supplies,}
\NormalTok{  ( }
\NormalTok{    (header: [*\textbackslash{}\#*], func: r =\textgreater{} r.index{-}alt ),}
\NormalTok{    (header: [*Name*], func: r =\textgreater{} r.name ), }
\NormalTok{  ),}
\NormalTok{  index: "index{-}alt", // set an aternate name for the automatically generated index property.}
\NormalTok{  fill: (\_, row) =\textgreater{} if calc.odd(row) \{ luma(240) \} else \{ luma(220) \}, }
\NormalTok{  stroke: none}
\NormalTok{)}
\end{Highlighting}
\end{Shaded}

\pandocbounded{\includesvg[keepaspectratio]{https://github.com/typst/packages/raw/main/packages/preview/tabut/1.0.2/doc/compiled-snippets/index-alternate.svg}}

\subsection[Transpose ]{\texorpdfstring{Transpose
\protect\hypertarget{transpose}{}{ }}{Transpose  }}\label{transpose}

This was annoying to implement, and I don’t know when you’d actually
use this, but here.

\begin{Shaded}
\begin{Highlighting}[]
\NormalTok{\#import "@preview/tabut:1.0.2": tabut}
\NormalTok{\#import "usd.typ": usd}
\NormalTok{\#import "example{-}data/supplies.typ": supplies}

\NormalTok{\#tabut(}
\NormalTok{  supplies,}
\NormalTok{  (}
\NormalTok{    (header: [*\textbackslash{}\#*], func: r =\textgreater{} r.\_index),}
\NormalTok{    (header: [*Name*], func: r =\textgreater{} r.name), }
\NormalTok{    (header: [*Price*], func: r =\textgreater{} usd(r.price)), }
\NormalTok{    (header: [*Quantity*], func: r =\textgreater{} r.quantity),}
\NormalTok{  ),}
\NormalTok{  transpose: true,  // set optional name arg \textasciigrave{}transpose\textasciigrave{} to \textasciigrave{}true\textasciigrave{}}
\NormalTok{  fill: (\_, row) =\textgreater{} if calc.odd(row) \{ luma(240) \} else \{ luma(220) \}, }
\NormalTok{  stroke: none}
\NormalTok{)}
\end{Highlighting}
\end{Shaded}

\pandocbounded{\includesvg[keepaspectratio]{https://github.com/typst/packages/raw/main/packages/preview/tabut/1.0.2/doc/compiled-snippets/transpose.svg}}

\subsection[Alignment ]{\texorpdfstring{Alignment
\protect\hypertarget{alignment}{}{ }}{Alignment  }}\label{alignment}

\begin{Shaded}
\begin{Highlighting}[]
\NormalTok{\#import "@preview/tabut:1.0.2": tabut}
\NormalTok{\#import "usd.typ": usd}
\NormalTok{\#import "example{-}data/supplies.typ": supplies}

\NormalTok{\#tabut(}
\NormalTok{  supplies,}
\NormalTok{  ( // Include \textasciigrave{}align\textasciigrave{} as an optional arg to a column def}
\NormalTok{    (header: [*\textbackslash{}\#*], func: r =\textgreater{} r.\_index),}
\NormalTok{    (header: [*Name*], align: right, func: r =\textgreater{} r.name), }
\NormalTok{    (header: [*Price*], align: right, func: r =\textgreater{} usd(r.price)), }
\NormalTok{    (header: [*Quantity*], align: right, func: r =\textgreater{} r.quantity),}
\NormalTok{  ),}
\NormalTok{  fill: (\_, row) =\textgreater{} if calc.odd(row) \{ luma(240) \} else \{ luma(220) \}, }
\NormalTok{  stroke: none}
\NormalTok{)}
\end{Highlighting}
\end{Shaded}

\pandocbounded{\includesvg[keepaspectratio]{https://github.com/typst/packages/raw/main/packages/preview/tabut/1.0.2/doc/compiled-snippets/align.svg}}

You can also define Alignment manually as in the the standard Table
Function.

\begin{Shaded}
\begin{Highlighting}[]
\NormalTok{\#import "@preview/tabut:1.0.2": tabut}
\NormalTok{\#import "usd.typ": usd}
\NormalTok{\#import "example{-}data/supplies.typ": supplies}

\NormalTok{\#tabut(}
\NormalTok{  supplies,}
\NormalTok{  ( }
\NormalTok{    (header: [*\textbackslash{}\#*], func: r =\textgreater{} r.\_index),}
\NormalTok{    (header: [*Name*], func: r =\textgreater{} r.name), }
\NormalTok{    (header: [*Price*], func: r =\textgreater{} usd(r.price)), }
\NormalTok{    (header: [*Quantity*], func: r =\textgreater{} r.quantity),}
\NormalTok{  ),}
\NormalTok{  align: (auto, right, right, right), // Alignment defined as in standard table function}
\NormalTok{  fill: (\_, row) =\textgreater{} if calc.odd(row) \{ luma(240) \} else \{ luma(220) \}, }
\NormalTok{  stroke: none}
\NormalTok{)}
\end{Highlighting}
\end{Shaded}

\pandocbounded{\includesvg[keepaspectratio]{https://github.com/typst/packages/raw/main/packages/preview/tabut/1.0.2/doc/compiled-snippets/align-manual.svg}}

\subsection[Column Width ]{\texorpdfstring{Column Width
\protect\hypertarget{column-width}{}{
}}{Column Width  }}\label{column-width}

\begin{Shaded}
\begin{Highlighting}[]
\NormalTok{\#import "@preview/tabut:1.0.2": tabut}
\NormalTok{\#import "usd.typ": usd}
\NormalTok{\#import "example{-}data/supplies.typ": supplies}

\NormalTok{\#box(}
\NormalTok{  width: 300pt,}
\NormalTok{  tabut(}
\NormalTok{    supplies,}
\NormalTok{    ( // Include \textasciigrave{}width\textasciigrave{} as an optional arg to a column def}
\NormalTok{      (header: [*\textbackslash{}\#*], func: r =\textgreater{} r.\_index),}
\NormalTok{      (header: [*Name*], width: 1fr, func: r =\textgreater{} r.name), }
\NormalTok{      (header: [*Price*], width: 20\%, func: r =\textgreater{} usd(r.price)), }
\NormalTok{      (header: [*Quantity*], width: 1.5in, func: r =\textgreater{} r.quantity),}
\NormalTok{    ),}
\NormalTok{    fill: (\_, row) =\textgreater{} if calc.odd(row) \{ luma(240) \} else \{ luma(220) \}, }
\NormalTok{    stroke: none,}
\NormalTok{  )}
\NormalTok{)}
\end{Highlighting}
\end{Shaded}

\pandocbounded{\includesvg[keepaspectratio]{https://github.com/typst/packages/raw/main/packages/preview/tabut/1.0.2/doc/compiled-snippets/width.svg}}

You can also define Columns manually as in the the standard Table
Function.

\begin{Shaded}
\begin{Highlighting}[]
\NormalTok{\#import "@preview/tabut:1.0.2": tabut}
\NormalTok{\#import "usd.typ": usd}
\NormalTok{\#import "example{-}data/supplies.typ": supplies}

\NormalTok{\#box(}
\NormalTok{  width: 300pt,}
\NormalTok{  tabut(}
\NormalTok{    supplies,}
\NormalTok{    (}
\NormalTok{      (header: [*\textbackslash{}\#*], func: r =\textgreater{} r.\_index),}
\NormalTok{      (header: [*Name*], func: r =\textgreater{} r.name), }
\NormalTok{      (header: [*Price*], func: r =\textgreater{} usd(r.price)), }
\NormalTok{      (header: [*Quantity*], func: r =\textgreater{} r.quantity),}
\NormalTok{    ),}
\NormalTok{    columns: (auto, 1fr, 20\%, 1.5in),  // Columns defined as in standard table}
\NormalTok{    fill: (\_, row) =\textgreater{} if calc.odd(row) \{ luma(240) \} else \{ luma(220) \}, }
\NormalTok{    stroke: none,}
\NormalTok{  )}
\NormalTok{)}
\end{Highlighting}
\end{Shaded}

\pandocbounded{\includesvg[keepaspectratio]{https://github.com/typst/packages/raw/main/packages/preview/tabut/1.0.2/doc/compiled-snippets/width-manual.svg}}

\subsection[Get Cells Only ]{\texorpdfstring{Get Cells Only
\protect\hypertarget{get-cells-only}{}{
}}{Get Cells Only  }}\label{get-cells-only}

\begin{Shaded}
\begin{Highlighting}[]
\NormalTok{\#import "@preview/tabut:1.0.2": tabut{-}cells}
\NormalTok{\#import "usd.typ": usd}
\NormalTok{\#import "example{-}data/supplies.typ": supplies}

\NormalTok{\#tabut{-}cells(}
\NormalTok{  supplies,}
\NormalTok{  ( }
\NormalTok{    (header: [Name], func: r =\textgreater{} r.name), }
\NormalTok{    (header: [Price], func: r =\textgreater{} usd(r.price)), }
\NormalTok{    (header: [Quantity], func: r =\textgreater{} r.quantity),}
\NormalTok{  )}
\NormalTok{)}
\end{Highlighting}
\end{Shaded}

\pandocbounded{\includesvg[keepaspectratio]{https://github.com/typst/packages/raw/main/packages/preview/tabut/1.0.2/doc/compiled-snippets/only-cells.svg}}

\subsection[Use with Tablex ]{\texorpdfstring{Use with Tablex
\protect\hypertarget{use-with-tablex}{}{
}}{Use with Tablex  }}\label{use-with-tablex}

\begin{Shaded}
\begin{Highlighting}[]
\NormalTok{\#import "@preview/tabut:1.0.2": tabut{-}cells}
\NormalTok{\#import "usd.typ": usd}
\NormalTok{\#import "example{-}data/supplies.typ": supplies}

\NormalTok{\#import "@preview/tablex:0.0.8": tablex, rowspanx, colspanx}

\NormalTok{\#tablex(}
\NormalTok{  auto{-}vlines: false,}
\NormalTok{  header{-}rows: 2,}

\NormalTok{  /* {-}{-}{-} header {-}{-}{-} */}
\NormalTok{  rowspanx(2)[*Name*], colspanx(2)[*Price*], (), rowspanx(2)[*Quantity*],}
\NormalTok{  (),                 [*Base*], [*W/Tax*], (),}
\NormalTok{  /* {-}{-}{-}{-}{-}{-}{-}{-}{-}{-}{-}{-}{-}{-} */}

\NormalTok{  ..tabut{-}cells(}
\NormalTok{    supplies,}
\NormalTok{    ( }
\NormalTok{      (header: [], func: r =\textgreater{} r.name), }
\NormalTok{      (header: [], func: r =\textgreater{} usd(r.price)), }
\NormalTok{      (header: [], func: r =\textgreater{} usd(r.price * 1.3)), }
\NormalTok{      (header: [], func: r =\textgreater{} r.quantity),}
\NormalTok{    ),}
\NormalTok{    headers: false}
\NormalTok{  )}
\NormalTok{)}
\end{Highlighting}
\end{Shaded}

\pandocbounded{\includesvg[keepaspectratio]{https://github.com/typst/packages/raw/main/packages/preview/tabut/1.0.2/doc/compiled-snippets/tablex.svg}}

While technically seperate from table display, the following are
examples of how to perform operations on data before it is displayed
with \texttt{\ tabut\ } .

Since \texttt{\ tabut\ } assumes an “array of dictionaries� format,
then most data operations can be performed easily with Typst’s native
array functions. \texttt{\ tabut\ } also provides several functions to
provide additional functionality.

\subsection[CSV Data ]{\texorpdfstring{CSV Data
\protect\hypertarget{csv-data}{}{ }}{CSV Data  }}\label{csv-data}

By default, imported CSV gives a “rows� or “array of arrays�
data format, which can not be directly used by \texttt{\ tabut\ } . To
convert, \texttt{\ tabut\ } includes a function
\texttt{\ rows-to-records\ } demonstrated below.

\begin{Shaded}
\begin{Highlighting}[]
\NormalTok{\#import "@preview/tabut:1.0.2": tabut, rows{-}to{-}records}
\NormalTok{\#import "example{-}data/supplies.typ": supplies}

\NormalTok{\#let titanic = \{}
\NormalTok{  let titanic{-}raw = csv("example{-}data/titanic.csv");}
\NormalTok{  rows{-}to{-}records(}
\NormalTok{    titanic{-}raw.first(), // The header row}
\NormalTok{    titanic{-}raw.slice(1, {-}1), // The rest of the rows}
\NormalTok{  )}
\NormalTok{\}}
\end{Highlighting}
\end{Shaded}

Imported CSV data are all strings, so it’s usefull to convert them to
\texttt{\ int\ } or \texttt{\ float\ } when possible.

\begin{Shaded}
\begin{Highlighting}[]
\NormalTok{\#import "@preview/tabut:1.0.2": tabut, rows{-}to{-}records}
\NormalTok{\#import "example{-}data/supplies.typ": supplies}

\NormalTok{\#let auto{-}type(input) = \{}
\NormalTok{  let is{-}int = (input.match(regex("\^{}{-}?\textbackslash{}d+$")) != none);}
\NormalTok{  if is{-}int \{ return int(input); \}}
\NormalTok{  let is{-}float = (input.match(regex("\^{}{-}?(inf|nan|\textbackslash{}d+|\textbackslash{}d*(\textbackslash{}.\textbackslash{}d+))$")) != none);}
\NormalTok{  if is{-}float \{ return float(input) \}}
\NormalTok{  input}
\NormalTok{\}}

\NormalTok{\#let titanic = \{}
\NormalTok{  let titanic{-}raw = csv("example{-}data/titanic.csv");}
\NormalTok{  rows{-}to{-}records( titanic{-}raw.first(), titanic{-}raw.slice(1, {-}1) )}
\NormalTok{  .map( r =\textgreater{} \{}
\NormalTok{    let new{-}record = (:);}
\NormalTok{    for (k, v) in r.pairs() \{ new{-}record.insert(k, auto{-}type(v)); \}}
\NormalTok{    new{-}record}
\NormalTok{  \})}
\NormalTok{\}}
\end{Highlighting}
\end{Shaded}

\texttt{\ tabut\ } includes a function, \texttt{\ records-from-csv\ } ,
to automatically perform this process.

\begin{Shaded}
\begin{Highlighting}[]
\NormalTok{\#import "@preview/tabut:1.0.2": records{-}from{-}csv}

\NormalTok{\#let titanic = records{-}from{-}csv(csv("example{-}data/titanic.csv"));}
\end{Highlighting}
\end{Shaded}

\subsection[Slice ]{\texorpdfstring{Slice \protect\hypertarget{slice}{}{
}}{Slice  }}\label{slice}

\begin{Shaded}
\begin{Highlighting}[]
\NormalTok{\#import "@preview/tabut:1.0.2": tabut, records{-}from{-}csv}
\NormalTok{\#import "usd.typ": usd}
\NormalTok{\#import "example{-}data/titanic.typ": titanic}

\NormalTok{\#let classes = (}
\NormalTok{  "N/A",}
\NormalTok{  "First", }
\NormalTok{  "Second", }
\NormalTok{  "Third"}
\NormalTok{);}

\NormalTok{\#let titanic{-}head = titanic.slice(0, 5);}

\NormalTok{\#tabut(}
\NormalTok{  titanic{-}head,}
\NormalTok{  ( }
\NormalTok{    (header: [*Name*], func: r =\textgreater{} r.Name), }
\NormalTok{    (header: [*Class*], func: r =\textgreater{} classes.at(r.Pclass)),}
\NormalTok{    (header: [*Fare*], func: r =\textgreater{} usd(r.Fare)), }
\NormalTok{    (header: [*Survived?*], func: r =\textgreater{} ("No", "Yes").at(r.Survived)), }
\NormalTok{  ),}
\NormalTok{  fill: (\_, row) =\textgreater{} if calc.odd(row) \{ luma(240) \} else \{ luma(220) \}, }
\NormalTok{  stroke: none}
\NormalTok{)}
\end{Highlighting}
\end{Shaded}

\pandocbounded{\includesvg[keepaspectratio]{https://github.com/typst/packages/raw/main/packages/preview/tabut/1.0.2/doc/compiled-snippets/slice.svg}}

\subsection[Sorting and Reversing ]{\texorpdfstring{Sorting and
Reversing \protect\hypertarget{sorting-and-reversing}{}{
}}{Sorting and Reversing  }}\label{sorting-and-reversing}

\begin{Shaded}
\begin{Highlighting}[]
\NormalTok{\#import "@preview/tabut:1.0.2": tabut}
\NormalTok{\#import "usd.typ": usd}
\NormalTok{\#import "example{-}data/titanic.typ": titanic, classes}

\NormalTok{\#tabut(}
\NormalTok{  titanic}
\NormalTok{  .sorted(key: r =\textgreater{} r.Fare)}
\NormalTok{  .rev()}
\NormalTok{  .slice(0, 5),}
\NormalTok{  ( }
\NormalTok{    (header: [*Name*], func: r =\textgreater{} r.Name), }
\NormalTok{    (header: [*Class*], func: r =\textgreater{} classes.at(r.Pclass)),}
\NormalTok{    (header: [*Fare*], func: r =\textgreater{} usd(r.Fare)), }
\NormalTok{    (header: [*Survived?*], func: r =\textgreater{} ("No", "Yes").at(r.Survived)), }
\NormalTok{  ),}
\NormalTok{  fill: (\_, row) =\textgreater{} if calc.odd(row) \{ luma(240) \} else \{ luma(220) \}, }
\NormalTok{  stroke: none}
\NormalTok{)}
\end{Highlighting}
\end{Shaded}

\pandocbounded{\includesvg[keepaspectratio]{https://github.com/typst/packages/raw/main/packages/preview/tabut/1.0.2/doc/compiled-snippets/sort.svg}}

\subsection[Filter ]{\texorpdfstring{Filter
\protect\hypertarget{filter}{}{ }}{Filter  }}\label{filter}

\begin{Shaded}
\begin{Highlighting}[]
\NormalTok{\#import "@preview/tabut:1.0.2": tabut}
\NormalTok{\#import "usd.typ": usd}
\NormalTok{\#import "example{-}data/titanic.typ": titanic, classes}

\NormalTok{\#tabut(}
\NormalTok{  titanic}
\NormalTok{  .filter(r =\textgreater{} r.Pclass == 1)}
\NormalTok{  .slice(0, 5),}
\NormalTok{  ( }
\NormalTok{    (header: [*Name*], func: r =\textgreater{} r.Name), }
\NormalTok{    (header: [*Class*], func: r =\textgreater{} classes.at(r.Pclass)),}
\NormalTok{    (header: [*Fare*], func: r =\textgreater{} usd(r.Fare)), }
\NormalTok{    (header: [*Survived?*], func: r =\textgreater{} ("No", "Yes").at(r.Survived)), }
\NormalTok{  ),}
\NormalTok{  fill: (\_, row) =\textgreater{} if calc.odd(row) \{ luma(240) \} else \{ luma(220) \}, }
\NormalTok{  stroke: none}
\NormalTok{)}
\end{Highlighting}
\end{Shaded}

\pandocbounded{\includesvg[keepaspectratio]{https://github.com/typst/packages/raw/main/packages/preview/tabut/1.0.2/doc/compiled-snippets/filter.svg}}

\subsection[Aggregation using Map and Sum ]{\texorpdfstring{Aggregation
using Map and Sum \protect\hypertarget{aggregation-using-map-and-sum}{}{
}}{Aggregation using Map and Sum  }}\label{aggregation-using-map-and-sum}

\begin{Shaded}
\begin{Highlighting}[]
\NormalTok{\#import "usd.typ": usd}
\NormalTok{\#import "example{-}data/titanic.typ": titanic, classes}

\NormalTok{\#table(}
\NormalTok{  columns: (auto, auto),}
\NormalTok{  [*Fare, Total:*], [\#usd(titanic.map(r =\textgreater{} r.Fare).sum())],}
\NormalTok{  [*Fare, Avg:*], [\#usd(titanic.map(r =\textgreater{} r.Fare).sum() / titanic.len())], }
\NormalTok{  stroke: none}
\NormalTok{)}
\end{Highlighting}
\end{Shaded}

\pandocbounded{\includesvg[keepaspectratio]{https://github.com/typst/packages/raw/main/packages/preview/tabut/1.0.2/doc/compiled-snippets/aggregation.svg}}

\subsection[Grouping ]{\texorpdfstring{Grouping
\protect\hypertarget{grouping}{}{ }}{Grouping  }}\label{grouping}

\begin{Shaded}
\begin{Highlighting}[]
\NormalTok{\#import "@preview/tabut:1.0.2": tabut, group}
\NormalTok{\#import "example{-}data/titanic.typ": titanic, classes}

\NormalTok{\#tabut(}
\NormalTok{  group(titanic, r =\textgreater{} r.Pclass),}
\NormalTok{  (}
\NormalTok{    (header: [*Class*], func: r =\textgreater{} classes.at(r.value)), }
\NormalTok{    (header: [*Passengers*], func: r =\textgreater{} r.group.len()), }
\NormalTok{  ),}
\NormalTok{  fill: (\_, row) =\textgreater{} if calc.odd(row) \{ luma(240) \} else \{ luma(220) \}, }
\NormalTok{  stroke: none}
\NormalTok{)}
\end{Highlighting}
\end{Shaded}

\pandocbounded{\includesvg[keepaspectratio]{https://github.com/typst/packages/raw/main/packages/preview/tabut/1.0.2/doc/compiled-snippets/group.svg}}

\begin{Shaded}
\begin{Highlighting}[]
\NormalTok{\#import "@preview/tabut:1.0.2": tabut, group}
\NormalTok{\#import "usd.typ": usd}
\NormalTok{\#import "example{-}data/titanic.typ": titanic, classes}

\NormalTok{\#tabut(}
\NormalTok{  group(titanic, r =\textgreater{} r.Pclass),}
\NormalTok{  (}
\NormalTok{    (header: [*Class*], func: r =\textgreater{} classes.at(r.value)), }
\NormalTok{    (header: [*Total Fare*], func: r =\textgreater{} usd(r.group.map(r =\textgreater{} r.Fare).sum())), }
\NormalTok{    (}
\NormalTok{      header: [*Avg Fare*], }
\NormalTok{      func: r =\textgreater{} usd(r.group.map(r =\textgreater{} r.Fare).sum() / r.group.len())}
\NormalTok{    ), }
\NormalTok{  ),}
\NormalTok{  fill: (\_, row) =\textgreater{} if calc.odd(row) \{ luma(240) \} else \{ luma(220) \}, }
\NormalTok{  stroke: none}
\NormalTok{)}
\end{Highlighting}
\end{Shaded}

\pandocbounded{\includesvg[keepaspectratio]{https://github.com/typst/packages/raw/main/packages/preview/tabut/1.0.2/doc/compiled-snippets/group-aggregation.svg}}

\subsection[\texttt{\ tabut\ } ]{\texorpdfstring{\texttt{\ tabut\ }
\protect\hypertarget{tabut}{}{ }}{ tabut   }}\label{tabut-1}

Takes data and column definitions and outputs a table.

\begin{Shaded}
\begin{Highlighting}[]
\NormalTok{tabut(}
\NormalTok{  data{-}raw, }
\NormalTok{  colDefs, }
\NormalTok{  columns: auto,}
\NormalTok{  align: auto,}
\NormalTok{  index: "\_index",}
\NormalTok{  transpose: false,}
\NormalTok{  headers: true,}
\NormalTok{  ..tableArgs}
\NormalTok{) {-}\textgreater{} content}
\end{Highlighting}
\end{Shaded}

\subsubsection{Parameters}\label{parameters}

\texttt{\ data-raw\ }\strut \\
This is the raw data that will be used to generate the table. The data
is expected to be in an array of dictionaries, where each dictionary
represents a single record or object.

\texttt{\ colDefs\ }\strut \\
These are the column definitions. An array of dictionaries, each
representing column definition. Must include the properties
\texttt{\ header\ } and a \texttt{\ func\ } . \texttt{\ header\ }
expects content, and specifies the label of the column.
\texttt{\ func\ } expects a function, the function takes a record
dictionary as input and returns the value to be displayed in the cell
corresponding to that record and column. There are also two optional
properties; \texttt{\ align\ } sets the alignment of the content within
the cells of the column, \texttt{\ width\ } sets the width of the
column.

\texttt{\ columns\ }\strut \\
(optional, default: \texttt{\ auto\ } ) Specifies the column widths. If
set to \texttt{\ auto\ } , the function automatically generates column
widths by each column’s column definition. Otherwise functions exactly
the \texttt{\ columns\ } paramater of the standard Typst
\texttt{\ table\ } function. Unlike the \texttt{\ tabut-cells\ } setting
this to \texttt{\ none\ } will break.

\texttt{\ align\ }\strut \\
(optional, default: \texttt{\ auto\ } ) Specifies the column alignment.
If set to \texttt{\ auto\ } , the function automatically generates
column alignment by each column’s column definition. If set to
\texttt{\ none\ } no \texttt{\ align\ } property is added to the output
arg. Otherwise functions exactly the \texttt{\ align\ } paramater of the
standard Typst \texttt{\ table\ } function.

\texttt{\ index\ }\strut \\
(optional, default: \texttt{\ "\_index"\ } ) Specifies the property name
for the index of each record. By default, an \texttt{\ \_index\ }
property is automatically added to each record. If set to
\texttt{\ none\ } , no index property is added.

\texttt{\ transpose\ }\strut \\
(optional, default: \texttt{\ false\ } ) If set to \texttt{\ true\ } ,
transposes the table, swapping rows and columns.

\texttt{\ headers\ }\strut \\
(optional, default: \texttt{\ true\ } ) Determines whether headers
should be included in the output. If set to \texttt{\ false\ } , headers
are not generated.

\texttt{\ tableArgs\ }\strut \\
(optional) Any additional arguments are passed to the \texttt{\ table\ }
function, can be used for styling or anything else.

\subsection[\texttt{\ tabut-cells\ }
]{\texorpdfstring{\texttt{\ tabut-cells\ }
\protect\hypertarget{tabut-cells}{}{
}}{ tabut-cells   }}\label{tabut-cells}

The \texttt{\ tabut-cells\ } function functions as \texttt{\ tabut\ } ,
but returns \texttt{\ arguments\ } for use in either the standard
\texttt{\ table\ } function or other tools such as \texttt{\ tablex\ } .
If you just want the array of cells, use the \texttt{\ pos\ } function
on the returned value, ex \texttt{\ tabut-cells(...).pos\ } .

\texttt{\ tabut-cells\ } is particularly useful when you need to
generate only the cell contents of a table or when these cells need to
be passed to another function for further processing or customization.

\subsubsection{Function Signature}\label{function-signature}

\begin{Shaded}
\begin{Highlighting}[]
\NormalTok{tabut{-}cells(}
\NormalTok{  data{-}raw, }
\NormalTok{  colDefs, }
\NormalTok{  columns: auto,}
\NormalTok{  align: auto,}
\NormalTok{  index: "\_index",}
\NormalTok{  transpose: false,}
\NormalTok{  headers: true,}
\NormalTok{) {-}\textgreater{} arguments}
\end{Highlighting}
\end{Shaded}

\subsubsection{Parameters}\label{parameters-1}

\texttt{\ data-raw\ }\strut \\
This is the raw data that will be used to generate the table. The data
is expected to be in an array of dictionaries, where each dictionary
represents a single record or object.

\texttt{\ colDefs\ }\strut \\
These are the column definitions. An array of dictionaries, each
representing column definition. Must include the properties
\texttt{\ header\ } and a \texttt{\ func\ } . \texttt{\ header\ }
expects content, and specifies the label of the column.
\texttt{\ func\ } expects a function, the function takes a record
dictionary as input and returns the value to be displayed in the cell
corresponding to that record and column. There are also two optional
properties; \texttt{\ align\ } sets the alignment of the content within
the cells of the column, \texttt{\ width\ } sets the width of the
column.

\texttt{\ columns\ }\strut \\
(optional, default: \texttt{\ auto\ } ) Specifies the column widths. If
set to \texttt{\ auto\ } , the function automatically generates column
widths by each column’s column definition. If set to \texttt{\ none\ }
no \texttt{\ column\ } property is added to the output arg. Otherwise
functions exactly the \texttt{\ columns\ } paramater of the standard
typst \texttt{\ table\ } function.

\texttt{\ align\ }\strut \\
(optional, default: \texttt{\ auto\ } ) Specifies the column alignment.
If set to \texttt{\ auto\ } , the function automatically generates
column alignment by each column’s column definition. If set to
\texttt{\ none\ } no \texttt{\ align\ } property is added to the output
arg. Otherwise functions exactly the \texttt{\ align\ } paramater of the
standard typst \texttt{\ table\ } function.

\texttt{\ index\ }\strut \\
(optional, default: \texttt{\ "\_index"\ } ) Specifies the property name
for the index of each record. By default, an \texttt{\ \_index\ }
property is automatically added to each record. If set to
\texttt{\ none\ } , no index property is added.

\texttt{\ transpose\ }\strut \\
(optional, default: \texttt{\ false\ } ) If set to \texttt{\ true\ } ,
transposes the table, swapping rows and columns.

\texttt{\ headers\ }\strut \\
(optional, default: \texttt{\ true\ } ) Determines whether headers
should be included in the output. If set to \texttt{\ false\ } , headers
are not generated.

\subsection[\texttt{\ records-from-csv\ }
]{\texorpdfstring{\texttt{\ records-from-csv\ }
\protect\hypertarget{records-from-csv}{}{
}}{ records-from-csv   }}\label{records-from-csv}

Automatically converts a CSV data into an array of records.

\begin{Shaded}
\begin{Highlighting}[]
\NormalTok{records{-}from{-}csv(}
\NormalTok{  data}
\NormalTok{) {-}\textgreater{} array}
\end{Highlighting}
\end{Shaded}

\subsubsection{Parameters}\label{parameters-2}

\texttt{\ data\ }\strut \\
The CSV data that needs to be converted, this can be obtained using the
native \texttt{\ csv\ } function, like
\texttt{\ records-from-csv(csv(file-path))\ } .

This function simplifies the process of converting CSV data into a
format compatible with \texttt{\ tabut\ } . It reads the CSV data,
extracts the headers, and converts each row into a dictionary, using the
headers as keys.

It also automatically converts data into floats or integers when
possible.

\subsection[\texttt{\ rows-to-records\ }
]{\texorpdfstring{\texttt{\ rows-to-records\ }
\protect\hypertarget{rows-to-records}{}{
}}{ rows-to-records   }}\label{rows-to-records}

Converts rows of data into an array of records based on specified
headers.

This function is useful for converting data in a “rows� format
(commonly found in CSV files) into an array of dictionaries format,
which is required for \texttt{\ tabut\ } and allows easy data processing
using the built in array functions.

\begin{Shaded}
\begin{Highlighting}[]
\NormalTok{rows{-}to{-}records(}
\NormalTok{  headers, }
\NormalTok{  rows, }
\NormalTok{  default: none}
\NormalTok{) {-}\textgreater{} array}
\end{Highlighting}
\end{Shaded}

\subsubsection{Parameters}\label{parameters-3}

\texttt{\ headers\ }\strut \\
An array representing the headers of the table. Each item in this array
corresponds to a column header.

\texttt{\ rows\ }\strut \\
An array of arrays, each representing a row of data. Each sub-array
contains the cell data for a corresponding row.

\texttt{\ default\ }\strut \\
(optional, default: \texttt{\ none\ } ) A default value to use when a
cell is empty or there is an error.

\subsection[\texttt{\ group\ } ]{\texorpdfstring{\texttt{\ group\ }
\protect\hypertarget{group}{}{ }}{ group   }}\label{group}

Groups data based on a specified function and returns an array of
grouped records.

\begin{Shaded}
\begin{Highlighting}[]
\NormalTok{group(}
\NormalTok{  data, }
\NormalTok{  function}
\NormalTok{) {-}\textgreater{} array}
\end{Highlighting}
\end{Shaded}

\subsubsection{Parameters}\label{parameters-4}

\texttt{\ data\ }\strut \\
An array of dictionaries. Each dictionary represents a single record or
object.

\texttt{\ function\ }\strut \\
A function that takes a record as input and returns a value based on
which the grouping is to be performed.

This function iterates over each record in the \texttt{\ data\ } ,
applies the \texttt{\ function\ } to determine the grouping value, and
organizes the records into groups based on this value. Each group record
is represented as a dictionary with two properties: \texttt{\ value\ }
(the result of the grouping function) and \texttt{\ group\ } (an array
of records belonging to this group).

In the context of \texttt{\ tabut\ } , the \texttt{\ group\ } function
is particularly useful for creating summary tables where records need to
be categorized and aggregated based on certain criteria, such as
calculating total or average values for each group.

\subsubsection{How to add}\label{how-to-add}

Copy this into your project and use the import as \texttt{\ tabut\ }

\begin{verbatim}
#import "@preview/tabut:1.0.2"
\end{verbatim}

\includesvg[width=0.16667in,height=0.16667in]{/assets/icons/16-copy.svg}

Check the docs for
\href{https://typst.app/docs/reference/scripting/\#packages}{more
information on how to import packages} .

\subsubsection{About}\label{about}

\begin{description}
\tightlist
\item[Author :]
\href{https://github.com/Amelia-Mowers}{Amelia Mowers}
\item[License:]
MIT
\item[Current version:]
1.0.2
\item[Last updated:]
April 16, 2024
\item[First released:]
January 29, 2024
\item[Archive size:]
9.40 kB
\href{https://packages.typst.org/preview/tabut-1.0.2.tar.gz}{\pandocbounded{\includesvg[keepaspectratio]{/assets/icons/16-download.svg}}}
\item[Repository:]
\href{https://github.com/Amelia-Mowers/typst-tabut}{GitHub}
\end{description}

\subsubsection{Where to report issues?}\label{where-to-report-issues}

This package is a project of Amelia Mowers . Report issues on
\href{https://github.com/Amelia-Mowers/typst-tabut}{their repository} .
You can also try to ask for help with this package on the
\href{https://forum.typst.app}{Forum} .

Please report this package to the Typst team using the
\href{https://typst.app/contact}{contact form} if you believe it is a
safety hazard or infringes upon your rights.

\phantomsection\label{versions}
\subsubsection{Version history}\label{version-history}

\begin{longtable}[]{@{}ll@{}}
\toprule\noalign{}
Version & Release Date \\
\midrule\noalign{}
\endhead
\bottomrule\noalign{}
\endlastfoot
1.0.2 & April 16, 2024 \\
\href{https://typst.app/universe/package/tabut/1.0.1/}{1.0.1} & January
31, 2024 \\
\href{https://typst.app/universe/package/tabut/1.0.0/}{1.0.0} & January
29, 2024 \\
\end{longtable}

Typst GmbH did not create this package and cannot guarantee correct
functionality of this package or compatibility with any version of the
Typst compiler or app.


\title{typst.app/universe/package/cetz-plot}

\phantomsection\label{banner}
\section{cetz-plot}\label{cetz-plot}

{ 0.1.0 }

Plotting module for CeTZ.

\phantomsection\label{readme}
CeTZ-Plot is a library that adds plots and charts to
\href{https://github.com/cetz-package/cetz}{CeTZ} , a library for
drawing with \href{https://typst.app/}{Typst} .

CeTZ-Plot requires CeTZ version ≥ 0.3.1!

\subsection{Examples}\label{examples}

\begin{longtable}[]{@{}lll@{}}
\toprule\noalign{}
\endhead
\bottomrule\noalign{}
\endlastfoot
\href{https://github.com/typst/packages/raw/main/packages/preview/cetz-plot/0.1.0/gallery/line.typ}{\includegraphics[width=2.60417in,height=\textheight,keepaspectratio]{https://github.com/typst/packages/raw/main/packages/preview/cetz-plot/0.1.0/gallery/line.png}}
&
\href{https://github.com/typst/packages/raw/main/packages/preview/cetz-plot/0.1.0/gallery/piechart.typ}{\includegraphics[width=2.60417in,height=\textheight,keepaspectratio]{https://github.com/typst/packages/raw/main/packages/preview/cetz-plot/0.1.0/gallery/piechart.png}}
&
\href{https://github.com/typst/packages/raw/main/packages/preview/cetz-plot/0.1.0/gallery/barchart.typ}{\includegraphics[width=2.60417in,height=\textheight,keepaspectratio]{https://github.com/typst/packages/raw/main/packages/preview/cetz-plot/0.1.0/gallery/barchart.png}} \\
Plot & Pie Chart & Clustered Barchart \\
\end{longtable}

\emph{Click on the example image to jump to the code.}

\subsection{Usage}\label{usage}

For information, see the
\href{https://github.com/cetz-package/cetz-plot/blob/stable/manual.pdf?raw=true}{manual
(stable)} .

To use this package, simply add the following code to your document:

\begin{verbatim}
#import "@preview/cetz:0.3.1"
#import "@preview/cetz-plot:0.1.0": plot, chart

#cetz.canvas({
  // Your plot/chart code goes here
})
\end{verbatim}

\subsection{Installing}\label{installing}

To install the CeTZ-Plot package under
\href{https://github.com/typst/packages?tab=readme-ov-file\#local-packages}{your
local typst package dir} you can use the \texttt{\ install\ } script
from the repository.

\subsubsection{Just}\label{just}

This project uses \href{https://github.com/casey/just}{just} , a handy
command runner.

You can run all commands without having \texttt{\ just\ } installed,
just have a look into the \texttt{\ justfile\ } . To install
\texttt{\ just\ } on your system, use your systems package manager. On
Windows, \href{https://doc.rust-lang.org/cargo/}{Cargo} (
\texttt{\ cargo\ install\ just\ } ),
\href{https://chocolatey.org/}{Chocolatey} (
\texttt{\ choco\ install\ just\ } ) and
\href{https://just.systems/man/en/chapter_4.html}{some other sources}
can be used. You need to run it from a \texttt{\ sh\ } compatible shell
on Windows (e.g git-bash).

\subsection{Testing}\label{testing}

This package comes with some unit tests under the \texttt{\ tests\ }
directory. To run all tests you can run the \texttt{\ just\ test\ }
target. You need to have
\href{https://github.com/tingerrr/typst-test/}{\texttt{\ typst-test\ }}
in your \texttt{\ PATH\ } :
\texttt{\ cargo\ install\ typst-test\ -\/-git\ https://github.com/tingerrr/typst-test\ }
.

\subsubsection{How to add}\label{how-to-add}

Copy this into your project and use the import as \texttt{\ cetz-plot\ }

\begin{verbatim}
#import "@preview/cetz-plot:0.1.0"
\end{verbatim}

\includesvg[width=0.16667in,height=0.16667in]{/assets/icons/16-copy.svg}

Check the docs for
\href{https://typst.app/docs/reference/scripting/\#packages}{more
information on how to import packages} .

\subsubsection{About}\label{about}

\begin{description}
\tightlist
\item[Author s :]
\href{https://github.com/johannes-wolf}{Johannes Wolf} \&
\href{https://github.com/fenjalien}{fenjalien}
\item[License:]
LGPL-3.0-or-later
\item[Current version:]
0.1.0
\item[Last updated:]
October 21, 2024
\item[First released:]
October 21, 2024
\item[Minimum Typst version:]
0.11.0
\item[Archive size:]
43.9 kB
\href{https://packages.typst.org/preview/cetz-plot-0.1.0.tar.gz}{\pandocbounded{\includesvg[keepaspectratio]{/assets/icons/16-download.svg}}}
\item[Repository:]
\href{https://github.com/cetz-package/cetz-plot}{GitHub}
\item[Categor y :]
\begin{itemize}
\tightlist
\item[]
\item
  \pandocbounded{\includesvg[keepaspectratio]{/assets/icons/16-chart.svg}}
  \href{https://typst.app/universe/search/?category=visualization}{Visualization}
\end{itemize}
\end{description}

\subsubsection{Where to report issues?}\label{where-to-report-issues}

This package is a project of Johannes Wolf and fenjalien . Report issues
on \href{https://github.com/cetz-package/cetz-plot}{their repository} .
You can also try to ask for help with this package on the
\href{https://forum.typst.app}{Forum} .

Please report this package to the Typst team using the
\href{https://typst.app/contact}{contact form} if you believe it is a
safety hazard or infringes upon your rights.

\phantomsection\label{versions}
\subsubsection{Version history}\label{version-history}

\begin{longtable}[]{@{}ll@{}}
\toprule\noalign{}
Version & Release Date \\
\midrule\noalign{}
\endhead
\bottomrule\noalign{}
\endlastfoot
0.1.0 & October 21, 2024 \\
\end{longtable}

Typst GmbH did not create this package and cannot guarantee correct
functionality of this package or compatibility with any version of the
Typst compiler or app.


\title{typst.app/universe/package/autofletcher}

\phantomsection\label{banner}
\section{autofletcher}\label{autofletcher}

{ 0.1.1 }

Easier diagrams with fletcher

\phantomsection\label{readme}
This small module provides functions to (sort of) abstract away manual
placement of coordinates.

See the
\href{https://raw.githubusercontent.com/3akev/autofletcher/main/manual.pdf}{manual}
for usage examples.

\subsection{Credits}\label{credits}

\href{https://github.com/Jollywatt/typst-fletcher}{fletcher}

\subsubsection{How to add}\label{how-to-add}

Copy this into your project and use the import as
\texttt{\ autofletcher\ }

\begin{verbatim}
#import "@preview/autofletcher:0.1.1"
\end{verbatim}

\includesvg[width=0.16667in,height=0.16667in]{/assets/icons/16-copy.svg}

Check the docs for
\href{https://typst.app/docs/reference/scripting/\#packages}{more
information on how to import packages} .

\subsubsection{About}\label{about}

\begin{description}
\tightlist
\item[Author :]
\href{https://github.com/3akev}{3akev}
\item[License:]
MIT
\item[Current version:]
0.1.1
\item[Last updated:]
May 23, 2024
\item[First released:]
May 14, 2024
\item[Archive size:]
2.67 kB
\href{https://packages.typst.org/preview/autofletcher-0.1.1.tar.gz}{\pandocbounded{\includesvg[keepaspectratio]{/assets/icons/16-download.svg}}}
\item[Repository:]
\href{https://github.com/3akev/autofletcher}{GitHub}
\item[Categor y :]
\begin{itemize}
\tightlist
\item[]
\item
  \pandocbounded{\includesvg[keepaspectratio]{/assets/icons/16-chart.svg}}
  \href{https://typst.app/universe/search/?category=visualization}{Visualization}
\end{itemize}
\end{description}

\subsubsection{Where to report issues?}\label{where-to-report-issues}

This package is a project of 3akev . Report issues on
\href{https://github.com/3akev/autofletcher}{their repository} . You can
also try to ask for help with this package on the
\href{https://forum.typst.app}{Forum} .

Please report this package to the Typst team using the
\href{https://typst.app/contact}{contact form} if you believe it is a
safety hazard or infringes upon your rights.

\phantomsection\label{versions}
\subsubsection{Version history}\label{version-history}

\begin{longtable}[]{@{}ll@{}}
\toprule\noalign{}
Version & Release Date \\
\midrule\noalign{}
\endhead
\bottomrule\noalign{}
\endlastfoot
0.1.1 & May 23, 2024 \\
\href{https://typst.app/universe/package/autofletcher/0.1.0/}{0.1.0} &
May 14, 2024 \\
\end{longtable}

Typst GmbH did not create this package and cannot guarantee correct
functionality of this package or compatibility with any version of the
Typst compiler or app.


\title{typst.app/universe/package/accelerated-jacow}

\phantomsection\label{banner}
\phantomsection\label{template-thumbnail}
\pandocbounded{\includegraphics[keepaspectratio]{https://packages.typst.org/preview/thumbnails/accelerated-jacow-0.1.1-small.webp}}

\section{accelerated-jacow}\label{accelerated-jacow}

{ 0.1.1 }

Paper template for conference proceedings in accelerator physics

\href{/app?template=accelerated-jacow&version=0.1.1}{Create project in
app}

\phantomsection\label{readme}
\href{https://github.com/eltos/accelerated-jacow}{\pandocbounded{\includegraphics[keepaspectratio]{https://img.shields.io/badge/GitHub\%20Repo-eltos\%2Faccelerated--jacow-lightgray}}}
\href{https://typst.app/universe/package/accelerated-jacow}{\pandocbounded{\includegraphics[keepaspectratio]{https://img.shields.io/badge/Typst\%20Universe-accelerated--jacow-\%23219dac}}}

Paper template for conference proceedings in accelerator physics.

Based on the JACoW guide for preparation of papers available at
\url{https://jacow.org/} .

\subsection{Usage}\label{usage}

\subsubsection{Typst web app}\label{typst-web-app}

In the \href{https://typst.app/}{typst web app} select “start from
template� and search for the accelerated-jacow template.

\subsubsection{Local installation}\label{local-installation}

Run these commands inside your terminal:

\begin{Shaded}
\begin{Highlighting}[]
\ExtensionTok{typst}\NormalTok{ init @preview/accelerated{-}jacow}
\BuiltInTok{cd}\NormalTok{ accelerated{-}jacow}
\ExtensionTok{typst}\NormalTok{ watch paper.typ}
\end{Highlighting}
\end{Shaded}

If you don’t yet have the \emph{TeX Gyre Termes} font family, you can
install it with \texttt{\ sudo\ apt\ install\ tex-gyre\ } .

\pandocbounded{\includegraphics[keepaspectratio]{https://github.com/typst/packages/raw/main/packages/preview/accelerated-jacow/0.1.1/thumbnail.webp}}

\subsection{Licence}\label{licence}

Files inside the template folder are licensed under MIT-0. You can use
them without restrictions.\\
The citation style (CSL) file is based on the IEEE style and licensed
under the \href{https://creativecommons.org/licenses/by-sa/4.0/}{CC BY
SA 4.0} compatible
\href{https://www.gnu.org/licenses/gpl-3.0.html}{GPLv3} license.\\
All other files are licensed under
\href{https://www.gnu.org/licenses/gpl-3.0.html}{GPLv3} .

\href{/app?template=accelerated-jacow&version=0.1.1}{Create project in
app}

\subsubsection{How to use}\label{how-to-use}

Click the button above to create a new project using this template in
the Typst app.

You can also use the Typst CLI to start a new project on your computer
using this command:

\begin{verbatim}
typst init @preview/accelerated-jacow:0.1.1
\end{verbatim}

\includesvg[width=0.16667in,height=0.16667in]{/assets/icons/16-copy.svg}

\subsubsection{About}\label{about}

\begin{description}
\tightlist
\item[Author :]
\href{https://github.com/eltos}{Philipp Niedermayer}
\item[License:]
GPL-3.0-only AND MIT-0
\item[Current version:]
0.1.1
\item[Last updated:]
November 21, 2024
\item[First released:]
October 30, 2024
\item[Archive size:]
27.1 kB
\href{https://packages.typst.org/preview/accelerated-jacow-0.1.1.tar.gz}{\pandocbounded{\includesvg[keepaspectratio]{/assets/icons/16-download.svg}}}
\item[Repository:]
\href{https://github.com/eltos/accelerated-jacow/}{GitHub}
\item[Discipline s :]
\begin{itemize}
\tightlist
\item[]
\item
  \href{https://typst.app/universe/search/?discipline=physics}{Physics}
\item
  \href{https://typst.app/universe/search/?discipline=engineering}{Engineering}
\end{itemize}
\item[Categor y :]
\begin{itemize}
\tightlist
\item[]
\item
  \pandocbounded{\includesvg[keepaspectratio]{/assets/icons/16-atom.svg}}
  \href{https://typst.app/universe/search/?category=paper}{Paper}
\end{itemize}
\end{description}

\subsubsection{Where to report issues?}\label{where-to-report-issues}

This template is a project of Philipp Niedermayer . Report issues on
\href{https://github.com/eltos/accelerated-jacow/}{their repository} .
You can also try to ask for help with this template on the
\href{https://forum.typst.app}{Forum} .

Please report this template to the Typst team using the
\href{https://typst.app/contact}{contact form} if you believe it is a
safety hazard or infringes upon your rights.

\phantomsection\label{versions}
\subsubsection{Version history}\label{version-history}

\begin{longtable}[]{@{}ll@{}}
\toprule\noalign{}
Version & Release Date \\
\midrule\noalign{}
\endhead
\bottomrule\noalign{}
\endlastfoot
0.1.1 & November 21, 2024 \\
\href{https://typst.app/universe/package/accelerated-jacow/0.1.0/}{0.1.0}
& October 30, 2024 \\
\end{longtable}

Typst GmbH did not create this template and cannot guarantee correct
functionality of this template or compatibility with any version of the
Typst compiler or app.


\title{typst.app/universe/package/ctheorems}

\phantomsection\label{banner}
\section{ctheorems}\label{ctheorems}

{ 1.1.3 }

Numbered theorem environments for typst.

\phantomsection\label{readme}
An implementation of numbered theorem environments in
\href{https://github.com/typst/typst}{typst} . Import using

\begin{Shaded}
\begin{Highlighting}[]
\NormalTok{\#import "@preview/ctheorems:1.1.3": *}
\NormalTok{\#show: thmrules}
\end{Highlighting}
\end{Shaded}

\subsubsection{Features}\label{features}

\begin{itemize}
\tightlist
\item
  Numbered theorem environments can be created and customized.
\item
  Environments can share the same counter, via same
  \texttt{\ identifier\ } s.
\item
  Environment counters can be \emph{attached} (just as subheadings are
  attached to headings) to other environments, headings, or keep a
  global count via \texttt{\ base\ } .
\item
  The depth of a counter can be manually set, via
  \texttt{\ base\_level\ } .
\item
  Environments can be \texttt{\ \textless{}label\textgreater{}\ }
  \textquotesingle d and \texttt{\ @reference\ } ’d.
\item
  Awesome presets (coming soon!)
\end{itemize}

\subsection{Manual and Examples}\label{manual-and-examples}

Get acquainted with \texttt{\ ctheorems\ } by checking out the minimal
example below!

You can read the
\href{https://github.com/typst/packages/raw/main/packages/preview/ctheorems/1.1.3/assets/manual.pdf}{manual}
for a full walkthrough of functionality offered by this module; flick
through
\href{https://github.com/typst/packages/raw/main/packages/preview/ctheorems/1.1.3/assets/manual_examples.pdf}{manual\_examples}
to just see the examples.

\pandocbounded{\includegraphics[keepaspectratio]{https://github.com/typst/packages/raw/main/packages/preview/ctheorems/1.1.3/assets/basic.png}}

\subsubsection{Preamble}\label{preamble}

\begin{Shaded}
\begin{Highlighting}[]
\NormalTok{\#import "@preview/ctheorems:1.1.3": *}
\NormalTok{\#show: thmrules.with(qed{-}symbol: $square$)}

\NormalTok{\#set page(width: 16cm, height: auto, margin: 1.5cm)}
\NormalTok{\#set heading(numbering: "1.1.")}

\NormalTok{\#let theorem = thmbox("theorem", "Theorem", fill: rgb("\#eeffee"))}
\NormalTok{\#let corollary = thmplain(}
\NormalTok{  "corollary",}
\NormalTok{  "Corollary",}
\NormalTok{  base: "theorem",}
\NormalTok{  titlefmt: strong}
\NormalTok{)}
\NormalTok{\#let definition = thmbox("definition", "Definition", inset: (x: 1.2em, top: 1em))}

\NormalTok{\#let example = thmplain("example", "Example").with(numbering: none)}
\NormalTok{\#let proof = thmproof("proof", "Proof")}
\end{Highlighting}
\end{Shaded}

\subsubsection{Document}\label{document}

\begin{Shaded}
\begin{Highlighting}[]
\NormalTok{= Prime numbers}

\NormalTok{\#definition[}
\NormalTok{  A natural number is called a \#highlight[\_prime number\_] if it is greater}
\NormalTok{  than 1 and cannot be written as the product of two smaller natural numbers.}
\NormalTok{]}
\NormalTok{\#example[}
\NormalTok{  The numbers $2$, $3$, and $17$ are prime.}
\NormalTok{  @cor\_largest\_prime shows that this list is not exhaustive!}
\NormalTok{]}

\NormalTok{\#theorem("Euclid")[}
\NormalTok{  There are infinitely many primes.}
\NormalTok{]}
\NormalTok{\#proof[}
\NormalTok{  Suppose to the contrary that $p\_1, p\_2, dots, p\_n$ is a finite enumeration}
\NormalTok{  of all primes. Set $P = p\_1 p\_2 dots p\_n$. Since $P + 1$ is not in our list,}
\NormalTok{  it cannot be prime. Thus, some prime factor $p\_j$ divides $P + 1$.  Since}
\NormalTok{  $p\_j$ also divides $P$, it must divide the difference $(P + 1) {-} P = 1$, a}
\NormalTok{  contradiction.}
\NormalTok{]}

\NormalTok{\#corollary[}
\NormalTok{  There is no largest prime number.}
\NormalTok{] \textless{}cor\_largest\_prime\textgreater{}}
\NormalTok{\#corollary[}
\NormalTok{  There are infinitely many composite numbers.}
\NormalTok{]}

\NormalTok{\#theorem[}
\NormalTok{  There are arbitrarily long stretches of composite numbers.}
\NormalTok{]}
\NormalTok{\#proof[}
\NormalTok{  For any $n \textgreater{} 2$, consider $}
\NormalTok{    n! + 2, quad n! + 3, quad ..., quad n! + n \#qedhere}
\NormalTok{  $}
\NormalTok{]}
\end{Highlighting}
\end{Shaded}

\subsection{Changelog}\label{changelog}

\subsubsection{v1.1.3}\label{v1.1.3}

\begin{itemize}
\tightlist
\item
  Fixed alignment and block-breaking issues resulting from breaking
  changes in Typst 0.12.
\end{itemize}

\subsubsection{v1.1.2}\label{v1.1.2}

\begin{itemize}
\tightlist
\item
  Introduced the \texttt{\ thmproof\ } function for creating proof
  environments.
\item
  Inserting \texttt{\ \#qedhere\ } in a block equation/list/enum item
  (in a proof) places the qed symbol on the same line. The qed symbol
  can be customized via \texttt{\ thmrules\ } .
\end{itemize}

\subsubsection{v1.1.1}\label{v1.1.1}

\begin{itemize}
\tightlist
\item
  Extra named arguments given to a theorem environment produced by
  \texttt{\ thmbox\ } (or \texttt{\ thmplain\ } ) are passed to
  \texttt{\ block\ } .
\end{itemize}

\subsubsection{v1.1.0}\label{v1.1.0}

\begin{itemize}
\tightlist
\item
  The \texttt{\ supplement\ } (for references) is no longer set in
  \texttt{\ thmenv\ } . It can be passed to the theorem environment
  directly, along with \texttt{\ refnumbering\ } to control the
  appearance of \texttt{\ @reference\ } s.
\item
  Extra named arguments given to \texttt{\ thmbox\ } are passed to
  \texttt{\ block\ } .
\item
  Fixed spacing bug for unnumbered environments.
\item
  Replaced dummy figure with labelled metadata.
\end{itemize}

\subsubsection{v1.0.0}\label{v1.0.0}

\begin{itemize}
\tightlist
\item
  Extra named arguments given to a theorem environment are passed to its
  formatting function \texttt{\ fmt\ } .
\item
  Removed \texttt{\ thmref\ } , introduced normal
  \texttt{\ \textless{}label\textgreater{}\ } s and
  \texttt{\ @reference\ } s.
\item
  Import must be followed by \texttt{\ show:\ thmrules\ } .
\item
  Removed \texttt{\ name:\ ...\ } from theorem environments; use
  \texttt{\ \#theorem("Euclid"){[}{]}\ } instead of
  \texttt{\ \#theorem(name:\ "Euclid"){[}{]}\ } .
\item
  Theorems are now wrapped in \texttt{\ figure\ } s.
\end{itemize}

\subsection{Credits}\label{credits}

\begin{itemize}
\tightlist
\item
  \href{https://github.com/sahasatvik}{sahasatvik (Satvik Saha)}
\item
  \href{https://github.com/MJHutchinson}{MJHutchinson (Michael
  Hutchinson)}
\item
  \href{https://github.com/rmolinari}{rmolinari (Rory Molinari)}
\item
  \href{https://github.com/PgBiel}{PgBiel}
\item
  \href{https://github.com/DVDTSB}{DVDTSB}
\end{itemize}

\subsubsection{How to add}\label{how-to-add}

Copy this into your project and use the import as \texttt{\ ctheorems\ }

\begin{verbatim}
#import "@preview/ctheorems:1.1.3"
\end{verbatim}

\includesvg[width=0.16667in,height=0.16667in]{/assets/icons/16-copy.svg}

Check the docs for
\href{https://typst.app/docs/reference/scripting/\#packages}{more
information on how to import packages} .

\subsubsection{About}\label{about}

\begin{description}
\tightlist
\item[Author s :]
sahasatvik (Satvik Saha) , rmolinari (Rory Molinari) , MJHutchinson
(Michael Hutchinson) , PgBiel , \& DVDTSB
\item[License:]
MIT
\item[Current version:]
1.1.3
\item[Last updated:]
October 23, 2024
\item[First released:]
September 13, 2023
\item[Archive size:]
4.68 kB
\href{https://packages.typst.org/preview/ctheorems-1.1.3.tar.gz}{\pandocbounded{\includesvg[keepaspectratio]{/assets/icons/16-download.svg}}}
\item[Repository:]
\href{https://github.com/sahasatvik/typst-theorems}{GitHub}
\end{description}

\subsubsection{Where to report issues?}\label{where-to-report-issues}

This package is a project of sahasatvik (Satvik Saha), rmolinari (Rory
Molinari), MJHutchinson (Michael Hutchinson), PgBiel, and DVDTSB .
Report issues on
\href{https://github.com/sahasatvik/typst-theorems}{their repository} .
You can also try to ask for help with this package on the
\href{https://forum.typst.app}{Forum} .

Please report this package to the Typst team using the
\href{https://typst.app/contact}{contact form} if you believe it is a
safety hazard or infringes upon your rights.

\phantomsection\label{versions}
\subsubsection{Version history}\label{version-history}

\begin{longtable}[]{@{}ll@{}}
\toprule\noalign{}
Version & Release Date \\
\midrule\noalign{}
\endhead
\bottomrule\noalign{}
\endlastfoot
1.1.3 & October 23, 2024 \\
\href{https://typst.app/universe/package/ctheorems/1.1.2/}{1.1.2} &
February 25, 2024 \\
\href{https://typst.app/universe/package/ctheorems/1.1.1/}{1.1.1} &
February 24, 2024 \\
\href{https://typst.app/universe/package/ctheorems/1.1.0/}{1.1.0} &
November 6, 2023 \\
\href{https://typst.app/universe/package/ctheorems/1.0.0/}{1.0.0} &
September 23, 2023 \\
\href{https://typst.app/universe/package/ctheorems/0.1.0/}{0.1.0} &
September 13, 2023 \\
\end{longtable}

Typst GmbH did not create this package and cannot guarantee correct
functionality of this package or compatibility with any version of the
Typst compiler or app.


\title{typst.app/universe/package/echarm}

\phantomsection\label{banner}
\section{echarm}\label{echarm}

{ 0.1.1 }

Run echarts in typst with the use of CtxJS.

\phantomsection\label{readme}
A typst plugin to run echarts in typst with the use of CtxJS.

\subsection{Examples}\label{examples}

\begin{longtable}[]{@{}lll@{}}
\toprule\noalign{}
\endhead
\bottomrule\noalign{}
\endlastfoot
\href{https://github.com/typst/packages/raw/main/packages/preview/echarm/0.1.1/examples/mixed_charts.typ}{\includegraphics[width=1\linewidth,height=\textheight,keepaspectratio]{https://github.com/typst/packages/raw/main/packages/preview/echarm/0.1.1/examples/mixed_charts.png}}
&
\href{https://github.com/typst/packages/raw/main/packages/preview/echarm/0.1.1/examples/radar.typ}{\includegraphics[width=1\linewidth,height=\textheight,keepaspectratio]{https://github.com/typst/packages/raw/main/packages/preview/echarm/0.1.1/examples/radar.png}}
&
\href{https://github.com/typst/packages/raw/main/packages/preview/echarm/0.1.1/examples/pie.typ}{\includegraphics[width=1\linewidth,height=\textheight,keepaspectratio]{https://github.com/typst/packages/raw/main/packages/preview/echarm/0.1.1/examples/pie.png}} \\
\href{https://github.com/typst/packages/raw/main/packages/preview/echarm/0.1.1/examples/mixed_charts.typ}{Source
Code} &
\href{https://github.com/typst/packages/raw/main/packages/preview/echarm/0.1.1/examples/radar.typ}{Source
Code} &
\href{https://github.com/typst/packages/raw/main/packages/preview/echarm/0.1.1/examples/pie.typ}{Source
Code} \\
\href{https://github.com/typst/packages/raw/main/packages/preview/echarm/0.1.1/examples/scatter.typ}{\includegraphics[width=1\linewidth,height=\textheight,keepaspectratio]{https://github.com/typst/packages/raw/main/packages/preview/echarm/0.1.1/examples/scatter.png}}
&
\href{https://github.com/typst/packages/raw/main/packages/preview/echarm/0.1.1/examples/gauge.typ}{\includegraphics[width=1\linewidth,height=\textheight,keepaspectratio]{https://github.com/typst/packages/raw/main/packages/preview/echarm/0.1.1/examples/gauge.png}}
&
\href{https://github.com/typst/packages/raw/main/packages/preview/echarm/0.1.1/examples/candlestick.typ}{\includegraphics[width=1\linewidth,height=\textheight,keepaspectratio]{https://github.com/typst/packages/raw/main/packages/preview/echarm/0.1.1/examples/candlestick.png}} \\
\href{https://github.com/typst/packages/raw/main/packages/preview/echarm/0.1.1/examples/scatter.typ}{Source
Code} &
\href{https://github.com/typst/packages/raw/main/packages/preview/echarm/0.1.1/examples/gauge.typ}{Source
Code} &
\href{https://github.com/typst/packages/raw/main/packages/preview/echarm/0.1.1/examples/candlestick.typ}{Source
Code} \\
\end{longtable}

For more examples see:

\url{https://echarts.apache.org/examples/en/index.html}

For the complete documentation for the configuration of echarts, see:

\url{https://echarts.apache.org/en/option.html}

\subsection{Usage}\label{usage}

\begin{Shaded}
\begin{Highlighting}[]
\NormalTok{\#import "@preview/echarm:0.1.1"}

\NormalTok{// options are echart options}
\NormalTok{\#echarm.render(width: 100\%, height: 100\%, options: (:))}
\end{Highlighting}
\end{Shaded}

\subsection{Infos}\label{infos}

The version is not the same as the echart version, so that I can update
independently. Animations are not supported here!

You can find more information about CtxJS here:

\url{https://typst.app/universe/package/ctxjs/}

\subsection{Versions}\label{versions}

\begin{longtable}[]{@{}ll@{}}
\toprule\noalign{}
Version & Echart-Version \\
\midrule\noalign{}
\endhead
\bottomrule\noalign{}
\endlastfoot
0.1.0 & 5.5.1 \\
0.1.1 & 5.5.1 \textsuperscript{1} \\
\end{longtable}

\textsuperscript{1} new eval-later feature

\subsubsection{How to add}\label{how-to-add}

Copy this into your project and use the import as \texttt{\ echarm\ }

\begin{verbatim}
#import "@preview/echarm:0.1.1"
\end{verbatim}

\includesvg[width=0.16667in,height=0.16667in]{/assets/icons/16-copy.svg}

Check the docs for
\href{https://typst.app/docs/reference/scripting/\#packages}{more
information on how to import packages} .

\subsubsection{About}\label{about}

\begin{description}
\tightlist
\item[Author :]
lublak
\item[License:]
MIT
\item[Current version:]
0.1.1
\item[Last updated:]
November 29, 2024
\item[First released:]
September 15, 2024
\item[Archive size:]
888 kB
\href{https://packages.typst.org/preview/echarm-0.1.1.tar.gz}{\pandocbounded{\includesvg[keepaspectratio]{/assets/icons/16-download.svg}}}
\item[Repository:]
\href{https://github.com/lublak/typst-echarm-package}{GitHub}
\end{description}

\subsubsection{Where to report issues?}\label{where-to-report-issues}

This package is a project of lublak . Report issues on
\href{https://github.com/lublak/typst-echarm-package}{their repository}
. You can also try to ask for help with this package on the
\href{https://forum.typst.app}{Forum} .

Please report this package to the Typst team using the
\href{https://typst.app/contact}{contact form} if you believe it is a
safety hazard or infringes upon your rights.

\phantomsection\label{versions}
\subsubsection{Version history}\label{version-history}

\begin{longtable}[]{@{}ll@{}}
\toprule\noalign{}
Version & Release Date \\
\midrule\noalign{}
\endhead
\bottomrule\noalign{}
\endlastfoot
0.1.1 & November 29, 2024 \\
\href{https://typst.app/universe/package/echarm/0.1.0/}{0.1.0} &
September 15, 2024 \\
\end{longtable}

Typst GmbH did not create this package and cannot guarantee correct
functionality of this package or compatibility with any version of the
Typst compiler or app.


