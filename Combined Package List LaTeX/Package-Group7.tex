\title{typst.app/universe/package/hidden-bib}

\phantomsection\label{banner}
\section{hidden-bib}\label{hidden-bib}

{ 0.1.1 }

Create hidden bibliographies or bibliographies with unmentioned (hidden)
citations.

\phantomsection\label{readme}
\href{https://github.com/pklaschka/typst-hidden-bib}{GitHub Repository
including Examples}

A Typst package to create hidden bibliographies or bibliographies with
unmentioned (hidden) citations.

\subsection{Use Cases}\label{use-cases}

\subsubsection{Hidden Bibliographies}\label{hidden-bibliographies}

In some documents, such as a letter, you may want to cite a reference
without printing a bibliography.

This can easily be achieved by wrapping your
\texttt{\ bibliography(...)\ } with the \texttt{\ hidden-bibliography\ }
function after importing the \texttt{\ hidden-bib\ } package.

The code then looks like this:

\begin{Shaded}
\begin{Highlighting}[]
\NormalTok{\#import "@preview/hidden{-}bib:0.1.0": hidden{-}bibliography}

\NormalTok{\#lorem(20) @example1}
\NormalTok{\#lorem(40) @example2[p. 2]}

\NormalTok{\#hidden{-}bibliography(}
\NormalTok{  bibliography("/refs.yml")}
\NormalTok{)}
\end{Highlighting}
\end{Shaded}

\emph{Note that this automatically sets the \texttt{\ style\ } option to
\texttt{\ "chicago-notes"\ } unless you specify a different style.}

\subsubsection{Hidden Citations}\label{hidden-citations}

In some documents, it may be necessary to include items in your
bibliography which weren’t explicitly cited at any specific point in
your document.

The code then looks like this:

\begin{Shaded}
\begin{Highlighting}[]
\NormalTok{\#import "@preview/hidden{-}bib:0.1.0": hidden{-}cite}

\NormalTok{\#hidden{-}cite("example1")}
\end{Highlighting}
\end{Shaded}

\subsubsection{Multiple Hidden
Citations}\label{multiple-hidden-citations}

If you want to include a large number of items in your bibliography
without having to use \texttt{\ hidden-cite\ } (to still get
autocompletion in the web editor), you can use the
\texttt{\ hidden-citations\ } environment.

The code then looks like this:

\begin{Shaded}
\begin{Highlighting}[]
\NormalTok{\#import "@preview/hidden{-}bib:0.1.0": hidden{-}citations}

\NormalTok{\#hidden{-}citations[}
\NormalTok{  @example1}
\NormalTok{  @example2}
\NormalTok{]}
\end{Highlighting}
\end{Shaded}

\subsection{FAQ}\label{faq}

\subsubsection{Why would I want to have hidden citations and a hidden
bibliography?}\label{why-would-i-want-to-have-hidden-citations-and-a-hidden-bibliography}

You don’t. While this package solves both (related) problems, you
should only use one of them at a time. Otherwise, you’ll simply see
nothing at all.

\subsubsection{Why would I want to have hidden
citations?}\label{why-would-i-want-to-have-hidden-citations}

That’s for you to decide. It essentially enables you to include
“uncited references�, similar to LaTeX’s
\texttt{\ \textbackslash{}nocite\{\}\ } command.

\subsection{License}\label{license}

This package is licensed under the MIT license. See the
\href{https://github.com/typst/packages/raw/main/packages/preview/hidden-bib/0.1.1/LICENSE}{LICENSE}
file for details.

\subsubsection{How to add}\label{how-to-add}

Copy this into your project and use the import as
\texttt{\ hidden-bib\ }

\begin{verbatim}
#import "@preview/hidden-bib:0.1.1"
\end{verbatim}

\includesvg[width=0.16667in,height=0.16667in]{/assets/icons/16-copy.svg}

Check the docs for
\href{https://typst.app/docs/reference/scripting/\#packages}{more
information on how to import packages} .

\subsubsection{About}\label{about}

\begin{description}
\tightlist
\item[Author :]
Zuri Klaschka
\item[License:]
MIT
\item[Current version:]
0.1.1
\item[Last updated:]
October 10, 2023
\item[First released:]
October 2, 2023
\item[Archive size:]
2.12 kB
\href{https://packages.typst.org/preview/hidden-bib-0.1.1.tar.gz}{\pandocbounded{\includesvg[keepaspectratio]{/assets/icons/16-download.svg}}}
\item[Repository:]
\href{https://github.com/pklaschka/typst-hidden-bib}{GitHub}
\end{description}

\subsubsection{Where to report issues?}\label{where-to-report-issues}

This package is a project of Zuri Klaschka . Report issues on
\href{https://github.com/pklaschka/typst-hidden-bib}{their repository} .
You can also try to ask for help with this package on the
\href{https://forum.typst.app}{Forum} .

Please report this package to the Typst team using the
\href{https://typst.app/contact}{contact form} if you believe it is a
safety hazard or infringes upon your rights.

\phantomsection\label{versions}
\subsubsection{Version history}\label{version-history}

\begin{longtable}[]{@{}ll@{}}
\toprule\noalign{}
Version & Release Date \\
\midrule\noalign{}
\endhead
\bottomrule\noalign{}
\endlastfoot
0.1.1 & October 10, 2023 \\
\href{https://typst.app/universe/package/hidden-bib/0.1.0/}{0.1.0} &
October 2, 2023 \\
\end{longtable}

Typst GmbH did not create this package and cannot guarantee correct
functionality of this package or compatibility with any version of the
Typst compiler or app.


\title{typst.app/universe/package/equate}

\phantomsection\label{banner}
\section{equate}\label{equate}

{ 0.2.1 }

Breakable equations with improved numbering.

{ } Featured Package

\phantomsection\label{readme}
A package for improved layout of equations and mathematical expressions.

When applied, this package will split up multi-line block equations into
multiple elements, so that each line can be assigned a separate number.
By default, the equation counter is incremented for each line, but this
behavior can be changed by setting the \texttt{\ sub-numbering\ }
argument to \texttt{\ true\ } . In this case, the equation counter will
only be incremented once for the entire block, and each line will be
assigned a sub-number like \texttt{\ 1a\ } , \texttt{\ 2.1\ } , or
similar, depending on the set equation numbering. You can also set the
\texttt{\ number-mode\ } argument to \texttt{\ "label"\ } to only number
labelled lines. If a label is only applied to the full equation, all
lines will be numbered.

This splitting also makes it possible to spread equations over page
boundaries while keeping alignment in place, which can be useful for
long derivations or proofs. This can be configured by the
\texttt{\ breakable\ } parameter of the \texttt{\ equate\ } function, or
by setting the \texttt{\ breakable\ } parameter of \texttt{\ block\ }
for equations via a show-set rule. Additionally, the alignment of the
equation number is improved, so that it always matches the baseline of
the equation.

If you want to create a “standard� equation with a single equation
number centered across all lines, you can attach the
\texttt{\ \textless{}equate:revoke\textgreater{}\ } label to the
equation. This will disable the effect of this package for the current
equation. This label can also be used in single lines of an equation to
disable numbering for that line only.

\subsection{Usage}\label{usage}

The package comes with a single \texttt{\ equate\ } function that is
supposed to be used as a template. It takes two optional arguments for
customization:

\begin{longtable}[]{@{}llll@{}}
\toprule\noalign{}
Argument & Type & Description & Default \\
\midrule\noalign{}
\endhead
\bottomrule\noalign{}
\endlastfoot
\texttt{\ breakable\ } & \texttt{\ boolean\ } , \texttt{\ auto\ } &
Whether to allow the equation to break across pages. &
\texttt{\ auto\ } \\
\texttt{\ sub-numbering\ } & \texttt{\ boolean\ } & Whether to assign
sub-numbers to each line of an equation. & \texttt{\ false\ } \\
\texttt{\ number-mode\ } & \texttt{\ "line"\ } , \texttt{\ "label"\ } &
Whether to number all lines or only those with a label. &
\texttt{\ "line"\ } \\
\end{longtable}

To reference a specific line of an equation, include the label at the
end of the line, like in the following example:

\begin{Shaded}
\begin{Highlighting}[]
\NormalTok{\#import "@preview/equate:0.2.1": equate}

\NormalTok{\#show: equate.with(breakable: true, sub{-}numbering: true)}
\NormalTok{\#set math.equation(numbering: "(1.1)")}

\NormalTok{The dot product of two vectors $arrow(a)$ and $arrow(b)$ can}
\NormalTok{be calculated as shown in @dot{-}product.}

\NormalTok{$}
\NormalTok{  angle.l a, b angle.r \&= arrow(a) dot arrow(b) \textbackslash{}}
\NormalTok{                       \&= a\_1 b\_1 + a\_2 b\_2 + ... a\_n b\_n \textbackslash{}}
\NormalTok{                       \&= sum\_(i=1)\^{}n a\_i b\_i. \#\textless{}sum\textgreater{}}
\NormalTok{$ \textless{}dot{-}product\textgreater{}}

\NormalTok{The sum notation in @sum is a useful way to express the dot}
\NormalTok{product of two vectors.}
\end{Highlighting}
\end{Shaded}

\pandocbounded{\includesvg[keepaspectratio]{https://github.com/typst/packages/raw/main/packages/preview/equate/0.2.1/assets/example-1.svg}}\\
\pandocbounded{\includesvg[keepaspectratio]{https://github.com/typst/packages/raw/main/packages/preview/equate/0.2.1/assets/example-2.svg}}

\subsubsection{Local Usage}\label{local-usage}

If you only want to use the package features on selected equations, you
can also apply the \texttt{\ equate\ } function directly to the
equation. This will override the default behavior for the current
equation only. Note, that this will require you to use the
\texttt{\ equate\ } function as a show rule for references, as shown in
the following example:

\begin{Shaded}
\begin{Highlighting}[]
\NormalTok{\#import "@preview/equate:0.2.1": equate}

\NormalTok{// Allow references to a line of the equation.}
\NormalTok{\#show ref: equate}

\NormalTok{\#set math.equation(numbering: "(1.1)", supplement: "Eq.")}

\NormalTok{\#equate($}
\NormalTok{  E \&= m c\^{}2 \#\textless{}short\textgreater{} \textbackslash{}}
\NormalTok{    \&= sqrt(p\^{}2 c\^{}2 + m\^{}2 c\^{}4) \#\textless{}long\textgreater{}}
\NormalTok{$)}

\NormalTok{While @short is the famous equation by Einstein, @long is a}
\NormalTok{more general form of the energy{-}momentum relation.}
\end{Highlighting}
\end{Shaded}

\pandocbounded{\includesvg[keepaspectratio]{https://github.com/typst/packages/raw/main/packages/preview/equate/0.2.1/assets/example-local.svg}}

As an alternative to the show rule, it is also possible to manually wrap
each reference in an \texttt{\ equate\ } function, though this is less
convenient and more prone to mistakes.

\subsubsection{How to add}\label{how-to-add}

Copy this into your project and use the import as \texttt{\ equate\ }

\begin{verbatim}
#import "@preview/equate:0.2.1"
\end{verbatim}

\includesvg[width=0.16667in,height=0.16667in]{/assets/icons/16-copy.svg}

Check the docs for
\href{https://typst.app/docs/reference/scripting/\#packages}{more
information on how to import packages} .

\subsubsection{About}\label{about}

\begin{description}
\tightlist
\item[Author :]
Eric Biedert
\item[License:]
MIT
\item[Current version:]
0.2.1
\item[Last updated:]
September 11, 2024
\item[First released:]
July 5, 2024
\item[Minimum Typst version:]
0.11.0
\item[Archive size:]
5.81 kB
\href{https://packages.typst.org/preview/equate-0.2.1.tar.gz}{\pandocbounded{\includesvg[keepaspectratio]{/assets/icons/16-download.svg}}}
\item[Repository:]
\href{https://github.com/EpicEricEE/typst-equate}{GitHub}
\item[Categor ies :]
\begin{itemize}
\tightlist
\item[]
\item
  \pandocbounded{\includesvg[keepaspectratio]{/assets/icons/16-layout.svg}}
  \href{https://typst.app/universe/search/?category=layout}{Layout}
\item
  \pandocbounded{\includesvg[keepaspectratio]{/assets/icons/16-list-unordered.svg}}
  \href{https://typst.app/universe/search/?category=model}{Model}
\end{itemize}
\end{description}

\subsubsection{Where to report issues?}\label{where-to-report-issues}

This package is a project of Eric Biedert . Report issues on
\href{https://github.com/EpicEricEE/typst-equate}{their repository} .
You can also try to ask for help with this package on the
\href{https://forum.typst.app}{Forum} .

Please report this package to the Typst team using the
\href{https://typst.app/contact}{contact form} if you believe it is a
safety hazard or infringes upon your rights.

\phantomsection\label{versions}
\subsubsection{Version history}\label{version-history}

\begin{longtable}[]{@{}ll@{}}
\toprule\noalign{}
Version & Release Date \\
\midrule\noalign{}
\endhead
\bottomrule\noalign{}
\endlastfoot
0.2.1 & September 11, 2024 \\
\href{https://typst.app/universe/package/equate/0.2.0/}{0.2.0} & July 5,
2024 \\
\href{https://typst.app/universe/package/equate/0.1.0/}{0.1.0} & July 5,
2024 \\
\end{longtable}

Typst GmbH did not create this package and cannot guarantee correct
functionality of this package or compatibility with any version of the
Typst compiler or app.


\title{typst.app/universe/package/cheda-seu-thesis}

\phantomsection\label{banner}
\phantomsection\label{template-thumbnail}
\pandocbounded{\includegraphics[keepaspectratio]{https://packages.typst.org/preview/thumbnails/cheda-seu-thesis-0.3.0-small.webp}}

\section{cheda-seu-thesis}\label{cheda-seu-thesis}

{ 0.3.0 }

东å?---大学本ç§`毕设与ç~''究ç''Ÿå­¦ä½?论æ--‡æ¨¡æ?¿ã€‚UNOFFICIAL
Southeast University Thesis.

\href{/app?template=cheda-seu-thesis&version=0.3.0}{Create project in
app}

\phantomsection\label{readme}
使ç''¨ Typst
å¤?刻东å?---大学「本ç§`毕业设计(论æ--‡ï¼‰æŠ¥å`Šã€?模æ?¿å'Œã€Œç~''究ç''Ÿå­¦ä½?论æ--‡ã€?模æ?¿ã€‚

请在
\href{https://github.com/typst/packages/raw/main/packages/preview/cheda-seu-thesis/0.3.0/init-files/}{\texttt{\ init-files\ }}
目录å†\ldots 查看 Demo PDF。

\begin{quote}
{[}!IMPORTANT{]}

此模æ?¿æ˜¯æ°`é---´æ¨¡æ?¿ï¼Œæœ‰ä¸?被学æ~¡è®¤å?¯çš„风险。

本模�虽已尽力�试�原原始 Word
模æ?¿ï¼Œä½†å?¯èƒ½ä»?然存在诸多æ~¼å¼?é---®é¢˜ã€‚

Typst
是一个ä»?在活跃开å?{}`ã€?å?¯èƒ½ä¼šæœ‰è¾ƒå¤§å?˜æ›´çš„æŽ'版工å\ldots·ï¼Œè¯·é€‰æ‹©æœ€æ--°ç‰ˆæ¨¡æ?¿ä¸Žæœ¬æ¨¡æ?¿å»ºè®®çš„
Typst 版本相é\ldots?å?ˆä½¿ç''¨ã€‚
\end{quote}

\begin{quote}
{[}!CAUTION{]}

本模�在
\href{https://github.com/csimide/SEU-Typst-Template/tree/c44b5172178c0c2380b322e50931750e2d761168}{\texttt{\ 0.2.2\ }}
-\textgreater{} \texttt{\ 0.3.0\ }
æ---¶è¿›è¡Œäº†ç~´å??性å?˜æ›´ã€‚有å\ldots³æ­¤æ¬¡å?˜æ›´çš„详细信æ?¯ï¼Œè¯·æŸ¥çœ‹
\href{https://github.com/typst/packages/raw/main/packages/preview/cheda-seu-thesis/0.3.0/CHANGELOG.md}{æ›´æ--°æ---¥å¿---}
\end{quote}

\begin{itemize}
\tightlist
\item
  \href{https://github.com/typst/packages/raw/main/packages/preview/cheda-seu-thesis/0.3.0/\#\%E4\%B8\%9C\%E5\%8D\%97\%E5\%A4\%A7\%E5\%AD\%A6\%E8\%AE\%BA\%E6\%96\%87\%E6\%A8\%A1\%E6\%9D\%BF}{东å?---大学论æ--‡æ¨¡æ?¿}

  \begin{itemize}
  \tightlist
  \item
    \href{https://github.com/typst/packages/raw/main/packages/preview/cheda-seu-thesis/0.3.0/\#\%E4\%BD\%BF\%E7\%94\%A8\%E6\%96\%B9\%E6\%B3\%95}{使ç''¨æ--¹æ³•}

    \begin{itemize}
    \tightlist
    \item
      \href{https://github.com/typst/packages/raw/main/packages/preview/cheda-seu-thesis/0.3.0/\#\%E6\%9C\%AC\%E5\%9C\%B0\%E4\%BD\%BF\%E7\%94\%A8}{本地使ç''¨}
    \item
      \href{https://github.com/typst/packages/raw/main/packages/preview/cheda-seu-thesis/0.3.0/\#web-app}{Web
      App}
    \end{itemize}
  \item
    \href{https://github.com/typst/packages/raw/main/packages/preview/cheda-seu-thesis/0.3.0/\#\%E6\%A8\%A1\%E6\%9D\%BF\%E5\%86\%85\%E5\%AE\%B9}{模æ?¿å†\ldots 容}

    \begin{itemize}
    \tightlist
    \item
      \href{https://github.com/typst/packages/raw/main/packages/preview/cheda-seu-thesis/0.3.0/\#\%E7\%A0\%94\%E7\%A9\%B6\%E7\%94\%9F\%E5\%AD\%A6\%E4\%BD\%8D\%E8\%AE\%BA\%E6\%96\%87\%E6\%A8\%A1\%E6\%9D\%BF}{ç~''究ç''Ÿå­¦ä½?论æ--‡æ¨¡æ?¿}
    \item
      \href{https://github.com/typst/packages/raw/main/packages/preview/cheda-seu-thesis/0.3.0/\#\%E6\%9C\%AC\%E7\%A7\%91\%E6\%AF\%95\%E4\%B8\%9A\%E8\%AE\%BE\%E8\%AE\%A1\%E8\%AE\%BA\%E6\%96\%87\%E6\%8A\%A5\%E5\%91\%8A\%E6\%A8\%A1\%E6\%9D\%BF}{本ç§`毕业设计(论æ--‡ï¼‰æŠ¥å`Šæ¨¡æ?¿}
    \end{itemize}
  \item
    \href{https://github.com/typst/packages/raw/main/packages/preview/cheda-seu-thesis/0.3.0/\#\%E7\%9B\%AE\%E5\%89\%8D\%E5\%AD\%98\%E5\%9C\%A8\%E7\%9A\%84\%E9\%97\%AE\%E9\%A2\%98}{ç›®å‰?存在的é---®é¢˜}

    \begin{itemize}
    \tightlist
    \item
      \href{https://github.com/typst/packages/raw/main/packages/preview/cheda-seu-thesis/0.3.0/\#\%E5\%8F\%82\%E8\%80\%83\%E6\%96\%87\%E7\%8C\%AE}{å?‚考æ--‡çŒ®}
    \end{itemize}
  \item
    \href{https://github.com/typst/packages/raw/main/packages/preview/cheda-seu-thesis/0.3.0/\#\%E5\%8F\%8B\%E6\%83\%85\%E9\%93\%BE\%E6\%8E\%A5}{å?‹æƒ\ldots é``¾æŽ¥}
  \item
    \href{https://github.com/typst/packages/raw/main/packages/preview/cheda-seu-thesis/0.3.0/\#\%E5\%BC\%80\%E5\%8F\%91\%E4\%B8\%8E\%E5\%8D\%8F\%E8\%AE\%AE}{å¼€å?{}`与å??è®®}

    \begin{itemize}
    \tightlist
    \item
      \href{https://github.com/typst/packages/raw/main/packages/preview/cheda-seu-thesis/0.3.0/\#\%E4\%BA\%8C\%E6\%AC\%A1\%E5\%BC\%80\%E5\%8F\%91}{二次开å?{}`}
    \end{itemize}
  \end{itemize}
\end{itemize}

\subsection{使ç''¨æ--¹æ³•}\label{uxe4uxbduxe7uxe6uxb9uxe6uxb3}

本模æ?¿éœ€è¦?使ç''¨ Typst 0.11.x ç¼--è¯`。

此模æ?¿å·²ä¸Šä¼~ Typst Universe ,å?¯ä»¥ä½¿ç''¨
\texttt{\ typst\ init\ } 功能åˆ?始åŒ--,也å?¯ä»¥ä½¿ç''¨ Web App
ç¼--è¾`。Typst Universe
上的模æ?¿å?¯èƒ½ä¸?是最æ--°ç‰ˆæœ¬ã€‚如果需è¦?使ç''¨æœ€æ--°ç‰ˆæœ¬çš„模æ?¿ï¼Œä»Žæœ¬
repo 中获å?--。

\subsubsection{本地使ç''¨}\label{uxe6ux153uxe5ux153uxe4uxbduxe7}

请å\ldots ˆå®‰è£\ldots ä½?于 \texttt{\ fonts\ }
目录å†\ldots çš„å\ldots¨éƒ¨å­---ä½``。然å?Žï¼Œæ‚¨å?¯ä»¥ä½¿ç''¨ä»¥ä¸‹ä¸¤ç§?æ--¹å¼?使ç''¨æœ¬æ¨¡æ?¿ï¼š

\begin{itemize}
\tightlist
\item
  下载/clone 本 repo çš„å\ldots¨éƒ¨æ--‡ä»¶ï¼Œç¼--è¾`
  \texttt{\ init-files\ } 目录å†\ldots 的示例æ--‡ä»¶ã€‚
\item
  使ç''¨ \texttt{\ typst\ init\ @preview/cheda-seu-thesis:0.2.2\ }
  æ?¥èŽ·å?--此模æ?¿ä¸Žåˆ?始åŒ--æ--‡ä»¶ã€‚
\end{itemize}

éš?å?Žï¼Œæ‚¨å?¯ä»¥é€šè¿‡ç¼--è¾`示例æ--‡ä»¶æ?¥ç''Ÿæˆ?想è¦?的论æ--‡ã€‚两ç§?论æ--‡æ~¼å¼?的说明都在对åº''的示例æ--‡æ¡£å†\ldots 。

如您使ç''¨ VSCode 作为ç¼--è¾`器,å?¯ä»¥å°?试使ç''¨
\href{https://marketplace.visualstudio.com/items?itemName=nvarner.typst-lsp}{Tinymist}
与
\href{https://marketplace.visualstudio.com/items?itemName=mgt19937.typst-preview}{Typst
Preview} æ?'件。如有本地åŒ\ldots äº`å?Œæ­¥éœ€æ±‚,å?¯ä»¥ä½¿ç''¨
\href{https://marketplace.visualstudio.com/items?itemName=OrangeX4.vscode-typst-sync}{Typst
Sync} æ?'件。更多ç¼--è¾`技巧,å?¯æŸ¥é˜
\href{https://github.com/nju-lug/modern-nju-thesis\#vs-code-\%E6\%9C\%AC\%E5\%9C\%B0\%E7\%BC\%96\%E8\%BE\%91\%E6\%8E\%A8\%E8\%8D\%90}{https://github.com/nju-lug/modern-nju-thesis\#vs-code-本地ç¼--è¾`推è??}
。

\subsubsection{Web App}\label{web-app}

\begin{quote}
{[}!NOTE{]}

ç''±äºŽå­---ä½``原å›~,ä¸?建议使ç''¨ Web App ç¼--è¾`此模æ?¿ã€‚
\end{quote}

请æ‰``å¼€ \url{https://typst.app/universe/package/cheda-seu-thesis}
并点击 \texttt{\ Create\ project\ in\ app\ } ,æˆ--在 Web App
中选择 \texttt{\ Start\ from\ a\ template\ } ,�选择
\texttt{\ cheda-seu-thesis\ } 。

然�,请将
\url{https://github.com/csimide/SEU-Typst-Template/tree/master/fonts}
å†\ldots çš„ \textbf{所有} å­---ä½``上ä¼~到 Typst Web App
å†\ldots 该项目的æ~¹ç›®å½•ã€‚注æ„?,之å?Žæ¯?次æ‰``开此项目,æµ?览器都会花费很长æ---¶é---´ä»Ž
Typst Web App çš„æœ?务器下载这一批å­---ä½``,ä½``验较差。

最å?Žï¼Œè¯·æŒ‰ç\ldots§è‡ªåŠ¨æ‰``开的æ--‡ä»¶çš„æ??示æ``?作。

\subsection{模æ?¿å†\ldots 容}\label{uxe6uxe6uxe5uxe5uxb9}

\subsubsection{ç~''究ç''Ÿå­¦ä½?论æ--‡æ¨¡æ?¿}\label{uxe7-uxe7uxe7uxffuxe5uxe4uxbduxe8uxbauxe6uxe6uxe6}

æ­¤ Typst 模æ?¿æŒ‰ç\ldots§
\href{https://seugs.seu.edu.cn/2023/0424/c26669a442680/page.htm}{《东å?---大学ç~''究ç''Ÿå­¦ä½?论æ--‡æ~¼å¼?规定》}
制作,制作æ---¶å?‚考了
\href{https://ctan.math.utah.edu/ctan/tex-archive/macros/latex/contrib/seuthesis/seuthesis.pdf}{SEUThesis
模�} 。

å½``å‰?æ''¯æŒ?进度:

\begin{itemize}
\tightlist
\item
  æ--‡æ¡£æž„件

  \begin{itemize}
  \tightlist
  \item
    {[}x{]} å°?é?¢
  \item
    {[}x{]} 中英æ--‡æ‰‰é¡µ
  \item
    {[}x{]} 中英æ--‡æ`˜è¦?
  \item
    {[}x{]} 目录
  \item
    {[}x{]} 术语表
  \item
    {[}x{]} æ­£æ--‡
  \item
    {[}x{]} 致谢
  \item
    {[}x{]} å?‚考æ--‡çŒ®
  \item
    {[}x{]} 附录
  \item
    {[} {]} 索引
  \item
    {[} {]} 作è€\ldots 简介
  \item
    {[} {]} å?Žè®°
  \item
    {[} {]} �底
  \end{itemize}
\item
  功能

  \begin{itemize}
  \tightlist
  \item
    {[} {]} 盲审
  \item
    {[}x{]}
    页ç~?ç¼--å?·ï¼šæ­£æ--‡å‰?使ç''¨ç½---马数å­---,正æ--‡å?Šæ­£æ--‡å?Žä½¿ç''¨é˜¿æ‹‰ä¼¯æ•°å­---
  \item
    {[}x{]} æ­£æ--‡ã€?附录图表ç¼--å?·æ~¼å¼?:详è§?ç~''院è¦?求
  \item
    {[}x{]}
    æ•°å­¦å\ldots¬å¼?æ''¾ç½®ä½?置:离页é?¢å·¦ä¾§ä¸¤ä¸ªæ±‰å­---è·?离
  \item
    {[}x{]} æ•°å­¦å\ldots¬å¼?ç¼--å?·ï¼šå\ldots¬å¼?å?---å?³ä¸‹
  \item
    {[}x{]} æ?'å\ldots¥ç©ºç™½é¡µï¼šæ--°ç«~节总在奇数页上
  \item
    {[}x{]}
    页眉:奇数页显示ç«~节å?·å'Œç«~节æ~‡é¢˜ï¼Œå?¶æ•°é¡µæ˜¾ç¤ºå›ºå®šå†\ldots 容
  \item
    {[}x{]} å?‚考æ--‡çŒ®ï¼šæ''¯æŒ?å?Œè¯­æ˜¾ç¤º
  \end{itemize}
\end{itemize}

\subsubsection{本ç§`毕业设计(论æ--‡ï¼‰æŠ¥å`Šæ¨¡æ?¿}\label{uxe6ux153uxe7uxe6uxe4ux161uxe8uxbeuxe8uxefuxbcux2c6uxe8uxbauxe6uxefuxbcuxe6ux161uxe5ux161uxe6uxe6}

æ­¤ Typst
模æ?¿åŸºäºŽä¸œå?---大学本ç§`毕业设计(论æ--‡ï¼‰æŠ¥å`Šæ¨¡æ?¿ï¼ˆ2024
å¹´ 1 月)仿制,原模æ?¿å?¯ä»¥åœ¨æ•™åŠ¡å¤„ç½`站上下载(
\href{https://jwc.seu.edu.cn/2021/1108/c21686a389963/page.htm}{2019 å¹´
9 月版} ,
\href{https://jwc.seu.edu.cn/2024/0117/c21686a479303/page.htm}{2024 å¹´
1 月版} )。

å½``å‰?æ''¯æŒ?进度:

\begin{itemize}
\tightlist
\item
  æ--‡æ¡£æž„件

  \begin{itemize}
  \tightlist
  \item
    {[}x{]} å°?é?¢
  \item
    {[}x{]} 中英æ--‡æ`˜è¦?
  \item
    {[}x{]} 目录
  \item
    {[}x{]} æ­£æ--‡
  \item
    {[}x{]} å?‚考æ--‡çŒ®
  \item
    {[}x{]} 附录
  \item
    {[}x{]} 致谢
  \item
    {[} {]} �底
  \end{itemize}
\item
  功能

  \begin{itemize}
  \tightlist
  \item
    {[} {]} 盲审
  \item
    {[}x{]}
    页ç~?ç¼--å?·ï¼šæ­£æ--‡å‰?使ç''¨ç½---马数å­---,正æ--‡å?Šæ­£æ--‡å?Žä½¿ç''¨é˜¿æ‹‰ä¼¯æ•°å­---
  \item
    {[}x{]}
    æ­£æ--‡ã€?附录图表ç¼--å?·æ~¼å¼?:详è§?本ç§`毕设è¦?求
  \item
    {[}x{]}
    æ•°å­¦å\ldots¬å¼?æ''¾ç½®ä½?置:离页é?¢å·¦ä¾§ä¸¤ä¸ªæ±‰å­---è·?离
  \item
    {[}x{]} æ•°å­¦å\ldots¬å¼?ç¼--å?·ï¼šå\ldots¬å¼?å?---å?³ä¾§ä¸­å¿ƒ
  \item
    {[}x{]} 页眉:显示固定å†\ldots 容
  \item
    {[}x{]} å?‚考æ--‡çŒ®ï¼šæ''¯æŒ?å?Œè¯­æ˜¾ç¤º
  \item
    {[} {]} \st{表æ~¼æ˜¾ç¤ºâ€œç»­è¡¨â€?}
    ç''±äºŽæ•™åŠ¡å¤„æ??供的模æ?¿ä¸­æ²¡æœ‰ç»™å‡ºâ€œç»­è¡¨â€?显示æ~·ä¾‹ï¼Œæ•\ldots æš‚ä¸?实现。
  \end{itemize}
\end{itemize}

\begin{quote}
{[}!NOTE{]}

å?¯ä»¥çœ‹çœ‹éš''å£? \url{https://github.com/TideDra/seu-thesis-typst/}
项目,也正在使ç''¨ Typst
实现毕业设计(论æ--‡ï¼‰æŠ¥å`Šæ¨¡æ?¿ï¼Œè¿˜æ??供了毕设翻è¯`模æ?¿ã€‚该项目的实现细节与本模æ?¿å¹¶ä¸?相å?Œï¼Œæ‚¨å?¯ä»¥æ~¹æ?®è‡ªå·±çš„å--œå¥½é€‰æ‹©ã€‚
\end{quote}

\subsection{ç›®å‰?存在的é---®é¢˜}\label{uxe7uxe5uxe5uxe5ux153uxe7ux161uxe9uxe9}

\begin{itemize}
\tightlist
\item
  中æ--‡é¦--段有æ---¶ä¼šè‡ªåŠ¨ç¼©è¿›ï¼Œæœ‰æ---¶ä¸?会。如果没有自动缩进,需è¦?使ç''¨
  \texttt{\ \#h(2em)\ } 手动缩进两个å­---符。
\item
  å?‚考æ--‡çŒ®æ~¼å¼?ä¸?完å\ldots¨ç¬¦å?ˆè¦?求。请è§?下æ--¹å?‚考æ--‡çŒ®å°?节。
\item
  è¡Œè·?ã€?è¾¹è·?等有å¾\ldots 继续调整。
\end{itemize}

\subsubsection{å?‚考æ--‡çŒ®}\label{uxe5uxe8ux192uxe6uxe7ux153}

å?‚考æ--‡çŒ®æ~¼å¼?ä¸?完å\ldots¨ç¬¦å?ˆè¦?求。Typst 自带的 GB/T
7714-2015 numeric
æ~¼å¼?与学æ~¡è¦?求æ~¼å¼?相æ¯'',有以下é---®é¢˜ï¼š

\begin{enumerate}
\item
  å­¦æ~¡è¦?求在作è€\ldots æ•°é‡?较多æ---¶ï¼Œè‹±æ--‡ä½¿ç''¨
  \texttt{\ et\ al.\ } 中æ--‡ä½¿ç''¨ \texttt{\ ç­‰\ }
  ��略。但是,Typst
  ç›®å‰?ä»\ldots å?¯ä»¥æ˜¾ç¤ºä¸ºå?•ä¸€è¯­è¨€ã€‚

  \textbf{A:} 该é---®é¢˜ç³» Typst çš„ CSL 解æž?器ä¸?æ''¯æŒ? CSL-M
  导致的。

  详细原å›

  \begin{itemize}
  \tightlist
  \item
    使ç''¨ CSL 实现这一 feature 需è¦?ç''¨åˆ°
    \href{https://citeproc-js.readthedocs.io/en/latest/csl-m/index.html\#cs-layout-extension}{CSL-M}
    扩展的多 \texttt{\ layout\ } 功能,而 Typst å°šä¸?æ''¯æŒ?
    CSL-M 扩展功能。详�
    \url{https://github.com/typst/typst/issues/2793} 与
    \url{https://github.com/typst/citationberg/issues/5} 。
  \item
    Typst 目�会忽视 BibTeX/CSL 中的 \texttt{\ language\ }
    å­---段。å?‚è§? \url{https://github.com/typst/hayagriva/pull/126}
    。
  \end{itemize}

  å›~为上述原å›~,目å‰?很难使ç''¨ Typst
  原ç''Ÿæ--¹æ³•å®žçŽ°æ~¹æ?®è¯­è¨€è‡ªåŠ¨é€‰ç''¨ \texttt{\ et\ al.\ } 与
  \texttt{\ 等\ } 。

  OrangeX4 å'Œæˆ`写了一个基于查找替æ?¢çš„
  \texttt{\ bilingual-bibliography\ } 功能,试图在 Typst æ''¯æŒ?
  CSL-M å‰?实现中æ--‡è¥¿æ--‡ä½¿ç''¨ä¸?å?Œçš„å\ldots³é''®è¯?。

  本模æ?¿çš„ Demo æ--‡æ¡£å†\ldots 已使ç''¨
  \texttt{\ bilingual-bibliography\ } 引ç''¨ï¼Œè¯·æŸ¥çœ‹ Demo
  æ--‡æ¡£ä»¥äº†è§£ç''¨æ³•ã€‚注æ„?,该功能ä»?在测试,很å?¯èƒ½æœ‰
  Bug,详�
  \url{https://github.com/csimide/SEU-Typst-Template/issues/1} 。

  \begin{quote}
  请在 \url{https://github.com/nju-lug/modern-nju-thesis/issues/3}
  查看更多有å\ldots³å?Œè¯­å?‚考æ--‡çŒ®å®žçŽ°çš„讨论。

  本模æ?¿æ›¾ç»?å°?试使ç''¨ \url{https://github.com/csimide/cslper}
  作为å?Œè¯­å?‚考æ--‡çŒ®çš„实现æ--¹æ³•ã€‚
  \end{quote}
\item
  å­¦æ~¡ç»™å‡ºçš„范例中,除了纯ç''µå­?资æº?,å?³ä½¿å¼•ç''¨æ--‡çŒ®æ?¥è‡ªçº¿ä¸Šæ¸~é?{}``,也å?‡ä¸?åŠ
  \texttt{\ OL\ } ã€?访é---®æ---¥æœŸã€?DOI 与 é``¾æŽ¥ã€‚但是,Typst
  å†\ldots 置的 GB/T 7714-2015 numeric æ~¼å¼?会为所有 bib
  å†\ldots 定义了é``¾æŽ¥/DOI çš„æ--‡çŒ®æ·»åŠ \texttt{\ OL\ }
  æ~‡è®°å'Œé``¾æŽ¥/DOI 。

  \textbf{A:} 该é---®é¢˜ç³»å­¦æ~¡çš„æ~‡å‡†ä¸Ž GB/T 7714-2015
  ä¸?完å\ldots¨ä¸€è‡´å¯¼è‡´çš„。

  请使ç''¨
  \texttt{\ style:\ "./seu-thesis/gb-t-7714-2015-numeric-seu.csl"\ }
  ,会自动ä¾?æ?®æ--‡çŒ®ç±»åž‹é€‰æ‹©æ˜¯å?¦æ˜¾ç¤º \texttt{\ OL\ }
  æ~‡è®°å'Œé``¾æŽ¥/DOI。

  \begin{quote}
  该 csl ä¿®æ''¹è‡ª
  \url{https://github.com/redleafnew/Chinese-STD-GB-T-7714-related-csl/blob/main/003gb-t-7714-2015-numeric-bilingual-no-url-doi.csl}

  原æ--‡ä»¶åŸºäºŽ CC-BY-SA 3.0 å??è®®å\ldots±äº«ã€‚
  \end{quote}
\item
  作è€\ldots 大å°?写(æˆ--è€\ldots å\ldots¶ä»--细节)与学æ~¡èŒƒä¾‹ä¸?一致。
\item
  å­¦ä½?论æ--‡ä¸­ï¼Œå­¦æ~¡è¦?求引ç''¨å\ldots¶ä»--å­¦ä½?论æ--‡çš„æ--‡çŒ®ç±»åž‹åº''å½``写作
  \texttt{\ {[}D{]}:\ {[}�士学�论文{]}.\ }
  æ~¼å¼?,但模æ?¿æ˜¾ç¤ºä¸º \texttt{\ {[}D{]}\ }
  ,�显示�类别。
\item
  å­¦ä½?论æ--‡ä¸­ï¼Œå­¦æ~¡ç»™å‡ºçš„范例使ç''¨å\ldots¨è§'符å?·ï¼Œå¦‚å\ldots¨è§'æ--¹æ‹¬å?·ã€?å\ldots¨è§'å?¥ç‚¹ç­‰ã€‚
\item
  引ç''¨æ?¡ç›®ä¸¢å¤± \texttt{\ .\ } ,如 \texttt{\ {[}M{]}2nd\ ed\ }
  。

  \textbf{3\textasciitilde6 A:} å­¦æ~¡ç''¨çš„是 GB/T 7714-2015
  çš„æ--¹è¨€ï¼Œæ›¾ç»?有学长把它å?«å?š GB/T 7714-SEU
  ,目å‰?没找到完美匹é\ldots?å­¦æ~¡è¦?求的
  CSL(ä¸?å?Œå­¦é™¢çš„è¦?求也ä¸?太一æ~·ï¼‰ï¼Œå?Žç»­ä¼šå†™ä¸€ä¸ªç¬¦å?ˆè¦?求的
  CSL æ--‡ä»¶ã€‚

  \textbf{2024-05-02 æ›´æ--°:} 现已åˆ?步实现 CSL。ä¸?å¾---ä¸?说
  Typst çš„ CSL æ''¯æŒ?æˆ?谜……目å‰?ä¿®å¤?æƒ\ldots 况如下:

  \begin{itemize}
  \tightlist
  \item
    é---®é¢˜ 3 已修å¤?ï¼›
  \item
    é---®é¢˜ 4 在学ä½?论æ--‡çš„ CSL å†\ldots 已修å¤?,但 Typst
    似乎ä¸?æ''¯æŒ?这一å­---段,å›~æ­¤æ---~法显示;
  \item
    é---®é¢˜ 5
    ä¸?准备修å¤?,查é˜\ldots 数篇已å?{}`表的学ä½?论æ--‡ï¼Œä½¿ç''¨çš„也是å?Šè§'符å?·ï¼›
  \item
    é---®é¢˜ 6 似乎是 Typst çš„ CSL
    æ''¯æŒ?çš„é---®é¢˜ï¼Œæœ¬æ¨¡æ?¿é™„带的 CSL
    æ--‡ä»¶å·²ç»?å?šäº†é¢?å¤--处ç?†ï¼Œåº''该ä¸?会丢 \texttt{\ .\ }
    了。
  \end{itemize}
\item
  引ç''¨å\ldots¶ä»--å­¦ä½?论æ--‡æ---¶ï¼ŒGB7714-2015/本ç§`毕设/å­¦ä½?论æ--‡å?‡è¦?求注明
  \texttt{\ 地点:\ å­¦æ~¡å??称,\ 年份.\ }
  。但是模��显示这一项。

  \textbf{A:} Typst ä¸?æ''¯æŒ? \texttt{\ school\ }
  \texttt{\ institution\ } 作为 \texttt{\ publisher\ }
  的别å??,亦ä¸?æ''¯æŒ?解æž? csl 中的 \texttt{\ institution\ } (
  \url{https://github.com/typst/hayagriva/issues/112}
  )。如需修å¤?,请手动修æ''¹ bib
  æ--‡ä»¶å†\ldots 对åº''æ?¡ç›®ï¼Œåœ¨
  \texttt{\ school\ =\ \{å­¦æ~¡å??称\},\ } 下åŠ~一行
  \texttt{\ publisher\ =\ \{å­¦æ~¡å??称\},\ } 。

  ä¿®æ''¹ç¤ºä¾‹

\begin{Shaded}
\begin{Highlighting}[]
\NormalTok{@phdthesis\{Example1,}
\NormalTok{  type = \{\{硕士学位论文\}\},}
\NormalTok{  title = \{\{摸鱼背景下的Typst模板使用研究\}\},}
\NormalTok{  author = \{王, 东南\},}
\NormalTok{  year = \{2024\},}
\NormalTok{  langid = \{chinese\},}
\NormalTok{  address = \{南京\},}
\NormalTok{  school = \{东南大学\},}
\NormalTok{  publisher = \{东南大学\},}
\NormalTok{\}}
\end{Highlighting}
\end{Shaded}
\item
  æ­£æ--‡ä¸­è¿žç»­å¼•ç''¨ï¼Œä¸Šæ~‡å?ˆå¹¶é''™è¯¯ï¼ˆä¾‹å¦‚,引ç''¨ 1 2 3 4
  åº''å½``显示为 {[}1-4{]} ,但是显示为 {[}1,4{]} )。

  \textbf{A:} 临æ---¶æ--¹æ¡ˆæ˜¯æŠŠ csl æ--‡ä»¶é‡Œ
  \texttt{\ after-collapse-delimiter=","\ } æ''¹æˆ?
  \texttt{\ after-collapse-delimiter="-"\ } 。本模�附带的 CSL
  æ--‡ä»¶å·²å?šæ­¤ä¿®æ''¹ã€‚

  详细原å›~请è§? \url{https://github.com/typst/hayagriva/issues/154}
  。

  \url{https://github.com/typst/hayagriva/pull/176} 正�试解决这一
  bug。 \textbf{该 bug ä¿®å¤?å?Žï¼Œè¯·å?Šæ---¶æ'¤é''€ä¸Šè¿°å¯¹ csl
  的临æ---¶ä¿®æ''¹ã€‚}
\end{enumerate}

\subsection{å?‹æƒ\ldots é``¾æŽ¥}\label{uxe5uxe6ux192uxe9uxbeuxe6ux17e}

\begin{itemize}
\tightlist
\item
  Typst Touying 东å?---大学主题幻ç?¯ç‰‡æ¨¡æ?¿ by QuadnucYard -
  \url{https://github.com/QuadnucYard/touying-theme-seu}
\item
  东å?---大学 Typst 本ç§`毕设模æ?¿ä¸Žæ¯•è®¾ç¿»è¯`模æ?¿ by Geary.Z
  (TideDra) - \url{https://github.com/TideDra/seu-thesis-typst}
\end{itemize}

\subsection{å¼€å?{}`与å??è®®}\label{uxe5uxbcuxe5uxe4ux17euxe5uxe8}

如果您在使ç''¨è¿‡ç¨‹ä¸­é?‡åˆ°ä»»ä½•é---®é¢˜ï¼Œè¯·æ??交
issue。本项目欢迎您的
PR。如果有å\ldots¶ä»--模æ?¿éœ€æ±‚也å?¯ä»¥åœ¨ issue 中æ??出。

除下述特殊说明的æ--‡ä»¶å¤--,此项目使ç''¨ MIT License 。

\begin{itemize}
\tightlist
\item
  \texttt{\ init-files/demo\_image/\ }
  路径下的æ--‡ä»¶æ?¥è‡ªä¸œå?---大学教务处本ç§`毕设模æ?¿ã€‚
\item
  \texttt{\ seu-thesis/assets/\ }
  路径下的æ--‡ä»¶æ˜¯ç''±ä¸œå?---大学教务处模æ?¿ç»?二次åŠ~å·¥å¾---到,æˆ--从东å?---大学视觉设计中å?--å¾---。
\item
  \texttt{\ fonts\ } 路径下的æ--‡ä»¶æ˜¯æ­¤æ¨¡æ?¿ç''¨åˆ°çš„å­---ä½``。
\item
  \texttt{\ 东�大学本科毕业设计(论文)�考模�\ (2024年1月修订).docx\ }
  是教务处æ??供的毕设论æ--‡æ¨¡æ?¿ã€‚
\end{itemize}

\subsubsection{二次开å?{}`}\label{uxe4uxbaux153uxe6uxe5uxbcuxe5}

本模æ?¿æ¬¢è¿ŽäºŒæ¬¡å¼€å?{}`。在二次开å?{}`å‰?,建议了解本模æ?¿çš„主è¦?特性与å\ldots³è?''çš„æ--‡ä»¶ï¼š

\begin{itemize}
\item
  有较为麻烦的图表ã€?å\ldots¬å¼?ç¼--å?·ï¼ˆå›¾è¡¨ç¼--å?·æ~¼å¼?ä¸?相å?Œï¼Œç''šè‡³é™„录与正æ--‡ä¸­å›¾è¡¨ç¼--å?·æ~¼å¼?也ä¸?相å?Œï¼›å›¾çš„å??称在图下æ--¹ï¼Œè¡¨çš„å??称在表上æ--¹ï¼›å\ldots¬å¼?ä¸?是å±\ldots 中对é½?,å\ldots¬å¼?ç¼--å?·ä½?ç½®ä¸?是å?³ä¾§ä¸Šä¸‹å±\ldots 中)。

  \begin{itemize}
  \tightlist
  \item
    å·²ç»?æ''¹ç''¨ \texttt{\ i-figured\ } åŒ\ldots 完æˆ?。
  \end{itemize}
\item
  (ä»\ldots ç~''究ç''Ÿå­¦ä½?论æ--‡ï¼‰å¥‡æ•°é¡µå?¶æ•°é¡µé¡µçœ‰ä¸?å?Œï¼Œä¸''有页眉中显示ç«~节å??称的需求。

  \begin{itemize}
  \tightlist
  \item
    该功能�于
    \texttt{\ seu-thesis/parts/main-body-degree-fn.typ\ } 。
  \item
    推è??æ''¹ç''¨ \texttt{\ chic-hdr\ }
    而ä¸?是自é€~è½®å­?,ç''±äºŽåŽ†å?²é?---ç•™é---®é¢˜æœ¬æ¨¡æ?¿æš‚未æ''¹ç''¨ã€‚
  \end{itemize}
\item
  æ''¯æŒ?å?Œè¯­æ˜¾ç¤ºå?‚考æ--‡çŒ®ï¼ˆè‡ªåŠ¨ä½¿ç''¨ \texttt{\ et\ al.\ }
  å'Œ \texttt{\ ç­‰\ } )

  \begin{itemize}
  \tightlist
  \item
    该功能�自 \texttt{\ bilingual-bibliography\ }
    ,å\ldots³è?''çš„æ--‡ä»¶æ˜¯
    \texttt{\ seu-thesis/utils/bilingual-bibliography.typ\ } 。
  \item
    有å\ldots³ \texttt{\ bilingual-bibliography\ }
    的更多信�,请查看
    \url{https://github.com/nju-lug/modern-nju-thesis/issues/3}
  \end{itemize}
\end{itemize}

\begin{quote}
{[}!NOTE{]}

本模æ?¿å†\ldots é€~çš„è½®å­?æ¯''较多,而ä¸''æˆ`的代ç~?è´¨é‡?一般,请é\ldots Œæƒ\ldots å?--ç''¨ã€‚
\end{quote}

\href{/app?template=cheda-seu-thesis&version=0.3.0}{Create project in
app}

\subsubsection{How to use}\label{how-to-use}

Click the button above to create a new project using this template in
the Typst app.

You can also use the Typst CLI to start a new project on your computer
using this command:

\begin{verbatim}
typst init @preview/cheda-seu-thesis:0.3.0
\end{verbatim}

\includesvg[width=0.16667in,height=0.16667in]{/assets/icons/16-copy.svg}

\subsubsection{About}\label{about}

\begin{description}
\tightlist
\item[Author :]
csimide
\item[License:]
MIT
\item[Current version:]
0.3.0
\item[Last updated:]
July 8, 2024
\item[First released:]
April 16, 2024
\item[Archive size:]
465 kB
\href{https://packages.typst.org/preview/cheda-seu-thesis-0.3.0.tar.gz}{\pandocbounded{\includesvg[keepaspectratio]{/assets/icons/16-download.svg}}}
\item[Repository:]
\href{https://github.com/csimide/SEU-Typst-Template}{GitHub}
\item[Categor y :]
\begin{itemize}
\tightlist
\item[]
\item
  \pandocbounded{\includesvg[keepaspectratio]{/assets/icons/16-mortarboard.svg}}
  \href{https://typst.app/universe/search/?category=thesis}{Thesis}
\end{itemize}
\end{description}

\subsubsection{Where to report issues?}\label{where-to-report-issues}

This template is a project of csimide . Report issues on
\href{https://github.com/csimide/SEU-Typst-Template}{their repository} .
You can also try to ask for help with this template on the
\href{https://forum.typst.app}{Forum} .

Please report this template to the Typst team using the
\href{https://typst.app/contact}{contact form} if you believe it is a
safety hazard or infringes upon your rights.

\phantomsection\label{versions}
\subsubsection{Version history}\label{version-history}

\begin{longtable}[]{@{}ll@{}}
\toprule\noalign{}
Version & Release Date \\
\midrule\noalign{}
\endhead
\bottomrule\noalign{}
\endlastfoot
0.3.0 & July 8, 2024 \\
\href{https://typst.app/universe/package/cheda-seu-thesis/0.2.2/}{0.2.2}
& May 23, 2024 \\
\href{https://typst.app/universe/package/cheda-seu-thesis/0.2.1/}{0.2.1}
& April 29, 2024 \\
\href{https://typst.app/universe/package/cheda-seu-thesis/0.2.0/}{0.2.0}
& April 16, 2024 \\
\end{longtable}

Typst GmbH did not create this template and cannot guarantee correct
functionality of this template or compatibility with any version of the
Typst compiler or app.


\title{typst.app/universe/package/chronos}

\phantomsection\label{banner}
\section{chronos}\label{chronos}

{ 0.2.0 }

A package to draw sequence diagrams with CeTZ

\phantomsection\label{readme}
A Typst package to draw sequence diagrams with CeTZ

\begin{center}\rule{0.5\linewidth}{0.5pt}\end{center}

This package lets you render sequence diagrams directly in Typst. The
following boilerplate code creates an empty sequence diagram with two
participants:

\begin{longtable}[]{@{}
  >{\raggedright\arraybackslash}p{(\linewidth - 2\tabcolsep) * \real{0.5000}}
  >{\raggedright\arraybackslash}p{(\linewidth - 2\tabcolsep) * \real{0.5000}}@{}}
\toprule\noalign{}
\endhead
\bottomrule\noalign{}
\endlastfoot
\textbf{Typst} & \textbf{Result} \\
\begin{minipage}[t]{\linewidth}\raggedright
\begin{Shaded}
\begin{Highlighting}[]
\NormalTok{\#import "@preview/chronos:0.2.0"}
\NormalTok{\#chronos.diagram(\{}
\NormalTok{  import chronos: *}
\NormalTok{  \_par("Alice")}
\NormalTok{  \_par("Bob")}
\NormalTok{\})}
\end{Highlighting}
\end{Shaded}
\end{minipage} &
\pandocbounded{\includegraphics[keepaspectratio]{https://github.com/typst/packages/raw/main/packages/preview/chronos/0.2.0/gallery/readme/boilerplate.png}} \\
\end{longtable}

\begin{quote}
\emph{Disclaimer}\\
The package cannot parse PlantUML syntax for the moment, and thus
requires the use of element functions, as shown in the examples. A
PlantUML parser is in the TODO list, just not the top priority
\end{quote}

\subsection{Basic sequences}\label{basic-sequences}

You can make basic sequences using the \texttt{\ \_seq\ } function:

\begin{longtable}[]{@{}
  >{\raggedright\arraybackslash}p{(\linewidth - 2\tabcolsep) * \real{0.5000}}
  >{\raggedright\arraybackslash}p{(\linewidth - 2\tabcolsep) * \real{0.5000}}@{}}
\toprule\noalign{}
\endhead
\bottomrule\noalign{}
\endlastfoot
\textbf{Typst} & \textbf{Result} \\
\begin{minipage}[t]{\linewidth}\raggedright
\begin{Shaded}
\begin{Highlighting}[]
\NormalTok{\#chronos.diagram(\{}
\NormalTok{  import chronos: *}
\NormalTok{  \_par("Alice")}
\NormalTok{  \_par("Bob")}

\NormalTok{  \_seq("Alice", "Bob", comment: "Hello")}
\NormalTok{  \_seq("Bob", "Bob", comment: "Think")}
\NormalTok{  \_seq("Bob", "Alice", comment: "Hi")}
\NormalTok{\})}
\end{Highlighting}
\end{Shaded}
\end{minipage} &
\pandocbounded{\includegraphics[keepaspectratio]{https://github.com/typst/packages/raw/main/packages/preview/chronos/0.2.0/gallery/readme/simple_sequence.png}} \\
\end{longtable}

You can make lifelines using the following parameters of the
\texttt{\ \_seq\ } function:

\begin{itemize}
\tightlist
\item
  \texttt{\ enable-dst\ } : enables the destination lifeline
\item
  \texttt{\ create-dst\ } : creates the destination lifeline and
  participant
\item
  \texttt{\ disable-dst\ } : disables the destination lifeline
\item
  \texttt{\ destroy-dst\ } : destroys the destination lifeline and
  participant
\item
  \texttt{\ disable-src\ } : disables the source lifeline
\item
  \texttt{\ destroy-src\ } : destroy the source lifeline and participant
\end{itemize}

\begin{longtable}[]{@{}
  >{\raggedright\arraybackslash}p{(\linewidth - 2\tabcolsep) * \real{0.5000}}
  >{\raggedright\arraybackslash}p{(\linewidth - 2\tabcolsep) * \real{0.5000}}@{}}
\toprule\noalign{}
\endhead
\bottomrule\noalign{}
\endlastfoot
\textbf{Typst} & \textbf{Result} \\
\begin{minipage}[t]{\linewidth}\raggedright
\begin{Shaded}
\begin{Highlighting}[]
\NormalTok{\#chronos.diagram(\{}
\NormalTok{  import chronos: *}
\NormalTok{  \_par("A", display{-}name: "Alice")}
\NormalTok{  \_par("B", display{-}name: "Bob")}
\NormalTok{  \_par("C", display{-}name: "Charlie")}
\NormalTok{  \_par("D", display{-}name: "Derek")}

\NormalTok{  \_seq("A", "B", comment: "hello", enable{-}dst: true)}
\NormalTok{  \_seq("B", "B", comment: "self call", enable{-}dst: true)}
\NormalTok{  \_seq("C", "B", comment: "hello from thread 2", enable{-}dst: true, lifeline{-}style: (fill: rgb("\#005500")))}
\NormalTok{  \_seq("B", "D", comment: "create", create{-}dst: true)}
\NormalTok{  \_seq("B", "C", comment: "done in thread 2", disable{-}src: true, dashed: true)}
\NormalTok{  \_seq("B", "B", comment: "rc", disable{-}src: true, dashed: true)}
\NormalTok{  \_seq("B", "D", comment: "delete", destroy{-}dst: true)}
\NormalTok{  \_seq("B", "A", comment: "success", disable{-}src: true, dashed: true)}
\NormalTok{\})}
\end{Highlighting}
\end{Shaded}
\end{minipage} &
\pandocbounded{\includegraphics[keepaspectratio]{https://github.com/typst/packages/raw/main/packages/preview/chronos/0.2.0/gallery/readme/lifelines.png}} \\
\end{longtable}

\subsection{Showcase}\label{showcase}

Several features have already been implemented in Chronos. Don’t
hesitate to checkout the examples in the
\href{https://github.com/typst/packages/raw/main/packages/preview/chronos/0.2.0/gallery}{gallery}
folder to see what you can do.

\paragraph{Quick example reference:}\label{quick-example-reference}

\begin{longtable}[]{@{}
  >{\raggedright\arraybackslash}p{(\linewidth - 2\tabcolsep) * \real{0.5000}}
  >{\raggedright\arraybackslash}p{(\linewidth - 2\tabcolsep) * \real{0.5000}}@{}}
\toprule\noalign{}
\endhead
\bottomrule\noalign{}
\endlastfoot
\textbf{Example} & \textbf{Features} \\
\begin{minipage}[t]{\linewidth}\raggedright
\texttt{\ example1\ }\strut \\
(
\href{https://github.com/typst/packages/raw/main/packages/preview/chronos/0.2.0/gallery/example1.pdf}{PDF}
\textbar{}
\href{https://github.com/typst/packages/raw/main/packages/preview/chronos/0.2.0/gallery/example1.typ}{Typst}
)\strut
\end{minipage} & Simple cases, color sequences, groups, separators,
gaps, self-sequences \\
\begin{minipage}[t]{\linewidth}\raggedright
\texttt{\ example2\ }\strut \\
(
\href{https://github.com/typst/packages/raw/main/packages/preview/chronos/0.2.0/gallery/example2.pdf}{PDF}
\textbar{}
\href{https://github.com/typst/packages/raw/main/packages/preview/chronos/0.2.0/gallery/example2.typ}{Typst}
)\strut
\end{minipage} & Lifelines, found/lost messages, synchronized sequences,
slanted sequences \\
\begin{minipage}[t]{\linewidth}\raggedright
\texttt{\ example3\ }\strut \\
(
\href{https://github.com/typst/packages/raw/main/packages/preview/chronos/0.2.0/gallery/example3.pdf}{PDF}
\textbar{}
\href{https://github.com/typst/packages/raw/main/packages/preview/chronos/0.2.0/gallery/example3.typ}{Typst}
)\strut
\end{minipage} & Participant shapes, sequence tips, hidden partipicant
ends \\
\begin{minipage}[t]{\linewidth}\raggedright
\texttt{\ notes\ }\strut \\
(
\href{https://github.com/typst/packages/raw/main/packages/preview/chronos/0.2.0/gallery/notes.pdf}{PDF}
\textbar{}
\href{https://github.com/typst/packages/raw/main/packages/preview/chronos/0.2.0/gallery/notes.typ}{Typst}
)\strut
\end{minipage} & Notes (duh), deferred participant creation \\
\end{longtable}

\begin{quote}
{[}!NOTE{]}

Many examples were taken/adapted from the PlantUML
\href{https://plantuml.com/sequence-diagram}{documentation} on sequence
diagrams
\end{quote}

\subsubsection{How to add}\label{how-to-add}

Copy this into your project and use the import as \texttt{\ chronos\ }

\begin{verbatim}
#import "@preview/chronos:0.2.0"
\end{verbatim}

\includesvg[width=0.16667in,height=0.16667in]{/assets/icons/16-copy.svg}

Check the docs for
\href{https://typst.app/docs/reference/scripting/\#packages}{more
information on how to import packages} .

\subsubsection{About}\label{about}

\begin{description}
\tightlist
\item[Author :]
\href{https://git.kb28.ch/HEL}{Louis Heredero}
\item[License:]
Apache-2.0
\item[Current version:]
0.2.0
\item[Last updated:]
November 12, 2024
\item[First released:]
October 1, 2024
\item[Minimum Typst version:]
0.12.0
\item[Archive size:]
537 kB
\href{https://packages.typst.org/preview/chronos-0.2.0.tar.gz}{\pandocbounded{\includesvg[keepaspectratio]{/assets/icons/16-download.svg}}}
\item[Repository:]
\href{https://git.kb28.ch/HEL/chronos}{git.kb28.ch}
\item[Categor y :]
\begin{itemize}
\tightlist
\item[]
\item
  \pandocbounded{\includesvg[keepaspectratio]{/assets/icons/16-chart.svg}}
  \href{https://typst.app/universe/search/?category=visualization}{Visualization}
\end{itemize}
\end{description}

\subsubsection{Where to report issues?}\label{where-to-report-issues}

This package is a project of Louis Heredero . Report issues on
\href{https://git.kb28.ch/HEL/chronos}{their repository} . You can also
try to ask for help with this package on the
\href{https://forum.typst.app}{Forum} .

Please report this package to the Typst team using the
\href{https://typst.app/contact}{contact form} if you believe it is a
safety hazard or infringes upon your rights.

\phantomsection\label{versions}
\subsubsection{Version history}\label{version-history}

\begin{longtable}[]{@{}ll@{}}
\toprule\noalign{}
Version & Release Date \\
\midrule\noalign{}
\endhead
\bottomrule\noalign{}
\endlastfoot
0.2.0 & November 12, 2024 \\
\href{https://typst.app/universe/package/chronos/0.1.0/}{0.1.0} &
October 1, 2024 \\
\end{longtable}

Typst GmbH did not create this package and cannot guarantee correct
functionality of this package or compatibility with any version of the
Typst compiler or app.


\title{typst.app/universe/package/game-theoryst}

\phantomsection\label{banner}
\section{game-theoryst}\label{game-theoryst}

{ 0.1.0 }

A package for typesetting games in Typst.

{ } Featured Package

\phantomsection\label{readme}
A package for typesetting games in Typst.

Full manual available
\href{https://github.com/typst/packages/raw/main/packages/preview/game-theoryst/0.1.0/doc/gtheoryst-manual.pdf}{here}

Work in progress â€`` \emph{coming soon!}

\subsection{Overview}\label{overview}

\paragraph{Simple Example}\label{simple-example}

The main function to make strategic (or \textbf{normal} ) form games is
\texttt{\ nfg\ } . For a basic 2x2 game, you can do

\begin{Shaded}
\begin{Highlighting}[]
\NormalTok{\#nfg(}
\NormalTok{  players: ("Jack", "Diane"),}
\NormalTok{  s1: ($C$, $D$),}
\NormalTok{  s2: ($C$, $D$),}
\NormalTok{  [$10, 10$], [$2, 20$], }
\NormalTok{  [$20, 2$], [$5, 5$],}
\NormalTok{)}
\end{Highlighting}
\end{Shaded}

\includegraphics[width=4.16667in,height=\textheight,keepaspectratio]{https://github.com/typst/packages/raw/main/packages/preview/game-theoryst/0.1.0/doc/gallery/simple-example.png}

\subsubsection{Importing}\label{importing}

Simply insert the following into your Typst code:

\begin{Shaded}
\begin{Highlighting}[]
\NormalTok{\#import "@preview/game{-}theoryst:0.1.0": *}
\end{Highlighting}
\end{Shaded}

This imports the \texttt{\ nfg()\ } function as well as the underlining
methods. If you want to tweak the helper functions for generating an
\texttt{\ nfg\ } , import them explicitly through the
\texttt{\ utils/\ } directory.

\paragraph{Full Example}\label{full-example}

\begin{Shaded}
\begin{Highlighting}[]
\NormalTok{\#nfg(}
\NormalTok{  players: ([A\textbackslash{} Joe], [Bas Pro]),}
\NormalTok{  s1: ([$x$], [a]),}
\NormalTok{  s2: ("x", "aaaa", [$a$]),}
\NormalTok{  pad: ("x": 12pt, "y": 10pt),}
\NormalTok{  eliminations: ("s11", "s21", "s22"),}
\NormalTok{  ejust: (}
\NormalTok{    s11: (x: (0pt, 36pt), y: ({-}3pt, {-}3.5pt)),}
\NormalTok{    s22: (x: ({-}10pt, {-}12pt), y: ({-}10pt, 10pt)),}
\NormalTok{    s21: (x: ({-}3pt, {-}9pt), y: ({-}10pt, 10pt)),}
\NormalTok{  ),}
\NormalTok{  mixings: (hmix: ($p$, $1{-}p$), vmix: ($q$, $r$, $1{-}q{-}r$)),}
\NormalTok{  custom{-}fills: (hp: maroon, vp: navy, hm: purple, vm: fuchsia, he: gray, ve: gray),}
\NormalTok{  [$0,vul(100000000)$], [$0,1$], [$0,0$],}
\NormalTok{  [$hul(1),1$], [$0, {-}1$], table.cell(fill: yellow.lighten(30\%), [$hful(0),vful(0)$])}
\NormalTok{)}
\end{Highlighting}
\end{Shaded}

\includegraphics[width=5.46875in,height=\textheight,keepaspectratio]{https://github.com/typst/packages/raw/main/packages/preview/game-theoryst/0.1.0/doc/gallery/full-example.png}

\subsubsection{Color}\label{color}

By default, player names, mixed-strategy parameters (called
\emph{mixings} ), and elimination lines are shown in color. These colors
can be turned off at the method-level by passing \texttt{\ bw:\ true\ }
, or at the document level by running the state helper-function
\texttt{\ \#colorless()\ } .

\texttt{\ nfg\ } accepts custom colors for all of the aforementioned
parameters by passing a \texttt{\ dictionary\ } of colors to the
\texttt{\ custom-fills\ } arg. The keys for this dictionary are as
follows ( \texttt{\ \textless{}defaults\textgreater{}\ } ):

\begin{itemize}
\tightlist
\item
  \texttt{\ hp\ } â€`` “horizontal playerâ€? (red)
\item
  \texttt{\ vp\ } â€`` “vertical playerâ€? (blue)
\item
  \texttt{\ hm\ } â€`` “hor. mixingâ€? (\#e64173)
\item
  \texttt{\ vm\ } â€`` “ver. mixingâ€? (eastern)
\item
  \texttt{\ he\ } â€`` “hor. eliminationâ€? line (orange)
\item
  \texttt{\ ve\ } â€`` “ver. eliminationâ€? line (olive)
\end{itemize}

\subsection{Cell Customization}\label{cell-customization}

Since the payoffs are implemented as argument sinks (
\texttt{\ ..args\ } ) which are passed directly to Typst’s
\texttt{\ \#table()\ } , underlining of non-math can be accomplished via
the standard \texttt{\ \#underline()\ } command. Similarly, any of the
payoff cells can be customized by using \texttt{\ table.cell()\ }
directly. For instance,
\texttt{\ table.cell(fill:\ yellow.lighten(30\%),\ {[}\$1,\ 1\${]})\ }
can be used to highlight a specific cell.

\subsubsection{Padding}\label{padding}

There are edge cases where the default padding may be off. These can be
mended by passing the optional \texttt{\ pad\ } argument to
\texttt{\ nfg()\ } . This should represent how much
\textbf{\emph{additional}} padding you want. The \texttt{\ pad\ } arg.
is interpreted as follows:

\begin{itemize}
\tightlist
\item
  If a \texttt{\ length\ } is provided, it assumes you want that much
  length added to all cell walls
\item
  If an array of the form \texttt{\ (L1,\ L2)\ } is provided, it assumes
  you want padding a horizontal ( \texttt{\ x\ } ) padding of
  \texttt{\ L1\ } and a vertical padding ( \texttt{\ y\ } ) of
  \texttt{\ L2\ }
\item
  If a \texttt{\ dictionary\ } is provided, it operates identically to
  that of the array, but you must specify the \texttt{\ x\ } /
  \texttt{\ y\ } keys yourself
\end{itemize}

\subsubsection{AUtomatic Cell Sizing}\label{automatic-cell-sizing}

Cell are automatically sized to equal heights/widths according to the
longest/tallest content. If you want to avoid this behavior, pass
\texttt{\ lazy-cells:\ true\ } to \texttt{\ nfg\ } . This behavior can
be combined with the custom \texttt{\ padding\ } argument.

\subsection{Semantic Game Theory
Features}\label{semantic-game-theory-features}

\subsubsection{Underlining}\label{underlining}

The package imports a small set of underlining utility functions.

The primary functions for underlining are

\begin{itemize}
\tightlist
\item
  \texttt{\ hul()\ } â€`` \emph{Horizontal Underline}
\item
  \texttt{\ vul()\ } â€`` \emph{Vertical Underline}
\item
  \texttt{\ bul()\ } â€`` \emph{Black Underline} These can be wrapped
  around values in math-mode ( \texttt{\ \$..\$\ } ) within the payoff
  matrix. The underlines for \texttt{\ hul\ } and \texttt{\ vul\ } are
  colored by default according to the default colors for names, but they
  accept an optional \texttt{\ col\ } parameter for changing the color
  of the underline. \texttt{\ bul()\ } produces a black underline.
\end{itemize}

\begin{Shaded}
\begin{Highlighting}[]
\NormalTok{\#nfg(}
\NormalTok{  players: ("Jack", "Diane"),}
\NormalTok{  s2: ($x$, $y$, $z$),}
\NormalTok{  s1: ($a$, $b$),}
\NormalTok{  [$hul(0),vul(0)$], [$1,1$], [$2,2$],}
\NormalTok{  [$3,3$], [$4,4$], [$5,5$],}
\NormalTok{)}
\end{Highlighting}
\end{Shaded}

By default, these commands leave the numbers themselves black, but
boldfaces them. \emph{Full Color} versions of \texttt{\ hul\ } and
\texttt{\ vul\ } , which color the numbers and under-lines identically,
are available via \texttt{\ hful()\ } and \texttt{\ vful()\ } . Like
their counterparts, they accept an optional \texttt{\ col\ } command for
the color.

Both of the colors can be modified individually via the general
\texttt{\ cul()\ } command, which takes in content ( \texttt{\ cont\ }
), an underline color ( \texttt{\ ucol\ } ), and the color for the text
value ( \texttt{\ tcol\ } ). For instance,

\begin{Shaded}
\begin{Highlighting}[]
\NormalTok{\#let new{-}ul(cont, col: olive, tcol: fuchsia) = \{ cul(cont, col, tcol) \}}
\end{Highlighting}
\end{Shaded}

will define a new command which underlines in olive and sets the text
(math) color to fuchsia.

\subsubsection{Mixed Strategies}\label{mixed-strategies}

You can optionally mark mixed strategies that a player will in a
\texttt{\ nfg\ } using the \texttt{\ mixing\ } argument. This can be a
\texttt{\ dictionary\ } with \texttt{\ hmix\ } and \texttt{\ vmix\ }
keys, or an \texttt{\ array\ } , interpreted as a dictionary with the
aforementioned keys in the \texttt{\ (hmix,\ vmix)\ } order. The
values/entries here should be arrays which mimic \texttt{\ s1\ } and
\texttt{\ s2\ } in size, with some parameter denoting the proportion of
time the relevant player uses that strategy. If you would like to omit a
strategy from this markup, pass \texttt{\ {[}{]}\ } in it’s place.

For example:

\begin{Shaded}
\begin{Highlighting}[]
\NormalTok{\#nfg(}
\NormalTok{  players: ("Chet", "North"),}
\NormalTok{  s1: ([$F$], [$G$], [$H$]),}
\NormalTok{  s2: ([$X$], [$Y$]),}
\NormalTok{  mixings: (}
\NormalTok{    hmix: ($p$, $1{-}p$), }
\NormalTok{    vmix: ($q$, [], $1{-}q$)),}
\NormalTok{  [$7,3$], [$2,4$], }
\NormalTok{  [$5,2$], [$6,1$], }
\NormalTok{  [$6,1$], [$5,4$]}
\NormalTok{)}
\end{Highlighting}
\end{Shaded}

\includegraphics[width=4.16667in,height=\textheight,keepaspectratio]{https://github.com/typst/packages/raw/main/packages/preview/game-theoryst/0.1.0/doc/gallery/mix-ex.png}

\subsubsection{Iterated Deletion (Elimination) of Dominated
Strategies}\label{iterated-deletion-elimination-of-dominated-strategies}

You can use the \texttt{\ pinit\ } package to cross out lines,
semantically eliminating strategies. \texttt{\ pinit\ } comes
pre-imported with \texttt{\ game-theoryst\ } by default.

You can tell \texttt{\ nfg\ } which strategies to eliminate with the
\texttt{\ eliminations\ } argument and the corresponding
\texttt{\ ejust\ } helper-argument. The \texttt{\ eliminations\ }
argument is simply an \texttt{\ array\ } of \texttt{\ strings\ } of the
form
\texttt{\ "s\textless{}i\textgreater{}\textless{}j\textgreater{}"\ } ,
where \texttt{\ \textless{}i\textgreater{}\ } is the player â€`` 1 or 2
â€`` and \texttt{\ \textless{}j\textgreater{}\ } is player
\texttt{\ i\ } ’s \texttt{\ \textless{}j\textgreater{}\ } th strategy,
in left-to-right / top-to-bottom order \emph{starting at 1} . These
strategy strings represent the rows/columns which you want to eliminate.
For instance, \texttt{\ ("s12",\ "s21")\ } denotes an elimination of
player 1’s second strategy as well as player 2’s first strategy.

Due to \texttt{\ context\ } dependence, the lines typically need manual
adjustments, which can be done via the \texttt{\ ejust\ } arg.
\texttt{\ ejust\ } needs to be a dictionary with keys of matching those
strings present in \texttt{\ eliminations\ } ( \texttt{\ s11\ } ,
\texttt{\ s21\ } , etc.). The values of one of these dictionary entries
is itself a dictionary: one with \texttt{\ x\ } and \texttt{\ y\ } keys.
Each of these keys needs an array consisting of 2 lengths, corresponding
to the starting/ending \texttt{\ dx/dy\ } adjustments from
\texttt{\ pinit-line\ } .

For example, one such \texttt{\ ejust\ } argument could be
\texttt{\ ("s12":\ (x:\ (5pt,\ -5pt),\ y:\ (-10pt,\ 3pt)))\ } . This
says to adjust the “s12� elimination line by \texttt{\ 5pt\ } in the
x direction and \texttt{\ -10pt\ } in the y direction for the starting
(strategy-) side of the line, and adjust by \texttt{\ -5pt\ } in x and
\texttt{\ 3pt\ } in y on the ending (payoff-) side of the line.

\begin{Shaded}
\begin{Highlighting}[]
\NormalTok{\#let just{-}arr = (}
\NormalTok{    "s12": (x: (0pt, 10pt), y: ({-}3pt, {-}3pt)),}
\NormalTok{    "s13": (x: (0pt, 10pt), y: ({-}3pt, {-}3pt)),}
\NormalTok{    "s14": (x: (0pt, 10pt), y: ({-}3pt, {-}3pt)),}
\NormalTok{    "s21": (x: ({-}6pt, {-}8pt), y: (3pt, 8pt)),}
\NormalTok{    "s22": (x: ({-}4pt, {-}8pt), y: (3pt, 8pt)),}
\NormalTok{    "s23": (x: ({-}4pt, {-}8pt), y: (3pt, 8pt)),}
\NormalTok{)}

\NormalTok{\#nfg(}
\NormalTok{  players: ("A", "B"),}
\NormalTok{  s1: ([$N$], [$S$], [$E$], [$W$] ),}
\NormalTok{  s2: ([$W$], [$E$], [$F$], [$A$]),}
\NormalTok{  eliminations: ("s12", "s13", "s14", "s21", "s22", "s23"),}
\NormalTok{  ejust: just{-}arr,}
\NormalTok{  [$6,4$], [$7,3$], [$5,5$], [$6,6$],}
\NormalTok{  [$7,3$], [$2,7$], [$4,6$], [$5,5$],}
\NormalTok{  [$8,2$], [$6,4$], [$3,7$], [$2,8$],}
\NormalTok{  [$3,7$], [$5,5$], [$4,6$], [$5,5$],}
\NormalTok{)}
\end{Highlighting}
\end{Shaded}

\includegraphics[width=4.16667in,height=\textheight,keepaspectratio]{https://github.com/typst/packages/raw/main/packages/preview/game-theoryst/0.1.0/doc/gallery/elim-ex.png}

\subsubsection{Debugging}\label{debugging}

If you want to see all of the lines for the table, including the ones
for a players, strategies, and mixings, set the following at the top of
your document.

\begin{Shaded}
\begin{Highlighting}[]
\NormalTok{\#set table.cell(stroke: (thickness: auto))}
\end{Highlighting}
\end{Shaded}

Note that cells are always present for mixings, they just have 0
width/height when no mixings of a specific variety are provided.

\subsection{License}\label{license}

game-theoryst Copyright © 2024 Connor T. Wiegand

This program is free software: you can redistribute it and/or modify it
under the terms of the GNU Affero General Public License as published by
the Free Software Foundation, either version 3 of the License, or (at
your option) any later version.

This program is distributed in the hope that it will be useful, but
WITHOUT ANY WARRANTY; without even the implied warranty of
MERCHANTABILITY or FITNESS FOR A PARTICULAR PURPOSE. See the GNU Affero
General Public License for more details.

You should have received a copy of the GNU Affero General Public License
along with this program. If not, see
\href{http://www.gnu.org/licenses/}{http:www.gnu.org/licenses/} .

\subsubsection{How to add}\label{how-to-add}

Copy this into your project and use the import as
\texttt{\ game-theoryst\ }

\begin{verbatim}
#import "@preview/game-theoryst:0.1.0"
\end{verbatim}

\includesvg[width=0.16667in,height=0.16667in]{/assets/icons/16-copy.svg}

Check the docs for
\href{https://typst.app/docs/reference/scripting/\#packages}{more
information on how to import packages} .

\subsubsection{About}\label{about}

\begin{description}
\tightlist
\item[Author :]
\href{https://github.com/connortwiegand}{Connor T. Wiegand}
\item[License:]
AGPL-3.0-only
\item[Current version:]
0.1.0
\item[Last updated:]
August 14, 2024
\item[First released:]
August 14, 2024
\item[Archive size:]
18.9 kB
\href{https://packages.typst.org/preview/game-theoryst-0.1.0.tar.gz}{\pandocbounded{\includesvg[keepaspectratio]{/assets/icons/16-download.svg}}}
\item[Repository:]
\href{https://github.com/connortwiegand/game-theoryst}{GitHub}
\item[Discipline s :]
\begin{itemize}
\tightlist
\item[]
\item
  \href{https://typst.app/universe/search/?discipline=economics}{Economics}
\item
  \href{https://typst.app/universe/search/?discipline=education}{Education}
\item
  \href{https://typst.app/universe/search/?discipline=mathematics}{Mathematics}
\end{itemize}
\end{description}

\subsubsection{Where to report issues?}\label{where-to-report-issues}

This package is a project of Connor T. Wiegand . Report issues on
\href{https://github.com/connortwiegand/game-theoryst}{their repository}
. You can also try to ask for help with this package on the
\href{https://forum.typst.app}{Forum} .

Please report this package to the Typst team using the
\href{https://typst.app/contact}{contact form} if you believe it is a
safety hazard or infringes upon your rights.

\phantomsection\label{versions}
\subsubsection{Version history}\label{version-history}

\begin{longtable}[]{@{}ll@{}}
\toprule\noalign{}
Version & Release Date \\
\midrule\noalign{}
\endhead
\bottomrule\noalign{}
\endlastfoot
0.1.0 & August 14, 2024 \\
\end{longtable}

Typst GmbH did not create this package and cannot guarantee correct
functionality of this package or compatibility with any version of the
Typst compiler or app.


\title{typst.app/universe/package/chicv}

\phantomsection\label{banner}
\phantomsection\label{template-thumbnail}
\pandocbounded{\includegraphics[keepaspectratio]{https://packages.typst.org/preview/thumbnails/chicv-0.1.0-small.webp}}

\section{chicv}\label{chicv}

{ 0.1.0 }

A minimal and fully-customizable CV template.

\href{/app?template=chicv&version=0.1.0}{Create project in app}

\phantomsection\label{readme}
A simple CV template for \href{https://typst.app/}{typst.app} .

\subsection{How To Use}\label{how-to-use}

\subsubsection{Quick Start}\label{quick-start}

Create a project on \href{https://typst.app/}{typst.app} , copy paste
everything in
\url{https://github.com/skyzh/chicv/blob/master/template/cv.typ} . All
done!

\subsubsection{Customize your CV}\label{customize-your-cv}

To change the text size, you can uncomment the lines in
\texttt{\ cv.typ\ } and set to your choice. (Recommended text size for
CV is from 10pt to 12pt)

You can also change the page margin in \texttt{\ cv.typ\ } to fit in
more contents in a single page. The margin default is set to
\texttt{\ (x:\ 0.9cm,\ y:\ 1.3cm)\ } .

Don’t forget to include \texttt{\ \#chiline()\ } every time you open a
new section, this line acts as a perfect split.

For basic typst syntax, check this template as a reference, it’s super
easy to understand and use!

For advanced topics, please refer to
\href{https://typst.app/docs/reference/}{official reference} by typst.

\subsection{Showcases}\label{showcases}

\subsubsection{Sample CV}\label{sample-cv}

\pandocbounded{\includegraphics[keepaspectratio]{https://github.com/typst/packages/raw/main/packages/preview/chicv/0.1.0/cv.png}}

\subsubsection{Chi’s CV}\label{chiuxe2s-cv}

\href{https://skyzh.github.io/files/cv.pdf}{cv.pdf}

\href{/app?template=chicv&version=0.1.0}{Create project in app}

\subsubsection{How to use}\label{how-to-use-1}

Click the button above to create a new project using this template in
the Typst app.

You can also use the Typst CLI to start a new project on your computer
using this command:

\begin{verbatim}
typst init @preview/chicv:0.1.0
\end{verbatim}

\includesvg[width=0.16667in,height=0.16667in]{/assets/icons/16-copy.svg}

\subsubsection{About}\label{about}

\begin{description}
\tightlist
\item[Author :]
@skyzh
\item[License:]
MIT
\item[Current version:]
0.1.0
\item[Last updated:]
June 5, 2024
\item[First released:]
June 5, 2024
\item[Archive size:]
4.79 kB
\href{https://packages.typst.org/preview/chicv-0.1.0.tar.gz}{\pandocbounded{\includesvg[keepaspectratio]{/assets/icons/16-download.svg}}}
\item[Repository:]
\href{https://github.com/skyzh/chicv}{GitHub}
\item[Categor y :]
\begin{itemize}
\tightlist
\item[]
\item
  \pandocbounded{\includesvg[keepaspectratio]{/assets/icons/16-user.svg}}
  \href{https://typst.app/universe/search/?category=cv}{CV}
\end{itemize}
\end{description}

\subsubsection{Where to report issues?}\label{where-to-report-issues}

This template is a project of @skyzh . Report issues on
\href{https://github.com/skyzh/chicv}{their repository} . You can also
try to ask for help with this template on the
\href{https://forum.typst.app}{Forum} .

Please report this template to the Typst team using the
\href{https://typst.app/contact}{contact form} if you believe it is a
safety hazard or infringes upon your rights.

\phantomsection\label{versions}
\subsubsection{Version history}\label{version-history}

\begin{longtable}[]{@{}ll@{}}
\toprule\noalign{}
Version & Release Date \\
\midrule\noalign{}
\endhead
\bottomrule\noalign{}
\endlastfoot
0.1.0 & June 5, 2024 \\
\end{longtable}

Typst GmbH did not create this template and cannot guarantee correct
functionality of this template or compatibility with any version of the
Typst compiler or app.


\title{typst.app/universe/package/minienvs}

\phantomsection\label{banner}
\section{minienvs}\label{minienvs}

{ 0.1.0 }

Theorem environments with minimal fuss

\phantomsection\label{readme}
Theorem environments in \href{https://typst.app/}{Typst} with minimal
fuss.

To use, import and add a show rule:

\begin{Shaded}
\begin{Highlighting}[]
\NormalTok{\#import "@preview/minienvs:0.1.0": *}
\NormalTok{\#show: minienvs}
\end{Highlighting}
\end{Shaded}

You can optionally pass a custom configuration in the show-rule via
\texttt{\ minienvs.with(…)\ } (see
\href{https://github.com/typst/packages/raw/main/packages/preview/minienvs/0.1.0/\#customization}{Customization}
).

You can now just add a theorem along with its proof using the term list
syntax. For example:

\begin{Shaded}
\begin{Highlighting}[]
\NormalTok{/ Theorem (Ville\textquotesingle{}s inequality):}
\NormalTok{  Let $X\_0, ...$ be a non{-}negative supermartingale. Then, for any real number $a \textgreater{} 0$,}

\NormalTok{  $ PP[sup\_(n\textgreater{}=0) X\_n \textgreater{}= a] \textless{}= EE[X\_0]/a. $}

\NormalTok{Let us now prove it:}

\NormalTok{/ Proof:}
\NormalTok{  Consider the stopping time $N = inf \{t \textgreater{}= 1 : X\_t \textgreater{}= a\}$.}
\NormalTok{  By the optional stopping theorem and the supermartingale convergence theorem, we have that}

\NormalTok{  $}
\NormalTok{    EE[X\_0] \textgreater{}= EE[X\_N]}
\NormalTok{    \&= EE[X\_N | N \textless{} oo] PP[N \textless{} oo] + EE[X\_oo | N = oo] PP[N = oo] \textbackslash{}}
\NormalTok{    \&\textgreater{}= EE[X\_N | N \textless{} oo] PP[N \textless{} oo]}
\NormalTok{    = EE[X\_N/a | N \textless{} oo] a PP[N \textless{} oo]. \textbackslash{}}
\NormalTok{  $}

\NormalTok{  And, therefore,}

\NormalTok{  $ PP[N \textless{} oo] \textless{}= EE[X\_0] \textbackslash{}/ a EE[X\_N/a | N \textless{} oo] \textless{}= EE[X\_0] \textbackslash{}/ a. $}
\end{Highlighting}
\end{Shaded}

\pandocbounded{\includegraphics[keepaspectratio]{https://github.com/typst/packages/raw/main/packages/preview/minienvs/0.1.0/assets/ville.png}}

\subsection{Labels and references}\label{labels-and-references}

Currently, in order to label a minienv one needs to use the
\texttt{\ envlabel\ } function. For example:

\begin{Shaded}
\begin{Highlighting}[]
\NormalTok{/ Lemma (Donsker and Varadhan\textquotesingle{}s variational formula) \#envlabel(\textless{}change{-}of{-}measure\textgreater{}):}
\NormalTok{  For any measureable, bounded function $h : Theta {-}\textgreater{} RR$ we have:}

\NormalTok{  $ log EE\_(theta \textasciitilde{} pi)[exp h(theta)] = sup\_(rho in cal(P)(Theta)) [ EE\_(theta\textasciitilde{}rho)[h(theta)] {-} KL(rho || pi) ]. $}

\NormalTok{As we will see, @change{-}of{-}measure is a fundamental building block of PAC{-}Bayes bounds.}
\end{Highlighting}
\end{Shaded}

\pandocbounded{\includegraphics[keepaspectratio]{https://github.com/typst/packages/raw/main/packages/preview/minienvs/0.1.0/assets/donsker-varadhan.png}}

\subsection{Customization}\label{customization}

You can customize the appearance of minienvs by providing a
configuration to the show-rule. For example, for the default
configuration, you can do:

\begin{Shaded}
\begin{Highlighting}[]
\NormalTok{\#show: minienvs.with(config: (}
\NormalTok{  // Whether to give numbers for environments.}
\NormalTok{  // If the environment is not mentioned in this dict, it has a number.}
\NormalTok{  no{-}numbering: (}
\NormalTok{    proof: true,}
\NormalTok{  ),}
\NormalTok{  // Additional options for the \textasciigrave{}block\textasciigrave{} containing the minienv (e.g., to put a box around the minienv).}
\NormalTok{  // If the environment is not mentioned in this dict, no additional options are passed.}
\NormalTok{  bbox: (:),}
\NormalTok{  // How to format the head of the minienv.}
\NormalTok{  // If the environment is not mentioned in this dict, then it is formatted in bold.}
\NormalTok{  head{-}style: (}
\NormalTok{    proof: it =\textgreater{} [\_\#\{it\}\_],}
\NormalTok{  ),}
\NormalTok{  // How to format the body of the minienv.}
\NormalTok{  // If the environment is not mentioned in this dict, then it is formatted in italic.}
\NormalTok{  transforms: (}
\NormalTok{    proof: it =\textgreater{} [\#it \#h(1fr) $space qed$],}
\NormalTok{  )}
\NormalTok{))}
\end{Highlighting}
\end{Shaded}

\subsection{Coming soon / Work in
progress}\label{coming-soon-work-in-progress}

\begin{itemize}
\tightlist
\item
  Presets for multiple languages
\item
  Separate counters
\item
  More customization
\end{itemize}

\subsubsection{How to add}\label{how-to-add}

Copy this into your project and use the import as \texttt{\ minienvs\ }

\begin{verbatim}
#import "@preview/minienvs:0.1.0"
\end{verbatim}

\includesvg[width=0.16667in,height=0.16667in]{/assets/icons/16-copy.svg}

Check the docs for
\href{https://typst.app/docs/reference/scripting/\#packages}{more
information on how to import packages} .

\subsubsection{About}\label{about}

\begin{description}
\tightlist
\item[Author :]
Daniel Csillag
\item[License:]
MIT
\item[Current version:]
0.1.0
\item[Last updated:]
December 14, 2023
\item[First released:]
December 14, 2023
\item[Archive size:]
3.13 kB
\href{https://packages.typst.org/preview/minienvs-0.1.0.tar.gz}{\pandocbounded{\includesvg[keepaspectratio]{/assets/icons/16-download.svg}}}
\end{description}

\subsubsection{Where to report issues?}\label{where-to-report-issues}

This package is a project of Daniel Csillag . You can also try to ask
for help with this package on the \href{https://forum.typst.app}{Forum}
.

Please report this package to the Typst team using the
\href{https://typst.app/contact}{contact form} if you believe it is a
safety hazard or infringes upon your rights.

\phantomsection\label{versions}
\subsubsection{Version history}\label{version-history}

\begin{longtable}[]{@{}ll@{}}
\toprule\noalign{}
Version & Release Date \\
\midrule\noalign{}
\endhead
\bottomrule\noalign{}
\endlastfoot
0.1.0 & December 14, 2023 \\
\end{longtable}

Typst GmbH did not create this package and cannot guarantee correct
functionality of this package or compatibility with any version of the
Typst compiler or app.


\title{typst.app/universe/package/grape-suite}

\phantomsection\label{banner}
\phantomsection\label{template-thumbnail}
\pandocbounded{\includegraphics[keepaspectratio]{https://packages.typst.org/preview/thumbnails/grape-suite-1.0.0-small.webp}}

\section{grape-suite}\label{grape-suite}

{ 1.0.0 }

Library of templates for exams, seminar papers, homeworks, etc.

{ } Featured Template

\href{/app?template=grape-suite&version=1.0.0}{Create project in app}

\phantomsection\label{readme}
The grape suite is a suite consisting of following templates:

\begin{itemize}
\item
  exercises (for exams, homework, etc.)
\item
  seminar papers
\item
  slides (using polylux)
\end{itemize}

\subsection{Exercises}\label{exercises}

\subsubsection{Setup}\label{setup}

\begin{Shaded}
\begin{Highlighting}[]
\NormalTok{\#import "@preview/grape{-}suite:1.0.0": exercise}
\NormalTok{\#import exercise: project, task, subtask}

\NormalTok{\#show: project.with(}
\NormalTok{    title: "Lorem ipsum dolor sit",}

\NormalTok{    university: [University],}
\NormalTok{    institute: [Institute],}
\NormalTok{    seminar: [Seminar],}

\NormalTok{    abstract: lorem(100),}
\NormalTok{    show{-}outline: true,}

\NormalTok{    author: "John Doe",}

\NormalTok{    show{-}solutions: false}
\NormalTok{)}
\end{Highlighting}
\end{Shaded}

\subsubsection{API-Documentation}\label{api-documentation}

\begin{longtable}[]{@{}ll@{}}
\toprule\noalign{}
\texttt{\ project\ } & \\
\midrule\noalign{}
\endhead
\bottomrule\noalign{}
\endlastfoot
\texttt{\ no\ } & optional, number, default: \texttt{\ none\ } , number
of the sheet in the series \\
\texttt{\ type\ } & optional, content, default: \texttt{\ {[}Exam{]}\ }
, type of the series, eg. exam, homework, protocol, … \\
\texttt{\ title\ } & optional, content, default: \texttt{\ none\ } ,
title of the document: if none, then generated from no, type and
suffix-title \\
\texttt{\ suffix-title\ } & optional, content, default:
\texttt{\ none\ } , used if title is none to generate the title of the
document \\
\texttt{\ show-outline\ } & optional, bool, default: \texttt{\ false\ }
, show outline after title iff true \\
\texttt{\ abstract\ } & optional, content, default: \texttt{\ none\ } ,
show abstract between outline and title \\
\texttt{\ document-title\ } & optional, content, default:
\texttt{\ none\ } , shown in the upper right corner of the page header:
if none, \texttt{\ title\ } is used \\
\texttt{\ show-hints\ } & optional, bool, default: \texttt{\ false\ } ,
generate hints from tasks iff true \\
\texttt{\ show-solutions\ } & optional, bool, default:
\texttt{\ false\ } , generate solutions from tasks iff true \\
\texttt{\ show-namefield\ } & optional, bool, default:
\texttt{\ false\ } , show namefield at the end of the left header iff
true \\
\texttt{\ namefield\ } & optional, content, default:
\texttt{\ {[}Name:{]}\ } , content shown iff
\texttt{\ show-namefield\ } \\
\texttt{\ show-timefield\ } & optional, bool, default:
\texttt{\ false\ } , show timefield at the end of right header iff
true \\
\texttt{\ timefield\ } & optional, function, default:
\texttt{\ (time)\ =\textgreater{}\ {[}Time:\ \#time\ min.{]}\ } , to
generate the content shown as the timefield iff
\texttt{\ show-timefield\ } is true \\
\texttt{\ max-time\ } & optional, number, default: \texttt{\ 0\ } , time
value used in the \texttt{\ timefield\ } function generateor \\
\texttt{\ show-lines\ } & optional, bool, default: \texttt{\ false\ } ,
draw automatic lines for each task, if \texttt{\ lines\ } parameter of
\texttt{\ task\ } is set \\
\texttt{\ show-point-distribution-in-tasks\ } & optional, bool, default:
\texttt{\ false\ } , show point distribution after tasks iff true \\
\texttt{\ show-point-distribution-in-solutions\ } & optional, bool,
default: \texttt{\ false\ } , show point distributions after solutions
iff true \\
\texttt{\ solutions-as-matrix\ } & optional, bool, default:
\texttt{\ false\ } , show solutions as a matrix iff true, \textbf{mind
that} : now the solution parameter of task expects a list of 2-tuples,
where the first element of the 2-tuple is the amount of points, a number
and the second element is content, how to achieve all points \\
\texttt{\ university\ } & optional, content, default:
\texttt{\ none\ } \\
\texttt{\ faculty\ } & optional, content, default: \texttt{\ none\ } \\
\texttt{\ institute\ } & optional, content, default:
\texttt{\ none\ } \\
\texttt{\ seminar\ } & optional, content, default: \texttt{\ none\ } \\
\texttt{\ semester\ } & optional, content, default: \texttt{\ none\ } \\
\texttt{\ docent\ } & optional, content, default: \texttt{\ none\ } \\
\texttt{\ author\ } & optional, content, default: \texttt{\ none\ } \\
\texttt{\ date\ } & optional, datetime, default:
\texttt{\ datetime.today()\ } \\
\texttt{\ header\ } & optional, content, default: \texttt{\ none\ } ,
overwrite page header \\
\texttt{\ header-right\ } & optional, content, default:
\texttt{\ none\ } , overwrite right header part \\
\texttt{\ header-middle\ } & optional, content, default:
\texttt{\ none\ } , overwrite middle header part \\
\texttt{\ header-left\ } & optional, content, default: \texttt{\ none\ }
, overwrite left header part \\
\texttt{\ footer\ } & optional, content, default: \texttt{\ none\ } ,
overwrite footer part \\
\texttt{\ footer-right\ } & optional, content, default:
\texttt{\ none\ } , overwrite right footer part \\
\texttt{\ footer-middle\ } & optional, content, default:
\texttt{\ none\ } , overwrite middle footer part \\
\texttt{\ footer-left\ } & optional, content, default: \texttt{\ none\ }
, overwrite left footer part \\
\texttt{\ task-type\ } & optional, content, default:
\texttt{\ {[}Task{]}\ } , content shown in task title box before
numbering \\
\texttt{\ extra-task-type\ } & optional, content, default:
\texttt{\ {[}Extra\ task{]}\ } , for tasks where the \texttt{\ extra\ }
parameter is true, content shown in title box before numbering \\
\texttt{\ box-task-title\ } & optional, content, default:
\texttt{\ {[}Task{]}\ } , shown as the title of a task box used by the
\texttt{\ slides\ } library \\
\texttt{\ box-hint-title\ } & optional, content, default:
\texttt{\ {[}Hint{]}\ } , shown as the title of a tasks colored hint
box \\
\texttt{\ box-solution-title\ } & optional, content, default:
\texttt{\ {[}Solution{]}\ } , shown as the title of a tasks colored
solution box \\
\texttt{\ box-definition-title\ } & optional, content, default:
\texttt{\ {[}Definition{]}\ } , shown as the title of a definition box
used by the \texttt{\ slides\ } library \\
\texttt{\ box-notice-title\ } & optional, content, default:
\texttt{\ {[}Notice{]}\ } , shown as the title of a notice box used by
the \texttt{\ slides\ } library \\
\texttt{\ box-example-title\ } & optional, content, default:
\texttt{\ {[}Example{]}\ } , shown as the title of a example box used by
the \texttt{\ slides\ } library \\
\texttt{\ hint-type\ } & optional, content, default:
\texttt{\ {[}Hint{]}\ } , title of a tasks hint version \\
\texttt{\ hints-title\ } & optional, content, default:
\texttt{\ {[}Hints{]}\ } , title of the hints section \\
\texttt{\ solution-type\ } & optional, content, default:
\texttt{\ {[}Suggested\ solution{]}\ } , title of a tasks solution
version \\
\texttt{\ solutions-title\ } & optional, content, default:
\texttt{\ {[}Suggested\ solutions{]}\ } , title of the solutions
section \\
\texttt{\ solution-matrix-task-header\ } & optional, content, default:
\texttt{\ {[}Tasks{]}\ } , first column header of solution matrix,
column contains the reasons on how to achieve the points \\
\texttt{\ solution-matrix-achieved-points-header\ } & optional, content,
default: \texttt{\ {[}Points\ achieved{]}\ } , second column header of
solution matrix, column contains the points the one achieved \\
\texttt{\ show-solution-matrix-comment-field\ } & optional, bool,
default: \texttt{\ false\ } , show comment field in solution matrix \\
\texttt{\ solution-matrix-comment-field-value\ } & optional, content,
default: \texttt{\ {[}*Note:*\ \#v(0.5cm){]}\ } , value of solution
matrix comment fields \\
\texttt{\ distribution-header-point-value\ } & optional, content,
default: \texttt{\ {[}Point{]}\ } , first row of point distribution,
used to indicate the points needed to get a specific grade \\
\texttt{\ distribution-header-point-grade\ } & optional, content,
default: \texttt{\ {[}Grade{]}\ } , second row of point distribution \\
\texttt{\ message\ } & optional, function, default:
\texttt{\ (points-sum,\ extrapoints-sum)\ =\textgreater{}\ {[}In\ sum\ \#points-sum\ +\ \#extrapoints-sum\ P.\ are\ achievable.\ You\ achieved\ \#box(line(stroke:\ purple,\ length:\ 1cm))\ out\ of\ \#points-sum\ points.{]}\ }
, used to generate the message part above the point distribution \\
\texttt{\ grade-scale\ } & optional, array, default:
\texttt{\ (({[}excellent{]},\ 0.9),\ ({[}very\ good{]},\ 0.8),\ ({[}good{]},\ 0.7),\ ({[}pass{]},\ 0.6),\ ({[}fail{]},\ 0.49))\ }
, list of grades and percentage of points to reach that grade \\
\texttt{\ page-margins\ } & optional, margins, default:
\texttt{\ none\ } , overwrite page margins \\
\texttt{\ fontsize\ } & optional, size, default: \texttt{\ 11pt\ } ,
overwrite font size \\
\texttt{\ show-todolist\ } & optional, bool, default: \texttt{\ true\ }
, show list of usages of the \texttt{\ todo\ } function after the
outline \\
\texttt{\ body\ } & content, document content \\
\end{longtable}

\texttt{\ task\ } creates a task element in an exercise project.

\begin{longtable}[]{@{}ll@{}}
\toprule\noalign{}
\texttt{\ task\ } & \\
\midrule\noalign{}
\endhead
\bottomrule\noalign{}
\endlastfoot
\texttt{\ lines\ } & optional, number, default: \texttt{\ 0\ } , number
of lines to draw if \texttt{\ show-lines\ } in exercise’s
\texttt{\ project\ } is set to \texttt{\ true\ } \\
\texttt{\ points\ } & optional, number, default: \texttt{\ 0\ } , number
of points achievable \\
\texttt{\ extra\ } & optional, bool, default: \texttt{\ false\ } ,
determines if the task is obligatory ( \texttt{\ false\ } ) or
additional ( \texttt{\ true\ } ) \\
\texttt{\ numbering-format\ } & optional, function, default:
\texttt{\ none\ } , \\
\texttt{\ title\ } & content, title of the task \\
\texttt{\ instruction\ } & content, instruction of the task,
highlighted \\
\texttt{\ ..args\ } & 1: content, task body; 2: content, task solution,
not highlighted (see \texttt{\ solution-as-matrix\ } of exercise’s
\texttt{\ project\ } ), 3: content, task hint \\
\end{longtable}

\texttt{\ subtask\ } creates a part of a task. Its points are added to
the parent task. \emph{\textbf{Subtasks are to be use inside of the
task’s body or inside of another subtask’s body.}}

\begin{longtable}[]{@{}ll@{}}
\toprule\noalign{}
\texttt{\ subtask\ } & \\
\midrule\noalign{}
\endhead
\bottomrule\noalign{}
\endlastfoot
\texttt{\ points\ } & optional, number, default: \texttt{\ 0\ } , points
achievable, adds to a tasks point \\
\texttt{\ tight\ } & optional, bool, default: \texttt{\ false\ } , enum
style \\
\texttt{\ markers\ } & optional, array, default:
\texttt{\ ("1.",\ "a)")\ } , numbering format for each level, fallback
is \texttt{\ i.\ } \\
\texttt{\ show-points\ } & optional, bool, default: \texttt{\ true\ } ,
show points next to subtask’s body iff \texttt{\ true\ } \\
\texttt{\ counter\ } & optional, counter, default: \texttt{\ none\ } ,
change number styled by the numbering format; if \texttt{\ none\ } ,
each level has an incrementel auto counter \\
\texttt{\ content\ } & content, subtask body \\
\end{longtable}

\subsection{Seminar paper}\label{seminar-paper}

\subsubsection{Setup}\label{setup-1}

\begin{Shaded}
\begin{Highlighting}[]
\NormalTok{\#import "@preview/grape{-}suite:1.0.0": seminar{-}paper}

\NormalTok{\#show: seminar{-}paper.project.with(}
\NormalTok{    title: "Die Intensionalität von dass{-}Sätzen",}
\NormalTok{    subtitle: "Intensionale Kontexte in philosophischen Argumenten",}

\NormalTok{    university: [Universität Musterstadt],}
\NormalTok{    faculty: [Exemplarische Fakultät],}
\NormalTok{    institute: [Institut für Philosophie],}
\NormalTok{    docent: [Dr. phil. Berta Beispielprüferin],}
\NormalTok{    seminar: [Beispielseminar],}

\NormalTok{    submit{-}to: [Eingereicht bei],}
\NormalTok{    submit{-}by: [Eingereicht durch],}

\NormalTok{    semester: german{-}dates.semester(datetime.today()),}

\NormalTok{    author: "Max Muster",}
\NormalTok{    email: "max.muster@uni{-}musterstadt.uni",}
\NormalTok{    address: [}
\NormalTok{        12345 Musterstadt \textbackslash{}}
\NormalTok{        Musterstraße 67}
\NormalTok{    ]}
\NormalTok{)}
\end{Highlighting}
\end{Shaded}

\subsubsection{Documentation}\label{documentation}

\begin{longtable}[]{@{}ll@{}}
\toprule\noalign{}
\texttt{\ project\ } & \\
\midrule\noalign{}
\endhead
\bottomrule\noalign{}
\endlastfoot
\texttt{\ title\ } & optional, content, default: \texttt{\ none\ } ,
title used on the title page \\
\texttt{\ subtitle\ } & optional, content, default: \texttt{\ none\ } ,
subtitle used on title page \\
\texttt{\ submit-to\ } & optional, content, default:
\texttt{\ "Submitted\ to"\ } , title for the assignees’s section \\
\texttt{\ submit-by\ } & optional, content, default:
\texttt{\ "Submitted\ by"\ } , title for the assigned’s section \\
\texttt{\ university\ } & optional, content, default:
\texttt{\ "UNIVERSITY"\ } \\
\texttt{\ faculty\ } & optional, content, default:
\texttt{\ "FACULTY"\ } \\
\texttt{\ institute\ } & optional, content, default:
\texttt{\ "INSTITUTE"\ } \\
\texttt{\ seminar\ } & optional, content, default:
\texttt{\ "SEMINAR"\ } \\
\texttt{\ semester\ } & optional, content, default:
\texttt{\ "SEMESTER"\ } \\
\texttt{\ docent\ } & optional, content, default:
\texttt{\ "DOCENT"\ } \\
\texttt{\ author\ } & optional, content, default:
\texttt{\ "AUTHOR"\ } \\
\texttt{\ email\ } & optional, content, default: \texttt{\ "EMAIL"\ } \\
\texttt{\ address\ } & optional, content, default:
\texttt{\ "ADDRESS"\ } \\
\texttt{\ title-page-part\ } & optional, content, default:
\texttt{\ none\ } , overwrite date, assignee and assigned section \\
\texttt{\ title-page-part-submit-date\ } & optional, content, default:
\texttt{\ none\ } , overwrite date section \\
\texttt{\ title-page-part-submit-to\ } & optional, content, default:
\texttt{\ none\ } , overwrite assignee section \\
\texttt{\ title-page-part-submit-by\ } & optional, content, default:
\texttt{\ none\ } , overwrite assigned section \\
\texttt{\ date\ } & optional, datetime, default:
\texttt{\ datetime.today()\ } \\
\texttt{\ date-format\ } & optional, function, default:
\texttt{\ (date)\ =\textgreater{}\ date.display("{[}day{]}.{[}month{]}.{[}year{]}")\ } \\
\texttt{\ header\ } & optional, content, default: \texttt{\ none\ } ,
overwrite page header \\
\texttt{\ header-right\ } & optional, content, default:
\texttt{\ none\ } , overwrite right header part \\
\texttt{\ header-middle\ } & optional, content, default:
\texttt{\ none\ } , overwrite middle header part \\
\texttt{\ header-left\ } & optional, content, default: \texttt{\ none\ }
, overwrite left header part \\
\texttt{\ footer\ } & optional, content, default: \texttt{\ none\ } ,
overwrite footer part \\
\texttt{\ footer-right\ } & optional, content, default:
\texttt{\ none\ } , overwrite right footer part \\
\texttt{\ footer-middle\ } & optional, content, default:
\texttt{\ none\ } , overwrite middle footer part \\
\texttt{\ footer-left\ } & optional, content, default: \texttt{\ none\ }
, overwrite left footer part \\
\texttt{\ show-outline\ } & optional, bool, default: \texttt{\ true\ } ,
show outline \\
\texttt{\ show-declaration-of-independent-work\ } & optional, bool,
default: \texttt{\ true\ } , show German declaration of independent
work \\
\texttt{\ page-margins\ } & optional, margins, default:
\texttt{\ none\ } , overwrite page margins \\
\texttt{\ fontsize\ } & optional, size, default: \texttt{\ 11pt\ } ,
overwrite fontsize \\
\texttt{\ show-todolist\ } & optional, bool, default: \texttt{\ true\ }
, show list of usages of the \texttt{\ todo\ } function after the
outline \\
\texttt{\ body\ } & content, document content \\
\end{longtable}

\begin{longtable}[]{@{}ll@{}}
\toprule\noalign{}
\texttt{\ sidenote\ } & \\
\midrule\noalign{}
\endhead
\bottomrule\noalign{}
\endlastfoot
\texttt{\ body\ } & sidenote content, which is a block with 3cm width
and will be displayed in the right margin of the page \\
\end{longtable}

\subsection{Slides}\label{slides}

\subsubsection{Setup}\label{setup-2}

\begin{Shaded}
\begin{Highlighting}[]
\NormalTok{\#import "@preview/grape{-}suite:1.0.0": slides}
\NormalTok{\#import slides: *}

\NormalTok{\#show: slides.with(}
\NormalTok{    no: 1,}
\NormalTok{    series: [Logik{-}Tutorium],}
\NormalTok{    title: [Organisatorisches und Einführung in die Logik],}

\NormalTok{    author: "Tristan Pieper",}
\NormalTok{    email: link("mailto:tristan.pieper@uni{-}rostock.de"),}
\NormalTok{)}
\end{Highlighting}
\end{Shaded}

\subsubsection{Documentation}\label{documentation-1}

\begin{longtable}[]{@{}ll@{}}
\toprule\noalign{}
\texttt{\ slides\ } & \\
\midrule\noalign{}
\endhead
\bottomrule\noalign{}
\endlastfoot
\texttt{\ no\ } & optional, number, default: \texttt{\ 0\ } , number in
the series \\
\texttt{\ series\ } & optional, content, default: \texttt{\ none\ } ,
name of the series \\
\texttt{\ title\ } & optional, content, default: \texttt{\ none\ } ,
title of the presentation \\
\texttt{\ topics\ } & optional, array, default: \texttt{\ ()\ } , topics
of the presentation \\
\texttt{\ author\ } & optional, content, default: \texttt{\ none\ } ,
author \\
\texttt{\ email\ } & optional, content, default: \texttt{\ none\ } ,
author’s email \\
\texttt{\ head-replacement\ } & optional, content, default:
\texttt{\ none\ } , replace head on title slide with given content \\
\texttt{\ title-replacement\ } & optional, content, default:
\texttt{\ none\ } , replace title below head on title slide with given
content \\
\texttt{\ footer\ } & optional, content, default: \texttt{\ none\ } ,
replace footer on slides with given content \\
\texttt{\ page-numbering\ } & optional, function, default:
\texttt{\ (n,\ total)\ =\textgreater{}\ \{...\}\ } , function that
creates the page numbering (where \texttt{\ n\ } is the current,
\texttt{\ total\ } is the last page) \\
\texttt{\ show-semester\ } & optional, bool, default: \texttt{\ true\ }
, show name of the semester (e.g. “SoSe 24�) \\
\texttt{\ show-date\ } & optional, bool, default: \texttt{\ true\ } ,
show date in german format \\
\texttt{\ show-outline\ } & optional, bool, default: \texttt{\ true\ } ,
show outline on the second slide \\
\texttt{\ box-task-title\ } & optional, content, default:
\texttt{\ {[}Task{]}\ } , shown as the title of a slide’s task box \\
\texttt{\ box-hint-title\ } & optional, content, default:
\texttt{\ {[}Hint{]}\ } , shown as the title of a slide’s tasks
colored \\
\texttt{\ box-solution-title\ } & optional, content, default:
\texttt{\ {[}Solution{]}\ } , shown as the title of a slide’s tasks
colored \\
\texttt{\ box-definition-title\ } & optional, content, default:
\texttt{\ {[}Definition{]}\ } , shown as the title of a slide’s
definition box \\
\texttt{\ box-notice-title\ } & optional, content, default:
\texttt{\ {[}Notice{]}\ } , shown as the title of a slide’s notice
box \\
\texttt{\ box-example-title\ } & optional, content, default:
\texttt{\ {[}Example{]}\ } , shown as the title of a slide’s example
box \\
\texttt{\ date\ } & optional, datetime, default:
\texttt{\ datetime.today()\ } \\
\texttt{\ show-todolist\ } & optional, bool, default: \texttt{\ true\ }
, show list of usages of the \texttt{\ todo\ } function after the
outline \\
\texttt{\ show-title-slide\ } & optional, bool, default:
\texttt{\ true\ } , show title slide \\
\texttt{\ show-author\ } & optional, bool, default: \texttt{\ true\ } ,
show author name on title slide \\
\texttt{\ show-footer\ } & optional, bool, default: \texttt{\ true\ } ,
show footer on slides \\
\texttt{\ show-page-numbers\ } & optional, bool, default:
\texttt{\ true\ } , show page numbering \\
\texttt{\ outline-title-text\ } & optional, content, default:
\texttt{\ "Outline"\ } , title for the outline \\
\texttt{\ body\ } & content, document content \\
\end{longtable}

\begin{itemize}
\tightlist
\item
  \texttt{\ slide\ } , \texttt{\ pause\ } , \texttt{\ only\ } ,
  \texttt{\ uncover\ } : imported from polylux
\end{itemize}

\subsubsection{Todos}\label{todos}

The following functions can be imported from \texttt{\ slides\ } ,
\texttt{\ exercise\ } and \texttt{\ seminar-paper\ } :

\begin{itemize}
\tightlist
\item
  \texttt{\ todo(content,\ ...)\ } - create a highlighted inline
  todo-note
\item
  \texttt{\ list-todos()\ } - create list of all todo-usages with page
  of usage and content
\item
  \texttt{\ hide-todos()\ } - hides all usages of \texttt{\ todo()\ } in
  the document
\end{itemize}

\subsubsection{Elements}\label{elements}

The following functions can be imported from \texttt{\ slides\ } ,
\texttt{\ exercise\ } and \texttt{\ seminar-paper\ } :
\texttt{\ definition\ }

\subsection{1.0.0}\label{section}

New:

\begin{itemize}
\tightlist
\item
  \texttt{\ todo\ } , \texttt{\ list-todos\ } , \texttt{\ hide-todos\ }
  in \texttt{\ todo.typ\ } , importable from \texttt{\ slides\ } ,
  \texttt{\ exercise.project\ } and \texttt{\ seminar-paper.project\ }
\item
  \texttt{\ show-todolist\ } attribute in above templates
\item
  \texttt{\ ignore-points\ } attribute in \texttt{\ task\ } and
  \texttt{\ subtask\ } of exercises, so that their points won’t be
  shown in the solution matrix or point distribution
\item
  comment field and a standard-value for solution matrix via
  \texttt{\ show-solution-matrix-comment-field\ } and
  \texttt{\ solution-matrix-comment-field-value\ } options in
  \texttt{\ exercise.project\ }
\item
  optional parameter \texttt{\ type\ } in \texttt{\ slides.task\ }
\item
  new parameters in \texttt{\ sllides.slides\ } :

  \begin{itemize}
  \tightlist
  \item
    \texttt{\ head-replacement\ }
  \item
    \texttt{\ title-replacement\ }
  \item
    \texttt{\ footer\ }
  \item
    \texttt{\ page-numbering\ }
  \item
    \texttt{\ show-title-slide\ }
  \item
    \texttt{\ show-author\ } (on title slide)
  \item
    \texttt{\ show-date\ }
  \item
    \texttt{\ show-footer\ }
  \item
    \texttt{\ show-page-numbers\ }
  \end{itemize}
\item
  optional parameter \texttt{\ show-outline\ } in
  \texttt{\ seminar-paper.project\ }
\end{itemize}

Changes:

\begin{itemize}
\tightlist
\item
  \texttt{\ dates.typ\ } becomes \texttt{\ german-dates.typ\ }
\end{itemize}

Fixes:

\begin{itemize}
\tightlist
\item
  remove forced German from the slides template
\item
  long headings are now properly aligned
\item
  subtask counter now resets for each part of task
\end{itemize}

\textbf{Breaking Changes:}

\begin{itemize}
\tightlist
\item
  \texttt{\ dates\ } becomes \texttt{\ german-dates\ }
\item
  changed all \texttt{\ with-outline\ } to \texttt{\ show-outline\ }
\end{itemize}

\href{/app?template=grape-suite&version=1.0.0}{Create project in app}

\subsubsection{How to use}\label{how-to-use}

Click the button above to create a new project using this template in
the Typst app.

You can also use the Typst CLI to start a new project on your computer
using this command:

\begin{verbatim}
typst init @preview/grape-suite:1.0.0
\end{verbatim}

\includesvg[width=0.16667in,height=0.16667in]{/assets/icons/16-copy.svg}

\subsubsection{About}\label{about}

\begin{description}
\tightlist
\item[Author :]
\href{mailto:tristanpieper080803@gmail.com}{Tristan Pieper}
\item[License:]
MIT
\item[Current version:]
1.0.0
\item[Last updated:]
July 22, 2024
\item[First released:]
May 3, 2024
\item[Minimum Typst version:]
0.11.0
\item[Archive size:]
15.7 kB
\href{https://packages.typst.org/preview/grape-suite-1.0.0.tar.gz}{\pandocbounded{\includesvg[keepaspectratio]{/assets/icons/16-download.svg}}}
\item[Repository:]
\href{https://github.com/piepert/grape-suite}{GitHub}
\item[Categor ies :]
\begin{itemize}
\tightlist
\item[]
\item
  \pandocbounded{\includesvg[keepaspectratio]{/assets/icons/16-layout.svg}}
  \href{https://typst.app/universe/search/?category=layout}{Layout}
\item
  \pandocbounded{\includesvg[keepaspectratio]{/assets/icons/16-atom.svg}}
  \href{https://typst.app/universe/search/?category=paper}{Paper}
\item
  \pandocbounded{\includesvg[keepaspectratio]{/assets/icons/16-presentation.svg}}
  \href{https://typst.app/universe/search/?category=presentation}{Presentation}
\end{itemize}
\end{description}

\subsubsection{Where to report issues?}\label{where-to-report-issues}

This template is a project of Tristan Pieper . Report issues on
\href{https://github.com/piepert/grape-suite}{their repository} . You
can also try to ask for help with this template on the
\href{https://forum.typst.app}{Forum} .

Please report this template to the Typst team using the
\href{https://typst.app/contact}{contact form} if you believe it is a
safety hazard or infringes upon your rights.

\phantomsection\label{versions}
\subsubsection{Version history}\label{version-history}

\begin{longtable}[]{@{}ll@{}}
\toprule\noalign{}
Version & Release Date \\
\midrule\noalign{}
\endhead
\bottomrule\noalign{}
\endlastfoot
1.0.0 & July 22, 2024 \\
\href{https://typst.app/universe/package/grape-suite/0.1.0/}{0.1.0} &
May 3, 2024 \\
\end{longtable}

Typst GmbH did not create this template and cannot guarantee correct
functionality of this template or compatibility with any version of the
Typst compiler or app.


\title{typst.app/universe/package/pyrunner}

\phantomsection\label{banner}
\section{pyrunner}\label{pyrunner}

{ 0.2.0 }

Run python code in typst.

\phantomsection\label{readme}
Run python code in \href{https://typst.app/}{typst} using
\href{https://github.com/RustPython/RustPython}{RustPython} .

\begin{Shaded}
\begin{Highlighting}[]
\NormalTok{\#import "@preview/pyrunner:0.1.0" as py}

\NormalTok{\#let compiled = py.compile(}
\NormalTok{\textasciigrave{}\textasciigrave{}\textasciigrave{}python}
\NormalTok{def find\_emails(string):}
\NormalTok{    import re}
\NormalTok{    return re.findall(r"\textbackslash{}b[a{-}zA{-}Z0{-}9.\_\%+{-}]+@[a{-}zA{-}Z0{-}9.{-}]+\textbackslash{}.[a{-}zA{-}Z]\{2,\}\textbackslash{}b", string)}

\NormalTok{def sum\_all(*array):}
\NormalTok{    return sum(array)}
\NormalTok{\textasciigrave{}\textasciigrave{}\textasciigrave{})}

\NormalTok{\#let txt = "My email address is john.doe@example.com and my friend\textquotesingle{}s email address is jane.doe@example.net."}

\NormalTok{\#py.call(compiled, "find\_emails", txt)}
\NormalTok{\#py.call(compiled, "sum\_all", 1, 2, 3)}
\end{Highlighting}
\end{Shaded}

Block mode is also available.

\begin{Shaded}
\begin{Highlighting}[]
\NormalTok{\#let code = \textasciigrave{}\textasciigrave{}\textasciigrave{}}
\NormalTok{f\textquotesingle{}\{a+b=\}\textquotesingle{}}
\NormalTok{\textasciigrave{}\textasciigrave{}\textasciigrave{}}

\NormalTok{\#py.block(code, globals: (a: 1, b: 2))}

\NormalTok{\#py.block(code, globals: (a: "1", b: "2"))}
\end{Highlighting}
\end{Shaded}

The result will be \texttt{\ a+b=3\ } and
\texttt{\ a+b=\textquotesingle{}12\textquotesingle{}\ } .

\subsection{Current limitations}\label{current-limitations}

Due to restrictions of typst and its plugin system, some Python function
will not work as expected:

\begin{itemize}
\tightlist
\item
  File and network IO will always raise an exception.
\item
  \texttt{\ datatime.now\ } will always return 1970-01-01.
\end{itemize}

Also, there is no way to import third-party modules. Only bundled stdlib
modules are available. We might find a way to lift this restriction, so
feel free to submit an issue if you want this functionality.

\subsection{API}\label{api}

\subsubsection{\texorpdfstring{\texttt{\ block\ }}{ block }}\label{block}

Run Python code block and get its result.

\paragraph{Arguments}\label{arguments}

\begin{itemize}
\tightlist
\item
  \texttt{\ code\ } : string \textbar{} raw content - The Python code to
  run.
\item
  \texttt{\ globals\ } : dict (named optional) - The global variables to
  bring into scope.
\end{itemize}

\paragraph{Returns}\label{returns}

The last expression of the code block.

\subsubsection{\texorpdfstring{\texttt{\ compile\ }}{ compile }}\label{compile}

Compile Python code to bytecode.

\paragraph{Arguments}\label{arguments-1}

\begin{itemize}
\tightlist
\item
  \texttt{\ code\ } : string \textbar{} raw content - The Python code to
  compile.
\end{itemize}

\paragraph{Returns}\label{returns-1}

The bytecode representation in \texttt{\ bytes\ } .

\subsubsection{\texorpdfstring{\texttt{\ call\ }}{ call }}\label{call}

Call a python function with arguments.

\paragraph{Arguments}\label{arguments-2}

\begin{itemize}
\tightlist
\item
  \texttt{\ compiled\ } : bytes - The bytecode representation of Python
  code.
\item
  \texttt{\ fn\_name\ } : string - The name of the function to be
  called.
\item
  \texttt{\ ..args\ } : any - The arguments to pass to the function.
\end{itemize}

\paragraph{Returns}\label{returns-2}

The result of the function call.

\subsection{Build from source}\label{build-from-source}

Install
\href{https://github.com/astrale-sharp/wasm-minimal-protocol}{\texttt{\ wasi-stub\ }}
. You should use a slightly modified one. See
\href{https://github.com/astrale-sharp/wasm-minimal-protocol/issues/22\#issuecomment-1827379467}{the
related issue} .

Build pyrunner.

\begin{verbatim}
rustup target add wasm32-wasi
cargo build --target wasm32-wasi
make pkg/typst-pyrunner.wasm
\end{verbatim}

Add to local package.

\begin{verbatim}
mkdir -p ~/.local/share/typst/packages/local/pyrunner/0.0.1
cp pkg/* ~/.local/share/typst/packages/local/pyrunner/0.0.1
\end{verbatim}

\subsubsection{How to add}\label{how-to-add}

Copy this into your project and use the import as \texttt{\ pyrunner\ }

\begin{verbatim}
#import "@preview/pyrunner:0.2.0"
\end{verbatim}

\includesvg[width=0.16667in,height=0.16667in]{/assets/icons/16-copy.svg}

Check the docs for
\href{https://typst.app/docs/reference/scripting/\#packages}{more
information on how to import packages} .

\subsubsection{About}\label{about}

\begin{description}
\tightlist
\item[Author :]
Peng Guanwen
\item[License:]
MIT
\item[Current version:]
0.2.0
\item[Last updated:]
March 18, 2024
\item[First released:]
December 4, 2023
\item[Minimum Typst version:]
0.10.0
\item[Archive size:]
5.89 MB
\href{https://packages.typst.org/preview/pyrunner-0.2.0.tar.gz}{\pandocbounded{\includesvg[keepaspectratio]{/assets/icons/16-download.svg}}}
\item[Repository:]
\href{https://github.com/peng1999/typst-pyrunner}{GitHub}
\item[Categor ies :]
\begin{itemize}
\tightlist
\item[]
\item
  \pandocbounded{\includesvg[keepaspectratio]{/assets/icons/16-code.svg}}
  \href{https://typst.app/universe/search/?category=scripting}{Scripting}
\item
  \pandocbounded{\includesvg[keepaspectratio]{/assets/icons/16-integration.svg}}
  \href{https://typst.app/universe/search/?category=integration}{Integration}
\end{itemize}
\end{description}

\subsubsection{Where to report issues?}\label{where-to-report-issues}

This package is a project of Peng Guanwen . Report issues on
\href{https://github.com/peng1999/typst-pyrunner}{their repository} .
You can also try to ask for help with this package on the
\href{https://forum.typst.app}{Forum} .

Please report this package to the Typst team using the
\href{https://typst.app/contact}{contact form} if you believe it is a
safety hazard or infringes upon your rights.

\phantomsection\label{versions}
\subsubsection{Version history}\label{version-history}

\begin{longtable}[]{@{}ll@{}}
\toprule\noalign{}
Version & Release Date \\
\midrule\noalign{}
\endhead
\bottomrule\noalign{}
\endlastfoot
0.2.0 & March 18, 2024 \\
\href{https://typst.app/universe/package/pyrunner/0.1.0/}{0.1.0} &
December 4, 2023 \\
\end{longtable}

Typst GmbH did not create this package and cannot guarantee correct
functionality of this package or compatibility with any version of the
Typst compiler or app.


\title{typst.app/universe/package/frame-it}

\phantomsection\label{banner}
\section{frame-it}\label{frame-it}

{ 1.0.0 }

Beautiful, flexible, and integrated. Display custom frames for theorems,
environments, and more. Attractive visuals with syntax that blends
seamlessly into the source.

\phantomsection\label{readme}
\pandocbounded{\includesvg[keepaspectratio]{https://raw.githubusercontent.com/marc-thieme/frame-it/refs/heads/assets/README.svg}}

\subsubsection{How to add}\label{how-to-add}

Copy this into your project and use the import as \texttt{\ frame-it\ }

\begin{verbatim}
#import "@preview/frame-it:1.0.0"
\end{verbatim}

\includesvg[width=0.16667in,height=0.16667in]{/assets/icons/16-copy.svg}

Check the docs for
\href{https://typst.app/docs/reference/scripting/\#packages}{more
information on how to import packages} .

\subsubsection{About}\label{about}

\begin{description}
\tightlist
\item[Author :]
Marc Thieme
\item[License:]
MIT
\item[Current version:]
1.0.0
\item[Last updated:]
November 18, 2024
\item[First released:]
November 18, 2024
\item[Archive size:]
8.36 kB
\href{https://packages.typst.org/preview/frame-it-1.0.0.tar.gz}{\pandocbounded{\includesvg[keepaspectratio]{/assets/icons/16-download.svg}}}
\item[Repository:]
\href{https://github.com/marc-thieme/frame-it}{GitHub}
\item[Categor ies :]
\begin{itemize}
\tightlist
\item[]
\item
  \pandocbounded{\includesvg[keepaspectratio]{/assets/icons/16-package.svg}}
  \href{https://typst.app/universe/search/?category=components}{Components}
\item
  \pandocbounded{\includesvg[keepaspectratio]{/assets/icons/16-layout.svg}}
  \href{https://typst.app/universe/search/?category=layout}{Layout}
\item
  \pandocbounded{\includesvg[keepaspectratio]{/assets/icons/16-speak.svg}}
  \href{https://typst.app/universe/search/?category=report}{Report}
\end{itemize}
\end{description}

\subsubsection{Where to report issues?}\label{where-to-report-issues}

This package is a project of Marc Thieme . Report issues on
\href{https://github.com/marc-thieme/frame-it}{their repository} . You
can also try to ask for help with this package on the
\href{https://forum.typst.app}{Forum} .

Please report this package to the Typst team using the
\href{https://typst.app/contact}{contact form} if you believe it is a
safety hazard or infringes upon your rights.

\phantomsection\label{versions}
\subsubsection{Version history}\label{version-history}

\begin{longtable}[]{@{}ll@{}}
\toprule\noalign{}
Version & Release Date \\
\midrule\noalign{}
\endhead
\bottomrule\noalign{}
\endlastfoot
1.0.0 & November 18, 2024 \\
\end{longtable}

Typst GmbH did not create this package and cannot guarantee correct
functionality of this package or compatibility with any version of the
Typst compiler or app.


\title{typst.app/universe/package/moderner-cv}

\phantomsection\label{banner}
\phantomsection\label{template-thumbnail}
\pandocbounded{\includegraphics[keepaspectratio]{https://packages.typst.org/preview/thumbnails/moderner-cv-0.1.0-small.webp}}

\section{moderner-cv}\label{moderner-cv}

{ 0.1.0 }

A resume template based on the moderncv LaTeX package.

\href{/app?template=moderner-cv&version=0.1.0}{Create project in app}

\phantomsection\label{readme}
This is a typst adaptation of LaTeX’s
\href{https://github.com/moderncv/moderncv}{moderncv} , a modern
curriculum vitae class.

\subsection{Requirements}\label{requirements}

This template uses FontAwesome icons via the
\href{https://typst.app/universe/package/fontawesome}{fontawesome typst
package} . In order to properly use it, you need to have fontawesome
installed on your system or have typst configured (via
\texttt{\ -\/-font-path\ } ) to use the fontawesome font files. You can
download fontawesome \href{https://fontawesome.com/download}{here} .

\subsection{Usage}\label{usage}

\begin{Shaded}
\begin{Highlighting}[]
\NormalTok{\#import "@preview/moderner{-}cv:0.1.0": *}

\NormalTok{\#show: moderner{-}cv.with(}
\NormalTok{  name: "Jane Doe",}
\NormalTok{  lang: "en",}
\NormalTok{  social: (}
\NormalTok{    email: "jane.doe@example.com",}
\NormalTok{    github: "jane{-}doe",}
\NormalTok{    linkedin: "jane{-}doe",}
\NormalTok{  ),}
\NormalTok{)}

\NormalTok{// ...}
\end{Highlighting}
\end{Shaded}

\subsection{Examples}\label{examples}

\pandocbounded{\includegraphics[keepaspectratio]{https://github.com/typst/packages/raw/main/packages/preview/moderner-cv/0.1.0/assets/example.png}}

\subsection{Building and Testing
Locally}\label{building-and-testing-locally}

To build and test the template locally, you can run
\texttt{\ pixi\ run\ watch\ } in the root of this repository. Please
ensure to have linked this package to your local typst packages, see
\href{https://github.com/typst/packages\#local-packages}{here} :

\begin{Shaded}
\begin{Highlighting}[]
\CommentTok{\# linux}
\FunctionTok{mkdir} \AttributeTok{{-}p}\NormalTok{ \textasciitilde{}/.local/share/typst/packages/preview/moderner{-}cv}
\FunctionTok{ln} \AttributeTok{{-}s} \VariableTok{$(}\BuiltInTok{pwd}\VariableTok{)}\NormalTok{ \textasciitilde{}/.local/share/typst/packages/preview/moderner{-}cv/0.1.0}

\CommentTok{\# macos}
\FunctionTok{mkdir} \AttributeTok{{-}p}\NormalTok{ \textasciitilde{}/Library/Application\textbackslash{} Support/typst/packages/preview/moderner{-}cv}
\FunctionTok{ln} \AttributeTok{{-}s} \VariableTok{$(}\BuiltInTok{pwd}\VariableTok{)}\NormalTok{ \textasciitilde{}/Library/Application\textbackslash{} Support/typst/packages/preview/moderner{-}cv/0.1.0}
\end{Highlighting}
\end{Shaded}

\href{/app?template=moderner-cv&version=0.1.0}{Create project in app}

\subsubsection{How to use}\label{how-to-use}

Click the button above to create a new project using this template in
the Typst app.

You can also use the Typst CLI to start a new project on your computer
using this command:

\begin{verbatim}
typst init @preview/moderner-cv:0.1.0
\end{verbatim}

\includesvg[width=0.16667in,height=0.16667in]{/assets/icons/16-copy.svg}

\subsubsection{About}\label{about}

\begin{description}
\tightlist
\item[Author :]
\href{https://github.com/pavelzw}{Pavel Zwerschke}
\item[License:]
MIT
\item[Current version:]
0.1.0
\item[Last updated:]
July 3, 2024
\item[First released:]
July 3, 2024
\item[Minimum Typst version:]
0.11.1
\item[Archive size:]
3.21 kB
\href{https://packages.typst.org/preview/moderner-cv-0.1.0.tar.gz}{\pandocbounded{\includesvg[keepaspectratio]{/assets/icons/16-download.svg}}}
\item[Repository:]
\href{https://github.com/pavelzw/moderner-cv}{GitHub}
\item[Categor y :]
\begin{itemize}
\tightlist
\item[]
\item
  \pandocbounded{\includesvg[keepaspectratio]{/assets/icons/16-user.svg}}
  \href{https://typst.app/universe/search/?category=cv}{CV}
\end{itemize}
\end{description}

\subsubsection{Where to report issues?}\label{where-to-report-issues}

This template is a project of Pavel Zwerschke . Report issues on
\href{https://github.com/pavelzw/moderner-cv}{their repository} . You
can also try to ask for help with this template on the
\href{https://forum.typst.app}{Forum} .

Please report this template to the Typst team using the
\href{https://typst.app/contact}{contact form} if you believe it is a
safety hazard or infringes upon your rights.

\phantomsection\label{versions}
\subsubsection{Version history}\label{version-history}

\begin{longtable}[]{@{}ll@{}}
\toprule\noalign{}
Version & Release Date \\
\midrule\noalign{}
\endhead
\bottomrule\noalign{}
\endlastfoot
0.1.0 & July 3, 2024 \\
\end{longtable}

Typst GmbH did not create this template and cannot guarantee correct
functionality of this template or compatibility with any version of the
Typst compiler or app.


\title{typst.app/universe/package/finite}

\phantomsection\label{banner}
\section{finite}\label{finite}

{ 0.3.2 }

Typst-setting finite automata with CeTZ

{ } Featured Package

\phantomsection\label{readme}
\textbf{finite} is a \href{https://github.com/typst/typst}{Typst}
package for rendering finite automata on top of
\href{https://github.com/johannes-wolf/typst-canvas}{CeTZ} .

\subsection{Usage}\label{usage}

For Typst 0.6.0 or later, import the package from the typst preview
repository:

\begin{Shaded}
\begin{Highlighting}[]
\NormalTok{\#import }\StringTok{"@preview/finite:0.3.2"}\OperatorTok{:}\NormalTok{ automaton}
\end{Highlighting}
\end{Shaded}

After importing the package, simply call \texttt{\ \#automaton()\ } with
a dictionary holding a transition table:

\begin{Shaded}
\begin{Highlighting}[]
\NormalTok{\#import }\StringTok{"@preview/finite:0.3.2"}\OperatorTok{:}\NormalTok{ automaton}

\NormalTok{\#}\FunctionTok{automaton}\NormalTok{((}
\NormalTok{  q0}\OperatorTok{:}\NormalTok{ (q1}\OperatorTok{:}\DecValTok{0}\OperatorTok{,}\NormalTok{ q0}\OperatorTok{:}\StringTok{"0,1"}\NormalTok{)}\OperatorTok{,}
\NormalTok{  q1}\OperatorTok{:}\NormalTok{ (q0}\OperatorTok{:}\NormalTok{(}\DecValTok{0}\OperatorTok{,}\DecValTok{1}\NormalTok{)}\OperatorTok{,}\NormalTok{ q2}\OperatorTok{:}\StringTok{"0"}\NormalTok{)}\OperatorTok{,}
\NormalTok{  q2}\OperatorTok{:}\NormalTok{ ()}\OperatorTok{,}
\NormalTok{))}
\end{Highlighting}
\end{Shaded}

The output should look like this:
\pandocbounded{\includegraphics[keepaspectratio]{https://github.com/typst/packages/raw/main/packages/preview/finite/0.3.2/assets/example.png}}

\subsection{Further documentation}\label{further-documentation}

See \texttt{\ manual.pdf\ } for a full manual of the package.

\subsection{Development}\label{development}

The documentation is created using
\href{https://github.com/jneug/typst-mantys}{Mantys} , a Typst template
for creating package documentation.

To compile the manual, Mantys needs to be available as a local package.
Refer to Mantys’ manual for instructions on how to do so.

\subsection{Changelog}\label{changelog}

\subsubsection{Version 0.3.2}\label{version-0.3.2}

\begin{itemize}
\tightlist
\item
  Fixed an issue with final states not beeing recognized properly (\#5)
\end{itemize}

\subsubsection{Version 0.3.1}\label{version-0.3.1}

\begin{itemize}
\tightlist
\item
  Added styling options for intial states:

  \begin{itemize}
  \tightlist
  \item
    \texttt{\ stroke\ } sets a stroke for the marking.
  \item
    \texttt{\ scale\ } scales the marking by a factor.
  \end{itemize}
\item
  Updated manual.
\end{itemize}

\subsubsection{Version 0.3.0}\label{version-0.3.0}

\begin{itemize}
\tightlist
\item
  Bumped tools4typst to v0.3.2.
\item
  Introducing automaton specs as a data structure.
\item
  Changes to \texttt{\ automaton\ } command:

  \begin{itemize}
  \tightlist
  \item
    Changed \texttt{\ label-format\ } argument to
    \texttt{\ state-format\ } and \texttt{\ input-format\ } .
  \item
    \texttt{\ layout\ } can now take a dictionary with (
    \texttt{\ state\ } : \texttt{\ coordinate\ } ) pairs to position
    states.
  \end{itemize}
\item
  Added \texttt{\ \#powerset\ } command, to transform a NFA into a DFA.
\item
  Added \texttt{\ \#add-trap\ } command, to complete a partial DFA.
\item
  Added \texttt{\ \#accepts\ } command, to test a word against an NFA or
  DFA.
\item
  Added \texttt{\ transpose-table\ } and \texttt{\ get-inputs\ }
  utilities.
\item
  Added “Start� label to the mark for initial states.

  \begin{itemize}
  \tightlist
  \item
    Added option to modify the mark label for initial states.
  \end{itemize}
\item
  Added anchor option for loops, to position the loop at one of the
  eight default anchors.
\item
  Changed \texttt{\ curve\ } option to be the height of the arc of the
  transition.

  \begin{itemize}
  \tightlist
  \item
    This makes styling more consistent over longer distances.
  \end{itemize}
\item
  Added \texttt{\ rest\ } key to custom layouts.
\end{itemize}

\subsubsection{Version 0.2.0}\label{version-0.2.0}

\begin{itemize}
\tightlist
\item
  Bumped CeTZ to v0.1.1.
\end{itemize}

\subsubsection{Version 0.1.0}\label{version-0.1.0}

\begin{itemize}
\tightlist
\item
  Initial release submitted to
  \href{https://github.com/typst/packages}{typst/packages} .
\end{itemize}

\subsubsection{How to add}\label{how-to-add}

Copy this into your project and use the import as \texttt{\ finite\ }

\begin{verbatim}
#import "@preview/finite:0.3.2"
\end{verbatim}

\includesvg[width=0.16667in,height=0.16667in]{/assets/icons/16-copy.svg}

Check the docs for
\href{https://typst.app/docs/reference/scripting/\#packages}{more
information on how to import packages} .

\subsubsection{About}\label{about}

\begin{description}
\tightlist
\item[Author :]
Jonas Neugebauer
\item[License:]
MIT
\item[Current version:]
0.3.2
\item[Last updated:]
September 30, 2024
\item[First released:]
September 3, 2023
\item[Archive size:]
13.6 kB
\href{https://packages.typst.org/preview/finite-0.3.2.tar.gz}{\pandocbounded{\includesvg[keepaspectratio]{/assets/icons/16-download.svg}}}
\item[Repository:]
\href{https://github.com/jneug/typst-finite}{GitHub}
\end{description}

\subsubsection{Where to report issues?}\label{where-to-report-issues}

This package is a project of Jonas Neugebauer . Report issues on
\href{https://github.com/jneug/typst-finite}{their repository} . You can
also try to ask for help with this package on the
\href{https://forum.typst.app}{Forum} .

Please report this package to the Typst team using the
\href{https://typst.app/contact}{contact form} if you believe it is a
safety hazard or infringes upon your rights.

\phantomsection\label{versions}
\subsubsection{Version history}\label{version-history}

\begin{longtable}[]{@{}ll@{}}
\toprule\noalign{}
Version & Release Date \\
\midrule\noalign{}
\endhead
\bottomrule\noalign{}
\endlastfoot
0.3.2 & September 30, 2024 \\
\href{https://typst.app/universe/package/finite/0.3.0/}{0.3.0} &
September 23, 2023 \\
\href{https://typst.app/universe/package/finite/0.1.0/}{0.1.0} &
September 3, 2023 \\
\end{longtable}

Typst GmbH did not create this package and cannot guarantee correct
functionality of this package or compatibility with any version of the
Typst compiler or app.


\title{typst.app/universe/package/inboisu}

\phantomsection\label{banner}
\phantomsection\label{template-thumbnail}
\pandocbounded{\includegraphics[keepaspectratio]{https://packages.typst.org/preview/thumbnails/inboisu-0.1.0-small.webp}}

\section{inboisu}\label{inboisu}

{ 0.1.0 }

Inboisu is a tool for creating Japanese invoices.

\href{/app?template=inboisu&version=0.1.0}{Create project in app}

\phantomsection\label{readme}
æ---¥æœ¬èªžã?®è«‹æ±‚書ã‚'作æˆ?ã?™ã‚‹ã?Ÿã‚?ã?® Typst
テンãƒ---レートã?§ã?™ã€‚

Inboisu is a Typst template for creating Japanese invoices.

\subsection{使ã?„æ--¹ / Usage}\label{uxe4uxbduxe3uxe6uxb9-usage}

\begin{Shaded}
\begin{Highlighting}[]
\NormalTok{\#include "@preview/inboisu:0.1.0": doc}

\NormalTok{\#show: doc(}
\NormalTok{    ... // ドキュメントを参照}
\NormalTok{)}
\end{Highlighting}
\end{Shaded}

\subsection{ドキュメント /
Documentation}\label{uxe3ux192uxe3uxe3ux192uxe3ux192uxe3ux192uxb3uxe3ux192ux2c6-documentation}

\begin{itemize}
\tightlist
\item
  \href{https://github.com/typst/packages/raw/main/packages/preview/inboisu/0.1.0/docs/documentation.pdf}{docs/documentation.pdf}
  (
  \href{https://github.com/typst/packages/raw/main/packages/preview/inboisu/0.1.0/docs/docs.typ}{Source}
  )
\end{itemize}

\subsection{テンãƒ---レート /
Templates}\label{uxe3ux192uxe3ux192uxb3uxe3ux192uxe3ux192uxe3ux192uxbcuxe3ux192ux2c6-templates}

\subsubsection{請求書 / Invoice}\label{uxe8uxe6uxe6-invoice}

\pandocbounded{\includegraphics[keepaspectratio]{https://github.com/typst/packages/raw/main/packages/preview/inboisu/0.1.0/images/invoice.png}}

\begin{itemize}
\tightlist
\item
  \href{https://github.com/typst/packages/raw/main/packages/preview/inboisu/0.1.0/template/invoice.typ}{invoice.typ}
\end{itemize}

\subsubsection{é~˜å?Žæ›¸ / Receipt}\label{uxe9-uxe5ux17euxe6-receipt}

\pandocbounded{\includegraphics[keepaspectratio]{https://github.com/typst/packages/raw/main/packages/preview/inboisu/0.1.0/images/receipt.png}}

\begin{itemize}
\tightlist
\item
  \href{https://github.com/typst/packages/raw/main/packages/preview/inboisu/0.1.0/template/receipt.typ}{receipt.typ}
\end{itemize}

\href{/app?template=inboisu&version=0.1.0}{Create project in app}

\subsubsection{How to use}\label{how-to-use}

Click the button above to create a new project using this template in
the Typst app.

You can also use the Typst CLI to start a new project on your computer
using this command:

\begin{verbatim}
typst init @preview/inboisu:0.1.0
\end{verbatim}

\includesvg[width=0.16667in,height=0.16667in]{/assets/icons/16-copy.svg}

\subsubsection{About}\label{about}

\begin{description}
\tightlist
\item[Author :]
\href{mailto:mkpoli@mkpo.li}{mkpoli}
\item[License:]
MIT-0
\item[Current version:]
0.1.0
\item[Last updated:]
November 21, 2024
\item[First released:]
November 21, 2024
\item[Archive size:]
5.01 kB
\href{https://packages.typst.org/preview/inboisu-0.1.0.tar.gz}{\pandocbounded{\includesvg[keepaspectratio]{/assets/icons/16-download.svg}}}
\item[Repository:]
\href{https://github.com/mkpoli/typst-inboisu}{GitHub}
\item[Discipline :]
\begin{itemize}
\tightlist
\item[]
\item
  \href{https://typst.app/universe/search/?discipline=business}{Business}
\end{itemize}
\item[Categor ies :]
\begin{itemize}
\tightlist
\item[]
\item
  \pandocbounded{\includesvg[keepaspectratio]{/assets/icons/16-layout.svg}}
  \href{https://typst.app/universe/search/?category=layout}{Layout}
\item
  \pandocbounded{\includesvg[keepaspectratio]{/assets/icons/16-envelope.svg}}
  \href{https://typst.app/universe/search/?category=office}{Office}
\end{itemize}
\end{description}

\subsubsection{Where to report issues?}\label{where-to-report-issues}

This template is a project of mkpoli . Report issues on
\href{https://github.com/mkpoli/typst-inboisu}{their repository} . You
can also try to ask for help with this template on the
\href{https://forum.typst.app}{Forum} .

Please report this template to the Typst team using the
\href{https://typst.app/contact}{contact form} if you believe it is a
safety hazard or infringes upon your rights.

\phantomsection\label{versions}
\subsubsection{Version history}\label{version-history}

\begin{longtable}[]{@{}ll@{}}
\toprule\noalign{}
Version & Release Date \\
\midrule\noalign{}
\endhead
\bottomrule\noalign{}
\endlastfoot
0.1.0 & November 21, 2024 \\
\end{longtable}

Typst GmbH did not create this template and cannot guarantee correct
functionality of this template or compatibility with any version of the
Typst compiler or app.


\title{typst.app/universe/package/fruitify}

\phantomsection\label{banner}
\section{fruitify}\label{fruitify}

{ 0.1.1 }

Replace letters in equations with fruit emoji

\phantomsection\label{readme}
Make your equations more fruity!

This package automatically replaces any single letters in equations with
fruit emoji.

Refer to
\href{https://codeberg.org/T0mstone/typst-fruitify/src/tag/0.1.1/example-documentation.pdf}{\texttt{\ example-documentation.pdf\ }}
for more detail.

\subsection{Emoji support}\label{emoji-support}

Until 0.12, typst did not have good emoji support for PDF. This meant
that even though this package worked as intended, the output would look
very wrong when exporting to PDF. Therefore, it is recommended to stick
with PNG export for those older typst versions.

\subsubsection{How to add}\label{how-to-add}

Copy this into your project and use the import as \texttt{\ fruitify\ }

\begin{verbatim}
#import "@preview/fruitify:0.1.1"
\end{verbatim}

\includesvg[width=0.16667in,height=0.16667in]{/assets/icons/16-copy.svg}

Check the docs for
\href{https://typst.app/docs/reference/scripting/\#packages}{more
information on how to import packages} .

\subsubsection{About}\label{about}

\begin{description}
\tightlist
\item[Author :]
\href{mailto:realt0mstone@gmail.com}{T0mstone}
\item[License:]
MIT-0
\item[Current version:]
0.1.1
\item[Last updated:]
October 16, 2024
\item[First released:]
October 11, 2023
\item[Minimum Typst version:]
0.9.0
\item[Archive size:]
8.42 kB
\href{https://packages.typst.org/preview/fruitify-0.1.1.tar.gz}{\pandocbounded{\includesvg[keepaspectratio]{/assets/icons/16-download.svg}}}
\item[Repository:]
\href{https://codeberg.org/T0mstone/typst-fruitify}{Codeberg}
\end{description}

\subsubsection{Where to report issues?}\label{where-to-report-issues}

This package is a project of T0mstone . Report issues on
\href{https://codeberg.org/T0mstone/typst-fruitify}{their repository} .
You can also try to ask for help with this package on the
\href{https://forum.typst.app}{Forum} .

Please report this package to the Typst team using the
\href{https://typst.app/contact}{contact form} if you believe it is a
safety hazard or infringes upon your rights.

\phantomsection\label{versions}
\subsubsection{Version history}\label{version-history}

\begin{longtable}[]{@{}ll@{}}
\toprule\noalign{}
Version & Release Date \\
\midrule\noalign{}
\endhead
\bottomrule\noalign{}
\endlastfoot
0.1.1 & October 16, 2024 \\
\href{https://typst.app/universe/package/fruitify/0.1.0/}{0.1.0} &
October 11, 2023 \\
\end{longtable}

Typst GmbH did not create this package and cannot guarantee correct
functionality of this package or compatibility with any version of the
Typst compiler or app.


\title{typst.app/universe/package/finely-crafted-cv}

\phantomsection\label{banner}
\phantomsection\label{template-thumbnail}
\pandocbounded{\includegraphics[keepaspectratio]{https://packages.typst.org/preview/thumbnails/finely-crafted-cv-0.1.0-small.webp}}

\section{finely-crafted-cv}\label{finely-crafted-cv}

{ 0.1.0 }

A modern résumé/curriculum vitæ template with high attention to
detail.

\href{/app?template=finely-crafted-cv&version=0.1.0}{Create project in
app}

\phantomsection\label{readme}
This Typst template provides a clean and professional format for
creating a curriculum vitae (CV) or résumé. It comes with functions
and styles to help you easily generate a well-structured document,
complete with sections for education, experience, skills, and more.

\subsection{Features}\label{features}

\begin{itemize}
\tightlist
\item
  \textbf{Modern Design:} Aesthetic and professional layout designed for
  readability.
\item
  \textbf{Responsive Header \& Footer:} Includes contact information
  dynamically.
\end{itemize}

\subsection{Usage}\label{usage}

To use this template, import it with the version number and utilize the
\texttt{\ resume\ } or \texttt{\ cv\ } function:

\begin{Shaded}
\begin{Highlighting}[]
\NormalTok{\#import "@preview/finely{-}crafted{-}cv:0.1.0": *}

\NormalTok{\#show: resume.with(}
\NormalTok{  name: "Amira Patel",}
\NormalTok{  tagline: "Innovative marine biologist with 15+ years of experience in ocean conservation and research.",}
\NormalTok{  keywords: "marine biology, conservation, research, education, patents",}
\NormalTok{  email: "amira.patel@oceandreams.org",}
\NormalTok{  phone: "+1{-}305{-}555{-}7890",}
\NormalTok{  linkedin{-}username: "amirapatel",}
\NormalTok{  thumbnail: image("assets/my{-}qr{-}code.svg"),}
\NormalTok{)}

\NormalTok{= Introduction}

\NormalTok{\#lorem(100)}

\NormalTok{= Experience}

\NormalTok{\#company{-}heading("Some Company", start: "March 2018", end: "Present", icon: image("icons/earth.svg"))[}
\NormalTok{  \#job{-}heading("Some Job", location: "Some Location")[}
\NormalTok{    {-} Here is an achievement}
\NormalTok{    {-} Here\textquotesingle{}s another one.}
\NormalTok{  ]}
\NormalTok{  // companies can have multiple jobs}
\NormalTok{  \#job{-}heading("First Job", location: "Some Location")[}
\NormalTok{    {-} Here is an achievement}
\NormalTok{    {-} Here\textquotesingle{}s another one.}
\NormalTok{  ]}
\NormalTok{]}

\NormalTok{// for companies which have less detail, you can use the \textasciigrave{}comment\textasciigrave{} instead of a}
\NormalTok{// body of tasks, as follows:}
\NormalTok{\#company{-}heading("Another Company", start: "July 2005", end: "August 2009", icon: image("icons/microscope.svg"))[}
\NormalTok{  \#job{-}heading("Another Job", location: "Another Location",}
\NormalTok{    comment: [Contributed to 7 published studies. \#footnote[Visit https://amirapatel.org/publications for full list of publications.]]}
\NormalTok{  )[]}
\NormalTok{]}

\NormalTok{= Education}

\NormalTok{// school{-}heading is an alias for company{-}heading, accepts the same parameters as company{-}heading}
\NormalTok{\#school{-}heading("University of California, San Diego", start: "Fall 2001", end: "Spring 2005", icon: image("icons/graduation{-}cap.svg"))[}
\NormalTok{  // degree{-}heading is an alias for job{-}heading, accepts the same parameters as job{-}heading}
\NormalTok{  \#degree{-}heading("Ph.D. in Marine Biology")[]}
\NormalTok{]}
\end{Highlighting}
\end{Shaded}

\subsection{Functions and Parameters}\label{functions-and-parameters}

\subsubsection{\texorpdfstring{\texttt{\ resume\ } or
\texttt{\ cv\ }}{ resume  or  cv }}\label{resume-or-cv}

This is the main function to create a CV document.

\begin{itemize}
\tightlist
\item
  \textbf{Parameters:}

  \begin{itemize}
  \tightlist
  \item
    \texttt{\ name\ } : (String) Your full name. Default is “YOUR NAME
    HERE�.
  \item
    \texttt{\ tagline\ } : (String) A brief description of your
    professional identity or mission.
  \item
    \texttt{\ paper\ } : (String) The paper size, default is
    “us-letter�.
  \item
    \texttt{\ heading-font\ } : (Font) Font for headings, customizable.
  \item
    \texttt{\ body-font\ } : (Font) Font for body text, customizable.
  \item
    \texttt{\ body-size\ } : (Size) Font size for body text.
  \item
    \texttt{\ email\ } : (String) Your email address.
  \item
    \texttt{\ phone\ } : (String) Your phone number.
  \item
    \texttt{\ linkedin-username\ } : (String) Your LinkedIn username.
  \item
    \texttt{\ keywords\ } : (String) Keywords for searchability.
  \item
    \texttt{\ thumbnail\ } : (Image) Thumbnail or QR code image,
    optional.
  \item
    \texttt{\ body\ } : (Block) The main content of your CV.
  \end{itemize}
\end{itemize}

\subsubsection{\texorpdfstring{\texttt{\ company-heading\ }}{ company-heading }}\label{company-heading}

Used to create a heading for a company or organization.

\begin{itemize}
\tightlist
\item
  \textbf{Parameters:}

  \begin{itemize}
  \tightlist
  \item
    \texttt{\ name\ } : (String) Name of the company.
  \item
    \texttt{\ start\ } : (String) Start date.
  \item
    \texttt{\ end\ } : (String) End date, optional.
  \item
    \texttt{\ icon\ } : (Image) Icon image associated with the company,
    optional.
  \item
    \texttt{\ body\ } : (Block) Content related to the company role or
    tasks.
  \end{itemize}
\end{itemize}

\subsubsection{\texorpdfstring{\texttt{\ job-heading\ }}{ job-heading }}\label{job-heading}

Defines a job title within a company heading.

\begin{itemize}
\tightlist
\item
  \textbf{Parameters:}

  \begin{itemize}
  \tightlist
  \item
    \texttt{\ title\ } : (String) Job title.
  \item
    \texttt{\ location\ } : (String) Location of the job, optional.
  \item
    \texttt{\ start\ } : (String) Start date, optional.
  \item
    \texttt{\ end\ } : (String) End date, optional.
  \item
    \texttt{\ comment\ } : (String) Additional comments or notes,
    optional.
  \item
    \texttt{\ body\ } : (Block) Tasks or responsibilities.
  \end{itemize}
\end{itemize}

\subsubsection{\texorpdfstring{\texttt{\ school-heading\ }}{ school-heading }}\label{school-heading}

Alias for \texttt{\ company-heading\ } , used for educational
institutions.

\subsubsection{\texorpdfstring{\texttt{\ degree-heading\ }}{ degree-heading }}\label{degree-heading}

Alias for \texttt{\ job-heading\ } , used for academic degrees or
certifications.

\subsection{License}\label{license}

This template is released under the MIT License.

\href{/app?template=finely-crafted-cv&version=0.1.0}{Create project in
app}

\subsubsection{How to use}\label{how-to-use}

Click the button above to create a new project using this template in
the Typst app.

You can also use the Typst CLI to start a new project on your computer
using this command:

\begin{verbatim}
typst init @preview/finely-crafted-cv:0.1.0
\end{verbatim}

\includesvg[width=0.16667in,height=0.16667in]{/assets/icons/16-copy.svg}

\subsubsection{About}\label{about}

\begin{description}
\tightlist
\item[Author :]
\href{mailto:steve@waits.net}{Stephen Waits}
\item[License:]
MIT
\item[Current version:]
0.1.0
\item[Last updated:]
October 22, 2024
\item[First released:]
October 22, 2024
\item[Archive size:]
28.5 kB
\href{https://packages.typst.org/preview/finely-crafted-cv-0.1.0.tar.gz}{\pandocbounded{\includesvg[keepaspectratio]{/assets/icons/16-download.svg}}}
\item[Repository:]
\href{https://github.com/swaits/typst-collection}{GitHub}
\item[Discipline s :]
\begin{itemize}
\tightlist
\item[]
\item
  \href{https://typst.app/universe/search/?discipline=business}{Business}
\item
  \href{https://typst.app/universe/search/?discipline=communication}{Communication}
\end{itemize}
\item[Categor y :]
\begin{itemize}
\tightlist
\item[]
\item
  \pandocbounded{\includesvg[keepaspectratio]{/assets/icons/16-user.svg}}
  \href{https://typst.app/universe/search/?category=cv}{CV}
\end{itemize}
\end{description}

\subsubsection{Where to report issues?}\label{where-to-report-issues}

This template is a project of Stephen Waits . Report issues on
\href{https://github.com/swaits/typst-collection}{their repository} .
You can also try to ask for help with this template on the
\href{https://forum.typst.app}{Forum} .

Please report this template to the Typst team using the
\href{https://typst.app/contact}{contact form} if you believe it is a
safety hazard or infringes upon your rights.

\phantomsection\label{versions}
\subsubsection{Version history}\label{version-history}

\begin{longtable}[]{@{}ll@{}}
\toprule\noalign{}
Version & Release Date \\
\midrule\noalign{}
\endhead
\bottomrule\noalign{}
\endlastfoot
0.1.0 & October 22, 2024 \\
\end{longtable}

Typst GmbH did not create this template and cannot guarantee correct
functionality of this template or compatibility with any version of the
Typst compiler or app.


\title{typst.app/universe/package/ouset}

\phantomsection\label{banner}
\section{ouset}\label{ouset}

{ 0.2.0 }

Package providing over- and underset functions for math mode.

\phantomsection\label{readme}
\href{https://github.com/ludwig-austermann/typst-ouset}{GitHub
Repository including Examples and Changelog}

This is a small package providing over- and underset functions for math
mode in \href{https://typst.app/}{typst} .

\subsection{Usage}\label{usage}

To use this package simply \texttt{\ \#import\ "@preview/ouset:0.2.0"\ }
. To import all functions use \texttt{\ :\ *\ } and for specific ones,
use either the module or as described in the
\href{https://typst.app/docs/reference/scripting\#modules}{typst docs} .

The main function provided in this package is \texttt{\ ouset\ } for
math environments. This function can take arbitrary many arguments, but
with the following rules:

\begin{itemize}
\tightlist
\item
  if the first argument is \texttt{\ \&\ } , a ‘alignpoint’ is
  inserted immediately before the symbol
\item
  next follows the symbol, then the content to put on top, and then the
  content to put at the bottom
\item
  if the last argument is \texttt{\ \&\ } , a ‘alignpoint’ is
  inserted immediately after the symbol
\end{itemize}

There is a named argument \texttt{\ insert-and\ } , which if false, does
not insert an ‘alignpoint’ in the above cases, but only clips at
these points.

This package provides furthermore 3 other functions:

\begin{itemize}
\tightlist
\item
  \texttt{\ overset(s,\ t,\ c:\ 0,\ insert-and:\ true)\ } : output the
  symbol s with t on top of it
\item
  \texttt{\ underset(s,\ b,\ c:\ 0\ insert-and:\ true)\ } : output the
  symbol s with b on below of it
\item
  \texttt{\ overunderset(s,\ t,\ b,\ c:\ 0,\ insert-and:\ true)\ } :
  output the symbol s with t on top of it and b below it
\end{itemize}

All functions put enough spacing around the operator, such that other
content does not interfere with it. However, this spacing can be
disabled, by setting \texttt{\ c\ } to 1, 2 or 3. This is a flag system
with

\begin{itemize}
\tightlist
\item
  \texttt{\ c=0\ } : normal spacing on the left and right
\item
  \texttt{\ c=1\ } : left spacing is according to the operator / symbol
  s and right spacing is normal
\item
  \texttt{\ c=2\ } : left spacing is normal and right spacing according
  to the operator / symbol s
\item
  \texttt{\ c=3\ } : both spacings are according to the operator /
  symbol s
\end{itemize}

Hence: clip param
\texttt{\ c\ ∈\ \{0,1,2,3\}\ ≜\ \{no\ clip,\ left\ clip,\ right\ clip,\ both\ clip\}\ }

\subsection{Example usage}\label{example-usage}

Try something like:

\begin{itemize}
\item
  \texttt{\ \$ouset(-\/-\textgreater{},,\ n-\textgreater{}oo)\$\ }
\item
  \texttt{\ \$ouset(-,1,2)\$\ }
\item
\begin{Shaded}
\begin{Highlighting}[]
\NormalTok{\#import "@preview/ouset:0.2.0": ouset}

\NormalTok{$ M \&= sum\_(k=0)\^{}oo q\^{}k = 1 + q + q\^{}2 + q\^{}3 + q\^{}4 + dots\textbackslash{}}
\NormalTok{    \&= 1 + q (1 + q + q\^{}2 + q\^{}3 + dots)\textbackslash{}}
\NormalTok{    ouset(\&, =, "Def.", "of" M) 1 + q dot M $}
\end{Highlighting}
\end{Shaded}
\end{itemize}

\subsubsection{How to add}\label{how-to-add}

Copy this into your project and use the import as \texttt{\ ouset\ }

\begin{verbatim}
#import "@preview/ouset:0.2.0"
\end{verbatim}

\includesvg[width=0.16667in,height=0.16667in]{/assets/icons/16-copy.svg}

Check the docs for
\href{https://typst.app/docs/reference/scripting/\#packages}{more
information on how to import packages} .

\subsubsection{About}\label{about}

\begin{description}
\tightlist
\item[Author :]
Ludwig Austermann
\item[License:]
MIT
\item[Current version:]
0.2.0
\item[Last updated:]
May 31, 2024
\item[First released:]
July 6, 2023
\item[Minimum Typst version:]
0.11.0
\item[Archive size:]
2.61 kB
\href{https://packages.typst.org/preview/ouset-0.2.0.tar.gz}{\pandocbounded{\includesvg[keepaspectratio]{/assets/icons/16-download.svg}}}
\item[Repository:]
\href{https://github.com/ludwig-austermann/typst-ouset}{GitHub}
\item[Categor y :]
\begin{itemize}
\tightlist
\item[]
\item
  \pandocbounded{\includesvg[keepaspectratio]{/assets/icons/16-layout.svg}}
  \href{https://typst.app/universe/search/?category=layout}{Layout}
\end{itemize}
\end{description}

\subsubsection{Where to report issues?}\label{where-to-report-issues}

This package is a project of Ludwig Austermann . Report issues on
\href{https://github.com/ludwig-austermann/typst-ouset}{their
repository} . You can also try to ask for help with this package on the
\href{https://forum.typst.app}{Forum} .

Please report this package to the Typst team using the
\href{https://typst.app/contact}{contact form} if you believe it is a
safety hazard or infringes upon your rights.

\phantomsection\label{versions}
\subsubsection{Version history}\label{version-history}

\begin{longtable}[]{@{}ll@{}}
\toprule\noalign{}
Version & Release Date \\
\midrule\noalign{}
\endhead
\bottomrule\noalign{}
\endlastfoot
0.2.0 & May 31, 2024 \\
\href{https://typst.app/universe/package/ouset/0.1.1/}{0.1.1} & July 7,
2023 \\
\href{https://typst.app/universe/package/ouset/0.1.0/}{0.1.0} & July 6,
2023 \\
\end{longtable}

Typst GmbH did not create this package and cannot guarantee correct
functionality of this package or compatibility with any version of the
Typst compiler or app.


\title{typst.app/universe/package/lambdabus}

\phantomsection\label{banner}
\section{lambdabus}\label{lambdabus}

{ 0.1.0 }

Easily parse, normalize and display simple λ-Calculus expressions.

\phantomsection\label{readme}
Lambdabus allows you to parse, normalize and display simple λ-Calculus
expressions in Typst with ease.

\subsection{Usage}\label{usage}

Lambdabus is available on the
\href{https://typst.app/universe/package/lambdabus/}{Typst Universe} and
it is thus recommended to be imported like this:

\begin{Shaded}
\begin{Highlighting}[]
\NormalTok{\#import "@preview/lambdabus:0.1.0" as lmd}
\end{Highlighting}
\end{Shaded}

\subsection{Features/Examples}\label{featuresexamples}

\pandocbounded{\includegraphics[keepaspectratio]{https://raw.github.com/luca-schlecker/typst-lambdabus/v0.1.0/gallery.png}}

\subsubsection{How to add}\label{how-to-add}

Copy this into your project and use the import as \texttt{\ lambdabus\ }

\begin{verbatim}
#import "@preview/lambdabus:0.1.0"
\end{verbatim}

\includesvg[width=0.16667in,height=0.16667in]{/assets/icons/16-copy.svg}

Check the docs for
\href{https://typst.app/docs/reference/scripting/\#packages}{more
information on how to import packages} .

\subsubsection{About}\label{about}

\begin{description}
\tightlist
\item[Author :]
\href{https://github.com/luca-schlecker}{Luca Schlecker}
\item[License:]
MIT
\item[Current version:]
0.1.0
\item[Last updated:]
November 4, 2024
\item[First released:]
November 4, 2024
\item[Archive size:]
3.90 kB
\href{https://packages.typst.org/preview/lambdabus-0.1.0.tar.gz}{\pandocbounded{\includesvg[keepaspectratio]{/assets/icons/16-download.svg}}}
\item[Repository:]
\href{https://github.com/luca-schlecker/typst-lambdabus}{GitHub}
\item[Categor y :]
\begin{itemize}
\tightlist
\item[]
\item
  \pandocbounded{\includesvg[keepaspectratio]{/assets/icons/16-smile.svg}}
  \href{https://typst.app/universe/search/?category=fun}{Fun}
\end{itemize}
\end{description}

\subsubsection{Where to report issues?}\label{where-to-report-issues}

This package is a project of Luca Schlecker . Report issues on
\href{https://github.com/luca-schlecker/typst-lambdabus}{their
repository} . You can also try to ask for help with this package on the
\href{https://forum.typst.app}{Forum} .

Please report this package to the Typst team using the
\href{https://typst.app/contact}{contact form} if you believe it is a
safety hazard or infringes upon your rights.

\phantomsection\label{versions}
\subsubsection{Version history}\label{version-history}

\begin{longtable}[]{@{}ll@{}}
\toprule\noalign{}
Version & Release Date \\
\midrule\noalign{}
\endhead
\bottomrule\noalign{}
\endlastfoot
0.1.0 & November 4, 2024 \\
\end{longtable}

Typst GmbH did not create this package and cannot guarantee correct
functionality of this package or compatibility with any version of the
Typst compiler or app.


\title{typst.app/universe/package/modern-unito-thesis}

\phantomsection\label{banner}
\phantomsection\label{template-thumbnail}
\pandocbounded{\includegraphics[keepaspectratio]{https://packages.typst.org/preview/thumbnails/modern-unito-thesis-0.1.0-small.webp}}

\section{modern-unito-thesis}\label{modern-unito-thesis}

{ 0.1.0 }

A thesis template of the University of Turin

\href{/app?template=modern-unito-thesis&version=0.1.0}{Create project in
app}

\phantomsection\label{readme}
This is a thesis template for the University of Turin (UniTO) based on
\href{https://github.com/eduardz1/Bachelor-Thesis}{my thesis} , since
there are no strict templates (notable mention to
\href{https://github.com/esenes/Unito-thesis-template}{Eugenio’s LaTeX
template though} ) take my choices with a grain of salt, different
supervisors may ask you to customize the template differently. My
choices are loosely based on this document:
\href{https://elearning.unito.it/sme/pluginfile.php/29485/mod_folder/content/0/format_TESI_2011-2012.pdf}{Indicazioni
per il Format della Tesi} .

If you find errors or ways to improve the template please open an issue
or contribute directly with a PR.

\subsection{Usage}\label{usage}

In the Typst web app simply click “Start from template� on the
dashboard and search for \texttt{\ modern-unito-thesis\ } .

From the CLI you can initialize the project with the command

\begin{Shaded}
\begin{Highlighting}[]
\ExtensionTok{typst}\NormalTok{ init @preview/modern{-}unito{-}thesis}
\end{Highlighting}
\end{Shaded}

A new directory with all the files needed to get started will be
created.

\subsection{Configuration}\label{configuration}

This template exports the \texttt{\ template\ } function with the
following named arguments:

\begin{itemize}
\tightlist
\item
  \texttt{\ title\ } : the title of the thesis
\item
  \texttt{\ academic-year\ } : the academic year (e.g. 2023/2024)
\item
  \texttt{\ subtitle\ } : e.g. “Bachelor’s Thesis�
\item
  \texttt{\ paper-size\ } (default \texttt{\ a4\ } ): the paper format
\item
  \texttt{\ candidate\ } : your name, surname and matricola (student id)
\item
  \texttt{\ supervisor\ } (relatore): your supervisor’s name and
  surname
\item
  \texttt{\ co-supervisor\ } (correlatore): an array of your
  co-supervisors’ names and surnames
\item
  \texttt{\ affiliation\ } : a dictionary that specifies
  \texttt{\ university\ } , \texttt{\ school\ } and \texttt{\ degree\ }
  keywords
\item
  \texttt{\ lang\ } : configurable between \texttt{\ en\ } for English
  and \texttt{\ it\ } for Italian
\item
  \texttt{\ bibliography-path\ } : the path to your bibliography file
  (e.g. \texttt{\ works.bib\ } )
\item
  \texttt{\ logo\ } (already set to UniTO’s logo by default): the path
  to your university’s logo
\item
  \texttt{\ abstract\ } : your thesis’ abstract, can be set to
  \texttt{\ none\ } if not needed
\item
  \texttt{\ acknowledgments\ } : your thesis’ acknowledgments, can be
  set to \texttt{\ none\ } if not needed
\item
  \texttt{\ keywords\ } : a list of keywords for the thesis, can be set
  to \texttt{\ none\ } if not needed
\end{itemize}

The template will initialize an example project with sensible defaults.

The template divides the level 1 headings in chapters under the
\texttt{\ chapters\ } directory, I suggest using this structure to keep
the project organized.

If you want to change an existing project to use this template, you can
add a show rule like this at the top of your file:

\begin{Shaded}
\begin{Highlighting}[]
\NormalTok{\#import "@preview/modern{-}unito{-}thesis:0.1.0": template}

\NormalTok{\#show: template.with(}
\NormalTok{  title: "My Beautiful Thesis",}
\NormalTok{  academic{-}year: [2023/2024],}
\NormalTok{  subtitle: "Bachelor\textquotesingle{}s Thesis",}
\NormalTok{  logo: image("path/to/your/logo.png"),}
\NormalTok{  candidate: (}
\NormalTok{    name: "Eduard Antonovic Occhipinti",}
\NormalTok{    matricola: 947847}
\NormalTok{  ),}
\NormalTok{  supervisor: (}
\NormalTok{    "Prof. Luigi Paperino"}
\NormalTok{  ),}
\NormalTok{  co{-}supervisor: (}
\NormalTok{    "Dott. Pluto Mario",}
\NormalTok{    "Dott. Minni Topolino"}
\NormalTok{  ),}
\NormalTok{  affiliation: (}
\NormalTok{    university: "Università degli Studi di Torino",}
\NormalTok{    school: "Scuola di Scienze della Natura",}
\NormalTok{    degree: "Corso di Laurea Triennale in Informatica",}
\NormalTok{  ),}
\NormalTok{  bibliography: bibliography("works.yml"),}
\NormalTok{  abstract: [Your abstract goes here],}
\NormalTok{  acknowledgments: [Your acknowledgments go here],}
\NormalTok{  keywords: [keyword1, keyword2, keyword3]}
\NormalTok{)}

\NormalTok{// Your content goes here}
\end{Highlighting}
\end{Shaded}

\subsection{Compile}\label{compile}

To compile the project from the CLI you just need to run

\begin{Shaded}
\begin{Highlighting}[]
\ExtensionTok{typst}\NormalTok{ compile main.typ}
\end{Highlighting}
\end{Shaded}

or if you want to watch for changes (recommended)

\begin{Shaded}
\begin{Highlighting}[]
\ExtensionTok{typst}\NormalTok{ watch main.typ}
\end{Highlighting}
\end{Shaded}

\subsection{Bibliography}\label{bibliography}

I integrated the bibliography as a
\href{https://github.com/typst/hayagriva}{Hayagriva} \texttt{\ yaml\ }
file under
\href{https://github.com/typst/packages/raw/main/packages/preview/modern-unito-thesis/0.1.0/template/works.yml}{works.yml}
, nonetheless using the more common \texttt{\ bib\ } format for your
bibliography management is as simple as passing a BibTex file to the
template \texttt{\ bibliography\ } parameter. Given that our university
is not strict in this regard I suggest using Hayagriva though :).

\href{/app?template=modern-unito-thesis&version=0.1.0}{Create project in
app}

\subsubsection{How to use}\label{how-to-use}

Click the button above to create a new project using this template in
the Typst app.

You can also use the Typst CLI to start a new project on your computer
using this command:

\begin{verbatim}
typst init @preview/modern-unito-thesis:0.1.0
\end{verbatim}

\includesvg[width=0.16667in,height=0.16667in]{/assets/icons/16-copy.svg}

\subsubsection{About}\label{about}

\begin{description}
\tightlist
\item[Author :]
Eduard Antonovic Occhipinti
\item[License:]
MIT
\item[Current version:]
0.1.0
\item[Last updated:]
March 20, 2024
\item[First released:]
March 20, 2024
\item[Minimum Typst version:]
0.11.0
\item[Archive size:]
19.6 kB
\href{https://packages.typst.org/preview/modern-unito-thesis-0.1.0.tar.gz}{\pandocbounded{\includesvg[keepaspectratio]{/assets/icons/16-download.svg}}}
\item[Repository:]
\href{https://github.com/eduardz1/unito-typst-template}{GitHub}
\item[Categor y :]
\begin{itemize}
\tightlist
\item[]
\item
  \pandocbounded{\includesvg[keepaspectratio]{/assets/icons/16-mortarboard.svg}}
  \href{https://typst.app/universe/search/?category=thesis}{Thesis}
\end{itemize}
\end{description}

\subsubsection{Where to report issues?}\label{where-to-report-issues}

This template is a project of Eduard Antonovic Occhipinti . Report
issues on \href{https://github.com/eduardz1/unito-typst-template}{their
repository} . You can also try to ask for help with this template on the
\href{https://forum.typst.app}{Forum} .

Please report this template to the Typst team using the
\href{https://typst.app/contact}{contact form} if you believe it is a
safety hazard or infringes upon your rights.

\phantomsection\label{versions}
\subsubsection{Version history}\label{version-history}

\begin{longtable}[]{@{}ll@{}}
\toprule\noalign{}
Version & Release Date \\
\midrule\noalign{}
\endhead
\bottomrule\noalign{}
\endlastfoot
0.1.0 & March 20, 2024 \\
\end{longtable}

Typst GmbH did not create this template and cannot guarantee correct
functionality of this template or compatibility with any version of the
Typst compiler or app.


\title{typst.app/universe/package/wenyuan-campaign}

\phantomsection\label{banner}
\phantomsection\label{template-thumbnail}
\pandocbounded{\includegraphics[keepaspectratio]{https://packages.typst.org/preview/thumbnails/wenyuan-campaign-0.1.0-small.webp}}

\section{wenyuan-campaign}\label{wenyuan-campaign}

{ 0.1.0 }

Easily write DnD 5e style campaign documents.

\href{/app?template=wenyuan-campaign&version=0.1.0}{Create project in
app}

\phantomsection\label{readme}
A template for writing RPG campaigns imitating the 5e theme. This was
made as a typst version of the \$\textbackslash LaTeX\$ package
\href{https://github.com/rpgtex/DND-5e-LaTeX-Template}{DnD 5e LaTeX
Template} , though it is not functionally nor entirely visually similar.

Packages:

\begin{itemize}
\tightlist
\item
  \texttt{\ droplet:0.3.1\ }
\end{itemize}

Fonts:

\begin{itemize}
\tightlist
\item
  TeX Gyre Bonum
\item
  Scaly Sans
\item
  Scaly Sans Caps
\item
  Royal Initalen
\item
  京è?¯è€?宋ä½`` KingHwa OldSong
\end{itemize}

\emph{\textbf{Please note: in an effort to reduce the file size of the
template, fonts are included in MY repository only, not in the typst
official one.}} You may find the fonts in my
\href{https://github.com/yanwenywan/typst-packages/tree/master/wenyuan-campaign/0.1.0/template/fonts}{github
repository in the fonts folder} , or download them yourself, or heck
provide your own fonts to your liking.

\begin{verbatim}
typst init @preview/wenyuan-campaign:0.1.0
\end{verbatim}

This will copy over all required fonts and comes prefilled with the
standard template so you can see how it works. To use this you need to
either install all the fonts locally or pass the folder into
-\/-font-path when compiling.

To initialise the style, do:

\begin{Shaded}
\begin{Highlighting}[]
\NormalTok{\#import "@preview/wenyuan{-}campaign:0.1.0": *}

\NormalTok{\#show: conf.with() }
\end{Highlighting}
\end{Shaded}

Very easy.

Optionally, you may set all the theme fonts from the configure function
(the defaults are shown):

\begin{Shaded}
\begin{Highlighting}[]
\NormalTok{\#import "@preview/wenyuan{-}campaign:0.1.0": *}

\NormalTok{\#show: conf.with(}
\NormalTok{    fontsize: 10pt,}
\NormalTok{    mainFont: ("TeX Gyre Bonum", "KingHwa\_OldSong"),}
\NormalTok{    titleFont: ("TeX Gyre Bonum", "KingHwa\_OldSong"),}
\NormalTok{    sansFont: ("Scaly Sans Remake", "KingHwa\_OldSong"),}
\NormalTok{    sansSmallcapsFont: ("Scaly Sans Caps", "KingHwa\_OldSong"),}
\NormalTok{    dropcapFont: "Royal Initialen"}
\NormalTok{) }
\end{Highlighting}
\end{Shaded}

You are encouraged to copy the template files and modify them if they
are not up to your liking.

\textbf{set-theme-colour} \texttt{\ (colour:\ color)\ } .\\
Sets a theme colour from the colours package of this module or any other
colour you wantâ€''up to you if it looks bad :)\\
The colours recommended are:

\begin{quote}
phbgreen, phbcyan, phbmauve, phbtan, dmglavender, dmgcoral, dmgslategrey
(-ay), dmglilac
\end{quote}

\begin{center}\rule{0.5\linewidth}{0.5pt}\end{center}

\textbf{make-title}
\texttt{\ (title:\ content,\ subtitle:\ content\ =\ {[}{]},\ author:\ content\ =\ {[}{]},\ date:\ content\ =\ {[}{]},\ anything-before:\ content\ =\ {[}{]},\ anything-after:\ content\ =\ {[}{]})\ }
.\\
Makes a simple title page.

Parameters:

\begin{itemize}
\tightlist
\item
  title: main book title
\item
  subtitle: (optional) subtitle
\item
  author: (optional)
\item
  date: (optional) â€`` just acts as a separate line, can be used for
  anything else
\item
  anything-before: (optional) this is put before the title
\item
  anything-after: (optional) this is put after the date
\end{itemize}

\begin{center}\rule{0.5\linewidth}{0.5pt}\end{center}

\textbf{drop-paragraph}
\texttt{\ (small-caps:\ string\ =\ "",\ body:\ content)\ } .\\
Makes a paragraph with a drop capital.

Parameters:

\begin{itemize}
\tightlist
\item
  small-caps: (optional) any text you wish to be rendered in small caps,
  like how DnD does it
\item
  body: anything else
\end{itemize}

\begin{center}\rule{0.5\linewidth}{0.5pt}\end{center}

\textbf{bump} \texttt{\ ()\ } .\\
Manually does a 1em paragraph space

\begin{center}\rule{0.5\linewidth}{0.5pt}\end{center}

\textbf{namedpar} \texttt{\ (title:\ content,\ content:\ content)\ } .\\
A paragraph with a bold italic name at the start.

Parameters:

\begin{itemize}
\tightlist
\item
  title: the bold italic name; a full stop/period is added immediately
  after for you
\item
  content: everything else
\end{itemize}

\begin{center}\rule{0.5\linewidth}{0.5pt}\end{center}

\textbf{namedpar-block}
\texttt{\ (title:\ content,\ content:\ content)\ } .\\
See \textbf{namedpar} , but this one is in a block environment.

\begin{center}\rule{0.5\linewidth}{0.5pt}\end{center}

\textbf{readaloud} \texttt{\ {[}{]}\ } .\\
A tan-coloured read-aloud box with some decorations.

\begin{center}\rule{0.5\linewidth}{0.5pt}\end{center}

\textbf{comment-box}
\texttt{\ (title:\ content\ =\ {[}{]},\ content:\ content)\ } .\\
A theme-coloured plain comment box.

Parameters:

\begin{itemize}
\tightlist
\item
  title: (optional) a title in bold small caps
\item
  content
\end{itemize}

\begin{center}\rule{0.5\linewidth}{0.5pt}\end{center}

\textbf{fancy-comment-box}
\texttt{\ (title:\ content\ =\ {[}{]},\ content:\ content)\ } .\\
A theme-coloured fancy comment box with decorations.

Parameters:

\begin{itemize}
\tightlist
\item
  title: (optional) a title in bold small caps
\item
  content
\end{itemize}

\begin{center}\rule{0.5\linewidth}{0.5pt}\end{center}

\textbf{dndtable} \texttt{\ (...)\ } . A theme-coloured dnd-style table.
Parameters are identical to table except stroke, fill, and inset are not
included.

\begin{center}\rule{0.5\linewidth}{0.5pt}\end{center}

\textbf{sctitle} \texttt{\ {[}{]}\ } .\\
Makes a small caps header block.

\begin{center}\rule{0.5\linewidth}{0.5pt}\end{center}

\textbf{begin-stat} \texttt{\ {[}{]}\ } .\\
Begins the monster statblock environment.

\begin{center}\rule{0.5\linewidth}{0.5pt}\end{center}

\textbf{begin-item} \texttt{\ {[}{]}\ } .\\
Begins the item environment.

\emph{\textbf{Important.}} Statblocks are provided under the
\texttt{\ stat\ } namespace, and will only work as intended in a
\texttt{\ beginStat\ } block. All statblock functions must be prepended
with \texttt{\ stat\ } .

\subsection{stat functions}\label{stat-functions}

\textbf{dice} \texttt{\ (value:\ str)\ }\\
Parses a dice string (e.g., \texttt{\ 3d6\ } , \texttt{\ 3d6+2\ } , or
\texttt{\ 3d6-1\ } ) and returns a formatted dice value (e.g., “10
(3d6)�). Specifically, the types of strings it accepts are:

\begin{quote}
\texttt{\ \textbackslash{}d+d\textbackslash{}d+({[}+-{]}\textbackslash{}d+)?\ }
(number \texttt{\ d\ } number \texttt{\ +/-\ } number)
\end{quote}

(You need to ensure the string is correct.)

\begin{center}\rule{0.5\linewidth}{0.5pt}\end{center}

\textbf{dice-raw}
\texttt{\ (num-dice:\ int,\ dice-face:\ int,\ modifier:\ int)\ }\\
A helper function for the above, optionally used. It takes all values as
integers and prints the correct formatting.

\begin{center}\rule{0.5\linewidth}{0.5pt}\end{center}

\textbf{statheading} \texttt{\ (title,\ desc\ =\ {[}{]})\ }\\
Takes a title and description, formatting it into a top-level monster
name heading. \texttt{\ desc\ } is the description of the monster, e.g.,
\emph{Medium humanoid, neutral evil} , but it can be anything.

\begin{center}\rule{0.5\linewidth}{0.5pt}\end{center}

\textbf{stroke} \texttt{\ ()\ } .\\
Draws a red stroke with a fading right edge.

\begin{center}\rule{0.5\linewidth}{0.5pt}\end{center}

\textbf{main-stats}
\texttt{\ (ac\ =\ "",\ hp-dice\ =\ "",\ speed\ =\ "30ft",\ hp-etc\ =\ "")\ }\\
Produces \textbf{Armor Class} , \textbf{Hit Points} , and \textbf{Speed}
in one go. All fields are optional. \texttt{\ hp-dice\ } accepts a
\emph{valid dice string only} â€''if you do not want to use dice, leave
it blank and use \texttt{\ hp-etc\ } . No restrictions on other fields.

\begin{center}\rule{0.5\linewidth}{0.5pt}\end{center}

\textbf{ability} \texttt{\ (str,\ dex,\ con,\ int,\ wis,\ cha)\ }\\
Takes the six ability scores (base values) as integers and formats them
into a table with appropriate modifiers.

\begin{center}\rule{0.5\linewidth}{0.5pt}\end{center}

\textbf{challenge} \texttt{\ (cr:\ str)\ }\\
Takes a numeric challenge rating (as a string) and formats it along with
the XP (if the challenge rating is valid). All CRs between 0â€``30 are
valid, including the fractional \texttt{\ 1/8\ } , \texttt{\ 1/4\ } ,
\texttt{\ 1/2\ } (which can also be written in decimal form, e.g.,
\texttt{\ 0.125\ } ).

\begin{center}\rule{0.5\linewidth}{0.5pt}\end{center}

\textbf{skill} \texttt{\ (title,\ contents)\ }\\
Takes a title and description, creating a single skills entry. For
example, \texttt{\ \#skill("Challenge",\ challenge(1))\ } will produce
(in red):

\begin{quote}
\textbf{Challenge} 1 (200 XP)
\end{quote}

(This uses \texttt{\ challenge\ } from above.)

\textbf{Section headers} such as \emph{Actions} or \emph{Reactions} are
done using the second-level header \texttt{\ ==\ }

\textbf{Action names} â€`` the names that go in front of actions /
abilities are done using the third level header \texttt{\ ===\ } (do not
leave a blank line between the header and its body text)

\emph{\textbf{Important.}} Basic item capability is provided under the
\texttt{\ item\ } namespace, and will only work as intended in a
\texttt{\ beginItem\ } block. All item functions must be prepended with
\texttt{\ item\ } .

\subsection{item functions}\label{item-functions}

\textbf{Item Name} is done with the top-level header \texttt{\ =\ }

\textbf{Section headers} are the second level header \texttt{\ ==\ }

\textbf{Abilities and named paragraphs} are the third level header
\texttt{\ ===\ }

\begin{center}\rule{0.5\linewidth}{0.5pt}\end{center}

\textbf{smalltext} \texttt{\ {[}{]}\ } . Half-size text for item
subheadings

\textbf{flavourtext} \texttt{\ {[}{]}\ } . Indented italic flavour text

\pandocbounded{\includegraphics[keepaspectratio]{https://github.com/typst/packages/raw/main/packages/preview/wenyuan-campaign/0.1.0/sample.png}}

\begin{itemize}
\tightlist
\item
  The overall style is based on the
  \href{https://github.com/rpgtex/DND-5e-LaTeX-Template}{Dnd 5e LaTeX
  Template} , which in turn replicate the base DnD aesthetic.
\item
  TeX Gyre Bonum by GUST e-Foundry is used for the body text
\item
  Scaly Sans and Scaly Sans Caps are part of
  \href{https://github.com/jonathonf/solbera-dnd-fonts}{Solbera’s CC
  Alternatives to DnD Fonts} and are used for main body text.
  \emph{\textbf{Note that these fonts are CC-BY-SA i.e. Share-Alike, so
  keep that in mind. This shouldn’t affect homebrew created using
  these fonts (just like how a painting made with a CC-BY-SA art program
  isn’t itself CC-BY-SA) but what do I know I’m not a lawyer.}}
\item
  \href{https://zhuanlan.zhihu.com/p/637491623}{KingHwa\_OldSong}
  (京è?¯è€?宋ä½``) is a traditional Chinese print font used for all
  CJK text (if present, mostly because I need it)
\end{itemize}

\href{/app?template=wenyuan-campaign&version=0.1.0}{Create project in
app}

\subsubsection{How to use}\label{how-to-use}

Click the button above to create a new project using this template in
the Typst app.

You can also use the Typst CLI to start a new project on your computer
using this command:

\begin{verbatim}
typst init @preview/wenyuan-campaign:0.1.0
\end{verbatim}

\includesvg[width=0.16667in,height=0.16667in]{/assets/icons/16-copy.svg}

\subsubsection{About}\label{about}

\begin{description}
\tightlist
\item[Author :]
\href{https://github.com/yanwenywan}{Yan Xin}
\item[License:]
Apache-2.0
\item[Current version:]
0.1.0
\item[Last updated:]
November 28, 2024
\item[First released:]
November 28, 2024
\item[Archive size:]
403 kB
\href{https://packages.typst.org/preview/wenyuan-campaign-0.1.0.tar.gz}{\pandocbounded{\includesvg[keepaspectratio]{/assets/icons/16-download.svg}}}
\item[Repository:]
\href{https://github.com/yanwenywan/typst-packages/tree/master/wenyuan-campaign}{GitHub}
\item[Categor ies :]
\begin{itemize}
\tightlist
\item[]
\item
  \pandocbounded{\includesvg[keepaspectratio]{/assets/icons/16-layout.svg}}
  \href{https://typst.app/universe/search/?category=layout}{Layout}
\item
  \pandocbounded{\includesvg[keepaspectratio]{/assets/icons/16-text.svg}}
  \href{https://typst.app/universe/search/?category=text}{Text}
\item
  \pandocbounded{\includesvg[keepaspectratio]{/assets/icons/16-docs.svg}}
  \href{https://typst.app/universe/search/?category=book}{Book}
\end{itemize}
\end{description}

\subsubsection{Where to report issues?}\label{where-to-report-issues}

This template is a project of Yan Xin . Report issues on
\href{https://github.com/yanwenywan/typst-packages/tree/master/wenyuan-campaign}{their
repository} . You can also try to ask for help with this template on the
\href{https://forum.typst.app}{Forum} .

Please report this template to the Typst team using the
\href{https://typst.app/contact}{contact form} if you believe it is a
safety hazard or infringes upon your rights.

\phantomsection\label{versions}
\subsubsection{Version history}\label{version-history}

\begin{longtable}[]{@{}ll@{}}
\toprule\noalign{}
Version & Release Date \\
\midrule\noalign{}
\endhead
\bottomrule\noalign{}
\endlastfoot
0.1.0 & November 28, 2024 \\
\end{longtable}

Typst GmbH did not create this template and cannot guarantee correct
functionality of this template or compatibility with any version of the
Typst compiler or app.


