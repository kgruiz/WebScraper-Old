\title{typst.app/universe/package/silky-report-insa}

\phantomsection\label{banner}
\phantomsection\label{template-thumbnail}
\pandocbounded{\includegraphics[keepaspectratio]{https://packages.typst.org/preview/thumbnails/silky-report-insa-0.4.0-small.webp}}

\section{silky-report-insa}\label{silky-report-insa}

{ 0.4.0 }

A template made for reports and other documents of INSA, a French
engineering school.

\href{/app?template=silky-report-insa&version=0.4.0}{Create project in
app}

\phantomsection\label{readme}
Typst Template for full documents and reports for the french engineering
school INSA.

\subsection{Table of contents}\label{table-of-contents}

\begin{enumerate}
\tightlist
\item
  \href{https://github.com/typst/packages/raw/main/packages/preview/silky-report-insa/0.4.0/\#examples}{Examples
  \& Usage}

  \begin{enumerate}
  \tightlist
  \item
    \href{https://github.com/typst/packages/raw/main/packages/preview/silky-report-insa/0.4.0/\#\%F0\%9F\%A7\%AA-tp-report}{🧪
    TP report}
  \item
    \href{https://github.com/typst/packages/raw/main/packages/preview/silky-report-insa/0.4.0/\#\%F0\%9F\%93\%9A-internship-report}{ðŸ``š
    Internship report}
  \item
    \href{https://github.com/typst/packages/raw/main/packages/preview/silky-report-insa/0.4.0/\#\%F0\%9F\%97\%92\%EF\%B8\%8F-blank-templates}{ðŸ---'ï¸?
    Blank templates}
  \end{enumerate}
\item
  \href{https://github.com/typst/packages/raw/main/packages/preview/silky-report-insa/0.4.0/\#fonts}{Fonts
  information}
\item
  \href{https://github.com/typst/packages/raw/main/packages/preview/silky-report-insa/0.4.0/\#notes}{Notes}
\item
  \href{https://github.com/typst/packages/raw/main/packages/preview/silky-report-insa/0.4.0/\#license}{License}
\item
  \href{https://github.com/typst/packages/raw/main/packages/preview/silky-report-insa/0.4.0/\#changelog}{Changelog}
\end{enumerate}

\subsection{Examples \& Usage}\label{examples-usage}

\subsubsection{🧪 TP report}\label{uxf0uxffuxaa-tp-report}

\pandocbounded{\includegraphics[keepaspectratio]{https://github.com/typst/packages/raw/main/packages/preview/silky-report-insa/0.4.0/thumbnail-insa-report.png}}

This is the default report for the \texttt{\ silky-report-insa\ }
package. It uses the \texttt{\ insa-report\ } show rule.\\
It is primarily used for reports of Practical Works (Travaux Pratiques).

\paragraph{Example}\label{example}

\begin{Shaded}
\begin{Highlighting}[]
\NormalTok{\#import "@preview/silky{-}report{-}insa:0.4.0": *}
\NormalTok{\#show: doc =\textgreater{} insa{-}report(}
\NormalTok{  id: 3,}
\NormalTok{  pre{-}title: "STPI 2",}
\NormalTok{  title: "Interférences et diffraction",}
\NormalTok{  authors: [}
\NormalTok{    *LE JEUNE Youenn*}

\NormalTok{    *MAUVY Eva*}
    
\NormalTok{    Groupe D}

\NormalTok{    Binôme 5}
\NormalTok{  ],}
\NormalTok{  date: "11/04/2023",}
\NormalTok{  insa: "rennes",}
\NormalTok{  doc)}

\NormalTok{= Introduction}
\NormalTok{Le but de ce TP est d’interpréter les figures de diffraction observées avec différents objets diffractants}
\NormalTok{et d’en déduire les dimensions de ces objets.}

\NormalTok{= Partie théorique {-} Phénomène d\textquotesingle{}interférence}
\NormalTok{== Diffraction par une fente double}
\NormalTok{Lors du passage de la lumière par une fente double de largeur $a$ et de distance $b$ entre les centres}
\NormalTok{des fentes...}
\end{Highlighting}
\end{Shaded}

\paragraph{Parameters}\label{parameters}

\begin{longtable}[]{@{}llll@{}}
\toprule\noalign{}
Parameter & Description & Type & Example \\
\midrule\noalign{}
\endhead
\bottomrule\noalign{}
\endlastfoot
\textbf{id} & TP number & int & \texttt{\ 1\ } \\
\textbf{pre-title} & Text written before the title & str &
\texttt{\ "STPI\ 2"\ } \\
\textbf{title} & Title of the TP & str &
\texttt{\ "Interférences\ et\ diffraction"\ } \\
\textbf{authors} & Authors & content &
\texttt{\ {[}\textbackslash{}*LE\ JEUNE\ Youenn\textbackslash{}*{]}\ } \\
\textbf{date} & Date of the TP & datetime/str &
\texttt{\ "11/04/2023"\ } \\
\textbf{insa} & INSA name ( \texttt{\ rennes\ } , \texttt{\ hdf\ } …)
& str & \texttt{\ "rennes"\ } \\
\textbf{lang} & Language & str & \texttt{\ "fr"\ } \\
\end{longtable}

\subsubsection{ðŸ``š Internship
report}\label{uxf0uxffux161-internship-report}

\pandocbounded{\includegraphics[keepaspectratio]{https://github.com/typst/packages/raw/main/packages/preview/silky-report-insa/0.4.0/thumbnail-insa-stage.png}}

If you want to make an internship report, you will need to use another
show rule: \texttt{\ insa-stage\ } .

\paragraph{Example}\label{example-1}

\begin{Shaded}
\begin{Highlighting}[]
\NormalTok{\#import "@preview/silky{-}report{-}insa:0.4.0": *}
\NormalTok{\#show: doc =\textgreater{} insa{-}stage(}
\NormalTok{  "Youenn LE JEUNE",}
\NormalTok{  "INFO",}
\NormalTok{  "2023{-}2024",}
\NormalTok{  "Real{-}time virtual interaction with deformable structure",}
\NormalTok{  "Sapienza University of Rome",}
\NormalTok{  image("logo{-}example.png"),}
\NormalTok{  "Marilena VENDITELLI",}
\NormalTok{  "Bertrand COUASNON",}
\NormalTok{  [}
\NormalTok{    Résumé du stage en français.}
\NormalTok{  ],}
\NormalTok{  [}
\NormalTok{    Summary of the internship in english.}
\NormalTok{  ],}
\NormalTok{  insa: "rennes",}
\NormalTok{  lang: "fr",}
\NormalTok{  doc}
\NormalTok{)}

\NormalTok{= Introduction}
\NormalTok{Présentation de l\textquotesingle{}entreprise, tout ça tout ça.}

\NormalTok{\#pagebreak()}
\NormalTok{= Travail réalisé}
\NormalTok{== Première partie}
\NormalTok{Blabla}

\NormalTok{== Seconde partie}
\NormalTok{Bleble}

\NormalTok{\#pagebreak()}
\NormalTok{= Conclusion}
\NormalTok{Conclusion random}

\NormalTok{\#pagebreak()}
\NormalTok{= Annexes}
\end{Highlighting}
\end{Shaded}

This template can also be used for a report that is written in english:
in this case, add the \texttt{\ lang:\ "en"\ } parameter to the function
call in the show rule.

\paragraph{Parameters}\label{parameters-1}

\begin{longtable}[]{@{}lllll@{}}
\toprule\noalign{}
\textbf{Parameter} & Required & Type & Description & Example \\
\midrule\noalign{}
\endhead
\bottomrule\noalign{}
\endlastfoot
\textbf{name} & yes & str & Name of the student &
\texttt{\ "Youenn\ LE\ JEUNE"\ } \\
\textbf{department} & yes & str & Department of the student &
\texttt{\ "INFO"\ } \\
\textbf{year} & yes & str & School year during the internship &
\texttt{\ "2023-2024"\ } \\
\textbf{title} & yes & str & Title of the internship &
\texttt{\ "Real-time\ virtual\ interaction\ with\ deformable\ structure"\ } \\
\textbf{company} & yes & str & Company &
\texttt{\ Sapienza\ University\ of\ Rome\ } \\
\textbf{company-logo} & yes & content & Logo of the company &
\texttt{\ image("logo-example.png")\ } \\
\textbf{company-tutor} & yes & str & Tutor in the company &
\texttt{\ "Marilena\ VENDITELLI"\ } \\
\textbf{insa-tutor} & yes & str & Tutor at INSA &
\texttt{\ "Bertrand\ COUASNON"\ } \\
\textbf{insa-tutor-suffix} & no & str & Suffix at the end of
“encadrant� in french & \texttt{\ "e"\ } \\
\textbf{summary-french} & yes & content & Summary in French &
\texttt{\ {[}\ Résumé\ du\ stage\ en\ français.\ {]}\ } \\
\textbf{summary-english} & yes & content & Summary in English &
\texttt{\ {[}\ Summary\ of\ the\ internship\ in\ english.\ {]}\ } \\
\textbf{student-suffix} & no & str & Suffix at the end of
“ingénieur� in french & \texttt{\ "e"\ } \\
\textbf{thanks-page} & no & content & Special thanks page. &
\texttt{\ {[}\ Thanks\ to\ my\ *supervisor*,\ blah\ blah\ blah.\ {]}\ } \\
\textbf{omit-outline} & no & bool & Whether to skip the outline page or
not & \texttt{\ false\ } \\
\textbf{insa} & no & str & INSA name ( \texttt{\ rennes\ } ,
\texttt{\ hdf\ } …) & \texttt{\ "rennes"\ } \\
\textbf{lang} & no & str & Language of the template. Some strings are
translated. & \texttt{\ "fr"\ } \\
\end{longtable}

\subsubsection{ðŸ---'ï¸? Blank
templates}\label{uxf0uxffuxef-blank-templates}

\pandocbounded{\includegraphics[keepaspectratio]{https://github.com/typst/packages/raw/main/packages/preview/silky-report-insa/0.4.0/thumbnail-insa-document.png}}

If you do not want the preformatted output with “TP x�, the title
and date in the header, etc. you can simply use the
\texttt{\ insa-document\ } show rule and customize all by yourself.

\paragraph{Blank template types}\label{blank-template-types}

The graphic charter provides 3 different document types, that are
translated in this Typst template under those names:

\begin{itemize}
\tightlist
\item
  \textbf{\texttt{\ light\ }} , which does not have many color and can
  be printed easily. Has 3 spots to write on the cover:
  \texttt{\ cover-top-left\ } , \texttt{\ cover-middle-left\ } and
  \texttt{\ cover-bottom-right\ } .
\item
  \textbf{\texttt{\ colored\ }} , which is beautiful but consumes a lot
  of ink to print. Only has 1 spot to write on the cover:
  \texttt{\ cover-top-left\ } .
\item
  \textbf{\texttt{\ pfe\ }} , which is primarily used for internship
  reports. Has 4 spots to write on both the front and back covers:
  \texttt{\ cover-top-left\ } , \texttt{\ cover-middle-left\ } ,
  \texttt{\ cover-bottom-right\ } and \texttt{\ back-cover\ } .
\end{itemize}

The document type must be the first argument of the
\texttt{\ insa-document\ } function.

\paragraph{Example}\label{example-2}

\begin{Shaded}
\begin{Highlighting}[]
\NormalTok{\#import "@preview/silky{-}report{-}insa:0.4.0": *}
\NormalTok{\#show: doc =\textgreater{} insa{-}document(}
\NormalTok{  "light",}
\NormalTok{  cover{-}top{-}left: [*Document important*],}
\NormalTok{  cover{-}middle{-}left: [}
\NormalTok{    NOM Prénom}

\NormalTok{    Département INFO}
\NormalTok{  ],}
\NormalTok{  cover{-}bottom{-}right: "uwu",}
\NormalTok{  page{-}header: "En{-}tête au pif",}
\NormalTok{  doc}
\NormalTok{)}
\end{Highlighting}
\end{Shaded}

\paragraph{Parameters}\label{parameters-2}

\begin{longtable}[]{@{}lll@{}}
\toprule\noalign{}
\textbf{Parameter} & Type & Description \\
\midrule\noalign{}
\endhead
\bottomrule\noalign{}
\endlastfoot
\textbf{cover-type} & str & ( \textbf{REQUIRED} ) Type of cover.
Available are: light, colored, pfe. \\
\textbf{cover-top-left} & content & \\
\textbf{cover-middle-left} & content & \\
\textbf{cover-bottom-right} & content & \\
\textbf{back-cover} & content & What to display on the back cover. \\
\textbf{page-header} & content & Header of the pages (except the front
and back). If \texttt{\ none\ } , will display the INSA logo. If not
empty, will display the passed content with an underline. \\
\textbf{page-footer} & content & Footer of the pages (except the front
and back). The page counter will be displayed at the right of the
footer, except if the page number is 0. \\
\textbf{include-back-cover} & bool & whether to add the back cover or
not. \\
\textbf{insa} & str & INSA name ( \texttt{\ rennes\ } , \texttt{\ hdf\ }
…) \\
\textbf{lang} & str & Language of the template. Some strings are
translated. \\
\textbf{metadata-title} & content & Title of the document that will be
embedded in the PDF metadata. \\
\textbf{metadata-authors} & str list & Authors that will be embedded in
the PDF metadata. \\
\textbf{metadata-date} & datetime & Date that will be set as the
document creation date. If not specified, will be set to now. \\
\end{longtable}

\subsection{Fonts}\label{fonts}

The graphic charter recommends the fonts \textbf{League Spartan} for
headings and \textbf{Source Serif} for regular text. To have the best
look, you should install those fonts.

\begin{quote}
You can download the fonts from
\href{https://github.com/SkytAsul/INSA-Typst-Template/tree/main/fonts}{here}
.
\end{quote}

To behave correctly on computers lacking those specific fonts, this
template will automatically fallback to similar ones:

\begin{itemize}
\tightlist
\item
  \textbf{League Spartan} -\textgreater{} \textbf{Arial} (approved by
  INSA’s graphic charter, by default in Windows) -\textgreater{}
  \textbf{Liberation Sans} (by default in most Linux)
\item
  \textbf{Source Serif} -\textgreater{} \textbf{Source Serif 4}
  (downloadable for free) -\textgreater{} \textbf{Georgia} (approved by
  the graphic charter) -\textgreater{} \textbf{Linux Libertine} (default
  Typst font)
\end{itemize}

\subsubsection{Note on variable fonts}\label{note-on-variable-fonts}

If you want to install those fonts on your computer, Typst might not
recognize them if you install their \emph{Variable} versions. You should
install the static versions ( \textbf{League Spartan Bold} and most
versions of \textbf{Source Serif} ).

Keep an eye on \href{https://github.com/typst/typst/issues/185}{the
issue in Typst bug tracker} to see when variable fonts will be used!

\subsection{Notes}\label{notes}

This template is being developed by Youenn LE JEUNE from the INSA de
Rennes in \href{https://github.com/SkytAsul/INSA-Typst-Template}{this
repository} .

For now it includes assets from the graphic charters of those INSAs:

\begin{itemize}
\tightlist
\item
  Rennes ( \texttt{\ rennes\ } )
\item
  Hauts de France ( \texttt{\ hdf\ } )
\item
  Centre Val de Loire ( \texttt{\ cvl\ } ) Users from other INSAs can
  open a pull request on the repository with the assets for their INSA.
\end{itemize}

If you have any other feature request, open an issue on the repository.

\subsection{License}\label{license}

The typst template is licensed under the
\href{https://github.com/SkytAsul/INSA-Typst-Template/blob/main/LICENSE}{MIT
license} . This does \emph{not} apply to the image assets. Those image
files are property of Groupe INSA.

\subsection{Changelog}\label{changelog}

\subsubsection{0.4.0}\label{section}

\begin{itemize}
\tightlist
\item
  Added INSA CVL assets
\item
  Added \texttt{\ insa-tutor-suffix\ } option to \texttt{\ insa-stage\ }
\end{itemize}

\subsubsection{0.3.1}\label{section-1}

\begin{itemize}
\tightlist
\item
  Added \texttt{\ insa\ } option to all templates
\item
  Added INSA HdF assets
\item
  Added \texttt{\ student-suffix\ } option to \texttt{\ insa-stage\ }
\item
  Made outline not shown in outline
\end{itemize}

\subsubsection{0.3.0}\label{section-2}

\begin{itemize}
\tightlist
\item
  Added \texttt{\ omit-outline\ } option to \texttt{\ insa-stage\ }
\item
  Added \texttt{\ thanks-page\ } parameter to \texttt{\ insa-stage\ }
\item
  Added metadata-related options to \texttt{\ insa-document\ }
\item
  Made some PDF metadata automatically exported for
  \texttt{\ insa-stage\ } and \texttt{\ insa-report\ }
\item
  Made page number not displayed if equals to 0
\item
  Adjusted positions of elements in back covers
\item
  Fixed some translations
\item
  Updated README to have changelog, visual examples of all documents and
  parameters table
\end{itemize}

\href{/app?template=silky-report-insa&version=0.4.0}{Create project in
app}

\subsubsection{How to use}\label{how-to-use}

Click the button above to create a new project using this template in
the Typst app.

You can also use the Typst CLI to start a new project on your computer
using this command:

\begin{verbatim}
typst init @preview/silky-report-insa:0.4.0
\end{verbatim}

\includesvg[width=0.16667in,height=0.16667in]{/assets/icons/16-copy.svg}

\subsubsection{About}\label{about}

\begin{description}
\tightlist
\item[Author :]
SkytAsul
\item[License:]
MIT
\item[Current version:]
0.4.0
\item[Last updated:]
November 21, 2024
\item[First released:]
March 19, 2024
\item[Archive size:]
4.48 MB
\href{https://packages.typst.org/preview/silky-report-insa-0.4.0.tar.gz}{\pandocbounded{\includesvg[keepaspectratio]{/assets/icons/16-download.svg}}}
\item[Repository:]
\href{https://github.com/SkytAsul/INSA-Typst-Template}{GitHub}
\item[Discipline s :]
\begin{itemize}
\tightlist
\item[]
\item
  \href{https://typst.app/universe/search/?discipline=engineering}{Engineering}
\item
  \href{https://typst.app/universe/search/?discipline=computer-science}{Computer
  Science}
\item
  \href{https://typst.app/universe/search/?discipline=mathematics}{Mathematics}
\item
  \href{https://typst.app/universe/search/?discipline=physics}{Physics}
\item
  \href{https://typst.app/universe/search/?discipline=education}{Education}
\end{itemize}
\item[Categor y :]
\begin{itemize}
\tightlist
\item[]
\item
  \pandocbounded{\includesvg[keepaspectratio]{/assets/icons/16-speak.svg}}
  \href{https://typst.app/universe/search/?category=report}{Report}
\end{itemize}
\end{description}

\subsubsection{Where to report issues?}\label{where-to-report-issues}

This template is a project of SkytAsul . Report issues on
\href{https://github.com/SkytAsul/INSA-Typst-Template}{their repository}
. You can also try to ask for help with this template on the
\href{https://forum.typst.app}{Forum} .

Please report this template to the Typst team using the
\href{https://typst.app/contact}{contact form} if you believe it is a
safety hazard or infringes upon your rights.

\phantomsection\label{versions}
\subsubsection{Version history}\label{version-history}

\begin{longtable}[]{@{}ll@{}}
\toprule\noalign{}
Version & Release Date \\
\midrule\noalign{}
\endhead
\bottomrule\noalign{}
\endlastfoot
0.4.0 & November 21, 2024 \\
\href{https://typst.app/universe/package/silky-report-insa/0.3.1/}{0.3.1}
& September 24, 2024 \\
\href{https://typst.app/universe/package/silky-report-insa/0.3.0/}{0.3.0}
& August 7, 2024 \\
\href{https://typst.app/universe/package/silky-report-insa/0.2.1/}{0.2.1}
& July 24, 2024 \\
\href{https://typst.app/universe/package/silky-report-insa/0.2.0/}{0.2.0}
& June 10, 2024 \\
\href{https://typst.app/universe/package/silky-report-insa/0.1.0/}{0.1.0}
& March 19, 2024 \\
\end{longtable}

Typst GmbH did not create this template and cannot guarantee correct
functionality of this template or compatibility with any version of the
Typst compiler or app.
