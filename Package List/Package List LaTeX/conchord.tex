\title{typst.app/universe/package/conchord}

\phantomsection\label{banner}
\section{conchord}\label{conchord}

{ 0.2.0 }

Easily write lyrics with chords, generate chord diagrams and tabs.

{ } Featured Package

\phantomsection\label{readme}
\begin{quote}
Notice: I’m preparing the update, so the documentation there is
referring to the new version.
\end{quote}

\texttt{\ conchord\ } (concise chord) is a
\href{https://github.com/typst/typst}{Typst} package to write lyrics
with chords and generate colorful fretboard diagram (aka chord diagram).
From \texttt{\ 0.1.1\ } there is also experimental tabs support (though
quite simple and unstable yet). It is inspired by
\href{https://github.com/ljgago/typst-chords}{chordx} package and my
previous tiny project for generating chord diagrams SVG-s.

\texttt{\ conchord\ } makes it easy to add new chords, both for diagrams
and lyrics. Unlike \href{https://github.com/ljgago/typst-chords}{chordx}
, you don’t need to think about layout and pass lots of arrays for
drawing barres. Just pass a string with held frets and it will work:

\begin{Shaded}
\begin{Highlighting}[]
\NormalTok{\#import "@preview/conchord:0.2.0": new{-}chordgen, overchord}

\NormalTok{\#let chord = new{-}chordgen()}

\NormalTok{\#box(chord("x32010", name: "C"))}
\NormalTok{\#box(chord("x33222", name: "F\#m/C\#"))}
\NormalTok{\#box(chord("x,9,7,8,9,9"))}
\end{Highlighting}
\end{Shaded}

\pandocbounded{\includegraphics[keepaspectratio]{https://github.com/typst/packages/raw/main/packages/preview/conchord/0.2.0/examples/simple.png}}

\begin{quote}
\texttt{\ x\ } means closed string, \texttt{\ 0\ } is open, other number
is a fret. In case of frets larger than \texttt{\ 9\ } frets should be
separated with commas, otherwise you can list them without any
separators.
\end{quote}

\begin{quote}
Chord diagram works like a usual block, so to put them into one line you
need to wrap them into boxes. In real code it is recommended to create a
wrapper function to customize box margins etc (see larger example
below).
\end{quote}

It is easy to customize the colors and styles of chords with
\texttt{\ colors\ } argument and \texttt{\ show\ } rules for text. You
can also put \texttt{\ !\ } and \texttt{\ *\ } marks in the end of the
string to force diagram generation. \texttt{\ !\ } forces barre,
\texttt{\ *\ } removes it:

\begin{Shaded}
\begin{Highlighting}[]
\NormalTok{\#let custom{-}chord = new{-}chordgen(string{-}number: 3,}
\NormalTok{    colors: (shadow{-}barre: orange,}
\NormalTok{        grid: gray.darken(30\%),}
\NormalTok{        hold: red,}
\NormalTok{        barre: purple)}
\NormalTok{)}

\NormalTok{\#set text(fill: purple)}
\NormalTok{\#box(custom{-}chord("320", name: "C"))}
\NormalTok{\#box(custom{-}chord("2,4,4,*", name: "Bm"))}
\NormalTok{\#box(custom{-}chord("2,2,2, *"))}
\NormalTok{\#box(custom{-}chord("x,3,2, !"))}
\end{Highlighting}
\end{Shaded}

\pandocbounded{\includegraphics[keepaspectratio]{https://github.com/typst/packages/raw/main/packages/preview/conchord/0.2.0/examples/crazy.png}}

\begin{quote}
NOTE: be careful when using \textbf{!} , if barre cannot be used, it
will result into nonsense.
\end{quote}

For lyrics, you don’t need to add chord to word and specify the number
of char in words (unlike
\href{https://github.com/ljgago/typst-chords}{chordx} ). Simply add
\texttt{\ \#overchord\ } to the place you want a chord. Compose with
native Typst stylistic things for non-plain look (you don’t need to
dig into \href{https://github.com/ljgago/typst-chords}{chordx} ’s
custom arguments):

\begin{Shaded}
\begin{Highlighting}[]
\NormalTok{\#let och(it) = overchord(strong(it))}

\NormalTok{=== \#raw("[Verse 1]")}

\NormalTok{\#och[Em] Another head }
\NormalTok{\#och[C] hangs lowly \textbackslash{}}
\NormalTok{\#och[G] Child is slowly}
\NormalTok{\#och[D] taken}

\NormalTok{...}
\end{Highlighting}
\end{Shaded}

\begin{quote}
Complete example of lyrics with chords (see
\href{https://github.com/typst/packages/raw/main/packages/preview/conchord/0.2.0/examples/zombie.typ}{full
source} ):
\end{quote}

\pandocbounded{\includegraphics[keepaspectratio]{https://github.com/typst/packages/raw/main/packages/preview/conchord/0.2.0/examples/zombie.png}}

I was quite amazed with general idea of
\href{https://github.com/ljgago/typst-chords}{chordx} , but a bit
frustated with implementation, so I decided to quickly rewrite my old js
code to Typst. I use \texttt{\ cetz\ } there, so code is quite clean.

\begin{quote}
Note: This package doesn’t use any piece of
\href{https://github.com/ljgago/typst-chords}{chordx} , only the general
idea is taken.
\end{quote}

Brief comparison may be seen there, some concepts explained below:

\pandocbounded{\includegraphics[keepaspectratio]{https://github.com/typst/packages/raw/main/packages/preview/conchord/0.2.0/examples/compare.png}}

\subsection{Think about frets, not
layout}\label{think-about-frets-not-layout}

Write frets for chord as you hold it, like a string like “123456�
(see examples above). You don’t need to think about layouting and
subtracting frets, \texttt{\ conchord\ } does it for you.

\begin{quote}
NOTE: I can’t guarantee that will be the best chord layout. Moreover,
the logic is quite simple: e.g., barre can’t be multiple and can’t
be put anywhere except first bar in the image. However, surprisingly, it
works well in almost all of the common cases, so the exceptions are
really rare.
\end{quote}

If you need to create something too \emph{custom/complex} \st{(but not
\emph{concise} )} , maybe it is worth to try
\href{https://github.com/ljgago/typst-chords}{chordx} . You can also try
using core function \texttt{\ render-chord\ } for more manual~control,
but it is still limited by one barre starting from one (but that barre
may be shifted). If you think that feature should be supported, you can
create issue there.

\subsection{Shadow barre}\label{shadow-barre}

Some chord generators put barre only where it \emph{ought to} be (any
less will not hold some strings). Others put it where it can be
(sometimes maximal size, sometimes some other logic). I use simple barre
where it \textbf{ought to} be, and add \emph{shadow barre} where it
\textbf{could} maximally be. You can easily disable it by either setting
\texttt{\ use-shadow-barre:\ false\ } on \texttt{\ new-chordgen\ } (only
necessary part of barre rendered) or by setting color of
\texttt{\ shadow-barre\ } the same as \texttt{\ barre\ } (maximal
possible barre).

\subsection{Name auto-scaling}\label{name-auto-scaling}

Chord name font size is \emph{reduced} for \emph{large} chord names, so
the name fits well into chord diagram (see example above). That makes it
much more pretty to stack several chords together. To achieve
chordx-like behavior, you can always use
\texttt{\ \#figure(chord("…"),\ caption:\ …)\ } .

\subsection{Easier chords for lyrics}\label{easier-chords-for-lyrics}

Just add chord labels above lyrics in arbitrary place, don’t think
about what letter exactly it should be located. By default it aligns the
chord label to the left, so it produces pretty results out-of-box. You
can pass other alignments to \texttt{\ alignment\ } argument, or use the
chords straight inside words.

The command is \emph{much} simpler than chordx (of course, it is a
trade-off):

\begin{Shaded}
\begin{Highlighting}[]
\NormalTok{\#let overchord(body, align: start, height: 1em, width: {-}0.25em) = box(place(align, body), height: 1em + height, width: width)}
\end{Highlighting}
\end{Shaded}

Feel free to use it for your purposes outside of the package.

It takes on default \texttt{\ -0.25em\ } width to remove one adjacent
space, so

\begin{itemize}
\tightlist
\item
  To make it work on monospace/other special fonts, you will need to
  adjust \texttt{\ width\ } argument. The problem is that I can’t
  \texttt{\ measure\ } space, but maybe that will be eventually fixed.
\item
  To add chord inside word, you have to add \emph{one} space, like
  \texttt{\ wo\ \#chord{[}Am{]}rd\ } .
\end{itemize}

\subsection{Colors}\label{colors}

Customize the colors of chord elements. \texttt{\ new-chordgen\ }
accepts the \texttt{\ colors\ } dictionary with following possible
fields:

\begin{itemize}
\tightlist
\item
  \texttt{\ grid\ } : color of grid, default is
  \texttt{\ gray.darken(20\%)\ }
\item
  \texttt{\ open\ } : color of circles for open strings, default is
  \texttt{\ black\ }
\item
  \texttt{\ muted\ } : color of crosses for muted strings, default is
  \texttt{\ black\ }
\item
  \texttt{\ hold\ } : color of held positions, default is
  \texttt{\ \#5d6eaf\ }
\item
  \texttt{\ barre\ } : color of main barre part, default is
  \texttt{\ \#5d6eaf\ }
\item
  \texttt{\ shadow-barre\ } : color of “unnecessary� barre part,
  default is \texttt{\ \#5d6eaf.lighten(30\%)\ }
\end{itemize}

\subsubsection{Customizing text}\label{customizing-text}

\textbf{Important} : \emph{frets} are rendered using \texttt{\ raw\ }
elements. So if you want to customize their font or color, please use
\texttt{\ \#show\ raw:\ set\ text(fill:\ ...)\ } or similar things.

The chord’s name, on the other hand, uses default font, so to set it,
just use \texttt{\ \#set\ text(font:\ ...)\ } in the corresponding
scope.

\subsection{Assertions}\label{assertions}

Currently \href{https://github.com/ljgago/typst-chords}{chordx} has
almost no checks inside for correctness of passed chords.
\texttt{\ conchord\ } , on the other side, checks for

\begin{itemize}
\tightlist
\item
  Number of passed\&parsed frets equal to set string-number
\item
  Only numbers and \texttt{\ x\ } passed as frets
\item
  All frets fitting in the diagram
\end{itemize}

\begin{quote}
Everything there is highly experimental and unstable
\end{quote}

\pandocbounded{\includegraphics[keepaspectratio]{https://github.com/typst/packages/raw/main/packages/preview/conchord/0.2.0/examples/tabs.png}}

\begin{Shaded}
\begin{Highlighting}[]
\NormalTok{\#let chord = new{-}chordgen(scale{-}length: 0.6pt)}

\NormalTok{\#let ending(n) = \{}
\NormalTok{    rect(stroke: (left: black, top: black), inset: 0.2em, n + h(3em))}
\NormalTok{    v(0.5em)}
\NormalTok{\}}
\NormalTok{*This thing doesn\textquotesingle{}t follow musical notation rules, it is used just for demonstration purposes*:}

\NormalTok{\#tabs.new(\textasciigrave{}\textasciigrave{}\textasciigrave{}}
\NormalTok{2/4 2/4{-}3 2/4{-}2 2/4{-}3 |}
\NormalTok{2/4{-}2 2/4{-}3 2/4 2/4 2/4 |}
\NormalTok{2/4{-}2 p 0/2{-}3 3/2{-}2}
\NormalTok{|:}

\NormalTok{0/1+0/6 0/1 0/1{-}3 2/1 | 3/1+3/5{-}2 3/1 3/1{-}3 5/1 | 2/1+0/4{-}2 2/1 0/1{-}3 3/2{-}3 | \textbackslash{} \textbackslash{}}
\NormalTok{3/2{-}2 \textasciigrave{}5/2{-}3}
\NormalTok{p{-}2}
\NormalTok{\#\#}
\NormalTok{  chord("022000", name: "Em")}
\NormalTok{\#\#south}
\NormalTok{0/2{-}3 3/2 | | \#\# [...] \#\# p{-}4. | | 7/1{-}3 0/1{-}2 p{-}3 0/1 3/1 }

\NormalTok{\#\#}
\NormalTok{    ending[1.]}
\NormalTok{\#\#west}

\NormalTok{|}
\NormalTok{2/1{-}3}
\NormalTok{2/1}
\NormalTok{3/1 0/1 2/1{-}2 p{-}3 0/2{-}3 3/2{-}3}
\NormalTok{\#\#}
\NormalTok{  ending[2.]}
\NormalTok{\#\#west}
\NormalTok{|}
\NormalTok{2/1{-}2 2/1 0/1{-}3 3/2 :| 0/6{-}2 | \^{}0/6{-}2 || \textbackslash{}}
\NormalTok{1/1 2/1 2/2 2/2 2/3 2/3 4/4 4/4 4/4 4/4 4/4 4/4 2/3 2/3 2/3 2/3  2/3 2/3 2/3 2/3 2/3 2/3 2/3 2/3 2/3}
\NormalTok{\#\#}
\NormalTok{[notice there is no manual break]}
\NormalTok{\#\#east}
\NormalTok{| 2/3 2/3 8/3 7/3 6/3 5/3 4/3 2/3  5/3 8/3 9/3  7/3 2/3 | 2/3 2/2 2/3 2/4 |}
\NormalTok{10/1{-}3 10/1{-}3 10/1{-}3 10/1{-}4 10/1{-}4 10/1{-}4 10/1{-}4 10/1{-}5. 10/1{-}5. 10/1{-}5 10/1{-}5 10/1{-}2 \textbackslash{}}
\NormalTok{1/3bfullr+2/5{-}2 1/2b1/2{-}1 2/3v{-}1}
\NormalTok{\textasciigrave{}\textasciigrave{}\textasciigrave{}, eval{-}scope: (chord: chord, ending: ending)}
\NormalTok{ )}


\NormalTok{Not a lot customization is available yet, but something is already possible:}

\NormalTok{\#show raw: set text(red.darken(30\%), font: "Comic Sans MS")}

\NormalTok{\#tabs.new("0/1+2/5{-}1 \^{}0/1+\textasciigrave{}3/5{-}2.. 2/3 |: 2/3{-}1 2/3 2/3 | 3/3 ||",}
\NormalTok{  scale{-}length: 0.2cm,}
\NormalTok{  one{-}beat{-}length: 12,}
\NormalTok{  s{-}num: 5,}
\NormalTok{  colors: (}
\NormalTok{    lines: gradient.linear(yellow, blue),}
\NormalTok{    bars: green,}
\NormalTok{    connects: red}
\NormalTok{  ),}
\NormalTok{  enable{-}scale: false}
\NormalTok{)}
\end{Highlighting}
\end{Shaded}

As you can see from example, you can use raw strings or code blocks to
write tabs, there is no real difference.

The general idea is very simple: to write a number on some line, write
\texttt{\ \textless{}fret\ number\textgreater{}/\textless{}string\textgreater{}\ }
.

\textbf{Spaces are important!} All notes and special symbols work well
only if properly separated.

\subsubsection{Duration}\label{duration}

By default they will be quarter notes. To change that, you have to
specify the duration:
\texttt{\ \textless{}fret\textgreater{}/\textless{}string\textgreater{}-\textless{}duration\textgreater{}\ }
, where duration is \$log\_2\$ from note duration. So a whole note will
be \texttt{\ -0\ } , a half: \texttt{\ -1\ } and so on. You can also use
as many dots as you want to multiply duration by 1.5, e.g.
\texttt{\ -2.\ }

Once you change the duration, all the following notes will use it, so
you have to specify duration every time it is changed (basically,
always, but it really depends on composition). Of course, you can just
ignore all that duration staff.

\subsubsection{Bars and repetitions}\label{bars-and-repetitions}

To add simple bar, just add \texttt{\ \textbar{}\ } . To add double bar
line, use \texttt{\ \textbar{}\ \textbar{}\ } . To add end
movement/composition, add \texttt{\ \textbar{}\textbar{}\ } . To add
repetitions, use \texttt{\ \textbar{}:\ } and \texttt{\ :\textbar{}\ }
respectively.

\subsubsection{Linebreaks}\label{linebreaks}

Notes and bars that don’t fit in line will be automatically moved to
next. However, sometimes it isn’t ideal and may be a bit bugged, so it
is recommended to do that manually, using \texttt{\ \textbackslash{}\ }
.

The line is autoscaled if it is possible and not too ugly. You can
change the maximum and minimum scaling size with \texttt{\ scale-max\ }
and \texttt{\ scale-min\ } . It is also possible to completely disable
scaling with \texttt{\ enable-scale:\ false\ } .

\subsubsection{Ties and slides}\label{ties-and-slides}

You can \emph{tie} notes or \emph{slide} between them. To use ties, you
have to add \texttt{\ \^{}\ } in front of \emph{second} tied note, like
\texttt{\ 1/1\ \^{}3/1\ } . To use slides you have to do the same, but
with `.

\emph{Current limitation:} tying and sliding works only on the same
string and may work really bad if tied/slided through line break.

\subsection{Bends and vibratos}\label{bends-and-vibratos}

Add \texttt{\ b\ } after note, but before the duration (e.g.
\texttt{\ 2/3b-2\ } ) to add a bend. After \texttt{\ b\ } you can write
custom text to be written on top (for example, \texttt{\ b1/2\ } ). Add
\texttt{\ r\ } to the end to add a release.

Adding vibratos works the same way, via adding \texttt{\ v\ } to the
note. The length of vibrato will be the same as the length of the note.

Unfortunately, they are all supported things for now. But wait, there is
still one cool thing left!

\subsubsection{Custom content}\label{custom-content}

Add any typst code you want between \texttt{\ \#\#\ …\ \#\#\ } . It
will be rendered with \texttt{\ cetz\ } on top of the line where you
wrote it. That means you can write \emph{lyrics, chords, add complex
things like endings} , even \textbf{draw the elements that are still
missing} (well, it is still worth to create issue there, I will try to
do something).

That code is evaluated with \texttt{\ eval\ } , so you will need to pass
dictionary to \texttt{\ eval-scope\ } with all things you want to use.

You can set align of these elements by writing cetz anchors after the
second (e.g., \texttt{\ west\ } , \texttt{\ south\ } and their
combinations, like \texttt{\ west-south\ } ).

Additionally, if you enjoy drawing missing things, you can also use
\texttt{\ preamble\ } and \texttt{\ extra\ } arguments in
\texttt{\ tabs.new\ } where you can put any \texttt{\ cetz\ } inner
things (tabs uses canvas, and that allow you drawing on it) before or
after the tabs are drawn.

\subsubsection{Plans}\label{plans}

\begin{enumerate}
\tightlist
\item
  Add \emph{(optional)} “rhythm section� under tabs
\item
  Add more signs\&lines
\item
  Add more built-in things to attach above tabs
\end{enumerate}

It is far from what I want to do, so maybe there will be much more! I
will be very glad to receive \emph{any feedback} , both issues and PR-s
are very welcome (though I can’t promise I will be able to work on it
immediately)!

\subsubsection{How to add}\label{how-to-add}

Copy this into your project and use the import as \texttt{\ conchord\ }

\begin{verbatim}
#import "@preview/conchord:0.2.0"
\end{verbatim}

\includesvg[width=0.16667in,height=0.16667in]{/assets/icons/16-copy.svg}

Check the docs for
\href{https://typst.app/docs/reference/scripting/\#packages}{more
information on how to import packages} .

\subsubsection{About}\label{about}

\begin{description}
\tightlist
\item[Author :]
sitandr
\item[License:]
MIT
\item[Current version:]
0.2.0
\item[Last updated:]
February 6, 2024
\item[First released:]
July 24, 2023
\item[Minimum Typst version:]
0.8.0
\item[Archive size:]
12.8 kB
\href{https://packages.typst.org/preview/conchord-0.2.0.tar.gz}{\pandocbounded{\includesvg[keepaspectratio]{/assets/icons/16-download.svg}}}
\item[Repository:]
\href{https://github.com/sitandr/conchord}{GitHub}
\end{description}

\subsubsection{Where to report issues?}\label{where-to-report-issues}

This package is a project of sitandr . Report issues on
\href{https://github.com/sitandr/conchord}{their repository} . You can
also try to ask for help with this package on the
\href{https://forum.typst.app}{Forum} .

Please report this package to the Typst team using the
\href{https://typst.app/contact}{contact form} if you believe it is a
safety hazard or infringes upon your rights.

\phantomsection\label{versions}
\subsubsection{Version history}\label{version-history}

\begin{longtable}[]{@{}ll@{}}
\toprule\noalign{}
Version & Release Date \\
\midrule\noalign{}
\endhead
\bottomrule\noalign{}
\endlastfoot
0.2.0 & February 6, 2024 \\
\href{https://typst.app/universe/package/conchord/0.1.1/}{0.1.1} &
September 19, 2023 \\
\href{https://typst.app/universe/package/conchord/0.1.0/}{0.1.0} & July
24, 2023 \\
\end{longtable}

Typst GmbH did not create this package and cannot guarantee correct
functionality of this package or compatibility with any version of the
Typst compiler or app.
