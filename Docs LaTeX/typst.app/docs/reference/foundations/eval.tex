\title{typst.app/docs/reference/foundations/eval}

\begin{itemize}
\tightlist
\item
  \href{/docs}{\includesvg[width=0.16667in,height=0.16667in]{/assets/icons/16-docs-dark.svg}}
\item
  \includesvg[width=0.16667in,height=0.16667in]{/assets/icons/16-arrow-right.svg}
\item
  \href{/docs/reference/}{Reference}
\item
  \includesvg[width=0.16667in,height=0.16667in]{/assets/icons/16-arrow-right.svg}
\item
  \href{/docs/reference/foundations/}{Foundations}
\item
  \includesvg[width=0.16667in,height=0.16667in]{/assets/icons/16-arrow-right.svg}
\item
  \href{/docs/reference/foundations/eval/}{Evaluate}
\end{itemize}

\section{\texorpdfstring{\texttt{\ eval\ }}{ eval }}\label{summary}

Evaluates a string as Typst code.

This function should only be used as a last resort.

\subsection{Example}\label{example}

\begin{verbatim}
#eval("1 + 1") \
#eval("(1, 2, 3, 4)").len() \
#eval("*Markup!*", mode: "markup") \
\end{verbatim}

\includegraphics[width=5in,height=\textheight,keepaspectratio]{/assets/docs/KZfqDZ_7V1ElK4um94vvjwAAAAAAAAAA.png}

\subsection{\texorpdfstring{{ Parameters
}}{ Parameters }}\label{parameters}

\phantomsection\label{parameters-tooltip}
Parameters are the inputs to a function. They are specified in
parentheses after the function name.

{ eval } (

{ \href{/docs/reference/foundations/str/}{str} , } {
\hyperref[parameters-mode]{mode :}
\href{/docs/reference/foundations/str/}{str} , } {
\hyperref[parameters-scope]{scope :}
\href{/docs/reference/foundations/dictionary/}{dictionary} , }

) -\textgreater{} { any }

\subsubsection{\texorpdfstring{\texttt{\ source\ }}{ source }}\label{parameters-source}

\href{/docs/reference/foundations/str/}{str}

{Required} {{ Positional }}

\phantomsection\label{parameters-source-positional-tooltip}
Positional parameters are specified in order, without names.

A string of Typst code to evaluate.

\subsubsection{\texorpdfstring{\texttt{\ mode\ }}{ mode }}\label{parameters-mode}

\href{/docs/reference/foundations/str/}{str}

The \href{/docs/reference/syntax/\#modes}{syntactical mode} in which the
string is parsed.

\begin{longtable}[]{@{}ll@{}}
\toprule\noalign{}
Variant & Details \\
\midrule\noalign{}
\endhead
\bottomrule\noalign{}
\endlastfoot
\texttt{\ "\ code\ "\ } & Evaluate as code, as after a hash. \\
\texttt{\ "\ markup\ "\ } & Evaluate as markup, like in a Typst file. \\
\texttt{\ "\ math\ "\ } & Evaluate as math, as in an equation. \\
\end{longtable}

Default: \texttt{\ }{\texttt{\ "code"\ }}\texttt{\ }

\includesvg[width=0.16667in,height=0.16667in]{/assets/icons/16-arrow-right.svg}
View example

\begin{verbatim}
#eval("= Heading", mode: "markup")
#eval("1_2^3", mode: "math")
\end{verbatim}

\includegraphics[width=5in,height=\textheight,keepaspectratio]{/assets/docs/4OYmfbro6ZT1td5j4R5wyAAAAAAAAAAA.png}

\subsubsection{\texorpdfstring{\texttt{\ scope\ }}{ scope }}\label{parameters-scope}

\href{/docs/reference/foundations/dictionary/}{dictionary}

A scope of definitions that are made available.

Default:
\texttt{\ }{\texttt{\ (\ }}\texttt{\ }{\texttt{\ :\ }}\texttt{\ }{\texttt{\ )\ }}\texttt{\ }

\includesvg[width=0.16667in,height=0.16667in]{/assets/icons/16-arrow-right.svg}
View example

\begin{verbatim}
#eval("x + 1", scope: (x: 2)) \
#eval(
  "abc/xyz",
  mode: "math",
  scope: (
    abc: $a + b + c$,
    xyz: $x + y + z$,
  ),
)
\end{verbatim}

\includegraphics[width=5in,height=\textheight,keepaspectratio]{/assets/docs/0vD-OzSZwxX0Gqmm8_Sk9AAAAAAAAAAA.png}

\href{/docs/reference/foundations/duration/}{\pandocbounded{\includesvg[keepaspectratio]{/assets/icons/16-arrow-right.svg}}}

{ Duration } { Previous page }

\href{/docs/reference/foundations/float/}{\pandocbounded{\includesvg[keepaspectratio]{/assets/icons/16-arrow-right.svg}}}

{ Float } { Next page }
