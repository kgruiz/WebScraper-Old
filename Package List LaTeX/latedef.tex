\title{typst.app/universe/package/latedef}

\phantomsection\label{banner}
\section{latedef}\label{latedef}

{ 0.1.0 }

Use now, define later

\phantomsection\label{readme}
\emph{Use now, define later!}

\subsection{Basic usage}\label{basic-usage}

This package exposes a single function, \texttt{\ latedef-setup\ } .

\begin{Shaded}
\begin{Highlighting}[]
\NormalTok{\#let (undef, def) = latedef{-}setup(simple: true)}

\NormalTok{My \#undef is \#undef.}
\NormalTok{\#def("dog")}
\NormalTok{\#def("cool")}
\end{Highlighting}
\end{Shaded}

\pandocbounded{\includegraphics[keepaspectratio]{https://github.com/typst/packages/raw/main/packages/preview/latedef/0.1.0/example-images/1.png}}

Note that the definition doesn’t actually have to come \emph{after}
the usage, but if you want to define something beforehand, you’re
better off using a variable instead.

\begin{Shaded}
\begin{Highlighting}[]
\NormalTok{\#let (undef, def) = latedef{-}setup(simple: true)}

\NormalTok{// Instead of}
\NormalTok{\#def("A")}
\NormalTok{The first letter is \#undef.}

\NormalTok{// you should use}
\NormalTok{\#let A = "A"}
\NormalTok{The first letter is \#A.}
\end{Highlighting}
\end{Shaded}

\pandocbounded{\includegraphics[keepaspectratio]{https://github.com/typst/packages/raw/main/packages/preview/latedef/0.1.0/example-images/2.png}}

\subsection{\texorpdfstring{The \texttt{\ simple\ }
parameter}{The  simple  parameter}}\label{the-simple-parameter}

When \texttt{\ simple:\ false\ } (which is the default),
\texttt{\ undef\ } becomes a function you have to call. It takes an
optional positional or named parameter \texttt{\ id\ } of type
\texttt{\ str\ } , which can be used to define things out of order.

\begin{Shaded}
\begin{Highlighting}[]
\NormalTok{\#let (undef, def) = latedef{-}setup() // or \textasciigrave{}latedef{-}setup(simple: false)\textasciigrave{}}

\NormalTok{// Note that you can still call it without an id, which works just like when \textasciigrave{}simple: true\textasciigrave{}.}
\NormalTok{My letters are \#undef("1"), \#undef(id: "2"), and \#undef().}

\NormalTok{// \textasciigrave{}def\textasciigrave{} now takes one positional and either another positional or a named parameter.}
\NormalTok{\#def("C")}
\NormalTok{\#def(id: "2", "B")}
\NormalTok{\#def("1", "A")}
\end{Highlighting}
\end{Shaded}

\pandocbounded{\includegraphics[keepaspectratio]{https://github.com/typst/packages/raw/main/packages/preview/latedef/0.1.0/example-images/3.png}}

\subsection{\texorpdfstring{The \texttt{\ footnote\ }
parameter}{The  footnote  parameter}}\label{the-footnote-parameter}

This is a convenience feature that automatically wraps
\texttt{\ undef\ } in \texttt{\ footnote\ } , either directly (when
\texttt{\ simple:\ true\ } ) or as a function (when
\texttt{\ simple:\ false\ } ).

This corresponds to LaTeX’s \texttt{\ \textbackslash{}footnotemark\ }
and \texttt{\ \textbackslash{}footnotetext\ } , hence the different
names in the example.

\begin{Shaded}
\begin{Highlighting}[]
\NormalTok{\#let (fmark, ftext) = latedef{-}setup(simple: true, footnote: true)}
\NormalTok{Do\#fmark you\#fmark believe\#fmark in God?\#fmark}

\NormalTok{\#let wdym = "What do you mean"}
\NormalTok{\#ftext[\#wdym "Do"?]}
\NormalTok{\#ftext[\#wdym "you"?]}
\NormalTok{\#ftext[\#wdym "believe"?]}
\NormalTok{\#ftext[And w\#wdym.slice(1) "God"?]}
\end{Highlighting}
\end{Shaded}

\pandocbounded{\includegraphics[keepaspectratio]{https://github.com/typst/packages/raw/main/packages/preview/latedef/0.1.0/example-images/4.png}}

\subsection{\texorpdfstring{The \texttt{\ stand-in\ }
parameter}{The  stand-in  parameter}}\label{the-stand-in-parameter}

This is a function that takes a single positional parameter (
\texttt{\ id\ } ) of type \texttt{\ none\ \textbar{}\ str\ } and
produces a stand-in value that gets shown when a late-defined value is
missing a corresponding definition.

\begin{Shaded}
\begin{Highlighting}[]
\NormalTok{\#let (undef, def) = latedef{-}setup()}
\NormalTok{// This is the default stand{-}in}
\NormalTok{\#undef()}
\NormalTok{\#undef("with an id")}

\NormalTok{// Custom stand{-}in}
\NormalTok{\#let (undef, def) = latedef{-}setup(stand{-}in: id =\textgreater{} emph[No \#id!])}
\NormalTok{\#undef()}
\NormalTok{\#undef("id")}
\end{Highlighting}
\end{Shaded}

\pandocbounded{\includegraphics[keepaspectratio]{https://github.com/typst/packages/raw/main/packages/preview/latedef/0.1.0/example-images/5.png}}

Since \texttt{\ stand-in\ } is a function, which is only called when a
definition is actually missing, you can even set it to panic to enforce
that all late-defined values have a definiton.

\begin{Shaded}
\begin{Highlighting}[]
\NormalTok{\#let (undef, def) = latedef{-}setup(stand{-}in: id =\textgreater{} panic("Missing definition for value with id " + repr(id)))}
\NormalTok{\#undef()}
\NormalTok{\#undef("id")}
\end{Highlighting}
\end{Shaded}

The output will look something like

\begin{verbatim}
error: panicked with: "Missing definition for value with id none"
  ┌─ example.typ:1:50
  │
  │ #let (undef, def) = latedef-setup(stand-in: id => panic("Missing definition for value with id " + repr(id)))
  │                                                   ^^^^^^^^^^^^^^^^^^^^^^^^^^^^^^^^^^^^^^^^^^^^^^^^^^^^^^^^^

error: panicked with: "Missing definition for value with id \"id\""
  ┌─ example.typ:1:50
  │
  │ #let (undef, def) = latedef-setup(stand-in: id => panic("Missing definition for value with id " + repr(id)))
  │                                                   ^^^^^^^^^^^^^^^^^^^^^^^^^^^^^^^^^^^^^^^^^^^^^^^^^^^^^^^^^
\end{verbatim}

And there is no error when everything has a definition:

\begin{Shaded}
\begin{Highlighting}[]
\NormalTok{\#let (undef, def) = latedef{-}setup(stand{-}in: id =\textgreater{} panic("Missing definition for value with id " + repr(id)))}
\NormalTok{\#undef() is \#undef("id").}
\NormalTok{\#def("This")}
\NormalTok{\#def("id", "fine")}
\end{Highlighting}
\end{Shaded}

\pandocbounded{\includegraphics[keepaspectratio]{https://github.com/typst/packages/raw/main/packages/preview/latedef/0.1.0/example-images/6.png}}

\subsection{\texorpdfstring{The \texttt{\ group\ }
parameter}{The  group  parameter}}\label{the-group-parameter}

Sometimes you may want to use multiple instances of \texttt{\ latedef\ }
in parallel. This is done using the \texttt{\ group\ } parameter, which
can be \texttt{\ none\ } (the default) or any \texttt{\ str\ } .

Note that using \texttt{\ footnote:\ true\ } sets the default group to
\texttt{\ "footnote"\ } instead.

\begin{Shaded}
\begin{Highlighting}[]
\NormalTok{// Use a group for the figure stuff...}
\NormalTok{\#let (caption{-}undef, caption) = latedef{-}setup(simple: true, group: "figure")}
\NormalTok{\#let figure = std.figure.with(caption: caption{-}undef)}
\NormalTok{// ...so you can still use the regular mechanism in parallel.}
\NormalTok{\#let (undef, def) = latedef{-}setup(simple: true)}

\NormalTok{\#figure(raw(block: true, lorem(5)))}
\NormalTok{\#caption[The \#undef \_lorem ipsum\_.]}
\NormalTok{\#def("classic")}
\end{Highlighting}
\end{Shaded}

\pandocbounded{\includegraphics[keepaspectratio]{https://github.com/typst/packages/raw/main/packages/preview/latedef/0.1.0/example-images/7.png}}

\subsubsection{How to add}\label{how-to-add}

Copy this into your project and use the import as \texttt{\ latedef\ }

\begin{verbatim}
#import "@preview/latedef:0.1.0"
\end{verbatim}

\includesvg[width=0.16667in,height=0.16667in]{/assets/icons/16-copy.svg}

Check the docs for
\href{https://typst.app/docs/reference/scripting/\#packages}{more
information on how to import packages} .

\subsubsection{About}\label{about}

\begin{description}
\tightlist
\item[Author :]
\href{mailto:realt0mstone@gmail.com}{T0mstone}
\item[License:]
MIT-0
\item[Current version:]
0.1.0
\item[Last updated:]
October 21, 2024
\item[First released:]
October 21, 2024
\item[Minimum Typst version:]
0.11.0
\item[Archive size:]
3.64 kB
\href{https://packages.typst.org/preview/latedef-0.1.0.tar.gz}{\pandocbounded{\includesvg[keepaspectratio]{/assets/icons/16-download.svg}}}
\item[Repository:]
\href{https://codeberg.org/T0mstone/typst-latedef}{Codeberg}
\end{description}

\subsubsection{Where to report issues?}\label{where-to-report-issues}

This package is a project of T0mstone . Report issues on
\href{https://codeberg.org/T0mstone/typst-latedef}{their repository} .
You can also try to ask for help with this package on the
\href{https://forum.typst.app}{Forum} .

Please report this package to the Typst team using the
\href{https://typst.app/contact}{contact form} if you believe it is a
safety hazard or infringes upon your rights.

\phantomsection\label{versions}
\subsubsection{Version history}\label{version-history}

\begin{longtable}[]{@{}ll@{}}
\toprule\noalign{}
Version & Release Date \\
\midrule\noalign{}
\endhead
\bottomrule\noalign{}
\endlastfoot
0.1.0 & October 21, 2024 \\
\end{longtable}

Typst GmbH did not create this package and cannot guarantee correct
functionality of this package or compatibility with any version of the
Typst compiler or app.
