\title{typst.app/universe/package/typearea}

\phantomsection\label{banner}
\section{typearea}\label{typearea}

{ 0.2.0 }

A KOMA-Script inspired package to better configure your typearea and
margins.

\phantomsection\label{readme}
A KOMA-Script inspired package to better configure your typearea and
margins.

\begin{Shaded}
\begin{Highlighting}[]
\NormalTok{\#import "@preview/typearea:0.2.0": typearea}

\NormalTok{\#show: typearea.with(}
\NormalTok{  paper: "a4",}
\NormalTok{  div: 9,}
\NormalTok{  binding{-}correction: 11mm,}
\NormalTok{)}

\NormalTok{= Hello World}
\end{Highlighting}
\end{Shaded}

\subsection{Reference}\label{reference}

\texttt{\ typearea\ } accepts the following options:

\subsubsection{two-sided}\label{two-sided}

Whether the document is two-sided. Defaults to \texttt{\ true\ } .

\subsubsection{binding-correction}\label{binding-correction}

Binding correction. Defaults to \texttt{\ 0pt\ } .

Additional margin on the inside of a page when two-sided is true. If
two-sided is false it will be on the left or right side, depending on
the value of \texttt{\ binding\ } . A \texttt{\ binding\ } value of
\texttt{\ auto\ } will currently default to \texttt{\ left\ } .

Note: Before version 0.2.0 this was called \texttt{\ bcor\ } .

\subsubsection{div}\label{div}

How many equal parts to split the page into. Controls the margins.
Defautls to \texttt{\ 9\ } .

The top and bottom margin will always be one and two parts respectively.
In two-sided mode the inside margin will be one part and the outside
margin two parts, so the combined margins between the text on the left
side and the text on the right side is the same as the margins from the
outer edge of the text to the outer edge of the page.

In one-sided mode the left and right margin will take 1.5 parts each.

\subsubsection{header-height /
footer-height}\label{header-height-footer-height}

The height of the page header/footer.

\subsubsection{header-sep / footer-sep}\label{header-sep-footer-sep}

The distance between the page header/footer and the text area.

\subsubsection{header-include /
footer-include}\label{header-include-footer-include}

Whether the header/footer should be counted as part of the text area
when calculating the margins. Defaults to \texttt{\ false\ } .

\subsubsection{…rest}\label{uxe2rest}

All other arguments are passed on to \texttt{\ page()\ } as is. You can
see which arguments \texttt{\ page()\ } accepts in the
\href{https://typst.app/docs/reference/layout/page/}{typst reference for
the page function} .

You should prefer this over calling or setting \texttt{\ page()\ }
directly as doing so could break some of \texttt{\ typearea\ } ’s
functionality.

\subsubsection{How to add}\label{how-to-add}

Copy this into your project and use the import as \texttt{\ typearea\ }

\begin{verbatim}
#import "@preview/typearea:0.2.0"
\end{verbatim}

\includesvg[width=0.16667in,height=0.16667in]{/assets/icons/16-copy.svg}

Check the docs for
\href{https://typst.app/docs/reference/scripting/\#packages}{more
information on how to import packages} .

\subsubsection{About}\label{about}

\begin{description}
\tightlist
\item[Author :]
Adrian Freund
\item[License:]
MIT
\item[Current version:]
0.2.0
\item[Last updated:]
June 13, 2024
\item[First released:]
October 29, 2023
\item[Archive size:]
2.39 kB
\href{https://packages.typst.org/preview/typearea-0.2.0.tar.gz}{\pandocbounded{\includesvg[keepaspectratio]{/assets/icons/16-download.svg}}}
\item[Repository:]
\href{https://github.com/freundTech/typst-typearea}{GitHub}
\end{description}

\subsubsection{Where to report issues?}\label{where-to-report-issues}

This package is a project of Adrian Freund . Report issues on
\href{https://github.com/freundTech/typst-typearea}{their repository} .
You can also try to ask for help with this package on the
\href{https://forum.typst.app}{Forum} .

Please report this package to the Typst team using the
\href{https://typst.app/contact}{contact form} if you believe it is a
safety hazard or infringes upon your rights.

\phantomsection\label{versions}
\subsubsection{Version history}\label{version-history}

\begin{longtable}[]{@{}ll@{}}
\toprule\noalign{}
Version & Release Date \\
\midrule\noalign{}
\endhead
\bottomrule\noalign{}
\endlastfoot
0.2.0 & June 13, 2024 \\
\href{https://typst.app/universe/package/typearea/0.1.0/}{0.1.0} &
October 29, 2023 \\
\end{longtable}

Typst GmbH did not create this package and cannot guarantee correct
functionality of this package or compatibility with any version of the
Typst compiler or app.
