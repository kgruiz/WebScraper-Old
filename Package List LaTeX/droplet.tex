\title{typst.app/universe/package/droplet}

\phantomsection\label{banner}
\section{droplet}\label{droplet}

{ 0.3.1 }

Customizable dropped capitals.

\phantomsection\label{readme}
A package for creating dropped capitals.

\subsection{Usage}\label{usage}

This package exports a single \texttt{\ dropcap\ } function that is used
to create dropped capitals. The function takes one or two positional
arguments, and several optional keyword arguments for customization:

\begin{longtable}[]{@{}llll@{}}
\toprule\noalign{}
Parameter & Type & Description & Default \\
\midrule\noalign{}
\endhead
\bottomrule\noalign{}
\endlastfoot
\texttt{\ height\ } & \texttt{\ integer\ } , \texttt{\ length\ } ,
\texttt{\ auto\ } & The height of the dropped capital. &
\texttt{\ 2\ } \\
\texttt{\ justify\ } & \texttt{\ boolean\ } , \texttt{\ auto\ } &
Whether the text should be justified. & \texttt{\ auto\ } \\
\texttt{\ gap\ } & \texttt{\ length\ } & The space between the dropped
capital and the text. & \texttt{\ 0pt\ } \\
\texttt{\ hanging-indent\ } & \texttt{\ length\ } & The indent of lines
after the first. & \texttt{\ 0pt\ } \\
\texttt{\ overhang\ } & \texttt{\ length\ } , \texttt{\ relative\ } ,
\texttt{\ ratio\ } & How much the dropped capital should hang into the
margin. & \texttt{\ 0pt\ } \\
\texttt{\ depth\ } & \texttt{\ integer\ } , \texttt{\ length\ } & The
space below the dropped capital. & \texttt{\ 0pt\ } \\
\texttt{\ transform\ } & \texttt{\ function\ } , \texttt{\ none\ } & A
function to be applied to the dropped capital. & \texttt{\ none\ } \\
\texttt{\ ..text-args\ } & & How to style the \texttt{\ text\ } of the
dropped capital. & \\
\end{longtable}

Some parameters allow values of different types for maximum flexibility:

\begin{itemize}
\tightlist
\item
  If \texttt{\ height\ } is given as an integer, it is interpreted as a
  number of lines. If given as \texttt{\ auto\ } , the dropped capital
  will not be scaled and remain at its original size.
\item
  If \texttt{\ overhang\ } has a relative part, it is resolved relative
  to the width of the dropped capital.
\item
  If \texttt{\ depth\ } is given as an integer, it is interpreted as a
  number of lines.
\item
  The \texttt{\ transform\ } function takes the extracted or passed
  dropped capital and returns the content to be shown.
\end{itemize}

If two positional arguments are given, the first is used as the dropped
capital, and the second as the text. If only one argument is given, the
dropped capital is automatically extracted from the text.

\subsubsection{Extraction}\label{extraction}

If no explicit dropped capital is passed, it is extracted automatically.
For this to work, the package looks into the content making up the first
paragraph and extracts the first letter of the first word. This letter
is then split off from the rest of the text and used as the dropped
capital. There are some special cases to consider:

\begin{itemize}
\tightlist
\item
  If the first element of the paragraph is a \texttt{\ box\ } , the
  whole box is used as the dropped capital.
\item
  If the first element is a list or enum item, it is assumed that the
  literal meaning of the list or enum syntax was intended, and the
  number or bullet is used as the dropped capital.
\end{itemize}

Affixes, such as punctuation, super- and subscripts, quotes, and spaces
will also be detected and stay with the dropped capital.

\subsubsection{Paragraph Splitting}\label{paragraph-splitting}

To wrap the text around the dropped capital, the paragraph is split into
two parts: the part next to the dropped capital and the part after it.
As Typst doesn’t natively support wrapping text around an element,
this package splits the paragraph at word boundaries and tries to fit as
much in the first part as possible. This approach comes with some
limitations:

\begin{itemize}
\tightlist
\item
  The paragraph is split at word boundaries, which makes hyphenation
  across the split impossible.
\item
  Some elements cannot be properly split, such as containers, lists, and
  context expressions.
\item
  The approach uses a greedy algorithm, which might not always find the
  optimal split.
\item
  If the split happens at a block element, the spacing between the two
  parts might be off.
\end{itemize}

To determine whether an elements fits into the first part, the position
of top edge of the element is crucial. If the top edge is above the
baseline of the dropped capital, the element is considered to be part of
the first part. This means that elements with a large height will be
part of the first part. This is done to avoid gaps between the two parts
of the paragraph.

\subsubsection{Styling}\label{styling}

In case you wish to style the dropped capital more than what is possible
with the arguments of the \texttt{\ text\ } function, you can use a
\texttt{\ transform\ } function. This function takes the extracted or
passed dropped capital and returns the content to be shown. The function
is provided with the context of the dropped capital.

Note that when using \texttt{\ em\ } units, they are resolved relative
to the font size of the dropped capital. When the dropped capital is
scaled to fit the given \texttt{\ height\ } parameter, the font size is
adjusted so that the \emph{bounds} of the transformed content match the
given height. For that, the \texttt{\ top-edge\ } and
\texttt{\ bottom-edge\ } parameters of \texttt{\ text-args\ } are set to
\texttt{\ bounds\ } by default.

\subsection{Example}\label{example}

\begin{Shaded}
\begin{Highlighting}[]
\NormalTok{\#import "@preview/droplet:0.3.1": dropcap}

\NormalTok{\#set par(justify: true)}

\NormalTok{\#dropcap(}
\NormalTok{  height: 3,}
\NormalTok{  gap: 4pt,}
\NormalTok{  hanging{-}indent: 1em,}
\NormalTok{  overhang: 8pt,}
\NormalTok{  font: "Curlz MT",}
\NormalTok{)[}
\NormalTok{  *Typst* is a new markup{-}based typesetting system that is designed to be as}
\NormalTok{  \_powerful\_ as LaTeX while being \_much easier\_ to learn and use. Typst has:}

\NormalTok{  {-} Built{-}in markup for the most common formatting tasks}
\NormalTok{  {-} Flexible functions for everything else}
\NormalTok{  {-} A tightly integrated scripting system}
\NormalTok{  {-} Math typesetting, bibliography management, and more}
\NormalTok{  {-} Fast compile times thanks to incremental compilation}
\NormalTok{  {-} Friendly error messages in case something goes wrong}
\NormalTok{]}
\end{Highlighting}
\end{Shaded}

\pandocbounded{\includesvg[keepaspectratio]{https://github.com/typst/packages/raw/main/packages/preview/droplet/0.3.1/assets/example.svg}}

\subsubsection{How to add}\label{how-to-add}

Copy this into your project and use the import as \texttt{\ droplet\ }

\begin{verbatim}
#import "@preview/droplet:0.3.1"
\end{verbatim}

\includesvg[width=0.16667in,height=0.16667in]{/assets/icons/16-copy.svg}

Check the docs for
\href{https://typst.app/docs/reference/scripting/\#packages}{more
information on how to import packages} .

\subsubsection{About}\label{about}

\begin{description}
\tightlist
\item[Author :]
Eric Biedert
\item[License:]
MIT
\item[Current version:]
0.3.1
\item[Last updated:]
November 21, 2024
\item[First released:]
July 5, 2024
\item[Minimum Typst version:]
0.11.0
\item[Archive size:]
7.82 kB
\href{https://packages.typst.org/preview/droplet-0.3.1.tar.gz}{\pandocbounded{\includesvg[keepaspectratio]{/assets/icons/16-download.svg}}}
\item[Repository:]
\href{https://github.com/EpicEricEE/typst-droplet}{GitHub}
\item[Categor y :]
\begin{itemize}
\tightlist
\item[]
\item
  \pandocbounded{\includesvg[keepaspectratio]{/assets/icons/16-text.svg}}
  \href{https://typst.app/universe/search/?category=text}{Text}
\end{itemize}
\end{description}

\subsubsection{Where to report issues?}\label{where-to-report-issues}

This package is a project of Eric Biedert . Report issues on
\href{https://github.com/EpicEricEE/typst-droplet}{their repository} .
You can also try to ask for help with this package on the
\href{https://forum.typst.app}{Forum} .

Please report this package to the Typst team using the
\href{https://typst.app/contact}{contact form} if you believe it is a
safety hazard or infringes upon your rights.

\phantomsection\label{versions}
\subsubsection{Version history}\label{version-history}

\begin{longtable}[]{@{}ll@{}}
\toprule\noalign{}
Version & Release Date \\
\midrule\noalign{}
\endhead
\bottomrule\noalign{}
\endlastfoot
0.3.1 & November 21, 2024 \\
\href{https://typst.app/universe/package/droplet/0.3.0/}{0.3.0} &
October 24, 2024 \\
\href{https://typst.app/universe/package/droplet/0.2.0/}{0.2.0} & July
5, 2024 \\
\href{https://typst.app/universe/package/droplet/0.1.0/}{0.1.0} & July
5, 2024 \\
\end{longtable}

Typst GmbH did not create this package and cannot guarantee correct
functionality of this package or compatibility with any version of the
Typst compiler or app.
