\title{typst.app/docs/reference/model/par}

\begin{itemize}
\tightlist
\item
  \href{/docs}{\includesvg[width=0.16667in,height=0.16667in]{/assets/icons/16-docs-dark.svg}}
\item
  \includesvg[width=0.16667in,height=0.16667in]{/assets/icons/16-arrow-right.svg}
\item
  \href{/docs/reference/}{Reference}
\item
  \includesvg[width=0.16667in,height=0.16667in]{/assets/icons/16-arrow-right.svg}
\item
  \href{/docs/reference/model/}{Model}
\item
  \includesvg[width=0.16667in,height=0.16667in]{/assets/icons/16-arrow-right.svg}
\item
  \href{/docs/reference/model/par/}{Paragraph}
\end{itemize}

\section{\texorpdfstring{\texttt{\ par\ } {{ Element
}}}{ par   Element }}\label{summary}

\phantomsection\label{element-tooltip}
Element functions can be customized with \texttt{\ set\ } and
\texttt{\ show\ } rules.

Arranges text, spacing and inline-level elements into a paragraph.

Although this function is primarily used in set rules to affect
paragraph properties, it can also be used to explicitly render its
argument onto a paragraph of its own.

\subsection{Example}\label{example}

\begin{verbatim}
#set par(
  first-line-indent: 1em,
  spacing: 0.65em,
  justify: true,
)

We proceed by contradiction.
Suppose that there exists a set
of positive integers $a$, $b$, and
$c$ that satisfies the equation
$a^n + b^n = c^n$ for some
integer value of $n > 2$.

Without loss of generality,
let $a$ be the smallest of the
three integers. Then, we ...
\end{verbatim}

\includegraphics[width=5in,height=\textheight,keepaspectratio]{/assets/docs/yrIipb0QYzuDEgQNZF57rwAAAAAAAAAA.png}

\subsection{\texorpdfstring{{ Parameters
}}{ Parameters }}\label{parameters}

\phantomsection\label{parameters-tooltip}
Parameters are the inputs to a function. They are specified in
parentheses after the function name.

{ par } (

{ \hyperref[parameters-leading]{leading :}
\href{/docs/reference/layout/length/}{length} , } {
\hyperref[parameters-spacing]{spacing :}
\href{/docs/reference/layout/length/}{length} , } {
\hyperref[parameters-justify]{justify :}
\href{/docs/reference/foundations/bool/}{bool} , } {
\hyperref[parameters-linebreaks]{linebreaks :}
\href{/docs/reference/foundations/auto/}{auto}
\href{/docs/reference/foundations/str/}{str} , } {
\hyperref[parameters-first-line-indent]{first-line-indent :}
\href{/docs/reference/layout/length/}{length} , } {
\hyperref[parameters-hanging-indent]{hanging-indent :}
\href{/docs/reference/layout/length/}{length} , } {
\href{/docs/reference/foundations/content/}{content} , }

) -\textgreater{} \href{/docs/reference/foundations/content/}{content}

\subsubsection{\texorpdfstring{\texttt{\ leading\ }}{ leading }}\label{parameters-leading}

\href{/docs/reference/layout/length/}{length}

{{ Settable }}

\phantomsection\label{parameters-leading-settable-tooltip}
Settable parameters can be customized for all following uses of the
function with a \texttt{\ set\ } rule.

The spacing between lines.

Leading defines the spacing between the
\href{/docs/reference/text/text/\#parameters-bottom-edge}{bottom edge}
of one line and the
\href{/docs/reference/text/text/\#parameters-top-edge}{top edge} of the
following line. By default, these two properties are up to the font, but
they can also be configured manually with a text set rule.

By setting top edge, bottom edge, and leading, you can also configure a
consistent baseline-to-baseline distance. You could, for instance, set
the leading to \texttt{\ }{\texttt{\ 1em\ }}\texttt{\ } , the top-edge
to \texttt{\ }{\texttt{\ 0.8em\ }}\texttt{\ } , and the bottom-edge to
\texttt{\ }{\texttt{\ -\ }}\texttt{\ }{\texttt{\ 0.2em\ }}\texttt{\ } to
get a baseline gap of exactly \texttt{\ }{\texttt{\ 2em\ }}\texttt{\ } .
The exact distribution of the top- and bottom-edge values affects the
bounds of the first and last line.

Default: \texttt{\ }{\texttt{\ 0.65em\ }}\texttt{\ }

\subsubsection{\texorpdfstring{\texttt{\ spacing\ }}{ spacing }}\label{parameters-spacing}

\href{/docs/reference/layout/length/}{length}

{{ Settable }}

\phantomsection\label{parameters-spacing-settable-tooltip}
Settable parameters can be customized for all following uses of the
function with a \texttt{\ set\ } rule.

The spacing between paragraphs.

Just like leading, this defines the spacing between the bottom edge of a
paragraph\textquotesingle s last line and the top edge of the next
paragraph\textquotesingle s first line.

When a paragraph is adjacent to a
\href{/docs/reference/layout/block/}{\texttt{\ block\ }} that is not a
paragraph, that block\textquotesingle s
\href{/docs/reference/layout/block/\#parameters-above}{\texttt{\ above\ }}
or
\href{/docs/reference/layout/block/\#parameters-below}{\texttt{\ below\ }}
property takes precedence over the paragraph spacing. Headings, for
instance, reduce the spacing below them by default for a better look.

Default: \texttt{\ }{\texttt{\ 1.2em\ }}\texttt{\ }

\subsubsection{\texorpdfstring{\texttt{\ justify\ }}{ justify }}\label{parameters-justify}

\href{/docs/reference/foundations/bool/}{bool}

{{ Settable }}

\phantomsection\label{parameters-justify-settable-tooltip}
Settable parameters can be customized for all following uses of the
function with a \texttt{\ set\ } rule.

Whether to justify text in its line.

Hyphenation will be enabled for justified paragraphs if the
\href{/docs/reference/text/text/\#parameters-hyphenate}{text
function\textquotesingle s \texttt{\ hyphenate\ } property} is set to
\texttt{\ }{\texttt{\ auto\ }}\texttt{\ } and the current language is
known.

Note that the current
\href{/docs/reference/layout/align/\#parameters-alignment}{alignment}
still has an effect on the placement of the last line except if it ends
with a
\href{/docs/reference/text/linebreak/\#parameters-justify}{justified
line break} .

Default: \texttt{\ }{\texttt{\ false\ }}\texttt{\ }

\subsubsection{\texorpdfstring{\texttt{\ linebreaks\ }}{ linebreaks }}\label{parameters-linebreaks}

\href{/docs/reference/foundations/auto/}{auto} {or}
\href{/docs/reference/foundations/str/}{str}

{{ Settable }}

\phantomsection\label{parameters-linebreaks-settable-tooltip}
Settable parameters can be customized for all following uses of the
function with a \texttt{\ set\ } rule.

How to determine line breaks.

When this property is set to \texttt{\ }{\texttt{\ auto\ }}\texttt{\ } ,
its default value, optimized line breaks will be used for justified
paragraphs. Enabling optimized line breaks for ragged paragraphs may
also be worthwhile to improve the appearance of the text.

\begin{longtable}[]{@{}
  >{\raggedright\arraybackslash}p{(\linewidth - 2\tabcolsep) * \real{0.5000}}
  >{\raggedright\arraybackslash}p{(\linewidth - 2\tabcolsep) * \real{0.5000}}@{}}
\toprule\noalign{}
\begin{minipage}[b]{\linewidth}\raggedright
Variant
\end{minipage} & \begin{minipage}[b]{\linewidth}\raggedright
Details
\end{minipage} \\
\midrule\noalign{}
\endhead
\bottomrule\noalign{}
\endlastfoot
\texttt{\ "\ simple\ "\ } & Determine the line breaks in a simple
first-fit style. \\
\texttt{\ "\ optimized\ "\ } & Optimize the line breaks for the whole
paragraph.

Typst will try to produce more evenly filled lines of text by
considering the whole paragraph when calculating line breaks. \\
\end{longtable}

Default: \texttt{\ }{\texttt{\ auto\ }}\texttt{\ }

\includesvg[width=0.16667in,height=0.16667in]{/assets/icons/16-arrow-right.svg}
View example

\begin{verbatim}
#set page(width: 207pt)
#set par(linebreaks: "simple")
Some texts feature many longer
words. Those are often exceedingly
challenging to break in a visually
pleasing way.

#set par(linebreaks: "optimized")
Some texts feature many longer
words. Those are often exceedingly
challenging to break in a visually
pleasing way.
\end{verbatim}

\includegraphics[width=4.3125in,height=\textheight,keepaspectratio]{/assets/docs/r-fawkmmJ6Sniwi8--ib5gAAAAAAAAAA.png}

\subsubsection{\texorpdfstring{\texttt{\ first-line-indent\ }}{ first-line-indent }}\label{parameters-first-line-indent}

\href{/docs/reference/layout/length/}{length}

{{ Settable }}

\phantomsection\label{parameters-first-line-indent-settable-tooltip}
Settable parameters can be customized for all following uses of the
function with a \texttt{\ set\ } rule.

The indent the first line of a paragraph should have.

Only the first line of a consecutive paragraph will be indented (not the
first one in a block or on the page).

By typographic convention, paragraph breaks are indicated either by some
space between paragraphs or by indented first lines. Consider reducing
the \href{/docs/reference/layout/block/\#parameters-spacing}{paragraph
spacing} to the
\href{/docs/reference/model/par/\#parameters-leading}{\texttt{\ leading\ }}
when using this property (e.g. using
\texttt{\ }{\texttt{\ \#\ }}\texttt{\ }{\texttt{\ set\ }}\texttt{\ }{\texttt{\ par\ }}\texttt{\ }{\texttt{\ (\ }}\texttt{\ spacing\ }{\texttt{\ :\ }}\texttt{\ }{\texttt{\ 0.65em\ }}\texttt{\ }{\texttt{\ )\ }}\texttt{\ }
).

Default: \texttt{\ }{\texttt{\ 0pt\ }}\texttt{\ }

\subsubsection{\texorpdfstring{\texttt{\ hanging-indent\ }}{ hanging-indent }}\label{parameters-hanging-indent}

\href{/docs/reference/layout/length/}{length}

{{ Settable }}

\phantomsection\label{parameters-hanging-indent-settable-tooltip}
Settable parameters can be customized for all following uses of the
function with a \texttt{\ set\ } rule.

The indent all but the first line of a paragraph should have.

Default: \texttt{\ }{\texttt{\ 0pt\ }}\texttt{\ }

\subsubsection{\texorpdfstring{\texttt{\ body\ }}{ body }}\label{parameters-body}

\href{/docs/reference/foundations/content/}{content}

{Required} {{ Positional }}

\phantomsection\label{parameters-body-positional-tooltip}
Positional parameters are specified in order, without names.

The contents of the paragraph.

\subsection{\texorpdfstring{{ Definitions
}}{ Definitions }}\label{definitions}

\phantomsection\label{definitions-tooltip}
Functions and types and can have associated definitions. These are
accessed by specifying the function or type, followed by a period, and
then the definition\textquotesingle s name.

\subsubsection{\texorpdfstring{\texttt{\ line\ } {{ Element
}}}{ line   Element }}\label{definitions-line}

\phantomsection\label{definitions-line-element-tooltip}
Element functions can be customized with \texttt{\ set\ } and
\texttt{\ show\ } rules.

A paragraph line.

This element is exclusively used for line number configuration through
set rules and cannot be placed.

The
\href{/docs/reference/model/par/\#definitions-line-numbering}{\texttt{\ numbering\ }}
option is used to enable line numbers by specifying a numbering format.

par { . } { line } (

{ \hyperref[definitions-line-parameters-numbering]{numbering :}
\href{/docs/reference/foundations/none/}{none}
\href{/docs/reference/foundations/str/}{str}
\href{/docs/reference/foundations/function/}{function} , } {
\hyperref[definitions-line-parameters-number-align]{number-align :}
\href{/docs/reference/foundations/auto/}{auto}
\href{/docs/reference/layout/alignment/}{alignment} , } {
\hyperref[definitions-line-parameters-number-margin]{number-margin :}
\href{/docs/reference/layout/alignment/}{alignment} , } {
\hyperref[definitions-line-parameters-number-clearance]{number-clearance
:} \href{/docs/reference/foundations/auto/}{auto}
\href{/docs/reference/layout/length/}{length} , } {
\hyperref[definitions-line-parameters-numbering-scope]{numbering-scope
:} \href{/docs/reference/foundations/str/}{str} , }

) -\textgreater{} \href{/docs/reference/foundations/content/}{content}

\begin{verbatim}
#set par.line(numbering: "1")

Roses are red. \
Violets are blue. \
Typst is there for you.
\end{verbatim}

\includegraphics[width=5in,height=\textheight,keepaspectratio]{/assets/docs/b257YLHUEagbFWlPeD4gEwAAAAAAAAAA.png}

The \texttt{\ numbering\ } option takes either a predefined
\href{/docs/reference/model/numbering/}{numbering pattern} or a function
returning styled content. You can disable line numbers for text inside
certain elements by setting the numbering to
\texttt{\ }{\texttt{\ none\ }}\texttt{\ } using show-set rules.

\begin{verbatim}
// Styled red line numbers.
#set par.line(
  numbering: n => text(red)[#n]
)

// Disable numbers inside figures.
#show figure: set par.line(
  numbering: none
)

Roses are red. \
Violets are blue.

#figure(
  caption: [Without line numbers.]
)[
  Lorem ipsum \
  dolor sit amet
]

The text above is a sample \
originating from distant times.
\end{verbatim}

\includegraphics[width=5in,height=\textheight,keepaspectratio]{/assets/docs/WJNwFvR3ObvODT-MbWqflAAAAAAAAAAA.png}

This element exposes further options which may be used to control other
aspects of line numbering, such as its
\href{/docs/reference/model/par/\#definitions-line-number-align}{alignment}
or
\href{/docs/reference/model/par/\#definitions-line-number-margin}{margin}
. In addition, you can control whether the numbering is reset on each
page through the
\href{/docs/reference/model/par/\#definitions-line-numbering-scope}{\texttt{\ numbering-scope\ }}
option.

\paragraph{\texorpdfstring{\texttt{\ numbering\ }}{ numbering }}\label{definitions-line-numbering}

\href{/docs/reference/foundations/none/}{none} {or}
\href{/docs/reference/foundations/str/}{str} {or}
\href{/docs/reference/foundations/function/}{function}

{{ Settable }}

\phantomsection\label{definitions-line-numbering-settable-tooltip}
Settable parameters can be customized for all following uses of the
function with a \texttt{\ set\ } rule.

How to number each line. Accepts a
\href{/docs/reference/model/numbering/}{numbering pattern or function} .

Default: \texttt{\ }{\texttt{\ none\ }}\texttt{\ }

\includesvg[width=0.16667in,height=0.16667in]{/assets/icons/16-arrow-right.svg}
View example

\begin{verbatim}
#set par.line(numbering: "I")

Roses are red. \
Violets are blue. \
Typst is there for you.
\end{verbatim}

\includegraphics[width=5in,height=\textheight,keepaspectratio]{/assets/docs/O-oJqYc-OwEoxappxK4AZAAAAAAAAAAA.png}

\paragraph{\texorpdfstring{\texttt{\ number-align\ }}{ number-align }}\label{definitions-line-number-align}

\href{/docs/reference/foundations/auto/}{auto} {or}
\href{/docs/reference/layout/alignment/}{alignment}

{{ Settable }}

\phantomsection\label{definitions-line-number-align-settable-tooltip}
Settable parameters can be customized for all following uses of the
function with a \texttt{\ set\ } rule.

The alignment of line numbers associated with each line.

The default of \texttt{\ }{\texttt{\ auto\ }}\texttt{\ } indicates a
smart default where numbers grow horizontally away from the text,
considering the margin they\textquotesingle re in and the current text
direction.

Default: \texttt{\ }{\texttt{\ auto\ }}\texttt{\ }

\includesvg[width=0.16667in,height=0.16667in]{/assets/icons/16-arrow-right.svg}
View example

\begin{verbatim}
#set par.line(
  numbering: "I",
  number-align: left,
)

Hello world! \
Today is a beautiful day \
For exploring the world.
\end{verbatim}

\includegraphics[width=5in,height=\textheight,keepaspectratio]{/assets/docs/XfwBMgYjt2fGeRgFr_kj4AAAAAAAAAAA.png}

\paragraph{\texorpdfstring{\texttt{\ number-margin\ }}{ number-margin }}\label{definitions-line-number-margin}

\href{/docs/reference/layout/alignment/}{alignment}

{{ Settable }}

\phantomsection\label{definitions-line-number-margin-settable-tooltip}
Settable parameters can be customized for all following uses of the
function with a \texttt{\ set\ } rule.

The margin at which line numbers appear.

\emph{Note:} In a multi-column document, the line numbers for paragraphs
inside the last column will always appear on the \texttt{\ end\ } margin
(right margin for left-to-right text and left margin for right-to-left),
regardless of this configuration. That behavior cannot be changed at
this moment.

Default: \texttt{\ start\ }

\includesvg[width=0.16667in,height=0.16667in]{/assets/icons/16-arrow-right.svg}
View example

\begin{verbatim}
#set par.line(
  numbering: "1",
  number-margin: right,
)

= Report
- Brightness: Dark, yet darker
- Readings: Negative
\end{verbatim}

\includegraphics[width=5in,height=\textheight,keepaspectratio]{/assets/docs/vf0ZBrlygVUABySMskTaKQAAAAAAAAAA.png}

\paragraph{\texorpdfstring{\texttt{\ number-clearance\ }}{ number-clearance }}\label{definitions-line-number-clearance}

\href{/docs/reference/foundations/auto/}{auto} {or}
\href{/docs/reference/layout/length/}{length}

{{ Settable }}

\phantomsection\label{definitions-line-number-clearance-settable-tooltip}
Settable parameters can be customized for all following uses of the
function with a \texttt{\ set\ } rule.

The distance between line numbers and text.

The default value of \texttt{\ }{\texttt{\ auto\ }}\texttt{\ } results
in a clearance that is adaptive to the page width and yields reasonable
results in most cases.

Default: \texttt{\ }{\texttt{\ auto\ }}\texttt{\ }

\includesvg[width=0.16667in,height=0.16667in]{/assets/icons/16-arrow-right.svg}
View example

\begin{verbatim}
#set par.line(
  numbering: "1",
  number-clearance: 4pt,
)

Typesetting \
Styling \
Layout
\end{verbatim}

\includegraphics[width=5in,height=\textheight,keepaspectratio]{/assets/docs/MgiUB3LoxE0JROWoHJPslgAAAAAAAAAA.png}

\paragraph{\texorpdfstring{\texttt{\ numbering-scope\ }}{ numbering-scope }}\label{definitions-line-numbering-scope}

\href{/docs/reference/foundations/str/}{str}

{{ Settable }}

\phantomsection\label{definitions-line-numbering-scope-settable-tooltip}
Settable parameters can be customized for all following uses of the
function with a \texttt{\ set\ } rule.

Controls when to reset line numbering.

\emph{Note:} The line numbering scope must be uniform across each page
run (a page run is a sequence of pages without an explicit pagebreak in
between). For this reason, set rules for it should be defined before any
page content, typically at the very start of the document.

\begin{longtable}[]{@{}ll@{}}
\toprule\noalign{}
Variant & Details \\
\midrule\noalign{}
\endhead
\bottomrule\noalign{}
\endlastfoot
\texttt{\ "\ document\ "\ } & Indicates the line number counter spans
the whole document, that is, is never automatically reset. \\
\texttt{\ "\ page\ "\ } & Indicates the line number counter should be
reset at the start of every new page. \\
\end{longtable}

Default: \texttt{\ }{\texttt{\ "document"\ }}\texttt{\ }

\includesvg[width=0.16667in,height=0.16667in]{/assets/icons/16-arrow-right.svg}
View example

\begin{verbatim}
#set par.line(
  numbering: "1",
  numbering-scope: "page",
)

First line \
Second line
#pagebreak()
First line again \
Second line again
\end{verbatim}

\includegraphics[width=5in,height=\textheight,keepaspectratio]{/assets/docs/MmmIOu-UB2sC4GlOg3oj9AAAAAAAAAAA.png}
\includegraphics[width=5in,height=\textheight,keepaspectratio]{/assets/docs/MmmIOu-UB2sC4GlOg3oj9AAAAAAAAAAB.png}

\href{/docs/reference/model/outline/}{\pandocbounded{\includesvg[keepaspectratio]{/assets/icons/16-arrow-right.svg}}}

{ Outline } { Previous page }

\href{/docs/reference/model/parbreak/}{\pandocbounded{\includesvg[keepaspectratio]{/assets/icons/16-arrow-right.svg}}}

{ Paragraph Break } { Next page }


\title{typst.app/docs/reference/model/list}

\begin{itemize}
\tightlist
\item
  \href{/docs}{\includesvg[width=0.16667in,height=0.16667in]{/assets/icons/16-docs-dark.svg}}
\item
  \includesvg[width=0.16667in,height=0.16667in]{/assets/icons/16-arrow-right.svg}
\item
  \href{/docs/reference/}{Reference}
\item
  \includesvg[width=0.16667in,height=0.16667in]{/assets/icons/16-arrow-right.svg}
\item
  \href{/docs/reference/model/}{Model}
\item
  \includesvg[width=0.16667in,height=0.16667in]{/assets/icons/16-arrow-right.svg}
\item
  \href{/docs/reference/model/list/}{Bullet List}
\end{itemize}

\section{\texorpdfstring{\texttt{\ list\ } {{ Element
}}}{ list   Element }}\label{summary}

\phantomsection\label{element-tooltip}
Element functions can be customized with \texttt{\ set\ } and
\texttt{\ show\ } rules.

A bullet list.

Displays a sequence of items vertically, with each item introduced by a
marker.

\subsection{Example}\label{example}

\begin{verbatim}
Normal list.
- Text
- Math
- Layout
- ...

Multiple lines.
- This list item spans multiple
  lines because it is indented.

Function call.
#list(
  [Foundations],
  [Calculate],
  [Construct],
  [Data Loading],
)
\end{verbatim}

\includegraphics[width=5in,height=\textheight,keepaspectratio]{/assets/docs/dGd96M9aTTHo-jKJ9y73kwAAAAAAAAAA.png}

\subsection{Syntax}\label{syntax}

This functions also has dedicated syntax: Start a line with a hyphen,
followed by a space to create a list item. A list item can contain
multiple paragraphs and other block-level content. All content that is
indented more than an item\textquotesingle s marker becomes part of that
item.

\subsection{\texorpdfstring{{ Parameters
}}{ Parameters }}\label{parameters}

\phantomsection\label{parameters-tooltip}
Parameters are the inputs to a function. They are specified in
parentheses after the function name.

{ list } (

{ \hyperref[parameters-tight]{tight :}
\href{/docs/reference/foundations/bool/}{bool} , } {
\hyperref[parameters-marker]{marker :}
\href{/docs/reference/foundations/content/}{content}
\href{/docs/reference/foundations/array/}{array}
\href{/docs/reference/foundations/function/}{function} , } {
\hyperref[parameters-indent]{indent :}
\href{/docs/reference/layout/length/}{length} , } {
\hyperref[parameters-body-indent]{body-indent :}
\href{/docs/reference/layout/length/}{length} , } {
\hyperref[parameters-spacing]{spacing :}
\href{/docs/reference/foundations/auto/}{auto}
\href{/docs/reference/layout/length/}{length} , } {
\hyperref[parameters-children]{..}
\href{/docs/reference/foundations/content/}{content} , }

) -\textgreater{} \href{/docs/reference/foundations/content/}{content}

\subsubsection{\texorpdfstring{\texttt{\ tight\ }}{ tight }}\label{parameters-tight}

\href{/docs/reference/foundations/bool/}{bool}

{{ Settable }}

\phantomsection\label{parameters-tight-settable-tooltip}
Settable parameters can be customized for all following uses of the
function with a \texttt{\ set\ } rule.

Defines the default
\href{/docs/reference/model/list/\#parameters-spacing}{spacing} of the
list. If it is \texttt{\ }{\texttt{\ false\ }}\texttt{\ } , the items
are spaced apart with
\href{/docs/reference/model/par/\#parameters-spacing}{paragraph spacing}
. If it is \texttt{\ }{\texttt{\ true\ }}\texttt{\ } , they use
\href{/docs/reference/model/par/\#parameters-leading}{paragraph leading}
instead. This makes the list more compact, which can look better if the
items are short.

In markup mode, the value of this parameter is determined based on
whether items are separated with a blank line. If items directly follow
each other, this is set to \texttt{\ }{\texttt{\ true\ }}\texttt{\ } ;
if items are separated by a blank line, this is set to
\texttt{\ }{\texttt{\ false\ }}\texttt{\ } . The markup-defined
tightness cannot be overridden with set rules.

Default: \texttt{\ }{\texttt{\ true\ }}\texttt{\ }

\includesvg[width=0.16667in,height=0.16667in]{/assets/icons/16-arrow-right.svg}
View example

\begin{verbatim}
- If a list has a lot of text, and
  maybe other inline content, it
  should not be tight anymore.

- To make a list wide, simply insert
  a blank line between the items.
\end{verbatim}

\includegraphics[width=5in,height=\textheight,keepaspectratio]{/assets/docs/4FUPGE5Zxz4-Z1S-m_IFCQAAAAAAAAAA.png}

\subsubsection{\texorpdfstring{\texttt{\ marker\ }}{ marker }}\label{parameters-marker}

\href{/docs/reference/foundations/content/}{content} {or}
\href{/docs/reference/foundations/array/}{array} {or}
\href{/docs/reference/foundations/function/}{function}

{{ Settable }}

\phantomsection\label{parameters-marker-settable-tooltip}
Settable parameters can be customized for all following uses of the
function with a \texttt{\ set\ } rule.

The marker which introduces each item.

Instead of plain content, you can also pass an array with multiple
markers that should be used for nested lists. If the list nesting depth
exceeds the number of markers, the markers are cycled. For total
control, you may pass a function that maps the list\textquotesingle s
nesting depth (starting from \texttt{\ }{\texttt{\ 0\ }}\texttt{\ } ) to
a desired marker.

Default:
\texttt{\ }{\texttt{\ (\ }}\texttt{\ }{\texttt{\ {[}\ }}\texttt{\ •\ }{\texttt{\ {]}\ }}\texttt{\ }{\texttt{\ ,\ }}\texttt{\ }{\texttt{\ {[}\ }}\texttt{\ ‣\ }{\texttt{\ {]}\ }}\texttt{\ }{\texttt{\ ,\ }}\texttt{\ }{\texttt{\ {[}\ }}\texttt{\ –\ }{\texttt{\ {]}\ }}\texttt{\ }{\texttt{\ )\ }}\texttt{\ }

\includesvg[width=0.16667in,height=0.16667in]{/assets/icons/16-arrow-right.svg}
View example

\begin{verbatim}
#set list(marker: [--])
- A more classic list
- With en-dashes

#set list(marker: ([•], [--]))
- Top-level
  - Nested
  - Items
- Items
\end{verbatim}

\includegraphics[width=5in,height=\textheight,keepaspectratio]{/assets/docs/rGFZOVIfGIEORB3iENBotQAAAAAAAAAA.png}

\subsubsection{\texorpdfstring{\texttt{\ indent\ }}{ indent }}\label{parameters-indent}

\href{/docs/reference/layout/length/}{length}

{{ Settable }}

\phantomsection\label{parameters-indent-settable-tooltip}
Settable parameters can be customized for all following uses of the
function with a \texttt{\ set\ } rule.

The indent of each item.

Default: \texttt{\ }{\texttt{\ 0pt\ }}\texttt{\ }

\subsubsection{\texorpdfstring{\texttt{\ body-indent\ }}{ body-indent }}\label{parameters-body-indent}

\href{/docs/reference/layout/length/}{length}

{{ Settable }}

\phantomsection\label{parameters-body-indent-settable-tooltip}
Settable parameters can be customized for all following uses of the
function with a \texttt{\ set\ } rule.

The spacing between the marker and the body of each item.

Default: \texttt{\ }{\texttt{\ 0.5em\ }}\texttt{\ }

\subsubsection{\texorpdfstring{\texttt{\ spacing\ }}{ spacing }}\label{parameters-spacing}

\href{/docs/reference/foundations/auto/}{auto} {or}
\href{/docs/reference/layout/length/}{length}

{{ Settable }}

\phantomsection\label{parameters-spacing-settable-tooltip}
Settable parameters can be customized for all following uses of the
function with a \texttt{\ set\ } rule.

The spacing between the items of the list.

If set to \texttt{\ }{\texttt{\ auto\ }}\texttt{\ } , uses paragraph
\href{/docs/reference/model/par/\#parameters-leading}{\texttt{\ leading\ }}
for tight lists and paragraph
\href{/docs/reference/model/par/\#parameters-spacing}{\texttt{\ spacing\ }}
for wide (non-tight) lists.

Default: \texttt{\ }{\texttt{\ auto\ }}\texttt{\ }

\subsubsection{\texorpdfstring{\texttt{\ children\ }}{ children }}\label{parameters-children}

\href{/docs/reference/foundations/content/}{content}

{Required} {{ Positional }}

\phantomsection\label{parameters-children-positional-tooltip}
Positional parameters are specified in order, without names.

{{ Variadic }}

\phantomsection\label{parameters-children-variadic-tooltip}
Variadic parameters can be specified multiple times.

The bullet list\textquotesingle s children.

When using the list syntax, adjacent items are automatically collected
into lists, even through constructs like for loops.

\includesvg[width=0.16667in,height=0.16667in]{/assets/icons/16-arrow-right.svg}
View example

\begin{verbatim}
#for letter in "ABC" [
  - Letter #letter
]
\end{verbatim}

\includegraphics[width=5in,height=\textheight,keepaspectratio]{/assets/docs/scttBXkLjYOvlJchbuo00wAAAAAAAAAA.png}

\subsection{\texorpdfstring{{ Definitions
}}{ Definitions }}\label{definitions}

\phantomsection\label{definitions-tooltip}
Functions and types and can have associated definitions. These are
accessed by specifying the function or type, followed by a period, and
then the definition\textquotesingle s name.

\subsubsection{\texorpdfstring{\texttt{\ item\ } {{ Element
}}}{ item   Element }}\label{definitions-item}

\phantomsection\label{definitions-item-element-tooltip}
Element functions can be customized with \texttt{\ set\ } and
\texttt{\ show\ } rules.

A bullet list item.

list { . } { item } (

{ \href{/docs/reference/foundations/content/}{content} }

) -\textgreater{} \href{/docs/reference/foundations/content/}{content}

\paragraph{\texorpdfstring{\texttt{\ body\ }}{ body }}\label{definitions-item-body}

\href{/docs/reference/foundations/content/}{content}

{Required} {{ Positional }}

\phantomsection\label{definitions-item-body-positional-tooltip}
Positional parameters are specified in order, without names.

The item\textquotesingle s body.

\href{/docs/reference/model/bibliography/}{\pandocbounded{\includesvg[keepaspectratio]{/assets/icons/16-arrow-right.svg}}}

{ Bibliography } { Previous page }

\href{/docs/reference/model/cite/}{\pandocbounded{\includesvg[keepaspectratio]{/assets/icons/16-arrow-right.svg}}}

{ Cite } { Next page }


\title{typst.app/docs/reference/model/enum}

\begin{itemize}
\tightlist
\item
  \href{/docs}{\includesvg[width=0.16667in,height=0.16667in]{/assets/icons/16-docs-dark.svg}}
\item
  \includesvg[width=0.16667in,height=0.16667in]{/assets/icons/16-arrow-right.svg}
\item
  \href{/docs/reference/}{Reference}
\item
  \includesvg[width=0.16667in,height=0.16667in]{/assets/icons/16-arrow-right.svg}
\item
  \href{/docs/reference/model/}{Model}
\item
  \includesvg[width=0.16667in,height=0.16667in]{/assets/icons/16-arrow-right.svg}
\item
  \href{/docs/reference/model/enum/}{Numbered List}
\end{itemize}

\section{\texorpdfstring{\texttt{\ enum\ } {{ Element
}}}{ enum   Element }}\label{summary}

\phantomsection\label{element-tooltip}
Element functions can be customized with \texttt{\ set\ } and
\texttt{\ show\ } rules.

A numbered list.

Displays a sequence of items vertically and numbers them consecutively.

\subsection{Example}\label{example}

\begin{verbatim}
Automatically numbered:
+ Preparations
+ Analysis
+ Conclusions

Manually numbered:
2. What is the first step?
5. I am confused.
+  Moving on ...

Multiple lines:
+ This enum item has multiple
  lines because the next line
  is indented.

Function call.
#enum[First][Second]
\end{verbatim}

\includegraphics[width=5in,height=\textheight,keepaspectratio]{/assets/docs/HrnJ1mRKvbXNf6U4DmZCaAAAAAAAAAAA.png}

You can easily switch all your enumerations to a different numbering
style with a set rule.

\begin{verbatim}
#set enum(numbering: "a)")

+ Starting off ...
+ Don't forget step two
\end{verbatim}

\includegraphics[width=5in,height=\textheight,keepaspectratio]{/assets/docs/hFb68a8DC-Rvf_eMOYtVMwAAAAAAAAAA.png}

You can also use
\href{/docs/reference/model/enum/\#definitions-item}{\texttt{\ enum.item\ }}
to programmatically customize the number of each item in the
enumeration:

\begin{verbatim}
#enum(
  enum.item(1)[First step],
  enum.item(5)[Fifth step],
  enum.item(10)[Tenth step]
)
\end{verbatim}

\includegraphics[width=5in,height=\textheight,keepaspectratio]{/assets/docs/uRQbjXrkv7FwltBxluVMMAAAAAAAAAAA.png}

\subsection{Syntax}\label{syntax}

This functions also has dedicated syntax:

\begin{itemize}
\tightlist
\item
  Starting a line with a plus sign creates an automatically numbered
  enumeration item.
\item
  Starting a line with a number followed by a dot creates an explicitly
  numbered enumeration item.
\end{itemize}

Enumeration items can contain multiple paragraphs and other block-level
content. All content that is indented more than an
item\textquotesingle s marker becomes part of that item.

\subsection{\texorpdfstring{{ Parameters
}}{ Parameters }}\label{parameters}

\phantomsection\label{parameters-tooltip}
Parameters are the inputs to a function. They are specified in
parentheses after the function name.

{ enum } (

{ \hyperref[parameters-tight]{tight :}
\href{/docs/reference/foundations/bool/}{bool} , } {
\hyperref[parameters-numbering]{numbering :}
\href{/docs/reference/foundations/str/}{str}
\href{/docs/reference/foundations/function/}{function} , } {
\hyperref[parameters-start]{start :}
\href{/docs/reference/foundations/int/}{int} , } {
\hyperref[parameters-full]{full :}
\href{/docs/reference/foundations/bool/}{bool} , } {
\hyperref[parameters-indent]{indent :}
\href{/docs/reference/layout/length/}{length} , } {
\hyperref[parameters-body-indent]{body-indent :}
\href{/docs/reference/layout/length/}{length} , } {
\hyperref[parameters-spacing]{spacing :}
\href{/docs/reference/foundations/auto/}{auto}
\href{/docs/reference/layout/length/}{length} , } {
\hyperref[parameters-number-align]{number-align :}
\href{/docs/reference/layout/alignment/}{alignment} , } {
\hyperref[parameters-children]{..}
\href{/docs/reference/foundations/content/}{content}
\href{/docs/reference/foundations/array/}{array} , }

) -\textgreater{} \href{/docs/reference/foundations/content/}{content}

\subsubsection{\texorpdfstring{\texttt{\ tight\ }}{ tight }}\label{parameters-tight}

\href{/docs/reference/foundations/bool/}{bool}

{{ Settable }}

\phantomsection\label{parameters-tight-settable-tooltip}
Settable parameters can be customized for all following uses of the
function with a \texttt{\ set\ } rule.

Defines the default
\href{/docs/reference/model/enum/\#parameters-spacing}{spacing} of the
enumeration. If it is \texttt{\ }{\texttt{\ false\ }}\texttt{\ } , the
items are spaced apart with
\href{/docs/reference/model/par/\#parameters-spacing}{paragraph spacing}
. If it is \texttt{\ }{\texttt{\ true\ }}\texttt{\ } , they use
\href{/docs/reference/model/par/\#parameters-leading}{paragraph leading}
instead. This makes the list more compact, which can look better if the
items are short.

In markup mode, the value of this parameter is determined based on
whether items are separated with a blank line. If items directly follow
each other, this is set to \texttt{\ }{\texttt{\ true\ }}\texttt{\ } ;
if items are separated by a blank line, this is set to
\texttt{\ }{\texttt{\ false\ }}\texttt{\ } . The markup-defined
tightness cannot be overridden with set rules.

Default: \texttt{\ }{\texttt{\ true\ }}\texttt{\ }

\includesvg[width=0.16667in,height=0.16667in]{/assets/icons/16-arrow-right.svg}
View example

\begin{verbatim}
+ If an enum has a lot of text, and
  maybe other inline content, it
  should not be tight anymore.

+ To make an enum wide, simply
  insert a blank line between the
  items.
\end{verbatim}

\includegraphics[width=5in,height=\textheight,keepaspectratio]{/assets/docs/CGCi1WYCPLux25Xc9ZWwDQAAAAAAAAAA.png}

\subsubsection{\texorpdfstring{\texttt{\ numbering\ }}{ numbering }}\label{parameters-numbering}

\href{/docs/reference/foundations/str/}{str} {or}
\href{/docs/reference/foundations/function/}{function}

{{ Settable }}

\phantomsection\label{parameters-numbering-settable-tooltip}
Settable parameters can be customized for all following uses of the
function with a \texttt{\ set\ } rule.

How to number the enumeration. Accepts a
\href{/docs/reference/model/numbering/}{numbering pattern or function} .

If the numbering pattern contains multiple counting symbols, they apply
to nested enums. If given a function, the function receives one argument
if \texttt{\ full\ } is \texttt{\ }{\texttt{\ false\ }}\texttt{\ } and
multiple arguments if \texttt{\ full\ } is
\texttt{\ }{\texttt{\ true\ }}\texttt{\ } .

Default: \texttt{\ }{\texttt{\ "1."\ }}\texttt{\ }

\includesvg[width=0.16667in,height=0.16667in]{/assets/icons/16-arrow-right.svg}
View example

\begin{verbatim}
#set enum(numbering: "1.a)")
+ Different
+ Numbering
  + Nested
  + Items
+ Style

#set enum(numbering: n => super[#n])
+ Superscript
+ Numbering!
\end{verbatim}

\includegraphics[width=5in,height=\textheight,keepaspectratio]{/assets/docs/b_5poTPc-SH9hcwOp4TcbAAAAAAAAAAA.png}

\subsubsection{\texorpdfstring{\texttt{\ start\ }}{ start }}\label{parameters-start}

\href{/docs/reference/foundations/int/}{int}

{{ Settable }}

\phantomsection\label{parameters-start-settable-tooltip}
Settable parameters can be customized for all following uses of the
function with a \texttt{\ set\ } rule.

Which number to start the enumeration with.

Default: \texttt{\ }{\texttt{\ 1\ }}\texttt{\ }

\includesvg[width=0.16667in,height=0.16667in]{/assets/icons/16-arrow-right.svg}
View example

\begin{verbatim}
#enum(
  start: 3,
  [Skipping],
  [Ahead],
)
\end{verbatim}

\includegraphics[width=5in,height=\textheight,keepaspectratio]{/assets/docs/NqaMIUfLtrq2fhf9xChjagAAAAAAAAAA.png}

\subsubsection{\texorpdfstring{\texttt{\ full\ }}{ full }}\label{parameters-full}

\href{/docs/reference/foundations/bool/}{bool}

{{ Settable }}

\phantomsection\label{parameters-full-settable-tooltip}
Settable parameters can be customized for all following uses of the
function with a \texttt{\ set\ } rule.

Whether to display the full numbering, including the numbers of all
parent enumerations.

Default: \texttt{\ }{\texttt{\ false\ }}\texttt{\ }

\includesvg[width=0.16667in,height=0.16667in]{/assets/icons/16-arrow-right.svg}
View example

\begin{verbatim}
#set enum(numbering: "1.a)", full: true)
+ Cook
  + Heat water
  + Add ingredients
+ Eat
\end{verbatim}

\includegraphics[width=5in,height=\textheight,keepaspectratio]{/assets/docs/ecL0fn92ARx_6xbLZYFkVAAAAAAAAAAA.png}

\subsubsection{\texorpdfstring{\texttt{\ indent\ }}{ indent }}\label{parameters-indent}

\href{/docs/reference/layout/length/}{length}

{{ Settable }}

\phantomsection\label{parameters-indent-settable-tooltip}
Settable parameters can be customized for all following uses of the
function with a \texttt{\ set\ } rule.

The indentation of each item.

Default: \texttt{\ }{\texttt{\ 0pt\ }}\texttt{\ }

\subsubsection{\texorpdfstring{\texttt{\ body-indent\ }}{ body-indent }}\label{parameters-body-indent}

\href{/docs/reference/layout/length/}{length}

{{ Settable }}

\phantomsection\label{parameters-body-indent-settable-tooltip}
Settable parameters can be customized for all following uses of the
function with a \texttt{\ set\ } rule.

The space between the numbering and the body of each item.

Default: \texttt{\ }{\texttt{\ 0.5em\ }}\texttt{\ }

\subsubsection{\texorpdfstring{\texttt{\ spacing\ }}{ spacing }}\label{parameters-spacing}

\href{/docs/reference/foundations/auto/}{auto} {or}
\href{/docs/reference/layout/length/}{length}

{{ Settable }}

\phantomsection\label{parameters-spacing-settable-tooltip}
Settable parameters can be customized for all following uses of the
function with a \texttt{\ set\ } rule.

The spacing between the items of the enumeration.

If set to \texttt{\ }{\texttt{\ auto\ }}\texttt{\ } , uses paragraph
\href{/docs/reference/model/par/\#parameters-leading}{\texttt{\ leading\ }}
for tight enumerations and paragraph
\href{/docs/reference/model/par/\#parameters-spacing}{\texttt{\ spacing\ }}
for wide (non-tight) enumerations.

Default: \texttt{\ }{\texttt{\ auto\ }}\texttt{\ }

\subsubsection{\texorpdfstring{\texttt{\ number-align\ }}{ number-align }}\label{parameters-number-align}

\href{/docs/reference/layout/alignment/}{alignment}

{{ Settable }}

\phantomsection\label{parameters-number-align-settable-tooltip}
Settable parameters can be customized for all following uses of the
function with a \texttt{\ set\ } rule.

The alignment that enum numbers should have.

By default, this is set to
\texttt{\ end\ }{\texttt{\ +\ }}\texttt{\ top\ } , which aligns enum
numbers towards end of the current text direction (in left-to-right
script, for example, this is the same as \texttt{\ right\ } ) and at the
top of the line. The choice of \texttt{\ end\ } for horizontal alignment
of enum numbers is usually preferred over \texttt{\ start\ } , as
numbers then grow away from the text instead of towards it, avoiding
certain visual issues. This option lets you override this behaviour,
however. (Also to note is that the
\href{/docs/reference/model/list/}{unordered list} uses a different
method for this, by giving the \texttt{\ marker\ } content an alignment
directly.).

Default: \texttt{\ end\ }{\texttt{\ +\ }}\texttt{\ top\ }

\includesvg[width=0.16667in,height=0.16667in]{/assets/icons/16-arrow-right.svg}
View example

\begin{verbatim}
#set enum(number-align: start + bottom)

Here are some powers of two:
1. One
2. Two
4. Four
8. Eight
16. Sixteen
32. Thirty two
\end{verbatim}

\includegraphics[width=5in,height=\textheight,keepaspectratio]{/assets/docs/s-zUl9r9z6yKdW4VnsLi_AAAAAAAAAAA.png}

\subsubsection{\texorpdfstring{\texttt{\ children\ }}{ children }}\label{parameters-children}

\href{/docs/reference/foundations/content/}{content} {or}
\href{/docs/reference/foundations/array/}{array}

{Required} {{ Positional }}

\phantomsection\label{parameters-children-positional-tooltip}
Positional parameters are specified in order, without names.

{{ Variadic }}

\phantomsection\label{parameters-children-variadic-tooltip}
Variadic parameters can be specified multiple times.

The numbered list\textquotesingle s items.

When using the enum syntax, adjacent items are automatically collected
into enumerations, even through constructs like for loops.

\includesvg[width=0.16667in,height=0.16667in]{/assets/icons/16-arrow-right.svg}
View example

\begin{verbatim}
#for phase in (
   "Launch",
   "Orbit",
   "Descent",
) [+ #phase]
\end{verbatim}

\includegraphics[width=5in,height=\textheight,keepaspectratio]{/assets/docs/9haSHPkr8gDAx-1cEtmf8QAAAAAAAAAA.png}

\subsection{\texorpdfstring{{ Definitions
}}{ Definitions }}\label{definitions}

\phantomsection\label{definitions-tooltip}
Functions and types and can have associated definitions. These are
accessed by specifying the function or type, followed by a period, and
then the definition\textquotesingle s name.

\subsubsection{\texorpdfstring{\texttt{\ item\ } {{ Element
}}}{ item   Element }}\label{definitions-item}

\phantomsection\label{definitions-item-element-tooltip}
Element functions can be customized with \texttt{\ set\ } and
\texttt{\ show\ } rules.

An enumeration item.

enum { . } { item } (

{ \hyperref[definitions-item-parameters-number]{}
\href{/docs/reference/foundations/none/}{none}
\href{/docs/reference/foundations/int/}{int} , } {
\href{/docs/reference/foundations/content/}{content} , }

) -\textgreater{} \href{/docs/reference/foundations/content/}{content}

\paragraph{\texorpdfstring{\texttt{\ number\ }}{ number }}\label{definitions-item-number}

\href{/docs/reference/foundations/none/}{none} {or}
\href{/docs/reference/foundations/int/}{int}

{{ Positional }}

\phantomsection\label{definitions-item-number-positional-tooltip}
Positional parameters are specified in order, without names.

{{ Settable }}

\phantomsection\label{definitions-item-number-settable-tooltip}
Settable parameters can be customized for all following uses of the
function with a \texttt{\ set\ } rule.

The item\textquotesingle s number.

Default: \texttt{\ }{\texttt{\ none\ }}\texttt{\ }

\paragraph{\texorpdfstring{\texttt{\ body\ }}{ body }}\label{definitions-item-body}

\href{/docs/reference/foundations/content/}{content}

{Required} {{ Positional }}

\phantomsection\label{definitions-item-body-positional-tooltip}
Positional parameters are specified in order, without names.

The item\textquotesingle s body.

\href{/docs/reference/model/link/}{\pandocbounded{\includesvg[keepaspectratio]{/assets/icons/16-arrow-right.svg}}}

{ Link } { Previous page }

\href{/docs/reference/model/numbering/}{\pandocbounded{\includesvg[keepaspectratio]{/assets/icons/16-arrow-right.svg}}}

{ Numbering } { Next page }


\title{typst.app/docs/reference/model/heading}

\begin{itemize}
\tightlist
\item
  \href{/docs}{\includesvg[width=0.16667in,height=0.16667in]{/assets/icons/16-docs-dark.svg}}
\item
  \includesvg[width=0.16667in,height=0.16667in]{/assets/icons/16-arrow-right.svg}
\item
  \href{/docs/reference/}{Reference}
\item
  \includesvg[width=0.16667in,height=0.16667in]{/assets/icons/16-arrow-right.svg}
\item
  \href{/docs/reference/model/}{Model}
\item
  \includesvg[width=0.16667in,height=0.16667in]{/assets/icons/16-arrow-right.svg}
\item
  \href{/docs/reference/model/heading/}{Heading}
\end{itemize}

\section{\texorpdfstring{\texttt{\ heading\ } {{ Element
}}}{ heading   Element }}\label{summary}

\phantomsection\label{element-tooltip}
Element functions can be customized with \texttt{\ set\ } and
\texttt{\ show\ } rules.

A section heading.

With headings, you can structure your document into sections. Each
heading has a \emph{level,} which starts at one and is unbounded
upwards. This level indicates the logical role of the following content
(section, subsection, etc.) A top-level heading indicates a top-level
section of the document (not the document\textquotesingle s title).

Typst can automatically number your headings for you. To enable
numbering, specify how you want your headings to be numbered with a
\href{/docs/reference/model/numbering/}{numbering pattern or function} .

Independently of the numbering, Typst can also automatically generate an
\href{/docs/reference/model/outline/}{outline} of all headings for you.
To exclude one or more headings from this outline, you can set the
\texttt{\ outlined\ } parameter to
\texttt{\ }{\texttt{\ false\ }}\texttt{\ } .

\subsection{Example}\label{example}

\begin{verbatim}
#set heading(numbering: "1.a)")

= Introduction
In recent years, ...

== Preliminaries
To start, ...
\end{verbatim}

\includegraphics[width=5in,height=\textheight,keepaspectratio]{/assets/docs/PajtbDMMN2eDYZCkAh9ZJwAAAAAAAAAA.png}

\subsection{Syntax}\label{syntax}

Headings have dedicated syntax: They can be created by starting a line
with one or multiple equals signs, followed by a space. The number of
equals signs determines the heading\textquotesingle s logical nesting
depth. The \texttt{\ offset\ } field can be set to configure the
starting depth.

\subsection{\texorpdfstring{{ Parameters
}}{ Parameters }}\label{parameters}

\phantomsection\label{parameters-tooltip}
Parameters are the inputs to a function. They are specified in
parentheses after the function name.

{ heading } (

{ \hyperref[parameters-level]{level :}
\href{/docs/reference/foundations/auto/}{auto}
\href{/docs/reference/foundations/int/}{int} , } {
\hyperref[parameters-depth]{depth :}
\href{/docs/reference/foundations/int/}{int} , } {
\hyperref[parameters-offset]{offset :}
\href{/docs/reference/foundations/int/}{int} , } {
\hyperref[parameters-numbering]{numbering :}
\href{/docs/reference/foundations/none/}{none}
\href{/docs/reference/foundations/str/}{str}
\href{/docs/reference/foundations/function/}{function} , } {
\hyperref[parameters-supplement]{supplement :}
\href{/docs/reference/foundations/none/}{none}
\href{/docs/reference/foundations/auto/}{auto}
\href{/docs/reference/foundations/content/}{content}
\href{/docs/reference/foundations/function/}{function} , } {
\hyperref[parameters-outlined]{outlined :}
\href{/docs/reference/foundations/bool/}{bool} , } {
\hyperref[parameters-bookmarked]{bookmarked :}
\href{/docs/reference/foundations/auto/}{auto}
\href{/docs/reference/foundations/bool/}{bool} , } {
\hyperref[parameters-hanging-indent]{hanging-indent :}
\href{/docs/reference/foundations/auto/}{auto}
\href{/docs/reference/layout/length/}{length} , } {
\href{/docs/reference/foundations/content/}{content} , }

) -\textgreater{} \href{/docs/reference/foundations/content/}{content}

\subsubsection{\texorpdfstring{\texttt{\ level\ }}{ level }}\label{parameters-level}

\href{/docs/reference/foundations/auto/}{auto} {or}
\href{/docs/reference/foundations/int/}{int}

{{ Settable }}

\phantomsection\label{parameters-level-settable-tooltip}
Settable parameters can be customized for all following uses of the
function with a \texttt{\ set\ } rule.

The absolute nesting depth of the heading, starting from one. If set to
\texttt{\ }{\texttt{\ auto\ }}\texttt{\ } , it is computed from
\texttt{\ offset\ }{\texttt{\ +\ }}\texttt{\ depth\ } .

This is primarily useful for usage in
\href{/docs/reference/styling/\#show-rules}{show rules} (either with
\href{/docs/reference/foundations/function/\#definitions-where}{\texttt{\ where\ }}
selectors or by accessing the level directly on a shown heading).

Default: \texttt{\ }{\texttt{\ auto\ }}\texttt{\ }

\includesvg[width=0.16667in,height=0.16667in]{/assets/icons/16-arrow-right.svg}
View example

\begin{verbatim}
#show heading.where(level: 2): set text(red)

= Level 1
== Level 2

#set heading(offset: 1)
= Also level 2
== Level 3
\end{verbatim}

\includegraphics[width=5in,height=\textheight,keepaspectratio]{/assets/docs/_pDm-P05bg_jGbl9uvGjlAAAAAAAAAAA.png}

\subsubsection{\texorpdfstring{\texttt{\ depth\ }}{ depth }}\label{parameters-depth}

\href{/docs/reference/foundations/int/}{int}

{{ Settable }}

\phantomsection\label{parameters-depth-settable-tooltip}
Settable parameters can be customized for all following uses of the
function with a \texttt{\ set\ } rule.

The relative nesting depth of the heading, starting from one. This is
combined with \texttt{\ offset\ } to compute the actual
\texttt{\ level\ } .

This is set by the heading syntax, such that
\texttt{\ }{\texttt{\ ==\ Heading\ }}\texttt{\ } creates a heading with
logical depth of 2, but actual level
\texttt{\ offset\ }{\texttt{\ +\ }}\texttt{\ }{\texttt{\ 2\ }}\texttt{\ }
. If you construct a heading manually, you should typically prefer this
over setting the absolute level.

Default: \texttt{\ }{\texttt{\ 1\ }}\texttt{\ }

\subsubsection{\texorpdfstring{\texttt{\ offset\ }}{ offset }}\label{parameters-offset}

\href{/docs/reference/foundations/int/}{int}

{{ Settable }}

\phantomsection\label{parameters-offset-settable-tooltip}
Settable parameters can be customized for all following uses of the
function with a \texttt{\ set\ } rule.

The starting offset of each heading\textquotesingle s \texttt{\ level\ }
, used to turn its relative \texttt{\ depth\ } into its absolute
\texttt{\ level\ } .

Default: \texttt{\ }{\texttt{\ 0\ }}\texttt{\ }

\includesvg[width=0.16667in,height=0.16667in]{/assets/icons/16-arrow-right.svg}
View example

\begin{verbatim}
= Level 1

#set heading(offset: 1, numbering: "1.1")
= Level 2

#heading(offset: 2, depth: 2)[
  I'm level 4
]
\end{verbatim}

\includegraphics[width=5in,height=\textheight,keepaspectratio]{/assets/docs/hKtWik5-HwMMejqOwDVKLAAAAAAAAAAA.png}

\subsubsection{\texorpdfstring{\texttt{\ numbering\ }}{ numbering }}\label{parameters-numbering}

\href{/docs/reference/foundations/none/}{none} {or}
\href{/docs/reference/foundations/str/}{str} {or}
\href{/docs/reference/foundations/function/}{function}

{{ Settable }}

\phantomsection\label{parameters-numbering-settable-tooltip}
Settable parameters can be customized for all following uses of the
function with a \texttt{\ set\ } rule.

How to number the heading. Accepts a
\href{/docs/reference/model/numbering/}{numbering pattern or function} .

Default: \texttt{\ }{\texttt{\ none\ }}\texttt{\ }

\includesvg[width=0.16667in,height=0.16667in]{/assets/icons/16-arrow-right.svg}
View example

\begin{verbatim}
#set heading(numbering: "1.a.")

= A section
== A subsection
=== A sub-subsection
\end{verbatim}

\includegraphics[width=5in,height=\textheight,keepaspectratio]{/assets/docs/dtIXlP8zFF4SfNqscPeLbAAAAAAAAAAA.png}

\subsubsection{\texorpdfstring{\texttt{\ supplement\ }}{ supplement }}\label{parameters-supplement}

\href{/docs/reference/foundations/none/}{none} {or}
\href{/docs/reference/foundations/auto/}{auto} {or}
\href{/docs/reference/foundations/content/}{content} {or}
\href{/docs/reference/foundations/function/}{function}

{{ Settable }}

\phantomsection\label{parameters-supplement-settable-tooltip}
Settable parameters can be customized for all following uses of the
function with a \texttt{\ set\ } rule.

A supplement for the heading.

For references to headings, this is added before the referenced number.

If a function is specified, it is passed the referenced heading and
should return content.

Default: \texttt{\ }{\texttt{\ auto\ }}\texttt{\ }

\includesvg[width=0.16667in,height=0.16667in]{/assets/icons/16-arrow-right.svg}
View example

\begin{verbatim}
#set heading(numbering: "1.", supplement: [Chapter])

= Introduction <intro>
In @intro, we see how to turn
Sections into Chapters. And
in @intro[Part], it is done
manually.
\end{verbatim}

\includegraphics[width=5in,height=\textheight,keepaspectratio]{/assets/docs/OZMUTnmWZCt9L0XUTDaRmQAAAAAAAAAA.png}

\subsubsection{\texorpdfstring{\texttt{\ outlined\ }}{ outlined }}\label{parameters-outlined}

\href{/docs/reference/foundations/bool/}{bool}

{{ Settable }}

\phantomsection\label{parameters-outlined-settable-tooltip}
Settable parameters can be customized for all following uses of the
function with a \texttt{\ set\ } rule.

Whether the heading should appear in the
\href{/docs/reference/model/outline/}{outline} .

Note that this property, if set to
\texttt{\ }{\texttt{\ true\ }}\texttt{\ } , ensures the heading is also
shown as a bookmark in the exported PDF\textquotesingle s outline (when
exporting to PDF). To change that behavior, use the
\texttt{\ bookmarked\ } property.

Default: \texttt{\ }{\texttt{\ true\ }}\texttt{\ }

\includesvg[width=0.16667in,height=0.16667in]{/assets/icons/16-arrow-right.svg}
View example

\begin{verbatim}
#outline()

#heading[Normal]
This is a normal heading.

#heading(outlined: false)[Hidden]
This heading does not appear
in the outline.
\end{verbatim}

\includegraphics[width=5in,height=\textheight,keepaspectratio]{/assets/docs/q3R6803Mv9D8hpPx5wD4TgAAAAAAAAAA.png}

\subsubsection{\texorpdfstring{\texttt{\ bookmarked\ }}{ bookmarked }}\label{parameters-bookmarked}

\href{/docs/reference/foundations/auto/}{auto} {or}
\href{/docs/reference/foundations/bool/}{bool}

{{ Settable }}

\phantomsection\label{parameters-bookmarked-settable-tooltip}
Settable parameters can be customized for all following uses of the
function with a \texttt{\ set\ } rule.

Whether the heading should appear as a bookmark in the exported
PDF\textquotesingle s outline. Doesn\textquotesingle t affect other
export formats, such as PNG.

The default value of \texttt{\ }{\texttt{\ auto\ }}\texttt{\ } indicates
that the heading will only appear in the exported PDF\textquotesingle s
outline if its \texttt{\ outlined\ } property is set to
\texttt{\ }{\texttt{\ true\ }}\texttt{\ } , that is, if it would also be
listed in Typst\textquotesingle s
\href{/docs/reference/model/outline/}{outline} . Setting this property
to either \texttt{\ }{\texttt{\ true\ }}\texttt{\ } (bookmark) or
\texttt{\ }{\texttt{\ false\ }}\texttt{\ } (don\textquotesingle t
bookmark) bypasses that behavior.

Default: \texttt{\ }{\texttt{\ auto\ }}\texttt{\ }

\includesvg[width=0.16667in,height=0.16667in]{/assets/icons/16-arrow-right.svg}
View example

\begin{verbatim}
#heading[Normal heading]
This heading will be shown in
the PDF's bookmark outline.

#heading(bookmarked: false)[Not bookmarked]
This heading won't be
bookmarked in the resulting
PDF.
\end{verbatim}

\includegraphics[width=5in,height=\textheight,keepaspectratio]{/assets/docs/_UvMUDZOtTdH4i83Hac2iwAAAAAAAAAA.png}

\subsubsection{\texorpdfstring{\texttt{\ hanging-indent\ }}{ hanging-indent }}\label{parameters-hanging-indent}

\href{/docs/reference/foundations/auto/}{auto} {or}
\href{/docs/reference/layout/length/}{length}

{{ Settable }}

\phantomsection\label{parameters-hanging-indent-settable-tooltip}
Settable parameters can be customized for all following uses of the
function with a \texttt{\ set\ } rule.

The indent all but the first line of a heading should have.

The default value of \texttt{\ }{\texttt{\ auto\ }}\texttt{\ } indicates
that the subsequent heading lines will be indented based on the width of
the numbering.

Default: \texttt{\ }{\texttt{\ auto\ }}\texttt{\ }

\includesvg[width=0.16667in,height=0.16667in]{/assets/icons/16-arrow-right.svg}
View example

\begin{verbatim}
#set heading(numbering: "1.")
#heading[A very, very, very, very, very, very long heading]
\end{verbatim}

\includegraphics[width=5in,height=\textheight,keepaspectratio]{/assets/docs/35Dg34kG-7rg1-RFp8FaIgAAAAAAAAAA.png}

\subsubsection{\texorpdfstring{\texttt{\ body\ }}{ body }}\label{parameters-body}

\href{/docs/reference/foundations/content/}{content}

{Required} {{ Positional }}

\phantomsection\label{parameters-body-positional-tooltip}
Positional parameters are specified in order, without names.

The heading\textquotesingle s title.

\href{/docs/reference/model/footnote/}{\pandocbounded{\includesvg[keepaspectratio]{/assets/icons/16-arrow-right.svg}}}

{ Footnote } { Previous page }

\href{/docs/reference/model/link/}{\pandocbounded{\includesvg[keepaspectratio]{/assets/icons/16-arrow-right.svg}}}

{ Link } { Next page }


\title{typst.app/docs/reference/model/link}

\begin{itemize}
\tightlist
\item
  \href{/docs}{\includesvg[width=0.16667in,height=0.16667in]{/assets/icons/16-docs-dark.svg}}
\item
  \includesvg[width=0.16667in,height=0.16667in]{/assets/icons/16-arrow-right.svg}
\item
  \href{/docs/reference/}{Reference}
\item
  \includesvg[width=0.16667in,height=0.16667in]{/assets/icons/16-arrow-right.svg}
\item
  \href{/docs/reference/model/}{Model}
\item
  \includesvg[width=0.16667in,height=0.16667in]{/assets/icons/16-arrow-right.svg}
\item
  \href{/docs/reference/model/link/}{Link}
\end{itemize}

\section{\texorpdfstring{\texttt{\ link\ } {{ Element
}}}{ link   Element }}\label{summary}

\phantomsection\label{element-tooltip}
Element functions can be customized with \texttt{\ set\ } and
\texttt{\ show\ } rules.

Links to a URL or a location in the document.

By default, links are not styled any different from normal text.
However, you can easily apply a style of your choice with a show rule.

\subsection{Example}\label{example}

\begin{verbatim}
#show link: underline

https://example.com \

#link("https://example.com") \
#link("https://example.com")[
  See example.com
]
\end{verbatim}

\includegraphics[width=5in,height=\textheight,keepaspectratio]{/assets/docs/mBfQJYO4ObjIyuLi_FjKfgAAAAAAAAAA.png}

\subsection{Syntax}\label{syntax}

This function also has dedicated syntax: Text that starts with
\texttt{\ http://\ } or \texttt{\ https://\ } is automatically turned
into a link.

\subsection{\texorpdfstring{{ Parameters
}}{ Parameters }}\label{parameters}

\phantomsection\label{parameters-tooltip}
Parameters are the inputs to a function. They are specified in
parentheses after the function name.

{ link } (

{ \href{/docs/reference/foundations/str/}{str}
\href{/docs/reference/foundations/label/}{label}
\href{/docs/reference/introspection/location/}{location}
\href{/docs/reference/foundations/dictionary/}{dictionary} , } {
\href{/docs/reference/foundations/content/}{content} , }

) -\textgreater{} \href{/docs/reference/foundations/content/}{content}

\subsubsection{\texorpdfstring{\texttt{\ dest\ }}{ dest }}\label{parameters-dest}

\href{/docs/reference/foundations/str/}{str} {or}
\href{/docs/reference/foundations/label/}{label} {or}
\href{/docs/reference/introspection/location/}{location} {or}
\href{/docs/reference/foundations/dictionary/}{dictionary}

{Required} {{ Positional }}

\phantomsection\label{parameters-dest-positional-tooltip}
Positional parameters are specified in order, without names.

The destination the link points to.

\begin{itemize}
\item
  To link to web pages, \texttt{\ dest\ } should be a valid URL string.
  If the URL is in the \texttt{\ mailto:\ } or \texttt{\ tel:\ } scheme
  and the \texttt{\ body\ } parameter is omitted, the email address or
  phone number will be the link\textquotesingle s body, without the
  scheme.
\item
  To link to another part of the document, \texttt{\ dest\ } can take
  one of three forms:

  \begin{itemize}
  \item
    A \href{/docs/reference/foundations/label/}{label} attached to an
    element. If you also want automatic text for the link based on the
    element, consider using a
    \href{/docs/reference/model/ref/}{reference} instead.
  \item
    A
    \href{/docs/reference/introspection/location/}{\texttt{\ location\ }}
    (typically retrieved from
    \href{/docs/reference/introspection/here/}{\texttt{\ here\ }} ,
    \href{/docs/reference/introspection/locate/}{\texttt{\ locate\ }} or
    \href{/docs/reference/introspection/query/}{\texttt{\ query\ }} ).
  \item
    A dictionary with a \texttt{\ page\ } key of type
    \href{/docs/reference/foundations/int/}{integer} and \texttt{\ x\ }
    and \texttt{\ y\ } coordinates of type
    \href{/docs/reference/layout/length/}{length} . Pages are counted
    from one, and the coordinates are relative to the
    page\textquotesingle s top left corner.
  \end{itemize}
\end{itemize}

\includesvg[width=0.16667in,height=0.16667in]{/assets/icons/16-arrow-right.svg}
View example

\begin{verbatim}
= Introduction <intro>
#link("mailto:hello@typst.app") \
#link(<intro>)[Go to intro] \
#link((page: 1, x: 0pt, y: 0pt))[
  Go to top
]
\end{verbatim}

\includegraphics[width=5in,height=\textheight,keepaspectratio]{/assets/docs/r-LwcI2C1K4OtUWhtvg8QgAAAAAAAAAA.png}

\subsubsection{\texorpdfstring{\texttt{\ body\ }}{ body }}\label{parameters-body}

\href{/docs/reference/foundations/content/}{content}

{Required} {{ Positional }}

\phantomsection\label{parameters-body-positional-tooltip}
Positional parameters are specified in order, without names.

The content that should become a link.

If \texttt{\ dest\ } is an URL string, the parameter can be omitted. In
this case, the URL will be shown as the link.

\href{/docs/reference/model/heading/}{\pandocbounded{\includesvg[keepaspectratio]{/assets/icons/16-arrow-right.svg}}}

{ Heading } { Previous page }

\href{/docs/reference/model/enum/}{\pandocbounded{\includesvg[keepaspectratio]{/assets/icons/16-arrow-right.svg}}}

{ Numbered List } { Next page }


\title{typst.app/docs/reference/model/terms}

\begin{itemize}
\tightlist
\item
  \href{/docs}{\includesvg[width=0.16667in,height=0.16667in]{/assets/icons/16-docs-dark.svg}}
\item
  \includesvg[width=0.16667in,height=0.16667in]{/assets/icons/16-arrow-right.svg}
\item
  \href{/docs/reference/}{Reference}
\item
  \includesvg[width=0.16667in,height=0.16667in]{/assets/icons/16-arrow-right.svg}
\item
  \href{/docs/reference/model/}{Model}
\item
  \includesvg[width=0.16667in,height=0.16667in]{/assets/icons/16-arrow-right.svg}
\item
  \href{/docs/reference/model/terms/}{Term List}
\end{itemize}

\section{\texorpdfstring{\texttt{\ terms\ } {{ Element
}}}{ terms   Element }}\label{summary}

\phantomsection\label{element-tooltip}
Element functions can be customized with \texttt{\ set\ } and
\texttt{\ show\ } rules.

A list of terms and their descriptions.

Displays a sequence of terms and their descriptions vertically. When the
descriptions span over multiple lines, they use hanging indent to
communicate the visual hierarchy.

\subsection{Example}\label{example}

\begin{verbatim}
/ Ligature: A merged glyph.
/ Kerning: A spacing adjustment
  between two adjacent letters.
\end{verbatim}

\includegraphics[width=5in,height=\textheight,keepaspectratio]{/assets/docs/qjdQTTJFa_RYtcfu42IiawAAAAAAAAAA.png}

\subsection{Syntax}\label{syntax}

This function also has dedicated syntax: Starting a line with a slash,
followed by a term, a colon and a description creates a term list item.

\subsection{\texorpdfstring{{ Parameters
}}{ Parameters }}\label{parameters}

\phantomsection\label{parameters-tooltip}
Parameters are the inputs to a function. They are specified in
parentheses after the function name.

{ terms } (

{ \hyperref[parameters-tight]{tight :}
\href{/docs/reference/foundations/bool/}{bool} , } {
\hyperref[parameters-separator]{separator :}
\href{/docs/reference/foundations/content/}{content} , } {
\hyperref[parameters-indent]{indent :}
\href{/docs/reference/layout/length/}{length} , } {
\hyperref[parameters-hanging-indent]{hanging-indent :}
\href{/docs/reference/layout/length/}{length} , } {
\hyperref[parameters-spacing]{spacing :}
\href{/docs/reference/foundations/auto/}{auto}
\href{/docs/reference/layout/length/}{length} , } {
\hyperref[parameters-children]{..}
\href{/docs/reference/foundations/content/}{content}
\href{/docs/reference/foundations/array/}{array} , }

) -\textgreater{} \href{/docs/reference/foundations/content/}{content}

\subsubsection{\texorpdfstring{\texttt{\ tight\ }}{ tight }}\label{parameters-tight}

\href{/docs/reference/foundations/bool/}{bool}

{{ Settable }}

\phantomsection\label{parameters-tight-settable-tooltip}
Settable parameters can be customized for all following uses of the
function with a \texttt{\ set\ } rule.

Defines the default
\href{/docs/reference/model/terms/\#parameters-spacing}{spacing} of the
term list. If it is \texttt{\ }{\texttt{\ false\ }}\texttt{\ } , the
items are spaced apart with
\href{/docs/reference/model/par/\#parameters-spacing}{paragraph spacing}
. If it is \texttt{\ }{\texttt{\ true\ }}\texttt{\ } , they use
\href{/docs/reference/model/par/\#parameters-leading}{paragraph leading}
instead. This makes the list more compact, which can look better if the
items are short.

In markup mode, the value of this parameter is determined based on
whether items are separated with a blank line. If items directly follow
each other, this is set to \texttt{\ }{\texttt{\ true\ }}\texttt{\ } ;
if items are separated by a blank line, this is set to
\texttt{\ }{\texttt{\ false\ }}\texttt{\ } . The markup-defined
tightness cannot be overridden with set rules.

Default: \texttt{\ }{\texttt{\ true\ }}\texttt{\ }

\includesvg[width=0.16667in,height=0.16667in]{/assets/icons/16-arrow-right.svg}
View example

\begin{verbatim}
/ Fact: If a term list has a lot
  of text, and maybe other inline
  content, it should not be tight
  anymore.

/ Tip: To make it wide, simply
  insert a blank line between the
  items.
\end{verbatim}

\includegraphics[width=5in,height=\textheight,keepaspectratio]{/assets/docs/skkuR2BgltlCHUy9cPpX7gAAAAAAAAAA.png}

\subsubsection{\texorpdfstring{\texttt{\ separator\ }}{ separator }}\label{parameters-separator}

\href{/docs/reference/foundations/content/}{content}

{{ Settable }}

\phantomsection\label{parameters-separator-settable-tooltip}
Settable parameters can be customized for all following uses of the
function with a \texttt{\ set\ } rule.

The separator between the item and the description.

If you want to just separate them with a certain amount of space, use
\texttt{\ }{\texttt{\ h\ }}\texttt{\ }{\texttt{\ (\ }}\texttt{\ }{\texttt{\ 2cm\ }}\texttt{\ }{\texttt{\ ,\ }}\texttt{\ weak\ }{\texttt{\ :\ }}\texttt{\ }{\texttt{\ true\ }}\texttt{\ }{\texttt{\ )\ }}\texttt{\ }
as the separator and replace \texttt{\ }{\texttt{\ 2cm\ }}\texttt{\ }
with your desired amount of space.

Default:
\texttt{\ }{\texttt{\ h\ }}\texttt{\ }{\texttt{\ (\ }}\texttt{\ amount\ }{\texttt{\ :\ }}\texttt{\ }{\texttt{\ 0.6em\ }}\texttt{\ }{\texttt{\ ,\ }}\texttt{\ weak\ }{\texttt{\ :\ }}\texttt{\ }{\texttt{\ true\ }}\texttt{\ }{\texttt{\ )\ }}\texttt{\ }

\includesvg[width=0.16667in,height=0.16667in]{/assets/icons/16-arrow-right.svg}
View example

\begin{verbatim}
#set terms(separator: [: ])

/ Colon: A nice separator symbol.
\end{verbatim}

\includegraphics[width=5in,height=\textheight,keepaspectratio]{/assets/docs/xyyblMI8l_99lTt1_p5kWgAAAAAAAAAA.png}

\subsubsection{\texorpdfstring{\texttt{\ indent\ }}{ indent }}\label{parameters-indent}

\href{/docs/reference/layout/length/}{length}

{{ Settable }}

\phantomsection\label{parameters-indent-settable-tooltip}
Settable parameters can be customized for all following uses of the
function with a \texttt{\ set\ } rule.

The indentation of each item.

Default: \texttt{\ }{\texttt{\ 0pt\ }}\texttt{\ }

\subsubsection{\texorpdfstring{\texttt{\ hanging-indent\ }}{ hanging-indent }}\label{parameters-hanging-indent}

\href{/docs/reference/layout/length/}{length}

{{ Settable }}

\phantomsection\label{parameters-hanging-indent-settable-tooltip}
Settable parameters can be customized for all following uses of the
function with a \texttt{\ set\ } rule.

The hanging indent of the description.

This is in addition to the whole item\textquotesingle s
\texttt{\ indent\ } .

Default: \texttt{\ }{\texttt{\ 2em\ }}\texttt{\ }

\includesvg[width=0.16667in,height=0.16667in]{/assets/icons/16-arrow-right.svg}
View example

\begin{verbatim}
#set terms(hanging-indent: 0pt)
/ Term: This term list does not
  make use of hanging indents.
\end{verbatim}

\includegraphics[width=5in,height=\textheight,keepaspectratio]{/assets/docs/6yYrKErT2JtRwBRmpS8r5wAAAAAAAAAA.png}

\subsubsection{\texorpdfstring{\texttt{\ spacing\ }}{ spacing }}\label{parameters-spacing}

\href{/docs/reference/foundations/auto/}{auto} {or}
\href{/docs/reference/layout/length/}{length}

{{ Settable }}

\phantomsection\label{parameters-spacing-settable-tooltip}
Settable parameters can be customized for all following uses of the
function with a \texttt{\ set\ } rule.

The spacing between the items of the term list.

If set to \texttt{\ }{\texttt{\ auto\ }}\texttt{\ } , uses paragraph
\href{/docs/reference/model/par/\#parameters-leading}{\texttt{\ leading\ }}
for tight term lists and paragraph
\href{/docs/reference/model/par/\#parameters-spacing}{\texttt{\ spacing\ }}
for wide (non-tight) term lists.

Default: \texttt{\ }{\texttt{\ auto\ }}\texttt{\ }

\subsubsection{\texorpdfstring{\texttt{\ children\ }}{ children }}\label{parameters-children}

\href{/docs/reference/foundations/content/}{content} {or}
\href{/docs/reference/foundations/array/}{array}

{Required} {{ Positional }}

\phantomsection\label{parameters-children-positional-tooltip}
Positional parameters are specified in order, without names.

{{ Variadic }}

\phantomsection\label{parameters-children-variadic-tooltip}
Variadic parameters can be specified multiple times.

The term list\textquotesingle s children.

When using the term list syntax, adjacent items are automatically
collected into term lists, even through constructs like for loops.

\includesvg[width=0.16667in,height=0.16667in]{/assets/icons/16-arrow-right.svg}
View example

\begin{verbatim}
#for (year, product) in (
  "1978": "TeX",
  "1984": "LaTeX",
  "2019": "Typst",
) [/ #product: Born in #year.]
\end{verbatim}

\includegraphics[width=5in,height=\textheight,keepaspectratio]{/assets/docs/wkvQM6jeTkSTRoaT9Y0lSQAAAAAAAAAA.png}

\subsection{\texorpdfstring{{ Definitions
}}{ Definitions }}\label{definitions}

\phantomsection\label{definitions-tooltip}
Functions and types and can have associated definitions. These are
accessed by specifying the function or type, followed by a period, and
then the definition\textquotesingle s name.

\subsubsection{\texorpdfstring{\texttt{\ item\ } {{ Element
}}}{ item   Element }}\label{definitions-item}

\phantomsection\label{definitions-item-element-tooltip}
Element functions can be customized with \texttt{\ set\ } and
\texttt{\ show\ } rules.

A term list item.

terms { . } { item } (

{ \href{/docs/reference/foundations/content/}{content} , } {
\href{/docs/reference/foundations/content/}{content} , }

) -\textgreater{} \href{/docs/reference/foundations/content/}{content}

\paragraph{\texorpdfstring{\texttt{\ term\ }}{ term }}\label{definitions-item-term}

\href{/docs/reference/foundations/content/}{content}

{Required} {{ Positional }}

\phantomsection\label{definitions-item-term-positional-tooltip}
Positional parameters are specified in order, without names.

The term described by the list item.

\paragraph{\texorpdfstring{\texttt{\ description\ }}{ description }}\label{definitions-item-description}

\href{/docs/reference/foundations/content/}{content}

{Required} {{ Positional }}

\phantomsection\label{definitions-item-description-positional-tooltip}
Positional parameters are specified in order, without names.

The description of the term.

\href{/docs/reference/model/table/}{\pandocbounded{\includesvg[keepaspectratio]{/assets/icons/16-arrow-right.svg}}}

{ Table } { Previous page }

\href{/docs/reference/text/}{\pandocbounded{\includesvg[keepaspectratio]{/assets/icons/16-arrow-right.svg}}}

{ Text } { Next page }


\title{typst.app/docs/reference/model/numbering}

\begin{itemize}
\tightlist
\item
  \href{/docs}{\includesvg[width=0.16667in,height=0.16667in]{/assets/icons/16-docs-dark.svg}}
\item
  \includesvg[width=0.16667in,height=0.16667in]{/assets/icons/16-arrow-right.svg}
\item
  \href{/docs/reference/}{Reference}
\item
  \includesvg[width=0.16667in,height=0.16667in]{/assets/icons/16-arrow-right.svg}
\item
  \href{/docs/reference/model/}{Model}
\item
  \includesvg[width=0.16667in,height=0.16667in]{/assets/icons/16-arrow-right.svg}
\item
  \href{/docs/reference/model/numbering/}{Numbering}
\end{itemize}

\section{\texorpdfstring{\texttt{\ numbering\ }}{ numbering }}\label{summary}

Applies a numbering to a sequence of numbers.

A numbering defines how a sequence of numbers should be displayed as
content. It is defined either through a pattern string or an arbitrary
function.

A numbering pattern consists of counting symbols, for which the actual
number is substituted, their prefixes, and one suffix. The prefixes and
the suffix are repeated as-is.

\subsection{Example}\label{example}

\begin{verbatim}
#numbering("1.1)", 1, 2, 3) \
#numbering("1.a.i", 1, 2) \
#numbering("I – 1", 12, 2) \
#numbering(
  (..nums) => nums
    .pos()
    .map(str)
    .join(".") + ")",
  1, 2, 3,
)
\end{verbatim}

\includegraphics[width=5in,height=\textheight,keepaspectratio]{/assets/docs/ViM4jxlRNjTCcZLHAqTQsQAAAAAAAAAA.png}

\subsection{Numbering patterns and numbering
functions}\label{numbering-patterns-and-numbering-functions}

There are multiple instances where you can provide a numbering pattern
or function in Typst. For example, when defining how to number
\href{/docs/reference/model/heading/}{headings} or
\href{/docs/reference/model/figure/}{figures} . Every time, the expected
format is the same as the one described below for the
\href{/docs/reference/model/numbering/\#parameters-numbering}{\texttt{\ numbering\ }}
parameter.

The following example illustrates that a numbering function is just a
regular \href{/docs/reference/foundations/function/}{function} that
accepts numbers and returns
\href{/docs/reference/foundations/content/}{\texttt{\ content\ }} .

\begin{verbatim}
#let unary(.., last) = "|" * last
#set heading(numbering: unary)
= First heading
= Second heading
= Third heading
\end{verbatim}

\includegraphics[width=5in,height=\textheight,keepaspectratio]{/assets/docs/y3Y2xT6PKYJ3nJF6y9bcPwAAAAAAAAAA.png}

\subsection{\texorpdfstring{{ Parameters
}}{ Parameters }}\label{parameters}

\phantomsection\label{parameters-tooltip}
Parameters are the inputs to a function. They are specified in
parentheses after the function name.

{ numbering } (

{ \href{/docs/reference/foundations/str/}{str}
\href{/docs/reference/foundations/function/}{function} , } {
\hyperref[parameters-numbers]{..}
\href{/docs/reference/foundations/int/}{int} , }

) -\textgreater{} { any }

\subsubsection{\texorpdfstring{\texttt{\ numbering\ }}{ numbering }}\label{parameters-numbering}

\href{/docs/reference/foundations/str/}{str} {or}
\href{/docs/reference/foundations/function/}{function}

{Required} {{ Positional }}

\phantomsection\label{parameters-numbering-positional-tooltip}
Positional parameters are specified in order, without names.

Defines how the numbering works.

\textbf{Counting symbols} are \texttt{\ 1\ } , \texttt{\ a\ } ,
\texttt{\ A\ } , \texttt{\ i\ } , \texttt{\ I\ } , \texttt{\ 一\ } ,
\texttt{\ 壹\ } , \texttt{\ �\ } , \texttt{\ �\ } ,
\texttt{\ ã‚¢\ } , \texttt{\ イ\ } , \texttt{\ ×?\ } , \texttt{\ ê°€\ }
, \texttt{\ ㄱ\ } , \texttt{\ *\ } , \texttt{\ â‘\ } , and
\texttt{\ ⓵\ } . They are replaced by the number in the sequence,
preserving the original case.

The \texttt{\ *\ } character means that symbols should be used to count,
in the order of \texttt{\ *\ } , \texttt{\ â€\ } , \texttt{\ ‡\ } ,
\texttt{\ §\ } , \texttt{\ ¶\ } , \texttt{\ ‖\ } . If there are more
than six items, the number is represented using repeated symbols.

\textbf{Suffixes} are all characters after the last counting symbol.
They are repeated as-is at the end of any rendered number.

\textbf{Prefixes} are all characters that are neither counting symbols
nor suffixes. They are repeated as-is at in front of their rendered
equivalent of their counting symbol.

This parameter can also be an arbitrary function that gets each number
as an individual argument. When given a function, the
\texttt{\ numbering\ } function just forwards the arguments to that
function. While this is not particularly useful in itself, it means that
you can just give arbitrary numberings to the \texttt{\ numbering\ }
function without caring whether they are defined as a pattern or
function.

\subsubsection{\texorpdfstring{\texttt{\ numbers\ }}{ numbers }}\label{parameters-numbers}

\href{/docs/reference/foundations/int/}{int}

{Required} {{ Positional }}

\phantomsection\label{parameters-numbers-positional-tooltip}
Positional parameters are specified in order, without names.

{{ Variadic }}

\phantomsection\label{parameters-numbers-variadic-tooltip}
Variadic parameters can be specified multiple times.

The numbers to apply the numbering to. Must be positive.

If \texttt{\ numbering\ } is a pattern and more numbers than counting
symbols are given, the last counting symbol with its prefix is repeated.

\href{/docs/reference/model/enum/}{\pandocbounded{\includesvg[keepaspectratio]{/assets/icons/16-arrow-right.svg}}}

{ Numbered List } { Previous page }

\href{/docs/reference/model/outline/}{\pandocbounded{\includesvg[keepaspectratio]{/assets/icons/16-arrow-right.svg}}}

{ Outline } { Next page }


\title{typst.app/docs/reference/model/document}

\begin{itemize}
\tightlist
\item
  \href{/docs}{\includesvg[width=0.16667in,height=0.16667in]{/assets/icons/16-docs-dark.svg}}
\item
  \includesvg[width=0.16667in,height=0.16667in]{/assets/icons/16-arrow-right.svg}
\item
  \href{/docs/reference/}{Reference}
\item
  \includesvg[width=0.16667in,height=0.16667in]{/assets/icons/16-arrow-right.svg}
\item
  \href{/docs/reference/model/}{Model}
\item
  \includesvg[width=0.16667in,height=0.16667in]{/assets/icons/16-arrow-right.svg}
\item
  \href{/docs/reference/model/document/}{Document}
\end{itemize}

\section{\texorpdfstring{\texttt{\ document\ } {{ Element
}}}{ document   Element }}\label{summary}

\phantomsection\label{element-tooltip}
Element functions can be customized with \texttt{\ set\ } and
\texttt{\ show\ } rules.

The root element of a document and its metadata.

All documents are automatically wrapped in a \texttt{\ document\ }
element. You cannot create a document element yourself. This function is
only used with \href{/docs/reference/styling/\#set-rules}{set rules} to
specify document metadata. Such a set rule must not occur inside of any
layout container.

\begin{verbatim}
#set document(title: [Hello])

This has no visible output, but
embeds metadata into the PDF!
\end{verbatim}

\includegraphics[width=5in,height=\textheight,keepaspectratio]{/assets/docs/g-R4wkXOtFnr5QmDRHynVAAAAAAAAAAA.png}

Note that metadata set with this function is not rendered within the
document. Instead, it is embedded in the compiled PDF file.

\subsection{\texorpdfstring{{ Parameters
}}{ Parameters }}\label{parameters}

\phantomsection\label{parameters-tooltip}
Parameters are the inputs to a function. They are specified in
parentheses after the function name.

{ document } (

{ \hyperref[parameters-title]{title :}
\href{/docs/reference/foundations/none/}{none}
\href{/docs/reference/foundations/content/}{content} , } {
\hyperref[parameters-author]{author :}
\href{/docs/reference/foundations/str/}{str}
\href{/docs/reference/foundations/array/}{array} , } {
\hyperref[parameters-keywords]{keywords :}
\href{/docs/reference/foundations/str/}{str}
\href{/docs/reference/foundations/array/}{array} , } {
\hyperref[parameters-date]{date :}
\href{/docs/reference/foundations/none/}{none}
\href{/docs/reference/foundations/auto/}{auto}
\href{/docs/reference/foundations/datetime/}{datetime} , }

) -\textgreater{} \href{/docs/reference/foundations/content/}{content}

\subsubsection{\texorpdfstring{\texttt{\ title\ }}{ title }}\label{parameters-title}

\href{/docs/reference/foundations/none/}{none} {or}
\href{/docs/reference/foundations/content/}{content}

{{ Settable }}

\phantomsection\label{parameters-title-settable-tooltip}
Settable parameters can be customized for all following uses of the
function with a \texttt{\ set\ } rule.

The document\textquotesingle s title. This is often rendered as the
title of the PDF viewer window.

While this can be arbitrary content, PDF viewers only support plain text
titles, so the conversion might be lossy.

Default: \texttt{\ }{\texttt{\ none\ }}\texttt{\ }

\subsubsection{\texorpdfstring{\texttt{\ author\ }}{ author }}\label{parameters-author}

\href{/docs/reference/foundations/str/}{str} {or}
\href{/docs/reference/foundations/array/}{array}

{{ Settable }}

\phantomsection\label{parameters-author-settable-tooltip}
Settable parameters can be customized for all following uses of the
function with a \texttt{\ set\ } rule.

The document\textquotesingle s authors.

Default:
\texttt{\ }{\texttt{\ (\ }}\texttt{\ }{\texttt{\ )\ }}\texttt{\ }

\subsubsection{\texorpdfstring{\texttt{\ keywords\ }}{ keywords }}\label{parameters-keywords}

\href{/docs/reference/foundations/str/}{str} {or}
\href{/docs/reference/foundations/array/}{array}

{{ Settable }}

\phantomsection\label{parameters-keywords-settable-tooltip}
Settable parameters can be customized for all following uses of the
function with a \texttt{\ set\ } rule.

The document\textquotesingle s keywords.

Default:
\texttt{\ }{\texttt{\ (\ }}\texttt{\ }{\texttt{\ )\ }}\texttt{\ }

\subsubsection{\texorpdfstring{\texttt{\ date\ }}{ date }}\label{parameters-date}

\href{/docs/reference/foundations/none/}{none} {or}
\href{/docs/reference/foundations/auto/}{auto} {or}
\href{/docs/reference/foundations/datetime/}{datetime}

{{ Settable }}

\phantomsection\label{parameters-date-settable-tooltip}
Settable parameters can be customized for all following uses of the
function with a \texttt{\ set\ } rule.

The document\textquotesingle s creation date.

If this is \texttt{\ }{\texttt{\ auto\ }}\texttt{\ } (default), Typst
uses the current date and time. Setting it to
\texttt{\ }{\texttt{\ none\ }}\texttt{\ } prevents Typst from embedding
any creation date into the PDF metadata.

The year component must be at least zero in order to be embedded into a
PDF.

If you want to create byte-by-byte reproducible PDFs, set this to
something other than \texttt{\ }{\texttt{\ auto\ }}\texttt{\ } .

Default: \texttt{\ }{\texttt{\ auto\ }}\texttt{\ }

\href{/docs/reference/model/cite/}{\pandocbounded{\includesvg[keepaspectratio]{/assets/icons/16-arrow-right.svg}}}

{ Cite } { Previous page }

\href{/docs/reference/model/emph/}{\pandocbounded{\includesvg[keepaspectratio]{/assets/icons/16-arrow-right.svg}}}

{ Emphasis } { Next page }


\title{typst.app/docs/reference/model/bibliography}

\begin{itemize}
\tightlist
\item
  \href{/docs}{\includesvg[width=0.16667in,height=0.16667in]{/assets/icons/16-docs-dark.svg}}
\item
  \includesvg[width=0.16667in,height=0.16667in]{/assets/icons/16-arrow-right.svg}
\item
  \href{/docs/reference/}{Reference}
\item
  \includesvg[width=0.16667in,height=0.16667in]{/assets/icons/16-arrow-right.svg}
\item
  \href{/docs/reference/model/}{Model}
\item
  \includesvg[width=0.16667in,height=0.16667in]{/assets/icons/16-arrow-right.svg}
\item
  \href{/docs/reference/model/bibliography/}{Bibliography}
\end{itemize}

\section{\texorpdfstring{\texttt{\ bibliography\ } {{ Element
}}}{ bibliography   Element }}\label{summary}

\phantomsection\label{element-tooltip}
Element functions can be customized with \texttt{\ set\ } and
\texttt{\ show\ } rules.

A bibliography / reference listing.

You can create a new bibliography by calling this function with a path
to a bibliography file in either one of two formats:

\begin{itemize}
\tightlist
\item
  A Hayagriva \texttt{\ .yml\ } file. Hayagriva is a new bibliography
  file format designed for use with Typst. Visit its
  \href{https://github.com/typst/hayagriva/blob/main/docs/file-format.md}{documentation}
  for more details.
\item
  A BibLaTeX \texttt{\ .bib\ } file.
\end{itemize}

As soon as you add a bibliography somewhere in your document, you can
start citing things with reference syntax (
\texttt{\ }{\texttt{\ @key\ }}\texttt{\ } ) or explicit calls to the
\href{/docs/reference/model/cite/}{citation} function (
\texttt{\ }{\texttt{\ \#\ }}\texttt{\ }{\texttt{\ cite\ }}\texttt{\ }{\texttt{\ (\ }}\texttt{\ }{\texttt{\ \textless{}key\textgreater{}\ }}\texttt{\ }{\texttt{\ )\ }}\texttt{\ }
). The bibliography will only show entries for works that were
referenced in the document.

\subsection{Styles}\label{styles}

Typst offers a wide selection of built-in
\href{/docs/reference/model/bibliography/\#parameters-style}{citation
and bibliography styles} . Beyond those, you can add and use custom
\href{https://citationstyles.org/}{CSL} (Citation Style Language) files.
Wondering which style to use? Here are some good defaults based on what
discipline you\textquotesingle re working in:

\begin{longtable}[]{@{}ll@{}}
\toprule\noalign{}
Fields & Typical Styles \\
\midrule\noalign{}
\endhead
\bottomrule\noalign{}
\endlastfoot
Engineering, IT & \texttt{\ }{\texttt{\ "ieee"\ }}\texttt{\ } \\
Psychology, Life Sciences &
\texttt{\ }{\texttt{\ "apa"\ }}\texttt{\ } \\
Social sciences &
\texttt{\ }{\texttt{\ "chicago-author-date"\ }}\texttt{\ } \\
Humanities & \texttt{\ }{\texttt{\ "mla"\ }}\texttt{\ } ,
\texttt{\ }{\texttt{\ "chicago-notes"\ }}\texttt{\ } ,
\texttt{\ }{\texttt{\ "harvard-cite-them-right"\ }}\texttt{\ } \\
Economics &
\texttt{\ }{\texttt{\ "harvard-cite-them-right"\ }}\texttt{\ } \\
Physics &
\texttt{\ }{\texttt{\ "american-physics-society"\ }}\texttt{\ } \\
\end{longtable}

\subsection{Example}\label{example}

\begin{verbatim}
This was already noted by
pirates long ago. @arrgh

Multiple sources say ...
@arrgh @netwok.

#bibliography("works.bib")
\end{verbatim}

\includegraphics[width=5in,height=\textheight,keepaspectratio]{/assets/docs/IJ3xnmEzh6yEddeM44ev3wAAAAAAAAAA.png}

\subsection{\texorpdfstring{{ Parameters
}}{ Parameters }}\label{parameters}

\phantomsection\label{parameters-tooltip}
Parameters are the inputs to a function. They are specified in
parentheses after the function name.

{ bibliography } (

{ \href{/docs/reference/foundations/str/}{str}
\href{/docs/reference/foundations/array/}{array} , } {
\hyperref[parameters-title]{title :}
\href{/docs/reference/foundations/none/}{none}
\href{/docs/reference/foundations/auto/}{auto}
\href{/docs/reference/foundations/content/}{content} , } {
\hyperref[parameters-full]{full :}
\href{/docs/reference/foundations/bool/}{bool} , } {
\hyperref[parameters-style]{style :}
\href{/docs/reference/foundations/str/}{str} , }

) -\textgreater{} \href{/docs/reference/foundations/content/}{content}

\subsubsection{\texorpdfstring{\texttt{\ path\ }}{ path }}\label{parameters-path}

\href{/docs/reference/foundations/str/}{str} {or}
\href{/docs/reference/foundations/array/}{array}

{Required} {{ Positional }}

\phantomsection\label{parameters-path-positional-tooltip}
Positional parameters are specified in order, without names.

Path(s) to Hayagriva \texttt{\ .yml\ } and/or BibLaTeX \texttt{\ .bib\ }
files.

\subsubsection{\texorpdfstring{\texttt{\ title\ }}{ title }}\label{parameters-title}

\href{/docs/reference/foundations/none/}{none} {or}
\href{/docs/reference/foundations/auto/}{auto} {or}
\href{/docs/reference/foundations/content/}{content}

{{ Settable }}

\phantomsection\label{parameters-title-settable-tooltip}
Settable parameters can be customized for all following uses of the
function with a \texttt{\ set\ } rule.

The title of the bibliography.

\begin{itemize}
\tightlist
\item
  When set to \texttt{\ }{\texttt{\ auto\ }}\texttt{\ } , an appropriate
  title for the \href{/docs/reference/text/text/\#parameters-lang}{text
  language} will be used. This is the default.
\item
  When set to \texttt{\ }{\texttt{\ none\ }}\texttt{\ } , the
  bibliography will not have a title.
\item
  A custom title can be set by passing content.
\end{itemize}

The bibliography\textquotesingle s heading will not be numbered by
default, but you can force it to be with a show-set rule:
\texttt{\ }{\texttt{\ show\ }}\texttt{\ }{\texttt{\ bibliography\ }}\texttt{\ }{\texttt{\ :\ }}\texttt{\ }{\texttt{\ set\ }}\texttt{\ }{\texttt{\ heading\ }}\texttt{\ }{\texttt{\ (\ }}\texttt{\ numbering\ }{\texttt{\ :\ }}\texttt{\ }{\texttt{\ "1."\ }}\texttt{\ }{\texttt{\ )\ }}\texttt{\ }

Default: \texttt{\ }{\texttt{\ auto\ }}\texttt{\ }

\subsubsection{\texorpdfstring{\texttt{\ full\ }}{ full }}\label{parameters-full}

\href{/docs/reference/foundations/bool/}{bool}

{{ Settable }}

\phantomsection\label{parameters-full-settable-tooltip}
Settable parameters can be customized for all following uses of the
function with a \texttt{\ set\ } rule.

Whether to include all works from the given bibliography files, even
those that weren\textquotesingle t cited in the document.

To selectively add individual cited works without showing them, you can
also use the \texttt{\ cite\ } function with
\href{/docs/reference/model/cite/\#parameters-form}{\texttt{\ form\ }}
set to \texttt{\ }{\texttt{\ none\ }}\texttt{\ } .

Default: \texttt{\ }{\texttt{\ false\ }}\texttt{\ }

\subsubsection{\texorpdfstring{\texttt{\ style\ }}{ style }}\label{parameters-style}

\href{/docs/reference/foundations/str/}{str}

{{ Settable }}

\phantomsection\label{parameters-style-settable-tooltip}
Settable parameters can be customized for all following uses of the
function with a \texttt{\ set\ } rule.

The bibliography style.

Should be either one of the built-in styles (see below) or a path to a
\href{https://citationstyles.org/}{CSL file} . Some of the styles listed
below appear twice, once with their full name and once with a short
alias.

\includesvg[width=0.16667in,height=0.16667in]{/assets/icons/16-arrow-right.svg}
View options

\begin{longtable}[]{@{}ll@{}}
\toprule\noalign{}
Variant & Details \\
\midrule\noalign{}
\endhead
\bottomrule\noalign{}
\endlastfoot
\texttt{\ "\ alphanumeric\ "\ } & Alphanumeric \\
\texttt{\ "\ american-anthropological-association\ "\ } & American
Anthropological Association \\
\texttt{\ "\ american-chemical-society\ "\ } & American Chemical
Society \\
\texttt{\ "\ american-geophysical-union\ "\ } & American Geophysical
Union \\
\texttt{\ "\ american-institute-of-aeronautics-and-astronautics\ "\ } &
American Institute of Aeronautics and Astronautics \\
\texttt{\ "\ american-institute-of-physics\ "\ } & American Institute of
Physics 4th edition \\
\texttt{\ "\ american-medical-association\ "\ } & American Medical
Association 11th edition \\
\texttt{\ "\ american-meteorological-society\ "\ } & American
Meteorological Society \\
\texttt{\ "\ american-physics-society\ "\ } & American Physical
Society \\
\texttt{\ "\ american-physiological-society\ "\ } & American
Physiological Society \\
\texttt{\ "\ american-political-science-association\ "\ } & American
Political Science Association \\
\texttt{\ "\ american-psychological-association\ "\ } & American
Psychological Association 7th edition \\
\texttt{\ "\ american-society-for-microbiology\ "\ } & American Society
for Microbiology \\
\texttt{\ "\ american-society-of-civil-engineers\ "\ } & American
Society of Civil Engineers \\
\texttt{\ "\ american-society-of-mechanical-engineers\ "\ } & American
Society of Mechanical Engineers \\
\texttt{\ "\ american-sociological-association\ "\ } & American
Sociological Association 6th/7th edition \\
\texttt{\ "\ angewandte-chemie\ "\ } & Angewandte Chemie International
Edition \\
\texttt{\ "\ annual-reviews\ "\ } & Annual Reviews (sorted by order of
appearance) \\
\texttt{\ "\ annual-reviews-author-date\ "\ } & Annual Reviews
(author-date) \\
\texttt{\ "\ associacao-brasileira-de-normas-tecnicas\ "\ } &
Associação Brasileira de Normas Técnicas (Português - Brasil) \\
\texttt{\ "\ association-for-computing-machinery\ "\ } & Association for
Computing Machinery \\
\texttt{\ "\ biomed-central\ "\ } & BioMed Central \\
\texttt{\ "\ bristol-university-press\ "\ } & Bristol University
Press \\
\texttt{\ "\ british-medical-journal\ "\ } & BMJ \\
\texttt{\ "\ cell\ "\ } & Cell \\
\texttt{\ "\ chicago-author-date\ "\ } & Chicago Manual of Style 17th
edition (author-date) \\
\texttt{\ "\ chicago-fullnotes\ "\ } & Chicago Manual of Style 17th
edition (full note) \\
\texttt{\ "\ chicago-notes\ "\ } & Chicago Manual of Style 17th edition
(note) \\
\texttt{\ "\ copernicus\ "\ } & Copernicus Publications \\
\texttt{\ "\ council-of-science-editors\ "\ } & Council of Science
Editors, Citation-Sequence (numeric, brackets) \\
\texttt{\ "\ council-of-science-editors-author-date\ "\ } & Council of
Science Editors, Name-Year (author-date) \\
\texttt{\ "\ current-opinion\ "\ } & Current Opinion journals \\
\texttt{\ "\ deutsche-gesellschaft-für-psychologie\ "\ } & Deutsche
Gesellschaft für Psychologie 5. Auflage (Deutsch) \\
\texttt{\ "\ deutsche-sprache\ "\ } & Deutsche Sprache (Deutsch) \\
\texttt{\ "\ elsevier-harvard\ "\ } & Elsevier - Harvard (with
titles) \\
\texttt{\ "\ elsevier-vancouver\ "\ } & Elsevier - Vancouver \\
\texttt{\ "\ elsevier-with-titles\ "\ } & Elsevier (numeric, with
titles) \\
\texttt{\ "\ frontiers\ "\ } & Frontiers journals \\
\texttt{\ "\ future-medicine\ "\ } & Future Medicine journals \\
\texttt{\ "\ future-science\ "\ } & Future Science Group \\
\texttt{\ "\ gb-7714-2005-numeric\ "\ } & China National Standard GB/T
7714-2005 (numeric, 中æ--‡) \\
\texttt{\ "\ gb-7714-2015-author-date\ "\ } & China National Standard
GB/T 7714-2015 (author-date, 中æ--‡) \\
\texttt{\ "\ gb-7714-2015-note\ "\ } & China National Standard GB/T
7714-2015 (note, 中æ--‡) \\
\texttt{\ "\ gb-7714-2015-numeric\ "\ } & China National Standard GB/T
7714-2015 (numeric, 中æ--‡) \\
\texttt{\ "\ gost-r-705-2008-numeric\ "\ } & Russian GOST R 7.0.5-2008
(numeric) \\
\texttt{\ "\ harvard-cite-them-right\ "\ } & Cite Them Right 12th
edition - Harvard \\
\texttt{\ "\ institute-of-electrical-and-electronics-engineers\ "\ } &
IEEE \\
\texttt{\ "\ institute-of-physics-numeric\ "\ } & Institute of Physics
(numeric) \\
\texttt{\ "\ iso-690-author-date\ "\ } & ISO-690 (author-date,
English) \\
\texttt{\ "\ iso-690-numeric\ "\ } & ISO-690 (numeric, English) \\
\texttt{\ "\ karger\ "\ } & Karger journals \\
\texttt{\ "\ mary-ann-liebert-vancouver\ "\ } & Mary Ann Liebert -
Vancouver \\
\texttt{\ "\ modern-humanities-research-association\ "\ } & Modern
Humanities Research Association 4th edition (note with bibliography) \\
\texttt{\ "\ modern-language-association\ "\ } & Modern Language
Association 9th edition \\
\texttt{\ "\ modern-language-association-8\ "\ } & Modern Language
Association 8th edition \\
\texttt{\ "\ multidisciplinary-digital-publishing-institute\ "\ } &
Multidisciplinary Digital Publishing Institute \\
\texttt{\ "\ nature\ "\ } & Nature \\
\texttt{\ "\ pensoft\ "\ } & Pensoft Journals \\
\texttt{\ "\ public-library-of-science\ "\ } & Public Library of
Science \\
\texttt{\ "\ royal-society-of-chemistry\ "\ } & Royal Society of
Chemistry \\
\texttt{\ "\ sage-vancouver\ "\ } & SAGE - Vancouver \\
\texttt{\ "\ sist02\ "\ } & SIST02 (æ---¥æœ¬èªž) \\
\texttt{\ "\ spie\ "\ } & SPIE journals \\
\texttt{\ "\ springer-basic\ "\ } & Springer - Basic (numeric,
brackets) \\
\texttt{\ "\ springer-basic-author-date\ "\ } & Springer - Basic
(author-date) \\
\texttt{\ "\ springer-fachzeitschriften-medizin-psychologie\ "\ } &
Springer - Fachzeitschriften Medizin Psychologie (Deutsch) \\
\texttt{\ "\ springer-humanities-author-date\ "\ } & Springer -
Humanities (author-date) \\
\texttt{\ "\ springer-lecture-notes-in-computer-science\ "\ } & Springer
- Lecture Notes in Computer Science \\
\texttt{\ "\ springer-mathphys\ "\ } & Springer - MathPhys (numeric,
brackets) \\
\texttt{\ "\ springer-socpsych-author-date\ "\ } & Springer - SocPsych
(author-date) \\
\texttt{\ "\ springer-vancouver\ "\ } & Springer - Vancouver
(brackets) \\
\texttt{\ "\ taylor-and-francis-chicago-author-date\ "\ } & Taylor \&
Francis - Chicago Manual of Style (author-date) \\
\texttt{\ "\ taylor-and-francis-national-library-of-medicine\ "\ } &
Taylor \& Francis - National Library of Medicine \\
\texttt{\ "\ the-institution-of-engineering-and-technology\ "\ } & The
Institution of Engineering and Technology \\
\texttt{\ "\ the-lancet\ "\ } & The Lancet \\
\texttt{\ "\ thieme\ "\ } & Thieme-German (Deutsch) \\
\texttt{\ "\ trends\ "\ } & Trends journals \\
\texttt{\ "\ turabian-author-date\ "\ } & Turabian 9th edition
(author-date) \\
\texttt{\ "\ turabian-fullnote-8\ "\ } & Turabian 8th edition (full
note) \\
\texttt{\ "\ vancouver\ "\ } & Vancouver \\
\texttt{\ "\ vancouver-superscript\ "\ } & Vancouver (superscript) \\
\end{longtable}

Default: \texttt{\ }{\texttt{\ "ieee"\ }}\texttt{\ }

\href{/docs/reference/model/}{\pandocbounded{\includesvg[keepaspectratio]{/assets/icons/16-arrow-right.svg}}}

{ Model } { Previous page }

\href{/docs/reference/model/list/}{\pandocbounded{\includesvg[keepaspectratio]{/assets/icons/16-arrow-right.svg}}}

{ Bullet List } { Next page }


\title{typst.app/docs/reference/model/ref}

\begin{itemize}
\tightlist
\item
  \href{/docs}{\includesvg[width=0.16667in,height=0.16667in]{/assets/icons/16-docs-dark.svg}}
\item
  \includesvg[width=0.16667in,height=0.16667in]{/assets/icons/16-arrow-right.svg}
\item
  \href{/docs/reference/}{Reference}
\item
  \includesvg[width=0.16667in,height=0.16667in]{/assets/icons/16-arrow-right.svg}
\item
  \href{/docs/reference/model/}{Model}
\item
  \includesvg[width=0.16667in,height=0.16667in]{/assets/icons/16-arrow-right.svg}
\item
  \href{/docs/reference/model/ref/}{Reference}
\end{itemize}

\section{\texorpdfstring{\texttt{\ ref\ } {{ Element
}}}{ ref   Element }}\label{summary}

\phantomsection\label{element-tooltip}
Element functions can be customized with \texttt{\ set\ } and
\texttt{\ show\ } rules.

A reference to a label or bibliography.

Produces a textual reference to a label. For example, a reference to a
heading will yield an appropriate string such as "Section 1" for a
reference to the first heading. The references are also links to the
respective element. Reference syntax can also be used to
\href{/docs/reference/model/cite/}{cite} from a bibliography.

Referenceable elements include
\href{/docs/reference/model/heading/}{headings} ,
\href{/docs/reference/model/figure/}{figures} ,
\href{/docs/reference/math/equation/}{equations} , and
\href{/docs/reference/model/footnote/}{footnotes} . To create a custom
referenceable element like a theorem, you can create a figure of a
custom
\href{/docs/reference/model/figure/\#parameters-kind}{\texttt{\ kind\ }}
and write a show rule for it. In the future, there might be a more
direct way to define a custom referenceable element.

If you just want to link to a labelled element and not get an automatic
textual reference, consider using the
\href{/docs/reference/model/link/}{\texttt{\ link\ }} function instead.

\subsection{Example}\label{example}

\begin{verbatim}
#set heading(numbering: "1.")
#set math.equation(numbering: "(1)")

= Introduction <intro>
Recent developments in
typesetting software have
rekindled hope in previously
frustrated researchers. @distress
As shown in @results, we ...

= Results <results>
We discuss our approach in
comparison with others.

== Performance <perf>
@slow demonstrates what slow
software looks like.
$ T(n) = O(2^n) $ <slow>

#bibliography("works.bib")
\end{verbatim}

\includegraphics[width=5in,height=\textheight,keepaspectratio]{/assets/docs/bzf3klNJ674BqVarCEGU8wAAAAAAAAAA.png}

\subsection{Syntax}\label{syntax}

This function also has dedicated syntax: A reference to a label can be
created by typing an \texttt{\ @\ } followed by the name of the label
(e.g.
\texttt{\ }{\texttt{\ =\ Introduction\ }}\texttt{\ }{\texttt{\ \textless{}intro\textgreater{}\ }}\texttt{\ }
can be referenced by typing \texttt{\ }{\texttt{\ @intro\ }}\texttt{\ }
).

To customize the supplement, add content in square brackets after the
reference:
\texttt{\ }{\texttt{\ @intro\ }{\texttt{\ {[}\ }}\texttt{\ Chapter\ }{\texttt{\ {]}\ }}\texttt{\ }}\texttt{\ }
.

\subsection{Customization}\label{customization}

If you write a show rule for references, you can access the referenced
element through the \texttt{\ element\ } field of the reference. The
\texttt{\ element\ } may be \texttt{\ }{\texttt{\ none\ }}\texttt{\ }
even if it exists if Typst hasn\textquotesingle t discovered it yet, so
you always need to handle that case in your code.

\begin{verbatim}
#set heading(numbering: "1.")
#set math.equation(numbering: "(1)")

#show ref: it => {
  let eq = math.equation
  let el = it.element
  if el != none and el.func() == eq {
    // Override equation references.
    link(el.location(),numbering(
      el.numbering,
      ..counter(eq).at(el.location())
    ))
  } else {
    // Other references as usual.
    it
  }
}

= Beginnings <beginning>
In @beginning we prove @pythagoras.
$ a^2 + b^2 = c^2 $ <pythagoras>
\end{verbatim}

\includegraphics[width=5in,height=\textheight,keepaspectratio]{/assets/docs/_2kRnAjhpZZ-kvJsytflygAAAAAAAAAA.png}

\subsection{\texorpdfstring{{ Parameters
}}{ Parameters }}\label{parameters}

\phantomsection\label{parameters-tooltip}
Parameters are the inputs to a function. They are specified in
parentheses after the function name.

{ ref } (

{ \href{/docs/reference/foundations/label/}{label} , } {
\hyperref[parameters-supplement]{supplement :}
\href{/docs/reference/foundations/none/}{none}
\href{/docs/reference/foundations/auto/}{auto}
\href{/docs/reference/foundations/content/}{content}
\href{/docs/reference/foundations/function/}{function} , }

) -\textgreater{} \href{/docs/reference/foundations/content/}{content}

\subsubsection{\texorpdfstring{\texttt{\ target\ }}{ target }}\label{parameters-target}

\href{/docs/reference/foundations/label/}{label}

{Required} {{ Positional }}

\phantomsection\label{parameters-target-positional-tooltip}
Positional parameters are specified in order, without names.

The target label that should be referenced.

Can be a label that is defined in the document or an entry from the
\href{/docs/reference/model/bibliography/}{\texttt{\ bibliography\ }} .

\subsubsection{\texorpdfstring{\texttt{\ supplement\ }}{ supplement }}\label{parameters-supplement}

\href{/docs/reference/foundations/none/}{none} {or}
\href{/docs/reference/foundations/auto/}{auto} {or}
\href{/docs/reference/foundations/content/}{content} {or}
\href{/docs/reference/foundations/function/}{function}

{{ Settable }}

\phantomsection\label{parameters-supplement-settable-tooltip}
Settable parameters can be customized for all following uses of the
function with a \texttt{\ set\ } rule.

A supplement for the reference.

For references to headings or figures, this is added before the
referenced number. For citations, this can be used to add a page number.

If a function is specified, it is passed the referenced element and
should return content.

Default: \texttt{\ }{\texttt{\ auto\ }}\texttt{\ }

\includesvg[width=0.16667in,height=0.16667in]{/assets/icons/16-arrow-right.svg}
View example

\begin{verbatim}
#set heading(numbering: "1.")
#set ref(supplement: it => {
  if it.func() == heading {
    "Chapter"
  } else {
    "Thing"
  }
})

= Introduction <intro>
In @intro, we see how to turn
Sections into Chapters. And
in @intro[Part], it is done
manually.
\end{verbatim}

\includegraphics[width=5in,height=\textheight,keepaspectratio]{/assets/docs/fh477CUxS1KmPvq1dqsQ5QAAAAAAAAAA.png}

\href{/docs/reference/model/quote/}{\pandocbounded{\includesvg[keepaspectratio]{/assets/icons/16-arrow-right.svg}}}

{ Quote } { Previous page }

\href{/docs/reference/model/strong/}{\pandocbounded{\includesvg[keepaspectratio]{/assets/icons/16-arrow-right.svg}}}

{ Strong Emphasis } { Next page }


\title{typst.app/docs/reference/model/table}

\begin{itemize}
\tightlist
\item
  \href{/docs}{\includesvg[width=0.16667in,height=0.16667in]{/assets/icons/16-docs-dark.svg}}
\item
  \includesvg[width=0.16667in,height=0.16667in]{/assets/icons/16-arrow-right.svg}
\item
  \href{/docs/reference/}{Reference}
\item
  \includesvg[width=0.16667in,height=0.16667in]{/assets/icons/16-arrow-right.svg}
\item
  \href{/docs/reference/model/}{Model}
\item
  \includesvg[width=0.16667in,height=0.16667in]{/assets/icons/16-arrow-right.svg}
\item
  \href{/docs/reference/model/table/}{Table}
\end{itemize}

\section{\texorpdfstring{\texttt{\ table\ } {{ Element
}}}{ table   Element }}\label{summary}

\phantomsection\label{element-tooltip}
Element functions can be customized with \texttt{\ set\ } and
\texttt{\ show\ } rules.

A table of items.

Tables are used to arrange content in cells. Cells can contain arbitrary
content, including multiple paragraphs and are specified in row-major
order. For a hands-on explanation of all the ways you can use and
customize tables in Typst, check out the
\href{/docs/guides/table-guide/}{table guide} .

Because tables are just grids with different defaults for some cell
properties (notably \texttt{\ stroke\ } and \texttt{\ inset\ } ), refer
to the \href{/docs/reference/layout/grid/}{grid documentation} for more
information on how to size the table tracks and specify the cell
appearance properties.

If you are unsure whether you should be using a table or a grid,
consider whether the content you are arranging semantically belongs
together as a set of related data points or similar or whether you are
just want to enhance your presentation by arranging unrelated content in
a grid. In the former case, a table is the right choice, while in the
latter case, a grid is more appropriate. Furthermore, Typst will
annotate its output in the future such that screenreaders will announce
content in \texttt{\ table\ } as tabular while a grid\textquotesingle s
content will be announced no different than multiple content blocks in
the document flow.

Note that, to override a particular cell\textquotesingle s properties or
apply show rules on table cells, you can use the
\href{/docs/reference/model/table/\#definitions-cell}{\texttt{\ table.cell\ }}
element. See its documentation for more information.

Although the \texttt{\ table\ } and the \texttt{\ grid\ } share most
properties, set and show rules on one of them do not affect the other.

To give a table a caption and make it
\href{/docs/reference/model/ref/}{referenceable} , put it into a
\href{/docs/reference/model/figure/}{figure} .

\subsection{Example}\label{example}

The example below demonstrates some of the most common table options.

\begin{verbatim}
#table(
  columns: (1fr, auto, auto),
  inset: 10pt,
  align: horizon,
  table.header(
    [], [*Volume*], [*Parameters*],
  ),
  image("cylinder.svg"),
  $ pi h (D^2 - d^2) / 4 $,
  [
    $h$: height \
    $D$: outer radius \
    $d$: inner radius
  ],
  image("tetrahedron.svg"),
  $ sqrt(2) / 12 a^3 $,
  [$a$: edge length]
)
\end{verbatim}

\includegraphics[width=5in,height=\textheight,keepaspectratio]{/assets/docs/KSzjBsOqtudzwvK6Zvp9uwAAAAAAAAAA.png}

Much like with grids, you can use
\href{/docs/reference/model/table/\#definitions-cell}{\texttt{\ table.cell\ }}
to customize the appearance and the position of each cell.

\begin{verbatim}
#set table(
  stroke: none,
  gutter: 0.2em,
  fill: (x, y) =>
    if x == 0 or y == 0 { gray },
  inset: (right: 1.5em),
)

#show table.cell: it => {
  if it.x == 0 or it.y == 0 {
    set text(white)
    strong(it)
  } else if it.body == [] {
    // Replace empty cells with 'N/A'
    pad(..it.inset)[_N/A_]
  } else {
    it
  }
}

#let a = table.cell(
  fill: green.lighten(60%),
)[A]
#let b = table.cell(
  fill: aqua.lighten(60%),
)[B]

#table(
  columns: 4,
  [], [Exam 1], [Exam 2], [Exam 3],

  [John], [], a, [],
  [Mary], [], a, a,
  [Robert], b, a, b,
)
\end{verbatim}

\includegraphics[width=5.66667in,height=\textheight,keepaspectratio]{/assets/docs/D_wYQ9Nqm8ZPq6ssgJwiZQAAAAAAAAAA.png}

\subsection{\texorpdfstring{{ Parameters
}}{ Parameters }}\label{parameters}

\phantomsection\label{parameters-tooltip}
Parameters are the inputs to a function. They are specified in
parentheses after the function name.

{ table } (

{ \hyperref[parameters-columns]{columns :}
\href{/docs/reference/foundations/auto/}{auto}
\href{/docs/reference/foundations/int/}{int}
\href{/docs/reference/layout/relative/}{relative}
\href{/docs/reference/layout/fraction/}{fraction}
\href{/docs/reference/foundations/array/}{array} , } {
\hyperref[parameters-rows]{rows :}
\href{/docs/reference/foundations/auto/}{auto}
\href{/docs/reference/foundations/int/}{int}
\href{/docs/reference/layout/relative/}{relative}
\href{/docs/reference/layout/fraction/}{fraction}
\href{/docs/reference/foundations/array/}{array} , } {
\hyperref[parameters-gutter]{gutter :}
\href{/docs/reference/foundations/auto/}{auto}
\href{/docs/reference/foundations/int/}{int}
\href{/docs/reference/layout/relative/}{relative}
\href{/docs/reference/layout/fraction/}{fraction}
\href{/docs/reference/foundations/array/}{array} , } {
\hyperref[parameters-column-gutter]{column-gutter :}
\href{/docs/reference/foundations/auto/}{auto}
\href{/docs/reference/foundations/int/}{int}
\href{/docs/reference/layout/relative/}{relative}
\href{/docs/reference/layout/fraction/}{fraction}
\href{/docs/reference/foundations/array/}{array} , } {
\hyperref[parameters-row-gutter]{row-gutter :}
\href{/docs/reference/foundations/auto/}{auto}
\href{/docs/reference/foundations/int/}{int}
\href{/docs/reference/layout/relative/}{relative}
\href{/docs/reference/layout/fraction/}{fraction}
\href{/docs/reference/foundations/array/}{array} , } {
\hyperref[parameters-fill]{fill :}
\href{/docs/reference/foundations/none/}{none}
\href{/docs/reference/visualize/color/}{color}
\href{/docs/reference/visualize/gradient/}{gradient}
\href{/docs/reference/foundations/array/}{array}
\href{/docs/reference/visualize/pattern/}{pattern}
\href{/docs/reference/foundations/function/}{function} , } {
\hyperref[parameters-align]{align :}
\href{/docs/reference/foundations/auto/}{auto}
\href{/docs/reference/foundations/array/}{array}
\href{/docs/reference/layout/alignment/}{alignment}
\href{/docs/reference/foundations/function/}{function} , } {
\hyperref[parameters-stroke]{stroke :}
\href{/docs/reference/foundations/none/}{none}
\href{/docs/reference/layout/length/}{length}
\href{/docs/reference/visualize/color/}{color}
\href{/docs/reference/visualize/gradient/}{gradient}
\href{/docs/reference/foundations/array/}{array}
\href{/docs/reference/visualize/stroke/}{stroke}
\href{/docs/reference/visualize/pattern/}{pattern}
\href{/docs/reference/foundations/dictionary/}{dictionary}
\href{/docs/reference/foundations/function/}{function} , } {
\hyperref[parameters-inset]{inset :}
\href{/docs/reference/layout/relative/}{relative}
\href{/docs/reference/foundations/array/}{array}
\href{/docs/reference/foundations/dictionary/}{dictionary}
\href{/docs/reference/foundations/function/}{function} , } {
\hyperref[parameters-children]{..}
\href{/docs/reference/foundations/content/}{content} , }

) -\textgreater{} \href{/docs/reference/foundations/content/}{content}

\subsubsection{\texorpdfstring{\texttt{\ columns\ }}{ columns }}\label{parameters-columns}

\href{/docs/reference/foundations/auto/}{auto} {or}
\href{/docs/reference/foundations/int/}{int} {or}
\href{/docs/reference/layout/relative/}{relative} {or}
\href{/docs/reference/layout/fraction/}{fraction} {or}
\href{/docs/reference/foundations/array/}{array}

{{ Settable }}

\phantomsection\label{parameters-columns-settable-tooltip}
Settable parameters can be customized for all following uses of the
function with a \texttt{\ set\ } rule.

The column sizes. See the \href{/docs/reference/layout/grid/}{grid
documentation} for more information on track sizing.

Default:
\texttt{\ }{\texttt{\ (\ }}\texttt{\ }{\texttt{\ )\ }}\texttt{\ }

\subsubsection{\texorpdfstring{\texttt{\ rows\ }}{ rows }}\label{parameters-rows}

\href{/docs/reference/foundations/auto/}{auto} {or}
\href{/docs/reference/foundations/int/}{int} {or}
\href{/docs/reference/layout/relative/}{relative} {or}
\href{/docs/reference/layout/fraction/}{fraction} {or}
\href{/docs/reference/foundations/array/}{array}

{{ Settable }}

\phantomsection\label{parameters-rows-settable-tooltip}
Settable parameters can be customized for all following uses of the
function with a \texttt{\ set\ } rule.

The row sizes. See the \href{/docs/reference/layout/grid/}{grid
documentation} for more information on track sizing.

Default:
\texttt{\ }{\texttt{\ (\ }}\texttt{\ }{\texttt{\ )\ }}\texttt{\ }

\subsubsection{\texorpdfstring{\texttt{\ gutter\ }}{ gutter }}\label{parameters-gutter}

\href{/docs/reference/foundations/auto/}{auto} {or}
\href{/docs/reference/foundations/int/}{int} {or}
\href{/docs/reference/layout/relative/}{relative} {or}
\href{/docs/reference/layout/fraction/}{fraction} {or}
\href{/docs/reference/foundations/array/}{array}

{{ Settable }}

\phantomsection\label{parameters-gutter-settable-tooltip}
Settable parameters can be customized for all following uses of the
function with a \texttt{\ set\ } rule.

The gaps between rows and columns. This is a shorthand for setting
\texttt{\ column-gutter\ } and \texttt{\ row-gutter\ } to the same
value. See the \href{/docs/reference/layout/grid/}{grid documentation}
for more information on gutters.

Default:
\texttt{\ }{\texttt{\ (\ }}\texttt{\ }{\texttt{\ )\ }}\texttt{\ }

\subsubsection{\texorpdfstring{\texttt{\ column-gutter\ }}{ column-gutter }}\label{parameters-column-gutter}

\href{/docs/reference/foundations/auto/}{auto} {or}
\href{/docs/reference/foundations/int/}{int} {or}
\href{/docs/reference/layout/relative/}{relative} {or}
\href{/docs/reference/layout/fraction/}{fraction} {or}
\href{/docs/reference/foundations/array/}{array}

{{ Settable }}

\phantomsection\label{parameters-column-gutter-settable-tooltip}
Settable parameters can be customized for all following uses of the
function with a \texttt{\ set\ } rule.

The gaps between columns. Takes precedence over \texttt{\ gutter\ } .
See the \href{/docs/reference/layout/grid/}{grid documentation} for more
information on gutters.

Default:
\texttt{\ }{\texttt{\ (\ }}\texttt{\ }{\texttt{\ )\ }}\texttt{\ }

\subsubsection{\texorpdfstring{\texttt{\ row-gutter\ }}{ row-gutter }}\label{parameters-row-gutter}

\href{/docs/reference/foundations/auto/}{auto} {or}
\href{/docs/reference/foundations/int/}{int} {or}
\href{/docs/reference/layout/relative/}{relative} {or}
\href{/docs/reference/layout/fraction/}{fraction} {or}
\href{/docs/reference/foundations/array/}{array}

{{ Settable }}

\phantomsection\label{parameters-row-gutter-settable-tooltip}
Settable parameters can be customized for all following uses of the
function with a \texttt{\ set\ } rule.

The gaps between rows. Takes precedence over \texttt{\ gutter\ } . See
the \href{/docs/reference/layout/grid/}{grid documentation} for more
information on gutters.

Default:
\texttt{\ }{\texttt{\ (\ }}\texttt{\ }{\texttt{\ )\ }}\texttt{\ }

\subsubsection{\texorpdfstring{\texttt{\ fill\ }}{ fill }}\label{parameters-fill}

\href{/docs/reference/foundations/none/}{none} {or}
\href{/docs/reference/visualize/color/}{color} {or}
\href{/docs/reference/visualize/gradient/}{gradient} {or}
\href{/docs/reference/foundations/array/}{array} {or}
\href{/docs/reference/visualize/pattern/}{pattern} {or}
\href{/docs/reference/foundations/function/}{function}

{{ Settable }}

\phantomsection\label{parameters-fill-settable-tooltip}
Settable parameters can be customized for all following uses of the
function with a \texttt{\ set\ } rule.

How to fill the cells.

This can be a color or a function that returns a color. The function
receives the cells\textquotesingle{} column and row indices, starting
from zero. This can be used to implement striped tables.

Default: \texttt{\ }{\texttt{\ none\ }}\texttt{\ }

\includesvg[width=0.16667in,height=0.16667in]{/assets/icons/16-arrow-right.svg}
View example

\begin{verbatim}
#table(
  fill: (x, _) =>
    if calc.odd(x) { luma(240) }
    else { white },
  align: (x, y) =>
    if y == 0 { center }
    else if x == 0 { left }
    else { right },
  columns: 4,
  [], [*Q1*], [*Q2*], [*Q3*],
  [Revenue:], [1000 €], [2000 €], [3000 €],
  [Expenses:], [500 €], [1000 €], [1500 €],
  [Profit:], [500 €], [1000 €], [1500 €],
)
\end{verbatim}

\includegraphics[width=5in,height=\textheight,keepaspectratio]{/assets/docs/HObhPJHvYkiYqHCjRK1JHwAAAAAAAAAA.png}

\subsubsection{\texorpdfstring{\texttt{\ align\ }}{ align }}\label{parameters-align}

\href{/docs/reference/foundations/auto/}{auto} {or}
\href{/docs/reference/foundations/array/}{array} {or}
\href{/docs/reference/layout/alignment/}{alignment} {or}
\href{/docs/reference/foundations/function/}{function}

{{ Settable }}

\phantomsection\label{parameters-align-settable-tooltip}
Settable parameters can be customized for all following uses of the
function with a \texttt{\ set\ } rule.

How to align the cells\textquotesingle{} content.

This can either be a single alignment, an array of alignments
(corresponding to each column) or a function that returns an alignment.
The function receives the cells\textquotesingle{} column and row
indices, starting from zero. If set to
\texttt{\ }{\texttt{\ auto\ }}\texttt{\ } , the outer alignment is used.

Default: \texttt{\ }{\texttt{\ auto\ }}\texttt{\ }

\includesvg[width=0.16667in,height=0.16667in]{/assets/icons/16-arrow-right.svg}
View example

\begin{verbatim}
#table(
  columns: 3,
  align: (left, center, right),
  [Hello], [Hello], [Hello],
  [A], [B], [C],
)
\end{verbatim}

\includegraphics[width=5in,height=\textheight,keepaspectratio]{/assets/docs/_fBgotCl-LtVjvGU4yJFLQAAAAAAAAAA.png}

\subsubsection{\texorpdfstring{\texttt{\ stroke\ }}{ stroke }}\label{parameters-stroke}

\href{/docs/reference/foundations/none/}{none} {or}
\href{/docs/reference/layout/length/}{length} {or}
\href{/docs/reference/visualize/color/}{color} {or}
\href{/docs/reference/visualize/gradient/}{gradient} {or}
\href{/docs/reference/foundations/array/}{array} {or}
\href{/docs/reference/visualize/stroke/}{stroke} {or}
\href{/docs/reference/visualize/pattern/}{pattern} {or}
\href{/docs/reference/foundations/dictionary/}{dictionary} {or}
\href{/docs/reference/foundations/function/}{function}

{{ Settable }}

\phantomsection\label{parameters-stroke-settable-tooltip}
Settable parameters can be customized for all following uses of the
function with a \texttt{\ set\ } rule.

How to \href{/docs/reference/visualize/stroke/}{stroke} the cells.

Strokes can be disabled by setting this to
\texttt{\ }{\texttt{\ none\ }}\texttt{\ } .

If it is necessary to place lines which can cross spacing between cells
produced by the \texttt{\ gutter\ } option, or to override the stroke
between multiple specific cells, consider specifying one or more of
\href{/docs/reference/model/table/\#definitions-hline}{\texttt{\ table.hline\ }}
and
\href{/docs/reference/model/table/\#definitions-vline}{\texttt{\ table.vline\ }}
alongside your table cells.

See the \href{/docs/reference/layout/grid/\#parameters-stroke}{grid
documentation} for more information on strokes.

Default:
\texttt{\ }{\texttt{\ 1pt\ }}\texttt{\ }{\texttt{\ +\ }}\texttt{\ black\ }

\subsubsection{\texorpdfstring{\texttt{\ inset\ }}{ inset }}\label{parameters-inset}

\href{/docs/reference/layout/relative/}{relative} {or}
\href{/docs/reference/foundations/array/}{array} {or}
\href{/docs/reference/foundations/dictionary/}{dictionary} {or}
\href{/docs/reference/foundations/function/}{function}

{{ Settable }}

\phantomsection\label{parameters-inset-settable-tooltip}
Settable parameters can be customized for all following uses of the
function with a \texttt{\ set\ } rule.

How much to pad the cells\textquotesingle{} content.

Default:
\texttt{\ }{\texttt{\ 0\%\ }}\texttt{\ }{\texttt{\ +\ }}\texttt{\ }{\texttt{\ 5pt\ }}\texttt{\ }

\includesvg[width=0.16667in,height=0.16667in]{/assets/icons/16-arrow-right.svg}
View example

\begin{verbatim}
#table(
  inset: 10pt,
  [Hello],
  [World],
)

#table(
  columns: 2,
  inset: (
    x: 20pt,
    y: 10pt,
  ),
  [Hello],
  [World],
)
\end{verbatim}

\includegraphics[width=5in,height=\textheight,keepaspectratio]{/assets/docs/f1kE1ENTTB02iZKKPoV_XwAAAAAAAAAA.png}

\subsubsection{\texorpdfstring{\texttt{\ children\ }}{ children }}\label{parameters-children}

\href{/docs/reference/foundations/content/}{content}

{Required} {{ Positional }}

\phantomsection\label{parameters-children-positional-tooltip}
Positional parameters are specified in order, without names.

{{ Variadic }}

\phantomsection\label{parameters-children-variadic-tooltip}
Variadic parameters can be specified multiple times.

The contents of the table cells, plus any extra table lines specified
with the
\href{/docs/reference/model/table/\#definitions-hline}{\texttt{\ table.hline\ }}
and
\href{/docs/reference/model/table/\#definitions-vline}{\texttt{\ table.vline\ }}
elements.

\subsection{\texorpdfstring{{ Definitions
}}{ Definitions }}\label{definitions}

\phantomsection\label{definitions-tooltip}
Functions and types and can have associated definitions. These are
accessed by specifying the function or type, followed by a period, and
then the definition\textquotesingle s name.

\subsubsection{\texorpdfstring{\texttt{\ cell\ } {{ Element
}}}{ cell   Element }}\label{definitions-cell}

\phantomsection\label{definitions-cell-element-tooltip}
Element functions can be customized with \texttt{\ set\ } and
\texttt{\ show\ } rules.

A cell in the table. Use this to position a cell manually or to apply
styling. To do the latter, you can either use the function to override
the properties for a particular cell, or use it in show rules to apply
certain styles to multiple cells at once.

Perhaps the most important use case of
\texttt{\ table\ }{\texttt{\ .\ }}\texttt{\ cell\ } is to make a cell
span multiple columns and/or rows with the \texttt{\ colspan\ } and
\texttt{\ rowspan\ } fields.

table { . } { cell } (

{ \href{/docs/reference/foundations/content/}{content} , } {
\hyperref[definitions-cell-parameters-x]{x :}
\href{/docs/reference/foundations/auto/}{auto}
\href{/docs/reference/foundations/int/}{int} , } {
\hyperref[definitions-cell-parameters-y]{y :}
\href{/docs/reference/foundations/auto/}{auto}
\href{/docs/reference/foundations/int/}{int} , } {
\hyperref[definitions-cell-parameters-colspan]{colspan :}
\href{/docs/reference/foundations/int/}{int} , } {
\hyperref[definitions-cell-parameters-rowspan]{rowspan :}
\href{/docs/reference/foundations/int/}{int} , } {
\hyperref[definitions-cell-parameters-fill]{fill :}
\href{/docs/reference/foundations/none/}{none}
\href{/docs/reference/foundations/auto/}{auto}
\href{/docs/reference/visualize/color/}{color}
\href{/docs/reference/visualize/gradient/}{gradient}
\href{/docs/reference/visualize/pattern/}{pattern} , } {
\hyperref[definitions-cell-parameters-align]{align :}
\href{/docs/reference/foundations/auto/}{auto}
\href{/docs/reference/layout/alignment/}{alignment} , } {
\hyperref[definitions-cell-parameters-inset]{inset :}
\href{/docs/reference/foundations/auto/}{auto}
\href{/docs/reference/layout/relative/}{relative}
\href{/docs/reference/foundations/dictionary/}{dictionary} , } {
\hyperref[definitions-cell-parameters-stroke]{stroke :}
\href{/docs/reference/foundations/none/}{none}
\href{/docs/reference/layout/length/}{length}
\href{/docs/reference/visualize/color/}{color}
\href{/docs/reference/visualize/gradient/}{gradient}
\href{/docs/reference/visualize/stroke/}{stroke}
\href{/docs/reference/visualize/pattern/}{pattern}
\href{/docs/reference/foundations/dictionary/}{dictionary} , } {
\hyperref[definitions-cell-parameters-breakable]{breakable :}
\href{/docs/reference/foundations/auto/}{auto}
\href{/docs/reference/foundations/bool/}{bool} , }

) -\textgreater{} \href{/docs/reference/foundations/content/}{content}

\begin{verbatim}
#show table.cell.where(y: 0): strong
#set table(
  stroke: (x, y) => if y == 0 {
    (bottom: 0.7pt + black)
  },
  align: (x, y) => (
    if x > 0 { center }
    else { left }
  )
)

#table(
  columns: 3,
  table.header(
    [Substance],
    [Subcritical °C],
    [Supercritical °C],
  ),
  [Hydrochloric Acid],
  [12.0], [92.1],
  [Sodium Myreth Sulfate],
  [16.6], [104],
  [Potassium Hydroxide],
  table.cell(colspan: 2)[24.7],
)
\end{verbatim}

\includegraphics[width=6.39583in,height=\textheight,keepaspectratio]{/assets/docs/2rQPm8gbRwbFqiITJlD6oAAAAAAAAAAA.png}

For example, you can override the fill, alignment or inset for a single
cell:

\begin{verbatim}
// You can also import those.
#import table: cell, header

#table(
  columns: 2,
  align: center,
  header(
    [*Trip progress*],
    [*Itinerary*],
  ),
  cell(
    align: right,
    fill: fuchsia.lighten(80%),
    [🚗],
  ),
  [Get in, folks!],
  [🚗], [Eat curbside hotdog],
  cell(align: left)[🌴🚗],
  cell(
    inset: 0.06em,
    text(1.62em)[🛖🌅🌊],
  ),
)
\end{verbatim}

\includegraphics[width=4.29167in,height=\textheight,keepaspectratio]{/assets/docs/VtayZlhMrUWzOmBAyEorDQAAAAAAAAAA.png}

You may also apply a show rule on \texttt{\ table.cell\ } to style all
cells at once. Combined with selectors, this allows you to apply styles
based on a cell\textquotesingle s position:

\begin{verbatim}
#show table.cell.where(x: 0): strong

#table(
  columns: 3,
  gutter: 3pt,
  [Name], [Age], [Strength],
  [Hannes], [36], [Grace],
  [Irma], [50], [Resourcefulness],
  [Vikram], [49], [Perseverance],
)
\end{verbatim}

\includegraphics[width=5in,height=\textheight,keepaspectratio]{/assets/docs/c2SP069qvMBzeFbrjVs8pwAAAAAAAAAA.png}

\paragraph{\texorpdfstring{\texttt{\ body\ }}{ body }}\label{definitions-cell-body}

\href{/docs/reference/foundations/content/}{content}

{Required} {{ Positional }}

\phantomsection\label{definitions-cell-body-positional-tooltip}
Positional parameters are specified in order, without names.

The cell\textquotesingle s body.

\paragraph{\texorpdfstring{\texttt{\ x\ }}{ x }}\label{definitions-cell-x}

\href{/docs/reference/foundations/auto/}{auto} {or}
\href{/docs/reference/foundations/int/}{int}

{{ Settable }}

\phantomsection\label{definitions-cell-x-settable-tooltip}
Settable parameters can be customized for all following uses of the
function with a \texttt{\ set\ } rule.

The cell\textquotesingle s column (zero-indexed). Functions identically
to the \texttt{\ x\ } field in
\href{/docs/reference/layout/grid/\#definitions-cell}{\texttt{\ grid.cell\ }}
.

Default: \texttt{\ }{\texttt{\ auto\ }}\texttt{\ }

\paragraph{\texorpdfstring{\texttt{\ y\ }}{ y }}\label{definitions-cell-y}

\href{/docs/reference/foundations/auto/}{auto} {or}
\href{/docs/reference/foundations/int/}{int}

{{ Settable }}

\phantomsection\label{definitions-cell-y-settable-tooltip}
Settable parameters can be customized for all following uses of the
function with a \texttt{\ set\ } rule.

The cell\textquotesingle s row (zero-indexed). Functions identically to
the \texttt{\ y\ } field in
\href{/docs/reference/layout/grid/\#definitions-cell}{\texttt{\ grid.cell\ }}
.

Default: \texttt{\ }{\texttt{\ auto\ }}\texttt{\ }

\paragraph{\texorpdfstring{\texttt{\ colspan\ }}{ colspan }}\label{definitions-cell-colspan}

\href{/docs/reference/foundations/int/}{int}

{{ Settable }}

\phantomsection\label{definitions-cell-colspan-settable-tooltip}
Settable parameters can be customized for all following uses of the
function with a \texttt{\ set\ } rule.

The amount of columns spanned by this cell.

Default: \texttt{\ }{\texttt{\ 1\ }}\texttt{\ }

\paragraph{\texorpdfstring{\texttt{\ rowspan\ }}{ rowspan }}\label{definitions-cell-rowspan}

\href{/docs/reference/foundations/int/}{int}

{{ Settable }}

\phantomsection\label{definitions-cell-rowspan-settable-tooltip}
Settable parameters can be customized for all following uses of the
function with a \texttt{\ set\ } rule.

The amount of rows spanned by this cell.

Default: \texttt{\ }{\texttt{\ 1\ }}\texttt{\ }

\paragraph{\texorpdfstring{\texttt{\ fill\ }}{ fill }}\label{definitions-cell-fill}

\href{/docs/reference/foundations/none/}{none} {or}
\href{/docs/reference/foundations/auto/}{auto} {or}
\href{/docs/reference/visualize/color/}{color} {or}
\href{/docs/reference/visualize/gradient/}{gradient} {or}
\href{/docs/reference/visualize/pattern/}{pattern}

{{ Settable }}

\phantomsection\label{definitions-cell-fill-settable-tooltip}
Settable parameters can be customized for all following uses of the
function with a \texttt{\ set\ } rule.

The cell\textquotesingle s
\href{/docs/reference/model/table/\#parameters-fill}{fill} override.

Default: \texttt{\ }{\texttt{\ auto\ }}\texttt{\ }

\paragraph{\texorpdfstring{\texttt{\ align\ }}{ align }}\label{definitions-cell-align}

\href{/docs/reference/foundations/auto/}{auto} {or}
\href{/docs/reference/layout/alignment/}{alignment}

{{ Settable }}

\phantomsection\label{definitions-cell-align-settable-tooltip}
Settable parameters can be customized for all following uses of the
function with a \texttt{\ set\ } rule.

The cell\textquotesingle s
\href{/docs/reference/model/table/\#parameters-align}{alignment}
override.

Default: \texttt{\ }{\texttt{\ auto\ }}\texttt{\ }

\paragraph{\texorpdfstring{\texttt{\ inset\ }}{ inset }}\label{definitions-cell-inset}

\href{/docs/reference/foundations/auto/}{auto} {or}
\href{/docs/reference/layout/relative/}{relative} {or}
\href{/docs/reference/foundations/dictionary/}{dictionary}

{{ Settable }}

\phantomsection\label{definitions-cell-inset-settable-tooltip}
Settable parameters can be customized for all following uses of the
function with a \texttt{\ set\ } rule.

The cell\textquotesingle s
\href{/docs/reference/model/table/\#parameters-inset}{inset} override.

Default: \texttt{\ }{\texttt{\ auto\ }}\texttt{\ }

\paragraph{\texorpdfstring{\texttt{\ stroke\ }}{ stroke }}\label{definitions-cell-stroke}

\href{/docs/reference/foundations/none/}{none} {or}
\href{/docs/reference/layout/length/}{length} {or}
\href{/docs/reference/visualize/color/}{color} {or}
\href{/docs/reference/visualize/gradient/}{gradient} {or}
\href{/docs/reference/visualize/stroke/}{stroke} {or}
\href{/docs/reference/visualize/pattern/}{pattern} {or}
\href{/docs/reference/foundations/dictionary/}{dictionary}

{{ Settable }}

\phantomsection\label{definitions-cell-stroke-settable-tooltip}
Settable parameters can be customized for all following uses of the
function with a \texttt{\ set\ } rule.

The cell\textquotesingle s
\href{/docs/reference/model/table/\#parameters-stroke}{stroke} override.

Default:
\texttt{\ }{\texttt{\ (\ }}\texttt{\ }{\texttt{\ :\ }}\texttt{\ }{\texttt{\ )\ }}\texttt{\ }

\paragraph{\texorpdfstring{\texttt{\ breakable\ }}{ breakable }}\label{definitions-cell-breakable}

\href{/docs/reference/foundations/auto/}{auto} {or}
\href{/docs/reference/foundations/bool/}{bool}

{{ Settable }}

\phantomsection\label{definitions-cell-breakable-settable-tooltip}
Settable parameters can be customized for all following uses of the
function with a \texttt{\ set\ } rule.

Whether rows spanned by this cell can be placed in different pages. When
equal to \texttt{\ }{\texttt{\ auto\ }}\texttt{\ } , a cell spanning
only fixed-size rows is unbreakable, while a cell spanning at least one
\texttt{\ }{\texttt{\ auto\ }}\texttt{\ } -sized row is breakable.

Default: \texttt{\ }{\texttt{\ auto\ }}\texttt{\ }

\subsubsection{\texorpdfstring{\texttt{\ hline\ } {{ Element
}}}{ hline   Element }}\label{definitions-hline}

\phantomsection\label{definitions-hline-element-tooltip}
Element functions can be customized with \texttt{\ set\ } and
\texttt{\ show\ } rules.

A horizontal line in the table.

Overrides any per-cell stroke, including stroke specified through the
table\textquotesingle s \texttt{\ stroke\ } field. Can cross spacing
between cells created through the table\textquotesingle s
\href{/docs/reference/model/table/\#parameters-column-gutter}{\texttt{\ column-gutter\ }}
option.

Use this function instead of the table\textquotesingle s
\texttt{\ stroke\ } field if you want to manually place a horizontal
line at a specific position in a single table. Consider using
\href{/docs/reference/model/table/\#parameters-stroke}{table\textquotesingle s
\texttt{\ stroke\ }} field or
\href{/docs/reference/model/table/\#definitions-cell-stroke}{\texttt{\ table.cell\ }
\textquotesingle s \texttt{\ stroke\ }} field instead if the line you
want to place is part of all your tables\textquotesingle{} designs.

table { . } { hline } (

{ \hyperref[definitions-hline-parameters-y]{y :}
\href{/docs/reference/foundations/auto/}{auto}
\href{/docs/reference/foundations/int/}{int} , } {
\hyperref[definitions-hline-parameters-start]{start :}
\href{/docs/reference/foundations/int/}{int} , } {
\hyperref[definitions-hline-parameters-end]{end :}
\href{/docs/reference/foundations/none/}{none}
\href{/docs/reference/foundations/int/}{int} , } {
\hyperref[definitions-hline-parameters-stroke]{stroke :}
\href{/docs/reference/foundations/none/}{none}
\href{/docs/reference/layout/length/}{length}
\href{/docs/reference/visualize/color/}{color}
\href{/docs/reference/visualize/gradient/}{gradient}
\href{/docs/reference/visualize/stroke/}{stroke}
\href{/docs/reference/visualize/pattern/}{pattern}
\href{/docs/reference/foundations/dictionary/}{dictionary} , } {
\hyperref[definitions-hline-parameters-position]{position :}
\href{/docs/reference/layout/alignment/}{alignment} , }

) -\textgreater{} \href{/docs/reference/foundations/content/}{content}

\begin{verbatim}
#set table.hline(stroke: .6pt)

#table(
  stroke: none,
  columns: (auto, 1fr),
  [09:00], [Badge pick up],
  [09:45], [Opening Keynote],
  [10:30], [Talk: Typst's Future],
  [11:15], [Session: Good PRs],
  table.hline(start: 1),
  [Noon], [_Lunch break_],
  table.hline(start: 1),
  [14:00], [Talk: Tracked Layout],
  [15:00], [Talk: Automations],
  [16:00], [Workshop: Tables],
  table.hline(),
  [19:00], [Day 1 Attendee Mixer],
)
\end{verbatim}

\includegraphics[width=5in,height=\textheight,keepaspectratio]{/assets/docs/Fl-W72wh8hlKb72YjlJ0PgAAAAAAAAAA.png}

\paragraph{\texorpdfstring{\texttt{\ y\ }}{ y }}\label{definitions-hline-y}

\href{/docs/reference/foundations/auto/}{auto} {or}
\href{/docs/reference/foundations/int/}{int}

{{ Settable }}

\phantomsection\label{definitions-hline-y-settable-tooltip}
Settable parameters can be customized for all following uses of the
function with a \texttt{\ set\ } rule.

The row above which the horizontal line is placed (zero-indexed).
Functions identically to the \texttt{\ y\ } field in
\href{/docs/reference/layout/grid/\#definitions-hline-y}{\texttt{\ grid.hline\ }}
.

Default: \texttt{\ }{\texttt{\ auto\ }}\texttt{\ }

\paragraph{\texorpdfstring{\texttt{\ start\ }}{ start }}\label{definitions-hline-start}

\href{/docs/reference/foundations/int/}{int}

{{ Settable }}

\phantomsection\label{definitions-hline-start-settable-tooltip}
Settable parameters can be customized for all following uses of the
function with a \texttt{\ set\ } rule.

The column at which the horizontal line starts (zero-indexed,
inclusive).

Default: \texttt{\ }{\texttt{\ 0\ }}\texttt{\ }

\paragraph{\texorpdfstring{\texttt{\ end\ }}{ end }}\label{definitions-hline-end}

\href{/docs/reference/foundations/none/}{none} {or}
\href{/docs/reference/foundations/int/}{int}

{{ Settable }}

\phantomsection\label{definitions-hline-end-settable-tooltip}
Settable parameters can be customized for all following uses of the
function with a \texttt{\ set\ } rule.

The column before which the horizontal line ends (zero-indexed,
exclusive).

Default: \texttt{\ }{\texttt{\ none\ }}\texttt{\ }

\paragraph{\texorpdfstring{\texttt{\ stroke\ }}{ stroke }}\label{definitions-hline-stroke}

\href{/docs/reference/foundations/none/}{none} {or}
\href{/docs/reference/layout/length/}{length} {or}
\href{/docs/reference/visualize/color/}{color} {or}
\href{/docs/reference/visualize/gradient/}{gradient} {or}
\href{/docs/reference/visualize/stroke/}{stroke} {or}
\href{/docs/reference/visualize/pattern/}{pattern} {or}
\href{/docs/reference/foundations/dictionary/}{dictionary}

{{ Settable }}

\phantomsection\label{definitions-hline-stroke-settable-tooltip}
Settable parameters can be customized for all following uses of the
function with a \texttt{\ set\ } rule.

The line\textquotesingle s stroke.

Specifying \texttt{\ }{\texttt{\ none\ }}\texttt{\ } removes any lines
previously placed across this line\textquotesingle s range, including
hlines or per-cell stroke below it.

Default:
\texttt{\ }{\texttt{\ 1pt\ }}\texttt{\ }{\texttt{\ +\ }}\texttt{\ black\ }

\paragraph{\texorpdfstring{\texttt{\ position\ }}{ position }}\label{definitions-hline-position}

\href{/docs/reference/layout/alignment/}{alignment}

{{ Settable }}

\phantomsection\label{definitions-hline-position-settable-tooltip}
Settable parameters can be customized for all following uses of the
function with a \texttt{\ set\ } rule.

The position at which the line is placed, given its row ( \texttt{\ y\ }
) - either \texttt{\ top\ } to draw above it or \texttt{\ bottom\ } to
draw below it.

This setting is only relevant when row gutter is enabled (and
shouldn\textquotesingle t be used otherwise - prefer just increasing the
\texttt{\ y\ } field by one instead), since then the position below a
row becomes different from the position above the next row due to the
spacing between both.

Default: \texttt{\ top\ }

\subsubsection{\texorpdfstring{\texttt{\ vline\ } {{ Element
}}}{ vline   Element }}\label{definitions-vline}

\phantomsection\label{definitions-vline-element-tooltip}
Element functions can be customized with \texttt{\ set\ } and
\texttt{\ show\ } rules.

A vertical line in the table. See the docs for
\href{/docs/reference/layout/grid/\#definitions-vline}{\texttt{\ grid.vline\ }}
for more information regarding how to use this element\textquotesingle s
fields.

Overrides any per-cell stroke, including stroke specified through the
table\textquotesingle s \texttt{\ stroke\ } field. Can cross spacing
between cells created through the table\textquotesingle s
\href{/docs/reference/model/table/\#parameters-row-gutter}{\texttt{\ row-gutter\ }}
option.

Similar to
\href{/docs/reference/model/table/\#definitions-hline}{\texttt{\ table.hline\ }}
, use this function if you want to manually place a vertical line at a
specific position in a single table and use the
\href{/docs/reference/model/table/\#parameters-stroke}{table\textquotesingle s
\texttt{\ stroke\ }} field or
\href{/docs/reference/model/table/\#definitions-cell-stroke}{\texttt{\ table.cell\ }
\textquotesingle s \texttt{\ stroke\ }} field instead if the line you
want to place is part of all your tables\textquotesingle{} designs.

table { . } { vline } (

{ \hyperref[definitions-vline-parameters-x]{x :}
\href{/docs/reference/foundations/auto/}{auto}
\href{/docs/reference/foundations/int/}{int} , } {
\hyperref[definitions-vline-parameters-start]{start :}
\href{/docs/reference/foundations/int/}{int} , } {
\hyperref[definitions-vline-parameters-end]{end :}
\href{/docs/reference/foundations/none/}{none}
\href{/docs/reference/foundations/int/}{int} , } {
\hyperref[definitions-vline-parameters-stroke]{stroke :}
\href{/docs/reference/foundations/none/}{none}
\href{/docs/reference/layout/length/}{length}
\href{/docs/reference/visualize/color/}{color}
\href{/docs/reference/visualize/gradient/}{gradient}
\href{/docs/reference/visualize/stroke/}{stroke}
\href{/docs/reference/visualize/pattern/}{pattern}
\href{/docs/reference/foundations/dictionary/}{dictionary} , } {
\hyperref[definitions-vline-parameters-position]{position :}
\href{/docs/reference/layout/alignment/}{alignment} , }

) -\textgreater{} \href{/docs/reference/foundations/content/}{content}

\paragraph{\texorpdfstring{\texttt{\ x\ }}{ x }}\label{definitions-vline-x}

\href{/docs/reference/foundations/auto/}{auto} {or}
\href{/docs/reference/foundations/int/}{int}

{{ Settable }}

\phantomsection\label{definitions-vline-x-settable-tooltip}
Settable parameters can be customized for all following uses of the
function with a \texttt{\ set\ } rule.

The column before which the horizontal line is placed (zero-indexed).
Functions identically to the \texttt{\ x\ } field in
\href{/docs/reference/layout/grid/\#definitions-vline}{\texttt{\ grid.vline\ }}
.

Default: \texttt{\ }{\texttt{\ auto\ }}\texttt{\ }

\paragraph{\texorpdfstring{\texttt{\ start\ }}{ start }}\label{definitions-vline-start}

\href{/docs/reference/foundations/int/}{int}

{{ Settable }}

\phantomsection\label{definitions-vline-start-settable-tooltip}
Settable parameters can be customized for all following uses of the
function with a \texttt{\ set\ } rule.

The row at which the vertical line starts (zero-indexed, inclusive).

Default: \texttt{\ }{\texttt{\ 0\ }}\texttt{\ }

\paragraph{\texorpdfstring{\texttt{\ end\ }}{ end }}\label{definitions-vline-end}

\href{/docs/reference/foundations/none/}{none} {or}
\href{/docs/reference/foundations/int/}{int}

{{ Settable }}

\phantomsection\label{definitions-vline-end-settable-tooltip}
Settable parameters can be customized for all following uses of the
function with a \texttt{\ set\ } rule.

The row on top of which the vertical line ends (zero-indexed,
exclusive).

Default: \texttt{\ }{\texttt{\ none\ }}\texttt{\ }

\paragraph{\texorpdfstring{\texttt{\ stroke\ }}{ stroke }}\label{definitions-vline-stroke}

\href{/docs/reference/foundations/none/}{none} {or}
\href{/docs/reference/layout/length/}{length} {or}
\href{/docs/reference/visualize/color/}{color} {or}
\href{/docs/reference/visualize/gradient/}{gradient} {or}
\href{/docs/reference/visualize/stroke/}{stroke} {or}
\href{/docs/reference/visualize/pattern/}{pattern} {or}
\href{/docs/reference/foundations/dictionary/}{dictionary}

{{ Settable }}

\phantomsection\label{definitions-vline-stroke-settable-tooltip}
Settable parameters can be customized for all following uses of the
function with a \texttt{\ set\ } rule.

The line\textquotesingle s stroke.

Specifying \texttt{\ }{\texttt{\ none\ }}\texttt{\ } removes any lines
previously placed across this line\textquotesingle s range, including
vlines or per-cell stroke below it.

Default:
\texttt{\ }{\texttt{\ 1pt\ }}\texttt{\ }{\texttt{\ +\ }}\texttt{\ black\ }

\paragraph{\texorpdfstring{\texttt{\ position\ }}{ position }}\label{definitions-vline-position}

\href{/docs/reference/layout/alignment/}{alignment}

{{ Settable }}

\phantomsection\label{definitions-vline-position-settable-tooltip}
Settable parameters can be customized for all following uses of the
function with a \texttt{\ set\ } rule.

The position at which the line is placed, given its column (
\texttt{\ x\ } ) - either \texttt{\ start\ } to draw before it or
\texttt{\ end\ } to draw after it.

The values \texttt{\ left\ } and \texttt{\ right\ } are also accepted,
but discouraged as they cause your table to be inconsistent between
left-to-right and right-to-left documents.

This setting is only relevant when column gutter is enabled (and
shouldn\textquotesingle t be used otherwise - prefer just increasing the
\texttt{\ x\ } field by one instead), since then the position after a
column becomes different from the position before the next column due to
the spacing between both.

Default: \texttt{\ start\ }

\subsubsection{\texorpdfstring{\texttt{\ header\ } {{ Element
}}}{ header   Element }}\label{definitions-header}

\phantomsection\label{definitions-header-element-tooltip}
Element functions can be customized with \texttt{\ set\ } and
\texttt{\ show\ } rules.

A repeatable table header.

You should wrap your tables\textquotesingle{} heading rows in this
function even if you do not plan to wrap your table across pages because
Typst will use this function to attach accessibility metadata to tables
in the future and ensure universal access to your document.

You can use the \texttt{\ repeat\ } parameter to control whether your
table\textquotesingle s header will be repeated across pages.

table { . } { header } (

{ \hyperref[definitions-header-parameters-repeat]{repeat :}
\href{/docs/reference/foundations/bool/}{bool} , } {
\hyperref[definitions-header-parameters-children]{..}
\href{/docs/reference/foundations/content/}{content} , }

) -\textgreater{} \href{/docs/reference/foundations/content/}{content}

\begin{verbatim}
#set page(height: 11.5em)
#set table(
  fill: (x, y) =>
    if x == 0 or y == 0 {
      gray.lighten(40%)
    },
  align: right,
)

#show table.cell.where(x: 0): strong
#show table.cell.where(y: 0): strong

#table(
  columns: 4,
  table.header(
    [], [Blue chip],
    [Fresh IPO], [Penny st'k],
  ),
  table.cell(
    rowspan: 6,
    align: horizon,
    rotate(-90deg, reflow: true)[
      *USD / day*
    ],
  ),
  [0.20], [104], [5],
  [3.17], [108], [4],
  [1.59], [84],  [1],
  [0.26], [98],  [15],
  [0.01], [195], [4],
  [7.34], [57],  [2],
)
\end{verbatim}

\includegraphics[width=5in,height=\textheight,keepaspectratio]{/assets/docs/IHpzp-b7mQ7ctAllSxEWfQAAAAAAAAAA.png}
\includegraphics[width=5in,height=\textheight,keepaspectratio]{/assets/docs/IHpzp-b7mQ7ctAllSxEWfQAAAAAAAAAB.png}

\paragraph{\texorpdfstring{\texttt{\ repeat\ }}{ repeat }}\label{definitions-header-repeat}

\href{/docs/reference/foundations/bool/}{bool}

{{ Settable }}

\phantomsection\label{definitions-header-repeat-settable-tooltip}
Settable parameters can be customized for all following uses of the
function with a \texttt{\ set\ } rule.

Whether this header should be repeated across pages.

Default: \texttt{\ }{\texttt{\ true\ }}\texttt{\ }

\paragraph{\texorpdfstring{\texttt{\ children\ }}{ children }}\label{definitions-header-children}

\href{/docs/reference/foundations/content/}{content}

{Required} {{ Positional }}

\phantomsection\label{definitions-header-children-positional-tooltip}
Positional parameters are specified in order, without names.

{{ Variadic }}

\phantomsection\label{definitions-header-children-variadic-tooltip}
Variadic parameters can be specified multiple times.

The cells and lines within the header.

\subsubsection{\texorpdfstring{\texttt{\ footer\ } {{ Element
}}}{ footer   Element }}\label{definitions-footer}

\phantomsection\label{definitions-footer-element-tooltip}
Element functions can be customized with \texttt{\ set\ } and
\texttt{\ show\ } rules.

A repeatable table footer.

Just like the
\href{/docs/reference/model/table/\#definitions-header}{\texttt{\ table.header\ }}
element, the footer can repeat itself on every page of the table. This
is useful for improving legibility by adding the column labels in both
the header and footer of a large table, totals, or other information
that should be visible on every page.

No other table cells may be placed after the footer.

table { . } { footer } (

{ \hyperref[definitions-footer-parameters-repeat]{repeat :}
\href{/docs/reference/foundations/bool/}{bool} , } {
\hyperref[definitions-footer-parameters-children]{..}
\href{/docs/reference/foundations/content/}{content} , }

) -\textgreater{} \href{/docs/reference/foundations/content/}{content}

\paragraph{\texorpdfstring{\texttt{\ repeat\ }}{ repeat }}\label{definitions-footer-repeat}

\href{/docs/reference/foundations/bool/}{bool}

{{ Settable }}

\phantomsection\label{definitions-footer-repeat-settable-tooltip}
Settable parameters can be customized for all following uses of the
function with a \texttt{\ set\ } rule.

Whether this footer should be repeated across pages.

Default: \texttt{\ }{\texttt{\ true\ }}\texttt{\ }

\paragraph{\texorpdfstring{\texttt{\ children\ }}{ children }}\label{definitions-footer-children}

\href{/docs/reference/foundations/content/}{content}

{Required} {{ Positional }}

\phantomsection\label{definitions-footer-children-positional-tooltip}
Positional parameters are specified in order, without names.

{{ Variadic }}

\phantomsection\label{definitions-footer-children-variadic-tooltip}
Variadic parameters can be specified multiple times.

The cells and lines within the footer.

\href{/docs/reference/model/strong/}{\pandocbounded{\includesvg[keepaspectratio]{/assets/icons/16-arrow-right.svg}}}

{ Strong Emphasis } { Previous page }

\href{/docs/reference/model/terms/}{\pandocbounded{\includesvg[keepaspectratio]{/assets/icons/16-arrow-right.svg}}}

{ Term List } { Next page }


\title{typst.app/docs/reference/model/figure}

\begin{itemize}
\tightlist
\item
  \href{/docs}{\includesvg[width=0.16667in,height=0.16667in]{/assets/icons/16-docs-dark.svg}}
\item
  \includesvg[width=0.16667in,height=0.16667in]{/assets/icons/16-arrow-right.svg}
\item
  \href{/docs/reference/}{Reference}
\item
  \includesvg[width=0.16667in,height=0.16667in]{/assets/icons/16-arrow-right.svg}
\item
  \href{/docs/reference/model/}{Model}
\item
  \includesvg[width=0.16667in,height=0.16667in]{/assets/icons/16-arrow-right.svg}
\item
  \href{/docs/reference/model/figure/}{Figure}
\end{itemize}

\section{\texorpdfstring{\texttt{\ figure\ } {{ Element
}}}{ figure   Element }}\label{summary}

\phantomsection\label{element-tooltip}
Element functions can be customized with \texttt{\ set\ } and
\texttt{\ show\ } rules.

A figure with an optional caption.

Automatically detects its kind to select the correct counting track. For
example, figures containing images will be numbered separately from
figures containing tables.

\subsection{Examples}\label{examples}

The example below shows a basic figure with an image:

\begin{verbatim}
@glacier shows a glacier. Glaciers
are complex systems.

#figure(
  image("glacier.jpg", width: 80%),
  caption: [A curious figure.],
) <glacier>
\end{verbatim}

\includegraphics[width=5in,height=\textheight,keepaspectratio]{/assets/docs/udw8J1zW8CDfoYB1XTzdLgAAAAAAAAAA.png}

You can also insert \href{/docs/reference/model/table/}{tables} into
figures to give them a caption. The figure will detect this and
automatically use a separate counter.

\begin{verbatim}
#figure(
  table(
    columns: 4,
    [t], [1], [2], [3],
    [y], [0.3s], [0.4s], [0.8s],
  ),
  caption: [Timing results],
)
\end{verbatim}

\includegraphics[width=5in,height=\textheight,keepaspectratio]{/assets/docs/_RaOJik9O5UoQO8Eq6OM9gAAAAAAAAAA.png}

This behaviour can be overridden by explicitly specifying the
figure\textquotesingle s \texttt{\ kind\ } . All figures of the same
kind share a common counter.

\subsection{Figure behaviour}\label{figure-behaviour}

By default, figures are placed within the flow of content. To make them
float to the top or bottom of the page, you can use the
\href{/docs/reference/model/figure/\#parameters-placement}{\texttt{\ placement\ }}
argument.

If your figure is too large and its contents are breakable across pages
(e.g. if it contains a large table), then you can make the figure itself
breakable across pages as well with this show rule:

\begin{verbatim}
#show figure: set block(breakable: true)
\end{verbatim}

See the
\href{/docs/reference/layout/block/\#parameters-breakable}{block}
documentation for more information about breakable and non-breakable
blocks.

\subsection{Caption customization}\label{caption-customization}

You can modify the appearance of the figure\textquotesingle s caption
with its associated
\href{/docs/reference/model/figure/\#definitions-caption}{\texttt{\ caption\ }}
function. In the example below, we emphasize all captions:

\begin{verbatim}
#show figure.caption: emph

#figure(
  rect[Hello],
  caption: [I am emphasized!],
)
\end{verbatim}

\includegraphics[width=5in,height=\textheight,keepaspectratio]{/assets/docs/_XYhSBTt1dmjYR9A4n_aCgAAAAAAAAAA.png}

By using a
\href{/docs/reference/foundations/function/\#definitions-where}{\texttt{\ where\ }}
selector, we can scope such rules to specific kinds of figures. For
example, to position the caption above tables, but keep it below for all
other kinds of figures, we could write the following show-set rule:

\begin{verbatim}
#show figure.where(
  kind: table
): set figure.caption(position: top)

#figure(
  table(columns: 2)[A][B][C][D],
  caption: [I'm up here],
)
\end{verbatim}

\includegraphics[width=5in,height=\textheight,keepaspectratio]{/assets/docs/5FXY-vQKID4Q1FYsR4Ix9AAAAAAAAAAA.png}

\subsection{\texorpdfstring{{ Parameters
}}{ Parameters }}\label{parameters}

\phantomsection\label{parameters-tooltip}
Parameters are the inputs to a function. They are specified in
parentheses after the function name.

{ figure } (

{ \href{/docs/reference/foundations/content/}{content} , } {
\hyperref[parameters-placement]{placement :}
\href{/docs/reference/foundations/none/}{none}
\href{/docs/reference/foundations/auto/}{auto}
\href{/docs/reference/layout/alignment/}{alignment} , } {
\hyperref[parameters-scope]{scope :}
\href{/docs/reference/foundations/str/}{str} , } {
\hyperref[parameters-caption]{caption :}
\href{/docs/reference/foundations/none/}{none}
\href{/docs/reference/foundations/content/}{content} , } {
\hyperref[parameters-kind]{kind :}
\href{/docs/reference/foundations/auto/}{auto}
\href{/docs/reference/foundations/str/}{str}
\href{/docs/reference/foundations/function/}{function} , } {
\hyperref[parameters-supplement]{supplement :}
\href{/docs/reference/foundations/none/}{none}
\href{/docs/reference/foundations/auto/}{auto}
\href{/docs/reference/foundations/content/}{content}
\href{/docs/reference/foundations/function/}{function} , } {
\hyperref[parameters-numbering]{numbering :}
\href{/docs/reference/foundations/none/}{none}
\href{/docs/reference/foundations/str/}{str}
\href{/docs/reference/foundations/function/}{function} , } {
\hyperref[parameters-gap]{gap :}
\href{/docs/reference/layout/length/}{length} , } {
\hyperref[parameters-outlined]{outlined :}
\href{/docs/reference/foundations/bool/}{bool} , }

) -\textgreater{} \href{/docs/reference/foundations/content/}{content}

\subsubsection{\texorpdfstring{\texttt{\ body\ }}{ body }}\label{parameters-body}

\href{/docs/reference/foundations/content/}{content}

{Required} {{ Positional }}

\phantomsection\label{parameters-body-positional-tooltip}
Positional parameters are specified in order, without names.

The content of the figure. Often, an
\href{/docs/reference/visualize/image/}{image} .

\subsubsection{\texorpdfstring{\texttt{\ placement\ }}{ placement }}\label{parameters-placement}

\href{/docs/reference/foundations/none/}{none} {or}
\href{/docs/reference/foundations/auto/}{auto} {or}
\href{/docs/reference/layout/alignment/}{alignment}

{{ Settable }}

\phantomsection\label{parameters-placement-settable-tooltip}
Settable parameters can be customized for all following uses of the
function with a \texttt{\ set\ } rule.

The figure\textquotesingle s placement on the page.

\begin{itemize}
\tightlist
\item
  \texttt{\ }{\texttt{\ none\ }}\texttt{\ } : The figure stays in-flow
  exactly where it was specified like other content.
\item
  \texttt{\ }{\texttt{\ auto\ }}\texttt{\ } : The figure picks
  \texttt{\ top\ } or \texttt{\ bottom\ } depending on which is closer.
\item
  \texttt{\ top\ } : The figure floats to the top of the page.
\item
  \texttt{\ bottom\ } : The figure floats to the bottom of the page.
\end{itemize}

The gap between the main flow content and the floating figure is
controlled by the
\href{/docs/reference/layout/place/\#parameters-clearance}{\texttt{\ clearance\ }}
argument on the \texttt{\ place\ } function.

Default: \texttt{\ }{\texttt{\ none\ }}\texttt{\ }

\includesvg[width=0.16667in,height=0.16667in]{/assets/icons/16-arrow-right.svg}
View example

\begin{verbatim}
#set page(height: 200pt)

= Introduction
#figure(
  placement: bottom,
  caption: [A glacier],
  image("glacier.jpg", width: 60%),
)
#lorem(60)
\end{verbatim}

\includegraphics[width=5in,height=\textheight,keepaspectratio]{/assets/docs/AvTTV4CvkxyZB8XrzNUT3wAAAAAAAAAA.png}
\includegraphics[width=5in,height=\textheight,keepaspectratio]{/assets/docs/AvTTV4CvkxyZB8XrzNUT3wAAAAAAAAAB.png}

\subsubsection{\texorpdfstring{\texttt{\ scope\ }}{ scope }}\label{parameters-scope}

\href{/docs/reference/foundations/str/}{str}

{{ Settable }}

\phantomsection\label{parameters-scope-settable-tooltip}
Settable parameters can be customized for all following uses of the
function with a \texttt{\ set\ } rule.

Relative to which containing scope the figure is placed.

Set this to \texttt{\ }{\texttt{\ "parent"\ }}\texttt{\ } to create a
full-width figure in a two-column document.

Has no effect if \texttt{\ placement\ } is
\texttt{\ }{\texttt{\ none\ }}\texttt{\ } .

\begin{longtable}[]{@{}ll@{}}
\toprule\noalign{}
Variant & Details \\
\midrule\noalign{}
\endhead
\bottomrule\noalign{}
\endlastfoot
\texttt{\ "\ column\ "\ } & Place into the current column. \\
\texttt{\ "\ parent\ "\ } & Place relative to the parent, letting the
content span over all columns. \\
\end{longtable}

Default: \texttt{\ }{\texttt{\ "column"\ }}\texttt{\ }

\includesvg[width=0.16667in,height=0.16667in]{/assets/icons/16-arrow-right.svg}
View example

\begin{verbatim}
#set page(height: 250pt, columns: 2)

= Introduction
#figure(
  placement: bottom,
  scope: "parent",
  caption: [A glacier],
  image("glacier.jpg", width: 60%),
)
#lorem(60)
\end{verbatim}

\includegraphics[width=5in,height=\textheight,keepaspectratio]{/assets/docs/_zX5K9NHfd2mYYCeJmag7wAAAAAAAAAA.png}
\includegraphics[width=5in,height=\textheight,keepaspectratio]{/assets/docs/_zX5K9NHfd2mYYCeJmag7wAAAAAAAAAB.png}

\subsubsection{\texorpdfstring{\texttt{\ caption\ }}{ caption }}\label{parameters-caption}

\href{/docs/reference/foundations/none/}{none} {or}
\href{/docs/reference/foundations/content/}{content}

{{ Settable }}

\phantomsection\label{parameters-caption-settable-tooltip}
Settable parameters can be customized for all following uses of the
function with a \texttt{\ set\ } rule.

The figure\textquotesingle s caption.

Default: \texttt{\ }{\texttt{\ none\ }}\texttt{\ }

\subsubsection{\texorpdfstring{\texttt{\ kind\ }}{ kind }}\label{parameters-kind}

\href{/docs/reference/foundations/auto/}{auto} {or}
\href{/docs/reference/foundations/str/}{str} {or}
\href{/docs/reference/foundations/function/}{function}

{{ Settable }}

\phantomsection\label{parameters-kind-settable-tooltip}
Settable parameters can be customized for all following uses of the
function with a \texttt{\ set\ } rule.

The kind of figure this is.

All figures of the same kind share a common counter.

If set to \texttt{\ }{\texttt{\ auto\ }}\texttt{\ } , the figure will
try to automatically determine its kind based on the type of its body.
Automatically detected kinds are
\href{/docs/reference/model/table/}{tables} and
\href{/docs/reference/text/raw/}{code} . In other cases, the inferred
kind is that of an \href{/docs/reference/visualize/image/}{image} .

Setting this to something other than
\texttt{\ }{\texttt{\ auto\ }}\texttt{\ } will override the automatic
detection. This can be useful if

\begin{itemize}
\tightlist
\item
  you wish to create a custom figure type that is not an
  \href{/docs/reference/visualize/image/}{image} , a
  \href{/docs/reference/model/table/}{table} or
  \href{/docs/reference/text/raw/}{code} ,
\item
  you want to force the figure to use a specific counter regardless of
  its content.
\end{itemize}

You can set the kind to be an element function or a string. If you set
it to an element function other than
\href{/docs/reference/model/table/}{\texttt{\ table\ }} ,
\href{/docs/reference/text/raw/}{\texttt{\ raw\ }} or
\href{/docs/reference/visualize/image/}{\texttt{\ image\ }} , you will
need to manually specify the figure\textquotesingle s supplement.

Default: \texttt{\ }{\texttt{\ auto\ }}\texttt{\ }

\includesvg[width=0.16667in,height=0.16667in]{/assets/icons/16-arrow-right.svg}
View example

\begin{verbatim}
#figure(
  circle(radius: 10pt),
  caption: [A curious atom.],
  kind: "atom",
  supplement: [Atom],
)
\end{verbatim}

\includegraphics[width=5in,height=\textheight,keepaspectratio]{/assets/docs/gnEhUtPlQLC9DmHftY4vzQAAAAAAAAAA.png}

\subsubsection{\texorpdfstring{\texttt{\ supplement\ }}{ supplement }}\label{parameters-supplement}

\href{/docs/reference/foundations/none/}{none} {or}
\href{/docs/reference/foundations/auto/}{auto} {or}
\href{/docs/reference/foundations/content/}{content} {or}
\href{/docs/reference/foundations/function/}{function}

{{ Settable }}

\phantomsection\label{parameters-supplement-settable-tooltip}
Settable parameters can be customized for all following uses of the
function with a \texttt{\ set\ } rule.

The figure\textquotesingle s supplement.

If set to \texttt{\ }{\texttt{\ auto\ }}\texttt{\ } , the figure will
try to automatically determine the correct supplement based on the
\texttt{\ kind\ } and the active
\href{/docs/reference/text/text/\#parameters-lang}{text language} . If
you are using a custom figure type, you will need to manually specify
the supplement.

If a function is specified, it is passed the first descendant of the
specified \texttt{\ kind\ } (typically, the figure\textquotesingle s
body) and should return content.

Default: \texttt{\ }{\texttt{\ auto\ }}\texttt{\ }

\includesvg[width=0.16667in,height=0.16667in]{/assets/icons/16-arrow-right.svg}
View example

\begin{verbatim}
#figure(
  [The contents of my figure!],
  caption: [My custom figure],
  supplement: [Bar],
  kind: "foo",
)
\end{verbatim}

\includegraphics[width=5in,height=\textheight,keepaspectratio]{/assets/docs/_ow3s-d4xSBN6VX-nVHVzQAAAAAAAAAA.png}

\subsubsection{\texorpdfstring{\texttt{\ numbering\ }}{ numbering }}\label{parameters-numbering}

\href{/docs/reference/foundations/none/}{none} {or}
\href{/docs/reference/foundations/str/}{str} {or}
\href{/docs/reference/foundations/function/}{function}

{{ Settable }}

\phantomsection\label{parameters-numbering-settable-tooltip}
Settable parameters can be customized for all following uses of the
function with a \texttt{\ set\ } rule.

How to number the figure. Accepts a
\href{/docs/reference/model/numbering/}{numbering pattern or function} .

Default: \texttt{\ }{\texttt{\ "1"\ }}\texttt{\ }

\subsubsection{\texorpdfstring{\texttt{\ gap\ }}{ gap }}\label{parameters-gap}

\href{/docs/reference/layout/length/}{length}

{{ Settable }}

\phantomsection\label{parameters-gap-settable-tooltip}
Settable parameters can be customized for all following uses of the
function with a \texttt{\ set\ } rule.

The vertical gap between the body and caption.

Default: \texttt{\ }{\texttt{\ 0.65em\ }}\texttt{\ }

\subsubsection{\texorpdfstring{\texttt{\ outlined\ }}{ outlined }}\label{parameters-outlined}

\href{/docs/reference/foundations/bool/}{bool}

{{ Settable }}

\phantomsection\label{parameters-outlined-settable-tooltip}
Settable parameters can be customized for all following uses of the
function with a \texttt{\ set\ } rule.

Whether the figure should appear in an
\href{/docs/reference/model/outline/}{\texttt{\ outline\ }} of figures.

Default: \texttt{\ }{\texttt{\ true\ }}\texttt{\ }

\subsection{\texorpdfstring{{ Definitions
}}{ Definitions }}\label{definitions}

\phantomsection\label{definitions-tooltip}
Functions and types and can have associated definitions. These are
accessed by specifying the function or type, followed by a period, and
then the definition\textquotesingle s name.

\subsubsection{\texorpdfstring{\texttt{\ caption\ } {{ Element
}}}{ caption   Element }}\label{definitions-caption}

\phantomsection\label{definitions-caption-element-tooltip}
Element functions can be customized with \texttt{\ set\ } and
\texttt{\ show\ } rules.

The caption of a figure. This element can be used in set and show rules
to customize the appearance of captions for all figures or figures of a
specific kind.

In addition to its \texttt{\ pos\ } and \texttt{\ body\ } , the
\texttt{\ caption\ } also provides the figure\textquotesingle s
\texttt{\ kind\ } , \texttt{\ supplement\ } , \texttt{\ counter\ } , and
\texttt{\ numbering\ } as fields. These parts can be used in
\href{/docs/reference/foundations/function/\#definitions-where}{\texttt{\ where\ }}
selectors and show rules to build a completely custom caption.

figure { . } { caption } (

{ \hyperref[definitions-caption-parameters-position]{position :}
\href{/docs/reference/layout/alignment/}{alignment} , } {
\hyperref[definitions-caption-parameters-separator]{separator :}
\href{/docs/reference/foundations/auto/}{auto}
\href{/docs/reference/foundations/content/}{content} , } {
\href{/docs/reference/foundations/content/}{content} , }

) -\textgreater{} \href{/docs/reference/foundations/content/}{content}

\begin{verbatim}
#show figure.caption: emph

#figure(
  rect[Hello],
  caption: [A rectangle],
)
\end{verbatim}

\includegraphics[width=5in,height=\textheight,keepaspectratio]{/assets/docs/_9Rae3k-14dcb00bWW4ciAAAAAAAAAAA.png}

\paragraph{\texorpdfstring{\texttt{\ position\ }}{ position }}\label{definitions-caption-position}

\href{/docs/reference/layout/alignment/}{alignment}

{{ Settable }}

\phantomsection\label{definitions-caption-position-settable-tooltip}
Settable parameters can be customized for all following uses of the
function with a \texttt{\ set\ } rule.

The caption\textquotesingle s position in the figure. Either
\texttt{\ top\ } or \texttt{\ bottom\ } .

Default: \texttt{\ bottom\ }

\includesvg[width=0.16667in,height=0.16667in]{/assets/icons/16-arrow-right.svg}
View example

\begin{verbatim}
#show figure.where(
  kind: table
): set figure.caption(position: top)

#figure(
  table(columns: 2)[A][B],
  caption: [I'm up here],
)

#figure(
  rect[Hi],
  caption: [I'm down here],
)

#figure(
  table(columns: 2)[A][B],
  caption: figure.caption(
    position: bottom,
    [I'm down here too!]
  )
)
\end{verbatim}

\includegraphics[width=5in,height=\textheight,keepaspectratio]{/assets/docs/IdFKmiavSqMTEqn8wUXuUgAAAAAAAAAA.png}

\paragraph{\texorpdfstring{\texttt{\ separator\ }}{ separator }}\label{definitions-caption-separator}

\href{/docs/reference/foundations/auto/}{auto} {or}
\href{/docs/reference/foundations/content/}{content}

{{ Settable }}

\phantomsection\label{definitions-caption-separator-settable-tooltip}
Settable parameters can be customized for all following uses of the
function with a \texttt{\ set\ } rule.

The separator which will appear between the number and body.

If set to \texttt{\ }{\texttt{\ auto\ }}\texttt{\ } , the separator will
be adapted to the current
\href{/docs/reference/text/text/\#parameters-lang}{language} and
\href{/docs/reference/text/text/\#parameters-region}{region} .

Default: \texttt{\ }{\texttt{\ auto\ }}\texttt{\ }

\includesvg[width=0.16667in,height=0.16667in]{/assets/icons/16-arrow-right.svg}
View example

\begin{verbatim}
#set figure.caption(separator: [ --- ])

#figure(
  rect[Hello],
  caption: [A rectangle],
)
\end{verbatim}

\includegraphics[width=5in,height=\textheight,keepaspectratio]{/assets/docs/F47AgUphmXiVn12oCb_ECAAAAAAAAAAA.png}

\paragraph{\texorpdfstring{\texttt{\ body\ }}{ body }}\label{definitions-caption-body}

\href{/docs/reference/foundations/content/}{content}

{Required} {{ Positional }}

\phantomsection\label{definitions-caption-body-positional-tooltip}
Positional parameters are specified in order, without names.

The caption\textquotesingle s body.

Can be used alongside \texttt{\ kind\ } , \texttt{\ supplement\ } ,
\texttt{\ counter\ } , \texttt{\ numbering\ } , and
\texttt{\ location\ } to completely customize the caption.

\includesvg[width=0.16667in,height=0.16667in]{/assets/icons/16-arrow-right.svg}
View example

\begin{verbatim}
#show figure.caption: it => [
  #underline(it.body) |
  #it.supplement
  #context it.counter.display(it.numbering)
]

#figure(
  rect[Hello],
  caption: [A rectangle],
)
\end{verbatim}

\includegraphics[width=5in,height=\textheight,keepaspectratio]{/assets/docs/JxID--FAnIhAECKLMVFiVwAAAAAAAAAA.png}

\href{/docs/reference/model/emph/}{\pandocbounded{\includesvg[keepaspectratio]{/assets/icons/16-arrow-right.svg}}}

{ Emphasis } { Previous page }

\href{/docs/reference/model/footnote/}{\pandocbounded{\includesvg[keepaspectratio]{/assets/icons/16-arrow-right.svg}}}

{ Footnote } { Next page }


\title{typst.app/docs/reference/model/parbreak}

\begin{itemize}
\tightlist
\item
  \href{/docs}{\includesvg[width=0.16667in,height=0.16667in]{/assets/icons/16-docs-dark.svg}}
\item
  \includesvg[width=0.16667in,height=0.16667in]{/assets/icons/16-arrow-right.svg}
\item
  \href{/docs/reference/}{Reference}
\item
  \includesvg[width=0.16667in,height=0.16667in]{/assets/icons/16-arrow-right.svg}
\item
  \href{/docs/reference/model/}{Model}
\item
  \includesvg[width=0.16667in,height=0.16667in]{/assets/icons/16-arrow-right.svg}
\item
  \href{/docs/reference/model/parbreak/}{Paragraph Break}
\end{itemize}

\section{\texorpdfstring{\texttt{\ parbreak\ } {{ Element
}}}{ parbreak   Element }}\label{summary}

\phantomsection\label{element-tooltip}
Element functions can be customized with \texttt{\ set\ } and
\texttt{\ show\ } rules.

A paragraph break.

This starts a new paragraph. Especially useful when used within code
like \href{/docs/reference/scripting/\#loops}{for loops} . Multiple
consecutive paragraph breaks collapse into a single one.

\subsection{Example}\label{example}

\begin{verbatim}
#for i in range(3) {
  [Blind text #i: ]
  lorem(5)
  parbreak()
}
\end{verbatim}

\includegraphics[width=5in,height=\textheight,keepaspectratio]{/assets/docs/Ugn0Cpe50EHdh4tXrmb4YAAAAAAAAAAA.png}

\subsection{Syntax}\label{syntax}

Instead of calling this function, you can insert a blank line into your
markup to create a paragraph break.

\subsection{\texorpdfstring{{ Parameters
}}{ Parameters }}\label{parameters}

\phantomsection\label{parameters-tooltip}
Parameters are the inputs to a function. They are specified in
parentheses after the function name.

{ parbreak } (

) -\textgreater{} \href{/docs/reference/foundations/content/}{content}

\href{/docs/reference/model/par/}{\pandocbounded{\includesvg[keepaspectratio]{/assets/icons/16-arrow-right.svg}}}

{ Paragraph } { Previous page }

\href{/docs/reference/model/quote/}{\pandocbounded{\includesvg[keepaspectratio]{/assets/icons/16-arrow-right.svg}}}

{ Quote } { Next page }


\title{typst.app/docs/reference/model/outline}

\begin{itemize}
\tightlist
\item
  \href{/docs}{\includesvg[width=0.16667in,height=0.16667in]{/assets/icons/16-docs-dark.svg}}
\item
  \includesvg[width=0.16667in,height=0.16667in]{/assets/icons/16-arrow-right.svg}
\item
  \href{/docs/reference/}{Reference}
\item
  \includesvg[width=0.16667in,height=0.16667in]{/assets/icons/16-arrow-right.svg}
\item
  \href{/docs/reference/model/}{Model}
\item
  \includesvg[width=0.16667in,height=0.16667in]{/assets/icons/16-arrow-right.svg}
\item
  \href{/docs/reference/model/outline/}{Outline}
\end{itemize}

\section{\texorpdfstring{\texttt{\ outline\ } {{ Element
}}}{ outline   Element }}\label{summary}

\phantomsection\label{element-tooltip}
Element functions can be customized with \texttt{\ set\ } and
\texttt{\ show\ } rules.

A table of contents, figures, or other elements.

This function generates a list of all occurrences of an element in the
document, up to a given depth. The element\textquotesingle s numbering
and page number will be displayed in the outline alongside its title or
caption. By default this generates a table of contents.

\subsection{Example}\label{example}

\begin{verbatim}
#outline()

= Introduction
#lorem(5)

= Prior work
#lorem(10)
\end{verbatim}

\includegraphics[width=5in,height=\textheight,keepaspectratio]{/assets/docs/pxzEoLgfS9GjzIb6I2LlEgAAAAAAAAAA.png}

\subsection{Alternative outlines}\label{alternative-outlines}

By setting the \texttt{\ target\ } parameter, the outline can be used to
generate a list of other kinds of elements than headings. In the example
below, we list all figures containing images by setting
\texttt{\ target\ } to
\texttt{\ figure\ }{\texttt{\ .\ }}\texttt{\ }{\texttt{\ where\ }}\texttt{\ }{\texttt{\ (\ }}\texttt{\ kind\ }{\texttt{\ :\ }}\texttt{\ image\ }{\texttt{\ )\ }}\texttt{\ }
. We could have also set it to just \texttt{\ figure\ } , but then the
list would also include figures containing tables or other material. For
more details on the \texttt{\ where\ } selector,
\href{/docs/reference/foundations/function/\#definitions-where}{see
here} .

\begin{verbatim}
#outline(
  title: [List of Figures],
  target: figure.where(kind: image),
)

#figure(
  image("tiger.jpg"),
  caption: [A nice figure!],
)
\end{verbatim}

\includegraphics[width=5in,height=\textheight,keepaspectratio]{/assets/docs/K0Fgir_M6IbOnlxFTpRoyAAAAAAAAAAA.png}

\subsection{Styling the outline}\label{styling-the-outline}

The outline element has several options for customization, such as its
\texttt{\ title\ } and \texttt{\ indent\ } parameters. If desired,
however, it is possible to have more control over the
outline\textquotesingle s look and style through the
\href{/docs/reference/model/outline/\#definitions-entry}{\texttt{\ outline.entry\ }}
element.

\subsection{\texorpdfstring{{ Parameters
}}{ Parameters }}\label{parameters}

\phantomsection\label{parameters-tooltip}
Parameters are the inputs to a function. They are specified in
parentheses after the function name.

{ outline } (

{ \hyperref[parameters-title]{title :}
\href{/docs/reference/foundations/none/}{none}
\href{/docs/reference/foundations/auto/}{auto}
\href{/docs/reference/foundations/content/}{content} , } {
\hyperref[parameters-target]{target :}
\href{/docs/reference/foundations/label/}{label}
\href{/docs/reference/foundations/selector/}{selector}
\href{/docs/reference/introspection/location/}{location}
\href{/docs/reference/foundations/function/}{function} , } {
\hyperref[parameters-depth]{depth :}
\href{/docs/reference/foundations/none/}{none}
\href{/docs/reference/foundations/int/}{int} , } {
\hyperref[parameters-indent]{indent :}
\href{/docs/reference/foundations/none/}{none}
\href{/docs/reference/foundations/auto/}{auto}
\href{/docs/reference/foundations/bool/}{bool}
\href{/docs/reference/layout/relative/}{relative}
\href{/docs/reference/foundations/function/}{function} , } {
\hyperref[parameters-fill]{fill :}
\href{/docs/reference/foundations/none/}{none}
\href{/docs/reference/foundations/content/}{content} , }

) -\textgreater{} \href{/docs/reference/foundations/content/}{content}

\subsubsection{\texorpdfstring{\texttt{\ title\ }}{ title }}\label{parameters-title}

\href{/docs/reference/foundations/none/}{none} {or}
\href{/docs/reference/foundations/auto/}{auto} {or}
\href{/docs/reference/foundations/content/}{content}

{{ Settable }}

\phantomsection\label{parameters-title-settable-tooltip}
Settable parameters can be customized for all following uses of the
function with a \texttt{\ set\ } rule.

The title of the outline.

\begin{itemize}
\tightlist
\item
  When set to \texttt{\ }{\texttt{\ auto\ }}\texttt{\ } , an appropriate
  title for the \href{/docs/reference/text/text/\#parameters-lang}{text
  language} will be used. This is the default.
\item
  When set to \texttt{\ }{\texttt{\ none\ }}\texttt{\ } , the outline
  will not have a title.
\item
  A custom title can be set by passing content.
\end{itemize}

The outline\textquotesingle s heading will not be numbered by default,
but you can force it to be with a show-set rule:
\texttt{\ }{\texttt{\ show\ }}\texttt{\ }{\texttt{\ outline\ }}\texttt{\ }{\texttt{\ :\ }}\texttt{\ }{\texttt{\ set\ }}\texttt{\ }{\texttt{\ heading\ }}\texttt{\ }{\texttt{\ (\ }}\texttt{\ numbering\ }{\texttt{\ :\ }}\texttt{\ }{\texttt{\ "1."\ }}\texttt{\ }{\texttt{\ )\ }}\texttt{\ }

Default: \texttt{\ }{\texttt{\ auto\ }}\texttt{\ }

\subsubsection{\texorpdfstring{\texttt{\ target\ }}{ target }}\label{parameters-target}

\href{/docs/reference/foundations/label/}{label} {or}
\href{/docs/reference/foundations/selector/}{selector} {or}
\href{/docs/reference/introspection/location/}{location} {or}
\href{/docs/reference/foundations/function/}{function}

{{ Settable }}

\phantomsection\label{parameters-target-settable-tooltip}
Settable parameters can be customized for all following uses of the
function with a \texttt{\ set\ } rule.

The type of element to include in the outline.

To list figures containing a specific kind of element, like a table, you
can write
\texttt{\ figure\ }{\texttt{\ .\ }}\texttt{\ }{\texttt{\ where\ }}\texttt{\ }{\texttt{\ (\ }}\texttt{\ kind\ }{\texttt{\ :\ }}\texttt{\ table\ }{\texttt{\ )\ }}\texttt{\ }
.

Default:
\texttt{\ heading\ }{\texttt{\ .\ }}\texttt{\ }{\texttt{\ where\ }}\texttt{\ }{\texttt{\ (\ }}\texttt{\ outlined\ }{\texttt{\ :\ }}\texttt{\ }{\texttt{\ true\ }}\texttt{\ }{\texttt{\ )\ }}\texttt{\ }

\includesvg[width=0.16667in,height=0.16667in]{/assets/icons/16-arrow-right.svg}
View example

\begin{verbatim}
#outline(
  title: [List of Tables],
  target: figure.where(kind: table),
)

#figure(
  table(
    columns: 4,
    [t], [1], [2], [3],
    [y], [0.3], [0.7], [0.5],
  ),
  caption: [Experiment results],
)
\end{verbatim}

\includegraphics[width=5in,height=\textheight,keepaspectratio]{/assets/docs/9oD_YO_3aaN85cAixeBP2gAAAAAAAAAA.png}

\subsubsection{\texorpdfstring{\texttt{\ depth\ }}{ depth }}\label{parameters-depth}

\href{/docs/reference/foundations/none/}{none} {or}
\href{/docs/reference/foundations/int/}{int}

{{ Settable }}

\phantomsection\label{parameters-depth-settable-tooltip}
Settable parameters can be customized for all following uses of the
function with a \texttt{\ set\ } rule.

The maximum level up to which elements are included in the outline. When
this argument is \texttt{\ }{\texttt{\ none\ }}\texttt{\ } , all
elements are included.

Default: \texttt{\ }{\texttt{\ none\ }}\texttt{\ }

\includesvg[width=0.16667in,height=0.16667in]{/assets/icons/16-arrow-right.svg}
View example

\begin{verbatim}
#set heading(numbering: "1.")
#outline(depth: 2)

= Yes
Top-level section.

== Still
Subsection.

=== Nope
Not included.
\end{verbatim}

\includegraphics[width=5in,height=\textheight,keepaspectratio]{/assets/docs/fYEfgTUmkbH0skbcMKeSFwAAAAAAAAAA.png}

\subsubsection{\texorpdfstring{\texttt{\ indent\ }}{ indent }}\label{parameters-indent}

\href{/docs/reference/foundations/none/}{none} {or}
\href{/docs/reference/foundations/auto/}{auto} {or}
\href{/docs/reference/foundations/bool/}{bool} {or}
\href{/docs/reference/layout/relative/}{relative} {or}
\href{/docs/reference/foundations/function/}{function}

{{ Settable }}

\phantomsection\label{parameters-indent-settable-tooltip}
Settable parameters can be customized for all following uses of the
function with a \texttt{\ set\ } rule.

How to indent the outline\textquotesingle s entries.

\begin{itemize}
\tightlist
\item
  \texttt{\ }{\texttt{\ none\ }}\texttt{\ } : No indent
\item
  \texttt{\ }{\texttt{\ auto\ }}\texttt{\ } : Indents the numbering of
  the nested entry with the title of its parent entry. This only has an
  effect if the entries are numbered (e.g., via
  \href{/docs/reference/model/heading/\#parameters-numbering}{heading
  numbering} ).
\item
  \href{/docs/reference/layout/relative/}{Relative length} : Indents the
  item by this length multiplied by its nesting level. Specifying
  \texttt{\ }{\texttt{\ 2em\ }}\texttt{\ } , for instance, would indent
  top-level headings (not nested) by
  \texttt{\ }{\texttt{\ 0em\ }}\texttt{\ } , second level headings by
  \texttt{\ }{\texttt{\ 2em\ }}\texttt{\ } (nested once), third-level
  headings by \texttt{\ }{\texttt{\ 4em\ }}\texttt{\ } (nested twice)
  and so on.
\item
  \href{/docs/reference/foundations/function/}{Function} : You can
  completely customize this setting with a function. That function
  receives the nesting level as a parameter (starting at 0 for top-level
  headings/elements) and can return a relative length or content making
  up the indent. For example,
  \texttt{\ n\ }{\texttt{\ =\textgreater{}\ }}\texttt{\ n\ }{\texttt{\ *\ }}\texttt{\ }{\texttt{\ 2em\ }}\texttt{\ }
  would be equivalent to just specifying
  \texttt{\ }{\texttt{\ 2em\ }}\texttt{\ } , while
  \texttt{\ n\ }{\texttt{\ =\textgreater{}\ }}\texttt{\ }{\texttt{\ {[}\ }}\texttt{\ →\ }{\texttt{\ {]}\ }}\texttt{\ }{\texttt{\ *\ }}\texttt{\ n\ }
  would indent with one arrow per nesting level.
\end{itemize}

\emph{Migration hints:} Specifying
\texttt{\ }{\texttt{\ true\ }}\texttt{\ } (equivalent to
\texttt{\ }{\texttt{\ auto\ }}\texttt{\ } ) or
\texttt{\ }{\texttt{\ false\ }}\texttt{\ } (equivalent to
\texttt{\ }{\texttt{\ none\ }}\texttt{\ } ) for this option is
deprecated and will be removed in a future release.

Default: \texttt{\ }{\texttt{\ none\ }}\texttt{\ }

\includesvg[width=0.16667in,height=0.16667in]{/assets/icons/16-arrow-right.svg}
View example

\begin{verbatim}
#set heading(numbering: "1.a.")

#outline(
  title: [Contents (Automatic)],
  indent: auto,
)

#outline(
  title: [Contents (Length)],
  indent: 2em,
)

#outline(
  title: [Contents (Function)],
  indent: n => [→ ] * n,
)

= About ACME Corp.
== History
=== Origins
#lorem(10)

== Products
#lorem(10)
\end{verbatim}

\includegraphics[width=5in,height=\textheight,keepaspectratio]{/assets/docs/VxzAmxCU1uGgVW2hebfhtwAAAAAAAAAA.png}

\subsubsection{\texorpdfstring{\texttt{\ fill\ }}{ fill }}\label{parameters-fill}

\href{/docs/reference/foundations/none/}{none} {or}
\href{/docs/reference/foundations/content/}{content}

{{ Settable }}

\phantomsection\label{parameters-fill-settable-tooltip}
Settable parameters can be customized for all following uses of the
function with a \texttt{\ set\ } rule.

Content to fill the space between the title and the page number. Can be
set to \texttt{\ }{\texttt{\ none\ }}\texttt{\ } to disable filling.

Default:
\texttt{\ }{\texttt{\ repeat\ }}\texttt{\ }{\texttt{\ (\ }}\texttt{\ body\ }{\texttt{\ :\ }}\texttt{\ }{\texttt{\ {[}\ }}\texttt{\ .\ }{\texttt{\ {]}\ }}\texttt{\ }{\texttt{\ )\ }}\texttt{\ }

\includesvg[width=0.16667in,height=0.16667in]{/assets/icons/16-arrow-right.svg}
View example

\begin{verbatim}
#outline(fill: line(length: 100%))

= A New Beginning
\end{verbatim}

\includegraphics[width=5in,height=\textheight,keepaspectratio]{/assets/docs/KQmhOQJ1ylUUEeut6OI0rQAAAAAAAAAA.png}

\subsection{\texorpdfstring{{ Definitions
}}{ Definitions }}\label{definitions}

\phantomsection\label{definitions-tooltip}
Functions and types and can have associated definitions. These are
accessed by specifying the function or type, followed by a period, and
then the definition\textquotesingle s name.

\subsubsection{\texorpdfstring{\texttt{\ entry\ } {{ Element
}}}{ entry   Element }}\label{definitions-entry}

\phantomsection\label{definitions-entry-element-tooltip}
Element functions can be customized with \texttt{\ set\ } and
\texttt{\ show\ } rules.

Represents each entry line in an outline, including the reference to the
outlined element, its page number, and the filler content between both.

This element is intended for use with show rules to control the
appearance of outlines. To customize an entry\textquotesingle s line,
you can build it from scratch by accessing the \texttt{\ level\ } ,
\texttt{\ element\ } , \texttt{\ body\ } , \texttt{\ fill\ } and
\texttt{\ page\ } fields on the entry.

outline { . } { entry } (

{ \href{/docs/reference/foundations/int/}{int} , } {
\href{/docs/reference/foundations/content/}{content} , } {
\href{/docs/reference/foundations/content/}{content} , } {
\href{/docs/reference/foundations/none/}{none}
\href{/docs/reference/foundations/content/}{content} , } {
\href{/docs/reference/foundations/content/}{content} , }

) -\textgreater{} \href{/docs/reference/foundations/content/}{content}

\begin{verbatim}
#set heading(numbering: "1.")

#show outline.entry.where(
  level: 1
): it => {
  v(12pt, weak: true)
  strong(it)
}

#outline(indent: auto)

= Introduction
= Background
== History
== State of the Art
= Analysis
== Setup
\end{verbatim}

\includegraphics[width=5in,height=\textheight,keepaspectratio]{/assets/docs/z5yX2QHZa1YP1epncxVx1wAAAAAAAAAA.png}

\paragraph{\texorpdfstring{\texttt{\ level\ }}{ level }}\label{definitions-entry-level}

\href{/docs/reference/foundations/int/}{int}

{Required} {{ Positional }}

\phantomsection\label{definitions-entry-level-positional-tooltip}
Positional parameters are specified in order, without names.

The nesting level of this outline entry. Starts at
\texttt{\ }{\texttt{\ 1\ }}\texttt{\ } for top-level entries.

\paragraph{\texorpdfstring{\texttt{\ element\ }}{ element }}\label{definitions-entry-element}

\href{/docs/reference/foundations/content/}{content}

{Required} {{ Positional }}

\phantomsection\label{definitions-entry-element-positional-tooltip}
Positional parameters are specified in order, without names.

The element this entry refers to. Its location will be available through
the
\href{/docs/reference/foundations/content/\#definitions-location}{\texttt{\ location\ }}
method on content and can be \href{/docs/reference/model/link/}{linked}
to.

\paragraph{\texorpdfstring{\texttt{\ body\ }}{ body }}\label{definitions-entry-body}

\href{/docs/reference/foundations/content/}{content}

{Required} {{ Positional }}

\phantomsection\label{definitions-entry-body-positional-tooltip}
Positional parameters are specified in order, without names.

The content which is displayed in place of the referred element at its
entry in the outline. For a heading, this would be its number followed
by the heading\textquotesingle s title, for example.

\paragraph{\texorpdfstring{\texttt{\ fill\ }}{ fill }}\label{definitions-entry-fill}

\href{/docs/reference/foundations/none/}{none} {or}
\href{/docs/reference/foundations/content/}{content}

{Required} {{ Positional }}

\phantomsection\label{definitions-entry-fill-positional-tooltip}
Positional parameters are specified in order, without names.

The content used to fill the space between the element\textquotesingle s
outline and its page number, as defined by the outline element this
entry is located in. When \texttt{\ }{\texttt{\ none\ }}\texttt{\ } ,
empty space is inserted in that gap instead.

Note that, when using show rules to override outline entries, it is
recommended to wrap the filling content in a
\href{/docs/reference/layout/box/}{\texttt{\ box\ }} with fractional
width. For example,
\texttt{\ }{\texttt{\ box\ }}\texttt{\ }{\texttt{\ (\ }}\texttt{\ width\ }{\texttt{\ :\ }}\texttt{\ }{\texttt{\ 1fr\ }}\texttt{\ }{\texttt{\ ,\ }}\texttt{\ }{\texttt{\ repeat\ }}\texttt{\ }{\texttt{\ {[}\ }}\texttt{\ -\ }{\texttt{\ {]}\ }}\texttt{\ }{\texttt{\ )\ }}\texttt{\ }
would show precisely as many \texttt{\ -\ } characters as necessary to
fill a particular gap.

\paragraph{\texorpdfstring{\texttt{\ page\ }}{ page }}\label{definitions-entry-page}

\href{/docs/reference/foundations/content/}{content}

{Required} {{ Positional }}

\phantomsection\label{definitions-entry-page-positional-tooltip}
Positional parameters are specified in order, without names.

The page number of the element this entry links to, formatted with the
numbering set for the referenced page.

\href{/docs/reference/model/numbering/}{\pandocbounded{\includesvg[keepaspectratio]{/assets/icons/16-arrow-right.svg}}}

{ Numbering } { Previous page }

\href{/docs/reference/model/par/}{\pandocbounded{\includesvg[keepaspectratio]{/assets/icons/16-arrow-right.svg}}}

{ Paragraph } { Next page }


\title{typst.app/docs/reference/model/footnote}

\begin{itemize}
\tightlist
\item
  \href{/docs}{\includesvg[width=0.16667in,height=0.16667in]{/assets/icons/16-docs-dark.svg}}
\item
  \includesvg[width=0.16667in,height=0.16667in]{/assets/icons/16-arrow-right.svg}
\item
  \href{/docs/reference/}{Reference}
\item
  \includesvg[width=0.16667in,height=0.16667in]{/assets/icons/16-arrow-right.svg}
\item
  \href{/docs/reference/model/}{Model}
\item
  \includesvg[width=0.16667in,height=0.16667in]{/assets/icons/16-arrow-right.svg}
\item
  \href{/docs/reference/model/footnote/}{Footnote}
\end{itemize}

\section{\texorpdfstring{\texttt{\ footnote\ } {{ Element
}}}{ footnote   Element }}\label{summary}

\phantomsection\label{element-tooltip}
Element functions can be customized with \texttt{\ set\ } and
\texttt{\ show\ } rules.

A footnote.

Includes additional remarks and references on the same page with
footnotes. A footnote will insert a superscript number that links to the
note at the bottom of the page. Notes are numbered sequentially
throughout your document and can break across multiple pages.

To customize the appearance of the entry in the footnote listing, see
\href{/docs/reference/model/footnote/\#definitions-entry}{\texttt{\ footnote.entry\ }}
. The footnote itself is realized as a normal superscript, so you can
use a set rule on the
\href{/docs/reference/text/super/}{\texttt{\ super\ }} function to
customize it. You can also apply a show rule to customize only the
footnote marker (superscript number) in the running text.

\subsection{Example}\label{example}

\begin{verbatim}
Check the docs for more details.
#footnote[https://typst.app/docs]
\end{verbatim}

\includegraphics[width=5in,height=\textheight,keepaspectratio]{/assets/docs/Rux1n4IPwOkOpn1s57WxpAAAAAAAAAAA.png}

The footnote automatically attaches itself to the preceding word, even
if there is a space before it in the markup. To force space, you can use
the string
\texttt{\ }{\texttt{\ \#\ }}\texttt{\ }{\texttt{\ "\ "\ }}\texttt{\ } or
explicit \href{/docs/reference/layout/h/}{horizontal spacing} .

By giving a label to a footnote, you can have multiple references to it.

\begin{verbatim}
You can edit Typst documents online.
#footnote[https://typst.app/app] <fn>
Checkout Typst's website. @fn
And the online app. #footnote(<fn>)
\end{verbatim}

\includegraphics[width=5in,height=\textheight,keepaspectratio]{/assets/docs/xECSHtr0VhzFh5onpw48GQAAAAAAAAAA.png}

\emph{Note:} Set and show rules in the scope where \texttt{\ footnote\ }
is called may not apply to the footnote\textquotesingle s content. See
\href{https://github.com/typst/typst/issues/1467\#issuecomment-1588799440}{here}
for more information.

\subsection{\texorpdfstring{{ Parameters
}}{ Parameters }}\label{parameters}

\phantomsection\label{parameters-tooltip}
Parameters are the inputs to a function. They are specified in
parentheses after the function name.

{ footnote } (

{ \hyperref[parameters-numbering]{numbering :}
\href{/docs/reference/foundations/str/}{str}
\href{/docs/reference/foundations/function/}{function} , } {
\href{/docs/reference/foundations/label/}{label}
\href{/docs/reference/foundations/content/}{content} , }

) -\textgreater{} \href{/docs/reference/foundations/content/}{content}

\subsubsection{\texorpdfstring{\texttt{\ numbering\ }}{ numbering }}\label{parameters-numbering}

\href{/docs/reference/foundations/str/}{str} {or}
\href{/docs/reference/foundations/function/}{function}

{{ Settable }}

\phantomsection\label{parameters-numbering-settable-tooltip}
Settable parameters can be customized for all following uses of the
function with a \texttt{\ set\ } rule.

How to number footnotes.

By default, the footnote numbering continues throughout your document.
If you prefer per-page footnote numbering, you can reset the footnote
\href{/docs/reference/introspection/counter/}{counter} in the page
\href{/docs/reference/layout/page/\#parameters-header}{header} . In the
future, there might be a simpler way to achieve this.

Default: \texttt{\ }{\texttt{\ "1"\ }}\texttt{\ }

\includesvg[width=0.16667in,height=0.16667in]{/assets/icons/16-arrow-right.svg}
View example

\begin{verbatim}
#set footnote(numbering: "*")

Footnotes:
#footnote[Star],
#footnote[Dagger]
\end{verbatim}

\includegraphics[width=5in,height=\textheight,keepaspectratio]{/assets/docs/CVlSBedIWhhzGwE8LefQmwAAAAAAAAAA.png}

\subsubsection{\texorpdfstring{\texttt{\ body\ }}{ body }}\label{parameters-body}

\href{/docs/reference/foundations/label/}{label} {or}
\href{/docs/reference/foundations/content/}{content}

{Required} {{ Positional }}

\phantomsection\label{parameters-body-positional-tooltip}
Positional parameters are specified in order, without names.

The content to put into the footnote. Can also be the label of another
footnote this one should point to.

\subsection{\texorpdfstring{{ Definitions
}}{ Definitions }}\label{definitions}

\phantomsection\label{definitions-tooltip}
Functions and types and can have associated definitions. These are
accessed by specifying the function or type, followed by a period, and
then the definition\textquotesingle s name.

\subsubsection{\texorpdfstring{\texttt{\ entry\ } {{ Element
}}}{ entry   Element }}\label{definitions-entry}

\phantomsection\label{definitions-entry-element-tooltip}
Element functions can be customized with \texttt{\ set\ } and
\texttt{\ show\ } rules.

An entry in a footnote list.

This function is not intended to be called directly. Instead, it is used
in set and show rules to customize footnote listings.

footnote { . } { entry } (

{ \href{/docs/reference/foundations/content/}{content} , } {
\hyperref[definitions-entry-parameters-separator]{separator :}
\href{/docs/reference/foundations/content/}{content} , } {
\hyperref[definitions-entry-parameters-clearance]{clearance :}
\href{/docs/reference/layout/length/}{length} , } {
\hyperref[definitions-entry-parameters-gap]{gap :}
\href{/docs/reference/layout/length/}{length} , } {
\hyperref[definitions-entry-parameters-indent]{indent :}
\href{/docs/reference/layout/length/}{length} , }

) -\textgreater{} \href{/docs/reference/foundations/content/}{content}

\begin{verbatim}
#show footnote.entry: set text(red)

My footnote listing
#footnote[It's down here]
has red text!
\end{verbatim}

\includegraphics[width=5in,height=\textheight,keepaspectratio]{/assets/docs/OQcOLIwIWFG81ucXxeuiVwAAAAAAAAAA.png}

\emph{Note:} Footnote entry properties must be uniform across each page
run (a page run is a sequence of pages without an explicit pagebreak in
between). For this reason, set and show rules for footnote entries
should be defined before any page content, typically at the very start
of the document.

\paragraph{\texorpdfstring{\texttt{\ note\ }}{ note }}\label{definitions-entry-note}

\href{/docs/reference/foundations/content/}{content}

{Required} {{ Positional }}

\phantomsection\label{definitions-entry-note-positional-tooltip}
Positional parameters are specified in order, without names.

The footnote for this entry. It\textquotesingle s location can be used
to determine the footnote counter state.

\includesvg[width=0.16667in,height=0.16667in]{/assets/icons/16-arrow-right.svg}
View example

\begin{verbatim}
#show footnote.entry: it => {
  let loc = it.note.location()
  numbering(
    "1: ",
    ..counter(footnote).at(loc),
  )
  it.note.body
}

Customized #footnote[Hello]
listing #footnote[World! 🌏]
\end{verbatim}

\includegraphics[width=5in,height=\textheight,keepaspectratio]{/assets/docs/pITXewKM6sSB5ed44fUp7wAAAAAAAAAA.png}

\paragraph{\texorpdfstring{\texttt{\ separator\ }}{ separator }}\label{definitions-entry-separator}

\href{/docs/reference/foundations/content/}{content}

{{ Settable }}

\phantomsection\label{definitions-entry-separator-settable-tooltip}
Settable parameters can be customized for all following uses of the
function with a \texttt{\ set\ } rule.

The separator between the document body and the footnote listing.

Default:
\texttt{\ }{\texttt{\ line\ }}\texttt{\ }{\texttt{\ (\ }}\texttt{\ length\ }{\texttt{\ :\ }}\texttt{\ }{\texttt{\ 30\%\ }}\texttt{\ }{\texttt{\ +\ }}\texttt{\ }{\texttt{\ 0pt\ }}\texttt{\ }{\texttt{\ ,\ }}\texttt{\ stroke\ }{\texttt{\ :\ }}\texttt{\ }{\texttt{\ 0.5pt\ }}\texttt{\ }{\texttt{\ )\ }}\texttt{\ }

\includesvg[width=0.16667in,height=0.16667in]{/assets/icons/16-arrow-right.svg}
View example

\begin{verbatim}
#set footnote.entry(
  separator: repeat[.]
)

Testing a different separator.
#footnote[
  Unconventional, but maybe
  not that bad?
]
\end{verbatim}

\includegraphics[width=5in,height=\textheight,keepaspectratio]{/assets/docs/2BZbfiOf16u6fje-JM2KhwAAAAAAAAAA.png}

\paragraph{\texorpdfstring{\texttt{\ clearance\ }}{ clearance }}\label{definitions-entry-clearance}

\href{/docs/reference/layout/length/}{length}

{{ Settable }}

\phantomsection\label{definitions-entry-clearance-settable-tooltip}
Settable parameters can be customized for all following uses of the
function with a \texttt{\ set\ } rule.

The amount of clearance between the document body and the separator.

Default: \texttt{\ }{\texttt{\ 1em\ }}\texttt{\ }

\includesvg[width=0.16667in,height=0.16667in]{/assets/icons/16-arrow-right.svg}
View example

\begin{verbatim}
#set footnote.entry(clearance: 3em)

Footnotes also need ...
#footnote[
  ... some space to breathe.
]
\end{verbatim}

\includegraphics[width=5in,height=\textheight,keepaspectratio]{/assets/docs/jGI_-Yxsz0NqX0MjmZS_qQAAAAAAAAAA.png}

\paragraph{\texorpdfstring{\texttt{\ gap\ }}{ gap }}\label{definitions-entry-gap}

\href{/docs/reference/layout/length/}{length}

{{ Settable }}

\phantomsection\label{definitions-entry-gap-settable-tooltip}
Settable parameters can be customized for all following uses of the
function with a \texttt{\ set\ } rule.

The gap between footnote entries.

Default: \texttt{\ }{\texttt{\ 0.5em\ }}\texttt{\ }

\includesvg[width=0.16667in,height=0.16667in]{/assets/icons/16-arrow-right.svg}
View example

\begin{verbatim}
#set footnote.entry(gap: 0.8em)

Footnotes:
#footnote[Spaced],
#footnote[Apart]
\end{verbatim}

\includegraphics[width=5in,height=\textheight,keepaspectratio]{/assets/docs/3sggupXU7L_bO6KYRBDWHQAAAAAAAAAA.png}

\paragraph{\texorpdfstring{\texttt{\ indent\ }}{ indent }}\label{definitions-entry-indent}

\href{/docs/reference/layout/length/}{length}

{{ Settable }}

\phantomsection\label{definitions-entry-indent-settable-tooltip}
Settable parameters can be customized for all following uses of the
function with a \texttt{\ set\ } rule.

The indent of each footnote entry.

Default: \texttt{\ }{\texttt{\ 1em\ }}\texttt{\ }

\includesvg[width=0.16667in,height=0.16667in]{/assets/icons/16-arrow-right.svg}
View example

\begin{verbatim}
#set footnote.entry(indent: 0em)

Footnotes:
#footnote[No],
#footnote[Indent]
\end{verbatim}

\includegraphics[width=5in,height=\textheight,keepaspectratio]{/assets/docs/-zkE_ejQDpF6KTPTlZZ3gwAAAAAAAAAA.png}

\href{/docs/reference/model/figure/}{\pandocbounded{\includesvg[keepaspectratio]{/assets/icons/16-arrow-right.svg}}}

{ Figure } { Previous page }

\href{/docs/reference/model/heading/}{\pandocbounded{\includesvg[keepaspectratio]{/assets/icons/16-arrow-right.svg}}}

{ Heading } { Next page }


\title{typst.app/docs/reference/model/strong}

\begin{itemize}
\tightlist
\item
  \href{/docs}{\includesvg[width=0.16667in,height=0.16667in]{/assets/icons/16-docs-dark.svg}}
\item
  \includesvg[width=0.16667in,height=0.16667in]{/assets/icons/16-arrow-right.svg}
\item
  \href{/docs/reference/}{Reference}
\item
  \includesvg[width=0.16667in,height=0.16667in]{/assets/icons/16-arrow-right.svg}
\item
  \href{/docs/reference/model/}{Model}
\item
  \includesvg[width=0.16667in,height=0.16667in]{/assets/icons/16-arrow-right.svg}
\item
  \href{/docs/reference/model/strong/}{Strong Emphasis}
\end{itemize}

\section{\texorpdfstring{\texttt{\ strong\ } {{ Element
}}}{ strong   Element }}\label{summary}

\phantomsection\label{element-tooltip}
Element functions can be customized with \texttt{\ set\ } and
\texttt{\ show\ } rules.

Strongly emphasizes content by increasing the font weight.

Increases the current font weight by a given \texttt{\ delta\ } .

\subsection{Example}\label{example}

\begin{verbatim}
This is *strong.* \
This is #strong[too.] \

#show strong: set text(red)
And this is *evermore.*
\end{verbatim}

\includegraphics[width=5in,height=\textheight,keepaspectratio]{/assets/docs/8PFV4SUNXNbbYe9uHW1ITAAAAAAAAAAA.png}

\subsection{Syntax}\label{syntax}

This function also has dedicated syntax: To strongly emphasize content,
simply enclose it in stars/asterisks ( \texttt{\ *\ } ). Note that this
only works at word boundaries. To strongly emphasize part of a word, you
have to use the function.

\subsection{\texorpdfstring{{ Parameters
}}{ Parameters }}\label{parameters}

\phantomsection\label{parameters-tooltip}
Parameters are the inputs to a function. They are specified in
parentheses after the function name.

{ strong } (

{ \hyperref[parameters-delta]{delta :}
\href{/docs/reference/foundations/int/}{int} , } {
\href{/docs/reference/foundations/content/}{content} , }

) -\textgreater{} \href{/docs/reference/foundations/content/}{content}

\subsubsection{\texorpdfstring{\texttt{\ delta\ }}{ delta }}\label{parameters-delta}

\href{/docs/reference/foundations/int/}{int}

{{ Settable }}

\phantomsection\label{parameters-delta-settable-tooltip}
Settable parameters can be customized for all following uses of the
function with a \texttt{\ set\ } rule.

The delta to apply on the font weight.

Default: \texttt{\ }{\texttt{\ 300\ }}\texttt{\ }

\includesvg[width=0.16667in,height=0.16667in]{/assets/icons/16-arrow-right.svg}
View example

\begin{verbatim}
#set strong(delta: 0)
No *effect!*
\end{verbatim}

\includegraphics[width=5in,height=\textheight,keepaspectratio]{/assets/docs/SC7LmnRUxtrvxQL331fpfAAAAAAAAAAA.png}

\subsubsection{\texorpdfstring{\texttt{\ body\ }}{ body }}\label{parameters-body}

\href{/docs/reference/foundations/content/}{content}

{Required} {{ Positional }}

\phantomsection\label{parameters-body-positional-tooltip}
Positional parameters are specified in order, without names.

The content to strongly emphasize.

\href{/docs/reference/model/ref/}{\pandocbounded{\includesvg[keepaspectratio]{/assets/icons/16-arrow-right.svg}}}

{ Reference } { Previous page }

\href{/docs/reference/model/table/}{\pandocbounded{\includesvg[keepaspectratio]{/assets/icons/16-arrow-right.svg}}}

{ Table } { Next page }


\title{typst.app/docs/reference/model/emph}

\begin{itemize}
\tightlist
\item
  \href{/docs}{\includesvg[width=0.16667in,height=0.16667in]{/assets/icons/16-docs-dark.svg}}
\item
  \includesvg[width=0.16667in,height=0.16667in]{/assets/icons/16-arrow-right.svg}
\item
  \href{/docs/reference/}{Reference}
\item
  \includesvg[width=0.16667in,height=0.16667in]{/assets/icons/16-arrow-right.svg}
\item
  \href{/docs/reference/model/}{Model}
\item
  \includesvg[width=0.16667in,height=0.16667in]{/assets/icons/16-arrow-right.svg}
\item
  \href{/docs/reference/model/emph/}{Emphasis}
\end{itemize}

\section{\texorpdfstring{\texttt{\ emph\ } {{ Element
}}}{ emph   Element }}\label{summary}

\phantomsection\label{element-tooltip}
Element functions can be customized with \texttt{\ set\ } and
\texttt{\ show\ } rules.

Emphasizes content by toggling italics.

\begin{itemize}
\tightlist
\item
  If the current
  \href{/docs/reference/text/text/\#parameters-style}{text style} is
  \texttt{\ }{\texttt{\ "normal"\ }}\texttt{\ } , this turns it into
  \texttt{\ }{\texttt{\ "italic"\ }}\texttt{\ } .
\item
  If it is already \texttt{\ }{\texttt{\ "italic"\ }}\texttt{\ } or
  \texttt{\ }{\texttt{\ "oblique"\ }}\texttt{\ } , it turns it back to
  \texttt{\ }{\texttt{\ "normal"\ }}\texttt{\ } .
\end{itemize}

\subsection{Example}\label{example}

\begin{verbatim}
This is _emphasized._ \
This is #emph[too.]

#show emph: it => {
  text(blue, it.body)
}

This is _emphasized_ differently.
\end{verbatim}

\includegraphics[width=5in,height=\textheight,keepaspectratio]{/assets/docs/p3cGCgaJdrkrScOita7QfgAAAAAAAAAA.png}

\subsection{Syntax}\label{syntax}

This function also has dedicated syntax: To emphasize content, simply
enclose it in underscores ( \texttt{\ \_\ } ). Note that this only works
at word boundaries. To emphasize part of a word, you have to use the
function.

\subsection{\texorpdfstring{{ Parameters
}}{ Parameters }}\label{parameters}

\phantomsection\label{parameters-tooltip}
Parameters are the inputs to a function. They are specified in
parentheses after the function name.

{ emph } (

{ \href{/docs/reference/foundations/content/}{content} }

) -\textgreater{} \href{/docs/reference/foundations/content/}{content}

\subsubsection{\texorpdfstring{\texttt{\ body\ }}{ body }}\label{parameters-body}

\href{/docs/reference/foundations/content/}{content}

{Required} {{ Positional }}

\phantomsection\label{parameters-body-positional-tooltip}
Positional parameters are specified in order, without names.

The content to emphasize.

\href{/docs/reference/model/document/}{\pandocbounded{\includesvg[keepaspectratio]{/assets/icons/16-arrow-right.svg}}}

{ Document } { Previous page }

\href{/docs/reference/model/figure/}{\pandocbounded{\includesvg[keepaspectratio]{/assets/icons/16-arrow-right.svg}}}

{ Figure } { Next page }


\title{typst.app/docs/reference/model/cite}

\begin{itemize}
\tightlist
\item
  \href{/docs}{\includesvg[width=0.16667in,height=0.16667in]{/assets/icons/16-docs-dark.svg}}
\item
  \includesvg[width=0.16667in,height=0.16667in]{/assets/icons/16-arrow-right.svg}
\item
  \href{/docs/reference/}{Reference}
\item
  \includesvg[width=0.16667in,height=0.16667in]{/assets/icons/16-arrow-right.svg}
\item
  \href{/docs/reference/model/}{Model}
\item
  \includesvg[width=0.16667in,height=0.16667in]{/assets/icons/16-arrow-right.svg}
\item
  \href{/docs/reference/model/cite/}{Cite}
\end{itemize}

\section{\texorpdfstring{\texttt{\ cite\ } {{ Element
}}}{ cite   Element }}\label{summary}

\phantomsection\label{element-tooltip}
Element functions can be customized with \texttt{\ set\ } and
\texttt{\ show\ } rules.

Cite a work from the bibliography.

Before you starting citing, you need to add a
\href{/docs/reference/model/bibliography/}{bibliography} somewhere in
your document.

\subsection{Example}\label{example}

\begin{verbatim}
This was already noted by
pirates long ago. @arrgh

Multiple sources say ...
@arrgh @netwok.

You can also call `cite`
explicitly. #cite(<arrgh>)

#bibliography("works.bib")
\end{verbatim}

\includegraphics[width=5in,height=\textheight,keepaspectratio]{/assets/docs/VelsLOKdUATbBc5AK51_FgAAAAAAAAAA.png}

If your source name contains certain characters such as slashes, which
are not recognized by the \texttt{\ \textless{}\textgreater{}\ } syntax,
you can explicitly call \texttt{\ label\ } instead.

\begin{verbatim}
Computer Modern is an example of a modernist serif typeface.
#cite(label("DBLP:books/lib/Knuth86a")).
\end{verbatim}

\subsection{Syntax}\label{syntax}

This function indirectly has dedicated syntax.
\href{/docs/reference/model/ref/}{References} can be used to cite works
from the bibliography. The label then corresponds to the citation key.

\subsection{\texorpdfstring{{ Parameters
}}{ Parameters }}\label{parameters}

\phantomsection\label{parameters-tooltip}
Parameters are the inputs to a function. They are specified in
parentheses after the function name.

{ cite } (

{ \href{/docs/reference/foundations/label/}{label} , } {
\hyperref[parameters-supplement]{supplement :}
\href{/docs/reference/foundations/none/}{none}
\href{/docs/reference/foundations/content/}{content} , } {
\hyperref[parameters-form]{form :}
\href{/docs/reference/foundations/none/}{none}
\href{/docs/reference/foundations/str/}{str} , } {
\hyperref[parameters-style]{style :}
\href{/docs/reference/foundations/auto/}{auto}
\href{/docs/reference/foundations/str/}{str} , }

) -\textgreater{} \href{/docs/reference/foundations/content/}{content}

\subsubsection{\texorpdfstring{\texttt{\ key\ }}{ key }}\label{parameters-key}

\href{/docs/reference/foundations/label/}{label}

{Required} {{ Positional }}

\phantomsection\label{parameters-key-positional-tooltip}
Positional parameters are specified in order, without names.

The citation key that identifies the entry in the bibliography that
shall be cited, as a label.

\includesvg[width=0.16667in,height=0.16667in]{/assets/icons/16-arrow-right.svg}
View example

\begin{verbatim}
// All the same
@netwok \
#cite(<netwok>) \
#cite(label("netwok"))
\end{verbatim}

\includegraphics[width=5in,height=\textheight,keepaspectratio]{/assets/docs/fyv1W7ZKnlPyBVM6_1DvjgAAAAAAAAAA.png}

\subsubsection{\texorpdfstring{\texttt{\ supplement\ }}{ supplement }}\label{parameters-supplement}

\href{/docs/reference/foundations/none/}{none} {or}
\href{/docs/reference/foundations/content/}{content}

{{ Settable }}

\phantomsection\label{parameters-supplement-settable-tooltip}
Settable parameters can be customized for all following uses of the
function with a \texttt{\ set\ } rule.

A supplement for the citation such as page or chapter number.

In reference syntax, the supplement can be added in square brackets:

Default: \texttt{\ }{\texttt{\ none\ }}\texttt{\ }

\includesvg[width=0.16667in,height=0.16667in]{/assets/icons/16-arrow-right.svg}
View example

\begin{verbatim}
This has been proven. @distress[p.~7]

#bibliography("works.bib")
\end{verbatim}

\includegraphics[width=5in,height=\textheight,keepaspectratio]{/assets/docs/yJ9a0jIezaQUawq1k-YqqwAAAAAAAAAA.png}

\subsubsection{\texorpdfstring{\texttt{\ form\ }}{ form }}\label{parameters-form}

\href{/docs/reference/foundations/none/}{none} {or}
\href{/docs/reference/foundations/str/}{str}

{{ Settable }}

\phantomsection\label{parameters-form-settable-tooltip}
Settable parameters can be customized for all following uses of the
function with a \texttt{\ set\ } rule.

The kind of citation to produce. Different forms are useful in different
scenarios: A normal citation is useful as a source at the end of a
sentence, while a "prose" citation is more suitable for inclusion in the
flow of text.

If set to \texttt{\ }{\texttt{\ none\ }}\texttt{\ } , the cited work is
included in the bibliography, but nothing will be displayed.

\begin{longtable}[]{@{}ll@{}}
\toprule\noalign{}
Variant & Details \\
\midrule\noalign{}
\endhead
\bottomrule\noalign{}
\endlastfoot
\texttt{\ "\ normal\ "\ } & Display in the standard way for the active
style. \\
\texttt{\ "\ prose\ "\ } & Produces a citation that is suitable for
inclusion in a sentence. \\
\texttt{\ "\ full\ "\ } & Mimics a bibliography entry, with full
information about the cited work. \\
\texttt{\ "\ author\ "\ } & Shows only the cited work\textquotesingle s
author(s). \\
\texttt{\ "\ year\ "\ } & Shows only the cited work\textquotesingle s
year. \\
\end{longtable}

Default: \texttt{\ }{\texttt{\ "normal"\ }}\texttt{\ }

\includesvg[width=0.16667in,height=0.16667in]{/assets/icons/16-arrow-right.svg}
View example

\begin{verbatim}
#cite(<netwok>, form: "prose")
show the outsized effects of
pirate life on the human psyche.
\end{verbatim}

\includegraphics[width=5in,height=\textheight,keepaspectratio]{/assets/docs/xCamzQ_SHz1kKaOAByx_rAAAAAAAAAAA.png}

\subsubsection{\texorpdfstring{\texttt{\ style\ }}{ style }}\label{parameters-style}

\href{/docs/reference/foundations/auto/}{auto} {or}
\href{/docs/reference/foundations/str/}{str}

{{ Settable }}

\phantomsection\label{parameters-style-settable-tooltip}
Settable parameters can be customized for all following uses of the
function with a \texttt{\ set\ } rule.

The citation style.

Should be either \texttt{\ }{\texttt{\ auto\ }}\texttt{\ } , one of the
built-in styles (see below) or a path to a
\href{https://citationstyles.org/}{CSL file} . Some of the styles listed
below appear twice, once with their full name and once with a short
alias.

When set to \texttt{\ }{\texttt{\ auto\ }}\texttt{\ } , automatically
use the
\href{/docs/reference/model/bibliography/\#parameters-style}{bibliography\textquotesingle s
style} for the citations.

\includesvg[width=0.16667in,height=0.16667in]{/assets/icons/16-arrow-right.svg}
View options

\begin{longtable}[]{@{}ll@{}}
\toprule\noalign{}
Variant & Details \\
\midrule\noalign{}
\endhead
\bottomrule\noalign{}
\endlastfoot
\texttt{\ "\ alphanumeric\ "\ } & Alphanumeric \\
\texttt{\ "\ american-anthropological-association\ "\ } & American
Anthropological Association \\
\texttt{\ "\ american-chemical-society\ "\ } & American Chemical
Society \\
\texttt{\ "\ american-geophysical-union\ "\ } & American Geophysical
Union \\
\texttt{\ "\ american-institute-of-aeronautics-and-astronautics\ "\ } &
American Institute of Aeronautics and Astronautics \\
\texttt{\ "\ american-institute-of-physics\ "\ } & American Institute of
Physics 4th edition \\
\texttt{\ "\ american-medical-association\ "\ } & American Medical
Association 11th edition \\
\texttt{\ "\ american-meteorological-society\ "\ } & American
Meteorological Society \\
\texttt{\ "\ american-physics-society\ "\ } & American Physical
Society \\
\texttt{\ "\ american-physiological-society\ "\ } & American
Physiological Society \\
\texttt{\ "\ american-political-science-association\ "\ } & American
Political Science Association \\
\texttt{\ "\ american-psychological-association\ "\ } & American
Psychological Association 7th edition \\
\texttt{\ "\ american-society-for-microbiology\ "\ } & American Society
for Microbiology \\
\texttt{\ "\ american-society-of-civil-engineers\ "\ } & American
Society of Civil Engineers \\
\texttt{\ "\ american-society-of-mechanical-engineers\ "\ } & American
Society of Mechanical Engineers \\
\texttt{\ "\ american-sociological-association\ "\ } & American
Sociological Association 6th/7th edition \\
\texttt{\ "\ angewandte-chemie\ "\ } & Angewandte Chemie International
Edition \\
\texttt{\ "\ annual-reviews\ "\ } & Annual Reviews (sorted by order of
appearance) \\
\texttt{\ "\ annual-reviews-author-date\ "\ } & Annual Reviews
(author-date) \\
\texttt{\ "\ associacao-brasileira-de-normas-tecnicas\ "\ } &
Associação Brasileira de Normas Técnicas (Português - Brasil) \\
\texttt{\ "\ association-for-computing-machinery\ "\ } & Association for
Computing Machinery \\
\texttt{\ "\ biomed-central\ "\ } & BioMed Central \\
\texttt{\ "\ bristol-university-press\ "\ } & Bristol University
Press \\
\texttt{\ "\ british-medical-journal\ "\ } & BMJ \\
\texttt{\ "\ cell\ "\ } & Cell \\
\texttt{\ "\ chicago-author-date\ "\ } & Chicago Manual of Style 17th
edition (author-date) \\
\texttt{\ "\ chicago-fullnotes\ "\ } & Chicago Manual of Style 17th
edition (full note) \\
\texttt{\ "\ chicago-notes\ "\ } & Chicago Manual of Style 17th edition
(note) \\
\texttt{\ "\ copernicus\ "\ } & Copernicus Publications \\
\texttt{\ "\ council-of-science-editors\ "\ } & Council of Science
Editors, Citation-Sequence (numeric, brackets) \\
\texttt{\ "\ council-of-science-editors-author-date\ "\ } & Council of
Science Editors, Name-Year (author-date) \\
\texttt{\ "\ current-opinion\ "\ } & Current Opinion journals \\
\texttt{\ "\ deutsche-gesellschaft-für-psychologie\ "\ } & Deutsche
Gesellschaft für Psychologie 5. Auflage (Deutsch) \\
\texttt{\ "\ deutsche-sprache\ "\ } & Deutsche Sprache (Deutsch) \\
\texttt{\ "\ elsevier-harvard\ "\ } & Elsevier - Harvard (with
titles) \\
\texttt{\ "\ elsevier-vancouver\ "\ } & Elsevier - Vancouver \\
\texttt{\ "\ elsevier-with-titles\ "\ } & Elsevier (numeric, with
titles) \\
\texttt{\ "\ frontiers\ "\ } & Frontiers journals \\
\texttt{\ "\ future-medicine\ "\ } & Future Medicine journals \\
\texttt{\ "\ future-science\ "\ } & Future Science Group \\
\texttt{\ "\ gb-7714-2005-numeric\ "\ } & China National Standard GB/T
7714-2005 (numeric, 中æ--‡) \\
\texttt{\ "\ gb-7714-2015-author-date\ "\ } & China National Standard
GB/T 7714-2015 (author-date, 中æ--‡) \\
\texttt{\ "\ gb-7714-2015-note\ "\ } & China National Standard GB/T
7714-2015 (note, 中æ--‡) \\
\texttt{\ "\ gb-7714-2015-numeric\ "\ } & China National Standard GB/T
7714-2015 (numeric, 中æ--‡) \\
\texttt{\ "\ gost-r-705-2008-numeric\ "\ } & Russian GOST R 7.0.5-2008
(numeric) \\
\texttt{\ "\ harvard-cite-them-right\ "\ } & Cite Them Right 12th
edition - Harvard \\
\texttt{\ "\ institute-of-electrical-and-electronics-engineers\ "\ } &
IEEE \\
\texttt{\ "\ institute-of-physics-numeric\ "\ } & Institute of Physics
(numeric) \\
\texttt{\ "\ iso-690-author-date\ "\ } & ISO-690 (author-date,
English) \\
\texttt{\ "\ iso-690-numeric\ "\ } & ISO-690 (numeric, English) \\
\texttt{\ "\ karger\ "\ } & Karger journals \\
\texttt{\ "\ mary-ann-liebert-vancouver\ "\ } & Mary Ann Liebert -
Vancouver \\
\texttt{\ "\ modern-humanities-research-association\ "\ } & Modern
Humanities Research Association 4th edition (note with bibliography) \\
\texttt{\ "\ modern-language-association\ "\ } & Modern Language
Association 9th edition \\
\texttt{\ "\ modern-language-association-8\ "\ } & Modern Language
Association 8th edition \\
\texttt{\ "\ multidisciplinary-digital-publishing-institute\ "\ } &
Multidisciplinary Digital Publishing Institute \\
\texttt{\ "\ nature\ "\ } & Nature \\
\texttt{\ "\ pensoft\ "\ } & Pensoft Journals \\
\texttt{\ "\ public-library-of-science\ "\ } & Public Library of
Science \\
\texttt{\ "\ royal-society-of-chemistry\ "\ } & Royal Society of
Chemistry \\
\texttt{\ "\ sage-vancouver\ "\ } & SAGE - Vancouver \\
\texttt{\ "\ sist02\ "\ } & SIST02 (æ---¥æœ¬èªž) \\
\texttt{\ "\ spie\ "\ } & SPIE journals \\
\texttt{\ "\ springer-basic\ "\ } & Springer - Basic (numeric,
brackets) \\
\texttt{\ "\ springer-basic-author-date\ "\ } & Springer - Basic
(author-date) \\
\texttt{\ "\ springer-fachzeitschriften-medizin-psychologie\ "\ } &
Springer - Fachzeitschriften Medizin Psychologie (Deutsch) \\
\texttt{\ "\ springer-humanities-author-date\ "\ } & Springer -
Humanities (author-date) \\
\texttt{\ "\ springer-lecture-notes-in-computer-science\ "\ } & Springer
- Lecture Notes in Computer Science \\
\texttt{\ "\ springer-mathphys\ "\ } & Springer - MathPhys (numeric,
brackets) \\
\texttt{\ "\ springer-socpsych-author-date\ "\ } & Springer - SocPsych
(author-date) \\
\texttt{\ "\ springer-vancouver\ "\ } & Springer - Vancouver
(brackets) \\
\texttt{\ "\ taylor-and-francis-chicago-author-date\ "\ } & Taylor \&
Francis - Chicago Manual of Style (author-date) \\
\texttt{\ "\ taylor-and-francis-national-library-of-medicine\ "\ } &
Taylor \& Francis - National Library of Medicine \\
\texttt{\ "\ the-institution-of-engineering-and-technology\ "\ } & The
Institution of Engineering and Technology \\
\texttt{\ "\ the-lancet\ "\ } & The Lancet \\
\texttt{\ "\ thieme\ "\ } & Thieme-German (Deutsch) \\
\texttt{\ "\ trends\ "\ } & Trends journals \\
\texttt{\ "\ turabian-author-date\ "\ } & Turabian 9th edition
(author-date) \\
\texttt{\ "\ turabian-fullnote-8\ "\ } & Turabian 8th edition (full
note) \\
\texttt{\ "\ vancouver\ "\ } & Vancouver \\
\texttt{\ "\ vancouver-superscript\ "\ } & Vancouver (superscript) \\
\end{longtable}

Default: \texttt{\ }{\texttt{\ auto\ }}\texttt{\ }

\href{/docs/reference/model/list/}{\pandocbounded{\includesvg[keepaspectratio]{/assets/icons/16-arrow-right.svg}}}

{ Bullet List } { Previous page }

\href{/docs/reference/model/document/}{\pandocbounded{\includesvg[keepaspectratio]{/assets/icons/16-arrow-right.svg}}}

{ Document } { Next page }


\title{typst.app/docs/reference/model/quote}

\begin{itemize}
\tightlist
\item
  \href{/docs}{\includesvg[width=0.16667in,height=0.16667in]{/assets/icons/16-docs-dark.svg}}
\item
  \includesvg[width=0.16667in,height=0.16667in]{/assets/icons/16-arrow-right.svg}
\item
  \href{/docs/reference/}{Reference}
\item
  \includesvg[width=0.16667in,height=0.16667in]{/assets/icons/16-arrow-right.svg}
\item
  \href{/docs/reference/model/}{Model}
\item
  \includesvg[width=0.16667in,height=0.16667in]{/assets/icons/16-arrow-right.svg}
\item
  \href{/docs/reference/model/quote/}{Quote}
\end{itemize}

\section{\texorpdfstring{\texttt{\ quote\ } {{ Element
}}}{ quote   Element }}\label{summary}

\phantomsection\label{element-tooltip}
Element functions can be customized with \texttt{\ set\ } and
\texttt{\ show\ } rules.

Displays a quote alongside an optional attribution.

\subsection{Example}\label{example}

\begin{verbatim}
Plato is often misquoted as the author of #quote[I know that I know
nothing], however, this is a derivation form his original quote:

#set quote(block: true)

#quote(attribution: [Plato])[
  ... ἔοικα γοῦν τούτου γε σμικρῷ τινι αὐτῷ τούτῳ σοφώτερος εἶναι, ὅτι
  ἃ μὴ οἶδα οὐδὲ οἴομαι εἰδέναι.
]
#quote(attribution: [from the Henry Cary literal translation of 1897])[
  ... I seem, then, in just this little thing to be wiser than this man at
  any rate, that what I do not know I do not think I know either.
]
\end{verbatim}

\includegraphics[width=5in,height=\textheight,keepaspectratio]{/assets/docs/SJpe1zkhE_liZRMF5cAy4gAAAAAAAAAA.png}

By default block quotes are padded left and right by
\texttt{\ }{\texttt{\ 1em\ }}\texttt{\ } , alignment and padding can be
controlled with show rules:

\begin{verbatim}
#set quote(block: true)
#show quote: set align(center)
#show quote: set pad(x: 5em)

#quote[
  You cannot pass... I am a servant of the Secret Fire, wielder of the
  flame of Anor. You cannot pass. The dark fire will not avail you,
  flame of Udûn. Go back to the Shadow! You cannot pass.
]
\end{verbatim}

\includegraphics[width=5in,height=\textheight,keepaspectratio]{/assets/docs/QLNv4Pfp0zBKSvwxIfby-wAAAAAAAAAA.png}

\subsection{\texorpdfstring{{ Parameters
}}{ Parameters }}\label{parameters}

\phantomsection\label{parameters-tooltip}
Parameters are the inputs to a function. They are specified in
parentheses after the function name.

{ quote } (

{ \hyperref[parameters-block]{block :}
\href{/docs/reference/foundations/bool/}{bool} , } {
\hyperref[parameters-quotes]{quotes :}
\href{/docs/reference/foundations/auto/}{auto}
\href{/docs/reference/foundations/bool/}{bool} , } {
\hyperref[parameters-attribution]{attribution :}
\href{/docs/reference/foundations/none/}{none}
\href{/docs/reference/foundations/label/}{label}
\href{/docs/reference/foundations/content/}{content} , } {
\href{/docs/reference/foundations/content/}{content} , }

) -\textgreater{} \href{/docs/reference/foundations/content/}{content}

\subsubsection{\texorpdfstring{\texttt{\ block\ }}{ block }}\label{parameters-block}

\href{/docs/reference/foundations/bool/}{bool}

{{ Settable }}

\phantomsection\label{parameters-block-settable-tooltip}
Settable parameters can be customized for all following uses of the
function with a \texttt{\ set\ } rule.

Whether this is a block quote.

Default: \texttt{\ }{\texttt{\ false\ }}\texttt{\ }

\includesvg[width=0.16667in,height=0.16667in]{/assets/icons/16-arrow-right.svg}
View example

\begin{verbatim}
An inline citation would look like
this: #quote(
  attribution: [René Descartes]
)[
  cogito, ergo sum
], and a block equation like this:
#quote(
  block: true,
  attribution: [JFK]
)[
  Ich bin ein Berliner.
]
\end{verbatim}

\includegraphics[width=5in,height=\textheight,keepaspectratio]{/assets/docs/bYLjzIuUOzRO9HYX7xT11wAAAAAAAAAA.png}

\subsubsection{\texorpdfstring{\texttt{\ quotes\ }}{ quotes }}\label{parameters-quotes}

\href{/docs/reference/foundations/auto/}{auto} {or}
\href{/docs/reference/foundations/bool/}{bool}

{{ Settable }}

\phantomsection\label{parameters-quotes-settable-tooltip}
Settable parameters can be customized for all following uses of the
function with a \texttt{\ set\ } rule.

Whether double quotes should be added around this quote.

The double quotes used are inferred from the \texttt{\ quotes\ }
property on \href{/docs/reference/text/smartquote/}{smartquote} , which
is affected by the \texttt{\ lang\ } property on
\href{/docs/reference/text/text/}{text} .

\begin{itemize}
\tightlist
\item
  \texttt{\ }{\texttt{\ true\ }}\texttt{\ } : Wrap this quote in double
  quotes.
\item
  \texttt{\ }{\texttt{\ false\ }}\texttt{\ } : Do not wrap this quote in
  double quotes.
\item
  \texttt{\ }{\texttt{\ auto\ }}\texttt{\ } : Infer whether to wrap this
  quote in double quotes based on the \texttt{\ block\ } property. If
  \texttt{\ block\ } is \texttt{\ }{\texttt{\ false\ }}\texttt{\ } ,
  double quotes are automatically added.
\end{itemize}

Default: \texttt{\ }{\texttt{\ auto\ }}\texttt{\ }

\includesvg[width=0.16667in,height=0.16667in]{/assets/icons/16-arrow-right.svg}
View example

\begin{verbatim}
#set text(lang: "de")

Ein deutsch-sprechender Author
zitiert unter umständen JFK:
#quote[Ich bin ein Berliner.]

#set text(lang: "en")

And an english speaking one may
translate the quote:
#quote[I am a Berliner.]
\end{verbatim}

\includegraphics[width=5in,height=\textheight,keepaspectratio]{/assets/docs/3Qsm4wm5qgO3MH7h3rFICAAAAAAAAAAA.png}

\subsubsection{\texorpdfstring{\texttt{\ attribution\ }}{ attribution }}\label{parameters-attribution}

\href{/docs/reference/foundations/none/}{none} {or}
\href{/docs/reference/foundations/label/}{label} {or}
\href{/docs/reference/foundations/content/}{content}

{{ Settable }}

\phantomsection\label{parameters-attribution-settable-tooltip}
Settable parameters can be customized for all following uses of the
function with a \texttt{\ set\ } rule.

The attribution of this quote, usually the author or source. Can be a
label pointing to a bibliography entry or any content. By default only
displayed for block quotes, but can be changed using a
\texttt{\ }{\texttt{\ show\ }}\texttt{\ } rule.

Default: \texttt{\ }{\texttt{\ none\ }}\texttt{\ }

\includesvg[width=0.16667in,height=0.16667in]{/assets/icons/16-arrow-right.svg}
View example

\begin{verbatim}
#quote(attribution: [René Descartes])[
  cogito, ergo sum
]

#show quote.where(block: false): it => {
  ["] + h(0pt, weak: true) + it.body + h(0pt, weak: true) + ["]
  if it.attribution != none [ (#it.attribution)]
}

#quote(
  attribution: link("https://typst.app/home")[typst.com]
)[
  Compose papers faster
]

#set quote(block: true)

#quote(attribution: <tolkien54>)[
  You cannot pass... I am a servant
  of the Secret Fire, wielder of the
  flame of Anor. You cannot pass. The
  dark fire will not avail you, flame
  of Udûn. Go back to the Shadow! You
  cannot pass.
]

#bibliography("works.bib", style: "apa")
\end{verbatim}

\includegraphics[width=5in,height=\textheight,keepaspectratio]{/assets/docs/bB0B3x32glSn_oATlkF6mQAAAAAAAAAA.png}

\subsubsection{\texorpdfstring{\texttt{\ body\ }}{ body }}\label{parameters-body}

\href{/docs/reference/foundations/content/}{content}

{Required} {{ Positional }}

\phantomsection\label{parameters-body-positional-tooltip}
Positional parameters are specified in order, without names.

The quote.

\href{/docs/reference/model/parbreak/}{\pandocbounded{\includesvg[keepaspectratio]{/assets/icons/16-arrow-right.svg}}}

{ Paragraph Break } { Previous page }

\href{/docs/reference/model/ref/}{\pandocbounded{\includesvg[keepaspectratio]{/assets/icons/16-arrow-right.svg}}}

{ Reference } { Next page }


