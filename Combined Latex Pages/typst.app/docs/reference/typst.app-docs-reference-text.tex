\title{typst.app/docs/reference/text/linebreak}

\begin{itemize}
\tightlist
\item
  \href{/docs}{\includesvg[width=0.16667in,height=0.16667in]{/assets/icons/16-docs-dark.svg}}
\item
  \includesvg[width=0.16667in,height=0.16667in]{/assets/icons/16-arrow-right.svg}
\item
  \href{/docs/reference/}{Reference}
\item
  \includesvg[width=0.16667in,height=0.16667in]{/assets/icons/16-arrow-right.svg}
\item
  \href{/docs/reference/text/}{Text}
\item
  \includesvg[width=0.16667in,height=0.16667in]{/assets/icons/16-arrow-right.svg}
\item
  \href{/docs/reference/text/linebreak/}{Line Break}
\end{itemize}

\section{\texorpdfstring{\texttt{\ linebreak\ } {{ Element
}}}{ linebreak   Element }}\label{summary}

\phantomsection\label{element-tooltip}
Element functions can be customized with \texttt{\ set\ } and
\texttt{\ show\ } rules.

Inserts a line break.

Advances the paragraph to the next line. A single trailing line break at
the end of a paragraph is ignored, but more than one creates additional
empty lines.

\subsection{Example}\label{example}

\begin{verbatim}
*Date:* 26.12.2022 \
*Topic:* Infrastructure Test \
*Severity:* High \
\end{verbatim}

\includegraphics[width=5in,height=\textheight,keepaspectratio]{/assets/docs/OEyyibskK4bIsTh7Qcp7OAAAAAAAAAAA.png}

\subsection{Syntax}\label{syntax}

This function also has dedicated syntax: To insert a line break, simply
write a backslash followed by whitespace. This always creates an
unjustified break.

\subsection{\texorpdfstring{{ Parameters
}}{ Parameters }}\label{parameters}

\phantomsection\label{parameters-tooltip}
Parameters are the inputs to a function. They are specified in
parentheses after the function name.

{ linebreak } (

{ \hyperref[parameters-justify]{justify :}
\href{/docs/reference/foundations/bool/}{bool} }

) -\textgreater{} \href{/docs/reference/foundations/content/}{content}

\subsubsection{\texorpdfstring{\texttt{\ justify\ }}{ justify }}\label{parameters-justify}

\href{/docs/reference/foundations/bool/}{bool}

{{ Settable }}

\phantomsection\label{parameters-justify-settable-tooltip}
Settable parameters can be customized for all following uses of the
function with a \texttt{\ set\ } rule.

Whether to justify the line before the break.

This is useful if you found a better line break opportunity in your
justified text than Typst did.

Default: \texttt{\ }{\texttt{\ false\ }}\texttt{\ }

\includesvg[width=0.16667in,height=0.16667in]{/assets/icons/16-arrow-right.svg}
View example

\begin{verbatim}
#set par(justify: true)
#let jb = linebreak(justify: true)

I have manually tuned the #jb
line breaks in this paragraph #jb
for an _interesting_ result. #jb
\end{verbatim}

\includegraphics[width=5in,height=\textheight,keepaspectratio]{/assets/docs/RlJnAEDPiPVRCZ7poOHTOwAAAAAAAAAA.png}

\href{/docs/reference/text/highlight/}{\pandocbounded{\includesvg[keepaspectratio]{/assets/icons/16-arrow-right.svg}}}

{ Highlight } { Previous page }

\href{/docs/reference/text/lorem/}{\pandocbounded{\includesvg[keepaspectratio]{/assets/icons/16-arrow-right.svg}}}

{ Lorem } { Next page }


\title{typst.app/docs/reference/text/highlight}

\begin{itemize}
\tightlist
\item
  \href{/docs}{\includesvg[width=0.16667in,height=0.16667in]{/assets/icons/16-docs-dark.svg}}
\item
  \includesvg[width=0.16667in,height=0.16667in]{/assets/icons/16-arrow-right.svg}
\item
  \href{/docs/reference/}{Reference}
\item
  \includesvg[width=0.16667in,height=0.16667in]{/assets/icons/16-arrow-right.svg}
\item
  \href{/docs/reference/text/}{Text}
\item
  \includesvg[width=0.16667in,height=0.16667in]{/assets/icons/16-arrow-right.svg}
\item
  \href{/docs/reference/text/highlight/}{Highlight}
\end{itemize}

\section{\texorpdfstring{\texttt{\ highlight\ } {{ Element
}}}{ highlight   Element }}\label{summary}

\phantomsection\label{element-tooltip}
Element functions can be customized with \texttt{\ set\ } and
\texttt{\ show\ } rules.

Highlights text with a background color.

\subsection{Example}\label{example}

\begin{verbatim}
This is #highlight[important].
\end{verbatim}

\includegraphics[width=5in,height=\textheight,keepaspectratio]{/assets/docs/QtpA6ir9UWFHeXPRr2gD9AAAAAAAAAAA.png}

\subsection{\texorpdfstring{{ Parameters
}}{ Parameters }}\label{parameters}

\phantomsection\label{parameters-tooltip}
Parameters are the inputs to a function. They are specified in
parentheses after the function name.

{ highlight } (

{ \hyperref[parameters-fill]{fill :}
\href{/docs/reference/foundations/none/}{none}
\href{/docs/reference/visualize/color/}{color}
\href{/docs/reference/visualize/gradient/}{gradient}
\href{/docs/reference/visualize/pattern/}{pattern} , } {
\hyperref[parameters-stroke]{stroke :}
\href{/docs/reference/foundations/none/}{none}
\href{/docs/reference/layout/length/}{length}
\href{/docs/reference/visualize/color/}{color}
\href{/docs/reference/visualize/gradient/}{gradient}
\href{/docs/reference/visualize/stroke/}{stroke}
\href{/docs/reference/visualize/pattern/}{pattern}
\href{/docs/reference/foundations/dictionary/}{dictionary} , } {
\hyperref[parameters-top-edge]{top-edge :}
\href{/docs/reference/layout/length/}{length}
\href{/docs/reference/foundations/str/}{str} , } {
\hyperref[parameters-bottom-edge]{bottom-edge :}
\href{/docs/reference/layout/length/}{length}
\href{/docs/reference/foundations/str/}{str} , } {
\hyperref[parameters-extent]{extent :}
\href{/docs/reference/layout/length/}{length} , } {
\hyperref[parameters-radius]{radius :}
\href{/docs/reference/layout/relative/}{relative}
\href{/docs/reference/foundations/dictionary/}{dictionary} , } {
\href{/docs/reference/foundations/content/}{content} , }

) -\textgreater{} \href{/docs/reference/foundations/content/}{content}

\subsubsection{\texorpdfstring{\texttt{\ fill\ }}{ fill }}\label{parameters-fill}

\href{/docs/reference/foundations/none/}{none} {or}
\href{/docs/reference/visualize/color/}{color} {or}
\href{/docs/reference/visualize/gradient/}{gradient} {or}
\href{/docs/reference/visualize/pattern/}{pattern}

{{ Settable }}

\phantomsection\label{parameters-fill-settable-tooltip}
Settable parameters can be customized for all following uses of the
function with a \texttt{\ set\ } rule.

The color to highlight the text with.

Default:
\texttt{\ }{\texttt{\ rgb\ }}\texttt{\ }{\texttt{\ (\ }}\texttt{\ }{\texttt{\ "\#fffd11a1"\ }}\texttt{\ }{\texttt{\ )\ }}\texttt{\ }

\includesvg[width=0.16667in,height=0.16667in]{/assets/icons/16-arrow-right.svg}
View example

\begin{verbatim}
This is #highlight(
  fill: blue
)[highlighted with blue].
\end{verbatim}

\includegraphics[width=5in,height=\textheight,keepaspectratio]{/assets/docs/oW--DyYpfs3nP_lZOIl65gAAAAAAAAAA.png}

\subsubsection{\texorpdfstring{\texttt{\ stroke\ }}{ stroke }}\label{parameters-stroke}

\href{/docs/reference/foundations/none/}{none} {or}
\href{/docs/reference/layout/length/}{length} {or}
\href{/docs/reference/visualize/color/}{color} {or}
\href{/docs/reference/visualize/gradient/}{gradient} {or}
\href{/docs/reference/visualize/stroke/}{stroke} {or}
\href{/docs/reference/visualize/pattern/}{pattern} {or}
\href{/docs/reference/foundations/dictionary/}{dictionary}

{{ Settable }}

\phantomsection\label{parameters-stroke-settable-tooltip}
Settable parameters can be customized for all following uses of the
function with a \texttt{\ set\ } rule.

The highlight\textquotesingle s border color. See the
\href{/docs/reference/visualize/rect/\#parameters-stroke}{rectangle\textquotesingle s
documentation} for more details.

Default:
\texttt{\ }{\texttt{\ (\ }}\texttt{\ }{\texttt{\ :\ }}\texttt{\ }{\texttt{\ )\ }}\texttt{\ }

\includesvg[width=0.16667in,height=0.16667in]{/assets/icons/16-arrow-right.svg}
View example

\begin{verbatim}
This is a #highlight(
  stroke: fuchsia
)[stroked highlighting].
\end{verbatim}

\includegraphics[width=5in,height=\textheight,keepaspectratio]{/assets/docs/VdZlBJnkRgdzR_4L65zKzQAAAAAAAAAA.png}

\subsubsection{\texorpdfstring{\texttt{\ top-edge\ }}{ top-edge }}\label{parameters-top-edge}

\href{/docs/reference/layout/length/}{length} {or}
\href{/docs/reference/foundations/str/}{str}

{{ Settable }}

\phantomsection\label{parameters-top-edge-settable-tooltip}
Settable parameters can be customized for all following uses of the
function with a \texttt{\ set\ } rule.

The top end of the background rectangle.

\begin{longtable}[]{@{}ll@{}}
\toprule\noalign{}
Variant & Details \\
\midrule\noalign{}
\endhead
\bottomrule\noalign{}
\endlastfoot
\texttt{\ "\ ascender\ "\ } & The font\textquotesingle s ascender, which
typically exceeds the height of all glyphs. \\
\texttt{\ "\ cap-height\ "\ } & The approximate height of uppercase
letters. \\
\texttt{\ "\ x-height\ "\ } & The approximate height of non-ascending
lowercase letters. \\
\texttt{\ "\ baseline\ "\ } & The baseline on which the letters rest. \\
\texttt{\ "\ bounds\ "\ } & The top edge of the glyph\textquotesingle s
bounding box. \\
\end{longtable}

Default: \texttt{\ }{\texttt{\ "ascender"\ }}\texttt{\ }

\includesvg[width=0.16667in,height=0.16667in]{/assets/icons/16-arrow-right.svg}
View example

\begin{verbatim}
#set highlight(top-edge: "ascender")
#highlight[a] #highlight[aib]

#set highlight(top-edge: "x-height")
#highlight[a] #highlight[aib]
\end{verbatim}

\includegraphics[width=5in,height=\textheight,keepaspectratio]{/assets/docs/33w6KWiqvSrMq41iE5ob_QAAAAAAAAAA.png}

\subsubsection{\texorpdfstring{\texttt{\ bottom-edge\ }}{ bottom-edge }}\label{parameters-bottom-edge}

\href{/docs/reference/layout/length/}{length} {or}
\href{/docs/reference/foundations/str/}{str}

{{ Settable }}

\phantomsection\label{parameters-bottom-edge-settable-tooltip}
Settable parameters can be customized for all following uses of the
function with a \texttt{\ set\ } rule.

The bottom end of the background rectangle.

\begin{longtable}[]{@{}ll@{}}
\toprule\noalign{}
Variant & Details \\
\midrule\noalign{}
\endhead
\bottomrule\noalign{}
\endlastfoot
\texttt{\ "\ baseline\ "\ } & The baseline on which the letters rest. \\
\texttt{\ "\ descender\ "\ } & The font\textquotesingle s descender,
which typically exceeds the depth of all glyphs. \\
\texttt{\ "\ bounds\ "\ } & The bottom edge of the
glyph\textquotesingle s bounding box. \\
\end{longtable}

Default: \texttt{\ }{\texttt{\ "descender"\ }}\texttt{\ }

\includesvg[width=0.16667in,height=0.16667in]{/assets/icons/16-arrow-right.svg}
View example

\begin{verbatim}
#set highlight(bottom-edge: "descender")
#highlight[a] #highlight[ap]

#set highlight(bottom-edge: "baseline")
#highlight[a] #highlight[ap]
\end{verbatim}

\includegraphics[width=5in,height=\textheight,keepaspectratio]{/assets/docs/tZTem6RAQXJ8OFzIrL6AnAAAAAAAAAAA.png}

\subsubsection{\texorpdfstring{\texttt{\ extent\ }}{ extent }}\label{parameters-extent}

\href{/docs/reference/layout/length/}{length}

{{ Settable }}

\phantomsection\label{parameters-extent-settable-tooltip}
Settable parameters can be customized for all following uses of the
function with a \texttt{\ set\ } rule.

The amount by which to extend the background to the sides beyond (or
within if negative) the content.

Default: \texttt{\ }{\texttt{\ 0pt\ }}\texttt{\ }

\includesvg[width=0.16667in,height=0.16667in]{/assets/icons/16-arrow-right.svg}
View example

\begin{verbatim}
A long #highlight(extent: 4pt)[background].
\end{verbatim}

\includegraphics[width=5in,height=\textheight,keepaspectratio]{/assets/docs/L2wf2ozgvgMg2iI3FR9LdQAAAAAAAAAA.png}

\subsubsection{\texorpdfstring{\texttt{\ radius\ }}{ radius }}\label{parameters-radius}

\href{/docs/reference/layout/relative/}{relative} {or}
\href{/docs/reference/foundations/dictionary/}{dictionary}

{{ Settable }}

\phantomsection\label{parameters-radius-settable-tooltip}
Settable parameters can be customized for all following uses of the
function with a \texttt{\ set\ } rule.

How much to round the highlight\textquotesingle s corners. See the
\href{/docs/reference/visualize/rect/\#parameters-radius}{rectangle\textquotesingle s
documentation} for more details.

Default:
\texttt{\ }{\texttt{\ (\ }}\texttt{\ }{\texttt{\ :\ }}\texttt{\ }{\texttt{\ )\ }}\texttt{\ }

\includesvg[width=0.16667in,height=0.16667in]{/assets/icons/16-arrow-right.svg}
View example

\begin{verbatim}
Listen #highlight(
  radius: 5pt, extent: 2pt
)[carefully], it will be on the test.
\end{verbatim}

\includegraphics[width=5in,height=\textheight,keepaspectratio]{/assets/docs/MdD0cA7uGh7p2z32380_kAAAAAAAAAAA.png}

\subsubsection{\texorpdfstring{\texttt{\ body\ }}{ body }}\label{parameters-body}

\href{/docs/reference/foundations/content/}{content}

{Required} {{ Positional }}

\phantomsection\label{parameters-body-positional-tooltip}
Positional parameters are specified in order, without names.

The content that should be highlighted.

\href{/docs/reference/text/}{\pandocbounded{\includesvg[keepaspectratio]{/assets/icons/16-arrow-right.svg}}}

{ Text } { Previous page }

\href{/docs/reference/text/linebreak/}{\pandocbounded{\includesvg[keepaspectratio]{/assets/icons/16-arrow-right.svg}}}

{ Line Break } { Next page }


\title{typst.app/docs/reference/text/smartquote}

\begin{itemize}
\tightlist
\item
  \href{/docs}{\includesvg[width=0.16667in,height=0.16667in]{/assets/icons/16-docs-dark.svg}}
\item
  \includesvg[width=0.16667in,height=0.16667in]{/assets/icons/16-arrow-right.svg}
\item
  \href{/docs/reference/}{Reference}
\item
  \includesvg[width=0.16667in,height=0.16667in]{/assets/icons/16-arrow-right.svg}
\item
  \href{/docs/reference/text/}{Text}
\item
  \includesvg[width=0.16667in,height=0.16667in]{/assets/icons/16-arrow-right.svg}
\item
  \href{/docs/reference/text/smartquote/}{Smartquote}
\end{itemize}

\section{\texorpdfstring{\texttt{\ smartquote\ } {{ Element
}}}{ smartquote   Element }}\label{summary}

\phantomsection\label{element-tooltip}
Element functions can be customized with \texttt{\ set\ } and
\texttt{\ show\ } rules.

A language-aware quote that reacts to its context.

Automatically turns into an appropriate opening or closing quote based
on the active \href{/docs/reference/text/text/\#parameters-lang}{text
language} .

\subsection{Example}\label{example}

\begin{verbatim}
"This is in quotes."

#set text(lang: "de")
"Das ist in Anführungszeichen."

#set text(lang: "fr")
"C'est entre guillemets."
\end{verbatim}

\includegraphics[width=5in,height=\textheight,keepaspectratio]{/assets/docs/dhrUjSwC3cH8VIWvplWrzwAAAAAAAAAA.png}

\subsection{Syntax}\label{syntax}

This function also has dedicated syntax: The normal quote characters (
\texttt{\ \textquotesingle{}\ } and \texttt{\ "\ } ). Typst
automatically makes your quotes smart.

\subsection{\texorpdfstring{{ Parameters
}}{ Parameters }}\label{parameters}

\phantomsection\label{parameters-tooltip}
Parameters are the inputs to a function. They are specified in
parentheses after the function name.

{ smartquote } (

{ \hyperref[parameters-double]{double :}
\href{/docs/reference/foundations/bool/}{bool} , } {
\hyperref[parameters-enabled]{enabled :}
\href{/docs/reference/foundations/bool/}{bool} , } {
\hyperref[parameters-alternative]{alternative :}
\href{/docs/reference/foundations/bool/}{bool} , } {
\hyperref[parameters-quotes]{quotes :}
\href{/docs/reference/foundations/auto/}{auto}
\href{/docs/reference/foundations/str/}{str}
\href{/docs/reference/foundations/array/}{array}
\href{/docs/reference/foundations/dictionary/}{dictionary} , }

) -\textgreater{} \href{/docs/reference/foundations/content/}{content}

\subsubsection{\texorpdfstring{\texttt{\ double\ }}{ double }}\label{parameters-double}

\href{/docs/reference/foundations/bool/}{bool}

{{ Settable }}

\phantomsection\label{parameters-double-settable-tooltip}
Settable parameters can be customized for all following uses of the
function with a \texttt{\ set\ } rule.

Whether this should be a double quote.

Default: \texttt{\ }{\texttt{\ true\ }}\texttt{\ }

\subsubsection{\texorpdfstring{\texttt{\ enabled\ }}{ enabled }}\label{parameters-enabled}

\href{/docs/reference/foundations/bool/}{bool}

{{ Settable }}

\phantomsection\label{parameters-enabled-settable-tooltip}
Settable parameters can be customized for all following uses of the
function with a \texttt{\ set\ } rule.

Whether smart quotes are enabled.

To disable smartness for a single quote, you can also escape it with a
backslash.

Default: \texttt{\ }{\texttt{\ true\ }}\texttt{\ }

\includesvg[width=0.16667in,height=0.16667in]{/assets/icons/16-arrow-right.svg}
View example

\begin{verbatim}
#set smartquote(enabled: false)

These are "dumb" quotes.
\end{verbatim}

\includegraphics[width=5in,height=\textheight,keepaspectratio]{/assets/docs/ykeeFPAnOBNAwmLXSDxTqAAAAAAAAAAA.png}

\subsubsection{\texorpdfstring{\texttt{\ alternative\ }}{ alternative }}\label{parameters-alternative}

\href{/docs/reference/foundations/bool/}{bool}

{{ Settable }}

\phantomsection\label{parameters-alternative-settable-tooltip}
Settable parameters can be customized for all following uses of the
function with a \texttt{\ set\ } rule.

Whether to use alternative quotes.

Does nothing for languages that don\textquotesingle t have alternative
quotes, or if explicit quotes were set.

Default: \texttt{\ }{\texttt{\ false\ }}\texttt{\ }

\includesvg[width=0.16667in,height=0.16667in]{/assets/icons/16-arrow-right.svg}
View example

\begin{verbatim}
#set text(lang: "de")
#set smartquote(alternative: true)

"Das ist in anderen Anführungszeichen."
\end{verbatim}

\includegraphics[width=5in,height=\textheight,keepaspectratio]{/assets/docs/lyTPNxNjIzyFJbp-JlBMpgAAAAAAAAAA.png}

\subsubsection{\texorpdfstring{\texttt{\ quotes\ }}{ quotes }}\label{parameters-quotes}

\href{/docs/reference/foundations/auto/}{auto} {or}
\href{/docs/reference/foundations/str/}{str} {or}
\href{/docs/reference/foundations/array/}{array} {or}
\href{/docs/reference/foundations/dictionary/}{dictionary}

{{ Settable }}

\phantomsection\label{parameters-quotes-settable-tooltip}
Settable parameters can be customized for all following uses of the
function with a \texttt{\ set\ } rule.

The quotes to use.

\begin{itemize}
\tightlist
\item
  When set to \texttt{\ }{\texttt{\ auto\ }}\texttt{\ } , the
  appropriate single quotes for the
  \href{/docs/reference/text/text/\#parameters-lang}{text language} will
  be used. This is the default.
\item
  Custom quotes can be passed as a string, array, or dictionary of
  either

  \begin{itemize}
  \tightlist
  \item
    \href{/docs/reference/foundations/str/}{string} : a string
    consisting of two characters containing the opening and closing
    double quotes (characters here refer to Unicode grapheme clusters)
  \item
    \href{/docs/reference/foundations/array/}{array} : an array
    containing the opening and closing double quotes
  \item
    \href{/docs/reference/foundations/dictionary/}{dictionary} : an
    array containing the double and single quotes, each specified as
    either \texttt{\ }{\texttt{\ auto\ }}\texttt{\ } , string, or array
  \end{itemize}
\end{itemize}

Default: \texttt{\ }{\texttt{\ auto\ }}\texttt{\ }

\includesvg[width=0.16667in,height=0.16667in]{/assets/icons/16-arrow-right.svg}
View example

\begin{verbatim}
#set text(lang: "de")
'Das sind normale Anführungszeichen.'

#set smartquote(quotes: "()")
"Das sind eigene Anführungszeichen."

#set smartquote(quotes: (single: ("[[", "]]"),  double: auto))
'Das sind eigene Anführungszeichen.'
\end{verbatim}

\includegraphics[width=5in,height=\textheight,keepaspectratio]{/assets/docs/bSqE_vffbQfTFgF9cX2J6AAAAAAAAAAA.png}

\href{/docs/reference/text/smallcaps/}{\pandocbounded{\includesvg[keepaspectratio]{/assets/icons/16-arrow-right.svg}}}

{ Small Capitals } { Previous page }

\href{/docs/reference/text/strike/}{\pandocbounded{\includesvg[keepaspectratio]{/assets/icons/16-arrow-right.svg}}}

{ Strikethrough } { Next page }


\title{typst.app/docs/reference/text/underline}

\begin{itemize}
\tightlist
\item
  \href{/docs}{\includesvg[width=0.16667in,height=0.16667in]{/assets/icons/16-docs-dark.svg}}
\item
  \includesvg[width=0.16667in,height=0.16667in]{/assets/icons/16-arrow-right.svg}
\item
  \href{/docs/reference/}{Reference}
\item
  \includesvg[width=0.16667in,height=0.16667in]{/assets/icons/16-arrow-right.svg}
\item
  \href{/docs/reference/text/}{Text}
\item
  \includesvg[width=0.16667in,height=0.16667in]{/assets/icons/16-arrow-right.svg}
\item
  \href{/docs/reference/text/underline/}{Underline}
\end{itemize}

\section{\texorpdfstring{\texttt{\ underline\ } {{ Element
}}}{ underline   Element }}\label{summary}

\phantomsection\label{element-tooltip}
Element functions can be customized with \texttt{\ set\ } and
\texttt{\ show\ } rules.

Underlines text.

\subsection{Example}\label{example}

\begin{verbatim}
This is #underline[important].
\end{verbatim}

\includegraphics[width=5in,height=\textheight,keepaspectratio]{/assets/docs/xV-Fy8zwdVIfyHyOpdk_9AAAAAAAAAAA.png}

\subsection{\texorpdfstring{{ Parameters
}}{ Parameters }}\label{parameters}

\phantomsection\label{parameters-tooltip}
Parameters are the inputs to a function. They are specified in
parentheses after the function name.

{ underline } (

{ \hyperref[parameters-stroke]{stroke :}
\href{/docs/reference/foundations/auto/}{auto}
\href{/docs/reference/layout/length/}{length}
\href{/docs/reference/visualize/color/}{color}
\href{/docs/reference/visualize/gradient/}{gradient}
\href{/docs/reference/visualize/stroke/}{stroke}
\href{/docs/reference/visualize/pattern/}{pattern}
\href{/docs/reference/foundations/dictionary/}{dictionary} , } {
\hyperref[parameters-offset]{offset :}
\href{/docs/reference/foundations/auto/}{auto}
\href{/docs/reference/layout/length/}{length} , } {
\hyperref[parameters-extent]{extent :}
\href{/docs/reference/layout/length/}{length} , } {
\hyperref[parameters-evade]{evade :}
\href{/docs/reference/foundations/bool/}{bool} , } {
\hyperref[parameters-background]{background :}
\href{/docs/reference/foundations/bool/}{bool} , } {
\href{/docs/reference/foundations/content/}{content} , }

) -\textgreater{} \href{/docs/reference/foundations/content/}{content}

\subsubsection{\texorpdfstring{\texttt{\ stroke\ }}{ stroke }}\label{parameters-stroke}

\href{/docs/reference/foundations/auto/}{auto} {or}
\href{/docs/reference/layout/length/}{length} {or}
\href{/docs/reference/visualize/color/}{color} {or}
\href{/docs/reference/visualize/gradient/}{gradient} {or}
\href{/docs/reference/visualize/stroke/}{stroke} {or}
\href{/docs/reference/visualize/pattern/}{pattern} {or}
\href{/docs/reference/foundations/dictionary/}{dictionary}

{{ Settable }}

\phantomsection\label{parameters-stroke-settable-tooltip}
Settable parameters can be customized for all following uses of the
function with a \texttt{\ set\ } rule.

How to \href{/docs/reference/visualize/stroke/}{stroke} the line.

If set to \texttt{\ }{\texttt{\ auto\ }}\texttt{\ } , takes on the
text\textquotesingle s color and a thickness defined in the current
font.

Default: \texttt{\ }{\texttt{\ auto\ }}\texttt{\ }

\includesvg[width=0.16667in,height=0.16667in]{/assets/icons/16-arrow-right.svg}
View example

\begin{verbatim}
Take #underline(
  stroke: 1.5pt + red,
  offset: 2pt,
  [care],
)
\end{verbatim}

\includegraphics[width=5in,height=\textheight,keepaspectratio]{/assets/docs/tbLKc9iYaghdhC9NcJaJOQAAAAAAAAAA.png}

\subsubsection{\texorpdfstring{\texttt{\ offset\ }}{ offset }}\label{parameters-offset}

\href{/docs/reference/foundations/auto/}{auto} {or}
\href{/docs/reference/layout/length/}{length}

{{ Settable }}

\phantomsection\label{parameters-offset-settable-tooltip}
Settable parameters can be customized for all following uses of the
function with a \texttt{\ set\ } rule.

The position of the line relative to the baseline, read from the font
tables if \texttt{\ }{\texttt{\ auto\ }}\texttt{\ } .

Default: \texttt{\ }{\texttt{\ auto\ }}\texttt{\ }

\includesvg[width=0.16667in,height=0.16667in]{/assets/icons/16-arrow-right.svg}
View example

\begin{verbatim}
#underline(offset: 5pt)[
  The Tale Of A Faraway Line I
]
\end{verbatim}

\includegraphics[width=5in,height=\textheight,keepaspectratio]{/assets/docs/p2tUWXcYq-E_ZbDtwzCDrAAAAAAAAAAA.png}

\subsubsection{\texorpdfstring{\texttt{\ extent\ }}{ extent }}\label{parameters-extent}

\href{/docs/reference/layout/length/}{length}

{{ Settable }}

\phantomsection\label{parameters-extent-settable-tooltip}
Settable parameters can be customized for all following uses of the
function with a \texttt{\ set\ } rule.

The amount by which to extend the line beyond (or within if negative)
the content.

Default: \texttt{\ }{\texttt{\ 0pt\ }}\texttt{\ }

\includesvg[width=0.16667in,height=0.16667in]{/assets/icons/16-arrow-right.svg}
View example

\begin{verbatim}
#align(center,
  underline(extent: 2pt)[Chapter 1]
)
\end{verbatim}

\includegraphics[width=5in,height=\textheight,keepaspectratio]{/assets/docs/tbT2BOLPtcXW-alQPb8q6wAAAAAAAAAA.png}

\subsubsection{\texorpdfstring{\texttt{\ evade\ }}{ evade }}\label{parameters-evade}

\href{/docs/reference/foundations/bool/}{bool}

{{ Settable }}

\phantomsection\label{parameters-evade-settable-tooltip}
Settable parameters can be customized for all following uses of the
function with a \texttt{\ set\ } rule.

Whether the line skips sections in which it would collide with the
glyphs.

Default: \texttt{\ }{\texttt{\ true\ }}\texttt{\ }

\includesvg[width=0.16667in,height=0.16667in]{/assets/icons/16-arrow-right.svg}
View example

\begin{verbatim}
This #underline(evade: true)[is great].
This #underline(evade: false)[is less great].
\end{verbatim}

\includegraphics[width=5in,height=\textheight,keepaspectratio]{/assets/docs/PaJc2qUpoh1s97E6NZYz0QAAAAAAAAAA.png}

\subsubsection{\texorpdfstring{\texttt{\ background\ }}{ background }}\label{parameters-background}

\href{/docs/reference/foundations/bool/}{bool}

{{ Settable }}

\phantomsection\label{parameters-background-settable-tooltip}
Settable parameters can be customized for all following uses of the
function with a \texttt{\ set\ } rule.

Whether the line is placed behind the content it underlines.

Default: \texttt{\ }{\texttt{\ false\ }}\texttt{\ }

\includesvg[width=0.16667in,height=0.16667in]{/assets/icons/16-arrow-right.svg}
View example

\begin{verbatim}
#set underline(stroke: (thickness: 1em, paint: maroon, cap: "round"))
#underline(background: true)[This is stylized.] \
#underline(background: false)[This is partially hidden.]
\end{verbatim}

\includegraphics[width=5in,height=\textheight,keepaspectratio]{/assets/docs/W98M7AlnFoSVnlt9g5bIsAAAAAAAAAAA.png}

\subsubsection{\texorpdfstring{\texttt{\ body\ }}{ body }}\label{parameters-body}

\href{/docs/reference/foundations/content/}{content}

{Required} {{ Positional }}

\phantomsection\label{parameters-body-positional-tooltip}
Positional parameters are specified in order, without names.

The content to underline.

\href{/docs/reference/text/text/}{\pandocbounded{\includesvg[keepaspectratio]{/assets/icons/16-arrow-right.svg}}}

{ Text } { Previous page }

\href{/docs/reference/text/upper/}{\pandocbounded{\includesvg[keepaspectratio]{/assets/icons/16-arrow-right.svg}}}

{ Uppercase } { Next page }


\title{typst.app/docs/reference/text/super}

\begin{itemize}
\tightlist
\item
  \href{/docs}{\includesvg[width=0.16667in,height=0.16667in]{/assets/icons/16-docs-dark.svg}}
\item
  \includesvg[width=0.16667in,height=0.16667in]{/assets/icons/16-arrow-right.svg}
\item
  \href{/docs/reference/}{Reference}
\item
  \includesvg[width=0.16667in,height=0.16667in]{/assets/icons/16-arrow-right.svg}
\item
  \href{/docs/reference/text/}{Text}
\item
  \includesvg[width=0.16667in,height=0.16667in]{/assets/icons/16-arrow-right.svg}
\item
  \href{/docs/reference/text/super/}{Superscript}
\end{itemize}

\section{\texorpdfstring{\texttt{\ super\ } {{ Element
}}}{ super   Element }}\label{summary}

\phantomsection\label{element-tooltip}
Element functions can be customized with \texttt{\ set\ } and
\texttt{\ show\ } rules.

Renders text in superscript.

The text is rendered smaller and its baseline is raised.

\subsection{Example}\label{example}

\begin{verbatim}
1#super[st] try!
\end{verbatim}

\includegraphics[width=5in,height=\textheight,keepaspectratio]{/assets/docs/052zwKrkvVHtZVzW65WFdQAAAAAAAAAA.png}

\subsection{\texorpdfstring{{ Parameters
}}{ Parameters }}\label{parameters}

\phantomsection\label{parameters-tooltip}
Parameters are the inputs to a function. They are specified in
parentheses after the function name.

{ super } (

{ \hyperref[parameters-typographic]{typographic :}
\href{/docs/reference/foundations/bool/}{bool} , } {
\hyperref[parameters-baseline]{baseline :}
\href{/docs/reference/layout/length/}{length} , } {
\hyperref[parameters-size]{size :}
\href{/docs/reference/layout/length/}{length} , } {
\href{/docs/reference/foundations/content/}{content} , }

) -\textgreater{} \href{/docs/reference/foundations/content/}{content}

\subsubsection{\texorpdfstring{\texttt{\ typographic\ }}{ typographic }}\label{parameters-typographic}

\href{/docs/reference/foundations/bool/}{bool}

{{ Settable }}

\phantomsection\label{parameters-typographic-settable-tooltip}
Settable parameters can be customized for all following uses of the
function with a \texttt{\ set\ } rule.

Whether to prefer the dedicated superscript characters of the font.

If this is enabled, Typst first tries to transform the text to
superscript codepoints. If that fails, it falls back to rendering raised
and shrunk normal letters.

Default: \texttt{\ }{\texttt{\ true\ }}\texttt{\ }

\includesvg[width=0.16667in,height=0.16667in]{/assets/icons/16-arrow-right.svg}
View example

\begin{verbatim}
N#super(typographic: true)[1]
N#super(typographic: false)[1]
\end{verbatim}

\includegraphics[width=5in,height=\textheight,keepaspectratio]{/assets/docs/1_zKQkbZObDWVLT4k-2LKQAAAAAAAAAA.png}

\subsubsection{\texorpdfstring{\texttt{\ baseline\ }}{ baseline }}\label{parameters-baseline}

\href{/docs/reference/layout/length/}{length}

{{ Settable }}

\phantomsection\label{parameters-baseline-settable-tooltip}
Settable parameters can be customized for all following uses of the
function with a \texttt{\ set\ } rule.

The baseline shift for synthetic superscripts. Does not apply if
\texttt{\ typographic\ } is true and the font has superscript codepoints
for the given \texttt{\ body\ } .

Default:
\texttt{\ }{\texttt{\ -\ }}\texttt{\ }{\texttt{\ 0.5em\ }}\texttt{\ }

\subsubsection{\texorpdfstring{\texttt{\ size\ }}{ size }}\label{parameters-size}

\href{/docs/reference/layout/length/}{length}

{{ Settable }}

\phantomsection\label{parameters-size-settable-tooltip}
Settable parameters can be customized for all following uses of the
function with a \texttt{\ set\ } rule.

The font size for synthetic superscripts. Does not apply if
\texttt{\ typographic\ } is true and the font has superscript codepoints
for the given \texttt{\ body\ } .

Default: \texttt{\ }{\texttt{\ 0.6em\ }}\texttt{\ }

\subsubsection{\texorpdfstring{\texttt{\ body\ }}{ body }}\label{parameters-body}

\href{/docs/reference/foundations/content/}{content}

{Required} {{ Positional }}

\phantomsection\label{parameters-body-positional-tooltip}
Positional parameters are specified in order, without names.

The text to display in superscript.

\href{/docs/reference/text/sub/}{\pandocbounded{\includesvg[keepaspectratio]{/assets/icons/16-arrow-right.svg}}}

{ Subscript } { Previous page }

\href{/docs/reference/text/text/}{\pandocbounded{\includesvg[keepaspectratio]{/assets/icons/16-arrow-right.svg}}}

{ Text } { Next page }


\title{typst.app/docs/reference/text/raw}

\begin{itemize}
\tightlist
\item
  \href{/docs}{\includesvg[width=0.16667in,height=0.16667in]{/assets/icons/16-docs-dark.svg}}
\item
  \includesvg[width=0.16667in,height=0.16667in]{/assets/icons/16-arrow-right.svg}
\item
  \href{/docs/reference/}{Reference}
\item
  \includesvg[width=0.16667in,height=0.16667in]{/assets/icons/16-arrow-right.svg}
\item
  \href{/docs/reference/text/}{Text}
\item
  \includesvg[width=0.16667in,height=0.16667in]{/assets/icons/16-arrow-right.svg}
\item
  \href{/docs/reference/text/raw/}{Raw Text / Code}
\end{itemize}

\section{\texorpdfstring{\texttt{\ raw\ } {{ Element
}}}{ raw   Element }}\label{summary}

\phantomsection\label{element-tooltip}
Element functions can be customized with \texttt{\ set\ } and
\texttt{\ show\ } rules.

Raw text with optional syntax highlighting.

Displays the text verbatim and in a monospace font. This is typically
used to embed computer code into your document.

\subsection{Example}\label{example}

\begin{verbatim}
Adding `rbx` to `rcx` gives
the desired result.

What is ```rust fn main()``` in Rust
would be ```c int main()``` in C.

```rust
fn main() {
    println!("Hello World!");
}
```

This has ``` `backticks` ``` in it
(but the spaces are trimmed). And
``` here``` the leading space is
also trimmed.
\end{verbatim}

\includegraphics[width=5in,height=\textheight,keepaspectratio]{/assets/docs/HG5qpETGRO7ndBI1Qrek9gAAAAAAAAAA.png}

You can also construct a
\href{/docs/reference/text/raw/}{\texttt{\ raw\ }} element
programmatically from a string (and provide the language tag via the
optional
\href{/docs/reference/text/raw/\#parameters-lang}{\texttt{\ lang\ }}
argument).

\begin{verbatim}
#raw("fn " + "main() {}", lang: "rust")
\end{verbatim}

\includegraphics[width=5in,height=\textheight,keepaspectratio]{/assets/docs/MNABiMKxTxPPaXzIwfuPPQAAAAAAAAAA.png}

\subsection{Syntax}\label{syntax}

This function also has dedicated syntax. You can enclose text in 1 or 3+
backticks ( \texttt{\ \textasciigrave{}\ } ) to make it raw. Two
backticks produce empty raw text. This works both in markup and code.

When you use three or more backticks, you can additionally specify a
language tag for syntax highlighting directly after the opening
backticks. Within raw blocks, everything (except for the language tag,
if applicable) is rendered as is, in particular, there are no escape
sequences.

The language tag is an identifier that directly follows the opening
backticks only if there are three or more backticks. If your text starts
with something that looks like an identifier, but no syntax highlighting
is needed, start the text with a single space (which will be trimmed) or
use the single backtick syntax. If your text should start or end with a
backtick, put a space before or after it (it will be trimmed).

\subsection{\texorpdfstring{{ Parameters
}}{ Parameters }}\label{parameters}

\phantomsection\label{parameters-tooltip}
Parameters are the inputs to a function. They are specified in
parentheses after the function name.

{ raw } (

{ \href{/docs/reference/foundations/str/}{str} , } {
\hyperref[parameters-block]{block :}
\href{/docs/reference/foundations/bool/}{bool} , } {
\hyperref[parameters-lang]{lang :}
\href{/docs/reference/foundations/none/}{none}
\href{/docs/reference/foundations/str/}{str} , } {
\hyperref[parameters-align]{align :}
\href{/docs/reference/layout/alignment/}{alignment} , } {
\hyperref[parameters-syntaxes]{syntaxes :}
\href{/docs/reference/foundations/str/}{str}
\href{/docs/reference/foundations/array/}{array} , } {
\hyperref[parameters-theme]{theme :}
\href{/docs/reference/foundations/none/}{none}
\href{/docs/reference/foundations/auto/}{auto}
\href{/docs/reference/foundations/str/}{str} , } {
\hyperref[parameters-tab-size]{tab-size :}
\href{/docs/reference/foundations/int/}{int} , }

) -\textgreater{} \href{/docs/reference/foundations/content/}{content}

\subsubsection{\texorpdfstring{\texttt{\ text\ }}{ text }}\label{parameters-text}

\href{/docs/reference/foundations/str/}{str}

{Required} {{ Positional }}

\phantomsection\label{parameters-text-positional-tooltip}
Positional parameters are specified in order, without names.

The raw text.

You can also use raw blocks creatively to create custom syntaxes for
your automations.

\includesvg[width=0.16667in,height=0.16667in]{/assets/icons/16-arrow-right.svg}
View example

\begin{verbatim}
// Parse numbers in raw blocks with the
// `mydsl` tag and sum them up.
#show raw.where(lang: "mydsl"): it => {
  let sum = 0
  for part in it.text.split("+") {
    sum += int(part.trim())
  }
  sum
}

```mydsl
1 + 2 + 3 + 4 + 5
```
\end{verbatim}

\includegraphics[width=5in,height=\textheight,keepaspectratio]{/assets/docs/6VperjQoP8Ey0LiUk5m0HQAAAAAAAAAA.png}

\subsubsection{\texorpdfstring{\texttt{\ block\ }}{ block }}\label{parameters-block}

\href{/docs/reference/foundations/bool/}{bool}

{{ Settable }}

\phantomsection\label{parameters-block-settable-tooltip}
Settable parameters can be customized for all following uses of the
function with a \texttt{\ set\ } rule.

Whether the raw text is displayed as a separate block.

In markup mode, using one-backtick notation makes this
\texttt{\ }{\texttt{\ false\ }}\texttt{\ } . Using three-backtick
notation makes it \texttt{\ }{\texttt{\ true\ }}\texttt{\ } if the
enclosed content contains at least one line break.

Default: \texttt{\ }{\texttt{\ false\ }}\texttt{\ }

\includesvg[width=0.16667in,height=0.16667in]{/assets/icons/16-arrow-right.svg}
View example

\begin{verbatim}
// Display inline code in a small box
// that retains the correct baseline.
#show raw.where(block: false): box.with(
  fill: luma(240),
  inset: (x: 3pt, y: 0pt),
  outset: (y: 3pt),
  radius: 2pt,
)

// Display block code in a larger block
// with more padding.
#show raw.where(block: true): block.with(
  fill: luma(240),
  inset: 10pt,
  radius: 4pt,
)

With `rg`, you can search through your files quickly.
This example searches the current directory recursively
for the text `Hello World`:

```bash
rg "Hello World"
```
\end{verbatim}

\includegraphics[width=5in,height=\textheight,keepaspectratio]{/assets/docs/PgXCmmr2Cn53ZnpWQOnMwwAAAAAAAAAA.png}

\subsubsection{\texorpdfstring{\texttt{\ lang\ }}{ lang }}\label{parameters-lang}

\href{/docs/reference/foundations/none/}{none} {or}
\href{/docs/reference/foundations/str/}{str}

{{ Settable }}

\phantomsection\label{parameters-lang-settable-tooltip}
Settable parameters can be customized for all following uses of the
function with a \texttt{\ set\ } rule.

The language to syntax-highlight in.

Apart from typical language tags known from Markdown, this supports the
\texttt{\ }{\texttt{\ "typ"\ }}\texttt{\ } ,
\texttt{\ }{\texttt{\ "typc"\ }}\texttt{\ } , and
\texttt{\ }{\texttt{\ "typm"\ }}\texttt{\ } tags for
\href{/docs/reference/syntax/\#markup}{Typst markup} ,
\href{/docs/reference/syntax/\#code}{Typst code} , and
\href{/docs/reference/syntax/\#math}{Typst math} , respectively.

Default: \texttt{\ }{\texttt{\ none\ }}\texttt{\ }

\includesvg[width=0.16667in,height=0.16667in]{/assets/icons/16-arrow-right.svg}
View example

\begin{verbatim}
```typ
This is *Typst!*
```

This is ```typ also *Typst*```, but inline!
\end{verbatim}

\includegraphics[width=5in,height=\textheight,keepaspectratio]{/assets/docs/bjU3PMlFs9msUi72QThHnAAAAAAAAAAA.png}

\subsubsection{\texorpdfstring{\texttt{\ align\ }}{ align }}\label{parameters-align}

\href{/docs/reference/layout/alignment/}{alignment}

{{ Settable }}

\phantomsection\label{parameters-align-settable-tooltip}
Settable parameters can be customized for all following uses of the
function with a \texttt{\ set\ } rule.

The horizontal alignment that each line in a raw block should have. This
option is ignored if this is not a raw block (if specified
\texttt{\ block:\ false\ } or single backticks were used in markup
mode).

By default, this is set to \texttt{\ start\ } , meaning that raw text is
aligned towards the start of the text direction inside the block by
default, regardless of the current context\textquotesingle s alignment
(allowing you to center the raw block itself without centering the text
inside it, for example).

Default: \texttt{\ start\ }

\includesvg[width=0.16667in,height=0.16667in]{/assets/icons/16-arrow-right.svg}
View example

\begin{verbatim}
#set raw(align: center)

```typc
let f(x) = x
code = "centered"
```
\end{verbatim}

\includegraphics[width=5in,height=\textheight,keepaspectratio]{/assets/docs/QoY61HWjc7MIUABTr8mvwwAAAAAAAAAA.png}

\subsubsection{\texorpdfstring{\texttt{\ syntaxes\ }}{ syntaxes }}\label{parameters-syntaxes}

\href{/docs/reference/foundations/str/}{str} {or}
\href{/docs/reference/foundations/array/}{array}

{{ Settable }}

\phantomsection\label{parameters-syntaxes-settable-tooltip}
Settable parameters can be customized for all following uses of the
function with a \texttt{\ set\ } rule.

One or multiple additional syntax definitions to load. The syntax
definitions should be in the
\href{https://www.sublimetext.com/docs/syntax.html}{\texttt{\ sublime-syntax\ }
file format} .

Default:
\texttt{\ }{\texttt{\ (\ }}\texttt{\ }{\texttt{\ )\ }}\texttt{\ }

\includesvg[width=0.16667in,height=0.16667in]{/assets/icons/16-arrow-right.svg}
View example

\begin{verbatim}
#set raw(syntaxes: "SExpressions.sublime-syntax")

```sexp
(defun factorial (x)
  (if (zerop x)
    ; with a comment
    1
    (* x (factorial (- x 1)))))
```
\end{verbatim}

\includegraphics[width=5in,height=\textheight,keepaspectratio]{/assets/docs/f1wEtKdjbuwy-LVNGIZ_igAAAAAAAAAA.png}

\subsubsection{\texorpdfstring{\texttt{\ theme\ }}{ theme }}\label{parameters-theme}

\href{/docs/reference/foundations/none/}{none} {or}
\href{/docs/reference/foundations/auto/}{auto} {or}
\href{/docs/reference/foundations/str/}{str}

{{ Settable }}

\phantomsection\label{parameters-theme-settable-tooltip}
Settable parameters can be customized for all following uses of the
function with a \texttt{\ set\ } rule.

The theme to use for syntax highlighting. Theme files should be in the
\href{https://www.sublimetext.com/docs/color_schemes_tmtheme.html}{\texttt{\ tmTheme\ }
file format} .

Applying a theme only affects the color of specifically highlighted
text. It does not consider the theme\textquotesingle s foreground and
background properties, so that you retain control over the color of raw
text. You can apply the foreground color yourself with the
\href{/docs/reference/text/text/}{\texttt{\ text\ }} function and the
background with a
\href{/docs/reference/layout/block/\#parameters-fill}{filled block} .
You could also use the
\href{/docs/reference/data-loading/xml/}{\texttt{\ xml\ }} function to
extract these properties from the theme.

Additionally, you can set the theme to
\texttt{\ }{\texttt{\ none\ }}\texttt{\ } to disable highlighting.

Default: \texttt{\ }{\texttt{\ auto\ }}\texttt{\ }

\includesvg[width=0.16667in,height=0.16667in]{/assets/icons/16-arrow-right.svg}
View example

\begin{verbatim}
#set raw(theme: "halcyon.tmTheme")
#show raw: it => block(
  fill: rgb("#1d2433"),
  inset: 8pt,
  radius: 5pt,
  text(fill: rgb("#a2aabc"), it)
)

```typ
= Chapter 1
#let hi = "Hello World"
```
\end{verbatim}

\includegraphics[width=5in,height=\textheight,keepaspectratio]{/assets/docs/_3ndU0y1KsOpDAMv999GwwAAAAAAAAAA.png}

\subsubsection{\texorpdfstring{\texttt{\ tab-size\ }}{ tab-size }}\label{parameters-tab-size}

\href{/docs/reference/foundations/int/}{int}

{{ Settable }}

\phantomsection\label{parameters-tab-size-settable-tooltip}
Settable parameters can be customized for all following uses of the
function with a \texttt{\ set\ } rule.

The size for a tab stop in spaces. A tab is replaced with enough spaces
to align with the next multiple of the size.

Default: \texttt{\ }{\texttt{\ 2\ }}\texttt{\ }

\includesvg[width=0.16667in,height=0.16667in]{/assets/icons/16-arrow-right.svg}
View example

\begin{verbatim}
#set raw(tab-size: 8)
```tsv
Year    Month   Day
2000    2   3
2001    2   1
2002    3   10
```
\end{verbatim}

\includegraphics[width=5in,height=\textheight,keepaspectratio]{/assets/docs/OAN98lLQ4wUhrTrjbVCTywAAAAAAAAAA.png}

\subsection{\texorpdfstring{{ Definitions
}}{ Definitions }}\label{definitions}

\phantomsection\label{definitions-tooltip}
Functions and types and can have associated definitions. These are
accessed by specifying the function or type, followed by a period, and
then the definition\textquotesingle s name.

\subsubsection{\texorpdfstring{\texttt{\ line\ } {{ Element
}}}{ line   Element }}\label{definitions-line}

\phantomsection\label{definitions-line-element-tooltip}
Element functions can be customized with \texttt{\ set\ } and
\texttt{\ show\ } rules.

A highlighted line of raw text.

This is a helper element that is synthesized by
\href{/docs/reference/text/raw/}{\texttt{\ raw\ }} elements.

It allows you to access various properties of the line, such as the line
number, the raw non-highlighted text, the highlighted text, and whether
it is the first or last line of the raw block.

raw { . } { line } (

{ \href{/docs/reference/foundations/int/}{int} , } {
\href{/docs/reference/foundations/int/}{int} , } {
\href{/docs/reference/foundations/str/}{str} , } {
\href{/docs/reference/foundations/content/}{content} , }

) -\textgreater{} \href{/docs/reference/foundations/content/}{content}

\paragraph{\texorpdfstring{\texttt{\ number\ }}{ number }}\label{definitions-line-number}

\href{/docs/reference/foundations/int/}{int}

{Required} {{ Positional }}

\phantomsection\label{definitions-line-number-positional-tooltip}
Positional parameters are specified in order, without names.

The line number of the raw line inside of the raw block, starts at 1.

\paragraph{\texorpdfstring{\texttt{\ count\ }}{ count }}\label{definitions-line-count}

\href{/docs/reference/foundations/int/}{int}

{Required} {{ Positional }}

\phantomsection\label{definitions-line-count-positional-tooltip}
Positional parameters are specified in order, without names.

The total number of lines in the raw block.

\paragraph{\texorpdfstring{\texttt{\ text\ }}{ text }}\label{definitions-line-text}

\href{/docs/reference/foundations/str/}{str}

{Required} {{ Positional }}

\phantomsection\label{definitions-line-text-positional-tooltip}
Positional parameters are specified in order, without names.

The line of raw text.

\paragraph{\texorpdfstring{\texttt{\ body\ }}{ body }}\label{definitions-line-body}

\href{/docs/reference/foundations/content/}{content}

{Required} {{ Positional }}

\phantomsection\label{definitions-line-body-positional-tooltip}
Positional parameters are specified in order, without names.

The highlighted raw text.

\href{/docs/reference/text/overline/}{\pandocbounded{\includesvg[keepaspectratio]{/assets/icons/16-arrow-right.svg}}}

{ Overline } { Previous page }

\href{/docs/reference/text/smallcaps/}{\pandocbounded{\includesvg[keepaspectratio]{/assets/icons/16-arrow-right.svg}}}

{ Small Capitals } { Next page }


\title{typst.app/docs/reference/text/lower}

\begin{itemize}
\tightlist
\item
  \href{/docs}{\includesvg[width=0.16667in,height=0.16667in]{/assets/icons/16-docs-dark.svg}}
\item
  \includesvg[width=0.16667in,height=0.16667in]{/assets/icons/16-arrow-right.svg}
\item
  \href{/docs/reference/}{Reference}
\item
  \includesvg[width=0.16667in,height=0.16667in]{/assets/icons/16-arrow-right.svg}
\item
  \href{/docs/reference/text/}{Text}
\item
  \includesvg[width=0.16667in,height=0.16667in]{/assets/icons/16-arrow-right.svg}
\item
  \href{/docs/reference/text/lower/}{Lowercase}
\end{itemize}

\section{\texorpdfstring{\texttt{\ lower\ }}{ lower }}\label{summary}

Converts a string or content to lowercase.

\subsection{Example}\label{example}

\begin{verbatim}
#lower("ABC") \
#lower[*My Text*] \
#lower[already low]
\end{verbatim}

\includegraphics[width=5in,height=\textheight,keepaspectratio]{/assets/docs/zbgdZcwg4Knc-ePylT0zpQAAAAAAAAAA.png}

\subsection{\texorpdfstring{{ Parameters
}}{ Parameters }}\label{parameters}

\phantomsection\label{parameters-tooltip}
Parameters are the inputs to a function. They are specified in
parentheses after the function name.

{ lower } (

{ \href{/docs/reference/foundations/str/}{str}
\href{/docs/reference/foundations/content/}{content} }

) -\textgreater{} \href{/docs/reference/foundations/str/}{str}
\href{/docs/reference/foundations/content/}{content}

\subsubsection{\texorpdfstring{\texttt{\ text\ }}{ text }}\label{parameters-text}

\href{/docs/reference/foundations/str/}{str} {or}
\href{/docs/reference/foundations/content/}{content}

{Required} {{ Positional }}

\phantomsection\label{parameters-text-positional-tooltip}
Positional parameters are specified in order, without names.

The text to convert to lowercase.

\href{/docs/reference/text/lorem/}{\pandocbounded{\includesvg[keepaspectratio]{/assets/icons/16-arrow-right.svg}}}

{ Lorem } { Previous page }

\href{/docs/reference/text/overline/}{\pandocbounded{\includesvg[keepaspectratio]{/assets/icons/16-arrow-right.svg}}}

{ Overline } { Next page }


\title{typst.app/docs/reference/text/sub}

\begin{itemize}
\tightlist
\item
  \href{/docs}{\includesvg[width=0.16667in,height=0.16667in]{/assets/icons/16-docs-dark.svg}}
\item
  \includesvg[width=0.16667in,height=0.16667in]{/assets/icons/16-arrow-right.svg}
\item
  \href{/docs/reference/}{Reference}
\item
  \includesvg[width=0.16667in,height=0.16667in]{/assets/icons/16-arrow-right.svg}
\item
  \href{/docs/reference/text/}{Text}
\item
  \includesvg[width=0.16667in,height=0.16667in]{/assets/icons/16-arrow-right.svg}
\item
  \href{/docs/reference/text/sub/}{Subscript}
\end{itemize}

\section{\texorpdfstring{\texttt{\ sub\ } {{ Element
}}}{ sub   Element }}\label{summary}

\phantomsection\label{element-tooltip}
Element functions can be customized with \texttt{\ set\ } and
\texttt{\ show\ } rules.

Renders text in subscript.

The text is rendered smaller and its baseline is lowered.

\subsection{Example}\label{example}

\begin{verbatim}
Revenue#sub[yearly]
\end{verbatim}

\includegraphics[width=5in,height=\textheight,keepaspectratio]{/assets/docs/q6m3B3bVOLKPuJFIogqIMwAAAAAAAAAA.png}

\subsection{\texorpdfstring{{ Parameters
}}{ Parameters }}\label{parameters}

\phantomsection\label{parameters-tooltip}
Parameters are the inputs to a function. They are specified in
parentheses after the function name.

{ sub } (

{ \hyperref[parameters-typographic]{typographic :}
\href{/docs/reference/foundations/bool/}{bool} , } {
\hyperref[parameters-baseline]{baseline :}
\href{/docs/reference/layout/length/}{length} , } {
\hyperref[parameters-size]{size :}
\href{/docs/reference/layout/length/}{length} , } {
\href{/docs/reference/foundations/content/}{content} , }

) -\textgreater{} \href{/docs/reference/foundations/content/}{content}

\subsubsection{\texorpdfstring{\texttt{\ typographic\ }}{ typographic }}\label{parameters-typographic}

\href{/docs/reference/foundations/bool/}{bool}

{{ Settable }}

\phantomsection\label{parameters-typographic-settable-tooltip}
Settable parameters can be customized for all following uses of the
function with a \texttt{\ set\ } rule.

Whether to prefer the dedicated subscript characters of the font.

If this is enabled, Typst first tries to transform the text to subscript
codepoints. If that fails, it falls back to rendering lowered and shrunk
normal letters.

Default: \texttt{\ }{\texttt{\ true\ }}\texttt{\ }

\includesvg[width=0.16667in,height=0.16667in]{/assets/icons/16-arrow-right.svg}
View example

\begin{verbatim}
N#sub(typographic: true)[1]
N#sub(typographic: false)[1]
\end{verbatim}

\includegraphics[width=5in,height=\textheight,keepaspectratio]{/assets/docs/eGuJ4coPHcIbozTvGKvULAAAAAAAAAAA.png}

\subsubsection{\texorpdfstring{\texttt{\ baseline\ }}{ baseline }}\label{parameters-baseline}

\href{/docs/reference/layout/length/}{length}

{{ Settable }}

\phantomsection\label{parameters-baseline-settable-tooltip}
Settable parameters can be customized for all following uses of the
function with a \texttt{\ set\ } rule.

The baseline shift for synthetic subscripts. Does not apply if
\texttt{\ typographic\ } is true and the font has subscript codepoints
for the given \texttt{\ body\ } .

Default: \texttt{\ }{\texttt{\ 0.2em\ }}\texttt{\ }

\subsubsection{\texorpdfstring{\texttt{\ size\ }}{ size }}\label{parameters-size}

\href{/docs/reference/layout/length/}{length}

{{ Settable }}

\phantomsection\label{parameters-size-settable-tooltip}
Settable parameters can be customized for all following uses of the
function with a \texttt{\ set\ } rule.

The font size for synthetic subscripts. Does not apply if
\texttt{\ typographic\ } is true and the font has subscript codepoints
for the given \texttt{\ body\ } .

Default: \texttt{\ }{\texttt{\ 0.6em\ }}\texttt{\ }

\subsubsection{\texorpdfstring{\texttt{\ body\ }}{ body }}\label{parameters-body}

\href{/docs/reference/foundations/content/}{content}

{Required} {{ Positional }}

\phantomsection\label{parameters-body-positional-tooltip}
Positional parameters are specified in order, without names.

The text to display in subscript.

\href{/docs/reference/text/strike/}{\pandocbounded{\includesvg[keepaspectratio]{/assets/icons/16-arrow-right.svg}}}

{ Strikethrough } { Previous page }

\href{/docs/reference/text/super/}{\pandocbounded{\includesvg[keepaspectratio]{/assets/icons/16-arrow-right.svg}}}

{ Superscript } { Next page }


\title{typst.app/docs/reference/text/overline}

\begin{itemize}
\tightlist
\item
  \href{/docs}{\includesvg[width=0.16667in,height=0.16667in]{/assets/icons/16-docs-dark.svg}}
\item
  \includesvg[width=0.16667in,height=0.16667in]{/assets/icons/16-arrow-right.svg}
\item
  \href{/docs/reference/}{Reference}
\item
  \includesvg[width=0.16667in,height=0.16667in]{/assets/icons/16-arrow-right.svg}
\item
  \href{/docs/reference/text/}{Text}
\item
  \includesvg[width=0.16667in,height=0.16667in]{/assets/icons/16-arrow-right.svg}
\item
  \href{/docs/reference/text/overline/}{Overline}
\end{itemize}

\section{\texorpdfstring{\texttt{\ overline\ } {{ Element
}}}{ overline   Element }}\label{summary}

\phantomsection\label{element-tooltip}
Element functions can be customized with \texttt{\ set\ } and
\texttt{\ show\ } rules.

Adds a line over text.

\subsection{Example}\label{example}

\begin{verbatim}
#overline[A line over text.]
\end{verbatim}

\includegraphics[width=5in,height=\textheight,keepaspectratio]{/assets/docs/BQmJqK4pMIkZOu3QEFxsZAAAAAAAAAAA.png}

\subsection{\texorpdfstring{{ Parameters
}}{ Parameters }}\label{parameters}

\phantomsection\label{parameters-tooltip}
Parameters are the inputs to a function. They are specified in
parentheses after the function name.

{ overline } (

{ \hyperref[parameters-stroke]{stroke :}
\href{/docs/reference/foundations/auto/}{auto}
\href{/docs/reference/layout/length/}{length}
\href{/docs/reference/visualize/color/}{color}
\href{/docs/reference/visualize/gradient/}{gradient}
\href{/docs/reference/visualize/stroke/}{stroke}
\href{/docs/reference/visualize/pattern/}{pattern}
\href{/docs/reference/foundations/dictionary/}{dictionary} , } {
\hyperref[parameters-offset]{offset :}
\href{/docs/reference/foundations/auto/}{auto}
\href{/docs/reference/layout/length/}{length} , } {
\hyperref[parameters-extent]{extent :}
\href{/docs/reference/layout/length/}{length} , } {
\hyperref[parameters-evade]{evade :}
\href{/docs/reference/foundations/bool/}{bool} , } {
\hyperref[parameters-background]{background :}
\href{/docs/reference/foundations/bool/}{bool} , } {
\href{/docs/reference/foundations/content/}{content} , }

) -\textgreater{} \href{/docs/reference/foundations/content/}{content}

\subsubsection{\texorpdfstring{\texttt{\ stroke\ }}{ stroke }}\label{parameters-stroke}

\href{/docs/reference/foundations/auto/}{auto} {or}
\href{/docs/reference/layout/length/}{length} {or}
\href{/docs/reference/visualize/color/}{color} {or}
\href{/docs/reference/visualize/gradient/}{gradient} {or}
\href{/docs/reference/visualize/stroke/}{stroke} {or}
\href{/docs/reference/visualize/pattern/}{pattern} {or}
\href{/docs/reference/foundations/dictionary/}{dictionary}

{{ Settable }}

\phantomsection\label{parameters-stroke-settable-tooltip}
Settable parameters can be customized for all following uses of the
function with a \texttt{\ set\ } rule.

How to \href{/docs/reference/visualize/stroke/}{stroke} the line.

If set to \texttt{\ }{\texttt{\ auto\ }}\texttt{\ } , takes on the
text\textquotesingle s color and a thickness defined in the current
font.

Default: \texttt{\ }{\texttt{\ auto\ }}\texttt{\ }

\includesvg[width=0.16667in,height=0.16667in]{/assets/icons/16-arrow-right.svg}
View example

\begin{verbatim}
#set text(fill: olive)
#overline(
  stroke: green.darken(20%),
  offset: -12pt,
  [The Forest Theme],
)
\end{verbatim}

\includegraphics[width=5in,height=\textheight,keepaspectratio]{/assets/docs/jXEAZxd9NFnCtgcbDVlzIQAAAAAAAAAA.png}

\subsubsection{\texorpdfstring{\texttt{\ offset\ }}{ offset }}\label{parameters-offset}

\href{/docs/reference/foundations/auto/}{auto} {or}
\href{/docs/reference/layout/length/}{length}

{{ Settable }}

\phantomsection\label{parameters-offset-settable-tooltip}
Settable parameters can be customized for all following uses of the
function with a \texttt{\ set\ } rule.

The position of the line relative to the baseline. Read from the font
tables if \texttt{\ }{\texttt{\ auto\ }}\texttt{\ } .

Default: \texttt{\ }{\texttt{\ auto\ }}\texttt{\ }

\includesvg[width=0.16667in,height=0.16667in]{/assets/icons/16-arrow-right.svg}
View example

\begin{verbatim}
#overline(offset: -1.2em)[
  The Tale Of A Faraway Line II
]
\end{verbatim}

\includegraphics[width=5in,height=\textheight,keepaspectratio]{/assets/docs/AUBIhMOFPefmpe2mV6TTrgAAAAAAAAAA.png}

\subsubsection{\texorpdfstring{\texttt{\ extent\ }}{ extent }}\label{parameters-extent}

\href{/docs/reference/layout/length/}{length}

{{ Settable }}

\phantomsection\label{parameters-extent-settable-tooltip}
Settable parameters can be customized for all following uses of the
function with a \texttt{\ set\ } rule.

The amount by which to extend the line beyond (or within if negative)
the content.

Default: \texttt{\ }{\texttt{\ 0pt\ }}\texttt{\ }

\includesvg[width=0.16667in,height=0.16667in]{/assets/icons/16-arrow-right.svg}
View example

\begin{verbatim}
#set overline(extent: 4pt)
#set underline(extent: 4pt)
#overline(underline[Typography Today])
\end{verbatim}

\includegraphics[width=5in,height=\textheight,keepaspectratio]{/assets/docs/11dFhng73-PPcouY1kGuxAAAAAAAAAAA.png}

\subsubsection{\texorpdfstring{\texttt{\ evade\ }}{ evade }}\label{parameters-evade}

\href{/docs/reference/foundations/bool/}{bool}

{{ Settable }}

\phantomsection\label{parameters-evade-settable-tooltip}
Settable parameters can be customized for all following uses of the
function with a \texttt{\ set\ } rule.

Whether the line skips sections in which it would collide with the
glyphs.

Default: \texttt{\ }{\texttt{\ true\ }}\texttt{\ }

\includesvg[width=0.16667in,height=0.16667in]{/assets/icons/16-arrow-right.svg}
View example

\begin{verbatim}
#overline(
  evade: false,
  offset: -7.5pt,
  stroke: 1pt,
  extent: 3pt,
  [Temple],
)
\end{verbatim}

\includegraphics[width=5in,height=\textheight,keepaspectratio]{/assets/docs/4typb8n1rt84GcGKwEvmQAAAAAAAAAAA.png}

\subsubsection{\texorpdfstring{\texttt{\ background\ }}{ background }}\label{parameters-background}

\href{/docs/reference/foundations/bool/}{bool}

{{ Settable }}

\phantomsection\label{parameters-background-settable-tooltip}
Settable parameters can be customized for all following uses of the
function with a \texttt{\ set\ } rule.

Whether the line is placed behind the content it overlines.

Default: \texttt{\ }{\texttt{\ false\ }}\texttt{\ }

\includesvg[width=0.16667in,height=0.16667in]{/assets/icons/16-arrow-right.svg}
View example

\begin{verbatim}
#set overline(stroke: (thickness: 1em, paint: maroon, cap: "round"))
#overline(background: true)[This is stylized.] \
#overline(background: false)[This is partially hidden.]
\end{verbatim}

\includegraphics[width=5in,height=\textheight,keepaspectratio]{/assets/docs/J1qF0GrkgS3hBoWTovrZ_AAAAAAAAAAA.png}

\subsubsection{\texorpdfstring{\texttt{\ body\ }}{ body }}\label{parameters-body}

\href{/docs/reference/foundations/content/}{content}

{Required} {{ Positional }}

\phantomsection\label{parameters-body-positional-tooltip}
Positional parameters are specified in order, without names.

The content to add a line over.

\href{/docs/reference/text/lower/}{\pandocbounded{\includesvg[keepaspectratio]{/assets/icons/16-arrow-right.svg}}}

{ Lowercase } { Previous page }

\href{/docs/reference/text/raw/}{\pandocbounded{\includesvg[keepaspectratio]{/assets/icons/16-arrow-right.svg}}}

{ Raw Text / Code } { Next page }


\title{typst.app/docs/reference/text/upper}

\begin{itemize}
\tightlist
\item
  \href{/docs}{\includesvg[width=0.16667in,height=0.16667in]{/assets/icons/16-docs-dark.svg}}
\item
  \includesvg[width=0.16667in,height=0.16667in]{/assets/icons/16-arrow-right.svg}
\item
  \href{/docs/reference/}{Reference}
\item
  \includesvg[width=0.16667in,height=0.16667in]{/assets/icons/16-arrow-right.svg}
\item
  \href{/docs/reference/text/}{Text}
\item
  \includesvg[width=0.16667in,height=0.16667in]{/assets/icons/16-arrow-right.svg}
\item
  \href{/docs/reference/text/upper/}{Uppercase}
\end{itemize}

\section{\texorpdfstring{\texttt{\ upper\ }}{ upper }}\label{summary}

Converts a string or content to uppercase.

\subsection{Example}\label{example}

\begin{verbatim}
#upper("abc") \
#upper[*my text*] \
#upper[ALREADY HIGH]
\end{verbatim}

\includegraphics[width=5in,height=\textheight,keepaspectratio]{/assets/docs/0rcLdDpP-7G0hFWoW3-J-wAAAAAAAAAA.png}

\subsection{\texorpdfstring{{ Parameters
}}{ Parameters }}\label{parameters}

\phantomsection\label{parameters-tooltip}
Parameters are the inputs to a function. They are specified in
parentheses after the function name.

{ upper } (

{ \href{/docs/reference/foundations/str/}{str}
\href{/docs/reference/foundations/content/}{content} }

) -\textgreater{} \href{/docs/reference/foundations/str/}{str}
\href{/docs/reference/foundations/content/}{content}

\subsubsection{\texorpdfstring{\texttt{\ text\ }}{ text }}\label{parameters-text}

\href{/docs/reference/foundations/str/}{str} {or}
\href{/docs/reference/foundations/content/}{content}

{Required} {{ Positional }}

\phantomsection\label{parameters-text-positional-tooltip}
Positional parameters are specified in order, without names.

The text to convert to uppercase.

\href{/docs/reference/text/underline/}{\pandocbounded{\includesvg[keepaspectratio]{/assets/icons/16-arrow-right.svg}}}

{ Underline } { Previous page }

\href{/docs/reference/math/}{\pandocbounded{\includesvg[keepaspectratio]{/assets/icons/16-arrow-right.svg}}}

{ Math } { Next page }


\title{typst.app/docs/reference/text/strike}

\begin{itemize}
\tightlist
\item
  \href{/docs}{\includesvg[width=0.16667in,height=0.16667in]{/assets/icons/16-docs-dark.svg}}
\item
  \includesvg[width=0.16667in,height=0.16667in]{/assets/icons/16-arrow-right.svg}
\item
  \href{/docs/reference/}{Reference}
\item
  \includesvg[width=0.16667in,height=0.16667in]{/assets/icons/16-arrow-right.svg}
\item
  \href{/docs/reference/text/}{Text}
\item
  \includesvg[width=0.16667in,height=0.16667in]{/assets/icons/16-arrow-right.svg}
\item
  \href{/docs/reference/text/strike/}{Strikethrough}
\end{itemize}

\section{\texorpdfstring{\texttt{\ strike\ } {{ Element
}}}{ strike   Element }}\label{summary}

\phantomsection\label{element-tooltip}
Element functions can be customized with \texttt{\ set\ } and
\texttt{\ show\ } rules.

Strikes through text.

\subsection{Example}\label{example}

\begin{verbatim}
This is #strike[not] relevant.
\end{verbatim}

\includegraphics[width=5in,height=\textheight,keepaspectratio]{/assets/docs/gYmGRzTLJUGSNzHzEZFB3gAAAAAAAAAA.png}

\subsection{\texorpdfstring{{ Parameters
}}{ Parameters }}\label{parameters}

\phantomsection\label{parameters-tooltip}
Parameters are the inputs to a function. They are specified in
parentheses after the function name.

{ strike } (

{ \hyperref[parameters-stroke]{stroke :}
\href{/docs/reference/foundations/auto/}{auto}
\href{/docs/reference/layout/length/}{length}
\href{/docs/reference/visualize/color/}{color}
\href{/docs/reference/visualize/gradient/}{gradient}
\href{/docs/reference/visualize/stroke/}{stroke}
\href{/docs/reference/visualize/pattern/}{pattern}
\href{/docs/reference/foundations/dictionary/}{dictionary} , } {
\hyperref[parameters-offset]{offset :}
\href{/docs/reference/foundations/auto/}{auto}
\href{/docs/reference/layout/length/}{length} , } {
\hyperref[parameters-extent]{extent :}
\href{/docs/reference/layout/length/}{length} , } {
\hyperref[parameters-background]{background :}
\href{/docs/reference/foundations/bool/}{bool} , } {
\href{/docs/reference/foundations/content/}{content} , }

) -\textgreater{} \href{/docs/reference/foundations/content/}{content}

\subsubsection{\texorpdfstring{\texttt{\ stroke\ }}{ stroke }}\label{parameters-stroke}

\href{/docs/reference/foundations/auto/}{auto} {or}
\href{/docs/reference/layout/length/}{length} {or}
\href{/docs/reference/visualize/color/}{color} {or}
\href{/docs/reference/visualize/gradient/}{gradient} {or}
\href{/docs/reference/visualize/stroke/}{stroke} {or}
\href{/docs/reference/visualize/pattern/}{pattern} {or}
\href{/docs/reference/foundations/dictionary/}{dictionary}

{{ Settable }}

\phantomsection\label{parameters-stroke-settable-tooltip}
Settable parameters can be customized for all following uses of the
function with a \texttt{\ set\ } rule.

How to \href{/docs/reference/visualize/stroke/}{stroke} the line.

If set to \texttt{\ }{\texttt{\ auto\ }}\texttt{\ } , takes on the
text\textquotesingle s color and a thickness defined in the current
font.

\emph{Note:} Please don\textquotesingle t use this for real redaction as
you can still copy paste the text.

Default: \texttt{\ }{\texttt{\ auto\ }}\texttt{\ }

\includesvg[width=0.16667in,height=0.16667in]{/assets/icons/16-arrow-right.svg}
View example

\begin{verbatim}
This is #strike(stroke: 1.5pt + red)[very stricken through]. \
This is #strike(stroke: 10pt)[redacted].
\end{verbatim}

\includegraphics[width=5in,height=\textheight,keepaspectratio]{/assets/docs/z5bibL2s5nJ9Rg5dVQco5QAAAAAAAAAA.png}

\subsubsection{\texorpdfstring{\texttt{\ offset\ }}{ offset }}\label{parameters-offset}

\href{/docs/reference/foundations/auto/}{auto} {or}
\href{/docs/reference/layout/length/}{length}

{{ Settable }}

\phantomsection\label{parameters-offset-settable-tooltip}
Settable parameters can be customized for all following uses of the
function with a \texttt{\ set\ } rule.

The position of the line relative to the baseline. Read from the font
tables if \texttt{\ }{\texttt{\ auto\ }}\texttt{\ } .

This is useful if you are unhappy with the offset your font provides.

Default: \texttt{\ }{\texttt{\ auto\ }}\texttt{\ }

\includesvg[width=0.16667in,height=0.16667in]{/assets/icons/16-arrow-right.svg}
View example

\begin{verbatim}
#set text(font: "Inria Serif")
This is #strike(offset: auto)[low-ish]. \
This is #strike(offset: -3.5pt)[on-top].
\end{verbatim}

\includegraphics[width=5in,height=\textheight,keepaspectratio]{/assets/docs/1OEdd7_f0OE1q_8jKEVHmQAAAAAAAAAA.png}

\subsubsection{\texorpdfstring{\texttt{\ extent\ }}{ extent }}\label{parameters-extent}

\href{/docs/reference/layout/length/}{length}

{{ Settable }}

\phantomsection\label{parameters-extent-settable-tooltip}
Settable parameters can be customized for all following uses of the
function with a \texttt{\ set\ } rule.

The amount by which to extend the line beyond (or within if negative)
the content.

Default: \texttt{\ }{\texttt{\ 0pt\ }}\texttt{\ }

\includesvg[width=0.16667in,height=0.16667in]{/assets/icons/16-arrow-right.svg}
View example

\begin{verbatim}
This #strike(extent: -2pt)[skips] parts of the word.
This #strike(extent: 2pt)[extends] beyond the word.
\end{verbatim}

\includegraphics[width=5in,height=\textheight,keepaspectratio]{/assets/docs/EqeD8OvCZeei8kbI8T5T0AAAAAAAAAAA.png}

\subsubsection{\texorpdfstring{\texttt{\ background\ }}{ background }}\label{parameters-background}

\href{/docs/reference/foundations/bool/}{bool}

{{ Settable }}

\phantomsection\label{parameters-background-settable-tooltip}
Settable parameters can be customized for all following uses of the
function with a \texttt{\ set\ } rule.

Whether the line is placed behind the content.

Default: \texttt{\ }{\texttt{\ false\ }}\texttt{\ }

\includesvg[width=0.16667in,height=0.16667in]{/assets/icons/16-arrow-right.svg}
View example

\begin{verbatim}
#set strike(stroke: red)
#strike(background: true)[This is behind.] \
#strike(background: false)[This is in front.]
\end{verbatim}

\includegraphics[width=5in,height=\textheight,keepaspectratio]{/assets/docs/5BzB-6LlvrhILN951-2KuQAAAAAAAAAA.png}

\subsubsection{\texorpdfstring{\texttt{\ body\ }}{ body }}\label{parameters-body}

\href{/docs/reference/foundations/content/}{content}

{Required} {{ Positional }}

\phantomsection\label{parameters-body-positional-tooltip}
Positional parameters are specified in order, without names.

The content to strike through.

\href{/docs/reference/text/smartquote/}{\pandocbounded{\includesvg[keepaspectratio]{/assets/icons/16-arrow-right.svg}}}

{ Smartquote } { Previous page }

\href{/docs/reference/text/sub/}{\pandocbounded{\includesvg[keepaspectratio]{/assets/icons/16-arrow-right.svg}}}

{ Subscript } { Next page }


\title{typst.app/docs/reference/text/text}

\begin{itemize}
\tightlist
\item
  \href{/docs}{\includesvg[width=0.16667in,height=0.16667in]{/assets/icons/16-docs-dark.svg}}
\item
  \includesvg[width=0.16667in,height=0.16667in]{/assets/icons/16-arrow-right.svg}
\item
  \href{/docs/reference/}{Reference}
\item
  \includesvg[width=0.16667in,height=0.16667in]{/assets/icons/16-arrow-right.svg}
\item
  \href{/docs/reference/text/}{Text}
\item
  \includesvg[width=0.16667in,height=0.16667in]{/assets/icons/16-arrow-right.svg}
\item
  \href{/docs/reference/text/text/}{Text}
\end{itemize}

\section{\texorpdfstring{\texttt{\ text\ } {{ Element
}}}{ text   Element }}\label{summary}

\phantomsection\label{element-tooltip}
Element functions can be customized with \texttt{\ set\ } and
\texttt{\ show\ } rules.

Customizes the look and layout of text in a variety of ways.

This function is used frequently, both with set rules and directly.
While the set rule is often the simpler choice, calling the
\texttt{\ text\ } function directly can be useful when passing text as
an argument to another function.

\subsection{Example}\label{example}

\begin{verbatim}
#set text(18pt)
With a set rule.

#emph(text(blue)[
  With a function call.
])
\end{verbatim}

\includegraphics[width=5in,height=\textheight,keepaspectratio]{/assets/docs/TE1TKvqGw3ajR6jn3phXugAAAAAAAAAA.png}

\subsection{\texorpdfstring{{ Parameters
}}{ Parameters }}\label{parameters}

\phantomsection\label{parameters-tooltip}
Parameters are the inputs to a function. They are specified in
parentheses after the function name.

{ text } (

{ \hyperref[parameters-font]{font :}
\href{/docs/reference/foundations/str/}{str}
\href{/docs/reference/foundations/array/}{array} , } {
\hyperref[parameters-fallback]{fallback :}
\href{/docs/reference/foundations/bool/}{bool} , } {
\hyperref[parameters-style]{style :}
\href{/docs/reference/foundations/str/}{str} , } {
\hyperref[parameters-weight]{weight :}
\href{/docs/reference/foundations/int/}{int}
\href{/docs/reference/foundations/str/}{str} , } {
\hyperref[parameters-stretch]{stretch :}
\href{/docs/reference/layout/ratio/}{ratio} , } {
\hyperref[parameters-size]{size :}
\href{/docs/reference/layout/length/}{length} , } {
\hyperref[parameters-fill]{fill :}
\href{/docs/reference/visualize/color/}{color}
\href{/docs/reference/visualize/gradient/}{gradient}
\href{/docs/reference/visualize/pattern/}{pattern} , } {
\hyperref[parameters-stroke]{stroke :}
\href{/docs/reference/foundations/none/}{none}
\href{/docs/reference/layout/length/}{length}
\href{/docs/reference/visualize/color/}{color}
\href{/docs/reference/visualize/gradient/}{gradient}
\href{/docs/reference/visualize/stroke/}{stroke}
\href{/docs/reference/visualize/pattern/}{pattern}
\href{/docs/reference/foundations/dictionary/}{dictionary} , } {
\hyperref[parameters-tracking]{tracking :}
\href{/docs/reference/layout/length/}{length} , } {
\hyperref[parameters-spacing]{spacing :}
\href{/docs/reference/layout/relative/}{relative} , } {
\hyperref[parameters-cjk-latin-spacing]{cjk-latin-spacing :}
\href{/docs/reference/foundations/none/}{none}
\href{/docs/reference/foundations/auto/}{auto} , } {
\hyperref[parameters-baseline]{baseline :}
\href{/docs/reference/layout/length/}{length} , } {
\hyperref[parameters-overhang]{overhang :}
\href{/docs/reference/foundations/bool/}{bool} , } {
\hyperref[parameters-top-edge]{top-edge :}
\href{/docs/reference/layout/length/}{length}
\href{/docs/reference/foundations/str/}{str} , } {
\hyperref[parameters-bottom-edge]{bottom-edge :}
\href{/docs/reference/layout/length/}{length}
\href{/docs/reference/foundations/str/}{str} , } {
\hyperref[parameters-lang]{lang :}
\href{/docs/reference/foundations/str/}{str} , } {
\hyperref[parameters-region]{region :}
\href{/docs/reference/foundations/none/}{none}
\href{/docs/reference/foundations/str/}{str} , } {
\hyperref[parameters-script]{script :}
\href{/docs/reference/foundations/auto/}{auto}
\href{/docs/reference/foundations/str/}{str} , } {
\hyperref[parameters-dir]{dir :}
\href{/docs/reference/foundations/auto/}{auto}
\href{/docs/reference/layout/direction/}{direction} , } {
\hyperref[parameters-hyphenate]{hyphenate :}
\href{/docs/reference/foundations/auto/}{auto}
\href{/docs/reference/foundations/bool/}{bool} , } {
\hyperref[parameters-costs]{costs :}
\href{/docs/reference/foundations/dictionary/}{dictionary} , } {
\hyperref[parameters-kerning]{kerning :}
\href{/docs/reference/foundations/bool/}{bool} , } {
\hyperref[parameters-alternates]{alternates :}
\href{/docs/reference/foundations/bool/}{bool} , } {
\hyperref[parameters-stylistic-set]{stylistic-set :}
\href{/docs/reference/foundations/none/}{none}
\href{/docs/reference/foundations/int/}{int}
\href{/docs/reference/foundations/array/}{array} , } {
\hyperref[parameters-ligatures]{ligatures :}
\href{/docs/reference/foundations/bool/}{bool} , } {
\hyperref[parameters-discretionary-ligatures]{discretionary-ligatures :}
\href{/docs/reference/foundations/bool/}{bool} , } {
\hyperref[parameters-historical-ligatures]{historical-ligatures :}
\href{/docs/reference/foundations/bool/}{bool} , } {
\hyperref[parameters-number-type]{number-type :}
\href{/docs/reference/foundations/auto/}{auto}
\href{/docs/reference/foundations/str/}{str} , } {
\hyperref[parameters-number-width]{number-width :}
\href{/docs/reference/foundations/auto/}{auto}
\href{/docs/reference/foundations/str/}{str} , } {
\hyperref[parameters-slashed-zero]{slashed-zero :}
\href{/docs/reference/foundations/bool/}{bool} , } {
\hyperref[parameters-fractions]{fractions :}
\href{/docs/reference/foundations/bool/}{bool} , } {
\hyperref[parameters-features]{features :}
\href{/docs/reference/foundations/array/}{array}
\href{/docs/reference/foundations/dictionary/}{dictionary} , } {
\href{/docs/reference/foundations/content/}{content} , } {
\href{/docs/reference/foundations/str/}{str} , }

) -\textgreater{} \href{/docs/reference/foundations/content/}{content}

\subsubsection{\texorpdfstring{\texttt{\ font\ }}{ font }}\label{parameters-font}

\href{/docs/reference/foundations/str/}{str} {or}
\href{/docs/reference/foundations/array/}{array}

{{ Settable }}

\phantomsection\label{parameters-font-settable-tooltip}
Settable parameters can be customized for all following uses of the
function with a \texttt{\ set\ } rule.

A font family name or priority list of font family names.

When processing text, Typst tries all specified font families in order
until it finds a font that has the necessary glyphs. In the example
below, the font \texttt{\ Inria\ Serif\ } is preferred, but since it
does not contain Arabic glyphs, the arabic text uses
\texttt{\ Noto\ Sans\ Arabic\ } instead.

The collection of available fonts differs by platform:

\begin{itemize}
\item
  In the web app, you can see the list of available fonts by clicking on
  the "Ag" button. You can provide additional fonts by uploading
  \texttt{\ .ttf\ } or \texttt{\ .otf\ } files into your project. They
  will be discovered automatically. The priority is: project fonts
  \textgreater{} server fonts.
\item
  Locally, Typst uses your installed system fonts or embedded fonts in
  the CLI, which are \texttt{\ Libertinus\ Serif\ } ,
  \texttt{\ New\ Computer\ Modern\ } ,
  \texttt{\ New\ Computer\ Modern\ Math\ } , and
  \texttt{\ DejaVu\ Sans\ Mono\ } . In addition, you can use the
  \texttt{\ -\/-font-path\ } argument or \texttt{\ TYPST\_FONT\_PATHS\ }
  environment variable to add directories that should be scanned for
  fonts. The priority is: \texttt{\ -\/-font-paths\ } \textgreater{}
  system fonts \textgreater{} embedded fonts. Run
  \texttt{\ typst\ fonts\ } to see the fonts that Typst has discovered
  on your system. Note that you can pass the
  \texttt{\ -\/-ignore-system-fonts\ } parameter to the CLI to ensure
  Typst won\textquotesingle t search for system fonts.
\end{itemize}

Default: \texttt{\ }{\texttt{\ "libertinus\ serif"\ }}\texttt{\ }

\includesvg[width=0.16667in,height=0.16667in]{/assets/icons/16-arrow-right.svg}
View example

\begin{verbatim}
#set text(font: "PT Sans")
This is sans-serif.

#set text(font: (
  "Inria Serif",
  "Noto Sans Arabic",
))

This is Latin. \
هذا عربي.
\end{verbatim}

\includegraphics[width=5in,height=\textheight,keepaspectratio]{/assets/docs/yZSlTN4UXKYq5EjwCcVgvwAAAAAAAAAA.png}

\subsubsection{\texorpdfstring{\texttt{\ fallback\ }}{ fallback }}\label{parameters-fallback}

\href{/docs/reference/foundations/bool/}{bool}

{{ Settable }}

\phantomsection\label{parameters-fallback-settable-tooltip}
Settable parameters can be customized for all following uses of the
function with a \texttt{\ set\ } rule.

Whether to allow last resort font fallback when the primary font list
contains no match. This lets Typst search through all available fonts
for the most similar one that has the necessary glyphs.

\emph{Note:} Currently, there are no warnings when fallback is disabled
and no glyphs are found. Instead, your text shows up in the form of
"tofus": Small boxes that indicate the lack of an appropriate glyph. In
the future, you will be able to instruct Typst to issue warnings so you
know something is up.

Default: \texttt{\ }{\texttt{\ true\ }}\texttt{\ }

\includesvg[width=0.16667in,height=0.16667in]{/assets/icons/16-arrow-right.svg}
View example

\begin{verbatim}
#set text(font: "Inria Serif")
هذا عربي

#set text(fallback: false)
هذا عربي
\end{verbatim}

\includegraphics[width=5in,height=\textheight,keepaspectratio]{/assets/docs/sa8VqsYbdClSlqi08qJyhAAAAAAAAAAA.png}

\subsubsection{\texorpdfstring{\texttt{\ style\ }}{ style }}\label{parameters-style}

\href{/docs/reference/foundations/str/}{str}

{{ Settable }}

\phantomsection\label{parameters-style-settable-tooltip}
Settable parameters can be customized for all following uses of the
function with a \texttt{\ set\ } rule.

The desired font style.

When an italic style is requested and only an oblique one is available,
it is used. Similarly, the other way around, an italic style can stand
in for an oblique one. When neither an italic nor an oblique style is
available, Typst selects the normal style. Since most fonts are only
available either in an italic or oblique style, the difference between
italic and oblique style is rarely observable.

If you want to emphasize your text, you should do so using the
\href{/docs/reference/model/emph/}{emph} function instead. This makes it
easy to adapt the style later if you change your mind about how to
signify the emphasis.

\begin{longtable}[]{@{}ll@{}}
\toprule\noalign{}
Variant & Details \\
\midrule\noalign{}
\endhead
\bottomrule\noalign{}
\endlastfoot
\texttt{\ "\ normal\ "\ } & The default, typically upright style. \\
\texttt{\ "\ italic\ "\ } & A cursive style with custom letterform. \\
\texttt{\ "\ oblique\ "\ } & Just a slanted version of the normal
style. \\
\end{longtable}

Default: \texttt{\ }{\texttt{\ "normal"\ }}\texttt{\ }

\includesvg[width=0.16667in,height=0.16667in]{/assets/icons/16-arrow-right.svg}
View example

\begin{verbatim}
#text(font: "Libertinus Serif", style: "italic")[Italic]
#text(font: "DejaVu Sans", style: "oblique")[Oblique]
\end{verbatim}

\includegraphics[width=5in,height=\textheight,keepaspectratio]{/assets/docs/S5xaZcVoGLtnT_0XwbPSUQAAAAAAAAAA.png}

\subsubsection{\texorpdfstring{\texttt{\ weight\ }}{ weight }}\label{parameters-weight}

\href{/docs/reference/foundations/int/}{int} {or}
\href{/docs/reference/foundations/str/}{str}

{{ Settable }}

\phantomsection\label{parameters-weight-settable-tooltip}
Settable parameters can be customized for all following uses of the
function with a \texttt{\ set\ } rule.

The desired thickness of the font\textquotesingle s glyphs. Accepts an
integer between \texttt{\ }{\texttt{\ 100\ }}\texttt{\ } and
\texttt{\ }{\texttt{\ 900\ }}\texttt{\ } or one of the predefined weight
names. When the desired weight is not available, Typst selects the font
from the family that is closest in weight.

If you want to strongly emphasize your text, you should do so using the
\href{/docs/reference/model/strong/}{strong} function instead. This
makes it easy to adapt the style later if you change your mind about how
to signify the strong emphasis.

\begin{longtable}[]{@{}ll@{}}
\toprule\noalign{}
Variant & Details \\
\midrule\noalign{}
\endhead
\bottomrule\noalign{}
\endlastfoot
\texttt{\ "\ thin\ "\ } & Thin weight (100). \\
\texttt{\ "\ extralight\ "\ } & Extra light weight (200). \\
\texttt{\ "\ light\ "\ } & Light weight (300). \\
\texttt{\ "\ regular\ "\ } & Regular weight (400). \\
\texttt{\ "\ medium\ "\ } & Medium weight (500). \\
\texttt{\ "\ semibold\ "\ } & Semibold weight (600). \\
\texttt{\ "\ bold\ "\ } & Bold weight (700). \\
\texttt{\ "\ extrabold\ "\ } & Extrabold weight (800). \\
\texttt{\ "\ black\ "\ } & Black weight (900). \\
\end{longtable}

Default: \texttt{\ }{\texttt{\ "regular"\ }}\texttt{\ }

\includesvg[width=0.16667in,height=0.16667in]{/assets/icons/16-arrow-right.svg}
View example

\begin{verbatim}
#set text(font: "IBM Plex Sans")

#text(weight: "light")[Light] \
#text(weight: "regular")[Regular] \
#text(weight: "medium")[Medium] \
#text(weight: 500)[Medium] \
#text(weight: "bold")[Bold]
\end{verbatim}

\includegraphics[width=5in,height=\textheight,keepaspectratio]{/assets/docs/HLYJEJyYVhBAwk1NcGJZjQAAAAAAAAAA.png}

\subsubsection{\texorpdfstring{\texttt{\ stretch\ }}{ stretch }}\label{parameters-stretch}

\href{/docs/reference/layout/ratio/}{ratio}

{{ Settable }}

\phantomsection\label{parameters-stretch-settable-tooltip}
Settable parameters can be customized for all following uses of the
function with a \texttt{\ set\ } rule.

The desired width of the glyphs. Accepts a ratio between
\texttt{\ }{\texttt{\ 50\%\ }}\texttt{\ } and
\texttt{\ }{\texttt{\ 200\%\ }}\texttt{\ } . When the desired width is
not available, Typst selects the font from the family that is closest in
stretch. This will only stretch the text if a condensed or expanded
version of the font is available.

If you want to adjust the amount of space between characters instead of
stretching the glyphs itself, use the
\href{/docs/reference/text/text/\#parameters-tracking}{\texttt{\ tracking\ }}
property instead.

Default: \texttt{\ }{\texttt{\ 100\%\ }}\texttt{\ }

\includesvg[width=0.16667in,height=0.16667in]{/assets/icons/16-arrow-right.svg}
View example

\begin{verbatim}
#text(stretch: 75%)[Condensed] \
#text(stretch: 100%)[Normal]
\end{verbatim}

\includegraphics[width=5in,height=\textheight,keepaspectratio]{/assets/docs/QhcCPECtjtdl-HaT2kdIoQAAAAAAAAAA.png}

\subsubsection{\texorpdfstring{\texttt{\ size\ }}{ size }}\label{parameters-size}

\href{/docs/reference/layout/length/}{length}

{{ Settable }}

\phantomsection\label{parameters-size-settable-tooltip}
Settable parameters can be customized for all following uses of the
function with a \texttt{\ set\ } rule.

The size of the glyphs. This value forms the basis of the
\texttt{\ em\ } unit: \texttt{\ }{\texttt{\ 1em\ }}\texttt{\ } is
equivalent to the font size.

You can also give the font size itself in \texttt{\ em\ } units. Then,
it is relative to the previous font size.

Default: \texttt{\ }{\texttt{\ 11pt\ }}\texttt{\ }

\includesvg[width=0.16667in,height=0.16667in]{/assets/icons/16-arrow-right.svg}
View example

\begin{verbatim}
#set text(size: 20pt)
very #text(1.5em)[big] text
\end{verbatim}

\includegraphics[width=5in,height=\textheight,keepaspectratio]{/assets/docs/blheA65DgOU1lkslOoHidgAAAAAAAAAA.png}

\subsubsection{\texorpdfstring{\texttt{\ fill\ }}{ fill }}\label{parameters-fill}

\href{/docs/reference/visualize/color/}{color} {or}
\href{/docs/reference/visualize/gradient/}{gradient} {or}
\href{/docs/reference/visualize/pattern/}{pattern}

{{ Settable }}

\phantomsection\label{parameters-fill-settable-tooltip}
Settable parameters can be customized for all following uses of the
function with a \texttt{\ set\ } rule.

The glyph fill paint.

Default:
\texttt{\ }{\texttt{\ luma\ }}\texttt{\ }{\texttt{\ (\ }}\texttt{\ }{\texttt{\ 0\%\ }}\texttt{\ }{\texttt{\ )\ }}\texttt{\ }

\includesvg[width=0.16667in,height=0.16667in]{/assets/icons/16-arrow-right.svg}
View example

\begin{verbatim}
#set text(fill: red)
This text is red.
\end{verbatim}

\includegraphics[width=5in,height=\textheight,keepaspectratio]{/assets/docs/hjdTrz3B1HnAtRXCRTTtGAAAAAAAAAAA.png}

\subsubsection{\texorpdfstring{\texttt{\ stroke\ }}{ stroke }}\label{parameters-stroke}

\href{/docs/reference/foundations/none/}{none} {or}
\href{/docs/reference/layout/length/}{length} {or}
\href{/docs/reference/visualize/color/}{color} {or}
\href{/docs/reference/visualize/gradient/}{gradient} {or}
\href{/docs/reference/visualize/stroke/}{stroke} {or}
\href{/docs/reference/visualize/pattern/}{pattern} {or}
\href{/docs/reference/foundations/dictionary/}{dictionary}

{{ Settable }}

\phantomsection\label{parameters-stroke-settable-tooltip}
Settable parameters can be customized for all following uses of the
function with a \texttt{\ set\ } rule.

How to stroke the text.

Default: \texttt{\ }{\texttt{\ none\ }}\texttt{\ }

\includesvg[width=0.16667in,height=0.16667in]{/assets/icons/16-arrow-right.svg}
View example

\begin{verbatim}
#text(stroke: 0.5pt + red)[Stroked]
\end{verbatim}

\includegraphics[width=5in,height=\textheight,keepaspectratio]{/assets/docs/9XI8EQ1M6rOusSDRRIbaPQAAAAAAAAAA.png}

\subsubsection{\texorpdfstring{\texttt{\ tracking\ }}{ tracking }}\label{parameters-tracking}

\href{/docs/reference/layout/length/}{length}

{{ Settable }}

\phantomsection\label{parameters-tracking-settable-tooltip}
Settable parameters can be customized for all following uses of the
function with a \texttt{\ set\ } rule.

The amount of space that should be added between characters.

Default: \texttt{\ }{\texttt{\ 0pt\ }}\texttt{\ }

\includesvg[width=0.16667in,height=0.16667in]{/assets/icons/16-arrow-right.svg}
View example

\begin{verbatim}
#set text(tracking: 1.5pt)
Distant text.
\end{verbatim}

\includegraphics[width=5in,height=\textheight,keepaspectratio]{/assets/docs/_W5ZMMvgiXlv5B8vI6sbcQAAAAAAAAAA.png}

\subsubsection{\texorpdfstring{\texttt{\ spacing\ }}{ spacing }}\label{parameters-spacing}

\href{/docs/reference/layout/relative/}{relative}

{{ Settable }}

\phantomsection\label{parameters-spacing-settable-tooltip}
Settable parameters can be customized for all following uses of the
function with a \texttt{\ set\ } rule.

The amount of space between words.

Can be given as an absolute length, but also relative to the width of
the space character in the font.

If you want to adjust the amount of space between characters rather than
words, use the
\href{/docs/reference/text/text/\#parameters-tracking}{\texttt{\ tracking\ }}
property instead.

Default:
\texttt{\ }{\texttt{\ 100\%\ }}\texttt{\ }{\texttt{\ +\ }}\texttt{\ }{\texttt{\ 0pt\ }}\texttt{\ }

\includesvg[width=0.16667in,height=0.16667in]{/assets/icons/16-arrow-right.svg}
View example

\begin{verbatim}
#set text(spacing: 200%)
Text with distant words.
\end{verbatim}

\includegraphics[width=5in,height=\textheight,keepaspectratio]{/assets/docs/NLatl7xe_PftXpK8eI1WSgAAAAAAAAAA.png}

\subsubsection{\texorpdfstring{\texttt{\ cjk-latin-spacing\ }}{ cjk-latin-spacing }}\label{parameters-cjk-latin-spacing}

\href{/docs/reference/foundations/none/}{none} {or}
\href{/docs/reference/foundations/auto/}{auto}

{{ Settable }}

\phantomsection\label{parameters-cjk-latin-spacing-settable-tooltip}
Settable parameters can be customized for all following uses of the
function with a \texttt{\ set\ } rule.

Whether to automatically insert spacing between CJK and Latin
characters.

Default: \texttt{\ }{\texttt{\ auto\ }}\texttt{\ }

\includesvg[width=0.16667in,height=0.16667in]{/assets/icons/16-arrow-right.svg}
View example

\begin{verbatim}
#set text(cjk-latin-spacing: auto)
第4章介绍了基本的API。

#set text(cjk-latin-spacing: none)
第4章介绍了基本的API。
\end{verbatim}

\includegraphics[width=5in,height=\textheight,keepaspectratio]{/assets/docs/VxUeM1bvsLzygleocZmQUAAAAAAAAAAA.png}

\subsubsection{\texorpdfstring{\texttt{\ baseline\ }}{ baseline }}\label{parameters-baseline}

\href{/docs/reference/layout/length/}{length}

{{ Settable }}

\phantomsection\label{parameters-baseline-settable-tooltip}
Settable parameters can be customized for all following uses of the
function with a \texttt{\ set\ } rule.

An amount to shift the text baseline by.

Default: \texttt{\ }{\texttt{\ 0pt\ }}\texttt{\ }

\includesvg[width=0.16667in,height=0.16667in]{/assets/icons/16-arrow-right.svg}
View example

\begin{verbatim}
A #text(baseline: 3pt)[lowered]
word.
\end{verbatim}

\includegraphics[width=5in,height=\textheight,keepaspectratio]{/assets/docs/Kc1E9Ts9m1i30dvtf5ymQgAAAAAAAAAA.png}

\subsubsection{\texorpdfstring{\texttt{\ overhang\ }}{ overhang }}\label{parameters-overhang}

\href{/docs/reference/foundations/bool/}{bool}

{{ Settable }}

\phantomsection\label{parameters-overhang-settable-tooltip}
Settable parameters can be customized for all following uses of the
function with a \texttt{\ set\ } rule.

Whether certain glyphs can hang over into the margin in justified text.
This can make justification visually more pleasing.

Default: \texttt{\ }{\texttt{\ true\ }}\texttt{\ }

\includesvg[width=0.16667in,height=0.16667in]{/assets/icons/16-arrow-right.svg}
View example

\begin{verbatim}
#set par(justify: true)
This justified text has a hyphen in
the paragraph's first line. Hanging
the hyphen slightly into the margin
results in a clearer paragraph edge.

#set text(overhang: false)
This justified text has a hyphen in
the paragraph's first line. Hanging
the hyphen slightly into the margin
results in a clearer paragraph edge.
\end{verbatim}

\includegraphics[width=5in,height=\textheight,keepaspectratio]{/assets/docs/MnBRs6VvAtjUYVDK-btjfgAAAAAAAAAA.png}

\subsubsection{\texorpdfstring{\texttt{\ top-edge\ }}{ top-edge }}\label{parameters-top-edge}

\href{/docs/reference/layout/length/}{length} {or}
\href{/docs/reference/foundations/str/}{str}

{{ Settable }}

\phantomsection\label{parameters-top-edge-settable-tooltip}
Settable parameters can be customized for all following uses of the
function with a \texttt{\ set\ } rule.

The top end of the conceptual frame around the text used for layout and
positioning. This affects the size of containers that hold text.

\begin{longtable}[]{@{}ll@{}}
\toprule\noalign{}
Variant & Details \\
\midrule\noalign{}
\endhead
\bottomrule\noalign{}
\endlastfoot
\texttt{\ "\ ascender\ "\ } & The font\textquotesingle s ascender, which
typically exceeds the height of all glyphs. \\
\texttt{\ "\ cap-height\ "\ } & The approximate height of uppercase
letters. \\
\texttt{\ "\ x-height\ "\ } & The approximate height of non-ascending
lowercase letters. \\
\texttt{\ "\ baseline\ "\ } & The baseline on which the letters rest. \\
\texttt{\ "\ bounds\ "\ } & The top edge of the glyph\textquotesingle s
bounding box. \\
\end{longtable}

Default: \texttt{\ }{\texttt{\ "cap-height"\ }}\texttt{\ }

\includesvg[width=0.16667in,height=0.16667in]{/assets/icons/16-arrow-right.svg}
View example

\begin{verbatim}
#set rect(inset: 0pt)
#set text(size: 20pt)

#set text(top-edge: "ascender")
#rect(fill: aqua)[Typst]

#set text(top-edge: "cap-height")
#rect(fill: aqua)[Typst]
\end{verbatim}

\includegraphics[width=5in,height=\textheight,keepaspectratio]{/assets/docs/LDeMc2Iiqb_9L3aj1lNrpgAAAAAAAAAA.png}

\subsubsection{\texorpdfstring{\texttt{\ bottom-edge\ }}{ bottom-edge }}\label{parameters-bottom-edge}

\href{/docs/reference/layout/length/}{length} {or}
\href{/docs/reference/foundations/str/}{str}

{{ Settable }}

\phantomsection\label{parameters-bottom-edge-settable-tooltip}
Settable parameters can be customized for all following uses of the
function with a \texttt{\ set\ } rule.

The bottom end of the conceptual frame around the text used for layout
and positioning. This affects the size of containers that hold text.

\begin{longtable}[]{@{}ll@{}}
\toprule\noalign{}
Variant & Details \\
\midrule\noalign{}
\endhead
\bottomrule\noalign{}
\endlastfoot
\texttt{\ "\ baseline\ "\ } & The baseline on which the letters rest. \\
\texttt{\ "\ descender\ "\ } & The font\textquotesingle s descender,
which typically exceeds the depth of all glyphs. \\
\texttt{\ "\ bounds\ "\ } & The bottom edge of the
glyph\textquotesingle s bounding box. \\
\end{longtable}

Default: \texttt{\ }{\texttt{\ "baseline"\ }}\texttt{\ }

\includesvg[width=0.16667in,height=0.16667in]{/assets/icons/16-arrow-right.svg}
View example

\begin{verbatim}
#set rect(inset: 0pt)
#set text(size: 20pt)

#set text(bottom-edge: "baseline")
#rect(fill: aqua)[Typst]

#set text(bottom-edge: "descender")
#rect(fill: aqua)[Typst]
\end{verbatim}

\includegraphics[width=5in,height=\textheight,keepaspectratio]{/assets/docs/l4WLB64gFfplM-bDPX7pEQAAAAAAAAAA.png}

\subsubsection{\texorpdfstring{\texttt{\ lang\ }}{ lang }}\label{parameters-lang}

\href{/docs/reference/foundations/str/}{str}

{{ Settable }}

\phantomsection\label{parameters-lang-settable-tooltip}
Settable parameters can be customized for all following uses of the
function with a \texttt{\ set\ } rule.

An \href{https://en.wikipedia.org/wiki/ISO_639}{ISO 639-1/2/3 language
code.}

Setting the correct language affects various parts of Typst:

\begin{itemize}
\tightlist
\item
  The text processing pipeline can make more informed choices.
\item
  Hyphenation will use the correct patterns for the language.
\item
  \href{/docs/reference/text/smartquote/}{Smart quotes} turns into the
  correct quotes for the language.
\item
  And all other things which are language-aware.
\end{itemize}

Default: \texttt{\ }{\texttt{\ "en"\ }}\texttt{\ }

\includesvg[width=0.16667in,height=0.16667in]{/assets/icons/16-arrow-right.svg}
View example

\begin{verbatim}
#set text(lang: "de")
#outline()

= Einleitung
In diesem Dokument, ...
\end{verbatim}

\includegraphics[width=5in,height=\textheight,keepaspectratio]{/assets/docs/pV_uneCLTlX_ftfk4ZJI1QAAAAAAAAAA.png}

\subsubsection{\texorpdfstring{\texttt{\ region\ }}{ region }}\label{parameters-region}

\href{/docs/reference/foundations/none/}{none} {or}
\href{/docs/reference/foundations/str/}{str}

{{ Settable }}

\phantomsection\label{parameters-region-settable-tooltip}
Settable parameters can be customized for all following uses of the
function with a \texttt{\ set\ } rule.

An \href{https://en.wikipedia.org/wiki/ISO_3166-1_alpha-2}{ISO 3166-1
alpha-2 region code.}

This lets the text processing pipeline make more informed choices.

Default: \texttt{\ }{\texttt{\ none\ }}\texttt{\ }

\subsubsection{\texorpdfstring{\texttt{\ script\ }}{ script }}\label{parameters-script}

\href{/docs/reference/foundations/auto/}{auto} {or}
\href{/docs/reference/foundations/str/}{str}

{{ Settable }}

\phantomsection\label{parameters-script-settable-tooltip}
Settable parameters can be customized for all following uses of the
function with a \texttt{\ set\ } rule.

The OpenType writing script.

The combination of \texttt{\ lang\ } and \texttt{\ script\ } determine
how font features, such as glyph substitution, are implemented.
Frequently the value is a modified (all-lowercase) ISO 15924 script
identifier, and the \texttt{\ math\ } writing script is used for
features appropriate for mathematical symbols.

When set to \texttt{\ }{\texttt{\ auto\ }}\texttt{\ } , the default and
recommended setting, an appropriate script is chosen for each block of
characters sharing a common Unicode script property.

Default: \texttt{\ }{\texttt{\ auto\ }}\texttt{\ }

\includesvg[width=0.16667in,height=0.16667in]{/assets/icons/16-arrow-right.svg}
View example

\begin{verbatim}
#set text(
  font: "Libertinus Serif",
  size: 20pt,
)

#let scedilla = [Ş]
#scedilla // S with a cedilla

#set text(lang: "ro", script: "latn")
#scedilla // S with a subscript comma

#set text(lang: "ro", script: "grek")
#scedilla // S with a cedilla
\end{verbatim}

\includegraphics[width=5in,height=\textheight,keepaspectratio]{/assets/docs/IJovpbe1c5rRr9DM_KRhvgAAAAAAAAAA.png}

\subsubsection{\texorpdfstring{\texttt{\ dir\ }}{ dir }}\label{parameters-dir}

\href{/docs/reference/foundations/auto/}{auto} {or}
\href{/docs/reference/layout/direction/}{direction}

{{ Settable }}

\phantomsection\label{parameters-dir-settable-tooltip}
Settable parameters can be customized for all following uses of the
function with a \texttt{\ set\ } rule.

The dominant direction for text and inline objects. Possible values are:

\begin{itemize}
\tightlist
\item
  \texttt{\ }{\texttt{\ auto\ }}\texttt{\ } : Automatically infer the
  direction from the \texttt{\ lang\ } property.
\item
  \texttt{\ ltr\ } : Layout text from left to right.
\item
  \texttt{\ rtl\ } : Layout text from right to left.
\end{itemize}

When writing in right-to-left scripts like Arabic or Hebrew, you should
set the \href{/docs/reference/text/text/\#parameters-lang}{text
language} or direction. While individual runs of text are automatically
layouted in the correct direction, setting the dominant direction gives
the bidirectional reordering algorithm the necessary information to
correctly place punctuation and inline objects. Furthermore, setting the
direction affects the alignment values \texttt{\ start\ } and
\texttt{\ end\ } , which are equivalent to \texttt{\ left\ } and
\texttt{\ right\ } in \texttt{\ ltr\ } text and the other way around in
\texttt{\ rtl\ } text.

If you set this to \texttt{\ rtl\ } and experience bugs or in some way
bad looking output, please get in touch with us through the
\href{https://forum.typst.app/}{Forum} ,
\href{https://discord.gg/2uDybryKPe}{Discord server} , or our
\href{https://typst.app/contact}{contact form} .

Default: \texttt{\ }{\texttt{\ auto\ }}\texttt{\ }

\includesvg[width=0.16667in,height=0.16667in]{/assets/icons/16-arrow-right.svg}
View example

\begin{verbatim}
#set text(dir: rtl)
هذا عربي.
\end{verbatim}

\includegraphics[width=5in,height=\textheight,keepaspectratio]{/assets/docs/KrWAMeKAPNsts-l34CremAAAAAAAAAAA.png}

\subsubsection{\texorpdfstring{\texttt{\ hyphenate\ }}{ hyphenate }}\label{parameters-hyphenate}

\href{/docs/reference/foundations/auto/}{auto} {or}
\href{/docs/reference/foundations/bool/}{bool}

{{ Settable }}

\phantomsection\label{parameters-hyphenate-settable-tooltip}
Settable parameters can be customized for all following uses of the
function with a \texttt{\ set\ } rule.

Whether to hyphenate text to improve line breaking. When
\texttt{\ }{\texttt{\ auto\ }}\texttt{\ } , text will be hyphenated if
and only if justification is enabled.

Setting the \href{/docs/reference/text/text/\#parameters-lang}{text
language} ensures that the correct hyphenation patterns are used.

Default: \texttt{\ }{\texttt{\ auto\ }}\texttt{\ }

\includesvg[width=0.16667in,height=0.16667in]{/assets/icons/16-arrow-right.svg}
View example

\begin{verbatim}
#set page(width: 200pt)

#set par(justify: true)
This text illustrates how
enabling hyphenation can
improve justification.

#set text(hyphenate: false)
This text illustrates how
enabling hyphenation can
improve justification.
\end{verbatim}

\includegraphics[width=4.16667in,height=\textheight,keepaspectratio]{/assets/docs/4Pafis8Dv1GSWE8dIkAx2wAAAAAAAAAA.png}

\subsubsection{\texorpdfstring{\texttt{\ costs\ }}{ costs }}\label{parameters-costs}

\href{/docs/reference/foundations/dictionary/}{dictionary}

{{ Settable }}

\phantomsection\label{parameters-costs-settable-tooltip}
Settable parameters can be customized for all following uses of the
function with a \texttt{\ set\ } rule.

The "cost" of various choices when laying out text. A higher cost means
the layout engine will make the choice less often. Costs are specified
as a ratio of the default cost, so
\texttt{\ }{\texttt{\ 50\%\ }}\texttt{\ } will make text layout twice as
eager to make a given choice, while
\texttt{\ }{\texttt{\ 200\%\ }}\texttt{\ } will make it half as eager.

Currently, the following costs can be customized:

\begin{itemize}
\tightlist
\item
  \texttt{\ hyphenation\ } : splitting a word across multiple lines
\item
  \texttt{\ runt\ } : ending a paragraph with a line with a single word
\item
  \texttt{\ widow\ } : leaving a single line of paragraph on the next
  page
\item
  \texttt{\ orphan\ } : leaving single line of paragraph on the previous
  page
\end{itemize}

Hyphenation is generally avoided by placing the whole word on the next
line, so a higher hyphenation cost can result in awkward justification
spacing.

Runts are avoided by placing more or fewer words on previous lines, so a
higher runt cost can result in more awkward in justification spacing.

Text layout prevents widows and orphans by default because they are
generally discouraged by style guides. However, in some contexts they
are allowed because the prevention method, which moves a line to the
next page, can result in an uneven number of lines between pages. The
\texttt{\ widow\ } and \texttt{\ orphan\ } costs allow disabling these
modifications. (Currently, \texttt{\ }{\texttt{\ 0\%\ }}\texttt{\ }
allows widows/orphans; anything else, including the default of
\texttt{\ }{\texttt{\ 100\%\ }}\texttt{\ } , prevents them. More nuanced
cost specification for these modifications is planned for the future.)

Default:
\texttt{\ }{\texttt{\ (\ }}\texttt{\ hyphenation\ }{\texttt{\ :\ }}\texttt{\ }{\texttt{\ 100\%\ }}\texttt{\ }{\texttt{\ ,\ }}\texttt{\ runt\ }{\texttt{\ :\ }}\texttt{\ }{\texttt{\ 100\%\ }}\texttt{\ }{\texttt{\ ,\ }}\texttt{\ widow\ }{\texttt{\ :\ }}\texttt{\ }{\texttt{\ 100\%\ }}\texttt{\ }{\texttt{\ ,\ }}\texttt{\ orphan\ }{\texttt{\ :\ }}\texttt{\ }{\texttt{\ 100\%\ }}\texttt{\ }{\texttt{\ ,\ }}\texttt{\ }{\texttt{\ )\ }}\texttt{\ }

\includesvg[width=0.16667in,height=0.16667in]{/assets/icons/16-arrow-right.svg}
View example

\begin{verbatim}
#set text(hyphenate: true, size: 11.4pt)
#set par(justify: true)

#lorem(10)

// Set hyphenation to ten times the normal cost.
#set text(costs: (hyphenation: 1000%))

#lorem(10)
\end{verbatim}

\includegraphics[width=5in,height=\textheight,keepaspectratio]{/assets/docs/k9JLw8qIVUINakYrnv50nAAAAAAAAAAA.png}

\subsubsection{\texorpdfstring{\texttt{\ kerning\ }}{ kerning }}\label{parameters-kerning}

\href{/docs/reference/foundations/bool/}{bool}

{{ Settable }}

\phantomsection\label{parameters-kerning-settable-tooltip}
Settable parameters can be customized for all following uses of the
function with a \texttt{\ set\ } rule.

Whether to apply kerning.

When enabled, specific letter pairings move closer together or further
apart for a more visually pleasing result. The example below
demonstrates how decreasing the gap between the "T" and "o" results in a
more natural look. Setting this to
\texttt{\ }{\texttt{\ false\ }}\texttt{\ } disables kerning by turning
off the OpenType \texttt{\ kern\ } font feature.

Default: \texttt{\ }{\texttt{\ true\ }}\texttt{\ }

\includesvg[width=0.16667in,height=0.16667in]{/assets/icons/16-arrow-right.svg}
View example

\begin{verbatim}
#set text(size: 25pt)
Totally

#set text(kerning: false)
Totally
\end{verbatim}

\includegraphics[width=5in,height=\textheight,keepaspectratio]{/assets/docs/7Gj4TjnwP0QfSOeJi7dKdAAAAAAAAAAA.png}

\subsubsection{\texorpdfstring{\texttt{\ alternates\ }}{ alternates }}\label{parameters-alternates}

\href{/docs/reference/foundations/bool/}{bool}

{{ Settable }}

\phantomsection\label{parameters-alternates-settable-tooltip}
Settable parameters can be customized for all following uses of the
function with a \texttt{\ set\ } rule.

Whether to apply stylistic alternates.

Sometimes fonts contain alternative glyphs for the same codepoint.
Setting this to \texttt{\ }{\texttt{\ true\ }}\texttt{\ } switches to
these by enabling the OpenType \texttt{\ salt\ } font feature.

Default: \texttt{\ }{\texttt{\ false\ }}\texttt{\ }

\includesvg[width=0.16667in,height=0.16667in]{/assets/icons/16-arrow-right.svg}
View example

\begin{verbatim}
#set text(
  font: "IBM Plex Sans",
  size: 20pt,
)

0, a, g, ß

#set text(alternates: true)
0, a, g, ß
\end{verbatim}

\includegraphics[width=5in,height=\textheight,keepaspectratio]{/assets/docs/I0B88ggX_x3jq5W1mvWVIgAAAAAAAAAA.png}

\subsubsection{\texorpdfstring{\texttt{\ stylistic-set\ }}{ stylistic-set }}\label{parameters-stylistic-set}

\href{/docs/reference/foundations/none/}{none} {or}
\href{/docs/reference/foundations/int/}{int} {or}
\href{/docs/reference/foundations/array/}{array}

{{ Settable }}

\phantomsection\label{parameters-stylistic-set-settable-tooltip}
Settable parameters can be customized for all following uses of the
function with a \texttt{\ set\ } rule.

Which stylistic sets to apply. Font designers can categorize alternative
glyphs forms into stylistic sets. As this value is highly font-specific,
you need to consult your font to know which sets are available.

This can be set to an integer or an array of integers, all of which must
be between \texttt{\ }{\texttt{\ 1\ }}\texttt{\ } and
\texttt{\ }{\texttt{\ 20\ }}\texttt{\ } , enabling the corresponding
OpenType feature(s) from \texttt{\ ss01\ } to \texttt{\ ss20\ } .
Setting this to \texttt{\ }{\texttt{\ none\ }}\texttt{\ } will disable
all stylistic sets.

Default:
\texttt{\ }{\texttt{\ (\ }}\texttt{\ }{\texttt{\ )\ }}\texttt{\ }

\includesvg[width=0.16667in,height=0.16667in]{/assets/icons/16-arrow-right.svg}
View example

\begin{verbatim}
#set text(font: "IBM Plex Serif")
ß vs #text(stylistic-set: 5)[ß] \
10 years ago vs #text(stylistic-set: (1, 2, 3))[10 years ago]
\end{verbatim}

\includegraphics[width=5in,height=\textheight,keepaspectratio]{/assets/docs/W4bB6oEym3iwH_NdeQRsEAAAAAAAAAAA.png}

\subsubsection{\texorpdfstring{\texttt{\ ligatures\ }}{ ligatures }}\label{parameters-ligatures}

\href{/docs/reference/foundations/bool/}{bool}

{{ Settable }}

\phantomsection\label{parameters-ligatures-settable-tooltip}
Settable parameters can be customized for all following uses of the
function with a \texttt{\ set\ } rule.

Whether standard ligatures are active.

Certain letter combinations like "fi" are often displayed as a single
merged glyph called a \emph{ligature.} Setting this to
\texttt{\ }{\texttt{\ false\ }}\texttt{\ } disables these ligatures by
turning off the OpenType \texttt{\ liga\ } and \texttt{\ clig\ } font
features.

Default: \texttt{\ }{\texttt{\ true\ }}\texttt{\ }

\includesvg[width=0.16667in,height=0.16667in]{/assets/icons/16-arrow-right.svg}
View example

\begin{verbatim}
#set text(size: 20pt)
A fine ligature.

#set text(ligatures: false)
A fine ligature.
\end{verbatim}

\includegraphics[width=5in,height=\textheight,keepaspectratio]{/assets/docs/IQnLFKoKsoxRyhR3pSbfiwAAAAAAAAAA.png}

\subsubsection{\texorpdfstring{\texttt{\ discretionary-ligatures\ }}{ discretionary-ligatures }}\label{parameters-discretionary-ligatures}

\href{/docs/reference/foundations/bool/}{bool}

{{ Settable }}

\phantomsection\label{parameters-discretionary-ligatures-settable-tooltip}
Settable parameters can be customized for all following uses of the
function with a \texttt{\ set\ } rule.

Whether ligatures that should be used sparingly are active. Setting this
to \texttt{\ }{\texttt{\ true\ }}\texttt{\ } enables the OpenType
\texttt{\ dlig\ } font feature.

Default: \texttt{\ }{\texttt{\ false\ }}\texttt{\ }

\subsubsection{\texorpdfstring{\texttt{\ historical-ligatures\ }}{ historical-ligatures }}\label{parameters-historical-ligatures}

\href{/docs/reference/foundations/bool/}{bool}

{{ Settable }}

\phantomsection\label{parameters-historical-ligatures-settable-tooltip}
Settable parameters can be customized for all following uses of the
function with a \texttt{\ set\ } rule.

Whether historical ligatures are active. Setting this to
\texttt{\ }{\texttt{\ true\ }}\texttt{\ } enables the OpenType
\texttt{\ hlig\ } font feature.

Default: \texttt{\ }{\texttt{\ false\ }}\texttt{\ }

\subsubsection{\texorpdfstring{\texttt{\ number-type\ }}{ number-type }}\label{parameters-number-type}

\href{/docs/reference/foundations/auto/}{auto} {or}
\href{/docs/reference/foundations/str/}{str}

{{ Settable }}

\phantomsection\label{parameters-number-type-settable-tooltip}
Settable parameters can be customized for all following uses of the
function with a \texttt{\ set\ } rule.

Which kind of numbers / figures to select. When set to
\texttt{\ }{\texttt{\ auto\ }}\texttt{\ } , the default numbers for the
font are used.

\begin{longtable}[]{@{}ll@{}}
\toprule\noalign{}
Variant & Details \\
\midrule\noalign{}
\endhead
\bottomrule\noalign{}
\endlastfoot
\texttt{\ "\ lining\ "\ } & Numbers that fit well with capital text (the
OpenType \texttt{\ lnum\ } font feature). \\
\texttt{\ "\ old-style\ "\ } & Numbers that fit well into a flow of
upper- and lowercase text (the OpenType \texttt{\ onum\ } font
feature). \\
\end{longtable}

Default: \texttt{\ }{\texttt{\ auto\ }}\texttt{\ }

\includesvg[width=0.16667in,height=0.16667in]{/assets/icons/16-arrow-right.svg}
View example

\begin{verbatim}
#set text(font: "Noto Sans", 20pt)
#set text(number-type: "lining")
Number 9.

#set text(number-type: "old-style")
Number 9.
\end{verbatim}

\includegraphics[width=5in,height=\textheight,keepaspectratio]{/assets/docs/Jl5yPI_4pX3UbcVjqxI5_QAAAAAAAAAA.png}

\subsubsection{\texorpdfstring{\texttt{\ number-width\ }}{ number-width }}\label{parameters-number-width}

\href{/docs/reference/foundations/auto/}{auto} {or}
\href{/docs/reference/foundations/str/}{str}

{{ Settable }}

\phantomsection\label{parameters-number-width-settable-tooltip}
Settable parameters can be customized for all following uses of the
function with a \texttt{\ set\ } rule.

The width of numbers / figures. When set to
\texttt{\ }{\texttt{\ auto\ }}\texttt{\ } , the default numbers for the
font are used.

\begin{longtable}[]{@{}ll@{}}
\toprule\noalign{}
Variant & Details \\
\midrule\noalign{}
\endhead
\bottomrule\noalign{}
\endlastfoot
\texttt{\ "\ proportional\ "\ } & Numbers with glyph-specific widths
(the OpenType \texttt{\ pnum\ } font feature). \\
\texttt{\ "\ tabular\ "\ } & Numbers of equal width (the OpenType
\texttt{\ tnum\ } font feature). \\
\end{longtable}

Default: \texttt{\ }{\texttt{\ auto\ }}\texttt{\ }

\includesvg[width=0.16667in,height=0.16667in]{/assets/icons/16-arrow-right.svg}
View example

\begin{verbatim}
#set text(font: "Noto Sans", 20pt)
#set text(number-width: "proportional")
A 12 B 34. \
A 56 B 78.

#set text(number-width: "tabular")
A 12 B 34. \
A 56 B 78.
\end{verbatim}

\includegraphics[width=5in,height=\textheight,keepaspectratio]{/assets/docs/6iCMWj0AW9bSFKBJ48tdiwAAAAAAAAAA.png}

\subsubsection{\texorpdfstring{\texttt{\ slashed-zero\ }}{ slashed-zero }}\label{parameters-slashed-zero}

\href{/docs/reference/foundations/bool/}{bool}

{{ Settable }}

\phantomsection\label{parameters-slashed-zero-settable-tooltip}
Settable parameters can be customized for all following uses of the
function with a \texttt{\ set\ } rule.

Whether to have a slash through the zero glyph. Setting this to
\texttt{\ }{\texttt{\ true\ }}\texttt{\ } enables the OpenType
\texttt{\ zero\ } font feature.

Default: \texttt{\ }{\texttt{\ false\ }}\texttt{\ }

\includesvg[width=0.16667in,height=0.16667in]{/assets/icons/16-arrow-right.svg}
View example

\begin{verbatim}
0, #text(slashed-zero: true)[0]
\end{verbatim}

\includegraphics[width=5in,height=\textheight,keepaspectratio]{/assets/docs/NqkvE1KDtvrmSgKJnmfRWwAAAAAAAAAA.png}

\subsubsection{\texorpdfstring{\texttt{\ fractions\ }}{ fractions }}\label{parameters-fractions}

\href{/docs/reference/foundations/bool/}{bool}

{{ Settable }}

\phantomsection\label{parameters-fractions-settable-tooltip}
Settable parameters can be customized for all following uses of the
function with a \texttt{\ set\ } rule.

Whether to turn numbers into fractions. Setting this to
\texttt{\ }{\texttt{\ true\ }}\texttt{\ } enables the OpenType
\texttt{\ frac\ } font feature.

It is not advisable to enable this property globally as it will mess
with all appearances of numbers after a slash (e.g., in URLs). Instead,
enable it locally when you want a fraction.

Default: \texttt{\ }{\texttt{\ false\ }}\texttt{\ }

\includesvg[width=0.16667in,height=0.16667in]{/assets/icons/16-arrow-right.svg}
View example

\begin{verbatim}
1/2 \
#text(fractions: true)[1/2]
\end{verbatim}

\includegraphics[width=5in,height=\textheight,keepaspectratio]{/assets/docs/ReL3WGljBzDnfTHeymCXGQAAAAAAAAAA.png}

\subsubsection{\texorpdfstring{\texttt{\ features\ }}{ features }}\label{parameters-features}

\href{/docs/reference/foundations/array/}{array} {or}
\href{/docs/reference/foundations/dictionary/}{dictionary}

{{ Settable }}

\phantomsection\label{parameters-features-settable-tooltip}
Settable parameters can be customized for all following uses of the
function with a \texttt{\ set\ } rule.

Raw OpenType features to apply.

\begin{itemize}
\tightlist
\item
  If given an array of strings, sets the features identified by the
  strings to \texttt{\ }{\texttt{\ 1\ }}\texttt{\ } .
\item
  If given a dictionary mapping to numbers, sets the features identified
  by the keys to the values.
\end{itemize}

Default:
\texttt{\ }{\texttt{\ (\ }}\texttt{\ }{\texttt{\ :\ }}\texttt{\ }{\texttt{\ )\ }}\texttt{\ }

\includesvg[width=0.16667in,height=0.16667in]{/assets/icons/16-arrow-right.svg}
View example

\begin{verbatim}
// Enable the `frac` feature manually.
#set text(features: ("frac",))
1/2
\end{verbatim}

\includegraphics[width=5in,height=\textheight,keepaspectratio]{/assets/docs/YY_AfHqvOwZWtTBzfgDvMwAAAAAAAAAA.png}

\subsubsection{\texorpdfstring{\texttt{\ body\ }}{ body }}\label{parameters-body}

\href{/docs/reference/foundations/content/}{content}

{Required} {{ Positional }}

\phantomsection\label{parameters-body-positional-tooltip}
Positional parameters are specified in order, without names.

Content in which all text is styled according to the other arguments.

\subsubsection{\texorpdfstring{\texttt{\ text\ }}{ text }}\label{parameters-text}

\href{/docs/reference/foundations/str/}{str}

{Required} {{ Positional }}

\phantomsection\label{parameters-text-positional-tooltip}
Positional parameters are specified in order, without names.

The text.

\href{/docs/reference/text/super/}{\pandocbounded{\includesvg[keepaspectratio]{/assets/icons/16-arrow-right.svg}}}

{ Superscript } { Previous page }

\href{/docs/reference/text/underline/}{\pandocbounded{\includesvg[keepaspectratio]{/assets/icons/16-arrow-right.svg}}}

{ Underline } { Next page }


\title{typst.app/docs/reference/text/smallcaps}

\begin{itemize}
\tightlist
\item
  \href{/docs}{\includesvg[width=0.16667in,height=0.16667in]{/assets/icons/16-docs-dark.svg}}
\item
  \includesvg[width=0.16667in,height=0.16667in]{/assets/icons/16-arrow-right.svg}
\item
  \href{/docs/reference/}{Reference}
\item
  \includesvg[width=0.16667in,height=0.16667in]{/assets/icons/16-arrow-right.svg}
\item
  \href{/docs/reference/text/}{Text}
\item
  \includesvg[width=0.16667in,height=0.16667in]{/assets/icons/16-arrow-right.svg}
\item
  \href{/docs/reference/text/smallcaps/}{Small Capitals}
\end{itemize}

\section{\texorpdfstring{\texttt{\ smallcaps\ } {{ Element
}}}{ smallcaps   Element }}\label{summary}

\phantomsection\label{element-tooltip}
Element functions can be customized with \texttt{\ set\ } and
\texttt{\ show\ } rules.

Displays text in small capitals.

\subsection{Example}\label{example}

\begin{verbatim}
Hello \
#smallcaps[Hello]
\end{verbatim}

\includegraphics[width=5in,height=\textheight,keepaspectratio]{/assets/docs/2GDSP4AltxmHWBvxVXZrwQAAAAAAAAAA.png}

\subsection{Smallcaps fonts}\label{smallcaps-fonts}

By default, this enables the OpenType \texttt{\ smcp\ } feature for the
font. Not all fonts support this feature. Sometimes smallcaps are part
of a dedicated font. This is, for example, the case for the \emph{Latin
Modern} family of fonts. In those cases, you can use a show-set rule to
customize the appearance of the text in smallcaps:

\begin{verbatim}
#show smallcaps: set text(font: "Latin Modern Roman Caps")
\end{verbatim}

In the future, this function will support synthesizing smallcaps from
normal letters, but this is not yet implemented.

\subsection{Smallcaps headings}\label{smallcaps-headings}

You can use a \href{/docs/reference/styling/\#show-rules}{show rule} to
apply smallcaps formatting to all your headings. In the example below,
we also center-align our headings and disable the standard bold font.

\begin{verbatim}
#set par(justify: true)
#set heading(numbering: "I.")

#show heading: smallcaps
#show heading: set align(center)
#show heading: set text(
  weight: "regular"
)

= Introduction
#lorem(40)
\end{verbatim}

\includegraphics[width=5in,height=\textheight,keepaspectratio]{/assets/docs/f0e4HVzW7NKFp4uqk6LvqgAAAAAAAAAA.png}

\subsection{\texorpdfstring{{ Parameters
}}{ Parameters }}\label{parameters}

\phantomsection\label{parameters-tooltip}
Parameters are the inputs to a function. They are specified in
parentheses after the function name.

{ smallcaps } (

{ \href{/docs/reference/foundations/content/}{content} }

) -\textgreater{} \href{/docs/reference/foundations/content/}{content}

\subsubsection{\texorpdfstring{\texttt{\ body\ }}{ body }}\label{parameters-body}

\href{/docs/reference/foundations/content/}{content}

{Required} {{ Positional }}

\phantomsection\label{parameters-body-positional-tooltip}
Positional parameters are specified in order, without names.

The content to display in small capitals.

\href{/docs/reference/text/raw/}{\pandocbounded{\includesvg[keepaspectratio]{/assets/icons/16-arrow-right.svg}}}

{ Raw Text / Code } { Previous page }

\href{/docs/reference/text/smartquote/}{\pandocbounded{\includesvg[keepaspectratio]{/assets/icons/16-arrow-right.svg}}}

{ Smartquote } { Next page }


\title{typst.app/docs/reference/text/lorem}

\begin{itemize}
\tightlist
\item
  \href{/docs}{\includesvg[width=0.16667in,height=0.16667in]{/assets/icons/16-docs-dark.svg}}
\item
  \includesvg[width=0.16667in,height=0.16667in]{/assets/icons/16-arrow-right.svg}
\item
  \href{/docs/reference/}{Reference}
\item
  \includesvg[width=0.16667in,height=0.16667in]{/assets/icons/16-arrow-right.svg}
\item
  \href{/docs/reference/text/}{Text}
\item
  \includesvg[width=0.16667in,height=0.16667in]{/assets/icons/16-arrow-right.svg}
\item
  \href{/docs/reference/text/lorem/}{Lorem}
\end{itemize}

\section{\texorpdfstring{\texttt{\ lorem\ }}{ lorem }}\label{summary}

Creates blind text.

This function yields a Latin-like \emph{Lorem Ipsum} blind text with the
given number of words. The sequence of words generated by the function
is always the same but randomly chosen. As usual for blind texts, it
does not make any sense. Use it as a placeholder to try layouts.

\subsection{Example}\label{example}

\begin{verbatim}
= Blind Text
#lorem(30)

= More Blind Text
#lorem(15)
\end{verbatim}

\includegraphics[width=5in,height=\textheight,keepaspectratio]{/assets/docs/ivKswpaSkeLwjU6-8qNTjgAAAAAAAAAA.png}

\subsection{\texorpdfstring{{ Parameters
}}{ Parameters }}\label{parameters}

\phantomsection\label{parameters-tooltip}
Parameters are the inputs to a function. They are specified in
parentheses after the function name.

{ lorem } (

{ \href{/docs/reference/foundations/int/}{int} }

) -\textgreater{} \href{/docs/reference/foundations/str/}{str}

\subsubsection{\texorpdfstring{\texttt{\ words\ }}{ words }}\label{parameters-words}

\href{/docs/reference/foundations/int/}{int}

{Required} {{ Positional }}

\phantomsection\label{parameters-words-positional-tooltip}
Positional parameters are specified in order, without names.

The length of the blind text in words.

\href{/docs/reference/text/linebreak/}{\pandocbounded{\includesvg[keepaspectratio]{/assets/icons/16-arrow-right.svg}}}

{ Line Break } { Previous page }

\href{/docs/reference/text/lower/}{\pandocbounded{\includesvg[keepaspectratio]{/assets/icons/16-arrow-right.svg}}}

{ Lowercase } { Next page }


