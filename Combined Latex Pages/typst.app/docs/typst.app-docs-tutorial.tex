\title{typst.app/docs/tutorial/advanced-styling}

\begin{itemize}
\tightlist
\item
  \href{/docs}{\includesvg[width=0.16667in,height=0.16667in]{/assets/icons/16-docs-dark.svg}}
\item
  \includesvg[width=0.16667in,height=0.16667in]{/assets/icons/16-arrow-right.svg}
\item
  \href{/docs/tutorial/}{Tutorial}
\item
  \includesvg[width=0.16667in,height=0.16667in]{/assets/icons/16-arrow-right.svg}
\item
  \href{/docs/tutorial/advanced-styling/}{Advanced Styling}
\end{itemize}

\section{Advanced Styling}\label{advanced-styling}

In the previous two chapters of this tutorial, you have learned how to
write a document in Typst and how to change its formatting. The report
you wrote throughout the last two chapters got a straight A and your
supervisor wants to base a conference paper on it! The report will of
course have to comply with the conference\textquotesingle s style guide.
Let\textquotesingle s see how we can achieve that.

Before we start, let\textquotesingle s create a team, invite your
supervisor and add them to the team. You can do this by going back to
the app dashboard with the back icon in the top left corner of the
editor. Then, choose the plus icon in the left toolbar and create a
team. Finally, click on the new team and go to its settings by clicking
\textquotesingle manage team\textquotesingle{} next to the team name.
Now you can invite your supervisor by email.

\pandocbounded{\includegraphics[keepaspectratio]{/assets/docs/3-advanced-team-settings.png}}

Next, move your project into the team: Open it, going to its settings by
choosing the gear icon in the left toolbar and selecting your new team
from the owners dropdown. Don\textquotesingle t forget to save your
changes!

Now, your supervisor can also edit the project and you can both see the
changes in real time. You can join our
\href{https://discord.gg/2uDybryKPe}{Discord server} to find other users
and try teams with them!

\subsection{The conference guidelines}\label{guidelines}

The layout guidelines are available on the conference website.
Let\textquotesingle s take a look at them:

\begin{itemize}
\tightlist
\item
  The font should be an 11pt serif font
\item
  The title should be in 17pt and bold
\item
  The paper contains a single-column abstract and two-column main text
\item
  The abstract should be centered
\item
  The main text should be justified
\item
  First level section headings should be 13pt, centered, and rendered in
  small capitals
\item
  Second level headings are run-ins, italicized and have the same size
  as the body text
\item
  Finally, the pages should be US letter sized, numbered in the center
  of the footer and the top right corner of each page should contain the
  title of the paper
\end{itemize}

We already know how to do many of these things, but for some of them,
we\textquotesingle ll need to learn some new tricks.

\subsection{Writing the right set rules}\label{set-rules}

Let\textquotesingle s start by writing some set rules for the document.

\begin{verbatim}
#set page(
  paper: "us-letter",
  header: align(right)[
    A fluid dynamic model for
    glacier flow
  ],
  numbering: "1",
)
#set par(justify: true)
#set text(
  font: "Libertinus Serif",
  size: 11pt,
)

#lorem(600)
\end{verbatim}

\includegraphics[width=12.75in,height=\textheight,keepaspectratio]{/assets/docs/p6Vtj1ockTIzscSwa5_kewAAAAAAAAAA.png}

You are already familiar with most of what is going on here. We set the
text size to \texttt{\ }{\texttt{\ 11pt\ }}\texttt{\ } and the font to
Libertinus Serif. We also enable paragraph justification and set the
page size to US letter.

The \texttt{\ header\ } argument is new: With it, we can provide content
to fill the top margin of every page. In the header, we specify our
paper\textquotesingle s title as requested by the conference style
guide. We use the \texttt{\ align\ } function to align the text to the
right.

Last but not least is the \texttt{\ numbering\ } argument. Here, we can
provide a \href{/docs/reference/model/numbering/}{numbering pattern}
that defines how to number the pages. By setting into to
\texttt{\ }{\texttt{\ "1"\ }}\texttt{\ } , Typst only displays the bare
page number. Setting it to \texttt{\ }{\texttt{\ "(1/1)"\ }}\texttt{\ }
would have displayed the current page and total number of pages
surrounded by parentheses. And we could even have provided a completely
custom function here to format things to our liking.

\subsection{Creating a title and abstract}\label{title-and-abstract}

Now, let\textquotesingle s add a title and an abstract.
We\textquotesingle ll start with the title. We center align it and
increase its font weight by enclosing it in
\texttt{\ }{\texttt{\ *stars*\ }}\texttt{\ } .

\begin{verbatim}
#align(center, text(17pt)[
  *A fluid dynamic model
  for glacier flow*
])
\end{verbatim}

\includegraphics[width=6.25in,height=\textheight,keepaspectratio]{/assets/docs/EYkMw9AAwHWkDqrGblkKBgAAAAAAAAAA.png}

This looks right. We used the \texttt{\ text\ } function to override the
previous text set rule locally, increasing the size to 17pt for the
function\textquotesingle s argument. Let\textquotesingle s also add the
author list: Since we are writing this paper together with our
supervisor, we\textquotesingle ll add our own and their name.

\begin{verbatim}
#grid(
  columns: (1fr, 1fr),
  align(center)[
    Therese Tungsten \
    Artos Institute \
    #link("mailto:tung@artos.edu")
  ],
  align(center)[
    Dr. John Doe \
    Artos Institute \
    #link("mailto:doe@artos.edu")
  ]
)
\end{verbatim}

\includegraphics[width=6.25in,height=\textheight,keepaspectratio]{/assets/docs/Iwl_3LT7ijX6dcpL71YOWAAAAAAAAAAA.png}

The two author blocks are laid out next to each other. We use the
\href{/docs/reference/layout/grid/}{\texttt{\ grid\ }} function to
create this layout. With a grid, we can control exactly how large each
column is and which content goes into which cell. The
\texttt{\ columns\ } argument takes an array of
\href{/docs/reference/layout/relative/}{relative lengths} or
\href{/docs/reference/layout/fraction/}{fractions} . In this case, we
passed it two equal fractional sizes, telling it to split the available
space into two equal columns. We then passed two content arguments to
the grid function. The first with our own details, and the second with
our supervisors\textquotesingle. We again use the \texttt{\ align\ }
function to center the content within the column. The grid takes an
arbitrary number of content arguments specifying the cells. Rows are
added automatically, but they can also be manually sized with the
\texttt{\ rows\ } argument.

Now, let\textquotesingle s add the abstract. Remember that the
conference wants the abstract to be set ragged and centered.

\begin{verbatim}
...

#align(center)[
  #set par(justify: false)
  *Abstract* \
  #lorem(80)
]
\end{verbatim}

\includegraphics[width=12.75in,height=\textheight,keepaspectratio]{/assets/docs/4IdrVTeq86rbgvB-RNog6gAAAAAAAAAA.png}

Well done! One notable thing is that we used a set rule within the
content argument of \texttt{\ align\ } to turn off justification for the
abstract. This does not affect the remainder of the document even though
it was specified after the first set rule because content blocks
\emph{scope} styling. Anything set within a content block will only
affect the content within that block.

Another tweak could be to save the paper title in a variable, so that we
do not have to type it twice, for header and title. We can do that with
the \texttt{\ }{\texttt{\ let\ }}\texttt{\ } keyword:

\begin{verbatim}
#let title = [
  A fluid dynamic model
  for glacier flow
]

...

#set page(
  header: align(
    right + horizon,
    title
  ),
  ...
)

#align(center, text(17pt)[
  *#title*
])

...
\end{verbatim}

\includegraphics[width=12.75in,height=\textheight,keepaspectratio]{/assets/docs/ZKsZ2Eei-RUgPTLeJrMEYgAAAAAAAAAA.png}

After we bound the content to the \texttt{\ title\ } variable, we can
use it in functions and also within markup (prefixed by \texttt{\ \#\ }
, like functions). This way, if we decide on another title, we can
easily change it in one place.

\subsection{Adding columns and headings}\label{columns-and-headings}

The paper above unfortunately looks like a wall of lead. To fix that,
let\textquotesingle s add some headings and switch our paper to a
two-column layout. Fortunately, that\textquotesingle s easy to do: We
just need to amend our \texttt{\ page\ } set rule with the
\texttt{\ columns\ } argument.

By adding \texttt{\ columns:\ }{\texttt{\ 2\ }}\texttt{\ } to the
argument list, we have wrapped the whole document in two columns.
However, that would also affect the title and authors overview. To keep
them spanning the whole page, we can wrap them in a function call to
\href{/docs/reference/layout/place/}{\texttt{\ place\ }} . Place expects
an alignment and the content it should place as positional arguments.
Using the named \texttt{\ scope\ } argument, we can decide if the items
should be placed relative to the current column or its parent (the
page). There is one more thing to configure: If no other arguments are
provided, \texttt{\ place\ } takes its content out of the flow of the
document and positions it over the other content without affecting the
layout of other content in its container:

\begin{verbatim}
#place(
  top + center,
  rect(fill: black),
)
#lorem(30)
\end{verbatim}

\includegraphics[width=5in,height=\textheight,keepaspectratio]{/assets/docs/30s7cU9X36lW286rJXE3RwAAAAAAAAAA.png}

If we hadn\textquotesingle t used \texttt{\ place\ } here, the square
would be in its own line, but here it overlaps the few lines of text
following it. Likewise, that text acts like as if there was no square.
To change this behavior, we can pass the argument
\texttt{\ float:\ }{\texttt{\ true\ }}\texttt{\ } to ensure that the
space taken up by the placed item at the top or bottom of the page is
not occupied by any other content.

\begin{verbatim}
#set page(
  paper: "us-letter",
  header: align(
    right + horizon,
    title
  ),
  numbering: "1",
  columns: 2,
)

#place(
  top + center,
  float: true,
  scope: "parent",
  clearance: 2em,
)[
  ...

  #par(justify: false)[
    *Abstract* \
    #lorem(80)
  ]
]

= Introduction
#lorem(300)

= Related Work
#lorem(200)
\end{verbatim}

\includegraphics[width=12.75in,height=\textheight,keepaspectratio]{/assets/docs/dAJVP8paZmMvnK23cMA_0AAAAAAAAAAA.png}

In this example, we also used the \texttt{\ clearance\ } argument of the
\texttt{\ place\ } function to provide the space between it and the body
instead of using the \href{/docs/reference/layout/v/}{\texttt{\ v\ }}
function. We can also remove the explicit
\texttt{\ }{\texttt{\ align\ }}\texttt{\ }{\texttt{\ (\ }}\texttt{\ center\ }{\texttt{\ ,\ }}\texttt{\ }{\texttt{\ ..\ }}\texttt{\ }{\texttt{\ )\ }}\texttt{\ }
calls around the various parts since they inherit the center alignment
from the placement.

Now there is only one thing left to do: Style our headings. We need to
make them centered and use small capitals. Because the
\texttt{\ heading\ } function does not offer a way to set any of that,
we need to write our own heading show rule.

\begin{verbatim}
#show heading: it => [
  #set align(center)
  #set text(13pt, weight: "regular")
  #block(smallcaps(it.body))
]

...
\end{verbatim}

\includegraphics[width=5.52083in,height=\textheight,keepaspectratio]{/assets/docs/ZJxJWdUySZNKlj1_Vn1NWgAAAAAAAAAA.png}

This looks great! We used a show rule that applies to all headings. We
give it a function that gets passed the heading as a parameter. That
parameter can be used as content but it also has some fields like
\texttt{\ title\ } , \texttt{\ numbers\ } , and \texttt{\ level\ } from
which we can compose a custom look. Here, we are center-aligning,
setting the font weight to
\texttt{\ }{\texttt{\ "regular"\ }}\texttt{\ } because headings are bold
by default, and use the
\href{/docs/reference/text/smallcaps/}{\texttt{\ smallcaps\ }} function
to render the heading\textquotesingle s title in small capitals.

The only remaining problem is that all headings look the same now. The
"Motivation" and "Problem Statement" subsections ought to be italic run
in headers, but right now, they look indistinguishable from the section
headings. We can fix that by using a \texttt{\ where\ } selector on our
set rule: This is a \href{/docs/reference/scripting/\#methods}{method}
we can call on headings (and other elements) that allows us to filter
them by their level. We can use it to differentiate between section and
subsection headings:

\begin{verbatim}
#show heading.where(
  level: 1
): it => block(width: 100%)[
  #set align(center)
  #set text(13pt, weight: "regular")
  #smallcaps(it.body)
]

#show heading.where(
  level: 2
): it => text(
  size: 11pt,
  weight: "regular",
  style: "italic",
  it.body + [.],
)
\end{verbatim}

\includegraphics[width=5.52083in,height=\textheight,keepaspectratio]{/assets/docs/eBNymJDskGFkYVAXkF9cuAAAAAAAAAAA.png}

This looks great! We wrote two show rules that each selectively apply to
the first and second level headings. We used a \texttt{\ where\ }
selector to filter the headings by their level. We then rendered the
subsection headings as run-ins. We also automatically add a period to
the end of the subsection headings.

Let\textquotesingle s review the conference\textquotesingle s style
guide:

\begin{itemize}
\tightlist
\item
  The font should be an 11pt serif font âœ``
\item
  The title should be in 17pt and bold âœ``
\item
  The paper contains a single-column abstract and two-column main text
  âœ``
\item
  The abstract should be centered âœ``
\item
  The main text should be justified âœ``
\item
  First level section headings should be centered, rendered in small
  caps and in 13pt âœ``
\item
  Second level headings are run-ins, italicized and have the same size
  as the body text âœ``
\item
  Finally, the pages should be US letter sized, numbered in the center
  and the top right corner of each page should contain the title of the
  paper âœ``
\end{itemize}

We are now in compliance with all of these styles and can submit the
paper to the conference! The finished paper looks like this:

\pandocbounded{\includegraphics[keepaspectratio]{/assets/docs/3-advanced-paper.png}}

\subsection{Review}\label{review}

You have now learned how to create headers and footers, how to use
functions and scopes to locally override styles, how to create more
complex layouts with the
\href{/docs/reference/layout/grid/}{\texttt{\ grid\ }} function and how
to write show rules for individual functions, and the whole document.
You also learned how to use the
\href{/docs/reference/styling/\#show-rules}{\texttt{\ where\ } selector}
to filter the headings by their level.

The paper was a great success! You\textquotesingle ve met a lot of
like-minded researchers at the conference and are planning a project
which you hope to publish at the same venue next year.
You\textquotesingle ll need to write a new paper using the same style
guide though, so maybe now you want to create a time-saving template for
you and your team?

In the next section, we will learn how to create templates that can be
reused in multiple documents. This is a more advanced topic, so feel
free to come back to it later if you don\textquotesingle t feel up to it
right now.

\href{/docs/tutorial/formatting/}{\pandocbounded{\includesvg[keepaspectratio]{/assets/icons/16-arrow-right.svg}}}

{ Formatting } { Previous page }

\href{/docs/tutorial/making-a-template/}{\pandocbounded{\includesvg[keepaspectratio]{/assets/icons/16-arrow-right.svg}}}

{ Making a Template } { Next page }


\title{typst.app/docs/tutorial/formatting}

\begin{itemize}
\tightlist
\item
  \href{/docs}{\includesvg[width=0.16667in,height=0.16667in]{/assets/icons/16-docs-dark.svg}}
\item
  \includesvg[width=0.16667in,height=0.16667in]{/assets/icons/16-arrow-right.svg}
\item
  \href{/docs/tutorial/}{Tutorial}
\item
  \includesvg[width=0.16667in,height=0.16667in]{/assets/icons/16-arrow-right.svg}
\item
  \href{/docs/tutorial/formatting/}{Formatting}
\end{itemize}

\section{Formatting}\label{formatting}

So far, you have written a report with some text, a few equations and
images. However, it still looks very plain. Your teaching assistant does
not yet know that you are using a new typesetting system, and you want
your report to fit in with the other student\textquotesingle s
submissions. In this chapter, we will see how to format your report
using Typst\textquotesingle s styling system.

\subsection{Set rules}\label{set-rules}

As we have seen in the previous chapter, Typst has functions that
\emph{insert} content (e.g. the
\href{/docs/reference/visualize/image/}{\texttt{\ image\ }} function)
and others that \emph{manipulate} content that they received as
arguments (e.g. the
\href{/docs/reference/layout/align/}{\texttt{\ align\ }} function). The
first impulse you might have when you want, for example, to justify the
report, could be to look for a function that does that and wrap the
complete document in it.

\begin{verbatim}
#par(justify: true)[
  = Background
  In the case of glaciers, fluid
  dynamics principles can be used
  to understand how the movement
  and behaviour of the ice is
  influenced by factors such as
  temperature, pressure, and the
  presence of other fluids (such as
  water).
]
\end{verbatim}

\includegraphics[width=5in,height=\textheight,keepaspectratio]{/assets/docs/Dijg8l-irnssXE7n_oJpJQAAAAAAAAAA.png}

Wait, shouldn\textquotesingle t all arguments of a function be specified
within parentheses? Why is there a second set of square brackets with
content \emph{after} the parentheses? The answer is that, as passing
content to a function is such a common thing to do in Typst, there is
special syntax for it: Instead of putting the content inside of the
argument list, you can write it in square brackets directly after the
normal arguments, saving on punctuation.

As seen above, that works. The
\href{/docs/reference/model/par/}{\texttt{\ par\ }} function justifies
all paragraphs within it. However, wrapping the document in countless
functions and applying styles selectively and in-situ can quickly become
cumbersome.

Fortunately, Typst has a more elegant solution. With \emph{set rules,}
you can apply style properties to all occurrences of some kind of
content. You write a set rule by entering the
\texttt{\ }{\texttt{\ set\ }}\texttt{\ } keyword, followed by the name
of the function whose properties you want to set, and a list of
arguments in parentheses.

\begin{verbatim}
#set par(justify: true)

= Background
In the case of glaciers, fluid
dynamics principles can be used
to understand how the movement
and behaviour of the ice is
influenced by factors such as
temperature, pressure, and the
presence of other fluids (such as
water).
\end{verbatim}

\includegraphics[width=5in,height=\textheight,keepaspectratio]{/assets/docs/JHqbSYpLaF9kuNFQoo1lAgAAAAAAAAAA.png}

Want to know in more technical terms what is happening here?

Set rules can be conceptualized as setting default values for some of
the parameters of a function for all future uses of that function.

\subsection{The autocomplete panel}\label{autocomplete}

If you followed along and tried a few things in the app, you might have
noticed that always after you enter a \texttt{\ \#\ } character, a panel
pops up to show you the available functions, and, within an argument
list, the available parameters. That\textquotesingle s the autocomplete
panel. It can be very useful while you are writing your document: You
can apply its suggestions by hitting the Return key or navigate to the
desired completion with the arrow keys. The panel can be dismissed by
hitting the Escape key and opened again by typing \texttt{\ \#\ } or
hitting { Ctrl } + { Space } . Use the autocomplete panel to discover
the right arguments for functions. Most suggestions come with a small
description of what they do.

\pandocbounded{\includegraphics[keepaspectratio]{/assets/docs/2-formatting-autocomplete.png}}

\subsection{Set up the page}\label{page-setup}

Back to set rules: When writing a rule, you choose the function
depending on what type of element you want to style. Here is a list of
some functions that are commonly used in set rules:

\begin{itemize}
\tightlist
\item
  \href{/docs/reference/text/text/}{\texttt{\ text\ }} to set font
  family, size, color, and other properties of text
\item
  \href{/docs/reference/layout/page/}{\texttt{\ page\ }} to set the page
  size, margins, headers, enable columns, and footers
\item
  \href{/docs/reference/model/par/}{\texttt{\ par\ }} to justify
  paragraphs, set line spacing, and more
\item
  \href{/docs/reference/model/heading/}{\texttt{\ heading\ }} to set the
  appearance of headings and enable numbering
\item
  \href{/docs/reference/model/document/}{\texttt{\ document\ }} to set
  the metadata contained in the PDF output, such as title and author
\end{itemize}

Not all function parameters can be set. In general, only parameters that
tell a function \emph{how} to do something can be set, not those that
tell it \emph{what} to do it with. The function reference pages indicate
which parameters are settable.

Let\textquotesingle s add a few more styles to our document. We want
larger margins and a serif font. For the purposes of the example,
we\textquotesingle ll also set another page size.

\begin{verbatim}
#set page(
  paper: "a6",
  margin: (x: 1.8cm, y: 1.5cm),
)
#set text(
  font: "New Computer Modern",
  size: 10pt
)
#set par(
  justify: true,
  leading: 0.52em,
)

= Introduction
In this report, we will explore the
various factors that influence fluid
dynamics in glaciers and how they
contribute to the formation and
behaviour of these natural structures.

...

#align(center + bottom)[
  #image("glacier.jpg", width: 70%)

  *Glaciers form an important
  part of the earth's climate
  system.*
]
\end{verbatim}

\includegraphics[width=6.19792in,height=\textheight,keepaspectratio]{/assets/docs/vXvjGwfGgpk5eo7U4CVWMQAAAAAAAAAA.png}

There are a few things of note here.

First is the \href{/docs/reference/layout/page/}{\texttt{\ page\ }} set
rule. It receives two arguments: the page size and margins for the page.
The page size is a string. Typst accepts
\href{/docs/reference/layout/page/\#parameters-paper}{many standard page
sizes,} but you can also specify a custom page size. The margins are
specified as a
\href{/docs/reference/foundations/dictionary/}{dictionary.} Dictionaries
are a collection of key-value pairs. In this case, the keys are
\texttt{\ x\ } and \texttt{\ y\ } , and the values are the horizontal
and vertical margins, respectively. We could also have specified
separate margins for each side by passing a dictionary with the keys
\texttt{\ left\ } , \texttt{\ right\ } , \texttt{\ top\ } , and
\texttt{\ bottom\ } .

Next is the set \href{/docs/reference/text/text/}{\texttt{\ text\ }} set
rule. Here, we set the font size to
\texttt{\ }{\texttt{\ 10pt\ }}\texttt{\ } and font family to
\texttt{\ }{\texttt{\ "New\ Computer\ Modern"\ }}\texttt{\ } . The Typst
app comes with many fonts that you can try for your document. When you
are in the text function\textquotesingle s argument list, you can
discover the available fonts in the autocomplete panel.

We have also set the spacing between lines (a.k.a. leading): It is
specified as a \href{/docs/reference/layout/length/}{length} value, and
we used the \texttt{\ em\ } unit to specify the leading relative to the
size of the font: \texttt{\ }{\texttt{\ 1em\ }}\texttt{\ } is equivalent
to the current font size (which defaults to
\texttt{\ }{\texttt{\ 11pt\ }}\texttt{\ } ).

Finally, we have bottom aligned our image by adding a vertical alignment
to our center alignment. Vertical and horizontal alignments can be
combined with the \texttt{\ }{\texttt{\ +\ }}\texttt{\ } operator to
yield a 2D alignment.

\subsection{A hint of sophistication}\label{sophistication}

To structure our document more clearly, we now want to number our
headings. We can do this by setting the \texttt{\ numbering\ } parameter
of the \href{/docs/reference/model/heading/}{\texttt{\ heading\ }}
function.

\begin{verbatim}
#set heading(numbering: "1.")

= Introduction
#lorem(10)

== Background
#lorem(12)

== Methods
#lorem(15)
\end{verbatim}

\includegraphics[width=5in,height=\textheight,keepaspectratio]{/assets/docs/4WtF0u81AczurIYrwpRdcwAAAAAAAAAA.png}

We specified the string \texttt{\ }{\texttt{\ "1."\ }}\texttt{\ } as the
numbering parameter. This tells Typst to number the headings with arabic
numerals and to put a dot between the number of each level. We can also
use \href{/docs/reference/model/numbering/}{letters, roman numerals, and
symbols} for our headings:

\begin{verbatim}
#set heading(numbering: "1.a")

= Introduction
#lorem(10)

== Background
#lorem(12)

== Methods
#lorem(15)
\end{verbatim}

\includegraphics[width=5in,height=\textheight,keepaspectratio]{/assets/docs/Llv0DrZ6U-QKf1vju2wP4QAAAAAAAAAA.png}

This example also uses the
\href{/docs/reference/text/lorem/}{\texttt{\ lorem\ }} function to
generate some placeholder text. This function takes a number as an
argument and generates that many words of \emph{Lorem Ipsum} text.

Did you wonder why the headings and text set rules apply to all text and
headings, even if they are not produced with the respective functions?

Typst internally calls the \texttt{\ heading\ } function every time you
write \texttt{\ }{\texttt{\ =\ Conclusion\ }}\texttt{\ } . In fact, the
function call
\texttt{\ }{\texttt{\ \#\ }}\texttt{\ }{\texttt{\ heading\ }}\texttt{\ }{\texttt{\ {[}\ }}\texttt{\ Conclusion\ }{\texttt{\ {]}\ }}\texttt{\ }
is equivalent to the heading markup above. Other markup elements work
similarly, they are only \emph{syntax sugar} for the corresponding
function calls.

\subsection{Show rules}\label{show-rules}

You are already pretty happy with how this turned out. But one last
thing needs to be fixed: The report you are writing is intended for a
larger project and that project\textquotesingle s name should always be
accompanied by a logo, even in prose.

You consider your options. You could add an
\texttt{\ }{\texttt{\ \#\ }}\texttt{\ }{\texttt{\ image\ }}\texttt{\ }{\texttt{\ (\ }}\texttt{\ }{\texttt{\ "logo.svg"\ }}\texttt{\ }{\texttt{\ )\ }}\texttt{\ }
call before every instance of the logo using search and replace. That
sounds very tedious. Instead, you could maybe
\href{/docs/reference/foundations/function/\#defining-functions}{define
a custom function} that always yields the logo with its image. However,
there is an even easier way:

With show rules, you can redefine how Typst displays certain elements.
You specify which elements Typst should show differently and how they
should look. Show rules can be applied to instances of text, many
functions, and even the whole document.

\begin{verbatim}
#show "ArtosFlow": name => box[
  #box(image(
    "logo.svg",
    height: 0.7em,
  ))
  #name
]

This report is embedded in the
ArtosFlow project. ArtosFlow is a
project of the Artos Institute.
\end{verbatim}

\includegraphics[width=5in,height=\textheight,keepaspectratio]{/assets/docs/349_Itx4-rTeNxzJmNofvgAAAAAAAAAA.png}

There is a lot of new syntax in this example: We write the
\texttt{\ }{\texttt{\ show\ }}\texttt{\ } keyword, followed by a string
of text we want to show differently and a colon. Then, we write a
function that takes the content that shall be shown as an argument.
Here, we called that argument \texttt{\ name\ } . We can now use the
\texttt{\ name\ } variable in the function\textquotesingle s body to
print the ArtosFlow name. Our show rule adds the logo image in front of
the name and puts the result into a box to prevent linebreaks from
occurring between logo and name. The image is also put inside of a box,
so that it does not appear in its own paragraph.

The calls to the first box function and the image function did not
require a leading \texttt{\ \#\ } because they were not embedded
directly in markup. When Typst expects code instead of markup, the
leading \texttt{\ \#\ } is not needed to access functions, keywords, and
variables. This can be observed in parameter lists, function
definitions, and \href{/docs/reference/scripting/}{code blocks} .

\subsection{Review}\label{review}

You now know how to apply basic formatting to your Typst documents. You
learned how to set the font, justify your paragraphs, change the page
dimensions, and add numbering to your headings with set rules. You also
learned how to use a basic show rule to change how text appears
throughout your document.

You have handed in your report. Your supervisor was so happy with it
that they want to adapt it into a conference paper! In the next section,
we will learn how to format your document as a paper using more advanced
show rules and functions.

\href{/docs/tutorial/writing-in-typst/}{\pandocbounded{\includesvg[keepaspectratio]{/assets/icons/16-arrow-right.svg}}}

{ Writing in Typst } { Previous page }

\href{/docs/tutorial/advanced-styling/}{\pandocbounded{\includesvg[keepaspectratio]{/assets/icons/16-arrow-right.svg}}}

{ Advanced Styling } { Next page }


\title{typst.app/docs/tutorial/writing-in-typst}

\begin{itemize}
\tightlist
\item
  \href{/docs}{\includesvg[width=0.16667in,height=0.16667in]{/assets/icons/16-docs-dark.svg}}
\item
  \includesvg[width=0.16667in,height=0.16667in]{/assets/icons/16-arrow-right.svg}
\item
  \href{/docs/tutorial/}{Tutorial}
\item
  \includesvg[width=0.16667in,height=0.16667in]{/assets/icons/16-arrow-right.svg}
\item
  \href{/docs/tutorial/writing-in-typst/}{Writing in Typst}
\end{itemize}

\section{Writing in Typst}\label{writing-in-typst}

Let\textquotesingle s get started! Suppose you got assigned to write a
technical report for university. It will contain prose, maths, headings,
and figures. To get started, you create a new project on the Typst app.
You\textquotesingle ll be taken to the editor where you see two panels:
A source panel where you compose your document and a preview panel where
you see the rendered document.

\pandocbounded{\includegraphics[keepaspectratio]{/assets/docs/1-writing-app.png}}

You already have a good angle for your report in mind. So
let\textquotesingle s start by writing the introduction. Enter some text
in the editor panel. You\textquotesingle ll notice that the text
immediately appears on the previewed page.

\begin{verbatim}
In this report, we will explore the
various factors that influence fluid
dynamics in glaciers and how they
contribute to the formation and
behaviour of these natural structures.
\end{verbatim}

\includegraphics[width=5in,height=\textheight,keepaspectratio]{/assets/docs/ePl1U-2a7w8qkmb3CLl_oAAAAAAAAAAA.png}

\emph{Throughout this tutorial, we\textquotesingle ll show code examples
like this one. Just like in the app, the first panel contains markup and
the second panel shows a preview. We shrunk the page to fit the examples
so you can see what\textquotesingle s going on.}

The next step is to add a heading and emphasize some text. Typst uses
simple markup for the most common formatting tasks. To add a heading,
enter the \texttt{\ =\ } character and to emphasize some text with
italics, enclose it in
\texttt{\ }{\texttt{\ \_underscores\_\ }}\texttt{\ } .

\begin{verbatim}
= Introduction
In this report, we will explore the
various factors that influence _fluid
dynamics_ in glaciers and how they
contribute to the formation and
behaviour of these natural structures.
\end{verbatim}

\includegraphics[width=5in,height=\textheight,keepaspectratio]{/assets/docs/p75v-z7QqVChplB2N8HZfwAAAAAAAAAA.png}

That was easy! To add a new paragraph, just add a blank line in between
two lines of text. If that paragraph needs a subheading, produce it by
typing \texttt{\ ==\ } instead of \texttt{\ =\ } . The number of
\texttt{\ =\ } characters determines the nesting level of the heading.

Now we want to list a few of the circumstances that influence glacier
dynamics. To do that, we use a numbered list. For each item of the list,
we type a \texttt{\ +\ } character at the beginning of the line. Typst
will automatically number the items.

\begin{verbatim}
+ The climate
+ The topography
+ The geology
\end{verbatim}

\includegraphics[width=5in,height=\textheight,keepaspectratio]{/assets/docs/U3IHQbhSNQ8ndkXIv_gPrgAAAAAAAAAA.png}

If we wanted to add a bulleted list, we would use the \texttt{\ -\ }
character instead of the \texttt{\ +\ } character. We can also nest
lists: For example, we can add a sub-list to the first item of the list
above by indenting it.

\begin{verbatim}
+ The climate
  - Temperature
  - Precipitation
+ The topography
+ The geology
\end{verbatim}

\includegraphics[width=5in,height=\textheight,keepaspectratio]{/assets/docs/xmS-BPiM_gDHkWk9_uhE_gAAAAAAAAAA.png}

\subsection{Adding a figure}\label{figure}

You think that your report would benefit from a figure.
Let\textquotesingle s add one. Typst supports images in the formats PNG,
JPEG, GIF, and SVG. To add an image file to your project, first open the
\emph{file panel} by clicking the box icon in the left sidebar. Here,
you can see a list of all files in your project. Currently, there is
only one: The main Typst file you are writing in. To upload another
file, click the button with the arrow in the top-right corner. This
opens the upload dialog, in which you can pick files to upload from your
computer. Select an image file for your report.

\pandocbounded{\includegraphics[keepaspectratio]{/assets/docs/1-writing-upload.png}}

We have seen before that specific symbols (called \emph{markup} ) have
specific meaning in Typst. We can use \texttt{\ =\ } , \texttt{\ -\ } ,
\texttt{\ +\ } , and \texttt{\ \_\ } to create headings, lists and
emphasized text, respectively. However, having a special symbol for
everything we want to insert into our document would soon become cryptic
and unwieldy. For this reason, Typst reserves markup symbols only for
the most common things. Everything else is inserted with
\emph{functions.} For our image to show up on the page, we use
Typst\textquotesingle s
\href{/docs/reference/visualize/image/}{\texttt{\ image\ }} function.

\begin{verbatim}
#image("glacier.jpg")
\end{verbatim}

\includegraphics[width=5in,height=\textheight,keepaspectratio]{/assets/docs/KwKlYCVb2uFZqZ8abt3-ggAAAAAAAAAA.png}

In general, a function produces some output for a set of
\emph{arguments} . When you \emph{call} a function within markup, you
provide the arguments and Typst inserts the result (the
function\textquotesingle s \emph{return value} ) into the document. In
our case, the \texttt{\ image\ } function takes one argument: The path
to the image file. To call a function in markup, we first need to type
the \texttt{\ \#\ } character, immediately followed by the name of the
function. Then, we enclose the arguments in parentheses. Typst
recognizes many different data types within argument lists. Our file
path is a short \href{/docs/reference/foundations/str/}{string of text}
, so we need to enclose it in double quotes.

The inserted image uses the whole width of the page. To change that,
pass the \texttt{\ width\ } argument to the \texttt{\ image\ } function.
This is a \emph{named} argument and therefore specified as a
\texttt{\ name:\ value\ } pair. If there are multiple arguments, they
are separated by commas, so we first need to put a comma behind the
path.

\begin{verbatim}
#image("glacier.jpg", width: 70%)
\end{verbatim}

\includegraphics[width=5in,height=\textheight,keepaspectratio]{/assets/docs/lpadKIOzcEsf_MGoSeZghAAAAAAAAAAA.png}

The \texttt{\ width\ } argument is a
\href{/docs/reference/layout/relative/}{relative length} . In our case,
we specified a percentage, determining that the image shall take up
\texttt{\ }{\texttt{\ 70\%\ }}\texttt{\ } of the page\textquotesingle s
width. We also could have specified an absolute value like
\texttt{\ }{\texttt{\ 1cm\ }}\texttt{\ } or
\texttt{\ }{\texttt{\ 0.7in\ }}\texttt{\ } .

Just like text, the image is now aligned at the left side of the page by
default. It\textquotesingle s also lacking a caption.
Let\textquotesingle s fix that by using the
\href{/docs/reference/model/figure/}{figure} function. This function
takes the figure\textquotesingle s contents as a positional argument and
an optional caption as a named argument.

Within the argument list of the \texttt{\ figure\ } function, Typst is
already in code mode. This means, you now have to remove the hash before
the image function call. The hash is only needed directly in markup (to
disambiguate text from function calls).

The caption consists of arbitrary markup. To give markup to a function,
we enclose it in square brackets. This construct is called a
\emph{content block.}

\begin{verbatim}
#figure(
  image("glacier.jpg", width: 70%),
  caption: [
    _Glaciers_ form an important part
    of the earth's climate system.
  ],
)
\end{verbatim}

\includegraphics[width=5in,height=\textheight,keepaspectratio]{/assets/docs/v5OnReUO8fD5Rfj2aJZVyQAAAAAAAAAA.png}

You continue to write your report and now want to reference the figure.
To do that, first attach a label to figure. A label uniquely identifies
an element in your document. Add one after the figure by enclosing some
name in angle brackets. You can then reference the figure in your text
by writing an \texttt{\ }{\texttt{\ @\ }}\texttt{\ } symbol followed by
that name. Headings and equations can also be labelled to make them
referenceable.

\begin{verbatim}
Glaciers as the one shown in
@glaciers will cease to exist if
we don't take action soon!

#figure(
  image("glacier.jpg", width: 70%),
  caption: [
    _Glaciers_ form an important part
    of the earth's climate system.
  ],
) <glaciers>
\end{verbatim}

\includegraphics[width=5in,height=\textheight,keepaspectratio]{/assets/docs/cwZ12iQ39B4L-_wQwhO2TAAAAAAAAAAA.png}

So far, we\textquotesingle ve passed content blocks (markup in square
brackets) and strings (text in double quotes) to our functions. Both
seem to contain text. What\textquotesingle s the difference?

A content block can contain text, but also any other kind of markup,
function calls, and more, whereas a string is really just a
\emph{sequence of characters} and nothing else.

For example, the image function expects a path to an image file. It
would not make sense to pass, e.g., a paragraph of text or another image
as the image\textquotesingle s path parameter. That\textquotesingle s
why only strings are allowed here. On the contrary, strings work
wherever content is expected because text is a valid kind of content.

\subsection{Adding a bibliography}\label{bibliography}

As you write up your report, you need to back up some of your claims.
You can add a bibliography to your document with the
\href{/docs/reference/model/bibliography/}{\texttt{\ bibliography\ }}
function. This function expects a path to a bibliography file.

Typst\textquotesingle s native bibliography format is
\href{https://github.com/typst/hayagriva/blob/main/docs/file-format.md}{Hayagriva}
, but for compatibility you can also use BibLaTeX files. As your
classmate has already done a literature survey and sent you a
\texttt{\ .bib\ } file, you\textquotesingle ll use that one. Upload the
file through the file panel to access it in Typst.

Once the document contains a bibliography, you can start citing from it.
Citations use the same syntax as references to a label. As soon as you
cite a source for the first time, it will appear in the bibliography
section of your document. Typst supports different citation and
bibliography styles. Consult the
\href{/docs/reference/model/bibliography/\#parameters-style}{reference}
for more details.

\begin{verbatim}
= Methods
We follow the glacier melting models
established in @glacier-melt.

#bibliography("works.bib")
\end{verbatim}

\includegraphics[width=5in,height=\textheight,keepaspectratio]{/assets/docs/QGPHT14ksdea0r_8vy01WAAAAAAAAAAA.png}

\subsection{Maths}\label{maths}

After fleshing out the methods section, you move on to the meat of the
document: Your equations. Typst has built-in mathematical typesetting
and uses its own math notation. Let\textquotesingle s start with a
simple equation. We wrap it in \texttt{\ \$\ } signs to let Typst know
it should expect a mathematical expression:

\begin{verbatim}
The equation $Q = rho A v + C$
defines the glacial flow rate.
\end{verbatim}

\includegraphics[width=5in,height=\textheight,keepaspectratio]{/assets/docs/_u5BjLoMFBZU2zg1OWULdgAAAAAAAAAA.png}

The equation is typeset inline, on the same line as the surrounding
text. If you want to have it on its own line instead, you should insert
a single space at its start and end:

\begin{verbatim}
The flow rate of a glacier is
defined by the following equation:

$ Q = rho A v + C $
\end{verbatim}

\includegraphics[width=5in,height=\textheight,keepaspectratio]{/assets/docs/GXI0mvGOqqSC165iRTK-QwAAAAAAAAAA.png}

We can see that Typst displayed the single letters \texttt{\ Q\ } ,
\texttt{\ A\ } , \texttt{\ v\ } , and \texttt{\ C\ } as-is, while it
translated \texttt{\ rho\ } into a Greek letter. Math mode will always
show single letters verbatim. Multiple letters, however, are interpreted
as symbols, variables, or function names. To imply a multiplication
between single letters, put spaces between them.

If you want to have a variable that consists of multiple letters, you
can enclose it in quotes:

\begin{verbatim}
The flow rate of a glacier is given
by the following equation:

$ Q = rho A v + "time offset" $
\end{verbatim}

\includegraphics[width=5in,height=\textheight,keepaspectratio]{/assets/docs/JSaojGBiKH-FLbIYqeWSgAAAAAAAAAAA.png}

You\textquotesingle ll also need a sum formula in your paper. We can use
the \texttt{\ sum\ } symbol and then specify the range of the summation
in sub- and superscripts:

\begin{verbatim}
Total displaced soil by glacial flow:

$ 7.32 beta +
  sum_(i=0)^nabla Q_i / 2 $
\end{verbatim}

\includegraphics[width=5in,height=\textheight,keepaspectratio]{/assets/docs/rTDyTGxJlXKPRHJub3ALRgAAAAAAAAAA.png}

To add a subscript to a symbol or variable, type a \texttt{\ \_\ }
character and then the subscript. Similarly, use the \texttt{\ \^{}\ }
character for a superscript. If your sub- or superscript consists of
multiple things, you must enclose them in round parentheses.

The above example also showed us how to insert fractions: Simply put a
\texttt{\ /\ } character between the numerator and the denominator and
Typst will automatically turn it into a fraction. Parentheses are
smartly resolved, so you can enter your expression as you would into a
calculator and Typst will replace parenthesized sub-expressions with the
appropriate notation.

\begin{verbatim}
Total displaced soil by glacial flow:

$ 7.32 beta +
  sum_(i=0)^nabla
    (Q_i (a_i - epsilon)) / 2 $
\end{verbatim}

\includegraphics[width=5in,height=\textheight,keepaspectratio]{/assets/docs/HgeB2Bx5Lh3a5NPfF6WEdwAAAAAAAAAA.png}

Not all math constructs have special syntax. Instead, we use functions,
just like the \texttt{\ image\ } function we have seen before. For
example, to insert a column vector, we can use the
\href{/docs/reference/math/vec/}{\texttt{\ vec\ }} function. Within math
mode, function calls don\textquotesingle t need to start with the
\texttt{\ \#\ } character.

\begin{verbatim}
$ v := vec(x_1, x_2, x_3) $
\end{verbatim}

\includegraphics[width=5in,height=\textheight,keepaspectratio]{/assets/docs/nj0pMnkuoX2t5FZ3_x6YKwAAAAAAAAAA.png}

Some functions are only available within math mode. For example, the
\href{/docs/reference/math/variants/\#functions-cal}{\texttt{\ cal\ }}
function is used to typeset calligraphic letters commonly used for sets.
The \href{/docs/reference/math/}{math section of the reference} provides
a complete list of all functions that math mode makes available.

One more thing: Many symbols, such as the arrow, have a lot of variants.
You can select among these variants by appending a dot and a modifier
name to a symbol\textquotesingle s name:

\begin{verbatim}
$ a arrow.squiggly b $
\end{verbatim}

\includegraphics[width=5in,height=\textheight,keepaspectratio]{/assets/docs/0GgQitNz41j-75F3FS6iAwAAAAAAAAAA.png}

This notation is also available in markup mode, but the symbol name must
be preceded with \texttt{\ \#sym.\ } there. See the
\href{/docs/reference/symbols/sym/}{symbols section} for a list of all
available symbols.

\subsection{Review}\label{review}

You have now seen how to write a basic document in Typst. You learned
how to emphasize text, write lists, insert images, align content, and
typeset mathematical expressions. You also learned about
Typst\textquotesingle s functions. There are many more kinds of content
that Typst lets you insert into your document, such as
\href{/docs/reference/model/table/}{tables} ,
\href{/docs/reference/visualize/}{shapes} , and
\href{/docs/reference/text/raw/}{code blocks} . You can peruse the
\href{/docs/reference/}{reference} to learn more about these and other
features.

For the moment, you have completed writing your report. You have already
saved a PDF by clicking on the download button in the top right corner.
However, you think the report could look a bit less plain. In the next
section, we\textquotesingle ll learn how to customize the look of our
document.

\href{/docs/tutorial/}{\pandocbounded{\includesvg[keepaspectratio]{/assets/icons/16-arrow-right.svg}}}

{ Tutorial } { Previous page }

\href{/docs/tutorial/formatting/}{\pandocbounded{\includesvg[keepaspectratio]{/assets/icons/16-arrow-right.svg}}}

{ Formatting } { Next page }


\title{typst.app/docs/tutorial/making-a-template}

\begin{itemize}
\tightlist
\item
  \href{/docs}{\includesvg[width=0.16667in,height=0.16667in]{/assets/icons/16-docs-dark.svg}}
\item
  \includesvg[width=0.16667in,height=0.16667in]{/assets/icons/16-arrow-right.svg}
\item
  \href{/docs/tutorial/}{Tutorial}
\item
  \includesvg[width=0.16667in,height=0.16667in]{/assets/icons/16-arrow-right.svg}
\item
  \href{/docs/tutorial/making-a-template/}{Making a Template}
\end{itemize}

\section{Making a Template}\label{making-a-template}

In the previous three chapters of this tutorial, you have learned how to
write a document in Typst, apply basic styles, and customize its
appearance in-depth to comply with a publisher\textquotesingle s style
guide. Because the paper you wrote in the previous chapter was a
tremendous success, you have been asked to write a follow-up article for
the same conference. This time, you want to take the style you created
in the previous chapter and turn it into a reusable template. In this
chapter you will learn how to create a template that you and your team
can use with just one show rule. Let\textquotesingle s get started!

\subsection{A toy template}\label{toy-template}

In Typst, templates are functions in which you can wrap your whole
document. To learn how to do that, let\textquotesingle s first review
how to write your very own functions. They can do anything you want them
to, so why not go a bit crazy?

\begin{verbatim}
#let amazed(term) = box[✨ #term ✨]

You are #amazed[beautiful]!
\end{verbatim}

\includegraphics[width=5in,height=\textheight,keepaspectratio]{/assets/docs/hf-0MuyTNtENvqMuqT5IlgAAAAAAAAAA.png}

This function takes a single argument, \texttt{\ term\ } , and returns a
content block with the \texttt{\ term\ } surrounded by sparkles. We also
put the whole thing in a box so that the term we are amazed by cannot be
separated from its sparkles by a line break.

Many functions that come with Typst have optional named parameters. Our
functions can also have them. Let\textquotesingle s add a parameter to
our function that lets us choose the color of the text. We need to
provide a default color in case the parameter isn\textquotesingle t
given.

\begin{verbatim}
#let amazed(term, color: blue) = {
  text(color, box[✨ #term ✨])
}

You are #amazed[beautiful]!
I am #amazed(color: purple)[amazed]!
\end{verbatim}

\includegraphics[width=5in,height=\textheight,keepaspectratio]{/assets/docs/DeOx9bmyxPapZywkKVbTFwAAAAAAAAAA.png}

Templates now work by wrapping our whole document in a custom function
like \texttt{\ amazed\ } . But wrapping a whole document in a giant
function call would be cumbersome! Instead, we can use an "everything"
show rule to achieve the same with cleaner code. To write such a show
rule, put a colon directly behind the show keyword and then provide a
function. This function is given the rest of the document as a
parameter. The function can then do anything with this content. Since
the \texttt{\ amazed\ } function can be called with a single content
argument, we can just pass it by name to the show rule.
Let\textquotesingle s try it:

\begin{verbatim}
#show: amazed
I choose to focus on the good
in my life and let go of any
negative thoughts or beliefs.
In fact, I am amazing!
\end{verbatim}

\includegraphics[width=5in,height=\textheight,keepaspectratio]{/assets/docs/gIv_i_LbdQ0VPwrL8LD78QAAAAAAAAAA.png}

Our whole document will now be passed to the \texttt{\ amazed\ }
function, as if we wrapped it around it. Of course, this is not
especially useful with this particular function, but when combined with
set rules and named arguments, it can be very powerful.

\subsection{Embedding set and show rules}\label{set-and-show-rules}

To apply some set and show rules to our template, we can use
\texttt{\ set\ } and \texttt{\ show\ } within a content block in our
function and then insert the document into that content block.

\begin{verbatim}
#let template(doc) = [
  #set text(font: "Inria Serif")
  #show "something cool": [Typst]
  #doc
]

#show: template
I am learning something cool today.
It's going great so far!
\end{verbatim}

\includegraphics[width=5in,height=\textheight,keepaspectratio]{/assets/docs/A-HDnb3ZV5ZLdSR0m_DP1QAAAAAAAAAA.png}

Just like we already discovered in the previous chapter, set rules will
apply to everything within their content block. Since the everything
show rule passes our whole document to the \texttt{\ template\ }
function, the text set rule and string show rule in our template will
apply to the whole document. Let\textquotesingle s use this knowledge to
create a template that reproduces the body style of the paper we wrote
in the previous chapter.

\begin{verbatim}
#let conf(title, doc) = {
  set page(
    paper: "us-letter",
    header: align(
      right + horizon,
      title
    ),
    columns: 2,
    ...
  )
  set par(justify: true)
  set text(
    font: "Libertinus Serif",
    size: 11pt,
  )

  // Heading show rules.
  ...

  doc
}

#show: doc => conf(
  [Paper title],
  doc,
)

= Introduction
#lorem(90)

...
\end{verbatim}

\includegraphics[width=12.75in,height=\textheight,keepaspectratio]{/assets/docs/Zq1nZR6oWo-01oCtP5oKDAAAAAAAAAAA.png}

We copy-pasted most of that code from the previous chapter. The two
differences are this:

\begin{enumerate}
\item
  We wrapped everything in the function \texttt{\ conf\ } using an
  everything show rule. The function applies a few set and show rules
  and echoes the content it has been passed at the end.
\item
  Moreover, we used a curly-braced code block instead of a content
  block. This way, we don\textquotesingle t need to prefix all set rules
  and function calls with a \texttt{\ \#\ } . In exchange, we cannot
  write markup directly in the code block anymore.
\end{enumerate}

Also note where the title comes from: We previously had it inside of a
variable. Now, we are receiving it as the first parameter of the
template function. To do so, we passed a closure (that\textquotesingle s
a function without a name that is used right away) to the everything
show rule. We did that because the \texttt{\ conf\ } function expects
two positional arguments, the title and the body, but the show rule will
only pass the body. Therefore, we add a new function definition that
allows us to set a paper title and use the single parameter from the
show rule.

\subsection{Templates with named arguments}\label{named-arguments}

Our paper in the previous chapter had a title and an author list.
Let\textquotesingle s add these things to our template. In addition to
the title, we want our template to accept a list of authors with their
affiliations and the paper\textquotesingle s abstract. To keep things
readable, we\textquotesingle ll add those as named arguments. In the
end, we want it to work like this:

\begin{verbatim}
#show: doc => conf(
  title: [Towards Improved Modelling],
  authors: (
    (
      name: "Theresa Tungsten",
      affiliation: "Artos Institute",
      email: "tung@artos.edu",
    ),
    (
      name: "Eugene Deklan",
      affiliation: "Honduras State",
      email: "e.deklan@hstate.hn",
    ),
  ),
  abstract: lorem(80),
  doc,
)

...
\end{verbatim}

Let\textquotesingle s build this new template function. First, we add a
default value to the \texttt{\ title\ } argument. This way, we can call
the template without specifying a title. We also add the named
\texttt{\ authors\ } and \texttt{\ abstract\ } parameters with empty
defaults. Next, we copy the code that generates title, abstract and
authors from the previous chapter into the template, replacing the fixed
details with the parameters.

The new \texttt{\ authors\ } parameter expects an
\href{/docs/reference/foundations/array/}{array} of
\href{/docs/reference/foundations/dictionary/}{dictionaries} with the
keys \texttt{\ name\ } , \texttt{\ affiliation\ } and \texttt{\ email\ }
. Because we can have an arbitrary number of authors, we dynamically
determine if we need one, two or three columns for the author list.
First, we determine the number of authors using the
\href{/docs/reference/foundations/array/\#definitions-len}{\texttt{\ .len()\ }}
method on the \texttt{\ authors\ } array. Then, we set the number of
columns as the minimum of this count and three, so that we never create
more than three columns. If there are more than three authors, a new row
will be inserted instead. For this purpose, we have also added a
\texttt{\ row-gutter\ } parameter to the \texttt{\ grid\ } function.
Otherwise, the rows would be too close together. To extract the details
about the authors from the dictionary, we use the
\href{/docs/reference/scripting/\#fields}{field access syntax} .

We still have to provide an argument to the grid for each author: Here
is where the array\textquotesingle s
\href{/docs/reference/foundations/array/\#definitions-map}{\texttt{\ map\ }
method} comes in handy. It takes a function as an argument that gets
called with each item of the array. We pass it a function that formats
the details for each author and returns a new array containing content
values. We\textquotesingle ve now got one array of values that
we\textquotesingle d like to use as multiple arguments for the grid. We
can do that by using the
\href{/docs/reference/foundations/arguments/}{\texttt{\ spread\ }
operator} . It takes an array and applies each of its items as a
separate argument to the function.

The resulting template function looks like this:

\begin{verbatim}
#let conf(
  title: none,
  authors: (),
  abstract: [],
  doc,
) = {
  // Set and show rules from before.
  ...

  set align(center)
  text(17pt, title)

  let count = authors.len()
  let ncols = calc.min(count, 3)
  grid(
    columns: (1fr,) * ncols,
    row-gutter: 24pt,
    ..authors.map(author => [
      #author.name \
      #author.affiliation \
      #link("mailto:" + author.email)
    ]),
  )

  par(justify: false)[
    *Abstract* \
    #abstract
  ]

  set align(left)
  doc
}
\end{verbatim}

\subsection{A separate file}\label{separate-file}

Most of the time, a template is specified in a different file and then
imported into the document. This way, the main file you write in is kept
clutter free and your template is easily reused. Create a new text file
in the file panel by clicking the plus button and name it
\texttt{\ conf.typ\ } . Move the \texttt{\ conf\ } function definition
inside of that new file. Now you can access it from your main file by
adding an import before the show rule. Specify the path of the file
between the \texttt{\ }{\texttt{\ import\ }}\texttt{\ } keyword and a
colon, then name the function that you want to import.

Another thing that you can do to make applying templates just a bit more
elegant is to use the
\href{/docs/reference/foundations/function/\#definitions-with}{\texttt{\ .with\ }}
method on functions to pre-populate all the named arguments. This way,
you can avoid spelling out a closure and appending the content argument
at the bottom of your template list. Templates on
\href{https://typst.app/universe/}{Typst Universe} are designed to work
with this style of function call.

\begin{verbatim}
#import "conf.typ": conf
#show: conf.with(
  title: [
    Towards Improved Modelling
  ],
  authors: (
    (
      name: "Theresa Tungsten",
      affiliation: "Artos Institute",
      email: "tung@artos.edu",
    ),
    (
      name: "Eugene Deklan",
      affiliation: "Honduras State",
      email: "e.deklan@hstate.hn",
    ),
  ),
  abstract: lorem(80),
)

= Introduction
#lorem(90)

== Motivation
#lorem(140)

== Problem Statement
#lorem(50)

= Related Work
#lorem(200)
\end{verbatim}

\includegraphics[width=12.75in,height=\textheight,keepaspectratio]{/assets/docs/BxllQV4yc0ikxppO7QP73AAAAAAAAAAA.png}

We have now converted the conference paper into a reusable template for
that conference! Why not share it in the
\href{https://forum.typst.app/}{Forum} or on
\href{https://discord.gg/2uDybryKPe}{Typst\textquotesingle s Discord
server} so that others can use it too?

\subsection{Review}\label{review}

Congratulations, you have completed Typst\textquotesingle s Tutorial! In
this section, you have learned how to define your own functions and how
to create and apply templates that define reusable document styles.
You\textquotesingle ve made it far and learned a lot. You can now use
Typst to write your own documents and share them with others.

We are still a super young project and are looking for feedback. If you
have any questions, suggestions or you found a bug, please let us know
in the \href{https://forum.typst.app/}{Forum} , on our
\href{https://discord.gg/2uDybryKPe}{Discord server} , on
\href{https://github.com/typst/typst/}{GitHub} , or via the web
app\textquotesingle s feedback form (always available in the Help menu).

So what are you waiting for? \href{https://typst.app}{Sign up} and write
something!

\href{/docs/tutorial/advanced-styling/}{\pandocbounded{\includesvg[keepaspectratio]{/assets/icons/16-arrow-right.svg}}}

{ Advanced Styling } { Previous page }

\href{/docs/reference/}{\pandocbounded{\includesvg[keepaspectratio]{/assets/icons/16-arrow-right.svg}}}

{ Reference } { Next page }


