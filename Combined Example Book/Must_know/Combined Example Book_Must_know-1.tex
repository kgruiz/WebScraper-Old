\title{sitandr.github.io/typst-examples-book/book/basics/must_know/spacing}

\section{\texorpdfstring{\hyperref[using-spacing]{Using
spacing}}{Using spacing}}\label{using-spacing}

Most time you will pass spacing into functions. There are special
function fields that take only \emph{size} . They are usually called
like
\texttt{\ }{\texttt{\ width,\ length,\ in(out)set,\ spacing\ }}\texttt{\ }
and so on.

Like in CSS, one of the ways to set up spacing in Typst is setting
margins and padding of elements. However, you can also insert spacing
directly using functions \texttt{\ }{\texttt{\ h\ }}\texttt{\ }
(horizontal spacing) and \texttt{\ }{\texttt{\ v\ }}\texttt{\ }
(vertical spacing).

\begin{quote}
Links to reference: \href{https://typst.app/docs/reference/layout/h/}{h}
, \href{https://typst.app/docs/reference/layout/v/}{v} .
\end{quote}

\begin{verbatim}
Horizontal #h(1cm) spacing.
#v(1cm)
And some vertical too!
\end{verbatim}

\pandocbounded{\includesvg[keepaspectratio]{typst-img/47b3ea7d16575780e489790177df9a624ad3c6c669594baa4127c1db516ebc94-1.svg}}

\section{\texorpdfstring{\hyperref[absolute-length-units]{Absolute
length units}}{Absolute length units}}\label{absolute-length-units}

\begin{quote}
Link to
\href{https://typst.app/docs/reference/layout/length/}{reference}
\end{quote}

Absolute length (aka just "length") units are not affected by outer
content and size of parent.

\begin{verbatim}
#set rect(height: 1em)
#table(
  columns: 2,
  [Points], rect(width: 72pt),
  [Millimeters], rect(width: 25.4mm),
  [Centimeters], rect(width: 2.54cm),
  [Inches], rect(width: 1in),
)
\end{verbatim}

\pandocbounded{\includesvg[keepaspectratio]{typst-img/073ad26fe313743ab62dca82f30208dbf2d57ff354d5c37f0b6d4c063dc37d76-1.svg}}

\subsection{\texorpdfstring{\hyperref[relative-to-current-font-size]{Relative
to current font
size}}{Relative to current font size}}\label{relative-to-current-font-size}

\texttt{\ }{\texttt{\ 1em\ =\ 1\ current\ font\ size\ }}\texttt{\ } :

\begin{verbatim}
#set rect(height: 1em)
#table(
  columns: 2,
  [Centimeters], rect(width: 2.54cm),
  [Relative to font size], rect(width: 6.5em)
)

Double font size: #box(stroke: red, baseline: 40%, height: 2em, width: 2em)
\end{verbatim}

\pandocbounded{\includesvg[keepaspectratio]{typst-img/7d62c9e2540f8bce40d8a3fc65a2779b161eb6b5b5682cf87247fee7f14145c2-1.svg}}

It is a very convenient unit, so it is used a lot in Typst.

\subsection{\texorpdfstring{\hyperref[combined]{Combined}}{Combined}}\label{combined}

\begin{verbatim}
Combined: #box(rect(height: 5pt + 1em))

#(5pt + 1em).abs
#(5pt + 1em).em
\end{verbatim}

\pandocbounded{\includesvg[keepaspectratio]{typst-img/c8a0cae6047f35c85c41ac44ff2a6b0d28a28d0e097ca61b367202f9a361136e-1.svg}}

\section{\texorpdfstring{\hyperref[ratio-length]{Ratio
length}}{Ratio length}}\label{ratio-length}

\begin{quote}
Link to \href{https://typst.app/docs/reference/layout/ratio/}{reference}
\end{quote}

\texttt{\ }{\texttt{\ 1\%\ =\ 1\%\ from\ parent\ size\ in\ that\ dimension\ }}\texttt{\ }

\begin{verbatim}
This line width is 50% of available page size (without margins):

#line(length: 50%)

This line width is 50% of the box width: #box(stroke: red, width: 4em, inset: (y: 0.5em), line(length: 50%))
\end{verbatim}

\pandocbounded{\includesvg[keepaspectratio]{typst-img/d478cb8be0a049380479b634cae709dc1e1ed406d323ecb1edbca1e582d7eafe-1.svg}}

\section{\texorpdfstring{\hyperref[relative-length]{Relative
length}}{Relative length}}\label{relative-length}

\begin{quote}
Link to
\href{https://typst.app/docs/reference/layout/relative/}{reference}
\end{quote}

You can \emph{combine} absolute and ratio lengths into \emph{relative
length} :

\begin{verbatim}
#rect(width: 100% - 50pt)

#(100% - 50pt).length \
#(100% - 50pt).ratio
\end{verbatim}

\pandocbounded{\includesvg[keepaspectratio]{typst-img/6b72620a1972e758e55ef1ecf49d3e843095037399ed4dd2dfcd262ebbbe803f-1.svg}}

\section{\texorpdfstring{\hyperref[fractional-length]{Fractional
length}}{Fractional length}}\label{fractional-length}

\begin{quote}
Link to
\href{https://typst.app/docs/reference/layout/fraction/}{reference}
\end{quote}

Single fraction length just takes \emph{maximum size possible} to fill
the parent:

\begin{verbatim}
Left #h(1fr) Right

#rect(height: 1em)[
  #h(1fr)
]
\end{verbatim}

\pandocbounded{\includesvg[keepaspectratio]{typst-img/b9c91f53b684699fff70c6889c8a47fccc57c5c540d7629b93c51a797eb2ef3c-1.svg}}

There are not many places you can use fractions, mainly those are
\texttt{\ }{\texttt{\ h\ }}\texttt{\ } and
\texttt{\ }{\texttt{\ v\ }}\texttt{\ } .

\subsection{\texorpdfstring{\hyperref[several-fractions]{Several
fractions}}{Several fractions}}\label{several-fractions}

If you use several fractions inside one parent, they will take all
remaining space \emph{proportional to their number} :

\begin{verbatim}
Left #h(1fr) Left-ish #h(2fr) Right
\end{verbatim}

\pandocbounded{\includesvg[keepaspectratio]{typst-img/45182cbcecf395256d133af78fccacd9d48e29073672317744cb17340d0bafd8-1.svg}}

\subsection{\texorpdfstring{\hyperref[nested-layout]{Nested
layout}}{Nested layout}}\label{nested-layout}

Remember that fractions work in parent only, don\textquotesingle t
\emph{rely on them in nested layout} :

\begin{verbatim}
Word: #h(1fr) #box(height: 1em, stroke: red)[
  #h(2fr)
]
\end{verbatim}

\pandocbounded{\includesvg[keepaspectratio]{typst-img/0c7ed8b25ea7e39a0907b1105b82027a0fb8b921b28978f30692f6c693bea5f7-1.svg}}


\title{sitandr.github.io/typst-examples-book/book/basics/must_know/tables}

\section{\texorpdfstring{\hyperref[tables-and-grids]{Tables and
grids}}{Tables and grids}}\label{tables-and-grids}

While tables are not that necessary to know if you don\textquotesingle t
plan to use them in your documents, grids may be very useful for
\emph{document layout} . We will use both of them them in the book
later.

Let\textquotesingle s not bother with copying examples from official
documentation. Just make sure to skim through it, okay?

\subsection{\texorpdfstring{\hyperref[basic-snippets]{Basic
snippets}}{Basic snippets}}\label{basic-snippets}

\subsubsection{\texorpdfstring{\hyperref[spreading]{Spreading}}{Spreading}}\label{spreading}

Spreading operators (see \href{../scripting/arguments.html}{there} ) may
be especially useful for the tables:

\begin{verbatim}
#set text(size: 9pt)

#let yield_cells(n) = {
  for i in range(0, n + 1) {
    for j in range(0, n + 1) {
      let product = if i * j != 0 {
        // math is used for the better look 
        if j <= i { $#{ j * i }$ } 
        else {
          // upper part of the table
          text(gray.darken(50%), str(i * j))
        }
      } else {
        if i == j {
          // the top right corner 
          $times$
        } else {
          // on of them is zero, we are at top/left
          $#{i + j}$
        }
      }
      // this is an array, for loops merge them together
      // into one large array of cells
      (
        table.cell(
          fill: if i == j and j == 0 { orange } // top right corner
          else if i == j { yellow } // the diagonal
          else if i * j == 0 { blue.lighten(50%) }, // multipliers
          product,),
      )
    }
  }
}

#let n = 10
#table(
  columns: (0.6cm,) * (n + 1), rows: (0.6cm,) * (n + 1), align: center + horizon, inset: 3pt, ..yield_cells(n),
)
\end{verbatim}

\pandocbounded{\includesvg[keepaspectratio]{typst-img/0640c1d0e5f79bdcb5e60f7675ff1b1eb18810078f5bbbdfaf1c5648b987706e-1.svg}}

\subsubsection{\texorpdfstring{\hyperref[highlighting-table-row]{Highlighting
table row}}{Highlighting table row}}\label{highlighting-table-row}

\begin{verbatim}
#table(
  columns: 2,
  fill: (x, y) => if y == 2 { highlight.fill },
  [A], [B],
  [C], [D],
  [E], [F],
  [G], [H],
)
\end{verbatim}

\pandocbounded{\includesvg[keepaspectratio]{typst-img/4ff8cbb75f85dbab08a336be31115bcb4cb8ca505799641534d937d444e88082-1.svg}}

For individual cells, use

\begin{verbatim}
#table(
  columns: 2,
  [A], [B],
  table.cell(fill: yellow)[C], table.cell(fill: yellow)[D],
  [E], [F],
  [G], [H],
)
\end{verbatim}

\pandocbounded{\includesvg[keepaspectratio]{typst-img/07676a86d4643ff83988c0907aa17995b3d1f8fa7b5be4f11959551afd674bc9-1.svg}}

\subsubsection{\texorpdfstring{\hyperref[splitting-tables]{Splitting
tables}}{Splitting tables}}\label{splitting-tables}

Tables are split between pages automatically.

\begin{verbatim}
#set page(height: 8em)
#(
table(
  columns: 5,
  [Aligner], [publication], [Indexing], [Pairwise alignment], [Max. read length  (bp)],
  [BWA], [2009], [BWT-FM], [Semi-Global], [125],
  [Bowtie], [2009], [BWT-FM], [HD], [76],
  [CloudBurst], [2009], [Hashing], [Landau-Vishkin], [36],
  [GNUMAP], [2009], [Hashing], [NW], [36]
  )
)
\end{verbatim}

\pandocbounded{\includesvg[keepaspectratio]{typst-img/34794c27fefc5c307a1dfdc9ad7958c1dcca0ff8fb64962047051c6a216e0ff7-1.svg}}

\pandocbounded{\includesvg[keepaspectratio]{typst-img/34794c27fefc5c307a1dfdc9ad7958c1dcca0ff8fb64962047051c6a216e0ff7-2.svg}}

However, if you want to make it breakable inside other element,
you\textquotesingle ll have to make that element breakable too:

\begin{verbatim}
#set page(height: 8em)
// Without this, the table fails to split upon several pages
#show figure: set block(breakable: true)
#figure(
table(
  columns: 5,
  [Aligner], [publication], [Indexing], [Pairwise alignment], [Max. read length  (bp)],
  [BWA], [2009], [BWT-FM], [Semi-Global], [125],
  [Bowtie], [2009], [BWT-FM], [HD], [76],
  [CloudBurst], [2009], [Hashing], [Landau-Vishkin], [36],
  [GNUMAP], [2009], [Hashing], [NW], [36]
  )
)
\end{verbatim}

\pandocbounded{\includesvg[keepaspectratio]{typst-img/5be04bf8770a33256599791fb50751bcb24fa5108c13d0e5e2807b675fed00fb-1.svg}}

\pandocbounded{\includesvg[keepaspectratio]{typst-img/5be04bf8770a33256599791fb50751bcb24fa5108c13d0e5e2807b675fed00fb-2.svg}}


\title{sitandr.github.io/typst-examples-book/book/basics/must_know/project_struct}

\section{\texorpdfstring{\hyperref[project-structure]{Project
structure}}{Project structure}}\label{project-structure}

\subsection{\texorpdfstring{\hyperref[large-document]{Large
document}}{Large document}}\label{large-document}

Once the document becomes large enough, it becomes harder to navigate
it. If you haven\textquotesingle t reached that size yet, you can ignore
that section.

For managing that I would recommend splitting your document into
\emph{chapters} . It is just a way to work with this, but once you
understand how it works, you can do anything you want.

Let\textquotesingle s say you have two chapters, then the recommended
structure will look like this:

\begin{verbatim}
#import "@preview/treet:0.1.1": *

#show list: tree-list
#set par(leading: 0.8em)
#show list: set text(font: "DejaVu Sans Mono", size: 0.8em)
- chapters/
  - chapter_1.typ
  - chapter_2.typ
- main.typ 👁 #text(gray)[← document entry point]
- template.typ
\end{verbatim}

\pandocbounded{\includesvg[keepaspectratio]{typst-img/291489e71b40beea77872ad05adb609349872e9a11fc3a9c3f2008c88e37c9d5-1.svg}}

The exact file names are up to you.

Let\textquotesingle s see what to put in each of these files.

\subsubsection{\texorpdfstring{\hyperref[template]{Template}}{Template}}\label{template}

In the "template" file goes \emph{all useful functions and variables}
you will use across the chapters. If you have your own template or want
to write one, you can write it there.

\begin{verbatim}
// template.typ

#let template = doc => {
    set page(header: "My super document")
    show "physics": "magic"
    doc
}

#let info-block = block.with(stroke: blue, fill: blue.lighten(70%))
#let author = "@sitandr"
\end{verbatim}

\subsubsection{\texorpdfstring{\hyperref[main]{Main}}{Main}}\label{main}

\textbf{This file should be compiled} to get the whole compiled
document.

\begin{verbatim}
// main.typ

#import "template.typ": *
// if you have a template
#show: template

= This is the document title

// some additional formatting

#show emph: set text(blue)

// but don't define functions or variables there!
// chapters will not see it

// Now the chapters themselves as some Typst content
#include("chapters/chapter_1.typ")
#include("chapters/chapter_1.typ")
\end{verbatim}

\subsubsection{\texorpdfstring{\hyperref[chapter]{Chapter}}{Chapter}}\label{chapter}

\begin{verbatim}
// chapter_1.typ

#import "../template.typ": *

That's just content with _styling_ and blocks:

#infoblock[Some information].

// just any content you want to include in the document
\end{verbatim}

\subsection{\texorpdfstring{\hyperref[notes]{Notes}}{Notes}}\label{notes}

Note that modules in Typst can see only what they created themselves or
imported. Anything else is invisible for them. That\textquotesingle s
why you need \texttt{\ }{\texttt{\ template.typ\ }}\texttt{\ } file to
define all functions within.

That means chapters \emph{don\textquotesingle t see each other either} ,
only what is in the template.

\subsection{\texorpdfstring{\hyperref[cyclic-imports]{Cyclic
imports}}{Cyclic imports}}\label{cyclic-imports}

\textbf{Important:} Typst \emph{forbids} cyclic imports. That means you
can\textquotesingle t import
\texttt{\ }{\texttt{\ chapter\_1\ }}\texttt{\ } from
\texttt{\ }{\texttt{\ chapter\_2\ }}\texttt{\ } and
\texttt{\ }{\texttt{\ chapter\_2\ }}\texttt{\ } from
\texttt{\ }{\texttt{\ chapter\_1\ }}\texttt{\ } at the same time!

But the good news is that you can always create some other file to
import variable from.


\title{sitandr.github.io/typst-examples-book/book/basics/must_know/place}

\section{\texorpdfstring{\hyperref[placing-moving-scale--hide]{Placing,
Moving, Scale \&
Hide}}{Placing, Moving, Scale \& Hide}}\label{placing-moving-scale--hide}

This is \textbf{a very important section} if you want to do arbitrary
things with layout, create custom elements and hacking a way around
current Typst limitations.

TODO: WIP, add text and better examples

\section{\texorpdfstring{\hyperref[place]{Place}}{Place}}\label{place}

\emph{Ignore layout} , just put some object somehow relative to parent
and current position. The placed object \emph{will not} affect layouting

\begin{quote}
Link to \href{https://typst.app/docs/reference/layout/place/}{reference}
\end{quote}

\begin{verbatim}
#set page(height: 60pt)
Hello, world!

#place(
  top + right, // place at the page right and top
  square(
    width: 20pt,
    stroke: 2pt + blue
  ),
)
\end{verbatim}

\pandocbounded{\includesvg[keepaspectratio]{typst-img/e0d4c250d0f288e1a110ebddcb06149e0acd11b626a0ccb0ca9feb1c1d7be359-1.svg}}

\subsubsection{\texorpdfstring{\hyperref[basic-floating-with-place]{Basic
floating with
place}}{Basic floating with place}}\label{basic-floating-with-place}

\begin{verbatim}
#set page(height: 150pt)
#let note(where, body) = place(
  center + where,
  float: true,
  clearance: 6pt,
  rect(body),
)

#lorem(10)
#note(bottom)[Bottom 1]
#note(bottom)[Bottom 2]
#lorem(40)
#note(top)[Top]
#lorem(10)
\end{verbatim}

\pandocbounded{\includesvg[keepaspectratio]{typst-img/b770cfef024690b5fc7ab82458797d6cfab0c5cc8f52078ecf2d61be17c13acc-1.svg}}

\pandocbounded{\includesvg[keepaspectratio]{typst-img/b770cfef024690b5fc7ab82458797d6cfab0c5cc8f52078ecf2d61be17c13acc-2.svg}}

\subsubsection{\texorpdfstring{\hyperref[dx-dy]{dx,
dy}}{dx, dy}}\label{dx-dy}

Manually change position by
\texttt{\ }{\texttt{\ (dx,\ dy)\ }}\texttt{\ } relative to intended.

\begin{verbatim}
#set page(height: 100pt)
#for i in range(16) {
  let amount = i * 4pt
  place(center, dx: amount - 32pt, dy: amount)[A]
}
\end{verbatim}

\pandocbounded{\includesvg[keepaspectratio]{typst-img/12464f1a2cfe81fb04623033345f3f88ff598af5dc77de378b9d7cf88fc1d5b3-1.svg}}

\section{\texorpdfstring{\hyperref[move]{Move}}{Move}}\label{move}

\begin{quote}
Link to \href{https://typst.app/docs/reference/layout/move/}{reference}
\end{quote}

\begin{verbatim}
#rect(inset: 0pt, move(
  dx: 6pt, dy: 6pt,
  rect(
    inset: 8pt,
    fill: white,
    stroke: black,
    [Abra cadabra]
  )
))
\end{verbatim}

\pandocbounded{\includesvg[keepaspectratio]{typst-img/3292aebf7b633a2d9574027f50867d723d80850e046a101b9df5ab5143eb8a8d-1.svg}}

\section{\texorpdfstring{\hyperref[scale]{Scale}}{Scale}}\label{scale}

Scale content \emph{without affecting the layout} .

\begin{quote}
Link to \href{https://typst.app/docs/reference/layout/scale/}{reference}
\end{quote}

\begin{verbatim}
#scale(x: -100%)[This is mirrored.]
\end{verbatim}

\pandocbounded{\includesvg[keepaspectratio]{typst-img/401c8cd6f306771a3b12432c3c51e097a3ec1d12656c131c0043a12c4c1c3a0e-1.svg}}

\begin{verbatim}
A#box(scale(75%)[A])A \
B#box(scale(75%, origin: bottom + left)[B])B
\end{verbatim}

\pandocbounded{\includesvg[keepaspectratio]{typst-img/204b55690645eb6cc623c8d2d74b5521d72e4ba38d58ea40ea5e2d4354a01836-1.svg}}

\section{\texorpdfstring{\hyperref[hide]{Hide}}{Hide}}\label{hide}

Don\textquotesingle t show content, but leave empty space there.

\begin{quote}
Link to \href{https://typst.app/docs/reference/layout/hide/}{reference}
\end{quote}

\begin{verbatim}
Hello Jane \
#hide[Hello] Joe
\end{verbatim}

\pandocbounded{\includesvg[keepaspectratio]{typst-img/610672d5e43baa3ce94fe61f8d6dd0307e405c785639359c6a9e84bdd66884ad-1.svg}}


\title{sitandr.github.io/typst-examples-book/book/basics/must_know/index}

\section{\texorpdfstring{\hyperref[must-know]{Must-know}}{Must-know}}\label{must-know}

This section contains things, that are not general enough to be part of
"tutorial", but still are very important to know for proper typesetting.

Feel free to skip through things you are sure you will not use.


\title{sitandr.github.io/typst-examples-book/book/basics/must_know/box_block}

\section{\texorpdfstring{\hyperref[boxing--blocking]{Boxing \&
Blocking}}{Boxing \& Blocking}}\label{boxing--blocking}

\begin{verbatim}
You can use boxes to wrap anything
into text: #box(image("../tiger.jpg", height: 2em)).

Blocks will always be "separate paragraphs".
They will not fit into a text: #block(image("../tiger.jpg", height: 2em))
\end{verbatim}

\pandocbounded{\includesvg[keepaspectratio]{typst-img/8e3bd89485b00259666bd636cf28586f92db9c3c3922f0adcdad765ee66a06b1-1.svg}}

Both have similar useful properties:

\begin{verbatim}
#box(stroke: red, inset: 1em)[Box text]
#block(stroke: red, inset: 1em)[Block text]
\end{verbatim}

\pandocbounded{\includesvg[keepaspectratio]{typst-img/9e3562619cb8a31b3d2311f53c3815a214f081e033a564e63dc003dfbc50d68d-1.svg}}

\subsection{\texorpdfstring{\hyperref[rect]{\texttt{\ }{\texttt{\ rect\ }}\texttt{\ }}}{  rect  }}\label{rect}

There is also \texttt{\ }{\texttt{\ rect\ }}\texttt{\ } that works like
\texttt{\ }{\texttt{\ block\ }}\texttt{\ } , but has useful default
inset and stroke:

\begin{verbatim}
#rect[Block text]
\end{verbatim}

\pandocbounded{\includesvg[keepaspectratio]{typst-img/c778d1e94a3663a4f258985368c02e294a1333554c550b6cfe0465275a2eef0f-1.svg}}

\subsection{\texorpdfstring{\hyperref[figures]{Figures}}{Figures}}\label{figures}

For the purposes of adding a \emph{figure} to your document, use
\texttt{\ }{\texttt{\ figure\ }}\texttt{\ } function.
Don\textquotesingle t try to use boxes or blocks there.

Figures are that things like centered images (probably with captions),
tables, even code.

\begin{verbatim}
@tiger shows a tiger. Tigers
are animals.

#figure(
  image("../tiger.jpg", width: 80%),
  caption: [A tiger.],
) <tiger>
\end{verbatim}

\pandocbounded{\includesvg[keepaspectratio]{typst-img/09a8b5b3c3bfffd81be7f34c31cc93ca5f8341b2594d022b2b92ac285aeb959d-1.svg}}

In fact, you can put there anything you want:

\begin{verbatim}
They told me to write a letter to you. Here it is:

#figure(
  text(size: 5em)[I],
  caption: [I'm cool, right?],
) 
\end{verbatim}

\pandocbounded{\includesvg[keepaspectratio]{typst-img/e009534c4572064346490dfac659ff94a5a11d7f46af7a2b46c2136d206088c6-1.svg}}


