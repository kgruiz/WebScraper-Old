\title{sitandr.github.io/typst-examples-book/book/basics/special_symbols}

\section{\texorpdfstring{\hyperref[special-symbols]{Special
symbols}}{Special symbols}}\label{special-symbols}

\begin{quote}
\emph{Important:} I\textquotesingle m not great with special symbols, so
I would additionally appreciate additions and corrections.
\end{quote}

Typst has a great support of \emph{unicode} . That also means it
supports \emph{special symbols} . They may be very useful for
typesetting.

In most cases, you shouldn\textquotesingle t use these symbols directly
often. If possible, use them with show rules (for example, replace all
\texttt{\ }{\texttt{\ -th\ }}\texttt{\ } with
\texttt{\ }{\texttt{\ \textbackslash{}u\ }}\texttt{\ }{\texttt{\ \{2011\}th\ }}\texttt{\ }
, a non-breaking hyphen).

\subsection{\texorpdfstring{\hyperref[non-breaking-symbols]{Non-breaking
symbols}}{Non-breaking symbols}}\label{non-breaking-symbols}

Non-breaking symbols can make sure the word/phrase will not be
separated. Typst will try to put them as a whole.

\subsubsection{\texorpdfstring{\hyperref[non-breaking-space]{Non-breaking
space}}{Non-breaking space}}\label{non-breaking-space}

\begin{quote}
\emph{Important:} As it is spacing symbols, copy-pasting it will not
help. Typst will see it as just a usual spacing symbol you used for your
source code to look nicer in your editor. Again, it will interpret it
\emph{as a basic space} .
\end{quote}

This is a symbol you should\textquotesingle t use often (use Typst boxes
instead), but it is a good demonstration of how non-breaking symbol
work:

\begin{verbatim}
#set page(width: 9em)

// Cruel and world are separated.
// Imagine this is a phrase that can't be split, what to do then?
Hello cruel world

// Let's connect them with a special space!

// No usual spacing is allowed, so either use semicolumn...
Hello cruel#sym.space.nobreak;world

// ...parentheses...
Hello cruel#(sym.space.nobreak)world

// ...or unicode code
Hello cruel\u{00a0}world

// Well, to achieve the same effect I recommend using box:
Hello #box[cruel world]
\end{verbatim}

\pandocbounded{\includesvg[keepaspectratio]{typst-img/be9e5cddfdd58a5f21a2b17e32227ac0c96e2d6eeffe764ef2809257aa416c59-1.svg}}

\subsubsection{\texorpdfstring{\hyperref[non-breaking-hyphen]{Non-breaking
hyphen}}{Non-breaking hyphen}}\label{non-breaking-hyphen}

\begin{verbatim}
#set page(width: 8em)

This is an $i$-th element.

This is an $i$\u{2011}th element.

// the best way would be
#show "-th": "\u{2011}th"

This is an $i$-th element.
\end{verbatim}

\pandocbounded{\includesvg[keepaspectratio]{typst-img/02baa9a61778ef23389d4ceb2fae4d2ac699d72b127b447ca6f25037096d2df9-1.svg}}

\subsection{\texorpdfstring{\hyperref[connectors-and-separators]{Connectors
and
separators}}{Connectors and separators}}\label{connectors-and-separators}

\subsubsection{\texorpdfstring{\hyperref[world-joiner]{World
joiner}}{World joiner}}\label{world-joiner}

Initially, world joiner indicates that no line break should occur at
this position. It is also a zero-width symbol (invisible), so it can be
used as a space removing thing:

\begin{verbatim}
#set page(width: 9em)
#set text(hyphenate: true)

Thisisawordthathastobreak

// Be careful, there is no line break at all now!
Thisi#sym.wj;sawordthathastobreak

// code from `physica` package
// word joiner here is used to avoid extra spacing
#let just-hbar = move(dy: -0.08em, strike(offset: -0.55em, extent: -0.05em, sym.planck))
#let hbar = (sym.wj, just-hbar, sym.wj).join()

$ a #just-hbar b, a hbar b$
\end{verbatim}

\pandocbounded{\includesvg[keepaspectratio]{typst-img/7df9031646c932030adb0fc5a97446e7560ca7d353ef935d4034dc0a4b8be5c1-1.svg}}

\subsubsection{\texorpdfstring{\hyperref[zero-width-space]{Zero width
space}}{Zero width space}}\label{zero-width-space}

Similar to word-joiner, but this is a \emph{space} . It
doesn\textquotesingle t prevent word break. On the contrary, it breaks
it without any hyphen at all!

\begin{verbatim}
#set page(width: 9em)
#set text(hyphenate: true)

// There is a space inside!
Thisisa#sym.zws;word

// Be careful, there is no hyphen at all now!
Thisisawo#sym.zws;rdthathastobreak
\end{verbatim}

\pandocbounded{\includesvg[keepaspectratio]{typst-img/7fd917d4e0422bc1bb72d451b6da6e38fb9fe28cd28152ab60bdfb7ad5d1cab1-1.svg}}


\title{sitandr.github.io/typst-examples-book/book/basics/index}

\section{\texorpdfstring{\hyperref[typst-basics]{Typst
Basics}}{Typst Basics}}\label{typst-basics}

This is a chapter that consistently introduces you to the most things
you need to know when writing with Typst.

It show much more things than official tutorial, so maybe it will be
interesting to read for some of the experienced users too.

Some examples are taken from
\href{https://typst.app/docs/tutorial/}{Official Tutorial} and
\href{https://typst.app/docs/reference/}{Official Reference} . Most are
created and edited specially for this book.

\begin{quote}
\emph{Important:} in most cases there will be used "clipped" examples of
your rendered documents (no margins, smaller width and so on).

To set up the spacing as you want, see
\href{https://typst.app/docs/guides/page-setup-guide/}{Official Page
Setup Guide} .
\end{quote}


\title{sitandr.github.io/typst-examples-book/book/basics/measure}

\section{\texorpdfstring{\hyperref[measure-layout]{Measure,
Layout}}{Measure, Layout}}\label{measure-layout}

This section is outdated. It may be still useful, but it is strongly
recommended to study new context system (using the reference).

\subsection{\texorpdfstring{\hyperref[style--measure]{Style \&
Measure}}{Style \& Measure}}\label{style--measure}

\begin{quote}
Style
\href{https://typst.app/docs/reference/foundations/style/}{documentation}
.
\end{quote}

\begin{quote}
Measure
\href{https://typst.app/docs/reference/layout/measure/}{documentation} .
\end{quote}

\texttt{\ }{\texttt{\ measure\ }}\texttt{\ } returns \emph{the element
size} . This command is extremely helpful when doing custom layout with
\texttt{\ }{\texttt{\ place\ }}\texttt{\ } .

However, there is a catch. Element size depends on styles, applied to
this element.

\begin{verbatim}
#let content = [Hello!]
#content
#set text(14pt)
#content
\end{verbatim}

\pandocbounded{\includesvg[keepaspectratio]{typst-img/00a6cbbc650947c03f34564786b0645eee60396f288d26333c591ff9059cc369-1.svg}}

So if we will set the big text size for some part of our text, to
measure the element\textquotesingle s size, we have to know \emph{where
the element is located} . Without knowing it, we can\textquotesingle t
tell what styles should be applied.

So we need a scheme similar to
\texttt{\ }{\texttt{\ locate\ }}\texttt{\ } .

This is what \texttt{\ }{\texttt{\ styles\ }}\texttt{\ } function is
used for. It is \emph{a content} , which, when located in document,
calls a function inside on \emph{current styles} .

Now, when we got fixed \texttt{\ }{\texttt{\ styles\ }}\texttt{\ } , we
can get the element\textquotesingle s size using
\texttt{\ }{\texttt{\ measure\ }}\texttt{\ } :

\begin{verbatim}
#let thing(body) = style(styles => {
  let size = measure(body, styles)
  [Width of "#body" is #size.width]
})

#thing[Hey] \
#thing[Welcome]
\end{verbatim}

\pandocbounded{\includesvg[keepaspectratio]{typst-img/5afe1855072b4ee8e343e5b5aa79affae5b17bc89738ffbe93dac245576cdd04-1.svg}}

\section{\texorpdfstring{\hyperref[layout]{Layout}}{Layout}}\label{layout}

Layout is similar to \texttt{\ }{\texttt{\ measure\ }}\texttt{\ } , but
it returns current scope \textbf{parent size} .

If you are putting elements in block, that will be
block\textquotesingle s size. If you are just putting right on the page,
that will be page\textquotesingle s size.

As parent\textquotesingle s size depends on it\textquotesingle s place
in document, it uses the similar scheme to
\texttt{\ }{\texttt{\ locate\ }}\texttt{\ } and
\texttt{\ }{\texttt{\ style\ }}\texttt{\ } :

\begin{verbatim}
#layout(size => {
  let half = 50% * size.width
  [Half a page is #half wide.]
})
\end{verbatim}

\pandocbounded{\includesvg[keepaspectratio]{typst-img/c68a166f6e6b1b3229fd56478ae302dbeb39c882e229c69d4c6ebb6c9c528985-1.svg}}

It may be extremely useful to combine
\texttt{\ }{\texttt{\ layout\ }}\texttt{\ } with
\texttt{\ }{\texttt{\ measure\ }}\texttt{\ } , to get width of things
that depend on parent\textquotesingle s size:

\begin{verbatim}
#let text = lorem(30)
#layout(size => style(styles => [
  #let (height,) = measure(
    block(width: size.width, text),
    styles,
  )
  This text is #height high with
  the current page width: \
  #text
]))
\end{verbatim}

\pandocbounded{\includesvg[keepaspectratio]{typst-img/93167dc0b22b02fe27aa92c6b03c2281665b4352624364a19c63f61a488aa75a-1.svg}}


\title{sitandr.github.io/typst-examples-book/book/basics/extra}

\section{\texorpdfstring{\hyperref[extra]{Extra}}{Extra}}\label{extra}

\subsection{\texorpdfstring{\hyperref[bibliography]{Bibliography}}{Bibliography}}\label{bibliography}

Typst supports bibliography using BibLaTex
\texttt{\ }{\texttt{\ .bib\ }}\texttt{\ } file or its own Hayagriva
\texttt{\ }{\texttt{\ .yml\ }}\texttt{\ } format.

BibLaTex is wider supported, but Hayagriva is easier to work with.

\begin{quote}
Link to Hayagriva
\href{https://github.com/typst/hayagriva/blob/main/docs/file-format.md}{documentation}
and some
\href{https://github.com/typst/hayagriva/blob/main/tests/data/basic.yml}{examples}
.
\end{quote}

\subsubsection{\texorpdfstring{\hyperref[citation-style]{Citation
Style}}{Citation Style}}\label{citation-style}

The style can be customized via CSL, citation style language, with more
than 10 000 styles available online. See
\href{https://github.com/citation-style-language/styles}{official
repository} .


