\title{typst.app/docs/reference/introspection/location}

\begin{itemize}
\tightlist
\item
  \href{/docs}{\includesvg[width=0.16667in,height=0.16667in]{/assets/icons/16-docs-dark.svg}}
\item
  \includesvg[width=0.16667in,height=0.16667in]{/assets/icons/16-arrow-right.svg}
\item
  \href{/docs/reference/}{Reference}
\item
  \includesvg[width=0.16667in,height=0.16667in]{/assets/icons/16-arrow-right.svg}
\item
  \href{/docs/reference/introspection/}{Introspection}
\item
  \includesvg[width=0.16667in,height=0.16667in]{/assets/icons/16-arrow-right.svg}
\item
  \href{/docs/reference/introspection/location/}{Location}
\end{itemize}

\section{\texorpdfstring{{ location }}{ location }}\label{summary}

Identifies an element in the document.

A location uniquely identifies an element in the document and lets you
access its absolute position on the pages. You can retrieve the current
location with the
\href{/docs/reference/introspection/here/}{\texttt{\ here\ }} function
and the location of a queried or shown element with the
\href{/docs/reference/foundations/content/\#definitions-location}{\texttt{\ location()\ }}
method on content.

\subsection{Locatable elements}\label{locatable}

Currently, only a subset of element functions is locatable. Aside from
headings and figures, this includes equations, references, quotes and
all elements with an explicit label. As a result, you \emph{can} query
for e.g. \href{/docs/reference/model/strong/}{\texttt{\ strong\ }}
elements, but you will find only those that have an explicit label
attached to them. This limitation will be resolved in the future.

\subsection{\texorpdfstring{{ Definitions
}}{ Definitions }}\label{definitions}

\phantomsection\label{definitions-tooltip}
Functions and types and can have associated definitions. These are
accessed by specifying the function or type, followed by a period, and
then the definition\textquotesingle s name.

\subsubsection{\texorpdfstring{\texttt{\ page\ }}{ page }}\label{definitions-page}

Returns the page number for this location.

Note that this does not return the value of the
\href{/docs/reference/introspection/counter/}{page counter} at this
location, but the true page number (starting from one).

If you want to know the value of the page counter, use
\texttt{\ }{\texttt{\ counter\ }}\texttt{\ }{\texttt{\ (\ }}\texttt{\ page\ }{\texttt{\ )\ }}\texttt{\ }{\texttt{\ .\ }}\texttt{\ }{\texttt{\ at\ }}\texttt{\ }{\texttt{\ (\ }}\texttt{\ loc\ }{\texttt{\ )\ }}\texttt{\ }
instead.

Can be used with
\href{/docs/reference/introspection/here/}{\texttt{\ here\ }} to
retrieve the physical page position of the current context:

self { . } { page } (

) -\textgreater{} \href{/docs/reference/foundations/int/}{int}

\begin{verbatim}
#context [
  I am located on
  page #here().page()
]
\end{verbatim}

\includegraphics[width=5in,height=\textheight,keepaspectratio]{/assets/docs/0ToVSLLUesTLkEw_YsnJkwAAAAAAAAAA.png}

\subsubsection{\texorpdfstring{\texttt{\ position\ }}{ position }}\label{definitions-position}

Returns a dictionary with the page number and the x, y position for this
location. The page number starts at one and the coordinates are measured
from the top-left of the page.

If you only need the page number, use \texttt{\ page()\ } instead as it
allows Typst to skip unnecessary work.

self { . } { position } (

) -\textgreater{}
\href{/docs/reference/foundations/dictionary/}{dictionary}

\subsubsection{\texorpdfstring{\texttt{\ page-numbering\ }}{ page-numbering }}\label{definitions-page-numbering}

Returns the page numbering pattern of the page at this location. This
can be used when displaying the page counter in order to obtain the
local numbering. This is useful if you are building custom indices or
outlines.

If the page numbering is set to
\texttt{\ }{\texttt{\ none\ }}\texttt{\ } at that location, this
function returns \texttt{\ }{\texttt{\ none\ }}\texttt{\ } .

self { . } { page-numbering } (

) -\textgreater{} \href{/docs/reference/foundations/none/}{none}
\href{/docs/reference/foundations/str/}{str}
\href{/docs/reference/foundations/function/}{function}

\href{/docs/reference/introspection/locate/}{\pandocbounded{\includesvg[keepaspectratio]{/assets/icons/16-arrow-right.svg}}}

{ Locate } { Previous page }

\href{/docs/reference/introspection/metadata/}{\pandocbounded{\includesvg[keepaspectratio]{/assets/icons/16-arrow-right.svg}}}

{ Metadata } { Next page }

\textbf{On this page}

\begin{itemize}
\tightlist
\item
  \hyperref[summary]{Summary}
\item
  \hyperref[locatable]{Locatable}
\item
  \hyperref[definitions]{Definitions}

  \begin{itemize}
  \tightlist
  \item
    \hyperref[definitions-page]{Page}
  \item
    \hyperref[definitions-position]{Position}
  \item
    \hyperref[definitions-page-numbering]{Page Numbering}
  \end{itemize}
\end{itemize}

\begin{itemize}
\tightlist
\item
  \href{/}{Home}
\item
  \href{/pricing/}{Pricing}
\item
  \href{/docs/}{Documentation}
\item
  \href{/universe/}{Universe}
\item
  \href{/about/}{About Us}
\item
  \href{/contact/}{Contact Us}
\item
  \href{/privacy/}{Privacy}
\item
  \href{https://typst.app/terms}{Terms and Conditions}
\item
  \href{/legal/}{Legal (Impressum)}
\end{itemize}

\begin{itemize}
\tightlist
\item
  \href{https://forum.typst.app}{Forum}
\item
  \href{/tools/}{Tools}
\item
  \href{/blog/}{Blog}
\item
  \href{https://github.com/typst/}{GitHub}
\item
  \href{https://discord.gg/2uDybryKPe}{Discord}
\item
  \href{https://mastodon.social/@typst}{Mastodon}
\item
  \href{https://bsky.app/profile/typst.app}{Bluesky}
\item
  \href{https://www.linkedin.com/company/typst/}{LinkedIn}
\item
  \href{https://instagram.com/typstapp/}{Instagram}
\end{itemize}

Made in Berlin
