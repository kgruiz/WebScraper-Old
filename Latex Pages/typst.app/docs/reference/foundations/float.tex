\title{typst.app/docs/reference/foundations/float}

\begin{itemize}
\tightlist
\item
  \href{/docs}{\includesvg[width=0.16667in,height=0.16667in]{/assets/icons/16-docs-dark.svg}}
\item
  \includesvg[width=0.16667in,height=0.16667in]{/assets/icons/16-arrow-right.svg}
\item
  \href{/docs/reference/}{Reference}
\item
  \includesvg[width=0.16667in,height=0.16667in]{/assets/icons/16-arrow-right.svg}
\item
  \href{/docs/reference/foundations/}{Foundations}
\item
  \includesvg[width=0.16667in,height=0.16667in]{/assets/icons/16-arrow-right.svg}
\item
  \href{/docs/reference/foundations/float/}{Float}
\end{itemize}

\section{\texorpdfstring{{ float }}{ float }}\label{summary}

A floating-point number.

A limited-precision representation of a real number. Typst uses 64 bits
to store floats. Wherever a float is expected, you can also pass an
\href{/docs/reference/foundations/int/}{integer} .

You can convert a value to a float with this type\textquotesingle s
constructor.

NaN and positive infinity are available as
\texttt{\ float\ }{\texttt{\ .\ }}\texttt{\ nan\ } and
\texttt{\ float\ }{\texttt{\ .\ }}\texttt{\ inf\ } respectively.

\subsection{Example}\label{example}

\begin{verbatim}
#3.14 \
#1e4 \
#(10 / 4)
\end{verbatim}

\includegraphics[width=5in,height=\textheight,keepaspectratio]{/assets/docs/Oh7PyPKhSHHcwVH4CSb0KwAAAAAAAAAA.png}

\subsection{\texorpdfstring{Constructor
{}}{Constructor }}\label{constructor}

\phantomsection\label{constructor-constructor-tooltip}
If a type has a constructor, you can call it like a function to create a
new value of the type.

Converts a value to a float.

\begin{itemize}
\tightlist
\item
  Booleans are converted to \texttt{\ 0.0\ } or \texttt{\ 1.0\ } .
\item
  Integers are converted to the closest 64-bit float. For integers with
  absolute value less than
  \texttt{\ calc\ }{\texttt{\ .\ }}\texttt{\ }{\texttt{\ pow\ }}\texttt{\ }{\texttt{\ (\ }}\texttt{\ }{\texttt{\ 2\ }}\texttt{\ }{\texttt{\ ,\ }}\texttt{\ }{\texttt{\ 53\ }}\texttt{\ }{\texttt{\ )\ }}\texttt{\ }
  , this conversion is exact.
\item
  Ratios are divided by 100\%.
\item
  Strings are parsed in base 10 to the closest 64-bit float. Exponential
  notation is supported.
\end{itemize}

{ float } (

{ \href{/docs/reference/foundations/bool/}{bool}
\href{/docs/reference/foundations/int/}{int}
\href{/docs/reference/foundations/float/}{float}
\href{/docs/reference/layout/ratio/}{ratio}
\href{/docs/reference/foundations/str/}{str}
\href{/docs/reference/foundations/decimal/}{decimal} }

) -\textgreater{} \href{/docs/reference/foundations/float/}{float}

\begin{verbatim}
#float(false) \
#float(true) \
#float(4) \
#float(40%) \
#float("2.7") \
#float("1e5")
\end{verbatim}

\includegraphics[width=5in,height=\textheight,keepaspectratio]{/assets/docs/PMa-HqZaL4--FN_1I0OHagAAAAAAAAAA.png}

\paragraph{\texorpdfstring{\texttt{\ value\ }}{ value }}\label{constructor-value}

\href{/docs/reference/foundations/bool/}{bool} {or}
\href{/docs/reference/foundations/int/}{int} {or}
\href{/docs/reference/foundations/float/}{float} {or}
\href{/docs/reference/layout/ratio/}{ratio} {or}
\href{/docs/reference/foundations/str/}{str} {or}
\href{/docs/reference/foundations/decimal/}{decimal}

{Required} {{ Positional }}

\phantomsection\label{constructor-value-positional-tooltip}
Positional parameters are specified in order, without names.

The value that should be converted to a float.

\subsection{\texorpdfstring{{ Definitions
}}{ Definitions }}\label{definitions}

\phantomsection\label{definitions-tooltip}
Functions and types and can have associated definitions. These are
accessed by specifying the function or type, followed by a period, and
then the definition\textquotesingle s name.

\subsubsection{\texorpdfstring{\texttt{\ is-nan\ }}{ is-nan }}\label{definitions-is-nan}

Checks if a float is not a number.

In IEEE 754, more than one bit pattern represents a NaN. This function
returns \texttt{\ true\ } if the float is any of those bit patterns.

self { . } { is-nan } (

) -\textgreater{} \href{/docs/reference/foundations/bool/}{bool}

\begin{verbatim}
#float.is-nan(0) \
#float.is-nan(1) \
#float.is-nan(float.nan)
\end{verbatim}

\includegraphics[width=5in,height=\textheight,keepaspectratio]{/assets/docs/9jd8hxPcunH7CdCSXWd1dwAAAAAAAAAA.png}

\subsubsection{\texorpdfstring{\texttt{\ is-infinite\ }}{ is-infinite }}\label{definitions-is-infinite}

Checks if a float is infinite.

Floats can represent positive infinity and negative infinity. This
function returns \texttt{\ }{\texttt{\ true\ }}\texttt{\ } if the float
is an infinity.

self { . } { is-infinite } (

) -\textgreater{} \href{/docs/reference/foundations/bool/}{bool}

\begin{verbatim}
#float.is-infinite(0) \
#float.is-infinite(1) \
#float.is-infinite(float.inf)
\end{verbatim}

\includegraphics[width=5in,height=\textheight,keepaspectratio]{/assets/docs/AIoKhvpoq-xeueiSPD9O7gAAAAAAAAAA.png}

\subsubsection{\texorpdfstring{\texttt{\ signum\ }}{ signum }}\label{definitions-signum}

Calculates the sign of a floating point number.

\begin{itemize}
\tightlist
\item
  If the number is positive (including
  \texttt{\ }{\texttt{\ +\ }}\texttt{\ }{\texttt{\ 0.0\ }}\texttt{\ } ),
  returns \texttt{\ }{\texttt{\ 1.0\ }}\texttt{\ } .
\item
  If the number is negative (including
  \texttt{\ }{\texttt{\ -\ }}\texttt{\ }{\texttt{\ 0.0\ }}\texttt{\ } ),
  returns
  \texttt{\ }{\texttt{\ -\ }}\texttt{\ }{\texttt{\ 1.0\ }}\texttt{\ } .
\item
  If the number is NaN, returns
  \texttt{\ float\ }{\texttt{\ .\ }}\texttt{\ nan\ } .
\end{itemize}

self { . } { signum } (

) -\textgreater{} \href{/docs/reference/foundations/float/}{float}

\begin{verbatim}
#(5.0).signum() \
#(-5.0).signum() \
#(0.0).signum() \
#float.nan.signum()
\end{verbatim}

\includegraphics[width=5in,height=\textheight,keepaspectratio]{/assets/docs/HHp-pldXJoLbqAEsg_2mmQAAAAAAAAAA.png}

\subsubsection{\texorpdfstring{\texttt{\ from-bytes\ }}{ from-bytes }}\label{definitions-from-bytes}

Converts bytes to a float.

float { . } { from-bytes } (

{ \href{/docs/reference/foundations/bytes/}{bytes} , } {
\hyperref[definitions-from-bytes-parameters-endian]{endian :}
\href{/docs/reference/foundations/str/}{str} , }

) -\textgreater{} \href{/docs/reference/foundations/float/}{float}

\begin{verbatim}
#float.from-bytes(bytes((0, 0, 0, 0, 0, 0, 240, 63))) \
#float.from-bytes(bytes((63, 240, 0, 0, 0, 0, 0, 0)), endian: "big")
\end{verbatim}

\includegraphics[width=5in,height=\textheight,keepaspectratio]{/assets/docs/TbCinqru71JKOm73kOJYdwAAAAAAAAAA.png}

\paragraph{\texorpdfstring{\texttt{\ bytes\ }}{ bytes }}\label{definitions-from-bytes-bytes}

\href{/docs/reference/foundations/bytes/}{bytes}

{Required} {{ Positional }}

\phantomsection\label{definitions-from-bytes-bytes-positional-tooltip}
Positional parameters are specified in order, without names.

The bytes that should be converted to a float.

Must be of length exactly 8 so that the result fits into a 64-bit float.

\paragraph{\texorpdfstring{\texttt{\ endian\ }}{ endian }}\label{definitions-from-bytes-endian}

\href{/docs/reference/foundations/str/}{str}

The endianness of the conversion.

\begin{longtable}[]{@{}ll@{}}
\toprule\noalign{}
Variant & Details \\
\midrule\noalign{}
\endhead
\bottomrule\noalign{}
\endlastfoot
\texttt{\ "\ big\ "\ } & Big-endian byte order: The highest-value byte
is at the beginning of the bytes. \\
\texttt{\ "\ little\ "\ } & Little-endian byte order: The lowest-value
byte is at the beginning of the bytes. \\
\end{longtable}

Default: \texttt{\ }{\texttt{\ "little"\ }}\texttt{\ }

\subsubsection{\texorpdfstring{\texttt{\ to-bytes\ }}{ to-bytes }}\label{definitions-to-bytes}

Converts a float to bytes.

self { . } { to-bytes } (

{ \hyperref[definitions-to-bytes-parameters-endian]{endian :}
\href{/docs/reference/foundations/str/}{str} }

) -\textgreater{} \href{/docs/reference/foundations/bytes/}{bytes}

\begin{verbatim}
#array(1.0.to-bytes(endian: "big")) \
#array(1.0.to-bytes())
\end{verbatim}

\includegraphics[width=5in,height=\textheight,keepaspectratio]{/assets/docs/oyz50tHIOoQRj_5WM6JIbAAAAAAAAAAA.png}

\paragraph{\texorpdfstring{\texttt{\ endian\ }}{ endian }}\label{definitions-to-bytes-endian}

\href{/docs/reference/foundations/str/}{str}

The endianness of the conversion.

\begin{longtable}[]{@{}ll@{}}
\toprule\noalign{}
Variant & Details \\
\midrule\noalign{}
\endhead
\bottomrule\noalign{}
\endlastfoot
\texttt{\ "\ big\ "\ } & Big-endian byte order: The highest-value byte
is at the beginning of the bytes. \\
\texttt{\ "\ little\ "\ } & Little-endian byte order: The lowest-value
byte is at the beginning of the bytes. \\
\end{longtable}

Default: \texttt{\ }{\texttt{\ "little"\ }}\texttt{\ }

\href{/docs/reference/foundations/eval/}{\pandocbounded{\includesvg[keepaspectratio]{/assets/icons/16-arrow-right.svg}}}

{ Evaluate } { Previous page }

\href{/docs/reference/foundations/function/}{\pandocbounded{\includesvg[keepaspectratio]{/assets/icons/16-arrow-right.svg}}}

{ Function } { Next page }
